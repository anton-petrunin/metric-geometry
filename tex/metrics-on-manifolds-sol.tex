\parbf{\ref{ex:besikovitch=}.}
Let us use the same notation as in the proof of \ref{thm:besikovitch}.

Consider the map $s\:x\mapsto(\distfun_A(x)_g,\distfun_B(x)_g)$.
From the proof of \ref{thm:besikovitch} we get that $\Im s\supset \square$.
Observe that in the case of equality we have that $\Im s= \square$.
Indeed,
the same argument shows that 
\[\vol(s^{-1}(\square),g)\ge \vol\square=1.\]
The set $s^{-1}(\RR^1\backslash \square)$ is an open subset of $\square$.
If it is nonempty, then it has positive volume.
In this case
\[\vol(\square,g)>\vol(s^{-1}(\square),g)\ge 1\]
--- a contradiction.

Summarizing the above discussion, there is a geodesic path of $g$-length $1$ connecting any point on one face of the cube to a point on the opposite face.

Moreover, for any pair of opposite faces and a point $p\in\square$, there is a geodesic path of $g$-length $1$ from one face to the other that passes thru $p$.
The latter can be shown by cutting $\square$ into two rectangles by a level set of $\distfun_A$ thru $p$,
applying the above statement to both rectangles and taking the concatenation of the obtained geodesic paths with end at $p$.
(The level surface might cut a rectangle with some topology, so we have to apply \ref{thm:besikovitch+} instead of \ref{thm:besikovitch}).

Let $\gamma$ be such a geodesic path from $A$ to $A'$.
Observe that $\gamma'(t)\z=\nabla_{\gamma(t)}\distfun_A$.
Therefore $\distfun_A$ is differentiable at every point $p\in \square$.
It follows that the map $s$ is differentiable.

Further, checking the equality case in each inequality in the proof of \ref{thm:besikovitch}, we get that $s$ is a bijection and the equalities
\[|d_{p}\distfun_A|= 1,\quad|d_{p}\distfun_B|=1,\quad \text{and}\quad \langle d_{p}\distfun_A,d_{p}\distfun_B\rangle= 0\]
hold for almost all $p\in\square$.
Since $d_{p}\distfun_A$ and $d_{p}\distfun_B$ are well defined, we get that the equalities hold everywhere.
That is, $s$ is an isometry.

\begin{wrapfigure}{r}{45 mm}
\vskip-4mm
\centering
\includegraphics{mppics/pic-27}
\end{wrapfigure}

\parbf{\ref{ex:hexagon}.}
Consider the hexagon with flat metric and curved sides shown on the diagram.
Observe that its area can be made arbitrary small while keeping the distances from the opposite sides at least 1.

\parbf{\ref{ex:cylinder};} \ref{SHORT.ex:cylinder:besicovitch}.
Let $\alpha$ be a shortest curve that runs between the boundary components of the cylinder.
Cut the cylinder along $\alpha$.
We get a square with Riemannian metric on it $(\square,g)$.

Two opposite sides of $\square$ correspond to the boundary components of the cylinder.
The other pair corresponds to the sides of the cut.
By assumption, the $g$-distance between the first pair of sides is at least 1.

Consider a shortest curve $\beta$ that connects this pair of sides;
let us keep the same notation for the projection of $\beta$ in the cylinder.

\begin{wrapfigure}{o}{45 mm}
\vskip2mm
\centering
\includegraphics{mppics/pic-23}
\end{wrapfigure}

Note that a cyclic concatenation $\gamma$ of $\beta$ with an arc of $\alpha$ is homotopic to a boundary circle.
Therefore $\length_g\gamma\ge1$.
Since $\alpha$ is a shortest path, its arc cannot be longer than any curve connecting its ends; therefore 
\[\length_g\beta\ge \tfrac 12\cdot\length_g \gamma\ge \tfrac 12.\]
That is, the other pair of sides of $\square$ lies on $g$-distance at least $\tfrac12$ from each other.
By \ref{thm:besikovitch+}, $\area(\square,g)\ge \tfrac12$, hence the result.

\parit{\ref{SHORT.ex:cylinder:coarea}.}
Note that any curve in the cylinder that is bordant to a boundary component has length at least $1$.
Therefore if $0\le t\le  1$, then the level sets 
\[L_t=\set{x\in \mathbb{S}^1\times[0,1]}{\distfun_{\mathbb{S}^1\times\{0\}}(x)_{g}=t}\] have length at least $1$.
Applying the coarea inequality, we get that
\[\area(\mathbb{S}^1\times[0,1],g)\ge 1.\]

\parit{Remark.}
The optimal bound is $\tfrac2{\sqrt{3}}$;
it is proved by Charles Loewner.

\parbf{\ref{ex:gadograph}.}
Without loss of generality, we may assume that $V$ lies in a unit cube~$\square$.
Consider a noncontinuous metric tensor $\bar g$ on $\square$ that coincides with $g$ inside $V$ and with the canonical flat metric tensor outside of $V$.

Observe that the $\bar g$-distances between opposite faces of $\square$ are at least 1.
Indeed this is true for the Euclidean metric and the assumption $\dist{p}{q}{g}\ge\dist{p}{q}{\EE^d}$  guarantees that one cannot make a shortcut in~$V$.
Therefore, the $\bar g$-distances between every pair of opposite faces is at least as large as 1 which is the Euclidean distance.

This metric tensor $\bar g$ is not continuous at $\Sigma$, but the same argument as in \ref{thm:besikovitch} can be applied to show that $\vol(\square,\bar g)\ge \vol\square$.
Whence the statement follows.


\parbf{\ref{ex:involution-of-sphere}.}
Let $x\in \mathbb{S}^2$ be a point that minimize the distance $|x-x'|_g$.
Consider a shortest path $\gamma$ from $x$ to $x'$.
We can assume that 
\[|x-x'|_g=\length \gamma=1.\]

Let $\gamma'$ be the antipodal arc to $\gamma$.
Note that $\gamma'$ intersects $\gamma$ only at the common endpoints $x$ and $x'$.
Indeed, if $p'=q$ for some $p,q\in\gamma$, then $|p-q|\ge 1$.
Since $\length \gamma=1$, the points $p$ and $q$ must be the ends of $\gamma$.

It follows that $\gamma$ together with $\gamma'$ forms a closed simple curve in $\mathbb{S}^2$
that divides the sphere into two disks $D$ and $D'$.

Let us divide $\gamma$ into two equal arcs $\gamma_1$ and $\gamma_2$; each of length $\tfrac12$.
Suppose that $p,q\in\gamma_1$, then 
\begin{align*}
|p-q'|_g&\ge |q-q'|_g-|p-q|_g\ge
\\
&\ge 1-\tfrac12=\tfrac12.
\end{align*}
That is, the minimal distance from $\gamma_1$ to $\gamma_1'$ is at least~$\tfrac12$.
The same way we get that the minimal distance from $\gamma_2$ to $\gamma_2'$ is at least~$\tfrac12$.
By Besicovitch inequality, we get that 
\[\area(D,g)\ge \tfrac14\quad\text{and}\quad \area(D',g)\ge \tfrac14.\]
Therefore 
\[\area(\mathbb{S}^2,g)\ge\tfrac12.\]

\parit{A better estimate.}
Let us indicate how to improve the obtained bound to
\[\area(\mathbb{S}^2,g)\ge1.\]

Suppose $x$, $x'$, $\gamma$ and $\gamma'$ are as above.
Consider the function
\[f(z)=\min_t \{\,|\gamma'(t)-z|_g+t\,\}.\]
Observe that $f$ is 1-Lipschitz.

Show that two points $\gamma'(c)$ and $\gamma(1-c)$ lie on one connected component of the level set $L_c=\set{z\in\mathbb{S}^2}{f(z)=c}$;
in particular 
\[\length L_c\ge 2\cdot|\gamma'(c)-\gamma(1-c)|_g.\]
By the triangle inequality, we have that
\begin{align*}
|\gamma'(c)-\gamma(1-c)|_g&\ge 1-|\gamma(c)-\gamma(1-c)|_g=
\\
&=1-|1-2\cdot c|.
\end{align*}

It remains to apply the coarea inequality \ref{cor:coarea}
\[\area(\mathbb{S}^2,g)\ge \int\limits_0^1\length L_c\cdot dc.\]

\parit{Remarks.}
The bound $\tfrac12$ was proved by Marcel Berger \cite{berger}. 
Christopher Croke conjectured that the optimal bound is $\tfrac4\pi$ and the round sphere is the only space that achieves this \cite[Conjecture 0.3 in][]{croke}.

\begin{wrapfigure}{r}{20 mm}
\vskip-0mm
\centering
\includegraphics{mppics/pic-1305}
\end{wrapfigure}

\parbf{\ref{ex:involution-of-3sphere}.}
Given $\eps>0$, construct a disk $\Delta$ in the plane with 
\[\length\partial \Delta<10\ \ \text{and}\ \ \area \Delta<\eps\]
that admits an continuous involution $\iota$ such that 
\[|\iota(x)-x|\ge 1\]
for any $x\in\partial \Delta$.

An example of $\Delta$ can be guessed from the picture;
the involution $\iota$ makes a length preserving half turn of its boundary $\partial \Delta$.


Take the product $\Delta\times \Delta\subset \RR^4$;
it is homeomorphic to the 4-ball.
Note that 
$$\vol_3[\partial(\Delta\times \Delta)]=2\cdot\area \Delta\cdot\length \partial \Delta<20\cdot\eps.$$
The boundary $\partial(\Delta\times \Delta)$ is homeomorphic to $\mathbb{S}^3$
and the restriction of the involution $(x,y)\z\mapsto (\iota(x),\iota(y))$ has the needed property.

It remains to smooth $\partial(\Delta\times \Delta)$ a  bit.

\parit{Remark.} This example is given by Christopher Croke \cite{croke}.
Note that according to \ref{thm:sys+}, 
the involution $\iota$ cannot be made isometric.

\parbf{\ref{ex:GH-vol}.}
Note that if $(M,g_\infty)$ is $e^{\pm\eps}$-bilipschitz to a cube, then applying Besicovitch inequality, we get that 
\[\liminf_{n\to\infty} \vol (M,g_n)\ge e^{-n\cdot \eps}\cdot\vol (M,g_\infty).\]

By the Vitali covering theorem, given $\eps>0$, we can cover the whole volume of $(M,g_\infty)$ by $e^{\pm\eps}$-bilipschitz cubes.
Applying the above observation and summing up the results, we get that 
\[\liminf_{n\to\infty} \vol (M,g_n)\ge e^{-n\cdot \eps}\cdot\vol (M,g_\infty).\]
The statement follows since $\eps$ is an arbitrary positive number.

\parit{Remark.}
A more general result was obtained by Sergei Ivanov~\cite{ivanov-1997}.
Note that the statement does for Gromov--Hausdorff convergence.
In fact any compact metric space $\spc{X}$ can be approximated by Riemannian surface in the sense of Gromov--Hausdorff with an arbitrary small area.
To show the latter statement, approximate $\spc{X}$ by a finite graph $\Gamma$, embed $\Gamma$ isometrically to the Euclidean space, and pass to the surface of its neighborhood.

\parbf{\ref{ex:sysT2}.}
Set $s=\sys(\TT^2,g)$.

Cut $\TT^2$ along a shortest closed noncontractible curve $\gamma$.
We get a cylinder $(\mathbb{S}^1,g)$ with a Riemannian metric on it.

Applying the argument in \ref{ex:cylinder:besicovitch}, we get that $g$-distance between the boundary components is at least $\tfrac s2$.
Then \ref{ex:cylinder:besicovitch} implies that the area of torus is at least $\tfrac{s^2}2$.

\begin{wrapfigure}{r}{44 mm}
\vskip-4mm
\centering
\includegraphics{mppics/pic-25}
\end{wrapfigure}

\parbf{\ref{ex:sysRP2}.}
Set $s\z=\sys (\RP^2,g)$.
Cut $(\RP^2,g)$ along a shortest noncontractible curve $\gamma$.
We obtain $(\DD^2,g)$ --- a disc with metric tensor which we still denote by $g$.

Divide $\gamma$ into two equal arcs $\alpha$ and $\beta$.
Denote by $A$ and $A'$ the two connected components of the inverse image of $\alpha$.
Similarly denote by $B$ and $B'$ the two connected components of the inverse image of $\beta$.

Let $\gamma_1$ be a path from $A$ to $A'$;
map it to $\RP^2$ and keep the same notation for it.
Note that $\gamma_1$ together with a subarc of $\alpha$ forms a closed noncontractible curve in $\RP^2$.
Since $\length\alpha=\tfrac s2$, we have that $\length\gamma_1\ge \tfrac s2$.
It follows that the distance between $A$ and $A'$ in $(\DD^2,g)$ is at least $\tfrac s2$.
The same way we show that the distance between $B$ and $B'$ in $(\DD^2,g)$ is at least $\tfrac s2$.

Note that $(\DD^2,g)$ can be parameterized by a square with sides $A$, $B$, $A'$ and $B'$ and apply \ref{thm:besikovitch} to show that 
\[\area(\RP^2,g)=\area(\DD^2,g)\ge \tfrac14\cdot s^2.\]

\parit{Remark.}
For the optimal constant was found by Pao Ming Pu see the remark on page \pageref{page:pu}.
His proof shows that any Riemannian metric on the disc with the boundary globally isometric to a unit circle with angle metric has the area at least as large as the unit hemisphere.
It is expected that the same inequality holds for any compact surface bounded by a single curve (not necessarily a disc);
this is the so-called the {}\emph{filling area conjecture} \cite[it is mentioned in 5.5.$\mathrm{B}'(\mathrm{e}')$ of][]{gromov-1983}.

\parbf{\ref{ex:sysSg}.} Cut the surface along a shortest noncontractible curve $\gamma$. 
We might get a surface with one or two components of the boundary.
In these two cases repeat the arguments in \ref{ex:sysRP2} or \ref{ex:sysT2} using \ref{thm:besikovitch+} instead of \ref{thm:besikovitch}.


\parbf{\ref{ex:sysS2xS1}.} Consider the product of a small 2-sphere with the unit circle.

\parbf{\ref{ex:besikovitch++}.}
Apply the same construction as in the original Besicovitch inequality, assuming that the target rectangle
$[0,d_1]\times\dots\times [0,d_n]$ equipped with the metric induced by the $\ell^\infty$ norm;
apply \ref{prop:bilip-measure} where it is appropriate.

\parbf{\ref{ex:2top-discs}.} Suppose that $\Delta_1\ne\Delta_2$.
Consider the map $f\:\mathbb{S}^n\to \spc{X}$ such that the restriction to north and south hemispheres describe $\Delta_1$ and $\Delta_2$ respectively.
Show that if $\Delta_1\ne\Delta_2$, then $\mathbb{S}^n$ can be parameterized by the boundary of unit cube in such a way that for any pair $A$, $A'$ of opposite faces their images $f(A)$, $f(A')$ do not overlap.

Since $\spc{X}$ is contractible, the map $f$ can be extended to a map of the whole cube.
It remains to apply \ref{ex:besikovitch++}.


\parbf{\ref{ex:macrodimension}.}
The following claim resembles Besicovitch inequality;
it is key to the proof:
\begin{itemize}
 \item[$({*})$] Let $a$ be a positive real number.
 Assume that a closed curve $\gamma$ in a metric space $\spc{X}$ can be subdivided into 4 arcs $\alpha$, $\beta$, $\alpha'$, and $\beta'$ in such a way that 
 \begin{itemize}
 \item $|x-x'|>a$ for any $x\in\alpha$ and $x'\in \alpha'$
 and
 \item $|y-y'|>a$ for any $y\in\beta$ and $y'\in \beta'$.
 \end{itemize}
 Then $\gamma$ is not contractible in its $\tfrac a2$-neighborhood.
\end{itemize}

To prove $({*})$, consider two functions defined on $\spc{X}$ as follows:
\begin{align*}
w_1(x)&=\min \{\,a,\distfun_{\alpha}(x)\,\}
\\
w_2(x)&=\min \{\,a,\distfun_{\beta}(x)\,\}
\end{align*}
and the map $\bm{w}\:\spc{X}\to [0,a]\times[0,a]$, defined by
\[\bm{w}\:x\mapsto(w_1(x),w_2(x)).\]

Note that 
\begin{align*}
\bm{w}(\alpha)&=0\times [0,a],
&
\bm{w}(\beta)&=[0,a]\times 0,
\\
\bm{w}(\alpha')&=a\times [0,a],
&
\bm{w}(\beta')&=[0,a]\times a.
\end{align*} 
Therefore, the composition $\bm{w}\circ\gamma$ is a degree 1 map 
\[\mathbb{S}^1\to \partial([0,a]\times[0,a]).\] 
It follows that if $h\:\DD\to \spc{X}$ shrinks $\gamma$, then there is a point $z\in\DD$ such that 
$\bm{w}\circ h(z)=(\tfrac a2,\tfrac a2)$.
Therefore, $h(z)$ lies at distance at least $\tfrac a2$ from $\alpha$, $\beta$, $\alpha'$, $\beta'$
and therefore from $\gamma$.
It proves the claim.

\medskip

Coming back to the problem, let $\{W_i\}$ be an open covering of the real line with multiplicity $2$ and $\rad W_i<R$ for each $i$;
for example take the covering by the intervals $((i-\tfrac23)\cdot R,(i+\tfrac23)\cdot R)$.

Choose a point $p\in \spc{X}$.
Denote by $\{V_j\}$ the connected components of $\distfun_p^{-1}(W_i)$ for all $i$.
Note that $\{V_j\}$ is an open finite cover of $\spc{X}$ with multiplicity at most 2.
It remains to show that $\rad V_j<100\cdot R$ for each $j$.

\begin{wrapfigure}{o}{31 mm}
\vskip-2mm
\centering
\includegraphics{mppics/pic-1310}
\end{wrapfigure}

Arguing by contradiction assume there is a pair of points  $x,y\in V_i$ 
such that $|x\z-y|_{\spc{X}}\z\ge 100\cdot R$.
Connect $x$ to $y$ with a curve $\tau$ in $V_j$.
Consider the closed curve $\sigma$ formed by $\tau$ and two shortest paths $[px]$, $[py]$.


Note that $|p-x|>40$.
Therefore, there is a point $m$ on $[px]$ such that $|m-x|=20$.

By the triangle inequality, the subdivision of $\sigma$ into the arcs $[pm]$, $[mx]$, $\tau$ and $[yp]$ satisfy the conditions of the claim $({*})$ for $a=10\cdot R$,
hence the statement.

\parit{The quasiconverse} does not hold.
As an example take a surface that looks like a long cylinder with two hats,
\begin{figure}[h!]
\vskip0mm
\centering
\includegraphics{mppics/pic-1315}
\end{figure}
it is a smooth surface diffeomorphic to a sphere.
Assuming the cylinder is thin, it has macroscopic dimension 1 at a given scale.
However, a circle formed by a section of cylinder around its midpoint by a plane parallel to the base is a circle that cannot be contracted in its small neighborhood.

\parit{Source:} \cite[Appendix $1(\text{E}_{2})$]{gromov-1983}.

\parbf{\ref{ex:width=suprad(inv)},} \textit{``only if'' part.}
Suppose $\width_n\spc{X}<R$.
Consider a covering $\{V_1,\dots,V_k\}$ of $\spc{X}$ guaranteed by the definition of width.
Let $\spc{N}$ be its nerve and $\psi\:\spc{X}\to \spc{N}$ be the map provided by \ref{prop:space->nerve}.

Since the multiplicity of the covering is at most $n+1$, we have $\dim \spc{N}\le n$.

Note that if $x\in \spc{N}$ lies in a star of a vertex $v_i$,
then $\psi^{-1}\{x\}\z\subset V_i$;
in particular, we have $\rad[\psi^{-1}\{x\}]<R$.

\parit{``If'' part.}
Choose $x\in \spc{N}$.
Since the inverse image $\psi^{-1}\{x\}$ is compact, $\psi$ is continuous, and $\rad[\psi^{-1}\{x\}]<R$,
there is a neighborhood $U\ni x$ such that the  $\rad[\psi^{-1}(U)]<R$.

Since $\spc{X}$ is compact,  there is a finite cover $\{U_i\}$ of $\spc{N}$ such that $\psi^{-1}(U_i)\subset\spc{X}$ has a radius smaller than $R$ for each $i$.
Since $\spc{N}$ has dimension $n$, we can inscribe%
\footnote{Recall that a covering $\{W_i\}$ is inscribed in the covering $\{U_i\}$ if for every $W_i$ is a subset of some $U_j$.} 
in $\{U_i\}$ a finite open cover $\{W_i\}$ with multiplicity at most $n+1$.
It remains to observe that $V_i=\psi^{-1}(W_i)$ defines a finite open cover of $\spc{X}$ with  multiplicity at most $n+1$ and $\rad V_i<R$ for any $i$. 

\parbf{\ref{ex:1D-case}.}
Assume that $\spc{P}$ is connected.

Let us show that $\diam\spc{P}<R$.
If this is not the case, then there are points $p,q\in\spc{P}$ on distance $R$ from each other.
Let $\gamma$ be a shortest path from $p$ to $q$.
Clearly $\length\gamma\ge R$ and $\gamma$ lies in $\oBall(p,R)$ except for the endpoint $q$.
Therefore, $\length[\oBall(p,R)_{\spc{P}}]\ge R$.
Since $\VolPro_{\spc{P}}(R)\z\ge \length[\oBall(p,R)_{\spc{P}}]$,
the latter contradicts $\VolPro_{\spc{P}}(R)<R$.

In general case, we get that each connected component of $\spc{P}$ has a radius smaller than $R$.
Whence the width of $\spc{P}$ is smaller than $R$.

\parit{Second part.} Again, we can assume that $\spc{P}$ is connected.

The examples of line segment or a circle show that the constant $c=\tfrac12$ cannot be improved.
It remains to show that the inequality holds with $c=\tfrac12$.

Choose $p\in\spc{P}$ such that the value
\[\rho(p)=\max\set{\dist{p}{q}{\spc{P}}}{q\in\spc{P}}\]
is minimal.
Suppose $\rho(p)\ge\tfrac 12\cdot R$.
Observe that there is a point $x\z\in \spc{P}\backslash\{p\}$ that lies on any shortest path starting from $p$ and length $\ge\tfrac 12\cdot R$.
Otherwise for any $r\in(0,\tfrac 12\cdot R)$ there would be at least two points on distance $r$ from $p$;
by coarea inequality we get that the total length of $\spc{P}\cap \oBall(p,\tfrac 12\cdot R)$ is at least $R$ --- a contradiction.

Moving $p$ toward $x$ reduces $\rho(p)$ which contradicts the choice of~$p$.

\parbf{\ref{ex:sys<width}.}
The inequality $6\cdot R<s$ used twice:
\begin{itemize}
\item to shrink the triangle $[p_ip_jp_k]$ to a point;
\item to extend the constructed homotopy on $\spc{M}^0$ to $\spc{M}^1$.
\end{itemize}

The first problem can be solved by passing to a barycentric subdivision of $\spc{N}^2$;
denote by $v_{ij}$ and $v_{ijk}$ the new vertexes in the subdivision that correspond to edge $[v_iv_j]$ and triangle $[v_iv_jv_k]$ respectively.

Further for each vertex $v_{ij}$ choose a point $p_{ij}\in V_i\cap V_j$ and set $f(v_{ij})=p_{ij}$.
Similarly for each vertex $v_{ijk}$ choose a point $p_{ijk}\z\in V_i\cap V_j\cap V_k$ and set $f(v_{ijk})=p_{ijk}$.

Note that 
\[|p_i-p_{ij}|<R,\quad |p_i-p_{ijk}|<R,\quad\text{and}\quad |p_{ij}-p_{ijk}|<2\cdot R.\]
Therefore, perimeter of the triangle $[p_ip_{ij}p_{ijk}]$ in the subdivision is less that $4\cdot R$.
It resolves the first issue.

The second issue disappears if one estimates the distances a bit more carefully.
 
\parbf{\ref{ex:fillrad-inj}.}
Choose a fine covering of $\spc{M}$ with multiplicity at most $n$.
Choose $\psi$ from $\spc{M}$ to the nerve $\spc{N}$ of the covering the same way as in the proof of \ref{thm:sys<width}.

It remains to construct $f\:\spc{N}\to\spc{M}$ and show that $f\circ\psi$ is homotopic to the identity map.
To do this, apply the same strategy as in the proof of \ref{thm:sys<width} together with the so-called \index{geodesic cone construction}\emph{geodesic cone construction}
described below.

Let $\triangle$ be a simplex in a barycentric subdivision of $\spc{N}$.
Suppose that a map $f$ is defined on one facet $\triangle'$ of $\triangle$ to $\spc{M}$ and $\oBall(p,r)\supset f(\triangle')$.
Then one can extend $f$ to whole $\triangle$ such that the remaining vertex $v$ maps to $p$.
Namely connect each point $f(x)$ to $p$ by minimizing geodesic path $\gamma_x$ (by assumption it is uniquely defined) and set
\[f
\:
t\cdot x\z+(1-t)\cdot v
\mapsto
\gamma_x(t).\]

\parbf{\ref{ex:connected-sum-essential}.}
Suppose $M$ is an essential manifold and $N$ is an arbitrary closed manifold.
Observe that shrinking $N$ to a point produces a map $N\#M\to M$ of degree 1.
In particular, there is a map $f\:N\#M\to M$ that sends the fundamental class of $N\#M$ to the fundamental class of $M$.

Since $M$ is essential, there is an aspherical space $K$ and a map $\iota\:M\to K$ that sends the fundamental class of $M$ to a nonzero homology class in $K$.
From above, the composition $\iota\circ f\:N\#M\to K$ sends the fundamental class of $N\#M$ to the same homology class in~$K$.


\parbf{\ref{ex:product-essential}.}
Suppose $\spc{M}_1$ and $\spc{M}_2$ are essential.
Let $\iota_1\:\spc{M}_1\to\spc{K}_1$ and $\iota_2\:\spc{M}_2\to\spc{K}_2$ are the maps to aspherical spaces as in the definition (\ref{def:essential}).
Show that the map
$(\iota_1,\iota_2)\:\spc{M}_1\times\spc{M}_2\to\spc{K}_1\times\spc{K}_2$
meets the definition.

\parit{Remark.}
Choose a group $G$.
Note that there is an aspherical connected space CW-complex $\spc{K}$ with fundamental group $G$.
The space $\spc{K}$ is called an \emph{Eilenberg--MacLane space of type $K(G,1)$}, or simply a $K(G,1)$ space.
Moreover it is not hard to check that
\begin{itemize}
\item $\spc{K}$ is uniquely defined up to a weak homotopy equivalence;
\item if $\spc{W}$ is a connected finite CW-complex.
Then any homomorphism $\pi_1(\spc{W},w)\to\pi_1(\spc{K},k)$ is induced by a continuous map $\phi\:(\spc{W},w)\to(\spc{K},k)$.
Moreover, $\phi$ is uniquely defined up to homotopy equivalence.
\end{itemize}

\begin{itemize}
 \item Suppose that $\spc{M}$ is a closed manifold, 
$\spc{K}$ is a $K(\pi_1(\spc{M}),1)$ space and a map $\iota\:\spc{M}\to \spc{K}$ induces an isomorphism of fundamental groups.
Then $\spc{M}$ is essential if and only if $\iota$ sends the fundamental class of $\spc{M}$ to a nonzero homology class of $\spc{K}$.
\end{itemize}

The last property described in the last statement is the original definition of essential manifold.
It can be used to prove a converse to the exercse;
namely \emph{the product of nonessential closed manifold with any closed manifold is \emph{not} essential}.



