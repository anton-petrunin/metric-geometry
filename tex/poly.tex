\chapter{Polyhedral spaces}\label{chapter:polyhedral}

This lecture is nearly a copy of \cite[Sections 1.7, 3.4, and 3.5]{alexander-kapovitch-petrunin-2019};
here we give a condition for polyhedral spaces that grantees that it is $\CAT(0)$.


\section{Space of directions and tangent space}
\label{sec:tangent-space+directions}

In this section we introduce a metric analog of (unit) tangent bundle that makes sense in Alexandrov geometry.

Let $\spc{X}$ be a metric space with defined angles for all hinges;
by \ref{ex:noncreasing} it holds for any $\CBB(\kappa)$ or $\CAT(\kappa)$ space.
Fix a point $p\in \spc{X}$. 

Consider the set $\mathfrak{S}_p$ 
of all nontrivial geodesics  that start at~$p$.
By \ref{ex:tringle-inq-angles}, the triangle inequality holds for $\mangle$ on $\mathfrak{S}_p$,
so
 $(\mathfrak{S}_p,\mangle)$ 
forms a \index{pseudometric space}\emph{pseudometric space};
that is, $\mangle$ satisfies all the conditions of a metric on $\mathfrak{S}_p$, except that  the angle between distinct geodesics might vanish.

The metric space corresponding to  $(\mathfrak{S}_p,\mangle)$ is called the \index{space of geodesic directions}\emph{space of geodesic directions} at $p$, denoted by $\Sigma'_p$ or $\Sigma'_p\spc{X}$.
Elements of $\Sigma'_p$ are called \index{geodesic directions}\emph{geodesic directions} at~$p$.
Each geodesic direction is formed by an equivalence class of geodesics in $\mathfrak{S}_p$
for the equivalence relation 
\[[px]\sim[py]\ \ \iff\ \ \mangle\hinge pxy=0.\]

The completion of $\Sigma'_p$ is called the 
\index{space of directions}\emph{space of directions} at $p$ and is denoted by $\Sigma_p$ or $\Sigma_p\spc{X}$.
Elements of $\Sigma_p$ are called \index{direction}\emph{directions} at~$p$.

The Euclidean cone $\Cone\Sigma_p$ over the space of directions $\Sigma_p$ is called the \index{tangent space}\emph{tangent space} at  $p$ and is denoted by $\T_p$ or $\T_p\spc{X}$.

\begin{thm}{Exercise}\label{ex:geod-CBA}
Assume $\spc{U}$ is a proper length $\CAT(0)$ space
 with extendable geodesics;
that is, any geodesic is an arc in a local geodesic $\RR\to \spc{U}$.

Show that the space of geodesic directions at any point in $\spc{U}$ is complete.

Does the statement remain true if $\spc{U}$ is complete, but not required to be proper?
\end{thm}


The tangent space $\T_p$ could also be defined directly, without introducing the space of directions.
To do so, consider the set $\mathfrak{T}_p$ of all geodesics with constant-speed parametrizations starting at~$p$. 
Given $\alpha,\beta\in \mathfrak{T}_p$,
set 
\[\dist{\alpha}{\beta}{\mathfrak{T}_p}
=
\lim_{\eps\to0} 
\frac{\dist{\alpha(\eps)}{\beta(\eps)}{\spc{X}}}\eps
\eqlbl{eq:dist-in-T_p}\]
Since the angles in $\spc{X}$ are defined, 
\ref{eq:dist-in-T_p}
defines a pseudometric on $\mathfrak{T}_p$.


The corresponding metric space admits a natural isometric identification with the cone $\T'_p=\Cone\Sigma'_p$.
The elements of $\T'_p$ are  equivalence classes for the relation 
\[\alpha\sim\beta\ \ \iff\ \ \dist{\alpha(t)}{\beta(t)}{\spc{X}}=o(t).\]
The completion of $\T'_p$ is therefore  naturally isometric to~$\T_p$.

Elements of $\T_p$ will be called 
\index{tangent vector}\emph{tangent vectors} 
at $p$, regardless of the fact that $\T_p$ is only a metric cone and need not be a vector space.
Elements of $\T'_p$ will be called 
\index{geodesic tangent vector}\emph{geodesic tangent vectors} 
at~$p$.

\begin{thm}{Exercise}\label{ex:tan(CAT)isCAT}
Let $\spc{X}$ be a complete length $\CAT(0)$ space.
Show that for any point $p\in \spc{X}$  the tangent space $\T_p\spc{X}$ is isometric to a subset of the ultra-tangent space $\T_p^\omega\spc{X}$ (defined on page~\pageref{page:ultratangent space}).

Use \ref{prop:cat-limit} to conclude that $\T_p\spc{X}$ is $\CAT(0)$.
\end{thm}

\begin{thm}{Exercise}\label{ex:tan(CAT)is-length}
Let $\spc{X}$ be a complete length $\CAT(0)$ space.
Show that for any point $p\in \spc{X}$ the tangent space $\T_p\spc{X}$ is a length space.
\end{thm}

\section{Suspension}

Suspension a spherical analog of cone construction defined on page~\pageref{page:cone}.

The \index{suspension}\emph{suspension} $\spc{V}=\Susp\spc{U}$ over a metric space $\spc{U}$
is defined as the metric space whose underlying set consists of equivalence classes in
$[0,\pi]\times \spc{U}$ with the equivalence relation ``$\sim$'' given by $(0,p)\z\sim (0,q)$ and $(\pi,p)\z\sim (\pi,q)$ for any points $p,q\in\spc{U}$,
and whose metric is given by the  spherical cosine rule
\[
\cos\dist{(p,s)}{(q,t)}{\Susp\spc{U}} 
=
\cos s\cdot\cos t-\sin s\cdot\sin t\cdot\cos\alpha,
\]
where $\alpha= \min\{\pi, \dist{p}{q}{\spc{U}}\}$.

The points in $\spc{V}$ formed by the equivalence classes of $0\times\spc{U}$ and $\pi\times\spc{U}$ are called  the {}\emph{north} and the \index{pole of suspension}\emph{south poles} of the suspension.

\begin{thm}{Exercise}\label{ex:product-cone}
Let $\spc{U}$ be a metric space.
Show that the spaces 
\[\RR\times \Cone\spc{U}\quad\text{and}\quad\Cone[\Susp\spc{U}]\]
are  isometric.
\end{thm}

The following statement is a direct analog of \ref{ex:cone+susp} and it can be proved along the same lines.

\begin{thm}{Proposition}\label{prop:susp}
Let $\spc{U}$ be a metric space.
Then $\Susp\spc{U}$ is  $\CAT(1)$ if and only if $\spc{U}$ is $\CAT(1)$.
\end{thm}


\section{Definitions}

\begin{thm}{Definition}\label{def:poly}
A length space $\spc{P}$ is called  
a \index{polyhedral space}\emph{(spherical) polyhedral space} 
if it admits a finite triangulation $\tau$ 
such that every simplex in $\tau$ is isometric to a simplex in a Euclidean space (or respectively a unit sphere) of appropriate dimension.

By a 
\index{triangulation of a polyhedral space}\emph{triangulation of a polyhedral space} 
we will always understand a triangulation as above. 
\end{thm}

Note that according to the above definition,
all polyhedral spaces are compact.
However, 
most of the statements below admit straightforward generalizations 
to \index{polyhedral spaces!locally polyhedral spaces}\emph{locally polyhedral spaces};
that is, complete length spaces,  
any point of which admits a closed neighborhood isometric to a polyhedral space.
The latter class of spaces includes in particular infinite covers of polyhedral spaces.

The \index{dimension of a polyhedral space}\emph{dimension} of a polyhedral space $\spc{P}$
is defined as the maximal dimension of the simplices 
in one (and therefore any) triangulation of~$\spc{P}$.

\parbf{Links.}
Let $\spc{P}$ be a polyhedral space
and $\sigma$ be a simplex in a triangulation $\tau$ of~$\spc{P}$.

The simplices that  contain $\sigma$
form an abstract simplicial complex called the \index{link}\emph{link} of $\sigma$, 
denoted by $\Link_\sigma$.
If $m$ is  the dimension of $\sigma$,
then the set of vertices of $\Link_\sigma$
is formed by the $(m+1)$-simplices that  contain $\sigma$;
the set of its edges are formed by the $(m+2)$-simplices 
that contain $\sigma$; and so on.

The link $\Link_\sigma$
can be identified with the subcomplex of $\tau$ 
formed by all the simplices $\sigma'$ 
such that $\sigma\cap\sigma'=\emptyset$ 
but both $\sigma$ and $\sigma'$ are faces of a simplex of~$\tau$.

The points in $\Link_\sigma$ can be identified with the normal directions to $\sigma$ at a point in its interior.
The angle metric between directions makes  $\Link_\sigma$ into a spherical polyhedral space.
We will always consider the link with this metric.

\parbf{Tangent space and space of directions.}
Let $\spc{P}$ be a polyhedral space (Euclidean or spherical) and  $\tau$ be its triangulation.
If a point $p\in \spc{P}$ 
lies in the interior of a $k$-simplex $\sigma$ of $\tau$ 
then the tangent space $\T_p=\T_p\spc{P}$
is  naturally isometric to
\[\EE^k\times(\Cone\Link_\sigma).\]
Equivalently, the space of directions $\Sigma_p=\Sigma_p\spc{P}$
can be isometrically identified with the 
$k$-times iterated suspension over $\Link_\sigma$;
that is, 
\[\Sigma_p\iso\Susp^{k}(\Link_\sigma).\]

If $\spc{P}$ is an $m$-dimensional polyhedral space,
then for any $p\in \spc{P}$
the space of directions $\Sigma_p$ is a spherical polyhedral space
of dimension at most $m-1$. 

In particular, 
for any point $p$ in $\sigma$,
the isometry class of $\Link_\sigma$ together with $k=\dim\sigma$
determines the isometry class of $\Sigma_p$, 
 and the other way around --- $\Sigma_p$ and $k$ determines the isometry class of $\Link_\sigma$.

A small neighborhood of $p$ is isometric to a neighborhood of the tip of $\Cone\Sigma_p$. 
(If $\spc{P}$ is a spherical polyhedral space, then a small neighborhood of $p$ is isometric to a neighborhood of the north pole in $\Susp\Sigma_p$.)
In fact, if this property holds at any point of a compact length space $\spc{P}$,
then  $\spc{P}$ is a polyhedral space \cite{lebedeva-petrunin}.

\section{CAT test}

The following theorem provides a combinatorial description of polyhedral spaces with curvature bounded above.


\begin{thm}{Theorem}\label{thm:PL-CAT}
Let $\spc{P}$ be a polyhedral space and $\tau$ be its triangulation. 
Then $\spc{P}$ is locally $\CAT(0)$ if and only if the link of each simplex in $\tau$ has no closed local geodesic shorter than $2\cdot\pi$.

Analogously, let $\spc{P}$ be a spherical polyhedral space and $\tau$ be its triangulation. 
Then $\spc{P}$ is $\CAT(1)$ if and only if neither $\spc{P}$ nor  the  link of any simplex in $\tau$ has a closed local geodesic shorter than $2\cdot\pi$.
\end{thm}





\parit{Proof.}
The ``only if'' part follows from \ref{cor:loc-geod-are-min}, \ref{prop:susp}, and \ref{ex:cone+susp}.

To prove the ``if'' part,
we apply induction on $\dim\spc{P}$.
The base case $\dim\spc{P}=0$ is evident.
Let us start with the $\CAT(1)$ case.

\parit{Step.}
Assume that the theorem is proved in the case $\dim\spc{P}<m$. Suppose  $\dim\spc{P}=m$.


Fix a point $p\in\spc{P}$.
A neighborhood of $p$ 
is isometric to a neighborhood of the north pole in the suspension over the space of directions~$\Sigma_p$.

Note that $\Sigma_p$ is a spherical polyhedral space, 
and its  links are isometric to  links of~$\spc{P}$. 
By the  induction hypothesis, $\Sigma_p$ is $\CAT(1)$.
Thus, by the second part of Exercise~\ref{ex:cone+susp}, $\spc{P}$ is locally  $\CAT(1)$.


Applying the second part of Corollary~\ref{cor:closed-geod-cat},
we get the statement.

The $\CAT(0)$ case is done in exactly the same way except we need to use the first part of Exercise~\ref{ex:cone+susp} and  the first part of Corollary~\ref{cor:closed-geod-cat} on the last step.
\qeds

\begin{thm}{Exercise}\label{ex:unique-geod=CAT}
Let $\spc{P}$
be a polyhedral space such that any two points can be connected by a unique geodesic.
Show that $\spc{P}$ is $\CAT(0)$.
\end{thm}

\begin{thm}{Advanced exercise}\label{ex:S3}
Construct a Euclidean polyhedral metric on $\mathbb{S}^3$
such that the total angle around each edge in its triangulation is at least $2\cdot \pi$.
\end{thm}


\section{Flag complexes}


\begin{thm}{Definition}\label{def:flag}
A simplicial complex $\mathcal{S}$ 
is called \index{flag complex}\emph{flag} if whenever $\{v^0,\z\dots,v^k\}$
is a set of distinct vertices of $\mathcal{S}$
that are pairwise joined by edges, then the vertices $v^0,\dots,v^k$
span a $k$-simplex in~$\mathcal{S}$.

If the above condition is satisfied for $k=2$, 
then we say that $\mathcal{S}$ satisfies 
the \index{no-triangle condition}\emph{no-triangle condition}.
\end{thm}

Note that every flag complex is determined by its one-skeleton.
Moreover, for any graph, its \index{clique}\emph{cliques} (that is, complete subgraphs) define a flag complex.
For that reason  flag complexes are also called  \index{clique complex}\emph{clique complexes}.

\begin{thm}{Exercise}\label{ex:baricenric-flag}
Show that the barycentric subdivision of any simplicial complex is a flag complex.

Use the flag condition (see \ref{thm:flag} below)
to conclude that any finite simplicial complex is homeomorphic to a proper length $\CAT(1)$ space.

\end{thm}


\begin{thm}{Proposition}\label{prop:no-trig}
A simplicial complex $\mathcal{S}$ is flag if and only if 
$\mathcal{S}$ as well as all the links of all its simplices
satisfy the no-triangle condition.
\end{thm}

From the definition of flag complex 
we get the following.

\begin{thm}{Observation}\label{obs:link-of-flag}
Any link of any simplex in a flag complex is flag.
\end{thm}


\parit{Proof of \ref{prop:no-trig}.}
By Observation~\ref{obs:link-of-flag}, the no-triangle condition holds 
for any flag complex and the  links of all its simplices.

Now assume that a complex $\spc{S}$ and all its links satisfy 
the no-triangle condition.
It follows that $\spc{S}$ includes a 2-simplex for each triangle.
Applying the same observation for each edge we get that $\spc{S}$ 
includes a 3-simplex for any complete graph with 4 vertices.
Repeating this observation 
for triangles, 
4-simplices,
5-simplices,
and so on, we get that $\spc{S}$ is flag.
\qeds


\parbf{All-right triangulation.}
A triangulation of a spherical polyhedral space 
is called an  \index{all-right triangulation}\emph{all-right triangulation} 
if each simplex of the triangulation is isometric 
to a spherical simplex all of whose angles are right.
Similarly, we say that a simplicial complex 
is equipped with an  \index{all-right spherical metric}\emph{all-right spherical metric}
if it is a length metric and each simplex is isometric 
to a spherical simplex all of whose angles are right.

Spherical polyhedral $\CAT(1)$ spaces glued from right-angled simplices
admit the following characterization 
discovered by Mikhael Gromov \cite[p.~122]{gromov-1987}.

\begin{thm}{Flag condition}\label{thm:flag}
Assume that a spherical polyhedral space $\spc{P}$
admits an all-right triangulation~$\tau$.
Then $\spc{P}$ is $\CAT(1)$
if and only if $\tau$ is flag.
\end{thm}

\parit{Proof; ``only if'' part.} 
Assume there are three vertices $v^1,v^2$ and $v^3$ of $\tau$
that are pairwise joined by edges 
but do not span a triangle.
Note that in this case 
\[
\mangle\hinge{v^1}{v^2}{v^3}=
\mangle\hinge{v^2}{v^3}{v^1}=
\mangle\hinge{v^3}{v^1}{v^2}=
\pi.
\]
Equivalently,
\begin{clm}{}\label{clm:3pi/2}
The product
of the geodesics $[v^1v^2]$, $[v^2v^3]$ and $[v^3v^1]$
forms a locally geodesic loop in~$\spc{P}$ of length $\tfrac32\cdot\pi$.
\end{clm}

Now assume that $\spc{P}$ is $\CAT(1)$.
Then by Theorem~\ref{thm:PL-CAT},
$\Link_\sigma\spc{P}$ is $\CAT(1)$ for every simplex $\sigma$ 
in~$\tau$. 

Each of these links is an all-right spherical complex and by Theorem \ref{thm:PL-CAT}, none of these links can contain a geodesic circle shorter than $2\cdot\pi$. 

Therefore Proposition~\ref{prop:no-trig} and \ref{clm:3pi/2} 
imply the ``only if'' part.

\parit{``If'' part.} 
By Observation~\ref{obs:link-of-flag} and Theorem~\ref{thm:PL-CAT},
it is sufficient to show that any closed local geodesic $\gamma$ 
in a flag complex $\spc{S}$ with all-right metric has length at least $2\cdot\pi$.

Recall that the  \index{star of vertex}\emph{closed star} of a vertex $v$ (briefly $\overline \Star_v$)
is formed by all the simplices containing~$v$. 
Similarly, $\Star_v$, the open star of $v$, is the union of all simplices containing $v$ with faces opposite $v$ removed.

Choose a vertex $v$ such that $\Star_v$ contains a point $\gamma(t_0)$ of $\gamma$.
Consider the maximal arc $\gamma_v$ of $\gamma$ 
that contains the point $\gamma(t_0)$
and runs in $\Star_v$.
Note that the distance $\dist{v}{\gamma_v(t)}{\spc{P}}$ behaves in exactly the same way 
as the distance from the north pole in~$\mathbb{S}^2$ to a geodesic in the north hemisphere;
that is, there is a geodesic $\tilde\gamma_v$ in the north hemisphere of $\mathbb{S}^2$ such that for any $t$ we have
\[\dist{v}{\gamma_v(t)}{\spc{P}}
=
\dist{n}{\tilde\gamma_v(t)}{\mathbb{S}^2},\]
where $n$ denotes the north pole of~$\mathbb{S}^2$.
In particular, 
\[\length\gamma_v=\pi;\]
that is, $\gamma$ spends time $\pi$ on every visit to $\Star_v$.


\begin{wrapfigure}{r}{45mm}
\vskip-2mm
\centering
\includegraphics{mppics/pic-1000}
\end{wrapfigure}

After leaving $\Star_v$,
the local geodesic $\gamma$ has to enter another simplex, 
say~$\sigma'$.
Since $\tau$ is flag, the simplex $\sigma'$
has a vertex $v'$ not joined to $v$ by an edge;
that is, 
\[\Star_v\cap\Star_{v'}=\emptyset\]

The same argument as above shows that $\gamma$ spends time $\pi$ on every visit to $\Star_{v'}$.
Therefore the total length of $\gamma$ is at least~$2\cdot\pi$.
\qeds

\begin{thm}{Exercise}\label{ex:flag>=pi/2}
Assume that a spherical polyhedral space $\spc{P}$
admits a triangulation $\tau$ such that all edge lengths of all simplices are at least~$\tfrac\pi2$.
Show that $\spc{P}$ is $\CAT(1)$
if $\tau$ is flag.
\end{thm}

\begin{thm}{Exercise}\label{ex:polyhedron-glue}
Let $P$ be a convex polyhedron in $\EE^3$ with $n$ faces $F_1,\dots, F_n$.
Suppose that each face of $P$ has only obtuse or right angles.
Let us take $2^n$ copies of $P$ indexed by $n$-bit array.
Glue two copies of $P$ along $F_i$ if their arrays differ only in $i$-th bit.
Show that the obtained space is a locally $\CAT(0)$ topological manifold.
\end{thm}

\parbf{The space of trees.}
The following construction is given by
Louis Billera,
Susan Holmes,
and  Karen Vogtmann in~\cite{billera-holmes-vogtmann}.

Let $\spc{T}_n$ be the set of all metric trees with 
$n$ end vertices
labeled by $a^1,\dots,a^n$.
To describe one tree in $\spc{T}_n$ we may fix a topological tree $t$ with end vertices $a^1,\dots,a^n$ 
and all  other vertices of degree 3,  
and prescribe the lengths of $2\cdot n-3$ edges.
If the length of an edge vanishes, we assume that this edge degenerates;
such a tree can be also described using a different topological tree~$t'$.
The subset of $\spc{T}_n$ corresponding to the given topological tree $t$ can be identified with the octant
\[\set{(x_1,\dots,x_{2\cdot n-3})\in\mathbb{R}^{2\cdot n-3}}{x_i\ge 0}.\]
Equip each such subset with the metric induced from $\mathbb{R}^{2\cdot n-3}$ and consider the length metric on $\spc{T}_n$ induced by these metrics.

\begin{thm}{Exercise}\label{ex:tree}
Show that $\spc{T}_n$ with the described metric is $\CAT(0)$.
\end{thm}

\section{Remarks}

Let us formulate a test for spaces with lower curvature bound.

\begin{thm}{Theorem}\label{thm:poly-CBB} Let $\spc{P}$ be a polyhedral space and $\tau$ be a triangulation of $\spc{P}$.
Then $\spc{P}$ is $\CBB(0)$ if and only if the following conditions hold.

\begin{subthm}{} $\tau$ is \index{pure complex}\emph{pure}; 
that is, any simplex in $\tau$ is  a face of some simplex of dimension exactly $m$. 
\end{subthm}

\begin{subthm}{thm:poly-CBB:m-1}
The link of any simplex of dimension $m-1$ is formed by single point or two points.
\end{subthm}

\begin{subthm}{thm:poly-CBB:connected}
The link of any simplex of dimension $\le m-2$ is connected.
\end{subthm}

\begin{subthm}{thm:poly-CBB:2pi}
Any link of any simplex of dimension $m-2$
has diameter at most~$\pi$.
\end{subthm}
\end{thm}

The proof relies on \ref{thm:cbb-globalization}.
The condition \ref{SHORT.thm:poly-CBB:connected}
can be reformulated in the following way:

\begin{itemize}
 \item[{\it \ref{SHORT.thm:poly-CBB:connected}$'$}] 
Any path $\gamma\:[0,1]\to \spc{P}$ can be approximated by paths
$\gamma_n\:[0,1]\to \spc{P}$ 
that cross only simplexes of dimension $m$ and $m-1$.
\end{itemize}

Further, modulo the other conditions,
the condition \ref{SHORT.thm:poly-CBB:2pi}
is equivalent to the following:


\begin{itemize}
 \item[{\it \ref{SHORT.thm:poly-CBB:2pi}$'$}] 
The link of any simplex of dimension $m-2$ is 
isometric to a circle of length $\le 2\cdot\pi$
or a closed real interval of length $\le \pi$.
\end{itemize}

