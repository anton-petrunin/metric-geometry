\parbf{\ref{ex:diam}.}
Suppose that $\dist{A}{B}{\Haus\spc{X}}<r$.
Choose a pair of points $a,a'\in A$ on maximal distance from each other.
Observe that there are points $b,b'\in B$ such that 
$\dist{a}{b}{\spc{X}},\dist{a'}{b'}{\spc{X}}<r$.
Whence 
\[\dist{a}{a'}{\spc{X}}-\dist{b}{b'}{\spc{X}}\le 2\cdot r\]
and therefore
\[\diam A-\diam B\le 2\cdot\dist{A}{B}{\Haus\spc{X}}.\]

It remains to swap $A$ and $B$ and repeat the argument.


\parbf{\ref{ex:Hausdorff-bry}}; \ref{SHORT.ex:Hausdorff-bry:conv}.
Denote by $A^r$ the closed $r$-neighborhood of a set $A\z\subset \RR^2$.
Observe  that 
\[(\Conv A)^r=\Conv(A^r),\]
and try to use it.

\parit{\ref{SHORT.ex:Hausdorff-bry:bry}.}
The answer is ``no'' in both parts.

For the first part let $A$ be a unit disk and $B$ a finite $\eps$-net in $A$.
Evidently, $|A-B|_{\Haus\RR^2}<\eps$, 
but
$|\partial A-\partial B|_{\Haus\RR^2}\approx 1$.

For the second part take $A$ to be a unit disk and $B=\partial A$ to be its boundary circle.
Note that $\partial A=\partial B$; in particular, $\dist{\partial A}{\partial B}{\Haus\RR^2}=0$ while $\dist{ A}{B}{\Haus\RR^2}=1$.

\parit{Remark.}
There are the so-called {}\emph{lakes of Wada} --- an example of three (and more) open bounded topological disks in the plane that have identical boundaries.
It can be used to construct more interesting examples for \ref{SHORT.ex:Hausdorff-bry:bry}.

\parbf{\ref{ex:H-sections}};
\ref{SHORT.ex:H-sections:S}.
Given $t\in (0,1]$, consider the real interval $\tilde C_t=[\tfrac 1t+t, \tfrac 1t+1]$.
Denote by $C_t$ the image of $\tilde C_t$ under the covering map $\pi\:\RR\to \mathbb{S}^1\z=\RR/\ZZ$.

Set $C_0=\mathbb{S}^1$.
Note that the Hausdorff distance from $C_0$ to $C_t$ is $\tfrac t2$.
Therefore $\{C_t\}_{t\in[0,1]}$ is a family of compact subsets in $\mathbb{S}^1$ that is continuous in the sense of Hausdorff.

Assume there is a continuous section $c(t)\in C_t$, for $t\in [0,1]$.
Since $\pi$ is a covering map,
we can lift the path $c$ to a path $\tilde c\:[0,1]\to \RR$ such that $\tilde c(t)\in \tilde C_t$ for all $t$.
In particular $\tilde c(t)\to\infty$ as $t\to0$,
a contradiction.

\parit{\ref{SHORT.ex:H-sections:R}.}
Consider path $c(t)\df\min C_t$.

\parit{Source:} The problem had been suggested by Stephan Stadler.

\parbf{\ref{ex:closure-union}.} 
Show that for any $\eps>0$ there is a positive integer $N$ such that $\bigcup_{n\le N} K_n$ is an $\eps$-net in the union $\bigcup_{n} K_n$.
Observe that $\bigcup_{n\le N} K_n$
is compact and apply \ref{ex:compact-net} and \ref{totally-bounded}.

\parbf{\ref{ex:Haus-length}}; \textit{if part.}
Choose two compact sets $A,B\subset \spc{X}$;
suppose that $\dist{A}{B}{\Haus\spc{X}}<r$.

Choose finite $\eps$-nets $\{a_1,\dots,a_m\}\subset A$ and $\{b_1,\dots b_n\}\subset B$.
For each pair $a_i,b_j$ construct a constant-speed path $\gamma_{i,j}$ from $a_i$ to $b_j$ such that 
\[\length \gamma_{i,j}<\dist{a_i}{b_j}{}+\eps.\]
Set 
\[C(t)=\set{\gamma_{i,j}(t)}{\dist{a_i}{b_j}{\spc{X}}<r+\eps}.\]
Observe that $C(t)$ is finite; in particular, it is compact.

Show and use that 
\begin{align*}
\dist{A}{C(t)}{\spc{X}}&<t\cdot r+10\cdot\eps,
\\
\dist{C(t)}{B}{\spc{X}}&<(1-t)\cdot r+10\cdot\eps.
\end{align*}
Apply \ref{ex:closure-union} and \ref{lem:mid>geod}.

\parit{Only-if part.}
Choose points $p,q\in\spc{X}$. 
Show that the existence of $\eps$-midpoints between $\{p\}$ and $\{q\}$ in $\Haus\spc{X}$ implies the existence of $\eps$-midpoints between $p$ and $q$ in $\spc{X}$.
Apply \ref{lem:mid>geod}.


\parbf{\ref{ex:Huas-perimeter-area}.}
Let $A$ be a compact convex set in the plane.
Denote by $A^r$ the closed $r$-neighborhood of $A$.
Recall that by Steiner's formula we have
\[\area A^r=\area A+r\cdot\perim A+\pi\cdot r^2.\]
Taking derivative and applying the coarea formula, we get 
\[\perim A^r=\perim A+2\cdot\pi\cdot r.\]

Observe that if $A$ lies in a compact set $B$ bounded by a closed curve, then 
\[\perim A\le \perim B.\]
Indeed the closest-point projection $\RR^2\to A$ is short and it maps $\partial B$ onto $\partial A$.

It remains to use the following observation: if $A_n\to A_\infty$, then for any $r>0$ we have that the inclusions
\[A_\infty^r\supset A_n
\quad\text{and}\quad
A_\infty\subset A_n^r\]
hold for all large $n$.

\begin{wrapfigure}{r}{27 mm}
\vskip-0mm
\centering
\includegraphics{mppics/pic-410}
\end{wrapfigure}

\parbf{\ref{ex:round-disc}.}
Note that almost all points on $\partial D$ have a defined tangent line.
In particular, for almost all pairs of points $a,b\z\in\partial D$ the two angles $\alpha$ and $\beta$ between the chord $[ab]$ and $ \partial D$ are defined.

The convexity of $D'$ implies that $\alpha=\beta$;
here we measure the angles $\alpha$ and $\beta$ on one side from $[ab]$.
Show that if the identity $\alpha=\beta$ holds for almost all chords, then $D$ is a round disk. 


\parbf{\ref{ex:generalized-selection}.}
Observe that all functions $\distfun_{A_n}$ are Lipschitz.

If they are not uniformly bounded on compact sets, then we can pass to a subsequence of $A_n$ so that $\distfun_{A_n}(x)\to\infty$ for any $x$;
in this case $A_n$ converges to the empty set.

If the functions are uniformly bounded, then, passing to a subsequence, we may assume that the sequence $\distfun_{A_n}$ converges to some function $f$.

Set $A_\infty=f^{-1}\{0\}$.
It remains to show that $f=\distfun_{A_\infty}$.

%%%%%%%%%%%%%%%%%%%%%%%%%%%%%%

\parbf{\ref{ex:d_GH-and-diam}};
\ref{SHORT.ex:d_GH-and-diam:point}.
Apply the definition for space $\spc{Z}$ obtained from $\spc{X}$ by adding a point that lies at distance $\tfrac12\cdot\diam \spc{X}$ from each point of $\spc{X}$.

\parit{\ref{SHORT.ex:d_GH-and-diam:scale}.}
Given a point $x\in\spc{X}$, denote by $a\cdot x$ and $b\cdot x$ the corresponding points in $a\cdot\spc{X}$ and $b\cdot \spc{X}$ respectively.
Show that there is a metric on $\spc{Z}\z=a\cdot\spc{X}\sqcup b\cdot\spc{X}$ such that 
\[|a\cdot x-b\cdot x|_{\spc{Z}}=\tfrac{|b-a|}2\cdot\diam\spc{X}\]
for any $x$ and the inclusions
$a\cdot\spc{X}\hookrightarrow\spc{Z}$,
$b\cdot\spc{X}\hookrightarrow\spc{Z}$ are distance preserving.

\parit{\ref{SHORT.ex:d_GH-and-diam:isometry}.}
Use \ref{SHORT.ex:d_GH-and-diam:point} and \ref{SHORT.ex:d_GH-and-diam:scale} to show that the isometry class of $\spc{O}$ is completely determined by the following property
\[|\spc{X}-\spc{Y}|_{\GH} \le \max\{\,|\spc{O}-\spc{X}|_{\GH},|\spc{O}-\spc{Y}|_{\GH}\,\}.\]
for any $\spc{X}$ and $\spc{Y}$.

\parit{Remark.}
In fact, \textit{the isometry group of space $\GH$ is trivial}.
The latter was proved by George Lowther \cite{lowther, ivanov-tuzhilin}.

\parbf{\ref{ex:GH<H}.}
Check a one-point set and the vertices of an equilateral triangle.
You may use \ref{ex:d_GH-and-diam:point}.


\parbf{\ref{ex:rectangle}.}
Arguing by contradiction,
we can identify $\spc{A}_r$ and $\spc{B}_r$ with subspaces of a space $\spc{Z}$
such that 
\[|\spc{A}_r-\spc{B}_r|_{\Haus \spc{Z}}<\tfrac1{10}\]
for large $r$; see the definition of Gromov--Hausdorff metric (\ref{def:GH}).

Set $n=\lceil r \rceil$.
Note that there are $2\cdot n$ integer points in~$\spc{A}_r$: 
$a_1\z=(0,0)$, $a_2=(1,0),\dots,a_{2\cdot n}=(n,1)$.
Choose a point $b_i\in \spc{B}_r$ that lies at the minimal distance from $a_i$.
Note that $|b_i-b_j|>\tfrac 45$ if $i\ne j$.
It follows that $r>\tfrac 45\cdot (2\cdot n-1)$.
The latter contradicts $n=\lceil r \rceil$ for large~$r$.

\parit{Remark.}
Try to show that $|\spc{A}_r-\spc{B}_r|_{\GH}=\tfrac12$ for all large $r$.

\parbf{\ref{ex:GH-inj}.}
Suppose that $|\spc{X}-\spc{Y}|_{\spc{U}}<\eps$;
we need to show that 
\[|\hat{\spc{X}}-\hat{\spc{Y}}|_{\GH}<2\cdot \eps.\]

Denote by $\hat{\spc{U}}$ the injective envelope of $\spc{U}$.
Recall that $\spc{U}$, $\spc{X}$, and $\spc{Y}$ can be considered as subspaces of $\hat{\spc{U}}$, $\hat{\spc{X}}$, and $\hat{\spc{Y}}$ respectively.

According to \ref{ex:d-p-inclusion}, the inclusions $\spc{X}\hookrightarrow\spc{U}$ and $\spc{Y}\hookrightarrow\spc{U}$ can be extended to distance-preserving inclusions $\hat{\spc{X}}\hookrightarrow\hat{\spc{U}}$ and $\hat{\spc{Y}}\hookrightarrow\hat{\spc{U}}$.
Therefore, we can and will consider  $\hat{\spc{X}}$ and $\hat{\spc{Y}}$ as subspaces of $\hat{\spc{U}}$.

Given $f\in \hat{\spc{U}}$,
let us find $g\in \hat{\spc{X}}$ such that 
\[|f(u)-g(u)|<2\cdot\eps\eqlbl{|g-f|}\]
for any $u\in\spc{U}$.
Note that the restriction $f|_{\spc{X}}$ is admissible on ${\spc{X}}$.
By \ref{obs:extremal:below}, there is $g\in \hat{\spc{X}}$ such that 
\[g(x)\le f(x)\eqlbl{g(x)=<f(x)}\]
for any $x\in\spc{X}$.

Recall that any extremal function is $1$-Lipschitz;
in particular, $f$ and $g$ are $1$-Lipschitz on $\spc{U}$.
Therefore, \ref{g(x)=<f(x)} and $|\spc{X}-\spc{Y}|_{\spc{U}}<\eps$ imply that
\[g(u)< f(u)+2\cdot \eps\]
for any $u\in\spc{U}$.
By \ref{ex:+-c}, we also have 
\[g(u)> f(u)-2\cdot \eps\]
for any $u\in\spc{U}$.
Whence \ref{|g-f|} follows.

It follows that $\hat{\spc{Y}}$ lies in a $2\cdot\eps$-neighborhood of $\hat{\spc{X}}$ in $\hat{\spc{U}}$.
The same way we show that $\hat{\spc{X}}$ lies in a $2\cdot\eps$-neighborhood of $\hat{\spc{Y}}$ in $\hat{\spc{U}}$.
The latter means that
$|\hat{\spc{X}}-\hat{\spc{Y}}|_{\Haus\hat{\spc{U}}}<2\cdot\eps$,
and therefore
$|\hat{\spc{X}}-\hat{\spc{Y}}|_{\GH}<2\cdot\eps$.

\parit{Remark.} 
This problem was discussed by Urs Lang, Maël Pavón, and Roger Züst \cite[3.1]{lang-pavon-zust}.
\begin{figure}[ht!]
\vskip-0mm
\centering
\includegraphics{mppics/pic-505}
\end{figure}
They also show that the constant 2 is optimal.
To see this, look at the injective envelopes of two 4-point metric spaces shown on the diagram and observe that the Gromov--Hausdorff distance between the 4-point metric spaces is 1, while the distance between their injective envelopes approaches 2 as $s\to\infty$. 

\parbf{\ref{ex:H-R}}; \textit{only-if part.}
Let us identify $\spc{X}$ and $\spc{Y}$ with subspaces of a metric space $\spc{Z}$ such that 
\[|\spc{X}-\spc{Y}|_{\Haus \spc{Z}}<\eps.\]

Set $x\approx y$ if and only if $\dist{x}{y}{\spc{Z}}<\eps$.
It remains to check that $\approx$ is an $\eps$-approximation.

\parit{If part.}
Show that we can assume that 
\[R=\set{(x,y)\in\spc{X}\times\spc{Y}}{x\approx y}\] is a compact subset of $\spc{X}\times\spc{Y}$.
Conclude that
\[\bigl|\dist{x}{x'}{\spc{X}}-\dist{y}{y'}{\spc{Y}}\bigr|<2\cdot\eps'\]
for some $\eps'<\eps$.

Show that there is a metric on $\spc{Z}=\spc{X}\sqcup\spc{Y}$ such that the inclusions $\spc{X}\hookrightarrow\spc{Z}$ and
$\spc{Y}\hookrightarrow\spc{Z}$ are distance preserving and $\dist{x}{y}{\spc{Z}}=\eps'$ if $x\approx y$.
Conclude that 
\[|\spc{X}-\spc{Y}|_{\Haus \spc{Z}}\le\eps'<\eps.\]

\parbf{\ref{ex:eps-isom}};
\ref{SHORT.ex:eps-isom:GH>isom}.
Let $\approx$ be an $\eps$-approximation provided by \ref{ex:H-R}.
For any $x\in\spc{X}$ choose a point $f(x)\in\spc{Y}$ such that $x\approx f(x)$.
Show that $x\mapsto f(x)$ is an $2\cdot\eps$-isometry.

\parit{\ref{SHORT.ex:eps-isom:isom>GH}.}
Let $x\in\spc{X}$ and $y\in\spc{Y}$.
Set $x\approx y$ if $\dist{y}{f(x)}{\spc{Y}}<\eps$.
Show that $\approx$ is an $\eps$-approximation. 
Apply \ref{ex:H-R}.

\parbf{\ref{ex:XYZ}.}
Consider the product space $[0,\eps]\times \ZZ_n$ with the natural $\ell^\infty$-product metric and make three variations of it by changing the size of the segments.

\parbf{\ref{ex:GH-SC}}; \ref{SHORT.ex:GH-SC:circle}.
Suppose $\spc{X}_n\GHto \spc{X}$ and $\spc{X}_n$ are simply-connected length metric space.
It is sufficient to show that any nontrivial covering map $f\:\tilde{\spc{X}}\to \spc{X}$ corresponds to a nontrivial covering map $f_n\:\tilde{\spc{X}}_n\to \spc{X}_n$ for large $n$.

The latter can be constructed by covering $\spc{X}_n$ with small balls that lie close to sets in $\spc{X}$ evenly covered by $f$, prepare a few copies of these sets and glue them the same way as the inverse images of the evenly covered sets in $\spc{X}$ glued to obtain $\tilde{\spc{X}}$.

\begin{wrapfigure}{r}{42 mm}
\vskip-4mm
\centering
\includegraphics{mppics/pic-2}
\end{wrapfigure}

\parit{\ref{SHORT.ex:GH-SC:nonsc-limit}.}
Let $\spc{V}$ be a cone over Hawaiian earrings.
Consider the {}\emph{doubled cone} $\spc{W}$ --- two copies of $\spc{V}$ with glued base points (see the diagram).

The space $\spc{W}$ can be equipped with a length metric
(for example, the induced length metric from the shown embedding).

Show that $\spc{V}$ is simply-connected, but $\spc{W}$ is not; follows from the van Kampen theorem.

If we delete from the earrings all small circles, then the obtained double cone becomes simply-connected and it remains to be close to $\spc{W}$.
That is $\spc{W}$ is a Gromov--Hausdorff limit of simply-connected spaces.

\parit{Remark.}
Note that the limit space in \ref{SHORT.ex:GH-SC:nonsc-limit}, does not admit a nontrivial covering.

\parbf{\ref{ex:sphere-to-ball},}
\textit{\ref{SHORT.ex:sphere-to-ball:2}.}
Suppose that a metric on $\mathbb{S}^2$ is close to the unit disk $\DD^2$.
Note that $\mathbb{S}^2$ contains a circle $\gamma$ that is close to the boundary curve of $\DD^2$.
By the Jordan curve theorem, $\gamma$ divides $\mathbb{S}^2$ into two disks, say $D_1$ and $D_2$.

By \ref{ex:GH-SC:circle}, the Gromov--Hausdorff limits of $D_1$ and $D_2$ have to contain the whole $\DD^2$, otherwise the limit would admit a nontrivial covering.

Consider points $p_1\in D_1$ and $p_2\in D_2$ that are close to the center of $\DD^2$.
If $n$ is large, the distance $\dist{p_1}{p_2}{n}$ has to be very small.
On the other hand, any curve from $p_1$ to $p_2$ must cross $\gamma$, so it has length about 2 --- a contradiction.

\parit{\ref{SHORT.ex:sphere-to-ball:3}.}
Make fine burrows in the standard 3-ball without changing its topology,
but at the same time come sufficiently close to any point in the ball.

Consider the \index{doubling}\emph{doubling} of the obtained ball along its boundary;
that is, two copies of the ball with identified corresponding points on their boundaries.
The obtained space is homeomorphic to $\mathbb{S}^3$.
Note that the burrows can be made 
so that the obtained space is sufficiently close to the original ball 
in the Gromov--Hausdorff metric.

\parit{Source:} \cite[Exercises 7.5.13 and 7.5.17]{burago-burago-ivanov}. 

\parbf{\ref{ex:utb+pack}.} Apply \ref{ex:pack-net}.

\parbf{\ref{pr:doubling}.}
Choose a space $\spc{X}$ in $\spc{Q}(C,D)$, denote a $C$-doubling measure by~$\mu$.
Without loss of generality, we may assume that $\mu(\spc{X})\z=1$.

The doubling condition implies that 
\[\mu[\oBall(p,\tfrac D{2^n})]\ge\tfrac 1{C^n}\]
for any point $x\in \spc{X}$.
It follows that 
\[\pack_{\frac D{2^n}}\spc{X}\le C^n.\]

By \ref{ex:pack-net}, for any $\eps\ge\frac D{2^{n-1}}$, the space $\spc{X}$ admits an $\eps$-net with at most $C^n$ points.
Whence $\spc{Q}(C,D)$ is uniformly totally bounded.

\parbf{\ref{pr:under}}; \ref{SHORT.pr:under:if}.
Choose $\eps>0$.
Since $\spc{Y}$ is compact, we can choose a finite $\eps$-net $\{y_1,\dots,y_{n}\}$ in $\spc{Y}$.

Suppose $f\:\spc{X}\to \spc{Y}$ be a distance-noncontracting map.
Choose one point $x_i$ in each nonempty subset $B_i=f^{-1}[\oBall(y_i,\eps)]$.
Note that the subset $B_i$ has diameter at most $2\cdot \eps$ and 
\[\spc{X}=\bigcup_iB_i.\]
Therefore, the set of points $\{x_i\}$ is a $2\cdot\eps$-net in $\spc{X}$.

\parit{\ref{SHORT.pr:under:only-if}.} Let $\spc{Q}$ be a uniformly totally bounded family of spaces. 
Suppose that each space in $\spc{Q}$ has an $\tfrac1{2^n}$-net with at most $M_n$ points; we may assume that $M_0=1$.

Consider the space $\spc{Y}$ of all infinite integer sequences $m_0,m_1,\dots$ such that $1\le m_n\le M_n$ for any $n$.
Given two sequences $\bm{\ell}\z=(\ell_1,\ell_2,\dots)$, and $\bm{m}\z=(m_1,m_2,\dots)$ of points in $\spc{Y}$, set 
\[\dist{\bm{\ell}}{\bm{m}}{\spc{Y}}=\tfrac C{2^{n}},\]
where $n$ is the minimal index such that $\ell_n\ne m_n$ and $C$ is a positive constant.

Observe that $\spc{Y}$ is compact.
Indeed it is complete and the sequences with constant tails, starting from index $n$, form a finite $\tfrac C{2^{n}}$-net in $\spc{Y}$.

Given a space $\spc{X}$ in $\spc{Q}$,
choose a sequence of $\tfrac1{2^n}$ nets 
$N_n\subset\spc{X}$ for each $n$.
We can assume that $|N_n|\le M_n$; let us label the points in $N_n$ by $\{1,\dots,M_n\}$.
Consider the map $f\:\spc{X}\to\spc{Y}$ defined by $f:x\z\mapsto (m_1(x),m_2(x),\dots)$ where $m_n(x)$ is the label of a point in $N_n$ that lies at the distance $<\tfrac1{2^n}$ from $x$.

If $\tfrac1{2^{n-2}}\ge \dist{x}{x'}{\spc{X}}>\tfrac1{2^{n-1}}$, then $m_n(x)\ne m_n(x')$.
It follows that $\dist{f(x)}{f(x')}{\spc{Y}}\ge \tfrac C{2^{n}}$.
In particular, if $C>10$, then 
\[\dist{f(x)}{f(x')}{\spc{Y}}\ge \dist{x}{x'}{\spc{X}}\]
for any $x,x'\in \spc{X}$.
That is, $f$ is a distance-noncontracting map $\spc{X}\to \spc{Y}$.

\parbf{\ref{ex-GH-length}}; \ref{SHORT.ex-GH-length:length}
Apply
\ref{ex:Haus-length},
\ref{prop:GH-with-fixed-Z},
\ref{lem:GH-complete},
and \ref{lem:mid>geod}.

\parit{\ref{SHORT.ex-GH-length:geodesic}.}
Choose two compact metric spaces $\spc{X}$ and $\spc{Y}$.
Show that there are subsets $\spc{X}'$, and $\spc{Y}'$ in the Urysohn space $\spc{U}$ that are isometric to $\spc{X}$ and $\spc{Y}$ respectively, and such that 
\[|\spc{X}-\spc{Y}|_{\GH} = |\spc{X}'-\spc{Y}'|_{\Haus\spc{U}}.\]

Further, construct a sequence of compact sets $\spc{Z}_n\subset \spc{U}$ such that $\spc{Z}_n$ is an $\tfrac1{2^n}$-midpoint of $\spc{X}'$, and $\spc{Y}'$ in $\Haus\spc{U}$ and 
\[|\spc{Z}_n-\spc{Z}_{n+1}|_{\Haus\spc{U}}<\tfrac1{2^n}\]
for any $n$.

Observe that the sequence $\spc{Z}_n$ converges in $\GH$, and its limit by $\spc{Z}$ is a midpoint of $\spc{X}$ and $\spc{Y}$.
Finally, apply \ref{lem:GH-complete} and \ref{lem:mid>geod}.

\parit{Source:} \cite{ivanov-nikolaeva-tuzhilin}.

\parbf{\ref{ex:GH-po}}; \ref{SHORT.ex:GH-po:a}.
To check that $\dist{*}{*}{\GH'}$ is a metric, it is sufficient to show that
\[\dist{\spc{X}}{\spc{Y}}{\GH'}=0 
\quad\Longrightarrow\quad
\spc{X}\iso\spc{Y};\]
the remaining conditions are trivial.

If $\dist{\spc{X}}{\spc{Y}}{\GH'}=0$, then there is a sequence of maps $f_n\:\spc{X}\to \spc{Y}$ such that 
\[\dist{f_n(x)}{f_n(x')}{\spc{Y}}\ge \dist{x}{x'}{\spc{X}}-\tfrac1n.\]

Choose a countable dense subset $S\subset \spc{X}$ and pass to a subsequence such that $f_n(x)$ converges for any $x\in S$; denote by $f_\infty\:S\to \spc{Y}$ the limit map.
Note that $f_\infty$ is distance-noncontracting, and it can be extended to a distance-noncontracting map $f_\infty\:\spc{X}\to \spc{Y}$.

The same way we can construct a distance-noncontracting map 
$g_\infty\:\spc{Y}\to \spc{X}$.

By \ref{ex:non-contracting-map}, the compositions $f_\infty\circ g_\infty\:\spc{Y}\to \spc{Y}$ and $g_\infty\z\circ f_\infty\:\spc{X}\to \spc{X}$ are isometries.
Therefore, $f_\infty$ and $g_\infty$ are isometries as well.

\parit{\ref{SHORT.ex:GH-po:b}.} The implication 
\[|\spc{X}_n-\spc{X}_\infty|_{\GH}\to 0 
\quad\Rightarrow\quad 
\dist{\spc{X}_n}{\spc{X}_\infty}{\GH'}\to 0\]
follows from \ref{ex:eps-isom:GH>isom}. 

Now suppose $\dist{\spc{X}_n}{\spc{X}_\infty}{\GH'}\to 0$.
Show that $\{\spc{X}_n\}$ is a uniformly totally bonded family.

If $\dist{\spc{X}_n}{\spc{X}_\infty}{\GH}\not\to 0$, then we can pass to a subsequence such that $\dist{\spc{X}_n}{\spc{X}_\infty}{\GH}\ge\eps$ for some $\eps>0$.
By Gromov selection theorem, we can assume that $\spc{X}_n$ converges in the sense of Gromov--Hausdorff.
From the first implication, the limit $\spc{X}_\infty'$ has to be isometric to $\spc{X}_\infty$;
on the other hand, $\dist{\spc{X}_\infty'}{\spc{X}_\infty}{\GH}\ge \eps$ --- a contradiction.

\parbf{\ref{ex:GH-urysohn}.}
Apply \ref{thm:compact-homogeneous} and \ref{prop:GH-with-fixed-Z}.

%%%%%%%%%%%%%%%%%%%%%%%%%%%%%%%%
