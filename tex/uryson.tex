\chapter{Universal spaces}

\section{Embedding in a normed space}

Recall that a function $v\mapsto |v|$ on a vector space $\spc{V}$ is called \index{norm}\emph{norm} if it satisfies the following condition for any two vectors $v,w\in \spc{V}$ and a scalar $\alpha$:
\begin{itemize}
\item $|v|\ge 0$;
\item $|\alpha\cdot v|=|\alpha|\cdot |v|$;
\item $|v|+|w|\ge|v+w|$.
\end{itemize}

As an example, consider \index{$\ell^\infty$}$\ell^\infty$ --- the space of real sequences equipped with \index{sup-norm}\emph{sup-norm}; that is, the norm of $\bm{a}=a_1,a_2,\dots$ is defined by
\[|\bm{a}|_{\ell^\infty}=\sup_n\{\,|a_n|\,\}.\]


It is straightforward to check that for any normed space the function $(v,w)\mapsto |v-w|$ defines a metric on it.
Therefore, any normed space is an example of metric space (in fact, it is a geodesic space).
Often we do not distinguish between normed space and the corresponding metric space.
(In fact by Mazur--Ulam theorem, the metric remembers the affine structure of the space; so to recover the original normed space we only need to specify the origin.
A slick proof of Mazur--Ulam theorem was given by Jussi V\"{a}is\"{a}l\"{a} \cite{vaisala}.)

Now let us show that reasonable metric spaces are isometric to subsets of $\ell^\infty$.

Recall that \index{diameter}\emph{diameter} of a metric space $\spc{X}$ (briefly $\diam \spc{X}$) is defined as the least upper bound on the distances between pairs of its points;
that is,
\[\diam \spc{X}=\sup\set{\dist{x}{y}{\spc{X}}}{x,y\in \spc{X}}.\]



\begin{thm}{Lemma}\label{lem:frechet}
Suppose $\spc{X}$ is a \index{bounded space}\emph{bounded} \index{separable space}\emph{separable} metric space;
that is, $\diam\spc{X}$ is finite and $\spc{X}$ contains a countable, dense set $\{w_n\}$.
Given $x\in \spc{X}$, set $a_n(x)=\dist{w_n}{x}{\spc{X}}$.
Then 
\[\iota\:x\mapsto (a_1(x), a_2(x),\dots)\]
defines a distance-preserving embedding $\iota\:\spc{X}\hookrightarrow \ell^\infty$.
\end{thm}

\parit{Proof.} 
By the triangle inequality 
\[|a_n(x)-a_n(y)|\le \dist{x}{y}{\spc{X}}.\eqlbl{eq:a-a=<dist}\]
Therefore, $\iota$ is \index{short map}\emph{short} (in other words, $\iota$ is distance-noncontracting).

Again by triangle inequality we have 
\[|a_n(x)-a_n(y)|\ge \dist{x}{y}{\spc{X}}-2\cdot\dist{w_n}{x}{\spc{X}}.\]
Since the set $\{w_n\}$ is dense, we can choose $w_n$ arbitrarily close to $x$.
Whence the value $|a_n(x)-a_n(y)|$ can be chosen arbitrarily close to $\dist{x}{y}{\spc{X}}$.
In other words 
\[\sup_n\left\{\,\bigl|\dist{w_n}{x}{\spc{X}}-\dist{w_n}{y}{\spc{X}}\bigr|\,\right\}
\ge
\dist{x}{y}{\spc{X}}.\]
Hence 
\[\sup_n\{\,|a_n(x)-a_n(y)|\,\}\ge \dist{x}{y}{\spc{X}};\eqlbl{eq:a-a>=dist}\]
that is, $\iota$ is distance-noncontracting.

Finally, observe that \ref{eq:a-a=<dist} and \ref{eq:a-a>=dist} imply the lemma.
\qeds

\begin{thm}{Exercise}\label{ex:compact-length}
Show that any compact metric space $\spc{K}$ is isometric to a subspace of a compact geodesic space. 
\end{thm}

The following exercise generalizes the lemma to arbitrary separable spaces.

\begin{thm}{Exercise}\label{ex:frechet}
Suppose $\{w_n\}$ is a countable, dense set in a metric space $\spc{X}$.
Choose $x_0\in \spc{X}$;
given $x\in \spc{X}$, set 
\[a_n(x)=\dist{w_n}{x}{\spc{X}}-\dist{w_n}{x_0}{\spc{X}}.\]
Show that $\iota\:x\mapsto (a_1(x), a_2(x),\dots)$ defines a distance-preserving embedding $\iota\:\spc{X}\hookrightarrow \ell^\infty$.
\end{thm}

The following lemma implies that {}\textit{any metric space is isometric to a subset of a normed vector space};
its proof is nearly identical to the proof of \ref{ex:frechet}.

\begin{thm}{Lemma}\label{lem:kuratowski}
Let $\spc{X}$ be arbitrary metric space.
Denote by $\ell^\infty(\spc{X})$ the space of all bounded functions on $\spc{X}$ equipped with sup-norm.

Then for any point $x_0\in \spc{X}$, the map $\iota\:\spc{X}\to \ell^\infty(\spc{X})$ defined by 
\[\iota\:x\mapsto (\distfun_x-\distfun_{x_0})\]
is distance-preserving.
\end{thm}

\section{Extension property}
\label{sec:Extension property}

If a metric space $\spc{X}$ is a subspace of a pseudometric space $\spc{X}'$, then we say that $\spc{X}'$ is an \index{extension}\emph{extension} of $\spc{X}$.
If in addition, $\diam\spc{X}'\le d$, then we say that $\spc{X}'$ is a {}\emph{$d$-extension}.

If the complement $\spc{X}'\setminus \spc{X}$ contains a single point, say $p$, we say that $\spc{X}'$ is a \index{one-point extension}\emph{one-point extension} of $\spc{X}$.
In this case, to define a metric on $\spc{X}'$, it is sufficient to specify the distance function from $p$; that is, a function $f\:\spc{X}\to\RR$ defined by 
\[f(x)=\dist{p}{x}{\spc{X}'}.\]
Any function $f$ of that type will be called \index{extension function}\emph{extension function}\label{page:extension function} or {}\emph{$d$-extension function} respectively.

The extension function $f$ cannot be taken arbitrary --- the triangle inequality implies that 
\[f(x)+f(y)\ge \dist{x}{y}{\spc{X}}\ge |f(x)-f(y)|\]
for any $x,y\in \spc{X}$.
In particular, $f$ is a non-negative 1-Lipschitz function on $\spc{X}$.
For a $d$-extension, we need to assume in addition that $\diam\spc{X}\z\le d$ and $f(x)\le d$ for any $x\in \spc{X}$.
It is easy to see that these conditions are necessary and sufficient.

\begin{thm}{Exercise}\label{ex:extension-of-extension}
Let $\spc{X}$ be a subspace of metric space $\spc{Y}$.
Assume $f$ is an extension function on $\spc{X}$.

\begin{subthm}{ex:extension-of-extension:a}
Show that 
\[\bar f(y)
\df
\inf_{x\in \spc{X}} \{\,f(x)+\dist{x}{y}{\spc{Y}}\,\}\]
defines an extension function on $\spc{Y}$.
\end{subthm}

\begin{subthm}{}
Assume that $\diam \spc{Y}\le d$ and $f(x)\le d$ for any $x\in  \spc{X}$.
Show that 
\[\bar f_d
\df
\min \{\, \bar f,d\,\}\]
is a $d$-extension function on $\spc{Y}$.
\end{subthm}

\end{thm}

The functions $\bar f$ and $\bar f_d$ in the above exercise are called \index{Katětov extensions}\emph{Katětov extensions} of $f$.

\begin{thm}{Definition}\label{def:finite+1}
A metric space $\spc{U}$ meets the \index{extension property}\emph{extension property}  if for any finite subspace $\spc{F}\subset\spc{U}$ and any extension function $f\:\spc{F}\to\RR$ there is a point $p\in \spc{U}$ such that $\dist{p}{x}{}=f(x)$ for any $x\in \spc{F}$.

If we assume in addition that $\diam \spc{U}\le d$ and instead of extension functions we consider only $d$-extension functions, then we arrive at a definition of {}\emph{$d$-extension property}.

If in addition, $\spc{U}$ is separable and complete, then it is called \index{Urysohn space}\emph{Urysohn space} or {}\emph{$d$-Urysohn space} respectively.
\end{thm}


\begin{thm}{Proposition}\label{prop:univeral-separable}
There is a separable metric space with the ($d$-) extension property (for any $d\ge 0$).
\end{thm}

\parit{Proof.}
Choose $d\ge 0$.
Let us construct a separable metric space with  the $d$-extension property.

Let $\spc{X}$ be a metric space such that $\diam\spc{X}\le d$.
Denote by $\spc{X}^d$ the space of all $d$-extension functions on $\spc{X}$ equipped with the metric defined by the sup-norm.
Note that the map $\spc{X} \to \spc{X}^d$ defined by $x\mapsto\distfun_x$ is a distance-preserving embedding,
so we can (and will) treat $\spc{X}$ as a subspace of $\spc{X}^d$, or, equivalently, $\spc{X}^d$ is an extension of $\spc{X}$.

Let us iterate this construction.
Start with a one-point space $\spc{X}_0$ and consider a sequence of spaces $(\spc{X}_n)$ defined by $\spc{X}_{n+1}\z=\spc{X}_n^d$.
Note that the sequence is nested;
that is, $\spc{X}_0\subset \spc{X}_1\subset\dots$
and the union
\[\spc{X}_\infty=\bigcup_n\spc{X}_n;\]
comes with metric such that
$\dist{x}{y}{\spc{X}_\infty} = \dist{x}{y}{\spc{X}_n}$
if $x,y\in\spc{X}_n$.

Note that if $\spc{X}$ is compact, then so is $\spc{X}^d$.
It follows that each space $\spc{X}_n$ is compact.
In particular, $\spc{X}_\infty$ is a countable union of compact spaces;
therefore $\spc{X}_\infty$ is separable.

Any finite subspace $\spc{F}$ of $\spc{X}_\infty$ lies in some $\spc{X}_n$ for $n<\infty$.
By construction, there is a point $p\in \spc{X}_{n+1}$ that meets the condition in \ref{def:finite+1} for any extension function $f\:\spc{F}\to\RR$.
That is, $\spc{X}_\infty$ has the $d$-extension property.

The construction of a separable metric space with the extension property requires only two changes.
First, the sequence should be defined by $\spc{X}_{n+1}\z=\spc{X}_n^{d_n}$, where $d_n$ is an increasing sequence such that $d_n\to\infty$.
Second, the point $p$ should be taken in $\spc{X}_{n+k}$ for sufficiently large $k$, so that $d_{n+k}>\max\{f(x)\}$
(here one has to apply \ref{ex:extension-of-extension:a}).
\qeds

Given a metric space $\spc{X}$, denote by $\spc{X}^\infty$ the space of all extension functions on $\spc{X}$ equipped with the metric defined by the sup-norm.

\begin{thm}{Exercise}\label{ex:inf-extension}
Construct a proper length space $\spc{X}$ such that $\spc{X}^\infty$ is not separable.
\end{thm}


\begin{thm}{Proposition}\label{prop:completion-univeral}
If a metric space $\spc{V}$ meets the ($d$-) extension property, then so does its completion.
\end{thm}

\parit{Proof.} 
Let us assume $\spc{V}$ meets the extension property.
We will show that its completion $\spc{U}=\bar{\spc{V}}$ meets the extension property as well.
The $d$-extension case can be proved along the same lines.

Note that $\spc{V}$ is a dense subset in a complete space $\spc{U}$.
Observe that $\spc{U}$ has the {}\emph{approximate extension property};
that is, if $\spc{F}\z\subset\spc{U}$ is a finite set, $\eps>0$, and $f\:\spc{F}\to \RR$ is an extension function, then
there exists $p\in \spc{U}$ such that
\[\dist{p}{x}{}\lg f(x)\pm\eps\eqlbl{eq:|p-x|><f(x)}\]
for any $x\in\spc{F}$.
Indeed, the Katětov extension  $\bar f\:\spc{U}\to\RR$ of $f$; see \ref{ex:extension-of-extension}.
Since $\spc{V}$ is dense in $\spc{U}$, we can choose a finite set $\spc{F}'\in \spc{V}$ such that for any $x\in \spc{F}$ there is $x'\in \spc{F}'$ with $\dist{x}{x'}{}<\tfrac\eps2$.
It remains to observe that the point $p$ provided by the extension property for the restriction $\bar f|_{\spc{F}'}$ meets \ref{eq:|p-x|><f(x)}.

It follows that there is a sequence of points $p_n\in \spc{U}$ such that for any $x\in \spc{F}$, 
\[\dist{p_n}{x}{}\lg f(x)\pm\tfrac1{2^n}.\]

Moreover, we can assume that 
\[\dist{p_n}{p_{n+1}}{} < \tfrac1{2^n}\eqlbl{eq:|pn-pn|}\]
for all large $n$.
Indeed, consider the sets $\spc{F}_n=\spc{F}\cup\{p_n\}$ and the functions $f_n\:\spc{F}_n\to\RR$ defined by $f_n(x)=f(x)$ if $x\ne p_n$ and
\[f_n(p_n)=\max\set{\bigl|\dist{p_n}{x}{}- f(x)\bigr|}{x\in \spc{F}}.\]
Observe that $f_n$ is an extension function for large $n$ and
$f_n(p_n)\z<\tfrac1{2^n}$.
Therefore, applying the approximate extension property recursively we get~\ref{eq:|pn-pn|}.

By \ref{eq:|pn-pn|}, the sequence $p_n$ is Cauchy and its limit meets the condition in the definition of extension property (\ref{def:finite+1}).
\qeds

Note that \ref{prop:univeral-separable} and \ref{prop:completion-univeral} imply the following:

\begin{thm}{Theorem}\label{thm:urysohn-exists}
Urysohn space and $d$-Urysohn space exist for any $d>0$.
\end{thm}

Here is a slightly stronger statement:

\begin{thm}{Theorem}\label{thm:urysohn-exists+}
Any separable metric space $\spc{X}$ admits a distance-preserving embedding into an Urysohn space $\spc{U}$ such that any isometry of $\spc{X}$ can be extended to an isometry of $\spc{U}$.
\end{thm}

\parit{Sketch of proof.}
Denote by $\spc{X}^\infty$ the space of all extension functions on $\spc{X}$ equipped with the metric defined by the sup-norm.
Note that $x\mapsto \distfun_x$ defines a distance preserving inclusion $\spc{X} \hookrightarrow\spc{X}^\infty$, and
any isometry $\spc{X}\to \spc{X}$ can be extended to a unique isometry $\spc{X}^\infty\z\to \spc{X}^\infty$.

Given a separable metric space $\spc{X}=\spc{X}_0$ consider a nested sequence of spaces 
\[\spc{X}_0\subset \spc{X}_1\subset\dots\]
such that $\spc{X}_{n+1}=\spc{X}_{n}^\infty$.
It is easy to modify the proofs of \ref{prop:univeral-separable} and \ref{prop:completion-univeral} to show that of the completion of the union $\bigcup_n\spc{X}_{n}$ is an Urysohn space, say $\spc{U}$, that comes with a distance-preserving inclusion $\spc{X}\hookrightarrow \spc{U}$.

From above, for any isometry $f\: \spc{X}\to \spc{X}$ there is a unique sequence of isometries $f_{n}\: \spc{X}_{n}\to \spc{X}_{n}$ such that $f_{n+1}$ is an extension of $f_n$ for any $n$.
Passing to a limit we get an isometry of $\spc{U}$.
\qeds


\section{Universality}

A metric space will be called \index{universal space}\emph{universal} if it includes as a subspace an isometric copy of any separable metric space.
In \ref{ex:frechet}, we proved that $\ell^\infty$ is a universal space. 
The following proposition shows that an Urysohn space is universal as well.
Unlike $\ell^\infty$, Urysohn spaces are separable;
so it might be considered as a \textit{better} universal space.
Theorem \ref{thm:compact-homogeneous} will give another reason why Urysohn spaces are better.

\begin{thm}{Proposition}\label{prop:sep-in-urys}
An Urysohn space is universal.
That is, if $\spc{U}$ is an Urysohn space, then any separable metric space $\spc{S}$ admits a distance-preserving embedding $\spc{S}\hookrightarrow\spc{U}$.

Moreover, for any finite subspace $\spc{F}\subset \spc{S}$,
any distance-preserving embedding $\spc{F}\hookrightarrow \spc{U}$ can be extended to a distance-preserving embedding $\spc{S}\hookrightarrow\spc{U}$.

A $d$-Urysohn space is $d$-universal;
that is, the above statements hold provided that $\diam\spc{S}\le d$.  
\end{thm}

\parit{Proof.}
We will prove the second statement;
the first statement is its partial case for $\spc{F}=\emptyset$.

The required isometry will be denoted by $x\mapsto x'$.

Choose a dense sequence of points $s_1,s_2,\dotsc\in\spc{S}$.
We may assume that $\spc{F}=\{s_1,\dots,s_n\}$, so $s_i'\in \spc{U}$ are defined for $i\le n$.

The sequence $s_i'$ for $i>n$ can be defined recursively using the extension property in $\spc{U}$.
Namely, suppose that $s_1',\dots,s_{i-1}'$ are already defined.
Since $\spc{U}$ meets the extension property, there is a point $s_i'\in \spc{U}$ such that
\[\dist{s_i'}{s_j'}{\spc{U}}=\dist{s_i}{s_j}{\spc{S}}\]
for any $j<i$.

The constructed map $s_i\mapsto s_i'$ is distance-preserving.
Therefore it can be continuously extended to the whole $\spc{S}$.
It remains to observe that the constructed map $\spc{S}\hookrightarrow\spc{U}$ is distance-preserving.
\qeds

\begin{thm}{Exercise}\label{ex:geodesics-urysohn}
Show that any two distinct points in an Urysohn space can be joined by an infinite number of geodesics.
\end{thm}

\begin{thm}{Exercise}\label{ex:compact-extension}
Modify the proofs of \ref{prop:completion-univeral} and \ref{prop:sep-in-urys} to prove the following theorem.
\end{thm}

\begin{thm}{Theorem}\label{thm:compact-extension}
Let $K$ be a compact set in a separable space $\spc{S}$.
Then any distance-preserving map from $K$ to an Urysohn space can be extended to 
a distance-preserving map on whole $\spc{S}$.
\end{thm}

\begin{thm}{Exercise}\label{ex:sc-urysohn}
Show that ($d$-) Urysohn space is simply-connected.
\end{thm}



\section{Uniqueness and homogeneity}

\begin{thm}{Theorem}\label{thm:urysohn-unique}
Suppose $\spc{F}\subset \spc{U}$ and $\spc{F}'\subset \spc{U}'$ be finite isometric subspaces in a pair of ($d$-)Urysohn spaces $\spc{U}$ and $\spc{U}'$.
Then any isometry $\iota\:\spc{F}\leftrightarrow \spc{F}'$ can be extended to an isometry $\spc{U}\leftrightarrow \spc{U}'$.

In particular, ($d$-)Urysohn space is unique up to isometry.
\end{thm}

While \ref{prop:sep-in-urys} implies that there are distance-preserving maps $\spc{U}\to \spc{U}'$ and $\spc{U}'\to \spc{U}$,
it does not solely imply the existence of an isometry $\spc{U}\leftrightarrow \spc{U}'$.
Its construction uses the idea of \ref{prop:sep-in-urys}, but it is applied {}\emph{back-and-forth} to ensure that the obtained distance-preserving map is onto.

\parit{Proof.}
Choose dense sequences $a_1,a_2,\dots{}\in \spc{U}$ and $b'_1,b'_2,\dots{}\in \spc{U}'$.
We can assume that $\spc{F}=\{a_1,\dots,a_n\}$, $\spc{F}'=\{b_1',\dots,b_n'\}$ and $\iota(a_i)=b_i$ for $i\le n$.

The required isometry $\spc{U}\leftrightarrow \spc{U}'$ will be denoted by $u \leftrightarrow u'$.
Set $a'_i=b'_i$ if $i\le n$.

Let us define recursively $a_{n+1}',b_{n+1}, a_{n+2}', b_{n+2},\dots$ --- on the odd step we define the images of $a_{n+1},a_{n+2},\dots$ and on the even steps we define inverse images of $b'_{n+1},b'_{n+2},\dots$
The same argument as in the proof of \ref{prop:sep-in-urys} shows that we can construct two sequences $a_1',a_2',\dots{}\in \spc{U}'$ and $b_1,b_2,\dots\in \spc{U}$ such that
\begin{align*}
\dist{a_i}{a_j}{\spc{U}}&=\dist{a_i'}{a_j'}{\spc{U}'}
\\
\dist{a_i}{b_j}{\spc{U}}&=\dist{a_i'}{b_j'}{\spc{U}'}
\\
\dist{b_i}{b_j}{\spc{U}}&=\dist{b_i'}{b_j'}{\spc{U}'}
\end{align*}
for all $i$ and $j$.

It remains to observe that the constructed distance-preserving bijection defined by $a_i\leftrightarrow a_i'$ and $b_i\leftrightarrow b_i'$ extends
continuously to an isometry $\spc{U}\leftrightarrow \spc{U}'$. 
\qeds

Observe that \ref{thm:urysohn-unique} implies that the Urysohn space (as well as the $d$-Urysohn space) is \index{homogeneous}\emph{finite-set-homogeneous}; that is,
\begin{itemize}
 \item any distance-preserving map from a finite subset to the whole space can be extended to an isometry.
\end{itemize}

Recall that $S(p,r)_{\spc{X}}$ denotes the sphere of radius $r$ centered at $p$ in a metric space $\spc{X}$;
that is, 
$$S(p,r)_{\spc{X}}=\set{x\in \spc{X}}{\dist{p}{x}{\spc{X}}=r}.$$

\begin{thm}{Exercise}\label{ex:sphere-in-urysohn}
Choose $d\in [0,\infty]$.
Denote by $\spc{U}_d$ the $d$-Urysohn space,
so $\spc{U}_\infty$ is the Urysohn space.

\begin{subthm}{ex:sphere-in-urysohn:sphere}
Assume that $L=S(p,r)_{\spc{U}_d}\ne \emptyset$.
Show that $L$ is isometric to $\spc{U}_{\ell}$; find $\ell$ in terms of $r$ and $d$.
\end{subthm}

\begin{subthm}{ex:sphere-in-urysohn:midpoint}
Let $\ell=\dist{p}{q}{\spc{U}_d}$.
Show that the subset $M\subset\spc{U}_d$ of midpoints between $p$ and $q$ is isometric to $\spc{U}_\ell$.
\end{subthm}

\begin{subthm}{ex:sphere-in-urysohn:homogeneous}
Show that $\spc{U}_d$ is \emph{not} countable-set-homogeneous;
that is, there is a distance-preserving map from a countable subset of $\spc{U}_d$ to $\spc{U}_d$ that cannot be extended to an isometry of $\spc{U}_d$.
\end{subthm}

\end{thm}

In fact the Urysohn space is compact-set-homogeneous; more precisely the following theorem holds.

\begin{thm}{Theorem}\label{thm:compact-homogeneous}
Let $K$ be a compact set in a ($d$-)Urysohn space~$\spc{U}$.
Then any distance-preserving map $K\to \spc{U}$ can be extended to an isometry of $\spc{U}$.
\end{thm}

A proof can be obtained by modifying the proofs of \ref{prop:completion-univeral} and \ref{thm:urysohn-unique}
the same way as it is done in \ref{ex:compact-extension}.

\begin{thm}{Exercise}\label{ex:shere}
Let $S$ be a unit sphere in the Urysohn space $\spc{U}$.
Show that for any two distinct points $x,y\in \spc{U}$ there is a point $z\in S$ such that 
$\dist{x}{z}{}\ne \dist{y}{z}{}$.

Conclude that two isometries of $\spc{U}$ coincide if they coincide on $S$.
\end{thm}

\begin{thm}{Exercise}\label{ex:ext(shere)}
Let $B$ be an open unit ball in the Urysohn space $\spc{U}$.
Show that $\spc{U}\setminus B$ is isometric to $\spc{U}$.

Use it to construct an isometry of a unit sphere $S$ in $\spc{U}$ that cannot be extended to an isometry of $\spc{U}$.
\end{thm}

The following exercise answers a question posted by Pavel Urysohn \cite[§$2(6)$]{urysohn}.
It was solved by Miroslav Katětov \cite{katetov},
but long after that, it was again mentioned as an open problem \cite[p. 83]{gromov-2007}.

\begin{thm}{Exercise}\label{ex:katetov}

\begin{subthm}{ex:katetov:inclusion}
Show that there is a distance-preserving inclusion of the Urysohn space $\iota\:\spc{U}\hookrightarrow \spc{U}$ 
such that $\spc{U}'=\iota(\spc{U})$ is nowhere dense in $\spc{U}$ and any isometry of $\spc{U}'$ 
can be extended to an isometry of the whole~$\spc{U}$.
\end{subthm}

\begin{subthm}{ex:katetov:sol}
Consider a nested sequence $\spc{U}_0\subset \spc{U}_1\subset\dots$ of Urysohn spaces 
with each inclusion $\spc{U}_n\hookrightarrow \spc{U}_{n+1}$ as in \ref{SHORT.ex:katetov:inclusion}.
Show that the union $\bigcup_n\spc{U}_n$ is a noncomplete finite-set-homogeneous metric space that meets the extension property.
\end{subthm}

\end{thm}

{\sloppy

\begin{thm}{Exercise}\label{ex:homogeneous}
Which of the following metric spaces are 
one-point-homogeneous, finite-set-homogeneous, compact-set-homogeneous, countable-set-homogeneous?

\begin{subthm}{ex:homogeneous:euclidean}
Euclidean plane,
\end{subthm}

\begin{subthm}{ex:homogeneous:hilbert}
 Hilbert space $\ell^2$,
\end{subthm}

\begin{subthm}{ex:homogeneous:ell-infty}
 $\ell^\infty$,
\end{subthm}

\begin{subthm}{ex:homogeneous:ell-1}
 $\ell^1$.
\end{subthm}
\end{thm}

}

\begin{thm}{Exercise}\label{ex:homogeneous-tree}
Show that any separable one-point-homogeneous metric tree is isometric to the real line $\RR$ or the one-point space.
\end{thm}


\section{Remarks}

The statement in \ref{ex:frechet} was proved by Maurice René Fréchet in the paper where he first defined metric spaces \cite{frechet};
its extension \ref{lem:kuratowski} was given by Kazimierz Kuratowski~\cite{kuratowski}.
The question about the existence of a separable universal space was posted by Maurice René Fréchet and answered by
Pavel Urysohn~\cite{urysohn}.

The idea of Urysohn's construction was reused in graph theory; it produces the so-called \index{Rado graph}\emph{Rado graph},
also known as {}\emph{Erd\H{o}s–R\'enyi graph} or \emph{random graph}; a good survey on the subject is given by Peter Cameron~\cite{cameron}.
In fact the Urysohn space is the random metric space in \textit{certain sense} \cite{vershik}.

\textit{The Urysohn space is homeomorphic to the Hilbert space};
the latter was proved by Vladimir Uspenskij \cite{uspenskij} using the so-called Toruńczyk’s criterion.

The finite-set-homogeneous spaces include Euclidean spaces, hyperbolic spaces, and spheres all with standard length metrics and arbitrary finite dimension.
In fact these are the only examples of locally compact three-point-homogeneous length spaces.
The latter was proved by Herbert Busemann \cite{busemann-1942}; it also follows from the more general result of Jacques Tits \cite{tits}.
The same conclusion holds for complete all-set-homogeneous geodesic spaces with local uniqueness of geodesics;
it was proved by Garrett Birkhoff \cite{birkhoff}.
The answer might be the same for complete separable all-set-homogeneous length spaces.
Without the separability condition, we also get the so-called \emph{universal metric trees} with finite valence; 
a transparent construction of these spaces is given by Anna Dyubina and Iosif Polterovich \cite{dyubina-polterovich}.  

{\sloppy

\begin{thm}{Exercise}\label{ex:RP-not}
Show that the real projective plane $\RP^2$ with the standard metric is two-point-homogeneous, but not three-point-homogeneous.
\end{thm}

}

\begin{thm}{Exercise}\label{ex:hom-cube}
Let $Q$ be the set of vertices on the $n$-dimensional cube;
assume $n$ is large.
Show that $Q$ is three-point-homogeneous, but not four-point-homogeneous.
\end{thm}

