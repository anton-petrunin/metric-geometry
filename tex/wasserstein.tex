\chapter{Metric plus measure}

\section{Borel sets}

Let us remind few definitions assuming knowleage of basic measure theory;
comprehensive treatments can be found in \cite{billingsley} and \cite{bogachev}.

Let $\spc{X}$ be a metric space.
\index{Borel set}\emph{Borel set} is any subset of $\spc{X}$ that can be formed from open sets using the countable union, countable intersection, and complement.
In other words, Borel sets form the minimal sigma-algebra that included open sets.

A measure on metric space will be always assumed to be \index{Borel measure}\emph{Borel};
that is, it is defined on the sigma-algebra of Borel sets.
A Borel measure can be uniquely determined by its values on all open (or closed) sets.

A measure $\mu$ on $\spc{X}$ is called \index{probability measure}\emph{probability measure} if $\mu\spc{X}=1$.

Recall that \index{delta-measure}\emph{delta-measure} is a probability measure with support at one point.
Delta-measure with support in $\{x\}$ will be denoted by~\index{$\delta_{x}$}$\delta_{x}$; so
\[\text{if}\quad x\in A,\quad\text{then}\quad \delta_x(A)=1,\quad\text{otherwise}\quad\delta_x(A)=0.\]

Let $\mu_n$ be a sequence of Borel measures on $\spc{X}$.
A measure $\mu_\infty$ is a \index{weak limit}\emph{weak limit} of $\mu_n$ if 
\[\int_{\spc{X}}f\cdot(\mu_n-\mu_\infty)\to0\gamma
\quad\text{as}\quad
n\to\infty
\]
for any continuous function $f\:\spc{X}\to \RR$.

Suppose $\mu$ is a measure on a metric space $\spc{X}$ and $f\:\spc{X}\to \spc{Y}$ is a measurable map;
that is, for any Borel set $B\subset \spc{Y}$, its inverse image $f^{-1}B$ is a Borel set in $\spc{Y}$.

Consider the unit interval with its Lebesgue mesure.
If $\spc{X}$ is a complete separable metric space with probability measure $\mu$, then there is a measurable map $[0,1]\to \spc{X}$

\section{Metric on measures}

Imagine that we need to transport dirt from one pile of a given shape to make another pile of a needed shape.
Suppose that cost of transporting a unit of dirt equals to the traveled distance.%
\footnote{This is the simplest cost function one can imagine.
One may consider other cost functions; for example, if the cost proportional to the square of the distance, then the problem has more applications.}
We are free to choose a destination point for dirt from a given place.
How to minimize the total cost?

To formalize this question,
suppose that the piles of dirt are described by Borel probability measures $\mu$ and $\nu$ on a metric space~$\spc{X}$.

To describe where each piece of dirt goes, we will use the so called \index{plan}\emph{plan} for $\mu$ and $\nu$.
It is a probability measure $\pi$ on the product $\spc{X}\times\spc{X}$ such that 
for all measurable sets $A \subset \spc{X}$, we have 
\[\mu A= \pi [A \times \spc{X}],
\quad\text{and}\quad
\nu A=\pi [\spc{X}\times A].
\eqlbl{eq:marginals}\]
Equivalently it can be described as a measure that satisfies the following identity
\begin{align*}
\int_{(x,y)\in \spc{X}\times\spc{X}}f(x)\cdot g(y) \cdot \pi
&=
\int_{x\in \spc{X}}f(x)\cdot \mu
\oldcdot \int_{y\in \spc{X}}g(y)\cdot \nu,
\end{align*}
for any continuous functions $f,g\:\spc{X}\to \RR$.

Given a measure $\pi$ on $\spc{X}\times\spc{X}$, the measures $\mu$ and $\nu$ defined by \ref{eq:marginals} are called first and second \index{marginal}\emph{marginals} of $\pi$;
so the statement \textit{$\pi$ is a plan for $\mu$ and $\nu$} is equivalent to \textit{$\mu$ and $\nu$ are the first and second marginals of $\pi$ respectively}.

\begin{thm}{Claim}\label{clm:plan-exists}
There is a plan $\pi$ for any two given Borel probability measures $\mu$ and $\nu$.
\end{thm}

The plan constructed in the proof distributes equally each piece of dirt in the new pile.
As we will see this plan is usually far from optimum.

\parit{Proof.}
Consider the measure $\pi$ that is uniquely defined  defined by the identity
\[\pi(A\times B)=\mu A\cdot \mu B\]
for any Borel subsets $A,B\subset\spc{X}$.
Observe that $\pi$ is a plan for $\mu$ and~$\nu$.
\qeds

Denote by $\Pi(\mu,\nu)$ the set of all plans for $\mu$ and $\nu$;
by \ref{clm:plan-exists}, $\Pi(\mu,\nu)\z\ne\emptyset$.
It is straightforwrd to check that the following formula defines a metric on the space of probability measures on $\spc{X}$.
\[\dist{\mu}{\nu}{\Wass_1\spc{X}}
\df
\inf_{\pi\in\Pi(\mu,\nu)}
\left\{\,\int_{(x,y)\in\spc{X}\times\spc{X}}\dist{x}{y}{\spc{X}}\cdot\pi\,\right\}.\]
This metric is called \index{Wasserstein distance}\emph{Wasserstein distance of order 1} between $\mu$ and $\nu$.

In genereral, the Wasserstein distance $\dist{\mu}{\nu}{}$ might take infinite value, but all measures with compact support lie on finite distance from each other in the obtained $\infty$-metric space.
The metric component of these measures is called \index{Wasserstein space}\emph{Wasserstein space} of order 1 over $\spc{X}$; 
it is denoted by $\Wass_1\spc{X}$.
In other words, $\Wass_1\spc{X}$ is the space of all Borel probability measures $\mu$ such that 
$\int\distfun_p\cdot\mu<\infty$ for some (and therefore any) point $p\in \spc{X}$.

\begin{thm}{Exercise}\label{ex:wasserstein-infty}
Construct two Borel probability measures $\mu$ and $\nu$ on $\RR$ with Wasserstein distance $\dist{\mu}{\nu}{}=\infty$.
\end{thm}


\begin{thm}{Exercise}\label{ex:wasserstein-compact}
Show that $\Wass_1\spc{X}$ is a compact if and only if so is~$\spc{X}$.
\end{thm}

\begin{thm}{Exercise}\label{ex:wasserstein-length}
Show that the Wasserstein space $\Wass_1\spc{X}$ is a geodesic space for any metric space $\spc{X}$.
\end{thm}

\section{Optimal plan}

A plan $\pi$ for given measures $\mu$ and $\nu$ is called \index{optimal plan}\emph{optimal} if 
\[\dist{\mu}{\nu}{\Wass_1\spc{X}}
=\int_{(x,y)\in\spc{X}\times\spc{X}}\dist{x}{y}{\spc{X}}\cdot\pi.\]

\begin{thm}{Theorem} %Vilani:Theorem 1.4
Let $\mu$ and $\nu$ be probability Borel measures on a compact metric space $\spc{X}$.
Then there is an optimal plan $\pi$ for $\mu$ and~$\nu$.
\end{thm}

\parit{Proof.}
By the definition of Wasserstein distance, we can choose a sequence of plans $\pi_n$ for $\mu$ and $\nu$ such that 
\[\int_{(x,y)\in\spc{X}\times\spc{X}}\dist{x}{y}{\spc{X}}\cdot\pi_n\to \dist{\mu}{\nu}{\Wass_1\spc{X}}\]
as $n\to \infty$.

Observe that $\pi_n$ has a weak partial limit, say $\pi$.
Moreover, $\pi$ is an optimal plan for $\mu$ and $\nu$.
\qeds

\begin{thm}{Theorem}
Any optimal plan $\pi$ is \index{cyclic monotonicity}\emph{cyclically monotonic}.
That is, suppose $\pi$ is an optimal plan for probability measures $\mu$ and $\nu$ on a metric space $\spc{X}$.
Then any sequence of pairs $(x_1,y_1),\dots,(x_n,y_n)\in\supp\pi\subset\spc{X}\times\spc{X}$ we have
\[\sum_i\dist{x_i}{y_i}{}
\le
\sum_i\dist{x_{i+1}}{y_i}{},\]
here the index $i$ in the sum is taken modulo $n$; in particular $x_{n+1}\z=x_1$.
\end{thm}

\parit{Proof.}
Assume that the cyclic monotonicity does not hold;
that is,
\[R=\sum_i\dist{x_i}{y_i}{}
-
\sum_i\dist{x_{i+1}}{y_i}{}>0,\]
for some $(x_0,y_0),\dots,(x_n,y_n)\in\supp\pi$.
We need to show that $\pi$ is not optimal;
in other words we need to construct another plan $\pi'$ for $\mu$ and $\nu$ such that 
\[\int_{(x,y)\in\spc{X}\times\spc{X}}\dist{x}{y}{\spc{X}}\cdot(\pi'-\pi)<0.\eqlbl{pi'<pi}\]

Assume $\spc{X}$ is finite.
In this case we can choose $\eps>0$ such that 
$\pi\{(x_i,y_i)\}>\eps$ for each $i$.
Let
\[\pi'=\pi-\eps\cdot\sum_i(\sigma_i-\sigma_i')\eqlbl{eq:pi'}\]
where $\sigma_i=\delta_{(x_i,y_i)}$ and $\sigma_i'=\delta_{(x_{i+1},y_i)}$.
It remains to observe that $\pi'$ is a plan for $\mu$ and $\nu$ that satisfies \ref{pi'<pi}.

The general case is similar, we only need to redefine $\eps$, $\sigma_i$, and~$\sigma_i'$.
Note that given $r>0$, we can choose a probability measures $\sigma_i$ with support in $\oBall((x_i,y_i),r)_{\spc{X}\times\spc{X}}$ such that $\eps\cdot \sigma_i<\pi$ for some fixed $\eps>0$ and every $i$.
Further denote by $\zeta_i$ and $\eta_i$ the first and second marginals of $\sigma_i$.
Observe that $\supp\zeta_i\subset\oBall(x_i,r)$ and $\supp\eta_i\subset\oBall(y_i,r)$ for all $i$.
Let $\sigma_i'$ be a plan for $\zeta_{i+1}$ and $\eta_i$.
Evidently 
\begin{align*}
\int_{(x,y)\in\spc{X}\times\spc{X}}\dist{x}{y}{}\cdot \sigma_i
&\lessgtr
\dist{x_i}{y_i}{}\pm 2\cdot r,
\\
\int_{(x,y)\in\spc{X}\times\spc{X}}\dist{x}{y}{}\cdot \sigma_i'
&\lessgtr
\dist{x_{i+1}}{y_i}{}\pm 2\cdot r.
\end{align*}
Taking $r<\tfrac R{10\cdot n}$, we get  \ref{pi'<pi}. 
\qeds




\section{Capitalistic approach}

Imagine that measures $\mu$ and $\nu$ describe the production and consumer of beer in the space.
A transportaition company transports beer from $\mu$ to $\nu$ and want to maximize its profit by adjusting price $f(x)$ of beer the point $x$; they buy beer at price $f(x)$ per unit, move it to an other point $y$ and sale it with (presumably higher) price $f(y)$.
However, the function $f$ is 1-Lipschitz condition;
otherwise the profit goes to second-hand dealers, or maybe it is a government regulation.
In other words, we need to maximize the following expression
\[\int_{\spc{X}} f\cdot(\mu-\nu)\]
for all $1$-Lipschitz functions $f$.
The maximal profit defines a metric

\begin{thm}{Theorem}
Let $\mu$ and $\nu$ be probability Borel measures on a compact metric space $\spc{X}$.
Then
\[\dist{\mu}{\nu}{\Wass_1\spc{X}}=\sup\int_{\spc{X}} f\cdot(\mu-\nu),\]
where the least upper bound is taken for all $1$-Lipschitz functions $f\:\spc{X}\z\to\RR$.
\end{thm}

The definition of Wassershtein metric described in the previous section reminds communist's planed economy.
The right-hand side in the above equation reminds capitalistic system.
Indeed, think that measures $\mu$ and $\nu$ describe the production and consumer of beer in the space.
A transportaition company trnasports beer from $\mu$ to $\nu$ and want to maximize its profit by adjusting price $f(x)$ of beer the point $x$.
However, the function $f$ is 1-Lipschitz condition --- this is a government regulation.




\parit{Proof.}
By the definition of Wasserstein metric, we can choose a sequence $\pi_n$ of plans  

Let us choose an optimal plan $\pi$ for $\mu$ and $\nu$; it exists by ???.
We need to find a 1-Lipschitz function $f\:\spc{X}\to\RR$ such that 
\[
\int_{\spc{X}} f\cdot(\mu-\nu)=\int_{(x,y)\in\spc{X}\times\spc{X}}\dist{x}{y}{\spc{X}}\cdot \pi.
\eqlbl{eq:f(mu-nu)}
\]

Choose $x_0\in \supp\mu$.
Note that adding a constant to $f$ does not change the left hand side in \ref{eq:f(mu-nu)}.
Therefore we can assume assume that $f(x_0)=0$ and set
\[f(x)=\sup\{\,|x_0-y_0|+\dots+|x_n-y_n|-(|x_1-y_0|+\dots+|x_n-y_{n-1}|)-|x-y_n|\,\}\]
where the least upper bound is taken for all sequences $(x_0,y_0),\z\dots,(x_n,y_n)\z\in\supp\pi$.

\qeds

\section{Metric-measure space}

A metric measure space is a metric $\spc{X}$ space with a choice of Borel probability measure $\vol$ on it.
In a metric-measure we ignore sets with vanishing volume; in other words, passing from $\spc{X}$ to the support of $\vol$ does not change the metric-measure space.

Alternatively we may start with unit interval $[0,1]$ equipped with Lebesgue measure and equip it with measurable semimetric $[0,1]\times [0,1]\to \RR$.





\section{Space of measures}


It can be equipped with the \index{Wasserstein metric}\emph{Wasserstein metric}
\[\dist{\mu}{\nu}{}\df\sup\left\{\,\int_{\spc{X}} f\cdot(\mu-\nu)\,\right\},\]
where the least upper bound is taken for all $1$-Lipschitz functions $f\:\spc{X}\to\RR$.

The Wasserstein distance $\dist{\mu}{\nu}{}$ might take infinite value, but all measures with compact support lie on finite distance from each other in the obtained $\infty$-metric space.
The metric component of these measures is called \index{Wasserstein space}\emph{Wasserstein space} of order 1 over $\spc{X}$; 
it is denoted by $\Wass_1\spc{X}$.



\section{Misc}

Suppose $\pi_n$ is a sequence of plans for $\mu$ and $\nu$.
Assume that $\pi_n$ weakly converges to a probability measure $\pi$ on $\spc{X}\times\spc{X}$.

is a weak limit of a sequence of plans $\pi_n$, then $\pi$ is a plan for $\mu$ and $\nu$ if for each $n$ $\pi_n$ is a plane for $\mu$ and $\nu$ 

Suppose that $f\:\spc{X}\to \RR$ is a 1-Lipschitz function,
so $f(x)-f(y)\le\dist{x}{y}{\spc{X}}$ for any $x,y\in \spc{X}$.
It follows that 
\begin{align*}
\int_{\spc{X}} f\cdot(\mu-\nu)&=\int_{x\in\spc{X}}f(x)\cdot\mu-\int_{y\in\spc{X}}f(y)\cdot\nu=
\\
&=\int_{(x,y)\in\spc{X}\times\spc{X}}[f(x)-f(y)]\cdot \pi\le
\\
&\le\int_{(x,y)\in\spc{X}\times\spc{X}}\dist{x}{y}{\spc{X}}\cdot \pi,
\end{align*}
where $\pi$ is a plan for $\mu$ and $\nu$.
By the definition of Wasserstein metric, we get  
\[\dist{\mu}{\nu}{\Wass_1\spc{X}}\le \int_{(x,y)\in\spc{X}\times\spc{X}}\dist{x}{y}{\spc{X}}\cdot\pi\eqlbl{wass=<int.plan}\]
for any plan $\pi$.

Next we want to show that equality holds in \ref{wass=<int.plan} for some plan $\pi$; such plans will be called \index{optimal plan}\emph{optimal}.


\parit{Proof.}
Choose a point $x_0\in \supp\mu$.
Given  $p\in \spc{X}$,
let
\[f(p)=\inf\left\{\sum_{i=0}^n\dist{x_i}{y_i}{}-\sum_{i=0}^n\dist{x_{i+1}}{y_i}{}-\dist{y_n}{p}{}\right\},
\eqlbl{eq:f(p)}\]
where the least upper bound is taken for all sequences of pairs 
\[(x_0,y_0),\z\dots,(x_n,y_n)\in \supp\pi.\eqlbl{eq:sequence}\]

Fix a sequence as in \ref{eq:sequence} and  denote by $F_\sigma(p)$ the expression under infimum in \ref{eq:f(p)}.

Let us show that 
\[F_\sigma(x_0)\ge 0.\]
Indeed, suppose $F_\sigma(x_0)<-\eps<0$.
Since $(x_i,y_i)\in \supp\pi$, we have $x_i\in\supp\mu$ and $y_i\in\supp\nu$ for any $i$.
Therefore we can choose sets $X_i\subset \oBall(x_i,\tfrac{\eps}{10\cdot n})$ and $Y_i\subset \oBall(y_i,\tfrac{\eps}{10\cdot n})$ such that 
$\mu(X_0)=\nu(Y_0)=\dots=\mu(X_n)=\nu(Y_n)$



Let us denote by $F(p)$ the expression under infimum in \ref{eq:f(p)}.
By the triangle inequality, 
\[F(q)\le F(p)+\dist{p}{q}{}.\]
Passing to the least upper bound in this inequality, we get
\[f(q)\le f(p)+\dist{p}{q}{}\]
for any $p,q\in\spc{X}$.
Hence $f$ is a 1-Lipschitz function.

Further, let us show that
\[(x,y)\in\supp\pi
\quad\Longrightarrow\quad
f(y)-f(x)=\dist{x}{y}{}\]





Suppose that cyclic monotonicity fails;
that is, there is a sequence of pairs $(x_1,y_1),\dots,(x_n,y_n)\in\spc{X}\times\spc{X}$ such that
\[\dist{x_1}{y_1}{}+\dots+\dist{x_n}{y_n}{}
>
\dist{x_1}{y_2}{}+\dots+\dist{x_{n-1}}{y_n}{}+\dist{x_{n}}{y_1}{}.\]
In this case, it would be more optimal to transport measure from a neighborhood of $x_i$ to a neighborhood of $y_{i+1}$ (
here and further we assume that the indexes are taken modulo $n$, so $n+1=1$).
The latter contradicts optimality of $\pi$.

The following argument makes it precise.
Choose small $\eps>0$.
For each $n$,
choose disjoint sets $X_i$ and $Y_i$ in $\eps$-neighborhood of $x_i$ and $y_i$
such that for some $\delta>0$ we have 
\[\pi [X_i\times Y_i]=\delta\]
for each $i$.

Let us modify the plan $\pi$ in the union $X_1\times Y_1 \cup\dots\cup X_n\times Y_n$ and such that 
$\pi'(X_i\times Y_{i+1})=\delta$ for each $i$;


Observe that
\[\int_{(x,y)\in\spc{X}\times\spc{X}}\dist{x}{y}{\spc{X}}\cdot(\pi'-\pi)>\]
\qeds
