\chapter{Metric plus measure}

This lecture devoted to interactions between metric measures.

\section{Space of measures}

A measure on a metric space is called \index{Borel measure}\emph{Borel} if it is defined on all open sets (and thus on all Borel sets).

Let us consider the space of Borel probability measures on a metric space $\spc{X}$.
It can be equipped with the \index{Wasserstein metric}\emph{Wasserstein metric}
\[\dist{\mu}{\nu}{}\df\sup\left\{\,\int_{\spc{X}} f\cdot(\mu-\nu)\,\right\},\]
where the least upper bound is taken for all $1$-Lipschitz functions $f\:\spc{X}\to\RR$.

The Wasserstein distance $\dist{\mu}{\nu}{}$ might take infinite value, but all measures with compact support belong lie on finite distance from each other in the obtained $\infty$-metric space.
the metric component of these measures is called \index{Wasserstein space}\emph{Wasserstein space} of order 1 over $\spc{X}$; 
it is denoted by $\Wass_1\spc{X}$.

\begin{thm}{Exercise}\label{ex:wasserstein-infty}
Construct two Borel probability measures $\mu$ and $\nu$ on $\RR$ with Wasserstein distance $\dist{\mu}{\nu}{}=\infty$.
\end{thm}


\begin{thm}{Exercise}\label{ex:wasserstein-compact}
Show that $\Wass_1\spc{X}$ is a compact if and only if so is~$\spc{X}$.
\end{thm}

\begin{thm}{Exercise}\label{ex:wasserstein-length}
Show that the Wasserstein space $\Wass_1\spc{X}$ is a geodesic space for any metric space $\spc{X}$.
\end{thm}

\section{Transportation plan}

Imagine that we need to transport dirt from one pile of a given shape to make another pile of a needed shape.
Suppose that cost of transporting a unit of dirt equals to the traveled distance.%
\footnote{This is the simplest cost function one can imagine.
One may consider other cost functions; for example, if the cost proportional to the square of the distance, then the problem has more applications.}
We are free to chose a destination point for dirt from a given place.
How to minimize the total cost?

Let us formalize this question.
Suppose that the piles of dirt are described by probability measures $\mu$ and $\nu$ on a metric space~$\spc{X}$.
To describe where each piece of dirt goes, we will use the so called \index{plan}\emph{plan} for $\mu$ and $\nu$.
It is a probability measure $\pi$ on the square $\spc{X}\times\spc{X}$ such that the following condition holds:
\begin{itemize}
\item For all measurable sets $A \subset \spc{X}$, we have 
\[\pi [A \times \spc{X}] = \mu A,\quad\text{and}\quad \pi [  \spc{X}\times A] = \nu A.\]
\end{itemize}

Suppose that $f\:\spc{X}\to \RR$ is a 1-Lipschitz function,
so $f(x)-f(y)\le\dist{x}{y}{\spc{X}}$ for any $x,y\in \spc{X}$.
It follows that 
\begin{align*}
\int_{\spc{X}} f\cdot(\mu-\nu)&=\int_{x\in\spc{X}}f(x)\cdot\mu-\int_{y\in\spc{X}}f(y)\cdot\nu=
\\
&=\int_{(x,y)\in\spc{X}\times\spc{X}}[f(x)-f(y)]\cdot \pi\le
\\
&\le\int_{(x,y)\in\spc{X}\times\spc{X}}\dist{x}{y}{\spc{X}}\cdot \pi,
\end{align*}
where $\pi$ is a plan for $\mu$ and $\nu$.
By the definition of Wasserstein metric, we get  
\[\dist{\mu}{\nu}{\Wass_1\spc{X}}\le \int_{(x,y)\in\spc{X}\times\spc{X}}\dist{x}{y}{\spc{X}}\cdot\pi\eqlbl{wass=<int.plan}\]
for any plan $\pi$.

Next we want to show that equality holds in \ref{wass=<int.plan} for some plan $\pi$; such plans will be called \index{optimal plan}\emph{optimal}.

First note that a plan $\pi$ for given measures $\mu$ and $\nu$ always exists.
For example, one may take the plan $\pi$ that is uniquely defined  defined by the identity
\[\pi(A\times B)=\mu A\cdot \mu B\]
for any Borel subsets $A,B\subset\spc{X}$.
This plan says that each piece of dirt distributed equally in the new pile; as we will see this plan is usually far from optimum.


Denote by $\Pi(\mu,\nu)$ the set of all plans for $\mu$ and $\nu$.
From above $\Pi(\mu,\nu)\ne\emptyset$.
It is straightforwrd to check that the following formula defines a metric on the space of probability measures on $\spc{X}$.
\[\dist{\mu}{\nu}{\Wass_1\spc{X}}
\df
\inf_{\pi\in\Pi(\mu,\nu)}
\left\{\,\int_{(x,y)\in\spc{X}\times\spc{X}}\dist{x}{y}{\spc{X}}\cdot\pi\,\right\}.\]
This metric is called \index{Wasserstein distance}\emph{Wasserstein distance of order 1} and the obtained metric space will be denoted by $\Wass_1\spc{X}$.

\section{Optimal plan}

A plan $\pi$ for given measures $\mu$ and $\nu$ is called \index{optimal plan}\emph{optimal} if 
\[\dist{\mu}{\nu}{\Wass_1\spc{X}}
=\int_{(x,y)\in\spc{X}\times\spc{X}}\dist{x}{y}{\spc{X}}\cdot\pi.\]

\begin{thm}{Theorem}
Let $\mu$ and $\nu$ be probability Borel measures on a compact metric space $\spc{X}$.
Then there is an optimal plan $\pi$ for $\mu$ and~$\nu$.

Moreover any optimal plan $\pi$ is cycliccaly monotonic;
that is, for any sequence of pairs $(x_0,y_0),\dots,(x_n,y_n)\in\supp\pi\subset\spc{X}\times\spc{X}$ we have
\[\dist{x_1}{y_1}{}+\dots+\dist{x_n}{y_n}{}
\le
\dist{x_1}{y_2}{}+\dots+\dist{x_{n-1}}{y_n}{}+\dist{x_{n}}{y_1}{}.\]
\end{thm}

\parit{Proof.}
By the definition of Wasserstein distance, we can choose a sequence of plans $\pi_n$ for $\mu$ and $\nu$ such that 
\[\int_{(x,y)\in\spc{X}\times\spc{X}}\dist{x}{y}{\spc{X}}\cdot\pi_n\to \dist{\mu}{\nu}{\Wass_1\spc{X}}\]
as $n\to \infty$.
Denote by $\pi$ a weak partial limit of $\pi_n$;
it exists by ???.
Note that $\pi$ is an optimal plan.

Suppose that cyclic monotonicity fails;
that is, there is a sequence of pairs $(x_1,y_1),\dots,(x_n,y_n)\in\spc{X}\times\spc{X}$ such that
\[\dist{x_1}{y_1}{}+\dots+\dist{x_n}{y_n}{}
>
\dist{x_1}{y_2}{}+\dots+\dist{x_{n-1}}{y_n}{}+\dist{x_{n}}{y_1}{}.\]
In this case, it would be more optimal to transport measure from a neighborhood of $x_i$ to a neighborhood of $y_{i+1}$ (
here and further we assume that the indexes are taken modulo $n$, so $n+1=1$).
The latter contradicts optimality of $\pi$.

The following argument makes it precise.
Choose small $\eps>0$.
For each $n$,
choose disjoint sets $X_i$ and $Y_i$ in $\eps$-neighborhood of $x_i$ and $y_i$
such that for some $\delta>0$ we have 
\[\pi [X_i\times Y_i]=\delta\]
for each $i$.

Let us modify the plan $\pi$ in the union $X_1\times Y_1 \cup\dots\cup X_n\times Y_n$ and such that 
$\pi'(X_i\times Y_{i+1})=\delta$ for each $i$;


Observe that
\[\int_{(x,y)\in\spc{X}\times\spc{X}}\dist{x}{y}{\spc{X}}\cdot(\pi'-\pi)>\]
\qeds


\section{Kantorovich duality}


\begin{thm}{Theorem}
Let $\mu$ and $\nu$ be probability Borel measures on a compact metric space $\spc{X}$.
Then
\[\dist{\mu}{\nu}{\Wass_1\spc{X}}=\sup\int_{\spc{X}} f\cdot(\mu-\nu),\]
where the least upper bound is taken for all $1$-Lipschitz functions $f\:\spc{X}\z\to\RR$.
\end{thm}


The following theorem gives an alternative way to define it.

\begin{thm}{Theorem}
Let $\mu$ and $\nu$ be two probability measures of a metric space $\spc{X}$.
Then 
\[\cost(\mu,\nu)
\df
\sup\set{\int_{x\in\spc{X}}\phi(x)\cdot\mu+\int_{y\in\spc{X}}\psi(y)\cdot\nu}{\phi(x)+\psi(y)\le \cost(x,y)},\]
where the functions $\phi$ and $\psi$ are assumed to be continuous.\footnote{In fact one }
\end{thm}




