\chapter{Space of sets}

\section{Hausdorff distance}

Let $\spc{X}$ be a metric space.
Given a subset $A\subset \spc{X}$,
consider the distance function to $A$
$$\distfun_A: \spc{X} \to [0,\infty)$$
defined as 
$$\distfun_A(x)
\df
\inf_{a\in A}\{\,\dist ax{\spc{X}}\,\}.$$

\begin{thm}{Definition}\label{def:hausdorff-convergence}
Let $A$ and $B$ be two compact subsets of a metric space $\spc{X}$.
Then the \index{Hausdorff distance}\emph{Hausdorff distance} between $A$ and $B$ is defined as 
$$|A-B|_{\Haus\spc{X}}
\df
\sup_{x\in \spc{X}}\{\,|\distfun_A(x)-\distfun_B(x)|\,\}.
$$

\end{thm}
 
Suppose $A$ and $B$ be two compact subsets of a metric space $\spc{X}$.
It is straightforward to check that $|A-B|_{\Haus\spc{X}}\le R$ if and only if 
$\distfun_A(b)\le R$ for any $b\in B$
and 
$\distfun_B(a)\le R$ for any $a\in A$.
In other words, $|A-B|_{\Haus\spc{X}}< R$ if and only if 
$B$ lies in a $R$-neighborhood of $A$, 
and 
$A$ lies in a $R$-neighborhood of $B$.
The latter property can be used as an alternative definition of the Hausdorff distance $|A-B|_{\Haus\spc{X}}$.

Note that the set of all nonempty compact subsets of a metric space $\spc{X}$ equipped with the Hausdorff metric forms a metric space.
This new metric space will be denoted as $\Haus\spc{X}$.


\begin{thm}{Exercise}\label{ex:diam}
Let $\spc{X}$ be a metric space.
Given a subset $A\subset \spc{X}$ define its diameter as 
$$\diam A\df\sup_{a,b\in A} |a-b|.$$

Show that 
$$\diam\:\Haus\spc{X}\to \RR$$ 
is a $2$-Lipschitz function;
that is,
\[|\diam A-\diam B|\le 2\cdot\dist{A}{B}{\Haus\spc{X}}\]
for any two compact nonempty sets $A,B\subset\spc{X}$.
\end{thm}


\begin{thm}{Exercise}\label{ex:Hausdorff-bry}
Let $X$ and $Y$ be two compact subsets in $\RR^2$.
Assume $|X-Y|_{\Haus\RR^2}<\eps$.

\begin{subthm}{ex:Hausdorff-bry:conv}
Show that $|\Conv X-\Conv Y|_{\Haus\RR^2}<\eps$, where $\Conv X$ denoted a convex hull of $X$.
\end{subthm}
\begin{subthm}{ex:Hausdorff-bry:bry}
Is it true that
$|\partial X-\partial Y|_{\Haus\RR^2}<\eps$,
where $\partial X$ denotes the boundary of $X$.

Does the converse holds? That is, assume $X$ and $Y$ be two compact subsets in $\RR^2$
and $|\partial X-\partial Y|_{\Haus\RR^2}<\eps$; 
is it true that $|X-Y|_{\Haus\RR^2}\z<\eps$?
\end{subthm}

\end{thm}

\section{Hausdorff convergence}

\begin{thm}{Blaschke selection theorem}\label{thm:compact+Hausdorff}
A metric space $\spc{X}$ is compact if and only if
so is $\Haus\spc{X}$.
\end{thm}

Hausdorff metric can be used to define convergence.
Namely, suppose $K_1,K_2,\dots$ and $K_\infty$ is a sequence of compact sets in a metric space $\spc{X}$.
If $|K_\infty-K_n|_{\Haus\spc{X}}\to0$ as $n\to\infty$, then we say that 
the sequence $(K_n)$ {}\emph{converges} to $K_\infty$ \index{convergence in the sense of Hausdorff}\emph{in the sense of Hausdorff};
or we can say that $K_\infty$ is \emph{Hausdorff limit} of the sequence $(K_n)$.

Note that the theorem implies that from any sequence of compact sets in $\spc{X}$ one can select a subsequence that converges in the sense of Hausdorff; 
by that reason it is called a selection theorem. 

\parit{Proof; ``only if'' part.}
Note that the map $\iota\:\spc{X}\to \Haus\spc{X}$, defined as $\iota\:x\mapsto\{x\}$
(that is, point $x$ mapped to the one-point subset $\{x\}$ of $\spc{X}$)
is distance-preserving.
Therefore $\spc{X}$ is isometric to the set $\iota(\spc{X})$ in $\Haus\spc{X})$.

Note that for a nonempty subset $A\subset \spc{X}$, we have $\diam A=0$ if and only if $A$ is a one-point set.
Therefore, from Exercise~\ref{ex:diam}, it follows 
that $\iota(\spc{X})$ is closed in $\Haus\spc{X}$.

Hence $\iota(\spc{X})$  is compact, as it is a closed subset of a compact space. 
Since $\spc{X}$ is isometric to $\iota(\spc{X})$,
``only if'' part follows.
\qeds

To prove ``if'' part we will need the following two lemmas.

\begin{thm}{Lemma}\label{lem:decreasing-converges}
Let $K_1\supset K_2\supset\dots$ be a sequence of nonempty compact sets in a metric space $\spc{X}$
then $K_\infty=\bigcap_n K_n$ is the Hausdorff limit of $K_n$;
that is, $|K_\infty-K_n|_{\Haus\spc{X}}\to0$ as $n\to\infty$.
\end{thm}

\parit{Proof.}
Note that $K_\infty$ is compact;
by finite intersection property, $K_\infty$ is nonempty.

If the assertion were false, 
then there is $\eps>0$ such that for each $n$ 
one can choose $x_n\in K_n$
such that $\distfun_{K_\infty}(x_n)\ge\eps$.
Note that $x_n\in K_1$ for each $n$.
Since $K_1$ is compact, 
there is 
a \index{partial limit}\emph{partial limit}%
\footnote{Partial limit is a limit of a subsequence.}
 $x_\infty$ of $x_n$.
Clearly $\distfun_{K_\infty}(x_\infty)\ge \eps$.

On the other hand, since $K_n$ is closed and $x_m\in K_n$ for $m\ge n$,
we get $x_\infty\in K_n$ for each $n$.
It follows that $x_\infty\in K_\infty$ and therefore $\distfun_{K_\infty}(x_\infty)=0$,
a contradiction.\qeds


\begin{thm}{Lemma}\label{lem:complete+Hausdorff}
If $\spc{X}$ is a compact metric space, then $\Haus\spc{X}$
is complete.
\end{thm}

\parit{Proof.}
Let $(Q_n)$ be a Cauchy sequence in $\Haus\spc{X}$.
Passing to a subsequence of $Q_n$ we may assume that 
$$|Q_n-Q_{n+1}|_{\Haus\spc{X}}\le \tfrac1{10^n}\eqlbl{eq:eps=1/10}$$
for each $n$.

Set 
\begin{align*}
K_n&= \set{x\in \spc{X}}{\distfun_{Q_n}(x)\le \tfrac1{10^n}}
\end{align*}
Since $\spc{X}$ is compact so is each $K_n$.

Clearly, $|Q_n-K_n|_{\Haus\spc{X}}\le \tfrac1{10^n}$ and from \ref{eq:eps=1/10}, we get
$K_n\supset K_{n+1}$ 
for each $n$.
Set 
$$K_\infty=\bigcap_{n=1}^\infty K_n.$$
Applying Lemma \ref{lem:decreasing-converges},
we get that $|K_n-K_\infty|_{\Haus\spc{X}}\to 0$ as $n\to\infty$.
Since $|Q_n-K_n|_{\Haus\spc{X}}\le \tfrac1{10^n}$, we get $|Q_n-K_\infty|_{\Haus\spc{X}}\to 0$ as $n\to\infty$ --- hence the lemma.
\qeds

\begin{thm}{Exercise}
Let $\spc{X}$ be a complete metric space and $K_n$ be a sequence of compact sets 
which converges in the sence of Hausdorff.
Show that closure of the union $\bigcup_{n=1}^\infty K_n$ is compact.

Use this to show that in Lemma~\ref{lem:complete+Hausdorff} compactness of $\spc{X}$ can be exchanged to completeness.
\end{thm}

\parit{Proof of ``if'' part in \ref{thm:compact+Hausdorff}.}
According to Lemma~\ref{lem:complete+Hausdorff},
$\Haus\spc{X}$ is complete.
It remains to show that $\Haus\spc{X}$ is totally bounded (\ref{totally-bounded});
that is, given $\eps>0$ there is a finite $\eps$-net in $\Haus\spc{X}$.

Choose a finite $\eps$-net $A$ in $\spc{X}$.
Denote by $\mathcal{A}$ the set of all subsets of $A$.
Note that  $\mathcal{A}$ is finite set in $\Haus\spc{X}$.
For each compact set $K\subset \spc{X}$, consider the subset $K'$ of all points $a\in A$
such that $\distfun_K(a)\le \eps$.
Then $K' \in \mathcal{A}$ and $|K-K'|_{\Haus\spc{X}}\le\eps$.
In other words $\mathcal{A}$ is a finite $\eps$-net in $\Haus\spc{X}$.
\qeds

\begin{thm}{Exercise}\label{ex:Haus-length}
Let $\spc{X}$ be a complete metric space.
Show that $\spc{X}$ is a length space if and only if so is $\Haus\spc{X}$.
\end{thm}

\section{An application}

The following statement is calle \index{isoperimetrical inequality}\emph{isoperimetrical inequality in the plane}.

\begin{thm}{Theorem}
Among the plane figures bounded by closed curves of length at most $\ell$ the round disc has maximal area.
\end{thm}

In this section we will sketch a proof of the isoperimetrical inequality that uses the Hausdorff convergence.
It is based on the following exercise.

\begin{thm}{Exercise}\label{ex:Huas-perimeter-area}
Let $\spc{C}$ be a subspace of $\Haus\RR^2$ formed by all compact convex subsets in $\RR^2$.
Show that perimeter\footnote{If the set degenerates to a line segment of length $\ell$, then its perimeter is defined as $2\cdot \ell$.} and area are continuous on~$\spc{C}$.
That is, if a sequence of convex compact plane sets $X_n$ converges to $X_\infty$ in the sense of Hausdorff, then 
\[\perim X_n\to \perim X_\infty\quad\text{and}\quad\area X_n\to\area X_\infty\]
as $n\to\infty$.
\end{thm}

\parit{Sketch.}
It is sufficient to consider only convex figures of given perimeter; if a figure is not convex pass to its convex hull and observe that it has larger area and smaller perimeter.


Note that the exercise guarantees existence of a figure $D$ with perimeter $\ell$ and maximal area.

It remains to show that $D$ is a round disc.
This turns out to be an easy problem;
further we indicate one way to do it.
An example way to prove that $D$ is a round disc is given by the so called {}\emph{Steiner's 4-joint method}; see \cite{blaschke}.


Let us cut $D$ along a chord $[ab]$ into two lens, $L_1$ and $L_2$.
Denote by $L_1'$ the reflection of $L_1$ across the perpendicular bisector of $[ab]$.
Note that $D$ and $D'=L_1'\cup L_2$ have the same perimeter and area.
That is, $D'$ has perimeter $\ell$ and maximal possible area;
in particular $D'$ is convex.
The following exercise will finish the proof.
\qeds

{

\begin{wrapfigure}{o}{50 mm}
\vskip-0mm
\centering
\includegraphics{mppics/pic-405}
\end{wrapfigure}

\begin{thm}{Exercise}
Suppose $D$ is a convex figure such that for any chord $[ab]$ of $D$ the above construction produces a convex figure $D'$.
Show that $D$ is a round disc.
\end{thm}


}

\section{Remarks}\label{sec:H-variation}

It seems that Hausdorff convergence was first introduced by Felix Hausdorff~\cite{hausdorff},
and a couple of years later an equivalent definition was given by Wilhelm Blaschke~\cite{blaschke}.

The following refinement of the definition was introduced by  Zden\v{e}k Frol\'{\i}k in \cite{frolik},
later it was rediscovered by Robert Wijsman in~\cite{wijsman}.  
This refinement takes an intermediate place between the original Hausdorff convergence and {}\emph{closed convergence}, also introduced by Hausdorff in \cite{hausdorff};
so we still call it Hausdorff convergence.

\begin{thm}{Definition}\label{def:gen-Haus-conv}
Let $(A_n)$ be a sequence of closed sets in a metric space $\spc{X}$.
We say that $(A_n)$ converges to a closed set $A_\infty$ in the sense of Hausdorff if $\distfun_{A_n}(x)\to \distfun_{A_\infty}(x)$ for any $x\in\spc{X}$.
\end{thm}

For example, suppose $\spc{X}$ is the Euclidean plane and $A_n$ is the circle with radius $n$ and center at $(n,0)$.
If we use the standard definition (\ref{def:hausdorff-convergence}), then the sequence $(A_n)$ diverges, but it converges to the $y$-axis in the sense of Definition~\ref{def:gen-Haus-conv}.

The following exercise is analogous to the Blaschke selection theorem (\ref{thm:compact+Hausdorff}).

\begin{thm}{Exercise}
Let $\spc{X}$ be a proper metric space
and $(A_n)_{n=1}^\infty$ be a sequence of closed sets in~$\spc{X}$.
Assume that for some (and therefore any) point  $x\in\spc{X}$, 
the sequence $a_n=\distfun_{A_n}(x)$ is bounded.
Show that the sequence  $(A_n)_{n=1}^\infty$ has a convergent subsequence in the sense of Definition~\ref{def:gen-Haus-conv}.
\end{thm}

\chapter{Space of spaces}

\section{Gromov--Hausdorff metric}

The goal of this section is to cook up a metric space out of metric spaces.
More precisely, we want to define the so called  Gromov--Hausdorff metric on the set of {}\emph{isometry classes} of compact metric spaces.
(Being isometric is an equivalence relation, 
and an isometry class is an equivalence class with respect to this equivalence relation.)

The obtained metric space will be denoted as $\GH$.
Given two metric spaces $\spc{X}$ and $\spc{Y}$,
denote by $[\spc{X}]$ and $[\spc{Y}]$ their isometry classes;
that is, $\spc{X}'\in [\spc{X}]$ if and only if $\spc{X}'\iso \spc{X}$.
Pedantically, the Gromov--Hausdorff distance from $[\spc{X}]$ 
to $[\spc{Y}]$ should be denoted as $|[\spc{X}]-[\spc{Y}]|_{\GH}$;
but we will write it as $|\spc{X}-\spc{Y}|_{\GH}$ and say (not quite correctly) 
``$|\spc{X}-\spc{Y}|_{\GH}$ is the Gromov--Hausdorff distance from  $\spc{X}$ 
to  $\spc{Y}$''.
In other words, from now on the term {}\emph{metric space} might stands for {}\emph{isometry class of this metric space}.

The metric on $\GH$ is maximal metric such that {}\emph{the distance between subspaces in a metric space is not greater than the Hausdorff distance between them}.
Here is a formal definition:

\begin{thm}{Definition}\label{def:GH}
Let $\spc{X}$ and $\spc{Y}$ be compact metric spaces. 
The Gromov--Hausdorff distance $|\spc{X}-\spc{Y}|_{\GH}$
between them is defined by the following
relation.
 
Given  $r > 0$, we have that $|\spc{X}-\spc{Y}|_{\GH} < r$ if and only if there exist a metric
space $\spc{Z}$ and subspaces $\spc{X}'$ and $\spc{Y}'$ in $\spc{Z}$ that are isometric to $\spc{X}$ and $\spc{Y}$
respectively and such that $|\spc{X}'-\spc{Y}'|_{\Haus\spc{Z}} < r$. 
(Here $|\spc{X}'-\spc{Y}'|_{\Haus\spc{Z}}$ denotes the Hausdorff distance between sets $\spc{X}'$ and $\spc{Y}'$ in $\spc{Z}$.)
\end{thm}

\begin{thm}{Theorem}\label{thm:GH-is-a-metric}
The set of isometry classes of compact metric spaces equipped with Gromov--Hausdorff metric forms a metric space (which is denoted by $\GH$).

In other words, for arbitrary  compact metric spaces $\spc{X}$, $\spc{Y}$ and $\spc{Z}$ the following conditions hold:

\begin{subthm}{GH-1} $|\spc{X}-\spc{Y}|_{\GH}\ge 0$;
\end{subthm}

\begin{subthm}{GH-2} $|\spc{X}-\spc{Y}|_{\GH}=0$ if and only if $\spc{X}$ is isometric to $\spc{Y}$;
\end{subthm}

\begin{subthm}{GH-3} $|\spc{X}-\spc{Y}|_{\GH}=|\spc{Y}-\spc{X}|_{\GH}$;
\end{subthm}

\begin{subthm}{GH-4} $|\spc{X}-\spc{Y}|_{\GH}+|\spc{Y}-\spc{Z}|_{\GH}\ge |\spc{X}-\spc{Z}|_{\GH}$.
\end{subthm}
\end{thm}


Note that \ref{SHORT.GH-1}, ``if''-part of \ref{SHORT.GH-2}, and \ref{SHORT.GH-3} follow directly from Definition \ref{def:GH}.
Part \ref{GH-4} will be proved in Section~\ref{sec:gluing+triangle}.
The ``only-if''-part of \ref{SHORT.GH-2} will be proved in Section???.

\begin{thm}{Exercise}\label{ex:d_GH-and-diam}
Let $\spc{X}$ and $\spc{Y}$ be two compact metric spaces.

\begin{subthm}{}
Prove that 
$$|\spc{X}-\spc{P}|_{\GH}=\tfrac12\cdot \diam \spc{X},$$
where $\spc{P}$ denotes the one-point metric space.
\end{subthm}

Recall that $a\cdot\spc{X}$ denotes $\spc{X}$ scaled by factor $a>0$;
that is, $a\cdot\spc{X}$ is a metric space with underlying set of $\spc{X}$ and the metric defined by
\[\dist{x}{y}{a\cdot\spc{X}}\df a\cdot\dist{x}{y}{\spc{X}}.\]

\begin{subthm}{}
Show that 
\[|a\cdot\spc{X}-b\cdot \spc{X}|_{\GH}=\tfrac12\cdot|a-b|\cdot\diam\spc{X}.\]
\end{subthm}


\end{thm}


\section{Metrics on disjoined union}

Definition~\ref{def:GH} deals with a huge class of metric spaces,
namely, all metric spaces $\spc{Z}$ that contain subspaces isometric to $\spc{X}$ and $\spc{Y}$.
The following proposition shows that it is possible to reduce this class to metrics on the disjoint unions of $\spc{X}$ and $\spc{Y}$. 

\begin{thm}{Proposition}\label{prop:GH=X+Y}
The Gromov--Hausdorff distance between two compact metric spaces $\spc{X}$
and $\spc{Y}$ is the infimum of $r > 0$ such that there exists a metric
$|{*}-{*}|_{\spc{U}}$ on the disjoint union $\spc{U}=\spc{X}\sqcup \spc{Y}$ 
such that the restrictions of $|{*}-{*}|_{\spc{U}}$ to $\spc{X}$ and $\spc{Y}$
coincide with $|{*}-{*}|_\spc{X}$ and $|{*}-{*}|_{\spc{Y}}$ 
and $|\spc{X}-\spc{Y}|_{\Haus\spc{U}} < r$. 
\end{thm}

\parit{Proof.}
Let $r>0$, $\spc{X}$, $\spc{Y}$, $\spc{X}'$, $\spc{Y}'$, and $\spc{W}$ be as in Definition~\ref{def:GH}.

If  $\spc{X}'$ and $\spc{Y}'$ are disjoint, then we can identify $\spc{X}$ with  $\spc{X}'$ and $\spc{Y}$ with $\spc{Y}'$ and identify $\spc{U}$ with the subspace formed by $\spc{X}'\cup\spc{Y}'$.

To do the general case,
choose isometries $f\: \spc{X} \to \spc{X}'$ and
$g\: \spc{Y} \z\to \spc{Y}'$, then define the metric by
\begin{align*}
|x-y|_{\spc{U}} &= |f (x)\z- g(y)|_{\spc{W}}+\eps,
\\
|x-x'|_{\spc{U}} &= |x- x'|_{\spc{X}},
\\
|y-y'|_{\spc{U}} &= |y- y'|_{\spc{Y}},
\end{align*}
for a small fixed $\eps>0$ and $x,x' \in \spc{X}$ and $y,y' \in \spc{Y}$.
We need to add $\eps$ to ensure that $|x-y|_{\spc{U}} > 0$ for any $x\in \spc{X}$ and $y\in \spc{Y}$;
so $|x-y|_{\spc{U}}$ is indeed a metric.

Since $\eps$ is small,
this yields a metric on $\spc{U}=\spc{X}\sqcup \spc{Y}$ for which
$|\spc{X}- \spc{Y}|_{\Haus\spc{U}} \z< r$.
\qeds

\begin{thm}{Advanced exercise}\label{ex:GH-inj}
Let $\spc{X}$ and $\spc{Y}$ be compact metric spaces;
denote by $\hat{\spc{X}}$ and $\hat{\spc{Y}}$ their injective envelops (see the definition on page \pageref{page:InjX}).
Show that 
\[|\hat{\spc{X}}-\hat{\spc{Y}}|_{\GH}\le 2\cdot|\spc{X}- \spc{Y}|_{\GH}.\] 

\end{thm}

\section{Gluing and triangle inequality}\label{sec:gluing+triangle}

Suppose 
$\spc{U}$ and $\spc{V}$ are metric spaces 
with isometric closed sets $A\subset\spc{U}$ and $A'\subset\spc{V}$;
let $\iota\:A\to A'$ be an isometry.
Consider the space $\spc{W}$ of all equivalence classes in $\spc{U}\sqcup\spc{V}$ with the equivalence relation given by $a\sim\iota(a)$ for any $a\in A$.

It is straightforward to see that the following defines a metric on $\spc{W}$:
\begin{align*}
\dist{x}{y}{\spc{W}}&\df\dist{x}{y}{\spc{U}}
\\
&\quad\text{if}\quad x,y\in \spc{U},
\\
\dist{x}{y}{\spc{W}}&\df\dist{x}{y}{\spc{V}}
\\
&\quad\text{if}\quad x,y\in \spc{V},\quad\text{and}
\\
\dist{x}{y}{\spc{W}}&\df\min\set{\dist{x}{a}{\spc{U}}+\dist{y}{\iota(a)}{\spc{V}}}{a\in A}
\\
&\quad\text{if}\quad x\in \spc{U}\quad\text{and}\quad y\in \spc{V}.
\end{align*}
Further, abusing notation, we denote by $x$ and $y$ the points in $\spc{U}\sqcup\spc{V}$ and their equivalence classes in $\spc{U}\sqcup\spc{V}/{{\sim}}$.

The  space $\spc{W}$ is called the \index{gluing}\emph{gluing} of $\spc{U}$ and  $\spc{V}$ along~$\iota$; or, we can write
$\spc{W}=\spc{U}\sqcup_\iota\spc{V}$.
If one applies this construction to two copies of one space $\spc{U}$ with a set $A\subset \spc{U}$ and the identity map $\iota\:A\to A$, then the obtained space is called the \index{double}\emph{double} of $\spc{U}$ along~$A$.

We can and will identify $\spc{U}$ and $\spc{V}$ with their images in $\spc{W}$;
this way the subsets $A$ and $A'$ will be identified and denoted further by~$A$.
Note that $A=\spc{U}\cap \spc{V}\subset \spc{W}$.

Let us use the described gluing construction we can prove the triangle inequality for Gromov--Hausdorff distance.

\parit{Proof of \ref{GH-4}.}
Choose arbitrary $a,b \in \mathbb{R}$ such that
$$a>|\spc{X}-\spc{Y}|_{\GH}
\quad\text{and}\quad
b>|\spc{Y}-\spc{Z}|_{\GH}.$$
Choose two metrics on $\spc{U}=\spc{X}\sqcup \spc{Y}$ and $\spc{V}=\spc{Y}\sqcup \spc{Z}$ so that
$|\spc{X}\z-\spc{Y}|_{\Haus\spc{U}}<a$ and $|\spc{Y}-\spc{Z}|_{\Haus\spc{V}}<b$ 
and the inclusions $\spc{X}\hookrightarrow \spc{U}$, $\spc{Y}\hookrightarrow \spc{U}$, $\spc{Y}\hookrightarrow \spc{V}$ and $\spc{Z}\hookrightarrow \spc{V}$ are distance-preserving.

Let $\spc{W}$ be the gluing of $\spc{U}$ and $\spc{V}$ along $\spc{X}$.
Observe that $a$-neighborhood of $\spc{X}$ in $\spc{W}$ contains $\spc{Y}$ and $b$-neighborhood of $\spc{Y}$ contains $\spc{Z}$.
Therefore $(a+b)$-neighborhood of $\spc{X}$ in $\spc{W}$ contains $\spc{Z}$.
The same way we can show that $(a+b)$-neighborhood of $\spc{Z}$ in $\spc{W}$ contains $\spc{X}$.
It follows that 
$$a+b\ge|\spc{X}-\spc{Y}|_{\GH}.$$
Whence the statement follows.
\qeds

\section{Fixed ambient space}

The following proposition says that the space $\spc{Z}$ in Definition~\ref{def:GH} can be exchanged to a fixed space, namely $\ell^\infty$ --- the space of bounded infinite sequences with the metric defined by $\sup$-norm.

\begin{thm}{Proposition}\label{prop:GH-with-fixed-Z}
Let $\spc{X}$ and $\spc{Y}$ be comact metric spaces.
Then
$$|\spc{X}-\spc{Y}|_{\GH} = \inf \{|\spc{X}'-\spc{Y}'|_{\Haus\ell^\infty}\}$$ 
where the infimum is taken over all pairs of sets $\spc{X}'$ and $\spc{Y}'$ in $\ell^\infty$
which isometric to  $\spc{X}$ and $\spc{Y}$ respectively.  
\end{thm}




\parit{Proof of \ref{prop:GH-with-fixed-Z}.}
By the definition, we have that 
\[|\spc{X}-\spc{Y}|_{\GH} \leq \inf \{|\spc{X}'-\spc{Y}'|_{\Haus\ell^\infty}\}.\]

Let $\spc{W}$ be an arbitrary metric space with the underlying set $\spc{X}\sqcup \spc{Y}$.
Note $\spc{W}$ is compact since it is union of two compact subsets $\spc{X},\spc{Y}\subset \spc{W}$.
In particular, $\spc{W}$ is separable.

By Lemma \ref{lem:frechet}, there is a distance-preserving embedding $\iota\:\spc{W}\z\to \ell^\infty$.
It remains to apply Proposition~\ref{prop:GH=X+Y}.
\qeds

\begin{thm}{Exercise}
Let $\spc{U}$ be the Urysohn space.
Given two compact set $A$ and $B$ in $\spc{U}$ define 
\[\|A-B\|=\inf\{|A-\iota(B)|_{\Haus\spc{U}}\},\]
where the infimum is taken for all isometrics $\iota$ of $\spc{U}$.
Show that $\|{*}\z-{*}\|$ defines a pseudometric on nonempty compact subsets of $\spc{U}$ and its corresponding metric space is isometric to $\GH$.
\end{thm}



\section{Almost isometries}\label{sec:alm-isom}

\begin{thm}{Definition} Let $\spc{X}$ and $\spc{Y}$ be metric spaces and $\eps > 0$. 
A  map\footnote{possibly noncontinuous} $f\: \spc{X} \z\to \spc{Y}$ is called an $\eps$-isometry 
if 
$$|f(x)-f(x')|_{\spc{Y}}\lege |x-x'|_{\spc{X}}\pm\eps$$
for any $x,x'\in \spc{X}$ 
and if $f(\spc{X})$ is an $\eps$-net in $\spc{Y}$.
\end{thm}

\begin{thm}{Exercise}\label{ex:alm-isom}

\begin{subthm}{ex:alm-isom:compositon}
Let $f\:\spc{X}\to \spc{Y}$ and $g\:\spc{Y}\to \spc{Z}$ be two $\eps$-isometries.
Show that $g\circ f\: \spc{X}\to \spc{Z}$ is a $(3\cdot\eps)$-isometry.
\end{subthm}

\begin{subthm}{ex:alm-isom:inverse}
Assume $f\: \spc{X} \z\to \spc{Y}$ is an $\eps$-isometry.
Show that there is a $(3\cdot\eps)$-isometry 
$g\: \spc{Y}\to \spc{X}$.
\end{subthm}

\begin{subthm}{ex:GH=>eps-isom}
 Assume $|\spc{X}-\spc{Y}|_{\GH}<\eps$, show that there is a $(2\cdot\eps)$-isometry 
$f\: \spc{X}\to \spc{Y}$.
\end{subthm}
\end{thm}

\begin{thm}{Proposition}\label{prop:alm-isom=>GH}
Let $\spc{X}$ and $\spc{Y}$ be metric spaces 
and let $f\: \spc{X}\to \spc{Y}$ be an $\eps$-isometry.
Then 
\[|\spc{X}-\spc{Y}|_{\GH}\le 2\cdot \eps.\]
\end{thm}

\parit{Proof.} Consider the set $\spc{W}=\spc{X}\sqcup \spc{Y}$.
Note that the following defines a metric on $\spc{W}$:
\begin{itemize}
\item  For any $x,x'\in \spc{X}$
$$|x-x'|_{\spc{W}}=|x-x'|_{\spc{X}};$$
\item For any $y,y'\in \spc{Y}$,
$$|y-y'|_{\spc{W}}=|y-y'|_{\spc{Y}}$$
\item For any $x\in \spc{X}$ and $y\in \spc{Y}$,
$$|x-y|_{\spc{W}}=\eps+\inf_{x'\in \spc{X}}\{|x-x'|_{\spc{X}}+|f(x')-y|_{\spc{Y}}\}.$$
\end{itemize}

Since $f(\spc{X})$ is an $\eps$-net in $\spc{Y}$,
for any $y\in \spc{Y}$ there is $x\in \spc{X}$ such that $|f(x)-y|_{\spc{Y}}\z\le\eps$;
therefore $|x-y|_{\spc{W}}\le 2\cdot\eps$.
On the other hand for any $x\in \spc{X}$, we have $|x-y|_{\spc{W}}\le\eps$
for $y=f(x)\in \spc{Y}$.

It follows that $|\spc{X}-\spc{Y}|_{\Haus\spc{W}}\le 2\cdot\eps$.
\qedsf

\begin{thm}{Proposition}\label{prop:eps-isometry=isometry}
Let $\spc{X}$ and $\spc{Y}$ be compact metric spaces.
Suppose that for any $\eps>0$ there is an $\eps$-isometry $\spc{X}\to\spc{Y}$.
Then there is an isometry $\spc{X}\to\spc{Y}$.
\end{thm}

\parit{Proof.}
For each positive integer $n$, choose a $\tfrac1n$-isometry $f_n\:\spc{X}\to\spc{Y}$.

Since $\spc{X}$ is compact, 
we can choose a countable dense set
$S$ in $\spc{X}$.
Use a diagonal procedure if necessary, to pass to a subsequence of $(f_n)$
such that for every $x \in S$ the sequence $(f_n(x))$ 
converges in $\spc{Y}$. 
Consider the pointwise limit map  $f_\infty \: S \to \spc{Y}$ defined by
 $$f_\infty(x) = \lim_{n\to\infty} f_n (x)$$ for every $x \in S$. 
Since $$|f_n (x)- f_n (x')|_{\spc{Y}}\lege |x- x'|_\spc{X} \pm\eps_n,$$ 
we have 
$$|f_\infty(x)-f_\infty (x')|_{\spc{Y}} 
= \lim_{n\to\infty} |f_n(x)-f_n (x')|_{\spc{Y}} 
= |x -x'|_\spc{X}$$ for all
$x, x' \in S$; 
that is, $f_\infty\:S\to \spc{Y}$ is a distance-preserving map. 
Therefore $f_\infty$ can be extended to a distance-preserving map from all of $\spc{X}$ to $\spc{Y}$.
The later can be done by setting 
$$f_\infty(x)=\lim_{n\to\infty} f_\infty(x_n)$$ 
for some sequence of points $(x_n)$ in $S$
which converges to $x$ in $\spc{X}$.
Indeed, if $x_n\to x$, then $(x_n)$ is Cauchy.
Since $f_\infty$ is distance-preserving, $y_n=f_\infty(x_n)$ is also a Cauchy sequence in $\spc{Y}$;
therefore it converges.
It remains to observe that this construction does not depend on the choice of the sequence $(x_n)$.

This way we obtain a distance-preserving map $f_\infty\:\spc{X}\to \spc{Y}$. 
It remains to show that $f_\infty$ is surjective; that is, $f_\infty(\spc{X})=\spc{Y}$.

By \ref{ex:alm-isom:inverse}, there is an $\eps$-siometry $\spc{Y}\z\to \spc{X}$ for any $\eps>0$.
The same construction as above shows that there is a distance-preserving map $g_\infty\:\spc{Y}\z\to \spc{X}$.
If $f_\infty$ is not surjective, then neither is $f_\infty\circ g_\infty\:\spc{Y}\to \spc{Y}$.
So $f_\infty \circ g_\infty$ is a distance-preserving map from a compact space to itself which is not an isometry.
The later contradicts \ref{ex:non-contracting-map}. 
\qeds

\parit{Proof of the ``only if''-part in \ref{GH-2}.}
Follows from \ref{ex:GH=>eps-isom} and \ref{prop:eps-isometry=isometry}.
\qeds


The Gromov--Hausdorff metric defines Gromov--Hausdorff convegence
and this is the only thing it is good for.
In other words in all applications, we use only topology on $\GH$
and we do not care about particular value of Gromov--Hausdorff distance between spaces.

In order to determine that a given sequence of metric spaces $(\spc{X}_n)$ converges in the Gromov--Hausdorff sense to $\spc{X}_\infty$, it is sufficient to estimate distances $|\spc{X}_n-\spc{X}_\infty|_{\GH}$ and  check if $|\spc{X}_n-\spc{X}_\infty|_{\GH}\to 0$.
This problem turns to be simpler than finding Gromov--Hausdorff distance between a particular pair of spaces.
The following proposition gives one way to do this.

\begin{thm}{Proposition}\label{prop:GH-e-isom}
A sequence of compact metric spaces $(\spc{X}_n)$ converges to  $\spc{X}_\infty$ in the sense of Gromov--Hausdorff if and only if there is a sequence $\eps_n\to0+$
and an $\eps_n$-isometry $f_n\:\spc{X}_n\to \spc{X}_\infty$ for each $n$.
\end{thm}

\parit{Proof.} Follows from \ref{prop:alm-isom=>GH} and \ref{ex:GH=>eps-isom}.
\qeds

\begin{thm}{Exercise}
Show that $\GH$ is a length space.
\end{thm}

Given two metric spaces $\spc{X}$ and $\spc{Y}$, we will write $\spc{X}\le \spc{Y}$ if there is a noncontracting map $f\:\spc{X}\to \spc{Y}$;
that is, if 
$$ |x-x'|_{\spc{X}}\le|f(x)-f(x')|_{\spc{Y}}$$
for any $x,x'\in \spc{X}$.

Further, given $\eps>0$, we will write $\spc{X}\le \spc{Y}+\eps$
if there is a map $f\:\spc{X}\to \spc{Y}$ such that 
$$|x-x'|_{\spc{X}}\le|f(x)-f(x')|_{\spc{Y}}+\eps$$
for any $x,x'\in \spc{X}$.

\begin{thm}{Exercise}\label{ex:GH-po}
\begin{subthm}{ex:GH-po:a}
Show that 
$$\dist{\spc{X}}{\spc{Y}}{\GH'}=\inf\set{\eps>0}{\spc{X}\le \spc{Y}+\eps
\quad\text{and}\quad
\spc{Y}\le \spc{X}+\eps}$$
defines a metric on the space of (isometry classes) of compact metric spaces.
\end{subthm}

\begin{subthm}{ex:GH-po:b}
Moreover $\dist{*}{*}{\GH'}$ is equivalent to the Gromov--Haudorff metric;
that is,
$$|\spc{X}_n-\spc{X}_\infty|_{\GH}\to 0 
\quad\iff\quad 
\dist{\spc{X}_n}{\spc{X}_\infty}{\GH'}\to 0$$ 
as $n\to\infty$.
\end{subthm}
\end{thm}


\section{Uniformly totally bonded families}

Let $\spc{Q}$ be a set of (isometry classes) of compact metric spaces.
Suppose that there is a sequence $\eps_n\to 0$ such that for any positive integer $n$ each space $\spc{X}$ in $\spc{Q}$ admits an $\eps_n$-net with at most $n$ points.
Then we say that $\spc{Q}$ is \index{uniformly totally bonded}\emph{uniformly totally bonded}.

Observe that in this case $\diam\spc{X}<\eps_1$ for any  $\spc{X}$ in $\spc{Q}$; that is diameters of spaces in $\spc{Q}$ are bounded above.

Fix a real constant $C$.
A measure $\mu$ on a metric space $\spc{X}$ is called \index{doubling space}\emph{$C$-doubling} if
\[\mu[\oBall(p,2\cdot r)< C\cdot\mu[\oBall(p,r)]\]
for any point $p\in \spc{X}$ and any positive real $r$.
A measure is called \index{doubling measure}\emph{doubling} if it is {}\emph{$C$-doubling} for a some real constant $C$.

\begin{thm}{Exercise}\label{pr:doubling}
Let $\spc{Q}(C,D)$ be the set of all the compact metric spaces with diameter at most $D$ that admit a $C$-doubling measure.
Show that $\spc{Q}(C,D)$ is totally bounded.
\end{thm}

Recall that we write $\spc{X}\le\spc{Y}$ if there is a distance-nondecreasing map $\spc{X}\to\spc{Y}$.

\begin{thm}{Exercise}\label{pr:under}

\begin{subthm}{pr:under:if}
Let $\spc{Y}$ be a compact metric space.
Show that the set of all spaces $\spc{X}$ such that $\spc{X}\le\spc{Y}$
is uniformly totally bounded.
\end{subthm}

\begin{subthm}{pr:under:only-if}
Show that for any uniformly totally bounded set $\spc{Q}\subset\GH$ there is a compact space $\spc{Y}$
such that $\spc{X}\le\spc{Y}$ for any $\spc{X}$ in $\spc{Q}$.
\end{subthm}

\end{thm}

\section{Gromov's selection theorem}

The following theorem is analogous to Blaschke selection theorems (\ref{thm:compact+Hausdorff}).

\begin{thm}{Gromov selection theorem}\label{thm:gromov-compactness}
Let $\spc{Q}$ be a closed and subset of $\GH$.
Then $\spc{Q}$ is compact if and only if it is totally bounded.
\end{thm}

\begin{thm}{Lemma}
The space $\GH$ is complete.
\end{thm}

\parit{Proof.}
Let $(\spc{X}_n)$ be a Cauchy sequence in $\GH$.
Passing to a subsequence if necessary, 
we can assume that $|\spc{X}_n-\spc{X}_{n+1}|_{\GH}<\tfrac1{2^n}$ for each~$n$.
In particular, for each $n$ one can equip $\spc{V}_n=\spc{X}_n \sqcup \spc{X}_{n+1}$ with a metric such that
inclusions $\spc{X}_n\hookrightarrow \spc{V}_n$ and $\spc{X}_{n+1}\hookrightarrow \spc{W}_n$ are distance-preserving,
and 
\[|\spc{X}_n-\spc{X}_{n+1}|_{\Haus\spc{V}_n}<\tfrac1{2^n}\]
for each $n$.

Let us glue $\spc{V}_1$ to $\spc{V}_2$ along $\spc{X}_2$;
to the obtained space glue $\spc{V}_3$ along $\spc{X}_3$, and so on.
The obtained metric space $\spc{W}$
has underlying set formed by the disjoint union of all $\spc{X}_n$ such that each inclusion $\spc{X}_n\hookrightarrow\spc{W}$ is distance preserving and
\[|\spc{X}_n-\spc{X}_{n+1}|_{\Haus\spc{W}}<\tfrac1{2^n}\]
for each $n$.
In particular 
\[|\spc{X}_m-\spc{X}_n|_{\Haus\spc{W}}<\tfrac1{2^{n-1}}\eqlbl{eq:|x_m-X_n|}\] 
if $m>n$.

Denote by $\bar{\spc{W}}$ the completion of $\spc{W}$.
Note that \ref{eq:|x_m-X_n|} implies that the union of $\spc{X}_1\cup \spc{X}_2\cup\dots\cup \spc{X}_n$ forms a $\tfrac1{2^{n-1}}$-net in $\bar{\spc{W}}$.
Since each $\spc{X}_i$ is compact, we get that $\bar{\spc{W}}$ admits a compact $\eps$-net for any $\eps>0$.
Whence $\bar{\spc{W}}$ is compact.

Applying Blaschke selection theorem (\ref{thm:compact+Hausdorff}),
we can pass to a subsequence of $(\spc{X}_n)$ that converges in $\Haus\bar{\spc{W}}$ and therefore in $\GH$.
\qeds

\parit{Proof of \ref{thm:gromov-compactness}; ``only if'' part.}
If there is no sequence $\eps_n\to0$ as described in the problem, then for a fixed $\delta>0$
there is a sequence of spaces $\spc{X}_n\in\spc{Q}$ such that $$\pack_\delta \spc{X}_n\to\infty
\quad\text{as}\quad
n\to\infty.$$
Since $\spc{Q}$ is compact, 
this sequence has a partial limit say $\spc{X}_\infty\in\spc{Q}$.
Observe that $\pack_{\delta} \spc{X}_\infty=\infty$.
Therefore $\spc{X}_\infty$ is not compact --- a contradiction.

\parit{``If'' part.}
Without loss of generality, we may assume that there is a sequence $\eps_n\to 0$ such that 
$\spc{Q}$ is the set of all compact metric spaces $\spc{X}$ such that
$\pack_{\eps_n} \spc{X}\le n$.

Note that $\diam \spc{X}\le \eps_1$ for any $\spc{X}\in \spc{Q}$.
Given positive integer $n$ consider set of all metric spaces $\spc{W}_n$
with number of points at most $n$ and diameter $\le \eps_1$.
Note that for each $n$ $\spc{W}_n$ forms a compact set in $\GH$.

Further a subspace formed by a maximal $\eps_n$-packing of any $\spc{X}\in\spc{Q}$ belongs to $\spc{W}_n$.
Therefore $\spc{W}_n\cap\spc{Q}$ is a comapct $\eps_n$-net in  $\spc{Q}$.
That is, $\spc{Q}$ has a compact $\eps$-net for any $\eps>0$.
Since $\spc{Q}$ is closed in a complete space $\GH$, it implies that $\spc{Q}$ is compact.
\qeds

In the following exercises {}\emph{converge} means {}\emph{converge in the sense of Gromov--Hausdorff}.

\begin{thm}{Exercise}\label{ex:GH-SC}
\begin{subthm}{ex:GH-SC:circle}
Show that a sequence of compact simply connected length spaces cannot converge to a circle.
\end{subthm}

\begin{subthm}{ex:GH-SC:nonsc-limit}
Construct a sequence of compact simply connected length spaces that converges to a compact nonsimply connected space.
\end{subthm}
\end{thm}

\begin{thm}{Exercise}\label{ex:sphere-to-ball}
\begin{subthm}{ex:sphere-to-ball:2}
Show that a sequence of lenght metrics on the 2-sphere cannot converge to a the unit disc.
\end{subthm}

\begin{subthm}{ex:sphere-to-ball:3}
Construct a sequence of lenght metrics on the 3-sphere that converges to a unit 3-ball.
\end{subthm}

\end{thm}

\section{Remarks}

Suppose $\spc{X}_n\GHto \spc{X}_\infty$, then there is a metric on the disjoint union 
\[\bm{X}=\bigsqcup_{n\in \NN\cup\{\infty\}} \spc{X}_n\]
such that the restriction of metric on each $\spc{X}_n$ and $\spc{X}_\infty$ coincides with its original metric and $\spc{X}_n\Hto \spc{X}_\infty$ as subsets in $\bm{X}$.

Indeed, since $\spc{X}_n\GHto \spc{X}_\infty$, there is a metric on $\spc{V}_n=\spc{X}_n\sqcup \spc{X}_\infty$ such that the restriction of metric on each $\spc{X}_n$ and $\spc{X}_\infty$ coincides with its original metric and $\dist{\spc{X}_n}{\spc{X}_\infty}{\Haus\spc{V}_n}<\eps_n$ for some sequence $\eps_n\to 0$.
Gluing all $\spc{V}_n$ along $\spc{X}_\infty$, we obtaine the required space $\bm{X}$.

In other words, the metric on $\bm{X}$ defines convergence $\spc{X}_n\GHto \spc{X}_\infty$.
This metric makes possible to talk about limits of sequences $x_n\in \spc{X}_n$ as $n\to\infty$, as well as weak limit of a sequence of measures $\mu_n$ on $\spc{X}_n$ and so on.
By that reason it might be useful to fix such metric on $\bm{X}$.
This approach can be also used to define Gromov--Hausdorff convergence of noncompact spaces which will which we are about to discuss.

We may consider a metric on $\bm{X}$ such that $\spc{X}_n\Hto \spc{X}_\infty$ without assuming that all the spaces $\spc{X}_n$ and $\spc{X}_\infty$ are compact; in this case we need to use the variation of Hausdorff convergence described in Section~\ref{sec:H-variation}.
In this case the limit spaces for this generalized convergence is not uniquely defined.
\begin{figure}[h!]
\vskip-0mm
\centering
\includegraphics{mppics/pic-500}
\end{figure}
For example if each space $\spc{X}_n$ in the sequence is isometric to the half-line, then its limit might be isometric to the half-line or to whole line.
The first convergence is evident and the second could be guessed from the diagram.



Often the isometry class of the limit can be fixed by marking a point $p_n$ in each space $\spc{X}_n$, it is called \index{pointed convergence}\emph{pointed Gromov--Haudorff convergence} --- we say that $(\spc{X}_n,p_n)$ converges to $(\spc{X}_\infty,p_\infty)$ if there is a metric on $\bm{X}$ such that $\spc{X}_n\Hto \spc{X}_\infty$ and $p_n\to p_\infty$.
For example the sequence $(\spc{X}_n,p_n)=(\RR_+,0)$ converges to $(\RR_+,0)$, while $(\spc{X}_n,p_n)=(\RR_+,n)$ converges to $(\RR,0)$.

This convergence works nicely for proper metric spaces.
The following theorem is an analog of Gromov's selection theorem for pointed Gromov--Haudorff convergence.

\begin{thm}{Theorem}\label{thm:pointed-gromov-compactness}%
Let $\spc{Q}$ be a set of isometry classes of pointed proper metric spaces
$(\spc{X},p)$.
Assume that for any $R>0$, the $R$-balls in the spaces centered at the marked points form a uniformly totally bounded family of spaces.
Then $\spc{Q}$ is precompact with respect to pointed Gromov--Haudorff convergence. 
\end{thm}
