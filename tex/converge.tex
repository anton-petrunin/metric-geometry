\chapter{Space of sets}

\section{Hausdorff convergence}

Let $\spc{X}$ be a metric space.
Given a subset $A\subset \spc{X}$,
consider the distance function to $A$
$$\distfun_A: \spc{X} \to [0,\infty)$$
defined as 
$$\distfun_A(x)
\df
\inf_{a\in A}\{\,\dist ax{\spc{X}}\,\}.$$

\begin{thm}{Definition}\label{def:hausdorff-convergence}
Let $A$ and $B$ be two compact subsets of a metric space $\spc{X}$.
Then the \emph{Hausdorff distance} between $A$ and $B$ is defined as 
$$|A-B|_{\mathcal{H}(\spc{X})}
\df
\sup_{x\in \spc{X}}\{\,|\distfun_A(x)-\distfun_B(x)|\,\}.
$$

\end{thm}
 
Suppose $A$ and $B$ be two compact subsets of a metric space $\spc{X}$.
It is straightforward to check that $|A-B|_{\mathcal{H}(\spc{X})}\le R$ if and only if 
$\distfun_A(b)\le R$ for any $b\in B$
and 
$\distfun_B(a)\le R$ for any $a\in A$.
In other words, $|A-B|_{\mathcal{H}(\spc{X})}< R$ if and only if 
$B$ lies in a $R$-neighborhood of $A$, 
and 
$A$ lies in a $R$-neighborhood of $B$.

Note that the set of all nonempty compact subsets of a metric space $\spc{X}$ equipped with the Hausdorff metric forms a metric space.
This new metric space will be denoted as $\mathcal{H}(\spc{X})$.


\begin{thm}{Exercise}\label{ex:diam}
Let $\spc{X}$ be a metric space.
Given a subset $A\subset \spc{X}$ define its diameter as 
$$\diam A\df\sup_{a,b\in A} |a-b|.$$

Show that 
$$\diam\:\mathcal{H}(\spc{X})\to \RR$$ 
is a $2$-Lipschitz function;
that is, $|\diam A-\diam B|\le 2\cdot\dist{A}{B}{\mathcal{H}(\spc{X})}$.
\end{thm}


\begin{thm}{Blaschke selection theorem}\label{thm:compact+Hausdorff}
Let $\spc{X}$ be a metric space.
Then the space $\mathcal{H}(\spc{X})$ is compact if and only if $\spc{X}$ is compact.
\end{thm}

Note that the theorem implies that from any sequence of compact sets in $\spc{X}$ one can select a subsequence converging in the sense of Hausdorff; 
by that reason it is called a selection theorem. 

\parit{Proof; ``only if'' part.}
Note that the map $\iota\:\spc{X}\to \mathcal{H}(\spc{X})$, defined as $\iota\:x\mapsto\{x\}$
(that is, point $x$ mapped to the one-point subset $\{x\}$ of $\spc{X}$)
is distance preserving.
Therefore $\spc{X}$ is isometric to the set $\iota(\spc{X})$ in $\mathcal{H}(\spc{X})$.

Note that for a nonempty subset $A\subset \spc{X}$, we have $\diam A=0$ if and only if $A$ is a one-point set.
Therefore, from Exercise~\ref{ex:diam}, it follows 
that $\iota(\spc{X})$ is closed in $\mathcal{H}(\spc{X})$.

Hence $\iota(\spc{X})$  is compact, as it is a closed subset of a compact space. 
Since $\spc{X}$ is isometric to $\iota(\spc{X})$,
``only if'' part follows.
\qeds

To prove ``if'' part we will need the following two lemmas.%???CHANGE THE PROOF???

\begin{thm}{Lemma}\label{lem:decreasing-converges}
Let $K_1\supset K_2\supset\dots$ be a sequence of nonempty compact sets in a metric space $\spc{X}$
then $K_\infty=\bigcap_n K_n$ is the Hausdorff limit of $K_n$;
that is, $|K_\infty-K_n|_{\mathcal{H}(\spc{X})}\to0$ as $n\to\infty$.
\end{thm}

\parit{Proof.}
Note that $K_\infty$ is compact;
by finite intersection property, $K_\infty$ is nonempty.

If the assertion were false, 
then there is $\eps>0$ such that for each $n$ 
one can choose $x_n\in K_n$
such that $\distfun_{K_\infty}(x_n)\ge\eps$.
Note that $x_n\in K_1$ for each $n$.
Since $K_1$ is compact, 
there is 
a \emph{partial limit}\index{partial limit}\footnote{Partial limit is a limit of a subsequence.}
 $x_\infty$ of $x_n$.
Clearly $\distfun_{K_\infty}(x_\infty)\ge \eps$.

On the other hand, since $K_n$ is closed and $x_m\in K_n$ for $m\ge n$,
we get $x_\infty\in K_n$ for each $n$.
It follows that $x_\infty\in K_\infty$ and therefore $\distfun_{K_\infty}(x_\infty)=0$,
a contradiction.\qeds


\begin{thm}{Lemma}\label{lem:complete+Hausdorff}
If $\spc{X}$ is a compact metric space then $\mathcal{H}(\spc{X})$
is complete.
\end{thm}

\parit{Proof.}
Let $(Q_n)$ be a Cauchy sequence in $\mathcal{H}(\spc{X})$.
Passing to a subsequence of $Q_n$ we may assume that 
$$|Q_n-Q_{n+1}|_{\mathcal{H}(\spc{X})}\le \tfrac1{10^n}\eqlbl{eq:eps=1/10}$$
for each $n$.

Set 
\begin{align*}
K_n&= \set{x\in \spc{X}}{\distfun_{Q_n}(x)\le \tfrac1{10^n}}
\end{align*}
Since $\spc{X}$ is compact so is each $K_n$.

Clearly, $|Q_n-K_n|_{\mathcal{H}(\spc{X})}\le \tfrac1{10^n}$ and from \ref{eq:eps=1/10}, we get
$K_n\supset K_{n+1}$ 
for each $n$.
Set 
$$K_\infty=\bigcap_{n=1}^\infty K_n.$$
Applying Lemma \ref{lem:decreasing-converges},
we get that $|K_n-K_\infty|_{\mathcal{H}(\spc{X})}\to 0$ as $n\to\infty$.
Since $|Q_n-K_n|_{\mathcal{H}(\spc{X})}\le \tfrac1{10^n}$, we get $|Q_n-K_\infty|_{\mathcal{H}(\spc{X})}\to 0$ as $n\to\infty$ --- hence the lemma.
\qeds

\begin{thm}{Exercise}
Let $\spc{X}$ be a complete metric space and $K_n$ be a sequence of compact sets 
which converges in the sence of Hausdorff.
Show that closure of the union $\bigcup_{n=1}^\infty K_n$ is compact.

Use this to show that in Lemma~\ref{lem:complete+Hausdorff} compactness of $\spc{X}$ can be exchanged to completeness.
\end{thm}

\parit{Proof of ``if'' part in \ref{thm:compact+Hausdorff}.}
According to Lemma~\ref{lem:complete+Hausdorff},
$\mathcal{H}(\spc{X})$ is complete.
It remains to show that $\mathcal{H}(\spc{X})$ is totally bounded (\ref{totally-bounded});
that is, given $\eps>0$ there is a finite $\eps$-net in $\mathcal{H}(\spc{X})$.

Choose a finite $\eps$-net $A$ in $\spc{X}$.
Denote by $\mathcal{A}$ the set of all subsets of $A$.
Note that  $\mathcal{A}$ is finite set in $\mathcal{H}(\spc{X})$.
For each compact set $K\subset \spc{X}$, consider the subset $K'$ of all points $a\in A$
such that $\distfun_K(a)\le \eps$.
Then $K' \in \mathcal{A}$ and $|K-K'|_{\mathcal{H}(\spc{X})}\le\eps$.
In other words $\mathcal{A}$ is a finite $\eps$-net in $\mathcal{H}(\spc{X})$.
\qeds

Hausdorff metric defines convergence of compact sets which is more important than metric itself.

\begin{thm}{Exercise}\label{ex:Hausdorff-bry}
Let $X$ and $Y$ be two compact subsets in $\RR^2$.
Assume $|X-Y|_{\mathcal{H}(\RR^2)}<\eps$, 
is it true that
$|\partial X-\partial Y|_{\mathcal{H}(\RR^2)}<\eps$,
where $\partial X$ denotes the boundary of $X$.

Does the converse holds? That is, assume $X$ and $Y$ be two compact subsets in $\RR^2$
and $|\partial X-\partial Y|_{\mathcal{H}(\RR^2)}<\eps$; 
is it true that $|X-Y|_{\mathcal{H}(\RR^2)}\z<\eps$?
\end{thm}

\begin{thm}{Exercise}\label{ex:Huas-perimeter-area}
Let $\spc{C}$ be a subspace of $\mathcal{H}(\RR^2)$ formed by all compact convex subsets in $\RR^2$.
Show that perimeter\footnote{If the set degenerates to a line segment of length $\ell$ then its perimeter is defined as $2\cdot \ell$.} and area are continuous on~$\spc{C}$.
That is, if a sequence of convex compact plane sets $X_n$ converges to $X_\infty$ in the sense of Hausdorff, then 
\[\perim X_n\to \perim X_\infty\quad\text{and}\quad\area X_n\to\area X_\infty\]
as $n\to\infty$.
\end{thm}

The above exercise can be used in a proof of isoperimetrical inequality in the plane;
it states that \emph{among the plane figures bounded by closed curves of length at most $\ell$ the round disc has maximal area}.

Indeed it is sufficient to consider only convex figures of given perimeter; if a figure is not convex pass to its convex hull and observe that it has larger area and smaller perimeter.
Further the exercise guarantees existence of a figure $D_\ell$ with perimeter $\ell$ and maximal area.
It remains to show that $D_\ell$ is a round disc.
The latter is easy to show, see for example Steiner's 4-joint method \cite{blaschke}.


\section{A variation}

It seems that \emph{Hausdorff convergence} was first introduced by Felix Hausdorff~\cite{hausdorff},
and a couple of years later an equivalent definition was given by Wilhelm Blaschke~\cite{blaschke}.

The following refinement of the definition was introduced by  Zden\v{e}k Frol\'{\i}k in \cite{frolik},
and later rediscovered by Robert Wijsman in~\cite{wijsman}.  
This refinement takes an intermediate place between the original Hausdorff convergence and {}\emph{closed convergence}, also introduced by Hausdorff in \cite{hausdorff};
so we still call it Hausdorff convergence.

\begin{thm}{Definition}\label{def:gen-Haus-conv}
Let $(A_n)$ be a sequence of closed sets in a metric space $\spc{X}$.
We say that $(A_n)$ converges to a closed set $A_\infty$ in the sense of Hausdorff if $\distfun_{A_n}(x)\to \distfun_{A_\infty}(x)$ for any $x\in\spc{X}$.
\end{thm}

For example, suppose $\spc{X}$ is the Euclidean plane and $A_n$ is the circle with radius $n$ and center at $(n,0)$.
If we use the standard definition (\ref{def:hausdorff-convergence}), then the sequence $(A_n)$ diverges, but it converges to the $y$-axis in the sense of Definition~\ref{def:gen-Haus-conv}.

The following exercise is analogous to the Blaschke selection theorem (\ref{thm:compact+Hausdorff}).

\begin{thm}{Exercise}
Let $\spc{X}$ be a proper metric space
and $(A_n)_{n=1}^\infty$ be a sequence of closed sets in~$\spc{X}$.
Assume that for some (and therefore any) point  $x\in\spc{X}$, 
the sequence $a_n=\distfun_{A_n}(x)$ is bounded.
Show that the sequence  $(A_n)_{n=1}^\infty$ has a convergent subsequence in the sense of Definition~\ref{def:gen-Haus-conv}.
\end{thm}

\chapter{Space of spaces}

\section{Gromov--Hausdorff metric}

The goal of this section is to cook up a metric space out of metric spaces.
More precisely, we want to define the so called  Gromov--Hausdorff metric on the set of \emph{isometry classes} of compact metric spaces.
(Being isometric is an equivalence relation, 
and an isometry class is an equivalence class with respect to this equivalence relation.)

The obtained metric space will be denoted as $\spc{M}$.
Given two metric spaces $\spc{X}$ and $\spc{Y}$,
denote by $[\spc{X}]$ and $[\spc{Y}]$ their isometry classes;
that is, $\spc{X}'\in [\spc{X}]$ if and only if $\spc{X}'\iso \spc{X}$.
Pedantically, the Gromov--Hausdorff distance from $[\spc{X}]$ 
to $[\spc{Y}]$ should be denoted as $|[\spc{X}]-[\spc{Y}]|_{\spc{M}}$;
but we will often write it as $|\spc{X}-\spc{Y}|_{\spc{M}}$ and say (not quite correctly) 
``$|\spc{X}-\spc{Y}|_{\spc{M}}$ is the Gromov--Hausdorff distance from  $\spc{X}$ 
to  $\spc{Y}$''.
In other words, from now on the term \emph{metric space} might stands for \emph{isometry class of this metric space}.

The metric on $\spc{M}$ is maximal metric such that \emph{the distance between subspaces in a metric space is not greater than the Hausdorff distance between them}.
Here is a formal definition:

\begin{thm}{Definition}\label{def:GH}
Let $\spc{X}$ and $\spc{Y}$ be compact metric spaces. 
The Gromov--Hausdorff distance $|\spc{X}-\spc{Y}|_{\spc{M}}$
between them is defined by the following
relation.
 
Given  $r > 0$, we have that $|\spc{X}-\spc{Y}|_{\spc{M}} < r$ if and only if there exist a metric
space $\spc{Z}$ and subspaces $\spc{X}'$ and $\spc{Y}'$ in $\spc{Z}$ that are isometric to $\spc{X}$ and $\spc{Y}$
respectively and such that $|\spc{X}'-\spc{Y}'|_{\mathcal{H}(\spc{Z})} < r$. 
(Here $|\spc{X}'-\spc{Y}'|_{\mathcal{H}(\spc{Z})}$ denotes the Hausdorff distance between sets $\spc{X}'$ and $\spc{Y}'$ in $\spc{Z}$.)
\end{thm}

Bit later (see \ref{thm:GH-is-a-metric}) we will show that \emph{Hausdorff metric} is indeed a metric.

We say that a sequence
of (isometry classes of) compact metric spaces $\spc{X}_n$ 
\emph{converges in the sense of Gromov--Hausdorff}\index{converges in the sense of Gromov--Hausdorff} to the (isometry classes of)
compact metric space $\spc{X}_\infty$ if $|\spc{X}_n - \spc{X}_\infty|_{\spc{M}} \to 0$ as $n\z\to\infty$;
in this case we write $\spc{X}_n\GHto \spc{X}_\infty$.

\section{Reformulations}

Let us discuss few alternative ways to define the Gromov--Hausdorff metric.

\parbf{Metrics on disjoined union.}
Definition~\ref{def:GH} deals with a huge class of metric spaces,
namely, all metric spaces $\spc{Z}$ that contain subspaces isometric to $\spc{X}$ and $\spc{Y}$.
It is possible to reduce this class to metrics on the disjoint unions of $\spc{X}$ and $\spc{Y}$. 
More precisely, 

\begin{thm}{Proposition}\label{prop:GH=X+Y}
The Gromov--Hausdorff distance between two compact metric spaces $\spc{X}$
and $\spc{Y}$ is the infimum of $r > 0$ such that there exists a metric
$|{*}-{*}|_{\spc{W}}$ on the disjoint union $\spc{W}=\spc{X}\sqcup \spc{Y}$ 
such that the restrictions of $|{*}-{*}|_{\spc{W}}$ to $\spc{X}$ and $\spc{Y}$
coincide with $|{*}-{*}|_\spc{X}$ and $|{*}-{*}|_{\spc{Y}}$ 
and $|\spc{X}-\spc{Y}|_{\mathcal{H}(\spc{W})} < r$. 
\end{thm}


\parit{Proof.}
Identify $\spc{X}\sqcup \spc{Y}$ with $\spc{X}'\cup \spc{Y}' \subset \spc{Z}$ 
(the notation
is from Definition~\ref{def:GH}). 

More formally, fix isometries $f\: \spc{X} \to \spc{X}'$ and
$g\: \spc{Y} \to \spc{Y}'$, then define the distance between $x \in \spc{X}$ and $y \in \spc{Y}$ by
$|x-y|_{\spc{W}} = |f (x)- g(y)|_{\spc{Z}}+\eps$ for small enuf $\eps>0$.%
\footnote{We add $\eps$ to ensure that $d(x, y) > 0$ for any $x\in \spc{X}$ and $y\in \spc{Y}$;
so $|x-y|_{\spc{W}}$ is indeed a metric.}
This yields a metric on $\spc{W}=\spc{X}\sqcup \spc{Y}$ for which
$|\spc{X}- \spc{Y}|_{\mathcal{H}(\spc{W})} \z< r$.
\qeds

\parbf{Fixed ambient space.}
The following proposition says that the space $\spc{Z}$ in Definition~\ref{def:GH} can be exchanged to a fixed space, namely $\ell^\infty$ --- the space of bounded infinite sequences with the metric defined by $\sup$-norm.

\begin{thm}{Proposition}\label{prop:GH-with-fixed-Z}
Let $\spc{X}$ and $\spc{Y}$ be comact metric spaces.
Then
$$|\spc{X}-\spc{Y}|_{\spc{M}} = \inf \{|\spc{X}'-\spc{Y}'|_{\mathcal{H}(\ell^\infty)}\}$$ 
where the infimum is taken over all pairs of sets $\spc{X}'$ and $\spc{Y}'$ in $\ell^\infty$
which isometric to  $\spc{X}$ and $\spc{Y}$ correspondingly.  
\end{thm}




\parit{Proof of \ref{prop:GH-with-fixed-Z}.}
By the definition, we have that 
\[|\spc{X}-\spc{Y}|_{\spc{M}} \leq \inf \{|\spc{X}'-\spc{Y}'|_{\mathcal{H}(\ell^\infty)}\}.\]

Let $\spc{W}$ be an arbitrary metric space with the underlying set $\spc{X}\sqcup \spc{Y}$.
Note $\spc{W}$ is compact since it is union of two compact subsets $\spc{X},\spc{Y}\subset \spc{W}$.
In particular, $\spc{W}$ is separable.

By Lemma \ref{lem:frechet}, there is an distance preserving embedding $\iota\:\spc{W}\z\to \ell^\infty$.
It remains to apply Proposition~\ref{prop:GH=X+Y}.
\qeds



\section{Almost isometries}\label{sec:alm-isom}

\begin{thm}{Definition} Let $\spc{X}$ and $\spc{Y}$ be metric spaces and $\eps > 0$. 
A  map\footnote{possibly noncontinuous} $f\: \spc{X} \z\to \spc{Y}$ is called an $\eps$-isometry 
if 
$$|f(x)-f(x')|_{\spc{Y}}\lege |x-x'|_{\spc{X}}\pm\eps$$
for any $x,x'\in \spc{X}$ 
and if $f(\spc{X})$ is an $\eps$-net in $\spc{Y}$.
\end{thm}

\begin{thm}{Exercise}\label{ex:alm-isom:compositon}\label{ex:alm-isom:inverse}\label{ex:GH=>eps-isom}
\begin{subthm}{}
Let $f\:\spc{X}\to \spc{Y}$ and $g\:\spc{Y}\to \spc{Z}$ be two $\eps$-isometries.
Show that $g\circ f\: \spc{X}\to \spc{Z}$ is a $(3\cdot\eps)$-isometry.
\end{subthm}

\begin{subthm}{}
Assume $f\: \spc{X} \z\to \spc{Y}$ is an $\eps$-isometry.
Show that there is a $(3\cdot\eps)$-isometry 
$g\: \spc{Y}\to \spc{X}$.
\end{subthm}

\begin{subthm}{}
 Assume $|\spc{X}-\spc{Y}|_{\spc{M}}<\eps$, show that there is a $(2\cdot\eps)$-isometry 
$f\: \spc{X}\to \spc{Y}$.
\end{subthm}
\end{thm}

\begin{thm}{Proposition}\label{prop:alm-isom=>GH}
Let $\spc{X}$ and $\spc{Y}$ be metric spaces 
and let $f\: \spc{X}\to \spc{Y}$ be an $\eps$-isometry.
Then 
\[|\spc{X}-\spc{Y}|_{\spc{M}}\le 2\cdot \eps.\]
\end{thm}

\parit{Proof.} Consider the set $\spc{W}=\spc{X}\sqcup \spc{Y}$.
Note that the following defines a metric on $\spc{W}$:
\begin{itemize}
\item  For any $x,x'\in \spc{X}$
$$|x-x'|_{\spc{W}}=|x-x'|_{\spc{X}};$$
\item For any $y,y'\in \spc{Y}$,
$$|y-y'|_{\spc{W}}=|y-y'|_{\spc{Y}}$$
\item For any $x\in \spc{X}$ and $y\in \spc{Y}$,
$$|x-y|_{\spc{W}}=\eps+\inf_{x'\in \spc{X}}\{|x-x'|_{\spc{X}}+|f(x')-y|_{\spc{Y}}\}.$$
\end{itemize}

Since $f(\spc{X})$ is an $\eps$-net in $\spc{Y}$,
for any $y\in \spc{Y}$ there is $x\in \spc{X}$ such that $|f(x)-y|_{\spc{Y}}\z\le\eps$;
therefore $|x-y|_{\spc{W}}\le 2\cdot\eps$.
On the other hand for any $x\in \spc{X}$, we have $|x-y|_{\spc{W}}\le\eps$
for $y=f(x)\in \spc{Y}$.

It follows that $|\spc{X}-\spc{Y}|_{\mathcal{H}(\spc{W})}\le 2\cdot\eps$.
\qedsf


\section{It is a metric}

\begin{thm}{Theorem}\label{thm:GH-is-a-metric}
The set of isometry classes of compact metric spaces equipped with Gromov--Hausdorff metric forms a metric space (which is denoted by $\spc{M}$).
\end{thm}

\parit{Proof.}
Let $\spc{X}$, $\spc{Y}$ and $\spc{Z}$ be arbitrary  compact metric spaces.
We need to check the following:
\begin{enumerate}[{\it (i)}]
\item\label{GH-1} $|\spc{X}-\spc{Y}|_{\spc{M}}\ge 0$;
\item\label{GH-2} $|\spc{X}-\spc{Y}|_{\spc{M}}=0$ if and only if $\spc{X}$ is isometric to $\spc{Y}$;
\item\label{GH-3} $|\spc{X}-\spc{Y}|_{\spc{M}}=|\spc{Y}-\spc{X}|_{\spc{M}}$;
\item\label{GH-4} $|\spc{X}-\spc{Y}|_{\spc{M}}+|\spc{Y}-\spc{Z}|_{\spc{M}}\ge |\spc{X}-\spc{Z}|_{\spc{M}}$.
\end{enumerate}


Note that {\it (\ref{GH-1})}, {\it(\ref{GH-3})} and ``if''-part of {\it(\ref{GH-2})} follow directly from Definition \ref{def:GH}.

\parit{(\ref{GH-4}).}
Choose arbitrary $a,b \in \mathbb{R}$ such that
$$a>|\spc{X}-\spc{Y}|_{\spc{M}}\ \ \text{and}\ \  b>|\spc{Y}-\spc{Z}|_{\spc{M}}.$$
Choose two metrics on $\spc{U}=\spc{X}\sqcup \spc{Y}$ and $\spc{V}=\spc{Y}\sqcup \spc{Z}$ so that
$|\spc{X}-\spc{Y}|_{\mathcal{H}(\spc{U})}<a$ and $|\spc{Y}-\spc{Z}|_{\mathcal{H}(\spc{V})}<b$ 
and the inclusions $\spc{X}\hookrightarrow \spc{U}$, $\spc{Y}\hookrightarrow \spc{U}$, $\spc{Y}\hookrightarrow \spc{V}$ and $\spc{Z}\hookrightarrow \spc{V}$ are distance preserving.

Consider the metric on $\spc{W}=\spc{X}\sqcup \spc{Z}$ 
so that inclusions $\spc{X}\hookrightarrow \spc{W}$ and $\spc{Z}\hookrightarrow \spc{W}$ are distance preserving
and 
$$|x-z|_{\spc{W}}=\inf_{y\in \spc{Y}}\{|x-y|_{\spc{U}}+|y-z|_{\spc{V}}\}.$$
Note that $|{*}-{*}|_{\spc{W}}$ is indeed a metric and 
$$|\spc{X}-\spc{Z}|_{\mathcal{H}(\spc{W})}<a+b.$$
Property {\it (\ref{GH-4})} follows since the last inequality holds for any $a>|\spc{X}\z-\spc{Y}|_{\spc{M}}$ and $b>|\spc{Y}-\spc{Z}|_{\spc{M}}$.


\parit{``Only if''-part of (\ref{GH-2}).}
According to Exercise~\ref{ex:GH=>eps-isom},
for any sequence $\eps_n\to0+$ there is a sequence of $\eps_n$-isometries 
$f_n\:\spc{X}\to \spc{Y}$.

Since $\spc{X}$ is compact, 
we can choose a countable dense set
$S$ in $\spc{X}$.
Use a diagonal procedure if necessary, to pass to a subsequence of $(f_n)$
such that for every $x \in S$ the sequence $(f_n(x))$ 
converges in $\spc{Y}$. 
Consider the pointwise limit map  $f_\infty \: S \to \spc{Y}$ defined by
 $$f_\infty(x) = \lim_{n\to\infty} f_n (x)$$ for every $x \in S$. 
Since $$|f_n (x)- f_n (x')|_{\spc{Y}}\lege |x- x'|_\spc{X} \pm\eps_n,$$ 
we have 
$$|f_\infty(x)-f_\infty (x')|_{\spc{Y}} 
= \lim_{n\to\infty} |f_n(x)-f_n (x')|_{\spc{Y}} 
= |x -x'|_\spc{X}$$ for all
$x, x' \in S$; 
that is, $f_\infty\:S\to \spc{Y}$ is a distance-preserving map. 
Therefore $f_\infty$ can be extended to a distance-preserving map from all of $\spc{X}$ to $\spc{Y}$.
The later is done by setting 
$$f_\infty(x)=\lim_{n\to\infty} f_\infty(x_n)$$ 
for some (and therefore any) sequence of points $(x_n)$ in $S$
which converges to $x$ in $\spc{X}$.
(Note that if $x_n\to x$ then $(x_n)$ is Cauchy.
Since $f_\infty$ is distance preserving, $y_n=f_\infty(x_n)$ is also a Cauchy sequence in $\spc{Y}$;
therefore it converges.)

This way we obtain a distance preserving map $f_\infty\:\spc{X}\to \spc{Y}$. 
It remains to show that $f_\infty$ is surjective; that is, $f_\infty(\spc{X})=\spc{Y}$.

Note that in the same way we can obtain a distance preserving map $g_\infty\:\spc{Y}\z\to \spc{X}$.
If $f_\infty$ is not surjective, then neither is $f_\infty\circ g_\infty\:\spc{Y}\to \spc{Y}$.
So $f_\infty \circ g_\infty$ is a distance preserving map from a compact space to itself which is not an isometry.
The later contradicts Exercise~\ref{ex:non-contracting-map}. 
\qeds

\begin{thm}{Exercise}\label{ex:d_GH-and-diam}
 Let $\spc{X}$ and $\spc{Y}$ be two compact metric spaces.
Prove that 
$$|\diam \spc{X} - \diam \spc{Y} |\le 2\cdot |\spc{X}-\spc{Y}|_{\spc{M}}.$$
In other words, $\diam\:\spc{M}\to\RR$ is a $2$-Lipschitz function.
\end{thm}

\begin{thm}{Exercise}
Show that $\spc{M}$ is a length metric space.
\end{thm}

Given two metric spaces $\spc{X}$ and $\spc{Y}$, we will write $\spc{X}\le \spc{Y}$ if there is a noncontracting map $f\:\spc{X}\to \spc{Y}$;
that is, if 
$$ |x-x'|_{\spc{X}}\le|f(x)-f(x')|_{\spc{Y}}$$
for any $x,x'\in \spc{X}$.

Further, given $\eps>0$, we will write $\spc{X}\le \spc{Y}+\eps$
if there is a map $f\:\spc{X}\to \spc{Y}$ such that 
$$|x-x'|_{\spc{X}}\le|f(x)-f(x')|_{\spc{Y}}+\eps$$
for any $x,x'\in \spc{X}$.

\begin{thm}{Exercise}\label{ex:GH-po}
Show that 
$$\dist{\spc{X}}{\spc{Y}}{\spc{M}'}=\inf\set{\eps}{\spc{X}\le \spc{Y}+\eps\ \ \text{and}\ \ \spc{Y}\le \spc{X}+\eps}$$
defines a metric on the space of (isometry classes) of compact metric spaces.

Moreover $\dist{*}{*}{\spc{M}'}$ is equivalent to the Gromov--Haudorff metric;
that is,
$$|\spc{X}_n-\spc{X}_\infty|_{\mathcal{M}}\to 0 \ \ \ \Leftrightarrow\ \ \ \dist{\spc{X}_n}{\spc{X}_\infty}{\spc{M}'}\to 0$$ 
as $n\to\infty$.
\end{thm}


\section{Gromov--Hausdorff convegence}

The Gromov--Hausdorff metric defines Gromov--Hausdorff convegence
and this is the only thing it is good for.
In other words in all applications, we use only topology on $\spc{M}$
and we do not care about particular value of Gromov--Hausdorff distance between spaces.

In order to determine that a given sequence of metric spaces $(\spc{X}_n)$ converges in the Gromov--Hausdorff sense to $\spc{X}_\infty$, it is sufficient to estimate distances $|\spc{X}_n-\spc{X}_\infty|_{\spc{M}}$ and  check if $|\spc{X}_n-\spc{X}_\infty|_{\spc{M}}\to 0$.
This problem turns to be simpler than finding Gromov--Hausdorff distance between a particular pair of spaces.
The proposition below gives one way to do this.

\begin{thm}{Proposition}\label{prop:GH-e-isom}
A sequence of compact metric spaces $(\spc{X}_n)$ converges to  $\spc{X}_\infty$ in the sense of Gromov--Hausdorff if and only if there is a sequence $\eps_n\to0+$
and an $\eps_n$-isometry $f_n\:\spc{X}_n\to \spc{X}_\infty$ for each $n$.
\end{thm}

\parit{Proof.} Follows from Propsition~\ref{prop:alm-isom=>GH} and Exercise~\ref{ex:GH=>eps-isom}
\qeds

Suppose $\spc{X}_n\GHto \spc{X}_\infty$, then there is a metric on the disjoint union 
\[\bm{X}=\bigsqcup_{n\in \NN\cup\{\infty\}} \spc{X}_n\]
such that the restriction of metric on each $\spc{X}_n$ and $\spc{X}_\infty$ coincides with its original metric and and $\spc{X}_n\Hto \spc{X}_\infty$ as subsets in $\bm{X}$.

Indeed, since $\spc{X}_n\GHto \spc{X}_\infty$, there is a metric on $\spc{V}_n=\spc{X}_n\sqcup \spc{X}_\infty$ such that the restriction of metric on each $\spc{X}_n$ and $\spc{X}_\infty$ coincides with its original metric and $\dist{\spc{X}_n}{\spc{X}_\infty}{\mathcal{H}(\spc{V}_n)}<\eps_n$ for some sequence $\eps_n\to 0$.
Arguing as in the proof of (\ref{GH-4}) in Theorem~\ref{thm:GH-is-a-metric}
we define metric on $\bm{X}$ by setting 
\begin{align*}
\dist{x_m}{x_n}{\bm{X}}&=\inf_{x_\infty}\set{\dist{x_m}{x_\infty}{\spc{V}_m}+\dist{x_n}{x_\infty}{\spc{V}_n}}{},
\\
\dist{x_n}{x_\infty}{\bm{X}}&=\dist{x_n}{x_\infty}{\spc{V}_n}
\\
\dist{x_n}{x'_n}{\bm{X}}&=\dist{x_n}{x_n'}{\spc{X}_n}
\end{align*}
where $x_n,x_n'\in\spc{X}_n$ for every $n\in \NN\cup\{\infty\}$.

In other words, the metric on $\bm{X}$ defines convergence $\spc{X}_n\GHto \spc{X}_\infty$.
This metric makes possible to talk about limits of sequences $x_n\in \spc{X}_n$ as $n\to\infty$, as well as weak limit of a sequence of measures $\mu_n$ on $\spc{X}_n$ and so on.
By that reason it might be useful to fix such metric on $\bm{X}$.
This approach can be also used to define Gromov--Hausdorff convergence of noncompact spaces which will be discussed latter.




\section{Uniformly totally bonded sets}

Let $\spc{Q}$ be a set of (isometry classes) of compact metric spaces.
Suppose that there is a sequence $\eps_n\to 0$ such that for any positive integer $n$ each space $\spc{X}$ in $\spc{Q}$ admits an $\eps_n$-net with at most $n$ points.
Then we say that $\spc{Q}$ is \emph{uniformly totally bonded}.

Observe that in this case $\diam\spc{X}<\eps_1$ for any  $\spc{X}$ in $\spc{Q}$; that is diameters of spaces in $\spc{Q}$ are bounded above.

Fix a real constant $C$.
A measure $\mu$ on a metric space $\spc{X}$ is called \emph{$C$-doubling} if
\[\mu[\oBall(p,2\cdot r)< C\cdot\mu[\oBall(p,r)]\]
for any point $p\in \spc{X}$ and any positive real $r$.
A measure is called \emph{doubling} if it is \emph{$C$-doubling} for a some real constant $C$.

\begin{thm}{Exercise}\label{pr:doubling}
Let $\spc{Q}(C,D)$ be the set of all the compact metric spaces with diameter at most $D$ that admit a $C$-doubling measure.
Show that $\spc{Q}(C,D)$ is totally bounded.
\end{thm}

Given two metric spaces $\spc{X}$ and $\spc{Y}$ we will write $\spc{X}\le\spc{Y}$ if there is a distance non-decreasing map $\spc{X}\to\spc{Y}$

\begin{thm}{Exercise}\label{pr:under}

\begin{subthm}{pr:under:if}
Let $\spc{Y}$ be a compact metric space.
Show that the set of all spaces $\spc{X}$ such that $\spc{X}\le\spc{Y}$
is uniformly totally bounded.
\end{subthm}

\begin{subthm}{pr:under:only-if}
Show that for any uniformly totally bounded set $\spc{Q}\subset\spc{M}$ there is a compact space $\spc{Y}$
such that $\spc{X}\le\spc{Y}$ for any $\spc{X}$ in $\spc{Q}$.
\end{subthm}

\end{thm}

\section{Gromov's selection theorem}

The following theorem is analogous to Blaschke selection theorems (\ref{thm:compact+Hausdorff}).

\begin{thm}{Gromov selection theorem}\label{thm:gromov-compactness}%
Let $\spc{Q}$ be a closed and totally bounded subset of $\spc{M}$.
Then $\spc{Q}$ is compact.
\end{thm}


