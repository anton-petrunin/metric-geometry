\chapter{Space of sets}

\section{Hausdorff distance}

Let $\spc{X}$ be a metric space.
Given a subset $A\subset \spc{X}$,
consider the distance function to $A$
$$\distfun_A: \spc{X} \to [0,\infty)$$
defined as 
$$\distfun_A(x)
\df
\inf_{a\in A}\{\,\dist ax{\spc{X}}\,\}.$$

\begin{thm}{Definition}\label{def:hausdorff-convergence}
Let $A$ and $B$ be two compact subsets of a metric space $\spc{X}$.
Then the \index{Hausdorff distance}\emph{Hausdorff distance} between $A$ and $B$ is defined as 
$$|A-B|_{\Haus\spc{X}}
\df
\sup_{x\in \spc{X}}\{\,|\distfun_A(x)-\distfun_B(x)|\,\}.
$$

\end{thm}

The following observation gives a useful reformulation of the definition:

\begin{thm}{Observation}\label{obs:Haus-nbhds}
Suppose $A$ and $B$ be two compact subsets of a metric space $\spc{X}$.
Then $|A-B|_{\Haus\spc{X}}< R$ if and only if and only if 
$B$ lies in an $R$-neighborhood of $A$, 
and 
$A$ lies in an $R$-neighborhood of~$B$.
\end{thm}



Note that the set of all nonempty compact subsets of a metric space $\spc{X}$ equipped with the Hausdorff metric forms a metric space.
This new metric space will be denoted as $\Haus\spc{X}$.


\begin{thm}{Exercise}\label{ex:diam}
Let $\spc{X}$ be a metric space.
Given a subset $A\subset \spc{X}$ define its \index{diameter}\emph{diameter} as 
$$\diam A\df\sup_{a,b\in A} |a-b|.$$

Show that 
$$\diam\:\Haus\spc{X}\to \RR$$ 
is a \index{Lipschitz function}\emph{$2$-Lipschitz function};
that is,
\[|\diam A-\diam B|\le 2\cdot\dist{A}{B}{\Haus\spc{X}}\]
for any two compact nonempty sets $A,B\subset\spc{X}$.
\end{thm}


\begin{thm}{Exercise}\label{ex:Hausdorff-bry}
Let $A$ and $B$ be two compact subsets in the Euclidean plane $\RR^2$.
Assume $|A-B|_{\Haus\RR^2}<\eps$.

\begin{subthm}{ex:Hausdorff-bry:conv}
Show that $|\Conv A-\Conv B|_{\Haus\RR^2}<\eps$, where $\Conv A$ denoted the convex hull of $A$.
\end{subthm}
\begin{subthm}{ex:Hausdorff-bry:bry}
Is it true that
$|\partial A-\partial B|_{\Haus\RR^2}<\eps$,
where $\partial A$ denotes the boundary of $A$.

Does the converse hold? That is, assume $A$ and $B$ be two compact subsets in $\RR^2$
and $|\partial A-\partial B|_{\Haus\RR^2}<\eps$; 
is it true that $|A-B|_{\Haus\RR^2}\z<\eps$?
\end{subthm}

\end{thm}

Note that part \ref{SHORT.ex:Hausdorff-bry:conv} implies that $A\mapsto \Conv A$ defines a short map $\Haus\RR^2\to \Haus\RR^2$. 

\begin{thm}{Exercise}\label{ex:Haus-func}
Let $A$ and $B$ be two compact subsets in metric space $\spc{X}$.
Show that 
\[\dist{A}{B}{\Haus\spc{X}}=\sup_f\, \{\,\max_{a\in A}\{f(a)\}-\max_{b\in B}\{f(b)\,\},\]
where the least upper bound is taken for all $1$-Lipschitz functions $f$.

\end{thm}


\section{Hausdorff convergence}

\begin{thm}{Blaschke selection theorem}\label{thm:compact+Hausdorff}
A metric space $\spc{X}$ is compact if and only if
so is $\Haus\spc{X}$.
\end{thm}

The Hausdorff metric can be used to define convergence.
Namely, suppose $K_1,K_2,\dots$, and $K_\infty$ are compact sets in a metric space $\spc{X}$.
If $|K_\infty-K_n|_{\Haus\spc{X}}\to0$ as $n\to\infty$, then we say that 
the sequence $(K_n)$ {}\emph{converges} to $K_\infty$ \index{convergence in the sense of Hausdorff}\emph{in the sense of Hausdorff};
or we can say that $K_\infty$ is \emph{Hausdorff limit} of the sequence $(K_n)$.

Note that the theorem implies that from any sequence of compact sets in $\spc{X}$ one can select a subsequence that converges in the sense of Hausdorff; 
for that reason, it is called a \emph{selection} theorem. 

\parit{Proof; ``only if'' part.}
Consider the map $\iota$ that sends point $x\in \spc{X}$ to the one-point subset $\{x\}$ of $\spc{X}$.
Note that $\iota\:\spc{X}\to \Haus\spc{X}$ is distance-preserving.

Suppose that $A\subset \spc{X}$.
Note that $\diam A=0$ if and only if $A$ is a one-point set.
Therefore, from Exercise~\ref{ex:diam}, it follows 
that $\iota(\spc{X})$ is a closed subset of the compact space $\Haus\spc{X}$.
Whence $\iota(\spc{X})$, and therefore $\spc{X}$, are compact.
\qeds

To prove the ``if'' part we will need the following two lemmas.

\begin{thm}{Monotone convergence}\label{lem:decreasing-converges}
Let $K_1\supset K_2\supset\dots$ be a nested sequence of nonempty compact sets in a metric space $\spc{X}$
then $K_\infty\z=\bigcap_n K_n$ is the Hausdorff limit of $K_n$;
that is, $|K_\infty-K_n|_{\Haus\spc{X}}\to0$ as $n\to\infty$.
\end{thm}

\parit{Proof.}
By finite intersection property, $K_\infty$ is a nonempty compact set.

If the assertion were false, 
then there is $\eps>0$ such that for each $n$ 
one can choose $x_n\in K_n$
such that $\distfun_{K_\infty}(x_n)\ge\eps$.
Note that $x_n\in K_1$ for each $n$.
Since $K_1$ is compact, 
there is 
a \index{partial limit}\emph{partial limit}%
\footnote{Partial limit is a limit of a subsequence.}
 $x_\infty$ of $x_n$.
Clearly, $\distfun_{K_\infty}(x_\infty)\ge \eps$.

On the other hand, since $K_n$ is closed and $x_m\in K_n$ for $m\ge n$,
we get $x_\infty\in K_n$ for each $n$.
It follows that $x_\infty\in K_\infty$ and therefore $\distfun_{K_\infty}(x_\infty)=0$ ---
a contradiction.\qeds


\begin{thm}{Lemma}\label{lem:complete+Hausdorff}
If $\spc{X}$ is a compact metric space, then $\Haus\spc{X}$
is complete.
\end{thm}

\parit{Proof.}
Let $(Q_n)$ be a Cauchy sequence in $\Haus\spc{X}$.
Passing to a subsequence of $Q_n$ we may assume that 
$$|Q_n-Q_{n+1}|_{\Haus\spc{X}}\le \tfrac1{10^n}\eqlbl{eq:eps=1/10}$$
for each $n$.

Denote by $K_n$ the closed $\tfrac1{10^n}$-neighborhood of $Q_n$;
that is,
\begin{align*}
K_n&= \set{x\in \spc{X}}{\distfun_{Q_n}(x)\le \tfrac1{10^n}}
\end{align*}
Since $\spc{X}$ is compact so is each $K_n$.

By \ref{obs:Haus-nbhds}, $|Q_n-K_n|_{\Haus\spc{X}}\le \tfrac1{10^n}$.
From \ref{eq:eps=1/10}, we get
$K_n\supset K_{n+1}$ 
for each $n$.
Set 
$$K_\infty=\bigcap_{n=1}^\infty K_n.$$
By the monotone convergence (\ref{lem:decreasing-converges}),
 $|K_n-K_\infty|_{\Haus\spc{X}}\to 0$ as $n\to\infty$.
Since $|Q_n-K_n|_{\Haus\spc{X}}\le \tfrac1{10^n}$, we get $|Q_n-K_\infty|_{\Haus\spc{X}}\to 0$ as $n\to\infty$ --- hence the lemma.
\qeds

\begin{thm}{Exercise}\label{ex:closure-union}
Let $\spc{X}$ be a complete metric space and $K_1,K_2,\dots$ be a sequence of compact sets 
that converges in the sense of Hausdorff.
Show that the union $K_1\cup K_2\cup\dots$ is a compact closure.

Use this statement to show that in Lemma~\ref{lem:complete+Hausdorff} compactness of $\spc{X}$ can be exchanged to completeness.
\end{thm}

\parit{Proof of ``if'' part in \ref{thm:compact+Hausdorff}.}
According to Lemma~\ref{lem:complete+Hausdorff},
$\Haus\spc{X}$ is complete.
It remains to show that $\Haus\spc{X}$ is totally bounded (\ref{totally-bounded});
that is, given $\eps>0$ there is a finite $\eps$-net in $\Haus\spc{X}$.

Choose a finite $\eps$-net $A$ in $\spc{X}$.
Denote by $B$ the set of all subsets of $A$.
Note that  $B$ is a finite set in $\Haus\spc{X}$.
For each compact set $K\subset \spc{X}$, consider the subset $K'$ of all points $a\in A$
such that $\distfun_K(a)\le \eps$.
Observe that $K' \in B$ and $|K-K'|_{\Haus\spc{X}}\le\eps$.
In other words, $B$ is a finite $\eps$-net in $\Haus\spc{X}$.
\qeds

\begin{thm}{Exercise}\label{ex:Haus-length}
Let $\spc{X}$ be a complete metric space.
Show that $\spc{X}$ is a length space if and only if so is $\Haus\spc{X}$.
\end{thm}

\section{An application}

The following statement is called \index{isoperimetric inequality}\emph{isoperimetric inequality in the plane}.

\begin{thm}{Theorem}\label{thm:isoperimetric}
Among the plane figures bounded by closed curves of length at most $\ell$ the round disk has the maximal area.
\end{thm}

In this section, we will sketch a proof of the isoperimetric inequality that uses the Hausdorff convergence.
It is based on the following exercise.

\begin{thm}{Exercise}\label{ex:Huas-perimeter-area}
Let $\spc{C}$ be a subspace of $\Haus\RR^2$ formed by all compact convex subsets in $\RR^2$.
Show that perimeter\footnote{If the set degenerates to a line segment of length $\ell$, then its perimeter is defined as $2\cdot \ell$.} and area are continuous on~$\spc{C}$.
That is, if a sequence of convex compact plane sets $X_n$ converges to $X_\infty$ in the sense of Hausdorff, then 
\[\perim X_n\to \perim X_\infty\quad\text{and}\quad\area X_n\to\area X_\infty\]
as $n\to\infty$.
\end{thm}

\parit{Semiproof of \ref{thm:isoperimetric}.}
It is sufficient to consider only convex figures of the given perimeter; if a figure is not convex, pass to its convex hull and observe that it has a larger area and smaller perimeter.


Note that the selection theorem (\ref{thm:compact+Hausdorff}) together with the exercise imply the existence of figure $D$ with perimeter $\ell$ and maximal area.

It remains to show that $D$ is a round disk.
This is a problem in elementary geometry.

Let us cut $D$ along a chord $[ab]$ into two lenses, $L_1$ and $L_2$.
Denote by $L_1'$ the reflection of $L_1$ across the perpendicular bisector of $[ab]$.
Note that $D$ and $D'=L_1'\cup L_2$ have the same perimeter and area.
That is, $D'$ has perimeter $\ell$ and maximal possible area;
in particular, $D'$ is convex.

The following exercise will finish the proof.
\qeds

{

\begin{wrapfigure}{o}{57 mm}
\vskip-5mm
\centering
\includegraphics{mppics/pic-405}
\end{wrapfigure}

\begin{thm}{Exercise}\label{ex:round-disc}
Suppose $D$ is a convex figure such that for any chord $[ab]$ of $D$ the above construction produces a convex figure $D'$.
Show that $D$ is a round disk.
\end{thm}


}

Another popular way to prove that $D$ is a round disk is given by the so-called {}\emph{Steiner's 4-joint method} \cite{blaschke}.

\section{Remarks}\label{sec:H-variation}

It seems that Hausdorff convergence was first introduced by Felix Hausdorff~\cite{hausdorff}.
A couple of years later an equivalent definition was given by Wilhelm Blaschke~\cite{blaschke}.

The following refinement of the definition was introduced by  Zdeněk Frolík \cite{frolik},
later it was rediscovered by Robert Wijsman~\cite{wijsman}.  
This refinement is also called \index{Hausdorff convergence}\emph{Hausdorff convergence};
in fact, it takes an intermediate place between the original Hausdorff convergence and {}\emph{closed convergence}, also introduced by Hausdorff in \cite{hausdorff}.

\begin{thm}{Definition}\label{def:gen-Haus-conv}
Let $(A_n)$ be a sequence of closed sets in a metric space $\spc{X}$.
We say that $(A_n)$ converges to a closed set $A_\infty$ in the sense of Hausdorff if for any $x\in\spc{X}$, we have
$\distfun_{A_n}(x)\to \distfun_{A_\infty}(x)$ as $n\to\infty$.
\end{thm}

For example, suppose $\spc{X}$ is the Euclidean plane and $A_n$ is the circle with radius $n$ and center at the point $(n,0)$.
If we use the standard definition (\ref{def:hausdorff-convergence}), then the sequence $(A_n)$ diverges, but it converges to the $y$-axis in the sense of Definition~\ref{def:gen-Haus-conv}.

The following exercise is analogous to the Blaschke selection theorem (\ref{thm:compact+Hausdorff}) for the modified Hausdorff convergence.

\begin{thm}{Exercise}\label{ex:generalized-selection}
Let $\spc{X}$ be a proper metric space
and $(A_n)_{n=1}^\infty$ be a sequence of closed sets in~$\spc{X}$.
Assume that for some (and therefore any) point  $x\in\spc{X}$, 
the sequence $a_n=\distfun_{A_n}(x)$ is bounded.
Show that the sequence  $(A_n)_{n=1}^\infty$ has a convergent subsequence in the sense of Definition~\ref{def:gen-Haus-conv}.
\end{thm}

\chapter{Space of spaces}

\section{Gromov--Hausdorff metric}

The goal of this section is to cook up a metric space out of metric spaces.
More precisely, we want to define the so-called  Gromov--Hausdorff metric on the set of {}\emph{isometry classes} of compact metric spaces.
(Being isometric is an equivalence relation, 
and an isometry class is an equivalence class with respect to this equivalence relation.)

The obtained metric space will be denoted by $\GH$.
Given two metric spaces $\spc{X}$ and $\spc{Y}$,
denote by $[\spc{X}]$ and $[\spc{Y}]$ their isometry classes;
that is, $\spc{X}'\in [\spc{X}]$ if and only if $\spc{X}'\iso \spc{X}$.
Pedantically, the Gromov--Hausdorff distance from $[\spc{X}]$ 
to $[\spc{Y}]$ should be denoted as $|[\spc{X}]-[\spc{Y}]|_{\GH}$;
but we will write it as $|\spc{X}\z-\spc{Y}|_{\GH}$ and say (not quite correctly) 
``$|\spc{X}\z-\spc{Y}|_{\GH}$ is the Gromov--Hausdorff distance from  $\spc{X}$ 
to  $\spc{Y}$''.
In other words, from now on the term {}\emph{metric space} might also stand for its {}\emph{isometry class}.

The metric on $\GH$ is defined as the maximal metric such that {}\emph{the distance between subspaces in a metric space is not greater than the Hausdorff distance between them}.
Here is a formal definition:

\begin{thm}{Definition}\label{def:GH}
Let $\spc{X}$ and $\spc{Y}$ be compact metric spaces. 
The Gromov--Hausdorff distance $|\spc{X}-\spc{Y}|_{\GH}$ is defined by the following
relation.
 
Given  $r > 0$, we have that $|\spc{X}-\spc{Y}|_{\GH} < r$ if and only if there exist a metric
space $\spc{Z}$ and subspaces $\spc{X}'$ and $\spc{Y}'$ in $\spc{Z}$ that are isometric to $\spc{X}$ and $\spc{Y}$
respectively and such that $|\spc{X}'-\spc{Y}'|_{\Haus\spc{Z}} < r$. 
(Here $|\spc{X}'-\spc{Y}'|_{\Haus\spc{Z}}$ denotes the Hausdorff distance between sets $\spc{X}'$ and $\spc{Y}'$ in $\spc{Z}$.)
\end{thm}

Note that passing to the subspace $\spc{X}'\cup\spc{Y}'$ of $\spc{Z}$ does not affect the definition.
Therefore we can always assume that $\spc{Z}$ is compact.

\begin{thm}{Theorem}\label{thm:GH-is-a-metric}
The set of isometry classes of compact metric spaces equipped with Gromov--Hausdorff metric forms a metric space (which is denoted by $\GH$).

In other words, for arbitrary  compact metric spaces $\spc{X}$, $\spc{Y}$ and $\spc{Z}$ the following conditions hold:

\begin{subthm}{GH-1} $|\spc{X}-\spc{Y}|_{\GH}\ge 0$;
\end{subthm}

\begin{subthm}{GH-2} $|\spc{X}-\spc{Y}|_{\GH}=0$ if and only if $\spc{X}$ is isometric to $\spc{Y}$;
\end{subthm}

\begin{subthm}{GH-3} $|\spc{X}-\spc{Y}|_{\GH}=|\spc{Y}-\spc{X}|_{\GH}$;
\end{subthm}

\begin{subthm}{GH-4} $|\spc{X}-\spc{Y}|_{\GH}+|\spc{Y}-\spc{Z}|_{\GH}\ge |\spc{X}-\spc{Z}|_{\GH}$.
\end{subthm}
\end{thm}


Note that \ref{SHORT.GH-1}, \ref{SHORT.GH-3},
and the ``if''-part of \ref{SHORT.GH-2} follow directly from Definition \ref{def:GH}.
Part \ref{SHORT.GH-4} will be proved in Section~\ref{sec:GH-approx}.
The ``only-if''-part of \ref{SHORT.GH-2} will be proved in Section~\ref{sec:alm-isom}.

Recall that $a\cdot\spc{X}$ denotes $\spc{X}$ \index{scaled space}\emph{scaled} by factor $a>0$;
that is, $a\cdot\spc{X}$ is a metric space with the underlying set of $\spc{X}$ and the metric defined by
\[\dist{x}{y}{a\cdot\spc{X}}\df a\cdot\dist{x}{y}{\spc{X}}.\]

\begin{thm}{Exercise}\label{ex:d_GH-and-diam}
Let $\spc{X}$ be a compact metric space,
$\spc{P}$ be the one-point metric space.

Prove that 
\begin{subthm}{ex:d_GH-and-diam:point}
\[|\spc{X}-\spc{P}|_{\GH}=\tfrac12\cdot \diam \spc{X}.\]

\end{subthm}

\begin{subthm}{ex:d_GH-and-diam:scale}
\[|a\cdot\spc{X}-b\cdot \spc{X}|_{\GH}=\tfrac12\cdot|a-b|\cdot\diam\spc{X}.\]
\end{subthm}


\end{thm}

\begin{thm}{Exercise}\label{ex:rectangle}
Let $\spc{A}_r$ be a rectangle $1$ by $r$ in the Euclidean plane 
and $\spc{B}_r$ be a closed line interval of length $r$.
Show that 
\[|\spc{A}_r-\spc{B}_r|_{\GH}>\tfrac1{10}\]
for all large $r$.
\end{thm}

\begin{thm}{Advanced exercise}\label{ex:GH-inj}
Let $\spc{X}$ and $\spc{Y}$ be compact metric spaces;
denote by $\hat{\spc{X}}$ and $\hat{\spc{Y}}$ their injective envelopes (see the definition on page \pageref{page:InjX}).
Show that 
\[|\hat{\spc{X}}-\hat{\spc{Y}}|_{\GH}\le 2\cdot|\spc{X}- \spc{Y}|_{\GH}.\] 

\end{thm}

\section{Approximations}\label{sec:GH-approx}

\begin{thm}{Definition}\label{ex:defGHR}
Let $\spc{X}$ and $\spc{Y}$ be two metric spaces.
A relation $\approx$ between points in $\spc{X}$ and $\spc{Y}$ is called $\eps$-approximation if the following conditions hold:
\begin{itemize}
\item For any $x\in  \spc{X}$ there is $y\in \spc{Y}$ such that $x\approx y$.
\item For any $y\in  \spc{Y}$ there is $x\in \spc{X}$ such that $x\approx y$.
\item If for some $x, x'\in  \spc{X}$ and $y,y'\in \spc{Y}$ we have $x\approx y$ and $x'\approx y'$, then 
\[\bigl|\dist{x}{x'}{\spc{X}}-\dist{y}{y'}{\spc{Y}}\bigr|<2\cdot\eps.\]
\end{itemize}

\end{thm}

\begin{thm}{Exercise}\label{ex:H-R}
Let $\spc{X}$ and $\spc{Y}$ be two compact metric spaces.
Show that
\[\dist{\spc{X}}{\spc{Y}}{\GH}<\eps\]
if and only if there is an $\eps$-approximation between $\spc{X}$ and $\spc{Y}$.

In other words $\dist{\spc{X}}{\spc{Y}}{\GH}$ is the greatest lower bound of values $\eps>0$ such that  there is an $\eps$-approximation between $\spc{X}$ and $\spc{Y}$.
\end{thm}

\parit{Proof of \ref{GH-4}.}
Suppose that 
\begin{itemize}
\item $\approx_1$ is a relation between points in $\spc{X}$ and $\spc{Y}$,
\item $\approx_2$ is a relation between points in $\spc{Y}$ and $\spc{Z}$.
\end{itemize}
Consider the relation $\approx_3$ between points in $\spc{X}$ and $\spc{Z}$ such that
$x\approx_3 z$ if and only if there is $y\in  \spc{Y}$ such that 
$x\approx_1 y$ and $y\approx_2 z$.

It is straightforward to check that if $\approx_1$ is an $\eps_1$-approximation and $\approx_2$ is an $\eps_2$-approximation, then $\approx_3$ is an $(\eps_1+\eps_2)$-approximation.

Applying \ref{ex:H-R}, we get that if 
\[|\spc{X}-\spc{Y}|_{\GH}<\eps_1
\quad\text{and}\quad
|\spc{Y}-\spc{Z}|_{\GH}<\eps_2
\]
then 
\[|\spc{X}-\spc{Z}|_{\GH}<\eps_1+\eps_2.\]
Hence \ref{GH-4} follows.
\qeds


\section{Almost isometries}\label{sec:alm-isom}

\begin{thm}{Definition} Let $\spc{X}$ and $\spc{Y}$ be metric spaces and $\eps>0$. 
A  map\footnote{possibly noncontinuous} $f\: \spc{X} \z\to \spc{Y}$ is called an \index{almost isometry}\emph{$\eps$-isometry} 
if $f(\spc{X})$ is an $\eps$-net in $\spc{Y}$ and
\[\bigl|\dist{x}{x'}{\spc{X}}-\dist{f(x)}{f(x')}{\spc{Y}}\bigr|<\eps.\]
for any $x,x'\in \spc{X}$.
\end{thm}

\begin{thm}{Exercise}\label{ex:eps-isom}
Let $\spc{X}$ and $\spc{Y}$ be compact metric spaces.

\begin{subthm}{ex:eps-isom:GH>isom}
If $\dist{\spc{X}}{\spc{Y}}{\GH}<\eps$, then there is a $2\cdot\eps$-isometry $f\:\spc{X}\to\spc{Y}$.
\end{subthm}

\begin{subthm}{ex:eps-isom:isom>GH}
If there is an $\eps$-isometry $f\:\spc{X}\to\spc{Y}$, then $\dist{\spc{X}}{\spc{Y}}{\GH}<\eps$.
\end{subthm}

\end{thm}

\parit{Proof of the ``only if''-part in \ref{GH-2}.}
\label{page:GH-2-proof}
Let $\spc{X}$ and $\spc{Y}$ be compact metric spaces.
Suppose that $\dist{\spc{X}}{\spc{Y}}{\GH}<\eps$ for any $\eps>0$;
we need to show that there is an isometry $\spc{X}\to\spc{Y}$.

By \ref{ex:eps-isom:GH>isom}, for each positive integer $n$, we can choose a $\tfrac1n$-isometry $f_n\:\spc{X}\to\spc{Y}$.

Since $\spc{X}$ is compact, 
we can choose a countable dense set
$S$ in~$\spc{X}$.
Applying the diagonal procedure if necessary, we can assume that for every $x \in S$ the sequence $(f_n(x))$ 
converges in $\spc{Y}$. 
Consider the pointwise limit map  $f_\infty \: S \to \spc{Y}$,
 $$f_\infty(x) \df \lim_{n\to\infty} f_n (x)$$ for every $x \in S$. 
Since $$|f_n (x)- f_n (x')|_{\spc{Y}}\lg |x- x'|_\spc{X} \pm\tfrac1n,$$ 
we have 
$$|f_\infty(x)-f_\infty (x')|_{\spc{Y}} 
= \lim_{n\to\infty} |f_n(x)-f_n (x')|_{\spc{Y}} 
= |x -x'|_\spc{X}$$ for all
$x, x' \in S$; 
that is, the map $f_\infty\:S\to \spc{Y}$ is distance-preserving. 
Therefore, $f_\infty$ can be extended to a distance-preserving map from the whole $\spc{X}$ to $\spc{Y}$.

The latter can be done by setting 
$$f_\infty(x)=\lim_{n\to\infty} f_\infty(x_n)$$ 
for some sequence of points $(x_n)$ in $S$
that converges to $x$ in $\spc{X}$.
Indeed, if $x_n\to x$, then $(x_n)$ is Cauchy.
Since $f_\infty$ is distance-preserving, $y_n=f_\infty(x_n)$ is also a Cauchy sequence in $\spc{Y}$;
therefore it converges.
It remains to observe that this construction does not depend on the choice of the sequence $(x_n)$.

This way we obtain a distance-preserving map $f_\infty\:\spc{X}\to \spc{Y}$. 
It remains to show that $f_\infty$ is surjective; that is, $f_\infty(\spc{X})=\spc{Y}$.

The same argument produces a distance-preserving map $g_\infty\:\spc{Y}\z\to \spc{X}$.
If $f_\infty$ is not surjective, then neither is the composition $f_\infty\z\circ g_\infty\:\spc{Y}\to \spc{Y}$.
So $f_\infty \z\circ g_\infty$ is a distance-preserving map from a compact space to itself which is not an isometry.
The latter contradicts \ref{ex:non-contracting-map}. 
\qeds

\section{Convergence}

The Gromov--Hausdorff metric is used to define Gromov--Hausdorff convergence.
Namely, a sequence of compact metric spaces $\spc{X}_n$ converges to compact metric spaces $\spc{X}_\infty$ in the sense of Gromov--Hausdorff if 
\[\dist{\spc{X}_n}{\spc{X}_\infty}{\GH}\to 0\quad\text{as}\quad n\to\infty.\]

This convergence is more important than the metric ---
in all applications, we use only the topology on $\GH$
and we do not care about the particular value of Gromov--Hausdorff distance between spaces.
The following observation follows from \ref{ex:eps-isom}:

\begin{thm}{Observation}\label{obs:GH-e-isom}
A sequence of compact metric spaces $(\spc{X}_n)$ converges to  $\spc{X}_\infty$ in the sense of Gromov--Hausdorff if and only if there is a sequence $\eps_n\to0+$
and an $\eps_n$-isometry $f_n\:\spc{X}_n\to \spc{X}_\infty$ for each $n$.
\end{thm}

In the following exercises {}\emph{converge} means {}\emph{converge in the sense of Gromov--Hausdorff}.

\begin{thm}{Exercise}\label{ex:GH-SC}
\begin{subthm}{ex:GH-SC:circle}
Show that a sequence of compact simply connected length spaces cannot converge to a circle.
\end{subthm}

\begin{subthm}{ex:GH-SC:nonsc-limit}
Construct a sequence of compact simply connected length spaces that converges to a compact non-simply connected space.
\end{subthm}
\end{thm}

\begin{thm}{Exercise}\label{ex:sphere-to-ball}
\begin{subthm}{ex:sphere-to-ball:2}
Show that a sequence of length metrics on the 2-sphere cannot converge to the unit disk.
\end{subthm}

\begin{subthm}{ex:sphere-to-ball:3}
Construct a sequence of length metrics on the 3-sphere that converges to a unit 3-ball.
\end{subthm}

\end{thm}

Given two metric spaces $\spc{X}$ and $\spc{Y}$, we will write $\spc{X}\le \spc{Y}$ if there is a noncontracting map $f\:\spc{X}\to \spc{Y}$;
that is, if 
$$ |x-x'|_{\spc{X}}\le|f(x)-f(x')|_{\spc{Y}}$$
for any $x,x'\in \spc{X}$.

Further, given $\eps>0$, we will write $\spc{X}\le \spc{Y}+\eps$
if there is a map $f\:\spc{X}\to \spc{Y}$ such that 
$$|x-x'|_{\spc{X}}\le|f(x)-f(x')|_{\spc{Y}}+\eps$$
for any $x,x'\in \spc{X}$.

\section{Uniformly totally bonded families}

\begin{thm}{Definition}\label{def:utb}
A family $\spc{Q}$ of (isometry classes) of compact metric spaces is called  \index{uniformly totally bonded family}\emph{uniformly totally bonded} if it meets the following two conditions:

\begin{subthm}{}
spaces in $\spc{Q}$ have uniformly bounded diameters; that is, there is $D\in\RR$ such that
\[\diam\spc{X}\le D\]
for any space $\spc{X}$ in $\spc{Q}$.
\end{subthm}

\begin{subthm}{}
For any $\eps>0$ there is $n\in\NN$ such that any space $\spc{X}$ in $\spc{Q}$ admits an $\eps$-net with at most $n$ points.
\end{subthm}
\end{thm}

\begin{thm}{Exercise}\label{ex:utb+pack}
Let $\spc{Q}$ be a family of compact spaces with uniformly bounded diameters.
Show that $\spc{Q}$ is uniformly totally bonded if for any $\eps>0$ there is $n\in\NN$ such that 
\[\pack_\eps\spc{X}\le n\]
for any space $\spc{X}$ in $\spc{Q}$.
\end{thm}


Fix a real constant $C$.
A Borel measure $\mu$ on a metric space $\spc{X}$ is called \index{doubling space}\emph{$C$-doubling} if
\[\mu[\oBall(p,2\cdot r)]< C\cdot\mu[\oBall(p,r)]\]
for any point $p\in \spc{X}$ and any $r>0$.
A Borel measure is called \index{doubling measure}\emph{doubling} if it is {}\emph{$C$-doubling} for some real constant $C$.

\begin{thm}{Exercise}\label{pr:doubling}
Let $\spc{Q}(C,D)$ be the set of all the compact metric spaces with diameter at most $D$ that admit a $C$-doubling measure.
Show that $\spc{Q}(C,D)$ is totally bounded.
\end{thm}

Recall that we write $\spc{X}\le\spc{Y}$ if there is a distance-nondecreasing map $\spc{X}\to\spc{Y}$.

\begin{thm}{Exercise}\label{pr:under}

\begin{subthm}{pr:under:if}
Let $\spc{Y}$ be a compact metric space.
Show that the set of all spaces $\spc{X}$ such that $\spc{X}\le\spc{Y}$
is uniformly totally bounded.
\end{subthm}

\begin{subthm}{pr:under:only-if}
Show that for any uniformly totally bounded set $\spc{Q}\subset\GH$ there is a compact space $\spc{Y}$
such that $\spc{X}\le\spc{Y}$ for any $\spc{X}$ in $\spc{Q}$.
\end{subthm}

\end{thm}

\section{Gromov's selection theorem}

The following theorem is analogous to Blaschke selection theorems (\ref{thm:compact+Hausdorff}).

\begin{thm}{Gromov selection theorem}\label{thm:gromov-compactness}
Let $\spc{Q}$ be a closed subset of $\GH$.
Then $\spc{Q}$ is compact if and only if it is totally bounded.
\end{thm}

\begin{thm}{Lemma}\label{lem:GH-complete}
The space $\GH$ is complete.
\end{thm}


Let us define gluing of metric spaces that will be used in the proof of the lemma.

Suppose 
$\spc{U}$ and $\spc{V}$ are metric spaces 
with isometric closed sets $A\subset\spc{U}$ and $A'\subset\spc{V}$;
let $\iota\:A\to A'$ be an isometry.
Consider the space $\spc{W}$ of all equivalence classes in $\spc{U}\sqcup\spc{V}$ with the equivalence relation given by $a\sim\iota(a)$ for any $a\in A$.

It is straightforward to check that the following defines a metric on~$\spc{W}$:
\begin{align*}
\dist{u}{u'}{\spc{W}}&\df\dist{u}{u'}{\spc{U}}
\\
\dist{v}{v'}{\spc{W}}&\df\dist{v}{v'}{\spc{V}}
\\
\dist{u}{v}{\spc{W}}&\df\min\set{\dist{u}{a}{\spc{U}}+\dist{v}{\iota(a)}{\spc{V}}}{a\in A}
\end{align*}
where $u,u'\in \spc{U}$ and $v,v'\in \spc{V}$.

The  space $\spc{W}$ is called the \index{gluing}\emph{gluing} of $\spc{U}$ and  $\spc{V}$ along~$\iota$; briefly, we can write
$\spc{W}=\spc{U}\sqcup_\iota\spc{V}$.
If one applies this construction to two copies of one space $\spc{U}$ with a set $A\subset \spc{U}$ and the identity map $\iota\:A\to A$, then the obtained space is called the \index{double}\emph{double} of $\spc{U}$ along~$A$; this space can be denoted by $\sqcup_A^2\spc{U}$.

Note that the inclusions $\spc{U}\hookrightarrow \spc{W}$ and $\spc{V}\hookrightarrow \spc{W}$ are distance preserving.
Therefore we can and will conside $\spc{U}$ and $\spc{V}$ as the subspaces of $\spc{W}$;
this way the subsets $A$ and $A'$ will be identified and denoted further by~$A$.
Note that $A=\spc{U}\cap \spc{V}\subset \spc{W}$.

\parit{Proof.}
Let $(\spc{X}_n)$ be a Cauchy sequence in $\GH$.
Passing to a subsequence if necessary, 
we can assume that $|\spc{X}_n-\spc{X}_{n+1}|_{\GH}<\tfrac1{2^n}$ for each~$n$.
In particular, for each $n$ there is a metric space $\spc{V}_n$ with distance preserving inclusions $\spc{X}_n\hookrightarrow \spc{V}_n$ and $\spc{X}_{n+1}\hookrightarrow \spc{V}_n$ such that
\[|\spc{X}_n-\spc{X}_{n+1}|_{\Haus\spc{V}_n}<\tfrac1{2^n}\]
for each $n$.
Moreover, we may assume that $\spc{V}_n=\spc{X}_n\cup\spc{X}_{n+1}$.

Let us glue $\spc{V}_1$ to $\spc{V}_2$ along $\spc{X}_2$;
to the obtained space glue $\spc{V}_3$ along $\spc{X}_3$, and so on.
The obtained metric space $\spc{W}$
has an underlying set formed by the disjoint union of all $\spc{X}_n$ such that each inclusion $\spc{X}_n\z\hookrightarrow\spc{W}$ is distance preserving and
\[|\spc{X}_n-\spc{X}_{n+1}|_{\Haus\spc{W}}<\tfrac1{2^n}\]
for each $n$.
In particular,
\[|\spc{X}_m-\spc{X}_n|_{\Haus\spc{W}}<\tfrac1{2^{n-1}}\eqlbl{eq:|x_m-X_n|}\] 
if $m>n$.

Denote by $\bar{\spc{W}}$ the completion of $\spc{W}$.
Observe that the union $\spc{X}_1\z\cup \spc{X}_2\cup\z\dots\cup \spc{X}_n$ is compact and \ref{eq:|x_m-X_n|} implies that it forms a $\tfrac1{2^{n-1}}$-net in $\bar{\spc{W}}$.
Whence $\bar{\spc{W}}$ is compact; see \ref{totally-bounded} and \ref{ex:compact-net}.

Applying Blaschke selection theorem (\ref{thm:compact+Hausdorff}),
we can pass to a subsequence of $(\spc{X}_n)$ that converges in $\Haus\bar{\spc{W}}$; denote its limit by $\spc{X}_\infty$.
It remains to observe that $\spc{X}_\infty$ is the Gromov--Hausdorff limit of $(\spc{X}_n)$.
\qeds

\parit{Proof of \ref{thm:gromov-compactness}; ``only if'' part.}
Suppose that there is no sequence $\eps_n\to0$ as described in \ref{def:utb}.
Observe that in this case
there is a sequence of spaces $\spc{X}_n\in\spc{Q}$ such that 
$$\pack_\delta \spc{X}_n\to\infty
\quad\text{as}\quad
n\to\infty$$
for some fixed $\delta>0$.

Since $\spc{Q}$ is compact, 
this sequence has a partial limit, say $\spc{X}_\infty\in\spc{Q}$.
Observe that $\pack_{\delta} \spc{X}_\infty=\infty$.
Therefore, $\spc{X}_\infty$ is not compact --- a contradiction.

\parit{``If'' part.}
Suppose sequence $(\eps_n)$ as in the definition of uniformly totally bonded families (\ref{def:utb}).

Note that $\diam \spc{X}\le \eps_1$ for any $\spc{X}\in \spc{Q}$.
Given a positive integer $n$ consider the set of all metric spaces $\spc{W}_n$
with the number of points at most $n$ and diameter $\le \eps_1$.
Note that $\spc{W}_n$ is a compact set in $\GH$ for each $n$.

Further, a subspace formed by a maximal $\eps_n$-net of any $\spc{X}\in\spc{Q}$ belongs to $\spc{W}_n$.
Therefore, $\spc{W}_n\cap\spc{Q}$ is a compact $\eps_n$-net in  $\spc{Q}$.
That is, $\spc{Q}$ has a compact $\eps$-net for any $\eps>0$.
Since $\spc{Q}$ is closed in a complete space $\GH$, it implies that $\spc{Q}$ is compact.
\qeds

\begin{thm}{Exercise}\label{ex-GH-length}
Show that the space $\GH$ is 

\begin{subthm}{ex-GH-length:length}
length,
\end{subthm}

\begin{subthm}{ex-GH-length:geodesic}
geodesic.
\end{subthm}

\end{thm}

\begin{thm}{Exercise}\label{ex:GH-po}
\begin{subthm}{ex:GH-po:a}
Show that 
$$\dist{\spc{X}}{\spc{Y}}{\GH'}=\inf\set{\eps>0}{\spc{X}\le \spc{Y}+\eps
\quad\text{and}\quad
\spc{Y}\le \spc{X}+\eps}$$
defines a metric on the space of (isometry classes) of compact metric spaces.
\end{subthm}

\begin{subthm}{ex:GH-po:b}
Moreover $\dist{*}{*}{\GH'}$ is equivalent to the Gromov--Hausdorff metric;
that is,
$$|\spc{X}_n-\spc{X}_\infty|_{\GH}\to 0 
\quad\iff\quad 
\dist{\spc{X}_n}{\spc{X}_\infty}{\GH'}\to 0$$ 
as $n\to\infty$.
\end{subthm}
\end{thm}

\section{Universal ambient space}

Recall that a metric space is called universal if it contains an isometric copy of any separable metric space (in particular, any compact metric space).
Examples of universal spaces include Urysohn space and $\ell^\infty$ --- the space of bounded infinite sequences with the metric defined by $\sup$-norm; see \ref{prop:sep-in-urys} and \ref{ex:frechet}.

The following proposition says that the space $\spc{W}$ in Definition~\ref{def:GH} can be exchanged to a fixed universal space.

\begin{thm}{Proposition}\label{prop:GH-with-fixed-Z}
Let $\spc{U}$ be a universal space.
Then for any compact metric spaces $\spc{X}$ and $\spc{Y}$ we have
$$|\spc{X}-\spc{Y}|_{\GH} = \inf \{|\spc{X}'-\spc{Y}'|_{\Haus\spc{U}}\}$$ 
where the greatest lower bound is taken over all pairs of sets $\spc{X}'$ and $\spc{Y}'$ in $\spc{U}$
which isometric to  $\spc{X}$ and $\spc{Y}$ respectively.  
\end{thm}




\parit{Proof of \ref{prop:GH-with-fixed-Z}.}
By the definition (\ref{def:GH}), we have that 
\[|\spc{X}-\spc{Y}|_{\GH} \le \inf \{|\spc{X}'-\spc{Y}'|_{\Haus\spc{U}}\};\]
it remains to prove the opposite inequality.

Suppse $|\spc{X}-\spc{Y}|_{\GH}<\eps$;
let $\spc{X}'$, $\spc{Y}'$ and $\spc{Z}$ be as in \ref{def:GH}.
We can assume that $\spc{Z}=\spc{X}'\cup\spc{Y}'$;
otherwise pass to the subspace $\spc{X}'\cup\spc{Y}'$ of~$\spc{Z}$.
In this case, $\spc{Z}$ is compact;
in particular, it is separable.

Since $\spc{U}$ is universal, there is a distance-preserving embedding of $\spc{Z}$ in $\spc{U}$;
let us keep the same notation for $\spc{X}'$, $\spc{Y}'$, and their images.
It follows that 
\[|\spc{X}'-\spc{Y}'|_{\Haus\spc{U}}<\eps,\]
--- hence the result.
\qeds

\begin{thm}{Exercise}\label{ex:GH-urysohn}
Let $\spc{U}_\infty$ be the Urysohn space.
Given two compact set $A$ and $B$ in $\spc{U}_\infty$ define 
\[\|A-B\|=\inf\{|A-\iota(B)|_{\Haus\spc{U}_\infty}\},\]
where the greatest lower bound is taken for all isometrics $\iota$ of $\spc{U}_\infty$.
Show that $\|{*}\z-{*}\|$ defines a pseudometric%
\footnote{The value $\|A-B\|$ is called \emph{Hausdorff distance up to isometry} from $A$ to $B$ in $\spc{U}_\infty$.}
on nonempty compact subsets of $\spc{U}_\infty$ and its corresponding metric space is isometric to $\GH$.
\end{thm}

\section{Remarks}

Suppose $\spc{X}_n\GHto \spc{X}_\infty$, then there is a metric on the disjoint union 
\[\bm{X}=\bigsqcup_{n\in \NN\cup\{\infty\}} \spc{X}_n\] 
that satisfies the following property:

\begin{thm}{Property}\label{propery:GH}
The restriction of metric on each $\spc{X}_n$ and $\spc{X}_\infty$ coincides with its original metric 
and $\spc{X}_n\Hto \spc{X}_\infty$ as subsets in $\bm{X}$.
\end{thm}


Indeed, since $\spc{X}_n\GHto \spc{X}_\infty$, there is a metric on $\spc{V}_n=\spc{X}_n\sqcup \spc{X}_\infty$ such that the restriction of metric on each $\spc{X}_n$ and $\spc{X}_\infty$ coincides with its original metric and $\dist{\spc{X}_n}{\spc{X}_\infty}{\Haus\spc{V}_n}<\eps_n$ for some sequence $\eps_n\to 0$.
Gluing all $\spc{V}_n$ along $\spc{X}_\infty$, we obtain the required space $\bm{X}$.

In other words, the metric on $\bm{X}$ defines the convergence $\spc{X}_n\z\GHto \spc{X}_\infty$.
This metric makes it possible to talk about limits of sequences $x_n\in \spc{X}_n$ as $n\to\infty$, as well as weak limits of a sequence of Borel measures $\mu_n$ on $\spc{X}_n$ and so on.

For that reason, it is useful to define convergence by specifying the metric on $\bm{X}$ that satisfies the property
for the variation of Hausdorff convergence described in Section~\ref{sec:H-variation}.
This approach is very flexible;
in particular, it can be used to define Gromov--Hausdorff convergence of arbitrary metric spaces (net necessarily compact).

In this case, a limit space for this generalized convergence is not uniquely defined.
\begin{figure}[h!]
\vskip-0mm
\centering
\includegraphics{mppics/pic-500}
\end{figure}
For example, if each space $\spc{X}_n$ in the sequence is isometric to the half-line, then its limit might be isometric to the half-line or the whole line.
The first convergence is evident and the second could be guessed from the diagram.



Often the isometry class of the limit can be fixed by marking a point $p_n$ in each space $\spc{X}_n$, it is called \index{pointed convergence}\emph{pointed Gromov--Hausdorff convergence} --- we say that $(\spc{X}_n,p_n)$ converges to $(\spc{X}_\infty,p_\infty)$ if there is a metric on $\bm{X}$ such that $\spc{X}_n\Hto \spc{X}_\infty$ and $p_n\to p_\infty$.
For example, the sequence $(\spc{X}_n,p_n)=(\RR_+,0)$ converges to $(\RR_+,0)$, while $(\spc{X}_n,p_n)=(\RR_+,n)$ converges to $(\RR,0)$.

The pointed convergence works nicely only for proper metric spaces;
the following theorem is an analog of Gromov's selection theorem for this convergence.

\begin{thm}{Theorem}\label{thm:pointed-gromov-compactness}%
Let $\spc{Q}$ be a set of isometry classes of pointed proper metric spaces
$(\spc{X},p)$.
Assume that for any $R>0$, the $R$-balls in the spaces centered at the marked points form a uniformly totally bounded family of spaces.
Then $\spc{Q}$ is precompact with respect to pointed Gromov--Hausdorff convergence. 
\end{thm}






