\chapter{Space of subsets}\label{chap:hausdorff}

In this lecture we define and study Hausdorff metric on subsets of a given metric space.

\section{Hausdorff distance}

Let $\spc{X}$ be a metric space.
Given a subset $A\subset \spc{X}$,
consider the distance function to $A$
$$\distfun_A: \spc{X} \to [0,\infty)$$
defined as 
$$\distfun_A(x)
\df
\inf\set{\dist ax{\spc{X}}}{a\in A}.$$

Further, we define the so-called Hausdorff metric on all nonempty compact subsets of a given metric space $\spc{X}$;
The obtained metric space will be denoted as $\Haus\spc{X}$.

\begin{thm}{Definition}\label{def:hausdorff-convergence}
Let $A$ and $B$ be two nonempty compact subsets of a metric space $\spc{X}$.
Then the \index{Hausdorff distance}\emph{Hausdorff distance} between $A$ and $B$ is defined as 
$$|A-B|_{\Haus\spc{X}}
\df
\sup_{x\in \spc{X}}\{\,|\distfun_A(x)-\distfun_B(x)|\,\}.
$$

\end{thm}

The following observation gives a useful reformulation of the definition:

\begin{thm}{Observation}\label{obs:Haus-nbhds}
Suppose $A$ and $B$ be two compact subsets of a metric space $\spc{X}$.
Then $|A-B|_{\Haus\spc{X}}< R$ if and only if and only if 
$B$ lies in an $R$-neighborhood of $A$, 
and 
$A$ lies in an $R$-neighborhood of~$B$.
\end{thm}

\begin{thm}{Exercise}\label{ex:diam}
Let $\spc{X}$ be a metric space.
Given a subset $A\subset \spc{X}$ define its \index{diameter}\emph{diameter} as 
$$\diam A\df\sup_{a,b\in A} |a-b|.$$

Show that 
$$\diam\:\Haus\spc{X}\to \RR$$ 
is a $2$-Lipschitz function;
that is,
\[|\diam A-\diam B|\le 2\cdot\dist{A}{B}{\Haus\spc{X}}\]
for any two compact nonempty sets $A,B\subset\spc{X}$.
\end{thm}


\begin{thm}{Exercise}\label{ex:Hausdorff-bry}
Let $A$ and $B$ be two compact subsets in the Euclidean plane $\RR^2$.
Assume $|A-B|_{\Haus\RR^2}<\eps$.

\begin{subthm}{ex:Hausdorff-bry:conv}
Show that $|\Conv A-\Conv B|_{\Haus\RR^2}<\eps$, where $\Conv A$ denoted the convex hull of $A$.
\end{subthm}
\begin{subthm}{ex:Hausdorff-bry:bry}
Is it true that
$|\partial A-\partial B|_{\Haus\RR^2}<\eps$,
where $\partial A$ denotes the boundary of $A$.

Does the converse hold? That is, assume $A$ and $B$ be two compact subsets in $\RR^2$
and $|\partial A-\partial B|_{\Haus\RR^2}<\eps$; 
is it true that $|A-B|_{\Haus\RR^2}\z<\eps$?
\end{subthm}

\end{thm}

Note that part \ref{SHORT.ex:Hausdorff-bry:conv} implies that $A\mapsto \Conv A$ defines a short map $\Haus\RR^2\to \Haus\RR^2$. 

\begin{thm}{Exercise}\label{ex:Haus-func}
Let $A$ and $B$ be compact subsets in metric space~$\spc{X}$.
Show that 
\[\dist{A}{B}{\Haus\spc{X}}=\sup_f\, \{\,\max_{a\in A}\{f(a)\}-\max_{b\in B}\{f(b)\,\},\]
where the least upper bound is taken for all $1$-Lipschitz functions $f$.

\end{thm}

Given a subset $A\subset \RR^n$,
The \index{support function}\emph{support function} $h_A\colon\mathbb{R}^n\to\mathbb{R}$ of a  nonempty closed set $A\subset \RR^n$ is defined as 
\[h_A(x)
\df
\sup\set{\langle x, a\rangle}{a\in A}.\]

\begin{thm}{Exercise}\label{ex:Haus-support}
Show that 
\[\dist{A}{B}{\Haus\RR^n}\ge \sup_{|u|=1}\{|h_A(u)-h_B(u)|\}\]
for any nonempty compact subsets $A,B\subset \RR^n$.

Moreover, equality holds if both $A$ and $B$ are convex.
\end{thm}


\begin{thm}{Advanced exercise}\label{ex:H-sections}
Suppose $C_t\subset \spc{X}$, $t\z\in [0,1]$ is a family of subsets.
A path $c\:[0,1]\to \spc{X}$ such that $c(t)\in C_t$ for all $t$ will be called a {}\emph{section} of $C_t$.

\begin{subthm}{ex:H-sections:S}
Construct a family of nonempty compact sets $C_t\subset\mathbb{S}^1$, $t\z\in [0,1]$ that is continuous in the Hausdorff topology, 
but does not admit a section.
\end{subthm}

\begin{subthm}{ex:H-sections:R}
Show that any family of nonempty compact sets $C_t\subset\RR$, $t\z\in [0,1]$ that is continuous in the Hausdorff topology, 
admits a section.
\end{subthm}

\end{thm}

\section{Hausdorff convergence}

\begin{thm}{Blaschke selection theorem}\label{thm:compact+Hausdorff}
A metric space $\spc{X}$ is compact if and only if
so is $\Haus\spc{X}$.
\end{thm}

The Hausdorff metric can be used to define convergence.
Namely, suppose $K_1,K_2,\dots$, and $K_\infty$ are compact sets in a metric space $\spc{X}$.
If $|K_\infty-K_n|_{\Haus\spc{X}}\to0$ as $n\to\infty$, then we say that 
the sequence $K_n$ {}\emph{converges} to $K_\infty$ \index{Hausdorff convergence}\emph{in the sense of Hausdorff};
equivalently, $K_\infty$ is the \index{Hausdorff limit}\emph{Hausdorff limit} of the sequence $K_n$.

Note that the theorem implies that from any sequence of nonempty compact sets in $\spc{X}$ one can select a convergent subsequence; 
for that reason, it is called a \textit{selection} theorem. 

\parit{Proof; if part.}
Consider the map $\iota$ that sends each point $x\in \spc{X}$ to the one-point subset $\{x\}$ of $\spc{X}$.
Note that $\iota\:\spc{X}\to \Haus\spc{X}$ is distance-preserving.

Suppose that $A\subset \spc{X}$.
Note that $\diam A=0$ if and only if $A$ is a one-point set.
By \ref{ex:diam}, $\iota(\spc{X})$ is a closed subset of the compact space $\Haus\spc{X}$.
It follows that $\iota(\spc{X})$, and therefore $\spc{X}$, are compact.
\qeds

Since the map $\iota$ above is distance-preserving, we can and will consider $\spc{X}$ as a subspace of $\Haus\spc{X}$.

\begin{thm}{Exercise}\label{ex:haus-contractible}
Let $\spc{X}$ be a compact length space.
Suppose that there is a short retraction $\Haus\spc{X}\to \spc{X}$.
Show that $\spc{X}$ is contractible.
\end{thm}


To prove the only-if part we will need the following two lemmas.

\begin{thm}{Monotone convergence}\label{lem:decreasing-converges}
Let $K_1\supset K_2\supset\dots$ be a nested sequence of nonempty compact sets in a metric space $\spc{X}$.
Then $K_\infty\z=\bigcap_n K_n$ is the Hausdorff limit of $K_n$;
that is, $|K_\infty-K_n|_{\Haus\spc{X}}\to0$ as $n\to\infty$.
\end{thm}

\parit{Proof.}
By finite intersection property, $K_\infty$ is a nonempty compact set.

Arguing by contradiction, assume that there is $\eps>0$ such that for each $n$ 
one can choose $x_n\in K_n$
such that $\distfun_{K_\infty}(x_n)\ge\eps$.
Note that $x_n\in K_1$ for each $n$.
Since $K_1$ is compact, 
there is 
a \index{partial limit}\emph{partial limit}
 $x_\infty$ of $x_n$;
that is, a limit of a subsequence.
Clearly, $\distfun_{K_\infty}(x_\infty)\ge \eps$.

On the other hand, since $K_n$ is closed and $x_m\in K_n$ for $m\ge n$,
we get $x_\infty\in K_n$ for each $n$.
It follows that $x_\infty\in K_\infty$ and therefore $\distfun_{K_\infty}(x_\infty)=0$ ---
a contradiction.\qeds


\begin{thm}{Lemma}\label{lem:complete+Hausdorff}
If $\spc{X}$ is a compact metric space, then $\Haus\spc{X}$
is complete.
\end{thm}

\parit{Proof.}
Let $Q_1,Q_2,\dots$ be a Cauchy sequence in $\Haus\spc{X}$.
Passing to a subsequence, we may assume that 
$$|Q_n-Q_{n+1}|_{\Haus\spc{X}}\le \tfrac1{10^n}\eqlbl{eq:eps=1/10}$$
for each $n$.

Denote by $K_n$ the closed $\tfrac2{10^n}$-neighborhood of $Q_n$;
that is,
\begin{align*}
K_n&= \set{x\in \spc{X}}{\distfun_{Q_n}(x)\le \tfrac2{10^n}}
\end{align*}
Since $\spc{X}$ is compact so is each $K_n$.

From \ref{eq:eps=1/10}, we get
$K_n\supset K_{n+1}$ 
for each $n$.
Set 
$$K_\infty=\bigcap_{n=1}^\infty K_n.$$
By the monotone convergence (\ref{lem:decreasing-converges}),
 $|K_n-K_\infty|_{\Haus\spc{X}}\to 0$ as $n\to\infty$.

By \ref{obs:Haus-nbhds}, $|Q_n-K_n|_{\Haus\spc{X}}\le \tfrac2{10^n}$.
Therefore, $|Q_n-K_\infty|_{\Haus\spc{X}}\to 0$ as $n\to\infty$ --- hence the lemma.
\qeds

\begin{thm}{Exercise}\label{ex:closure-union}
Let $\spc{X}$ be a complete metric space and $K_1,K_2,\dots$ be a sequence of compact sets 
that converges in the sense of Hausdorff.
Show that the union $K_1\cup K_2\cup\dots$ has compact closure.

Use this statement to show that in Lemma~\ref{lem:complete+Hausdorff} compactness of $\spc{X}$ can be exchanged to completeness.
\end{thm}

\parit{Proof of only-if part in \ref{thm:compact+Hausdorff}.}
According to Lemma~\ref{lem:complete+Hausdorff},
$\Haus\spc{X}$ is complete.
It remains to show that $\Haus\spc{X}$ is totally bounded (\ref{totally-bounded});
that is, given $\eps>0$ there is a finite $\eps$-net in $\Haus\spc{X}$.

Choose a finite $\eps$-net $A$ in $\spc{X}$.
Denote by $B$ the set of all nonempty subsets of $A$.
Note that  $B$ is a finite set in $\Haus\spc{X}$.
For each compact set $K\subset \spc{X}$, consider the subset $K'$ of all points $a\in A$
such that $\distfun_K(a)\le \eps$.
Observe that $K' \in B$ and $|K-K'|_{\Haus\spc{X}}\le\eps$.
In other words, $B$ is a finite $\eps$-net in $\Haus\spc{X}$.
\qeds

\begin{thm}{Exercise}\label{ex:Haus-length}
Let $\spc{X}$ be a complete metric space.
Show that $\spc{X}$ is a length space if and only if so is $\Haus\spc{X}$.
\end{thm}

\section{An application}


In this section, we will sketch a proof of the isoperimetric inequality in the plane that uses the Hausdorff convergence.

It is based on the following exercise.

\begin{thm}{Exercise}\label{ex:Huas-perimeter-area}
Let $\spc{C}$ be the set of all nonempty compact convex subsets in $\RR^2$.
Show that $\spc{C}$ is a closed subset of $\Haus\RR^2$ and 
perimeter and area are continuous on~$\spc{C}$.
(If the set degenerates to a line segment of length $\ell$, then its perimeter is defined as $2\cdot \ell$.)

More precisely, if a sequence of convex compact plane sets $X_n$ converges to $X_\infty$ in the sense of Hausdorff, then $X_\infty$ is convex,
\[\perim X_n\to \perim X_\infty\quad\text{and}\quad\area X_n\to\area X_\infty\]
as $n\to\infty$.
\end{thm}


\begin{thm}{Isoperimetric inequality}\label{thm:isoperimetric}
Among the plane figures bounded by closed curves of length at most $\ell$, the round disk has the maximal area.
\end{thm}

\parit{Sketch.}
It is sufficient to consider only convex figures of the given perimeter; if a figure is not convex, pass to its convex hull and observe that it has a larger area and smaller perimeter.


Note that the selection theorem (\ref{thm:compact+Hausdorff}) together with the exercise imply the existence of figure $D$ with perimeter $\ell$ and maximal area.

It remains to show that $D$ is a round disk;
it will be done by means of elementary geometry.

Let us cut $D$ along a chord $[ab]$ into two lenses, $L_1$ and $L_2$.
Denote by $L_1'$ the reflection of $L_1$ across the perpendicular bisector of $[ab]$.
Note that $D$ and $D'=L_1'\cup L_2$ have the same perimeter and area.
That is, $D'$ has perimeter $\ell$ and maximal possible area;
in particular, $D'$ is convex.

The following exercise will finish the proof.
\qeds

{

\begin{wrapfigure}{o}{57 mm}
\vskip-5mm
\centering
\includegraphics{mppics/pic-405}
\end{wrapfigure}

\begin{thm}{Exercise}\label{ex:round-disc}
Suppose $D$ is a convex figure such that for any chord $[ab]$ of $D$ the above construction produces a convex figure $D'$.
Show that $D$ is a round disk.
\end{thm}

}

Another popular way to prove that $D$ is a round disk is given by the so-called \textit{Steiner's 4-joint method} \cite{blaschke}.

\section{Remarks}\label{sec:H-variation}

It seems that Hausdorff convergence was first introduced by Felix Hausdorff~\cite{hausdorff}.
A couple of years later an equivalent definition was given by Wilhelm Blaschke~\cite{blaschke}.

The following refinement was introduced by  Zdeněk Frolík \cite{frolik} and rediscovered by Robert Wijsman~\cite{wijsman}.  
This refinement is also called \index{Hausdorff convergence}\emph{Hausdorff convergence};
in fact, it takes an intermediate place between the original Hausdorff convergence and the so-called \textit{closed convergence}, also introduced by Hausdorff in \cite{hausdorff}.

\begin{thm}{Definition}\label{def:gen-Haus-conv}
Let $A_1,A_2,\dots$ be a sequence of closed sets in a metric space $\spc{X}$.
We say that the sequence $A_n$ converges to a closed set $A_\infty$ in the sense of Hausdorff if, for any $x\in\spc{X}$, we have
$\distfun_{A_n}(x)\z\to \distfun_{A_\infty}(x)$ as $n\to\infty$.
\end{thm}

For example, suppose $\spc{X}$ is the Euclidean plane and $A_n$ is the circle with radius $n$ and center at the point $(0,n)$.
If we use the standard definition (\ref{def:hausdorff-convergence}), then the sequence $A_1,A_2,\dots$ diverges, but it converges to the $x$-axis in the sense of Definition~\ref{def:gen-Haus-conv}.

\begin{figure}[ht!]
\vskip-0mm
\centering
\includegraphics{mppics/pic-415}
\end{figure}

Further, consider the sequence of one-point sets $B_n=\{(n,0)\}$ in the Euclidean plane.
It diverges in the sense of the standard definition, but, in the sense of \ref{def:gen-Haus-conv}, it converges to the empty set;
indeed, for any point $x$ we have $\distfun_{B_n}(x)\to\infty$ as $n\to \infty$ and $\distfun_{\emptyset}(x)= \infty$ for any~$x$.

The following exercise is analogous to the Blaschke selection theorem (\ref{thm:compact+Hausdorff}) for the modified Hausdorff convergence.

\begin{thm}{Exercise}\label{ex:generalized-selection}
Let $\spc{X}$ be a proper metric space
and $A_1,A_2,\dots$ be a sequence of closed sets in~$\spc{X}$.
Show that the sequence  $A_1,A_2,\dots$ has a convergent subsequence in the sense of Definition~\ref{def:gen-Haus-conv}.
\end{thm}

\chapter{Space of spaces}\label{chap:GH}

In this lecture we define and study the so-called Gromov--Hausdorff metric on the isometry classes of compact metric spaces.

\section{Gromov--Hausdorff metric}

The goal of this section is to cook up a metric space out of all compact metric spaces.
More precisely, we want to define the so-called  Gromov--Hausdorff metric on the set of \textit{isometry classes} of compact metric spaces.
(Being isometric is an equivalence relation, 
and an \index{isometry class}\emph{isometry class} is an equivalence class with respect to this relation.)

The obtained metric space will be denoted by $\GH$.
Given two metric spaces $\spc{X}$ and $\spc{Y}$,
denote by $[\spc{X}]$ and $[\spc{Y}]$ their isometry classes;
that is, $\spc{X}'\in [\spc{X}]$ if and only if $\spc{X}'\iso \spc{X}$.
Pedantically, the Gromov--Hausdorff distance from $[\spc{X}]$ 
to $[\spc{Y}]$ should be denoted as $|[\spc{X}]-[\spc{Y}]|_{\GH}$;
but we will write it as $|\spc{X}\z-\spc{Y}|_{\GH}$ and say (not quite correctly) that 
\textit{$|\spc{X}\z-\spc{Y}|_{\GH}$ is the Gromov--Hausdorff distance from  $\spc{X}$ 
to  $\spc{Y}$}.
In other words, from now on the term \textit{metric space} might also stand for its \textit{isometry class}.

The metric on $\GH$ is defined as the maximal metric such that \textit{the distance between subspaces in a metric space is not greater than the Hausdorff distance between them}.
Here is a formal definition:

\begin{thm}{Definition}\label{def:GH}
The \index{Gromov--Hausdorff distance}\emph{Gromov--Hausdorff distance} $|\spc{X}-\spc{Y}|_{\GH}$ between compact metric spaces $\spc{X}$ and $\spc{Y}$
is defined by the following
relation.
 
Given  $r > 0$, we have that $|\spc{X}-\spc{Y}|_{\GH} < r$ if and only if there exists a metric
space $\spc{W}$ and subspaces $\spc{X}'$ and $\spc{Y}'$ in $\spc{W}$ that are isometric to $\spc{X}$ and $\spc{Y}$
respectively such that $|\spc{X}'-\spc{Y}'|_{\Haus\spc{W}} < r$. 
(Here $|\spc{X}'-\spc{Y}'|_{\Haus\spc{W}}$ denotes the Hausdorff distance between sets $\spc{X}'$ and $\spc{Y}'$ in $\spc{W}$.)
\end{thm}

\begin{thm}{Theorem}\label{thm:GH-is-a-metric}
The set of isometry classes of compact metric spaces equipped with Gromov--Hausdorff metric forms a metric space (which is denoted by $\GH$).

In other words, for arbitrary compact metric spaces $\spc{X}$, $\spc{Y}$, and $\spc{Z}$ the following conditions hold

\begin{subthm}{GH-1} $|\spc{X}-\spc{Y}|_{\GH}\ge 0$;
\end{subthm}

\begin{subthm}{GH-2} $|\spc{X}-\spc{Y}|_{\GH}=0$ if and only if $\spc{X}$ is isometric to $\spc{Y}$;
\end{subthm}

\begin{subthm}{GH-3} $|\spc{X}-\spc{Y}|_{\GH}=|\spc{Y}-\spc{X}|_{\GH}$;
\end{subthm}

\begin{subthm}{GH-4} $|\spc{X}-\spc{Y}|_{\GH}+|\spc{Y}-\spc{Z}|_{\GH}\ge |\spc{X}-\spc{Z}|_{\GH}$.
\end{subthm}
\end{thm}


Note that \ref{SHORT.GH-1}, \ref{SHORT.GH-3},
and the if part of \ref{SHORT.GH-2} follow directly from \ref{def:GH}.
Part \ref{SHORT.GH-4} will be proved in Section~\ref{sec:GH-approx}.
The only-if part of \ref{SHORT.GH-2} will be proved in Section~\ref{sec:extfun=GH}.

Recall that $a\cdot\spc{X}$ denotes $\spc{X}$ \index{rescaled space}\emph{rescaled} by a factor $a>0$;
that is, $a\cdot\spc{X}$ is a metric space with the underlying set of $\spc{X}$ and the metric defined by
\[\dist{x}{y}{a\cdot\spc{X}}\df a\cdot\dist{x}{y}{\spc{X}}.\]

\begin{thm}{Exercise}\label{ex:d_GH-and-diam}
Let $\spc{X}$ be a compact metric space,
$\spc{O}$ be the one-point metric space.
Prove the following.

\begin{subthm}{ex:d_GH-and-diam:point}
$|\spc{X}-\spc{O}|_{\GH}=\tfrac12\cdot \diam \spc{X}.$
\end{subthm}

\begin{subthm}{ex:d_GH-and-diam:scale}
$|a\cdot\spc{X}-b\cdot \spc{X}|_{\GH}=\tfrac12\cdot|a-b|\cdot\diam\spc{X}.$
\end{subthm}

\begin{subthm}{ex:d_GH-and-diam:isometry}
$\iota[\spc{O}]=[\spc{O}]$ for any isometry $\iota\:\GH\to\GH$.
\end{subthm}


\end{thm}




\begin{thm}{Exercise}\label{ex:GH<H}
Find two subsets $A,B\subset\RR^2$ such that 
\[|A-B|_{\GH}>|A-\iota(B)|_{\Haus\RR^2}\]
for any isometry $\iota$ of $\RR^2$.
\end{thm}


\begin{thm}{Exercise}\label{ex:rectangle}
Let $\spc{A}_r$ be a rectangle $1$ by $r$ in the Euclidean plane 
and $\spc{B}_r$ be a closed line interval of length $r$.
Show that 
\[|\spc{A}_r-\spc{B}_r|_{\GH}>\tfrac1{10}\]
for all large $r$.
\end{thm}

\begin{thm}{Advanced exercise}\label{ex:GH-inj}
Let $\spc{X}$ and $\spc{Y}$ be compact metric spaces;
denote by $\hat{\spc{X}}$ and $\hat{\spc{Y}}$ their injective envelopes (see \ref{sec:extremal-functions}).
Show that 
\[|\hat{\spc{X}}-\hat{\spc{Y}}|_{\GH}\le 2\cdot|\spc{X}- \spc{Y}|_{\GH}.\] 
In other words, $\spc{X}\mapsto \hat{\spc{X}}$ defines a $2$-Lipschitz map $\GH\to\GH$.

\end{thm}



\section{Approximations and almost isometries}\label{sec:GH-approx}

\begin{thm}{Definition}\label{ex:defGHR}
Let $\spc{X}$ and $\spc{Y}$ be two metric spaces.
A relation $\approx$ between points in $\spc{X}$ and $\spc{Y}$ is called \index{$\eps$-approximation}\emph{$\eps$-approximation} if the following conditions hold:
\begin{itemize}
\item For any $x\in  \spc{X}$ there is $y\in \spc{Y}$ such that $x\approx y$.
\item For any $y\in  \spc{Y}$ there is $x\in \spc{X}$ such that $x\approx y$.
\item If $x\approx y$ and $x'\approx y'$ for some $x, x'\in  \spc{X}$ and $y,y'\in \spc{Y}$, then 
\[\dist{x}{x'}{\spc{X}}\lg\dist{y}{y'}{\spc{Y}}\bigr|\pm2\cdot\eps.\]
\end{itemize}

\end{thm}

\begin{thm}{Exercise}\label{ex:H-R}
Let $\spc{X}$ and $\spc{Y}$ be two compact metric spaces.
Show that
\[\dist{\spc{X}}{\spc{Y}}{\GH}<\eps\]
if and only if there is an $\eps$-approximation between $\spc{X}$ and $\spc{Y}$.

In other words, $\dist{\spc{X}}{\spc{Y}}{\GH}$ is the greatest lower bound of values $\eps>0$ such that  there is an $\eps$-approximation between $\spc{X}$ and $\spc{Y}$.
\end{thm}

\parit{Proof of \ref{GH-4}.}
Suppose that 
\begin{itemize}
\item $\approx_1$ is a relation between points in $\spc{X}$ and $\spc{Y}$,
\item $\approx_2$ is a relation between points in $\spc{Y}$ and $\spc{Z}$.
\end{itemize}
Consider the relation $\approx_3$ between points in $\spc{X}$ and $\spc{Z}$ such that
$x\approx_3 z$ if and only if there is $y\in  \spc{Y}$ such that 
$x\approx_1 y$ and $y\approx_2 z$.

It is straightforward to check that if $\approx_1$ is an $\eps_1$-approximation and $\approx_2$ is an $\eps_2$-approximation, then $\approx_3$ is an $(\eps_1+\eps_2)$-approximation.

Applying \ref{ex:H-R}, we get that if 
\[|\spc{X}-\spc{Y}|_{\GH}<\eps_1
\quad\text{and}\quad
|\spc{Y}-\spc{Z}|_{\GH}<\eps_2,
\]
then 
\[|\spc{X}-\spc{Z}|_{\GH}<\eps_1+\eps_2.\]
Hence \ref{GH-4} follows.
\qeds

The following weakened version of isometry is closely related to $\eps$-approximations.

\begin{thm}{Definition} Let $\spc{X}$ and $\spc{Y}$ be metric spaces and $\eps>0$. 
A  map\footnote{possibly noncontinuous} $f\: \spc{X} \z\to \spc{Y}$ is called an \index{almost isometry}\emph{$\eps$-isometry} 
if $f(\spc{X})$ is an $\eps$-net in $\spc{Y}$ and
\[\dist{x}{x'}{\spc{X}}\lg\dist{f(x)}{f(x')}{\spc{Y}}\bigr|\pm\eps\]
for any $x,x'\in \spc{X}$.
\end{thm}

\begin{thm}{Exercise}\label{ex:eps-isom}
Let $\spc{X}$ and $\spc{Y}$ be compact metric spaces.

\begin{subthm}{ex:eps-isom:GH>isom}
If $\dist{\spc{X}}{\spc{Y}}{\GH}<\eps$, then there is a $2\cdot\eps$-isometry $f\:\spc{X}\to\spc{Y}$.
\end{subthm}

\begin{subthm}{ex:eps-isom:isom>GH}
If there is an $\eps$-isometry $f\:\spc{X}\to\spc{Y}$, then $\dist{\spc{X}}{\spc{Y}}{\GH}<\eps$.
\end{subthm}

\end{thm}

\section{Optimal realization}\label{sec:extfun=GH}

Note that
\[\dist{\spc{X}'}{\spc{Y}'}{\Haus\spc{W}}\ge \dist{\spc{X}}{\spc{Y}}{\GH},\]
where $\spc{X}$, $\spc{Y}$, $\spc{X}'$, $\spc{Y}'$, and $\spc{W}$ are as in \ref{def:GH}.
The following proposition states that equality holds for some choice of $\spc{X}'$, $\spc{Y}'$, and $\spc{W}$.

\begin{thm}{Proposition}\label{prop:GH=H}
For any two compact metric spaces $\spc{X}$ and $\spc{Y}$ there is a metric space $\spc{W}$
with subsets $\spc{X}'$ and $\spc{Y}'$ such that 
$\spc{X}'\iso\spc{X}$, $\spc{Y}'\iso\spc{Y}$, and 
\[\dist{\spc{X}'}{\spc{Y}'}{\Haus\spc{W}}=\dist{\spc{X}}{\spc{Y}}{\GH}.\]
\end{thm}

Let us introduce the so-called \textit{appropriate functions} and use them in a reinterpretation of Gromov--Hausdorff distance.

Suppose $\spc{X}$, $\spc{Y}$, $\spc{X}'$, $\spc{Y}'$, and $\spc{W}$ are as in \ref{def:GH}.
By passing to the subspace $\spc{X}'\cup\spc{Y}'$ in $\spc{W}$, we can assume that $\spc{W}=\spc{X}'\cup\spc{Y}'$.
Note that in this case the metric on $\spc{W}$ is completely determined by the function $f\:\spc{X}\times \spc{Y}\to\RR$ defined by
\[f(x,y)
\df
\dist{x}{y}{\spc{W}};\]
a function $f$ that can appear this way will be called \index{appropriate function}\emph{appropriate}.

Note that a function $f\:\spc{X}\times\spc{Y}\to\RR$ is appropriate if and only if
$x\mapsto f(x,y)$ and $y\mapsto f(x,y)$ are extension functions [see \ref{sec:Extension property}];
that is, if
\[
\begin{aligned}
f(x,y)+f(x,y')
&\ge \dist{y}{y'}{\spc{Y}}\ge |f(x,y)-f(x,y')|,\quad\text{and}
\\
f(x,y)+f(x',y)
&\ge \dist{x}{x'}{\spc{X}}\ge |f(x,y)-f(x',y)|
\end{aligned}
\eqlbl{eq:appropriate}
\]
for any $x,x',\in\spc{X}$ and  $y,y'\in\spc{X}$.
In other words, the following defines a semimetric on $\spc{X}\sqcup\spc{Y}$
\[\dist{x}{y}{\spc{X}\sqcup\spc{Y}}\df
\begin{cases}
\dist{x}{y}{\spc{X}}&\text{if\ } x,y\in \spc{X},
\\
\dist{x}{y}{\spc{Y}}&\text{if\ } x,y\in \spc{Y},
\\
f(x,y)&\text{if\ } x\in \spc{X}\ \text{and}\ y\in \spc{Y},
\end{cases}
\]
and the corresponding metric space $\spc{W}$ contains isometric copies of $\spc{X}$ and $\spc{Y}$.

\begin{thm}{Observation}\label{obs:GH=min-appropriate}
Let $\spc{X}$, $\spc{Y}$ be metric spaces.
Given an appropriate function $f\:\spc{X}\times\spc{Y}\to\RR$, set 
\begin{align*}
a_f&=\max_{x\in \spc{X}}\{\min_{y\in\spc{Y}} \{f(x,y)\}\},
\\
b_f&=\max_{y\in \spc{Y}}\{\min_{x\in\spc{X}} \{f(x,y)\}\},
\\
c_f&=\max\{\,a_f,b_f\,\}.
\end{align*}
Then 
\[\dist{\spc{X}}{\spc{Y}}{\GH}=\inf\{\,c_f\,\},\]
where the greatest lower bound is taken for all appropriate functions $f\:\spc{X}\times\spc{Y}\to\RR$.
\end{thm}

\parit{Proof of \ref{prop:GH=H}.}
Equip the product $\spc{X}\times\spc{Y}$ with $\ell_1$-metric;
that is,
\[\dist{(x,y)}{(x',y')}{\spc{X}\times\spc{Y}}
\df
\dist{x}{x'}{\spc{X}}+\dist{y}{y'}{\spc{Y}}\]
Note that any appropriate functions $f\:\spc{X}\times\spc{Y}\to\RR$ is $1$-Lipschitz.

Let us equip the space of appropriate functions $\spc{X}\times\spc{Y}\to\RR$ with supnorm.
Observe that the functional $f\mapsto c_f$ is continuous.
By the Arzelà--Ascoli theorem, we can choose an appropriate function $f$ 
with minimal possible value $c_f$.
It remains to apply \ref{obs:GH=min-appropriate}.
\qeds

\begin{thm}{Exercise}\label{ex:XYZ}
Construct three compact metric spaces $\spc{X}$, $\spc{Y}$, and $\spc{Z}$
such that for any metric space $\spc{W}$
with subsets $\spc{X}'$, $\spc{Y}'$, and $\spc{Z}'$ such that 
$\spc{X}'\iso\spc{X}$, $\spc{Y}'\iso\spc{Y}$, and $\spc{Z}'\iso\spc{Z}$
at least one of the following three inequalities is strict
\begin{align*}
\dist{\spc{X}'}{\spc{Y}'}{\Haus\spc{W}}&\ge \dist{\spc{X}}{\spc{Y}}{\GH},
\\
\dist{\spc{Y}'}{\spc{Z}'}{\Haus\spc{W}}&\ge\dist{\spc{Y}}{\spc{Z}}{\GH},
\\
\dist{\spc{Z}'}{\spc{X}'}{\Haus\spc{W}}&\ge\dist{\spc{Z}}{\spc{X}}{\GH}.
\end{align*}
\end{thm}

\section{Convergence}

The Gromov--Hausdorff metric defines Gromov--Hausdorff \index{Gromov--Hausdorff convergence}\emph{convergence}.
Namely, a sequence of compact metric spaces $\spc{X}_n$ converges to compact metric spaces $\spc{X}_\infty$ in the sense of Gromov--Hausdorff if 
\[\dist{\spc{X}_n}{\spc{X}_\infty}{\GH}\to 0\quad\text{as}\quad n\to\infty.\]

This convergence is more important than the metric ---
in all applications, we use only the topology on $\GH$
and we do not care about the particular value of the Gromov--Hausdorff distance between spaces.
The following observation follows from \ref{ex:eps-isom}:

\begin{thm}{Observation}\label{obs:GH-e-isom}
A sequence of compact metric spaces $(\spc{X}_n)$ converges to  $\spc{X}_\infty$ in the sense of Gromov--Hausdorff if and only if there is a sequence $\eps_n\to0+$
and an $\eps_n$-isometry $f_n\:\spc{X}_n\to \spc{X}_\infty$ for each $n$.
\end{thm}

In the following exercises, \textit{convergence} is understood in the sense of Gromov--Hausdorff.

\begin{thm}{Exercise}\label{ex:GH-SC}
\begin{subthm}{ex:GH-SC:circle}
Show that a sequence of compact simply-connected length spaces cannot converge to a circle.
\end{subthm}

\begin{subthm}{ex:GH-SC:nonsc-limit}
Construct a sequence of compact simply-connected length spaces that converges to a compact non-simply-connected space.
\end{subthm}
\end{thm}

\begin{thm}{Exercise}\label{ex:sphere-to-ball}
\begin{subthm}{ex:sphere-to-ball:2}
Show that a sequence of length metrics on the 2-sphere cannot converge to the unit disk.
\end{subthm}

\begin{subthm}{ex:sphere-to-ball:3}
Construct a sequence of length metrics on the 3-sphere that converges to the unit 3-ball.
\end{subthm}

\end{thm}

\section{Uniformly totally bonded families}

\begin{thm}{Definition}\label{def:utb}
A family $\bm{Q}$ of (isometry classes) of compact metric spaces is called  \index{uniformly totally bonded family}\emph{uniformly totally bonded} if it meets the following two conditions:

\begin{subthm}{}
spaces in $\bm{Q}$ have uniformly bounded diameters; that is, there is $D\in\RR$ such that
\[\diam\spc{X}\le D\]
for any space $\spc{X}$ in $\bm{Q}$.
\end{subthm}

\begin{subthm}{}
For any $\eps>0$ there is $n\in\NN$ such that any space $\spc{X}$ in $\bm{Q}$ admits an $\eps$-net with at most $n$ points.
\end{subthm}
\end{thm}

\begin{thm}{Exercise}\label{ex:utb+pack}
Let $\bm{Q}$ be a family of compact spaces with uniformly bounded diameters.
Show that $\bm{Q}$ is uniformly totally bonded if for any $\eps>0$ there is $n\in\NN$ such that 
\[\pack_\eps\spc{X}\le n\]
for any space $\spc{X}$ in $\bm{Q}$.
\end{thm}


Fix a real constant $C$.
A Borel measure $\mu$ on a metric space $\spc{X}$ is called \index{doubling space}\emph{$C$-doubling} if
\[\mu[\oBall(p,2\cdot r)]< C\cdot\mu[\oBall(p,r)]\]
for any point $p\in \spc{X}$ and any $r>0$.
A Borel measure is called \index{doubling measure}\emph{doubling} if it is {}\emph{$C$-doubling} for some real constant $C$.

\begin{thm}{Exercise}\label{pr:doubling}
Let $\bm{Q}(C,D)$ be the set of all the compact metric spaces with diameter at most $D$ that admit a $C$-doubling measure.
Show that $\bm{Q}(C,D)$ is totally bounded.
\end{thm}

Given two metric spaces $\spc{X}$ and $\spc{Y}$, we will write $\spc{X}\le \spc{Y}$ if there is a distance-noncontracting map $f\:\spc{X}\to \spc{Y}$;
that is, if 
$$ |x-x'|_{\spc{X}}\le|f(x)-f(x')|_{\spc{Y}}$$
for any $x,x'\in \spc{X}$.

\begin{thm}{Exercise}\label{pr:under}

\begin{subthm}{pr:under:if}
Let $\spc{Y}$ be a compact metric space.
Show that the set of all spaces $\spc{X}$ such that $\spc{X}\le\spc{Y}$
is uniformly totally bounded.
\end{subthm}

\begin{subthm}{pr:under:only-if}
Show that for any uniformly totally bounded set $\bm{Q}\subset\GH$ there is a compact space $\spc{Y}$
such that $\spc{X}\le\spc{Y}$ for any $\spc{X}$ in $\bm{Q}$.
\end{subthm}

\end{thm}

\section{Gromov selection theorem}

The following theorem is analogous to Blaschke selection theorems (\ref{thm:compact+Hausdorff}).

\begin{thm}{Gromov selection theorem}\label{thm:gromov-compactness}
Let $\bm{Q}$ be a closed subset of $\GH$.
Then $\bm{Q}$ is compact if and only if the spaces in $\bm{Q}$ are uniformly totally bounded.
\end{thm}

\begin{thm}{Lemma}\label{lem:GH-complete}
The space $\GH$ is complete.
\end{thm}

Suppose 
$\spc{U}$ and $\spc{V}$ are metric spaces 
with isometric closed sets $A\subset\spc{U}$ and $A'\subset\spc{V}$;
let $\iota\:A\to A'$ be an isometry.
Consider the gluing $\spc{W}=\spc{U}\sqcup_\iota\spc{V}$ of $\spc{U}$ and $\spc{V}$ along $\iota$ [see \ref{sec:max+glue}].

Let us identify points of $\spc{U}$ and $\spc{V}$ with their images in $\spc{W}$.
It is straightforward to check that the metric on~$\spc{W}$ is defined by
\begin{align*}
\dist{u}{u'}{\spc{W}}&\df\dist{u}{u'}{\spc{U}},
\\
\dist{v}{v'}{\spc{W}}&\df\dist{v}{v'}{\spc{V}},
\\
\dist{u}{v}{\spc{W}}&\df\min\set{\dist{u}{a}{\spc{U}}+\dist{v}{\iota(a)}{\spc{V}}}{a\in A},
\end{align*}
where $u,u'\in \spc{U}$ and $v,v'\in \spc{V}$.

If one applies this construction to two copies of one space $\spc{U}$ with a set $A\subset \spc{U}$ and the identity map $\iota\:A\to A$, then the obtained space is called the \index{doubling}\emph{doubling} of $\spc{U}$ along~$A$; this space can be denoted by $\sqcup_A^2\spc{U}$.

Note that the inclusions $\spc{U}\hookrightarrow \spc{W}$ and $\spc{V}\hookrightarrow \spc{W}$ are distance-preserving.
Therefore we can and will consider $\spc{U}$ and $\spc{V}$ as the subspaces of $\spc{W}$;
this way the subsets $A$ and $A'$ will be identified and denoted further by~$A$.
Note that $A=\spc{U}\cap \spc{V}\subset \spc{W}$.

\parit{Proof.}
Let $\spc{X}_1,\spc{X}_2,\dots$ be a Cauchy sequence in $\GH$.
Passing to a subsequence if necessary, 
we can assume that $|\spc{X}_n-\spc{X}_{n+1}|_{\GH}<\tfrac1{2^n}$ for each~$n$.
In particular, for each $n$ there is a metric space $\spc{V}_n$ with distance-preserving inclusions $\spc{X}_n\hookrightarrow \spc{V}_n$ and $\spc{X}_{n+1}\hookrightarrow \spc{V}_n$ such that
\[|\spc{X}_n-\spc{X}_{n+1}|_{\Haus\spc{V}_n}<\tfrac1{2^n}\]
for each $n$.
Moreover, we may assume that $\spc{V}_n=\spc{X}_n\cup\spc{X}_{n+1}$.

Let us glue $\spc{V}_1$ to $\spc{V}_2$ along $\spc{X}_2$;
to the obtained space glue $\spc{V}_3$ along $\spc{X}_3$, and so on.
The obtained metric space $\spc{W}$
has an underlying set formed by the disjoint union of all $\spc{X}_n$ such that each inclusion $\spc{X}_n\z\hookrightarrow\spc{W}$ is distance-preserving and
\[|\spc{X}_n-\spc{X}_{n+1}|_{\Haus\spc{W}}<\tfrac1{2^n}\]
for each $n$.
In particular,
\[|\spc{X}_m-\spc{X}_n|_{\Haus\spc{W}}<\tfrac1{2^{n-1}}\eqlbl{eq:|x_m-X_n|}\] 
if $m>n$.

Denote by $\bar{\spc{W}}$ the completion of $\spc{W}$.
Observe that the union $\spc{X}_1\z\cup \spc{X}_2\cup\z\dots\cup \spc{X}_n$ is compact and \ref{eq:|x_m-X_n|} implies that it forms a $\tfrac1{2^{n-1}}$-net in $\bar{\spc{W}}$.
Whence $\bar{\spc{W}}$ is compact; see \ref{totally-bounded} and \ref{ex:compact-net}.

Applying the Blaschke selection theorem (\ref{thm:compact+Hausdorff}),
we can pass to a subsequence of $\spc{X}_n$ that converges in $\Haus\bar{\spc{W}}$; denote its limit by $\spc{X}_\infty$.
It remains to observe that $\spc{X}_\infty$ is the Gromov--Hausdorff limit of $\spc{X}_n$.
\qeds

\parit{Proof of \ref{thm:gromov-compactness}; only-if part.}
Suppose that there is no sequence $\eps_n\to0$ as described in \ref{def:utb}.
Observe that in this case
there is a sequence of spaces $\spc{X}_n\in\bm{Q}$ such that 
$$\pack_\delta \spc{X}_n\to\infty
\quad\text{as}\quad
n\to\infty$$
for some fixed $\delta>0$.

Since $\bm{Q}$ is compact, 
this sequence has a partial limit, say $\spc{X}_\infty\in\bm{Q}$.
Observe that $\pack_{\delta} \spc{X}_\infty=\infty$.
Therefore, $\spc{X}_\infty$ is not compact --- a contradiction.

\parit{If part.}
Given a positive integer $n$ consider the set of all nonempty metric spaces $\spc{W}_n$
with the number of points at most $n$ and diameter $\le D$.
Note that $\spc{W}_n$ is a compact set in $\GH$ for each $n$.

Let $D$ and $n=n(\eps)$ be as in the definition of uniformly totally bonded families (\ref{def:utb}).

Note that an $\eps$-net of any $\spc{X}\in\bm{Q}$ belongs to $\spc{W}_{n(\eps)}$.
Therefore, $\spc{W}_{n(\eps)}$ is a compact $\eps$-net of $\bm{Q}$ for any $\eps>0$.
Since $\bm{Q}$ is closed in a complete space $\GH$, it implies that $\bm{Q}$ is compact.
\qeds

\begin{thm}{Exercise}\label{ex-GH-length}
Show that the space $\GH$ is 
\begin{multicols}{3}

\begin{subthm}{ex-GH-length:separable}
separable,
\end{subthm}

\begin{subthm}{ex-GH-length:length}
length, and
\end{subthm}

\begin{subthm}{ex-GH-length:geodesic}
geodesic.
\end{subthm}

\end{multicols}


\end{thm}

\begin{thm}{Exercise}\label{ex:GH-po}
For two metric spaces $\spc{X}$ and $\spc{Y}$,
we write $\spc{X}\le \spc{Y}+\eps$ if
there is a map $f\:\spc{X}\to \spc{Y}$ such that 
\[\dist{x}{x'}{\spc{X}}\le \dist{f(x)}{f(x')}{\spc{Y}}+\eps\]
for any $x,x'\in \spc{X}$.

\begin{subthm}{ex:GH-po:a}
Show that 
$$\dist{\spc{X}}{\spc{Y}}{\GH'}
\df
\inf\set{\eps>0}{\spc{X}\le \spc{Y}+\eps
\quad\text{and}\quad
\spc{Y}\le \spc{X}+\eps}$$
defines a metric on the space of (isometry classes) of compact metric spaces.
\end{subthm}

\begin{subthm}{ex:GH-po:b}
Moreover, $\dist{*}{*}{\GH'}$ is equivalent to the Gromov--Hausdorff metric;
that is,
$$|\spc{X}_n-\spc{X}_\infty|_{\GH}\to 0 
\quad\iff\quad 
\dist{\spc{X}_n}{\spc{X}_\infty}{\GH'}\to 0$$ 
as $n\to\infty$.
\end{subthm}
\end{thm}

\section{Universal ambient space}

Recall that a metric space is called universal if it contains an isometric copy of any separable metric space (in particular, any compact metric space).
Examples of universal spaces include $\spc{U}_\infty$ --- the Urysohn space and $\ell^\infty$ --- the space of bounded infinite sequences with the metric defined by $\sup$-norm; see \ref{prop:sep-in-urys} and \ref{ex:frechet}.

The following proposition says that the space $\spc{W}$ in Definition~\ref{def:GH} can be exchanged to a fixed universal space.

\begin{thm}{Proposition}\label{prop:GH-with-fixed-Z}
Let $\spc{U}$ be a universal space.
Then for any compact metric spaces $\spc{X}$ and $\spc{Y}$ we have
$$|\spc{X}-\spc{Y}|_{\GH} = \inf \{|\spc{X}'-\spc{Y}'|_{\Haus\spc{U}}\},$$ 
where the greatest lower bound is taken over all pairs of sets $\spc{X}'$ and $\spc{Y}'$ in $\spc{U}$
which isometric to  $\spc{X}$ and $\spc{Y}$ respectively.  
\end{thm}




\parit{Proof of \ref{prop:GH-with-fixed-Z}.}
By the definition (\ref{def:GH}), we have that 
\[|\spc{X}-\spc{Y}|_{\GH} \le \inf \{|\spc{X}'-\spc{Y}'|_{\Haus\spc{U}}\};\]
it remains to prove the opposite inequality.

Suppse $|\spc{X}-\spc{Y}|_{\GH}<\eps$;
let $\spc{X}'$, $\spc{Y}'$ and $\spc{W}$ be as in \ref{def:GH}.
We can assume that $\spc{W}=\spc{X}'\cup\spc{Y}'$;
otherwise, pass to the subspace $\spc{X}'\cup\spc{Y}'$ of~$\spc{W}$.
In this case, $\spc{W}$ is compact;
in particular, it is separable.

Since $\spc{U}$ is universal, there is a distance-preserving embedding of $\spc{W}$ in $\spc{U}$;
let us keep the same notation for $\spc{X}'$, $\spc{Y}'$, and their images.
It follows that 
\[|\spc{X}'-\spc{Y}'|_{\Haus\spc{U}}<\eps,\]
--- hence the result.
\qeds

\begin{thm}{Exercise}\label{ex:GH-urysohn}
Let $\spc{U}_\infty$ be the Urysohn space.
Given two compact sets $A$ and $B$ in $\spc{U}_\infty$, define 
\[\|A-B\|\df\inf\{|A-\iota(B)|_{\Haus\spc{U}_\infty}\},\]
where the greatest lower bound is taken for all isometrics $\iota$ of $\spc{U}_\infty$.
Show that $\|{*}\z-{*}\|$ defines a semimetric 
on nonempty compact subsets of $\spc{U}_\infty$ and its corresponding metric space is isometric to $\GH$.
\end{thm}

The value $\|A-B\|$ is called Hausdorff distance \index{Hausdorff distance!up to isometry}\emph{up to isometry} from $A$ to $B$ in $\spc{U}_\infty$.

\section{Remarks}\label{sec:remarks-GH}

Suppose $\spc{X}_n\GHto \spc{X}_\infty$, then there is a metric on the disjoint union 
\[\bm{X}=\bigsqcup_{n\in \NN\cup\{\infty\}} \spc{X}_n\] 
that satisfies the following property:

\begin{thm}{Property}\label{propery:GH}
The restriction of metric on each $\spc{X}_n$ and $\spc{X}_\infty$ coincides with its original metric, 
and $\spc{X}_n\Hto \spc{X}_\infty$ as subsets in $\bm{X}$.
\end{thm}

Indeed, since $\spc{X}_n\GHto \spc{X}_\infty$, there is a metric on $\spc{V}_n=\spc{X}_n\sqcup \spc{X}_\infty$ such that the restriction of metric on each $\spc{X}_n$ and $\spc{X}_\infty$ coincides with its original metric, and $\dist{\spc{X}_n}{\spc{X}_\infty}{\Haus\spc{V}_n}<\eps_n$ for some sequence $\eps_n\to 0$.
Gluing all $\spc{V}_n$ along $\spc{X}_\infty$, we obtain the required space $\bm{X}$.

In other words, the metric on $\bm{X}$ \textit{defines} the convergence $\spc{X}_n\z\GHto \spc{X}_\infty$.
This metric makes it possible to talk about limits of sequences $x_n\in \spc{X}_n$ as $n\to\infty$, as well as weak limits of a sequence of Borel measures $\mu_n$ on $\spc{X}_n$ and so on.

For that reason, it is useful to define \index{Hausdorff convergence}\emph{convergence} by specifying the metric on $\bm{X}$ that satisfies the property
for the variation of Hausdorff convergence described in Section~\ref{sec:H-variation}.

This approach is more flexible;
in particular, it can be used to define the Gromov--Hausdorff convergence of arbitrary metric spaces (not necessarily compact).
A limit space for this generalized convergence is not uniquely defined.
For example, if each space $\spc{X}_n$ in the sequence is isometric to the half-line, then its limit might be isometric to the half-line or the whole line.
The first convergence is evident and the second could be guessed from the diagram.

\begin{figure}[ht!]
\vskip-0mm
\centering
\includegraphics{mppics/pic-500}
\end{figure}

Often the isometry class of the limit can be fixed by marking a point $p_n$ in each space $\spc{X}_n$, it is called \index{pointed convergence}\emph{pointed Gromov--Hausdorff convergence} --- we say that $(\spc{X}_n,p_n)$ converges to $(\spc{X}_\infty,p_\infty)$ if there is a metric on $\bm{X}$ as in \ref{propery:GH} such that $\spc{X}_n\Hto \spc{X}_\infty$ and $p_n\to p_\infty$.
For example, the sequence $(\spc{X}_n,p_n)=(\RR_+,0)$ converges to $(\RR_+,0)$, while $(\spc{X}_n,p_n)=(\RR_+,n)$ converges to $(\RR,0)$.

The pointed convergence works nicely for proper metric spaces;
the following theorem is an analog of Gromov's selection theorem for this convergence.

\begin{thm}{Theorem}\label{thm:pointed-gromov-compactness}%
Let $\bm{Q}$ be a set of isometry classes of pointed proper metric spaces.
Assume that for any $R>0$, the $R$-balls in the spaces centered at the marked points form a uniformly totally bounded family of spaces.
Then $\bm{Q}$ is precompact with respect to the pointed Gromov--Hausdorff convergence. 
\end{thm}
