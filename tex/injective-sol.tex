\parbf{\ref{ex:+-c}.}
Note that if $c<0$, then $r>s$.
The latter is impossible since $r$ is extremal and $s$ is admissible.

Observe that the function $\bar r=\min\{\,r,s+c\,\}$ is admissible.
Indeed, choose $x,y\in \spc{X}$.
If $\bar r(x)=r(x)$ and $\bar r(y)=r(y)$, then 
\[\bar r(x)+\bar r(y)=r(x)+ r(y)\ge \dist{x}{y}{}.\]
Further, if $\bar r(x)=s(x)+c$, then 
\begin{align*}
\bar r(x)+\bar r(y)&\ge [s(x)+c]+ [s(y)-c]= 
\\
&=s(x)+s(y) \ge 
\\
&\ge\dist{x}{y}{}.
\end{align*}

Since $r$ is extremal, we have $r=\bar r$;
that is, $r\le s+c$.

\parbf{\ref{ex:circle};} \textit{only-if part}.
Suppose $r$ is extremal.
By \ref{lem:extremal-lipschitz}, $r$ is $1$-Lipschitz.
Since $\mathbb{S}^1$ is compact, \ref{lem:opposite-compact} implies that for any $p\in \mathbb{S}^1$ there is $q\in \mathbb{S}^1$ such that 
\[r(p)+r (q) = \dist{p}{q}{\mathbb{S}^1}.\]
Therefore
\begin{align*}
\pi&=\dist{p}{(-p)}{\mathbb{S}^1}\le 
\\
&\le 
r(p)+r(-p)=
\\
&=
r(p)+r(q) +r(-p) -r(q)\le
\\
&\le
\dist{p}{q}{\mathbb{S}^1}+\dist{q}{(-p)}{\mathbb{S}^1}=
\\
&=\pi.
\end{align*}
It implies that we have equalities in both places.
Hence the only-if part follows.

\parit{If part.}
Assume $r$ is 1-Lipschitz function such that $r(p)+r(-p)=\pi$.
Then 
\begin{align*}
\dist pq{\mathbb{S}^1}&=
\dist{p}{(-p)}{\mathbb{S}^1}-\dist{q}{(-p)}{\mathbb{S}^1}\ge
\\
&\ge\pi -(r(-p)-r(q))=
\\
&=r(p)+r(q).
\end{align*}
Therefore $r$ is admissible.

Finally, if $r$ is not extremal, then there is an admissible function $s\le r$ such that $s(p)<r(p)$ for some $p$.
The latter contradicts the equality $r(p)+r(-p)=\pi$.

\parit{Source:} \cite[Proposition 2.7]{zuest}.

\parbf{\ref{ex:inj=complete-geodesic-contractible}.}
Choose an injective space $\spc{Y}$.

\textit{\ref{SHORT.ex:inj=complete-geodesic-contractible:complete}.}
Fix a Cauchy sequence $x_n$ in $\spc{Y}$;
we need to show that it has a limit $x_\infty\in \spc{Y}$.
Consider metric on $\spc{X}=\NN\cup\{\infty\}$ defined by 
\begin{align*}
\dist{m}{n}{\spc{X}}&\df\dist{x_m}{x_n}{\spc{Y}},
\\
\dist{m}{\infty}{\spc{X}}&\df\lim_{n\to\infty}\dist{x_m}{x_n}{\spc{Y}}.
\end{align*}
Since the sequence is Cauchy, so is the sequence $\ell_n=\dist{x_m}{x_n}{\spc{Y}}$ for any $m$.
Therefore, the last limit is defined.

By construction, the map $n\mapsto x_n$ is distance-preserving on $\NN\subset \spc{X}$.
Since $\spc{Y}$ is injective, this map can be extended to $\infty$ as a short map; set $\infty\mapsto x_\infty$.
Since $\dist{x_n}{x_\infty}{\spc{Y}}\le \dist{n}{\infty}{\spc{X}}$ 
and $\dist{n}{\infty}{\spc{X}}\to 0$, we get that
$x_n\to x_\infty$ as $n\to\infty$.

\textit{\ref{SHORT.ex:inj=complete-geodesic-contractible:geodesic}.}
Applying the definition of injective space, we get a midpoint for any pair of points in $\spc{Y}$.
By \ref{SHORT.ex:inj=complete-geodesic-contractible:complete},
$\spc{Y}$ is a complete space.
It remains to apply \ref{lem:mid>geod:geod}.

\textit{\ref{SHORT.ex:inj=complete-geodesic-contractible:contractible}.}
Let $k\:\spc{Y}\hookrightarrow \ell^\infty(\spc{Y})$ be the Kuratowski embedding (\ref{lem:kuratowski}).
Observe that $\ell^\infty(\spc{Y})$ is contractible;
in particular, there is a homotopy $k_t\:\spc{Y}\hookrightarrow \ell^\infty(\spc{Y})$ such that $k_0=k$ and $k_1$ is a constant map.
(In fact, one can take $k_t=(1-t)\cdot k$.)

Since $k$ is distance-preserving and $\spc{Y}$ is injective,
there is a short map $f\:\ell^\infty(\spc{Y})\to \spc{Y}$ such that the composition $f\circ k$ is the identity map on $\spc{Y}$.
The composition $f\circ k_t\:\spc{Y}\hookrightarrow \spc{Y}$ provides the needed homotopy. 

\parbf{\ref{ex:injective-spaces}.}
Suppose that a short map $f\:A\to\spc{Y}$ is defined on a subset $A$ of a metric space $\spc{X}$.
We need to construct a short extension $F$ of $f$.
Without loss of generality, we may assume that $A\ne\emptyset$, otherwise map the whole $\spc{X}$ to a single point.
By Zorn's lemma, it is sufficient to enlarge $A$ by a single point $x\notin A$.

\parit{\ref{SHORT.ex:injective-spaces:R}.}
Suppose $\spc{Y}=\RR$.
Set 
\[F(x)=\inf\set{f(a)-\dist{a}{x}{}}{a\in A}.\] 
Observe that $F$ is short and $F(a)=f(a)$ for any $a\in A$.

\parit{\ref{SHORT.ex:injective-spaces:tree}.}
Suppose  $\spc{Y}$ is a complete metric tree.
Fix points $p\in \spc{X}$ and $q\in\spc{Y}$.
Given a point $a\in A$,
let $x_a\in\cBall[f(a),\dist{a}{p}{}]$ be the point closest to $f(x)$.
Note that $x_a\in[q\,f(a)]$ and either $x_a=q$ or $x_a$ lies on distance $\dist{a}{p}{}$ from $f(a)$.

Note that the geodesics $[q\,x_a]$ are nested;
that is, for any $a,b\in A$ we have either $[q\,x_a]\subset [q\,x_b]$ or $[q\,x_b]\subset [q\,x_a]$.
Moreover, in the first case we have $\dist{x_b}{f(a)}{}\le \dist{p}{a}{}$ and in the second $\dist{x_a}{f(b)}{}\le \dist{p}{b}{}$.

It follows that the closure of the union of all geodesics $[q\,x_a]$ for $a\in\spc{A}$ is a geodesic.
Denote by $x$ its endpoint; it exists since $\spc{Y}$ is complete.
It remains to observe that $\dist{x}{f(a)}{}\le \dist{p}{a}{}$ for any $a\in\spc{A}$;
that is, one can take $f(p)=x$.

\parit{\ref{SHORT.ex:injective-spaces:ell-infty}.}
In this case, $\spc{Y}=(\RR^2,\ell^\infty)$.
Note that $\spc{X}\to (\RR^2,\ell^\infty)$ is a short map if and only if both of its coordinate projections are short.
It remains to apply \ref{SHORT.ex:injective-spaces:R}.

More generally, \textit{any $\ell^\infty$-product of injective spaces is injective};
in particular, if $\spc{Y}$ and $\spc{Z}$ are injective then the product $\spc{Y}\times\spc{Z}$ equipped with the metric 
\[\dist{(y,z)}{(y',z')}{\spc{Y}\times\spc{Z}}=\max\{\,\dist{y}{y'}{\spc{Y}},\dist{z}{z'}{\spc{Z}}\,\}\]
is injective as well.


\parbf{\ref{ex:ultrametric}.}
Choose three points $x,y,z\in\spc{X}$ and set $\spc{A}=\{x,z\}$.
Let $f\:\spc{A}\z\to \spc{A}$ be the identity map.
Then $F(y)=x$ or $F(y)=z$.
The strong triangle inequality easily follows in both cases.

\parbf{\ref{ex:ultrametric-converse}}; \textit{main part.}
Choose a maximal subset $A\supset K$ that admits a short retraction $f\:A\to K$;
it exists by Zorn's lemma.
If $A$ is the whole space, then the problem is solved.
Otherwise, choose $p\notin A$.

Choose a sequence of points $a_n\in A$ such that $\dist{a_n}{p}{}$ converge to the exact lower bound on the distances from points in $A$ to $p$.
Since $K$ is compact, we can pass to a subsequence of $a_n$ such that $f(a_n)$ converges.
Set $f(p)=\lim f(a_n)$.

It remains to check that 
\[\dist{f(a)}{f(p)}{}\le\dist{a}{p}{}\eqlbl{eq:short-retract}\]
for any $a\in A$.
Choose $\eps>0$; note that 
\[\dist{a_n}{p}{}<\dist{a}{p}{}+\eps
\quad\text{and}\quad
\dist{f(a_n)}{f(p)}{}<\dist{f(a)}{f(a_n)}{}+\eps
\]
for all large~$n$.
Therefore, 
\begin{align*}
\dist{f(a)}{f(p)}{}&\le \max\{\,\dist{f(a)}{f(a_n)}{},\dist{f(a_n)}{f(p)}{}\,\}\le
\\
&\le \dist{f(a)}{f(a_n)}{}+\eps\le
\\
&\le \dist{a}{a_n}{} +\eps\le 
\\
&\le \max\{\,\dist{a}{p}{},\dist{a_n}{p}{}\,\}+\eps< 
\\
&< \dist{a}{p}{}+2\cdot\eps.
\end{align*}
Since $\eps>0$ is arbitrary, we get \ref{eq:short-retract}.

\parit{Example.}
Consider set of $\{\infty,1,2,\dots\}$ with metric defined by 
\[|m-n|=1+\frac1{\min\{m,n\}}\]
for $m\ne n$.
Observe that the space is complete, the subset $\{1,2,\dots\}$ is closed, but it is not a short retract of the ambient space.

\parbf{\ref{ex:one-point-gluing}.}
Apply \ref{thm:injective=hyperconvex:balls}.

\parbf{\ref{ex:urysohn-hyperconvex}.}
Denote by $\spc{U}_d$ the $d$-Urysohn space,
so $\spc{U}_\infty$ is the Urysohn space.

The extension property implies finite hyperconvexity.
It remains to show that $\spc{U}_d$ is not countably hyperconvex.

Suppose that $d<\infty$.
Then $\diam\spc{U}_d=d$ and for any point $x\in\spc{U}_d$ there is a point $y\in\spc{U}_d$ such that $\dist{x}{y}{\spc{U}_d}=d$.
It follows that there is no point $z\in\spc{U}_d$ such that $\dist{z}{x}{\spc{U}_d}\le \tfrac d2$ for any $x\in\spc{U}_d$.
Whence $\spc{U}_d$ is not countably hyperconvex.

Use \ref{ex:sphere-in-urysohn:midpoint} to reduce the case $d=\infty$ to the case $d<\infty$.

\parbf{\ref{ex:Inj(compact)}.}
Observe and use that the functions in $\Inj\spc{X}$ are 1-Lipschitz and uniformly bounded.

\parbf{\ref{ex:tripod+square}}; \ref{SHORT.ex:tripod+square:tripod}.
Let $f$ be an extremal function.
By \ref{lem:opposite-compact}, at least two of the numbers $f(a)+f(b)$, $f(b)+f(c)$, and $f(c)+f(a)$ are $1$.
It follows that for some $x\in[0,\tfrac12]$, we have 
\begin{align*}
f(a)&=1\pm x,&
f(b)&=1\pm x,&
f(c)&=1\pm x,
\end{align*}
where we have one ``minus'' and two ``pluses'' in these three formulas.

\begin{wrapfigure}{o}{30 mm}
\vskip-0mm
\centering
\includegraphics{mppics/pic-3}
\end{wrapfigure}

Suppose that
\begin{align*}
g(a)&=1\pm y,& g(b)&=1\pm y,& g(c)&=1\pm y
\end{align*}
is another extremal function.
Then $|f-g|\z=|x-y|$ if $g$ has ``minus'' at the same place as $f$ and $|f-g|=|x+y|$ otherwise.

It follows that $\Inj\spc{X}$ is isometric to a {}\emph{tripod} --- three segments of length $\tfrac12$ glued at one end.

\begin{wrapfigure}{o}{30 mm}
\vskip-0mm
\centering
\includegraphics{mppics/pic-4}
\end{wrapfigure}

\parit{\ref{SHORT.ex:tripod+square:square}.}
Assume $f$ is an extremal function.
Use \ref{lem:opposite-compact} to show that
\[f(x)+f(y)=f(p)+f(q)=2;\]
in particular, two values $a=f(x)-1$ and $b=f(p)-1$ completely describe the function $f$.
Since $f$ is extremal, we also have that 
\[(1\pm a)+(1\pm b)\ge 1\]
for all 4 choices of signs;
equivalently, 
\[|a|+|b|\le 1.\]

It follows that $\Inj\spc{X}$ is isometric to the rhombus $|a|+|b|\le 1$ in the $(a,b)$-plane with the metric induced by the $\ell^\infty$-norm.


\parbf{\ref{ex:4-on-a-line}.}
Recall that 
\[\dist{f}{g}{\Inj\spc{X}}=\sup\set{|f(x)-g(x)|}{x\in\spc{X}}\]
and 
\[\dist{f}{p}{\Inj\spc{X}}=f(p)\]
for any $f,g\in \Inj\spc{X}$ and $p\in \spc{X}$.

Since $\spc{X}$ is compact we can find a point $p\in\spc{X}$ such that 
\[\dist{f}{g}{\Inj\spc{X}}=|f(p)-g(p)|=\left|\dist{f}{p}{\Inj\spc{X}}-\dist{g}{p}{\Inj\spc{X}}\right|.\]
Without loss of generality, we may assume that 
\[\dist{f}{p}{\Inj\spc{X}}
=
\dist{g}{p}{\Inj\spc{X}}
+
\dist{f}{g}{\Inj\spc{X}}.\]
Applying \ref{lem:opposite-compact}, we can find a point $q\in\spc{X}$ such that 
\[\dist{q}{p}{\Inj\spc{X}}
=
\dist{f}{p}{\Inj\spc{X}}
+
\dist{f}{q}{\Inj\spc{X}},\]
whence the result.

Since $\Inj\spc{X}$ is injective (\ref{prop:InjX-is-injective}), by \ref{ex:inj=complete-geodesic-contractible:geodesic} it has to be geodesic. It remains to note that the concatenation of geodesics $[pq]$, $[gf]$, and $[fq]$ is a required geodesic $[pq]$.

\parbf{\ref{ex:d-p-inclusion}.}
Show that there is a pair of short maps 
$\Inj\spc{X}\to\Inj\spc{U}\to\Inj\spc{X}$ 
such that their composition is the identity on $\spc{X}$.
Make a conclusion.

