\parbf{\ref{ex:almost-min}.}
Assume the statement is wrong. 
Then for any point $x\in \spc{X}$, there is a point $x'\in \spc{X}$ such that 
\[\dist{x}{x'}{}< \rho(x)
\quad\text{and}\quad
\rho(x')\le\frac{\rho(x)}{1+\eps}.\]
Consider a sequence of points $(x_n)$ such that $x_{n+1}\z=x_n'$.
Clearly 
\[\dist{x_{n+1}}{x_n}{}
\le
\frac{\rho(x_0)}{\eps\cdot(1+\eps)^n}
\quad\hbox{and}\quad
\rho(x_n)\le \frac{\rho(x_0)}{(1+\eps)^n}.\] 
Therefore $(x_n)$ is a Cauchy sequence.
Since $\spc{X}$ is complete, the sequence $(x_n)$ converges;
denote its limit by $x_\infty$.
Since $\rho$ is a continuous function we get
\begin{align*}\rho(x_\infty)&=\lim_{n\to\infty}\rho(x_n)=
\\&=0.
\end{align*}

The latter contradicts that $\rho>0$.


\parbf{\ref{ex:non-contracting-map}.}
Given any pair of point $x_0,y_0\in \spc{K}$, 
consider two sequences $x_0,x_1,\dots$ and $y_0,y_1,\dots$
such that $x_{n+1}=f(x_n)$ and $y_{n+1}\z=f(y_n)$ for each $n$.

Since $\spc{K}$ is compact, 
we can choose an increasing sequence of integers $n_k$
such that both sequences $(x_{n_i})_{i=1}^\infty$ and $(y_{n_i})_{i=1}^\infty$
converge.
In particular, both are Cauchy;
that is,
\[
|x_{n_i}-x_{n_j}|_{\spc{K}}, |y_{n_i}-y_{n_j}|_{\spc{K}}\to 0
\quad
\text{as}
\quad\min\{i,j\}\to\infty.
\]


Since $f$ is non-contracting, we get
\[
|x_0-x_{|n_i-n_j|}|
\le 
|x_{n_i}-x_{n_j}|.
\]

It follows that  
there is a sequence $m_i\to\infty$ such that
\[
x_{m_i}\to x_0\quad\text{and}\quad y_{m_i}\to y_0\quad\text{as}\quad i\to\infty.
\leqno({*})\]

Set \[\ell_n=|x_n-y_n|_{\spc{K}}.\]
Since $f$ is non-contracting, the sequence $(\ell_n)$ is nondecreasing.

By $({*})$,  $\ell_{m_i}\to\ell_0$ as $m_i\to\infty$.
It follows that $(\ell_n)$ is a constant sequence.

In particular 
\[|x_0-y_0|_{\spc{K}}=\ell_0=\ell_1=|f(x_0)-f(y_0)|_{\spc{K}}\]
for any pair of points $(x_0,y_0)$ in $\spc{K}$.
That is, $f$ is distance-preserving, in particular injective.

From $({*})$, we also get that $f(\spc{K})$ is everywhere dense.
Since $\spc{K}$ is compact $f\:\spc{K}\to \spc{K}$ is surjective. Hence the result follows.

\parit{Remarks.}
This is a basic lemma in the introduction to Gromov--Hausdorff distance \cite[see 7.3.30 in][]{burago-burago-ivanov}.
This proof is not quite standard,
I learned this proof from Travis Morrison, 
a student in my MASS class at Penn State, Fall 2011.

Note that as an easy corollary one can see that any surjective non-expanding map from a compact metric space to itself is an isometry.


\parbf{\ref{ex:gross}.}
If such number does not exist then the ranges of average distance functions have empty intersection.
Since $X$ is a compact length-metric space, the range of any continuous function on $X$ is a closed interval.
By 1-dimesional Helly's theorem, there is a pair of such range intervals that do not intersect.
That is, for two point-arrays $(x_1,\dots,x_n)$ and $(y_1,\dots,y_m)$
and their average distance functions 
\[f(z)=\tfrac1n\cdot\sum_i|x_i-z|_X\quad\text{and}\quad h(z)=\tfrac1m\cdot\sum_j|y_j-z|_X,\] we have 
$$\min\set{f(z)}{z\in X}>\max\set{h(z)}{z\in X}.\leqno({*})$$

Note that 
$$\tfrac1m\cdot\sum_j f(y_j)=\tfrac1{m\cdot n}\cdot\sum_{i,j}|x_i-y_j|_X=\tfrac1n\cdot\sum_i h(x_i);$$
that is, the average value of $f(y_j)$ coincides with the average value of $h(x_i)$, 
which contradicts $({*})$.

\parit{Remarks.}
This is a result of Oliver Gross \cite{gross}. 
The value $\ell$ is called the \index{rendezvous value}\emph{rendezvous value} of $X$;
in fact it is uniquely defined.

\parbf{\ref{ex:no-geod}.}
We assume that the space is nontrivial, otherwise a one-point space is an example.

Consider the unit ball $(B,\rho_0)$
in the space $c_0$ of all sequences converging to zero equipped with the sup-norm.

Consider another metric $\rho_1$ which is different from $\rho_0$ by the conformal factor
\[\phi(\bm{x})=2+\tfrac{1}2\cdot x_1+\tfrac{1}4\cdot x_2+\tfrac{1}8\cdot x_3+\dots,\]
where $\bm{x}=(x_1,x_2\,\dots)\in B$.
That is, if $\bm{x}(t)$, $t\in[0,\ell]$, is a curve parametrized by $\rho_0$-length 
then its $\rho_1$-length is defined by
\[\length_{\rho_1}\bm{x}\df\int\limits_0^\ell\phi\circ\bm{x}(t)\cdot dt.\]
Note that the metric $\rho_1$ is bi-Lipschitz to~$\rho_0$.

Assume $\bm{x}(t)$ and $\bm{x}'(t)$ are two curves parametrized by $\rho_0$-length that differ only in the $m$-th coordinate, denoted by $x_m(t)$ and $x_m'(t)$ respectively.
Note that if $x'_m(t)\le x_m(t)$ for any $t$ and 
the function $x'_m(t)$ is locally $1$-Lipschitz at all $t$ such that $x'_m(t)< x_m(t)$, then 
\[\length_{\rho_1}\bm{x}'\le \length_{\rho_1}\bm{x}.\]
Moreover this inequality is strict if $x'_m(t)< x_m(t)$ for some~$t$.

Fix a curve $\bm{x}(t)$, $t\in[0,\ell]$, parametrized by  $\rho_0$-length.
We can choose $m$ large, so that $x_m(t)$ is sufficiently close to $0$ for any~$t$.
In particular, for some values $t$, we have $y_m(t)<x_m(t)$, where
\[y_m(t)=(1-\tfrac t\ell)\cdot x_m(0)
+\tfrac t\ell\cdot x_m(\ell)
-\tfrac 1{100}\cdot \min\{t,\ell-t\}.\]
Consider the curve $\bm{x}'(t)$ as above with
\[x'_m(t)=\min\{x_m(t),y_m(t)\}.\]
Note that $\bm{x}'(t)$ and $\bm{x}(t)$ have the same end points, and by the above
\[\length_{\rho_1}\bm{x}'<\length_{\rho_1}\bm{x}.\]
That is, for any curve $\bm{x}(t)$ in $(B,\rho_1)$, we can find a shorter curve $\bm{x}'(t)$ with the same end points.
In particular, $(B,\rho_1)$ has no geodesics.

\parit{Remarks.}
This solution was suggested by Fedor Nazarov~\cite{nazarov}.

\parbf{\ref{ex:compact+connceted}.}
Choose a sequence $\varepsilon_n\to 0$ and a $\varepsilon_n$-net $N_n$ of $K$ for each $n$.
Assume $N_0$ is a one-point set, so $\eps_0>\diam K$.
Connect each point $x\in N_{k+1}$ to a point $y\in N_{k}$ by a curve of length at most $\eps_k$.

Consider the union $K'$ of all these curves with $K$; observe that $K'$ is compact and path connected.

\parit{Source:} This problem was stated by Eugene Bilokopytov \cite{bilokopytov}.

\parbf{\ref{ex:compact=>complete}.}
Choose a Cauchy sequence $(x_n)$ in $(\spc{X},\|*-*\|)$; it sufficient to show that a subsequence of $(x_n)$ converges.

Note that the sequence $(x_n)$ is Cauchy in $(\spc{X},|*-*|)$;
denote its limit by $x_\infty$.

After passing to a subsequence, we can assume that $\|x_n-x_{n+1}\|\z<\tfrac1{2^n}$.
It follows that there is a 1-Lipschitz path $\gamma$ in $(\spc{X},\|*-*\|)$ such that $x_n=\gamma(\tfrac1{2^n})$ for each $n$ and $x_\infty=\gamma(0)$.

It follows that
\begin{align*}
\|x_\infty-x_n\|&\le \length\gamma|_{[0,\frac1{2^n}]}\le
\\
&\le \tfrac1{2^n}.
\end{align*}
In particular $x_n$ converges.

\parit{Source:} \cite[Lemma 2.3]{petrunin-stadler}.


\begin{wrapfigure}{r}{20 mm}
\vskip-0mm
\centering
\includegraphics{mppics/pic-1}
\end{wrapfigure}

\parbf{\ref{exercise from BH}.}
Consider the following subset of $\RR^2$ equipped with the induced length metric
\[
\spc{X}
=
\bigl((0,1]\times\{0,1\}\bigr)
\cup
\bigl(\{1,\tfrac12,\tfrac13,\dots\}\times[0,1]\bigr)
\]
Note that $\spc{X}$ is locally compact and geodesic.

Its completion $\bar{\spc{X}}$ is isometric to the closure of $\spc{X}$ equipped with the induced length metric.
Note that $\bar{\spc{X}}$ is obtained from $\spc{X}$ by adding two points $p=(0,0)$ and $q=(0,1)$.

Observe that the point $p$ admits no compact neighborhood in $\bar{\spc{X}}$ 
and there is no geodesic connecting $p$ to $q$ in~$\bar{\spc{X}}$. 

\parit{Source:} \cite[I.3.6(4)]{bridson-haefliger}.

%%%%%%%%%%%%%%%%%%%%%%%%%%%%%%
