\chapter{Width and systole}

This lecture is based on a paper of Alexander Nabutovsky \cite{nabutovsky}.

\section{Partition of unity}

\begin{thm}{Proposition}\label{thm:part-unit}
 Let $\{V_i\}$ be a finite open covering of a compact metric space ${\spc{X}}$.
Then there are Lipschitz functions $\psi_i\:{\spc{X}}\z\to[0,1]$ such that (1) if $\psi_i(x)>0$, then $x\in V_i$ and (2) for any $x\in {\spc{X}}$ we have
$$\sum_i\psi_i(x)=1.$$

\end{thm}

A collection of functions $\{\psi_i\}$ that meets the conditions in \ref{thm:part-unit} is called 
a \index{partition of unity}\emph{partition of unity subordinate to the covering} $\{V_i\}$.

\parit{Proof.}
Denote by $\phi_i(x)$ the distance from $x$ to the complement of $V_i$;
that is,
$$\phi_i(x)=\distfun_{{\spc{X}}\setminus V_i}(x).$$
Note $\phi_i$ is $1$-Lipschitz
for any $i$
and $\phi_i(x)>0$ if and only if $x\in V_i$.
Since $\{V_i\}$ is a covering, we have that
$$\Phi(x)\df\sum_i\phi_i(x)>0\ \ \text{for any}\ \ x\in {\spc{X}}.$$
Since $\spc{X}$ is compact, $\Phi>\delta$ for some $\delta>0$.
It follows that $x\mapsto\tfrac1{\Phi(x)}$ is a bounded Lipschitz function. 

Set 
$$\psi_k(x)=\frac{\phi_k(x)}{\Phi(x)}.$$
Observe that by construction the functions $\psi_i$ meet the conditions in the proposition.
\qedsf

\section{Nerves}

Let $\{V_1,\dots,V_k\}$ be a finite open cover of a compact metric space $\spc{X}$.
Consider an abstract simplicial complex $\spc{N}$, with one vertex $v_i$ for each set $V_i$ such that a simplex with vertices $v_{i_1},\dots, v_{i_m}$ is included in $\spc{N}$ if 
the intersection $V_{i_1}\cap\dots\cap V_{i_m}$ is nonempty.
\begin{figure}[h!]
\vskip-0mm
\centering
\includegraphics{mppics/pic-1402}
\end{figure}
The obtained simplicial complex $\spc{N}$ is called the \index{nerve}\emph{nerve of the covering $\{V_i\}$}.
Evidently $\spc{N}$ is a finite simplicial complex ---
it is a subcomplex of a simplex with the vertices $\{v_1,\dots,v_k\}$.

Note that the nerve $\spc{N}$ has dimension at most $n$ if and only if the covering $\{V_1,\dots,V_k\}$ has \index{multiplicity of covering}\emph{multiplicity} at most $n+1$;
that is, any point $x\in\spc{X}$ belongs to
at most $n+1$ sets of the covering.

Suppose $\{\psi_i\}$ is  
a partition of unity subordinate to the covering $\{V_1,\dots,V_k\}$.
Choose a point $x\in {\spc{X}}$.
Note that the set
$$\{v_{i_1},\dots,v_{i_n}\}=\set{v_i}{\psi_i(x)>0}$$
form vertices of a simplex in $\spc{N}$.
Therefore 
$$\bm{\psi}\:x\mapsto \psi_1(x)\cdot v_1+\psi_2(x)\cdot v_2+\dots+\psi_k(x)\cdot v_n.$$
describes a Lipschitz map from ${\spc{X}}$ to the nerve $\spc{N}$ of $\{V_i\}$.
In other words, $\bm{\psi}$ maps a point $x$ to the point in $\spc{N}$ with \index{barycentric coordinates}\emph{barycentric coordinates} $(\psi_1(x),\dots,\psi_k(x))$.

Recall that the \index{star}\emph{star} of a vertex $v_i$ (briefly $\Star_{v_i}$) is defined as the union of the interiors of all simplicies that have $v_i$ as a vertex.
Recall that $\psi_i(x)>0$ implies $x\in V_i$.
Therefore we get the following:

\begin{thm}{Proposition}\label{prop:space->nerve}
Let $\spc{N}$ be a nerve of an open covering $\{V_1,\z\dots,V_k\}$ of a compact metric space $\spc{X}$.
Denote by $v_i$ the vertex of $\spc{N}$ that corresponds to $V_i$.

Then there is a Lipschitz map $\bm{\psi}\:\spc{X}\to\spc{N}$ such that $\bm{\psi}(V_i)\z\subset\Star_{v_i}$ for every $i$.
\end{thm}


\section{Width}

Suppose $A$ is a subset of a metric space $\spc{X}$.
The radius of $A$ (briefly $\rad A$) is defined as the least upper bound on the values $R>0$ such that $\oBall(x,R)\supset A$ for some $x\in \spc{X}$.

\begin{thm}{Definition}\label{def:width}
Let $\spc{X}$ be a metric space.
The \index{width}\emph{$n$-th width} of $\spc{X}$ (briefly $\width_n\spc{X}$) is the least upper bound on values $R>0$ such that $\spc{X}$ admits a finite open covering $\{V_i\}$ with multiplicity at most $n+1$ and $\rad V_i< R$ for each $i$.
\end{thm}

\parbf{Remarks.}

\begin{itemize} 
\item Observe that 
\[\width_0\spc{X}\ge\width_1\spc{X}\ge\width_2\spc{X}\ge\dots\]
for any compact metric space $\spc{X}$.
Moreover, if $\spc{X}$ is connected, then 
\[\width_0\spc{X}=\rad\spc{X}.\]

\item Usually width is defined using diameter instead of radius, but the results differ at most twice.
Namely, if $r$ is the $n$-th radius-width and $d$ --- the $n$-th diameter-width, then 
$r\le d\le 2\cdot r$.

\item Note that \index{Lebesgue covering dimension}\emph{Lebesgue covering dimension} of $\spc{X}$ can be defined as the least number $n$ such that $\width_n\spc{X}=0$.

\item Another closely related notion is the so-called \index{macroscopic dimension}\emph{macroscopic dimension on scale $R$};
it is defined as the  least number $n$ such that $\width_n\spc{X}<R$.
\end{itemize}



\begin{thm}{Exercise}\label{ex:macrodimension}
Suppose $\spc{X}$ is a compact metric space such that any closed curve $\gamma$ in $\spc{X}$ can be contracted in its $R$-neighborhood.
Show that macroscopic dimension of $\spc{X}$ on scale $100\cdot R$ is at most 1.

What about quasiconverse? That is, suppose a simply connected compact metric space $\spc{X}$ has macroscopic dimension at most 1 on scale $R$, is it true that any closed curve $\gamma$ in $\spc{X}$ can be contracted in its $100\cdot R$-neighborhood?
\end{thm}


The following exercise gives a good reason for the choice of term \index{width}\emph{width}; it also can be used as an alternative definition.

\begin{thm}{Exercise}\label{ex:width=suprad(inv)}
Suppose $\spc{X}$ is a compact metric space.
Show that $\width_n\spc{X}<R$ if and only if there is a finite $n$-dimensional simplicial complex $\spc{N}$ and a continuous map $\bm{\psi}\:\spc{X}\to \spc{N}$
such that 
\[\rad[\bm{\psi}^{-1}(s)]<R\]
for any $s\in \spc{N}$.
\end{thm}

\section{Riemannian polyhedrons}

A \index{Riemannian!polyhedron}\emph{Riemannian polyhedron} is defined as a finite simplicial complex with a metric tensor on each simplex such that the restriction of the metric tensor to a subsimplex coincides with the metric on the subsimplex.

The {}\emph{dimension} of a Riemannian polyhedron is defined as the largest dimension in its triangulation.
For Riemannian polyhedrons one can define length of curves and volume the same way as for Riemannian manifolds.

The obtained metric space will be called \emph{Riemannian polyhedron} as well.
A \index{triangulation}\emph{triangulation} of Riemannian polyhedron  will always be assumed to have the above property on the metric tensor.

Further we will apply the notion of width only to compact Riemannian polyhedrons.
If $\spc{P}$ is an $n$-dimensional Riemannian polyhedron, then 
we suppose that
\[\width\spc{P}\df\width_{n-1}\spc{P}.\]


Suppose that $\spc{P}$ is an $n$-dimensional Riemannian polyhedron;
in this case we will use short cut $\vol$ for $\vol_n$.
Let us define \index{volume profile}\emph{volume profile} of $\spc{P}$ as a function 
returning largest volume of $r$-ball in~$\spc{P}$;
that is, the volume profile of $\spc{P}$ is a function $\VolPro_{\spc{P}}\:\RR_+\to\RR_+$ defined by 
\[\VolPro_{\spc{P}}(r)\df \sup\set{\vol \oBall(p,r)}{p\in\spc{P}}.\]
Note that 
$r\mapsto \VolPro_{\spc{P}}(r)$ is nondecreasing  and
\[\VolPro_{\spc{P}}(r)\le\vol\spc{P}\]
for any $r$.
Moreover, if $\spc{P}$ is connected, then the equality $\VolPro_{\spc{P}}(r)=\vol\spc{P}$ holds
for $r\ge \rad \spc{P}$.

Note that if $\spc{P}$ is a connected 1-dimensional Riemannian polyhedron, then 
\[\width\spc{P}=\width_0\spc{P}=\rad\spc{P}.\]

\begin{thm}{Exercise}\label{ex:1D-case}
Let $\spc{P}$ be a 1-dimensional Riemannian polyhedron.
Suppose that $\VolPro_{\spc{P}}(R)<R$ for some $R>0$.
Show that 
\[\width \spc{P}<R.\]
Try to show that $c=\tfrac 12$ is the optimal constant for which the following inequality holds: 
\[\width \spc{P}<c\cdot R.\]
\end{thm}

\section{Volume profile bounds width}

\begin{thm}{Theorem}\label{thm:width<volpro}
Let $\spc{P}$ be an $n$-dimensional Riemannian polyhedron. 
If the inequality 
\[R> n\cdot \sqrt[n]{\VolPro_{\spc{P}}(R)}\]
holds for {}\emph{some} $R>0$, then 
\[\width\spc{P}\le  R.\]
\end{thm}

Since $\VolPro_{\spc{P}}(R)\le \vol\spc{P}$ for any $R>0$,
we get the following:

\begin{thm}{Corollary}\label{thm:width<vol}
For any $n$-dimensional Riemannian polyhedron $\spc{P}$, we have
\[\width\spc{P}\le n\cdot \sqrt[n]{\vol\spc{P}}.\]

\end{thm}

The proof of \ref{thm:width<volpro} will be given at the very end of this section,
after discussing {}\emph{separating polyhedrons}. 

Let us start three technical statements.
The first statement can be obtained by modifying a smoothing procedure for functions defined on Euclidean space. 

A function $f$ defined on a Riemannian polyhedron $\spc{P}$ is called \index{piecewise smooth}\emph{piecewise smooth} if there is a triangulation of $\spc{P}$ such that restriction of $f$ to every simplex is smooth.


\begin{thm}{Smoothing procedure}\label{smoothing-procedure}
Let $\spc{P}$ be a Riemannian polyhedron and $f\:\spc{P}\to \RR$ be a 1-Lipschitz function.
Then for any $\delta>0$ there is a piecewise smooth 1-Lipschitz function $\tilde f\:\spc{P}\to \RR$ such that 
\[|\tilde f(x)-f(x)|<\delta\]
for any $x\in  \spc{P}$.
\end{thm}

The following statement can be proved by applying the classical Sard's theorem to each simplex of a Riemannian polyhedron.

\begin{thm}{Sard's theorem}\label{sard}\index{Sard's theorem}
Let $\spc{P}$ be an $n$-dimensional Riemannian polyhedron and $f\:\spc{P}\to \RR$ be a piecewise smooth function.
Then for almost all values $a\in\RR$, the inverse image $f^{-1}\{a\}$  is a Riemannian polyhedron of dimension at most $n-1$ (we assume that $f^{-1}\{a\}$ is equipped with the induced length metric).
\end{thm}

The following statement can be proved by applying the coarea inequality (\ref{cor:coarea}) to the restriction of $f$ to each simplex of the polyhedron and summing up the results.

\begin{thm}{Coarea inequality}\index{coarea inequality}\label{poly-coarea}
Let $\spc{P}$ be an $n$-dimensional Riemannian polyhedron and $f\:\spc{P}\to \RR$ be a piecewise smooth 1-Lipschitz function.
Set $v\z=\vol_n (f^{-1}[r,R])$ and $a(t)=\vol_{n-1}(f^{-1}\{t\})$.
Then 
\[\int_r^Ra(t)\cdot dt\ge v .\]
In particular there is a subset of positive measure $S\subset [r,R]$ such that the inequality 
\[a(t)\ge \frac v{R-r}\]
holds for any $t\in S$.
\end{thm}

\section*{Separating subpolyhedrons}

\begin{thm}{Definition}
Let $\spc{P}$ be an $n$-dimensional Riemannian polyhedron.
An $(n-1)$-dimensional subpolyhedron $\spc{Q}\subset\spc{P}$ is called \index{separating subpolyhedron}\emph{$R$-separating} if for each connected component $U$ of the complement $\spc{P}\setminus \spc{Q}$ we have 
\[\rad U<R.\]

\end{thm}



\begin{thm}{Lemma}\label{lem:separating}
Let $\spc{P}$ be an $n$-dimensional Riemannian polyhedron.
Then given $R>0$ and $\eps>0$ there is a $R$-separating subpolyhedron $\spc{Q}\subset\spc{P}$ such that for any $r_0<r_1\le R$ we have
\[\VolPro_{\spc{Q}}(r_0)< \tfrac1{r_1-r_0}\cdot \VolPro_{\spc{P}}(r_1)+\eps.\]

\end{thm}

The proof reminds the proof of the following statement about minimal surfaces: 
\textit{if a point $p$ lies on an compact area-minimizing surface $\Sigma$ and $\partial\Sigma \cap \oBall(p,r)=\emptyset$, then
\[\area(\Sigma\cap \oBall(p,r))\le \tfrac12\cdot \area\mathbb{S}^2\cdot r^2.\]
}


\parit{Proof.}
Choose a small $\delta>0$.
Applying the smoothing procedure (\ref{smoothing-procedure}), we can exchange each distance function $\distfun_p$ on $\spc{P}$ by $\delta$-close piecewise smooth 1-Lipschitz function, which will be denoted by $\widetilde \distfun_p$.

By Sard's theorem (\ref{sard}), for almost all values $c\z\in(r_0\z+\delta, r_1-\delta)$, the level set
\[\tilde S_c(p)=\set{x\in \spc{P}}{\widetilde \distfun_p(x)=c}\]
is a Riemannian polyhedron of dimension at most $n-1$.
Since $\delta$ is small, the coarea inequality (\ref{poly-coarea}) implies that $c$ can be chosen so that in addition the following inequality holds:
\begin{align*}
\vol_{n-1}\tilde S_c(p)&\le \tfrac1{r_1-r_0-2\cdot\delta}\cdot\vol_n[\oBall(p,r_1)]<
\\
&<\tfrac1{r_1-r_0}\cdot \VolPro_{\spc{P}}(r_1)+\tfrac\eps2.
\end{align*}

Suppose $\spc{Q}$ is an $R$-separating subpolyhedron in $\spc{P}$ with almost minimal volume;
say its volume is at most $\tfrac\eps2$-far from the greatest lower bound.
Note that cutting from $\spc{Q}$ everything inside $\tilde S_c(p)$ and adding $\tilde S_c(p)$ produces a $R$-separating subpolyhedron, say $\spc{Q}'$.%
\footnote{If $\dim\tilde S_c(p)<n-1$, then it might happen that $\dim\spc{Q}'<n-1$; so, by the definition, $\spc{Q}'$ is not separating.
It can be fixed by adding a tiny $(n-1)$-dimensional piece to $\spc{Q}'$.}

Since $\spc{Q}$ has almost minimal volume, we have
\[\vol_{n-1}[\spc{Q}\cap \oBall(p,r_0)_{\spc{P}}]-\tfrac\eps2\le \vol_{n-1}S_c(p).\]
Therefore 
\[\vol_{n-1}[\spc{Q}\cap \oBall(p,r_0)_{\spc{P}}]\le\tfrac1{r_1-r_0}\cdot \VolPro_{\spc{P}}(r_1)+\eps.
\eqlbl{eq:volQ<ProP}\]
Recall that $\spc{Q}$ is equipped with the induced length metric;
therefore $\dist{p}{q}{\spc{Q}}\ge \dist{p}{q}{\spc{P}}$ for any $p,q\in \spc{Q}$;
in particular, 
\[\oBall(p,r_0)_{\spc{Q}}\subset \spc{Q}\cap \oBall(p,r_0)_{\spc{P}}\]
for any $p\in \spc{Q}$ and $r_0\ge 0$.
Hence, \ref{eq:volQ<ProP} implies the lemma.
\qeds

\begin{thm}{Lemma}\label{lem:separating-width}
Let $\spc{Q}$ be an $R$-separating subpolyhedron in an $n$-dimensional Riemannian polyhedron $\spc{P}$.
Then 
\[\width\spc{Q}\le R
\quad\Longrightarrow\quad
\width\spc{P}\le R.\]
\end{thm}

\parit{Proof.}
Choose an open covering $\{V_1,\dots,V_k\}$ of $\spc{Q}$ as in the definition of width (\ref{def:width});
that is, it has multiplicity at most $n$ and $\rad V_i<R$ for any $i$. 

Note that $\{V_1,\dots,V_k\}$ can be converted into an open covering of
a small neighbourhood of $\spc{Q}$ in $\spc{P}$ without increasing the multiplicity.
This can be done by setting 
\[V_i'=\bigcup_{x\in V_i}\oBall(x,r_x),\]
where $r_x\df\tfrac1{10}\cdot\inf\set{\dist{x}{y}{}}{y\in \spc{Q}\setminus V_i}$.

By adding to  $\{V_i'\}$ all the components of $\spc{P}\setminus \spc{Q}$,
we increase the multiplicity by at most 1 and obtain a covering of $\spc{P}$.
The statement follows since $\dim \spc{P}= \dim \spc{Q}\z+1$.
\qeds

\section*{Proof assembling}

\parit{Proof of \ref{thm:width<volpro}.}
We apply induction on the dimension $n=\dim\spc{P}$.
The base case $n=1$ is given in \ref{ex:1D-case}.

Suppose that the  $(n-1)$-dimensional case is proved.
Consider an $n$-dimensional Riemannian polyhedron $\spc{P}$ and suppose
\[n\cdot \sqrt[n]{\VolPro\spc{P}(R)}< R\]
for some $R>0$.
Let $\spc{Q}$ be an $R$-separating subpolyhedron in $\spc{P}$ provided by \ref{lem:separating} for a small $\eps>0$.

Applying  \ref{lem:separating} for $r=\tfrac{n-1}n\cdot R$ and $R$, we have that 
\begin{align*}
\VolPro_\spc{Q}(r) &< \frac 1{R-r}\cdot \VolPro_\spc{P}(R)+\eps<
\\
&<\frac {n}{R}\cdot\left(\frac{R}{n}\right)^n=
\\
&=\left(\frac{r}{n-1}\right)^{n-1};
\end{align*}
that is, $(n-1)\cdot \sqrt[n-1]{\VolPro\spc{Q}(r)}< r$.
Since $\dim\spc{Q}= n-1$, by the induction hypothesis, we get that
\[\width\spc{Q}\le r<R.\]
It remains to apply \ref{lem:separating-width}.
\qeds





\section{Width bounds systole}

Recall that a topological space $K$ is called \index{aspherical space}\emph{aspherical} if any continuous map $\mathbb{S}^k\to K$ for $k\ge 2$ is null-homotopic.

\begin{thm}{Theorem}\label{thm:sys<width}
Suppose $\spc{M}$ is a compact aspherical $n$-dimensional Riemannian manifold.
Then 
\[\sys\spc{M}\le 6 \cdot \width \spc{M}.\]
\end{thm}

\begin{thm}{Lemma}\label{lem:aspherical-homotopy}
Let $K$ be an aspherical space and $\spc{W}$ a connected CW-complex.
Denote by $\spc{W}^k$ the k-skeleton of $\spc{W}$.
Then any continuous map $f\:\spc{W}^2\to K$ can be extended to a continuous map $\bar f\:\spc{W}\to K$

Moreover, if $p\in \spc{W}$ is a 0-cell and $q\in K$.
Then a continuous maps of pairs $\phi_0,\phi_1\:(\spc{W},p)\to(K,q)$ are homotopic if and only if $\phi_0$ and $\phi_1$ induce the same homomorphism on fundamental groups $\pi_1(\spc{W},p)\to\pi_1(K,q)$.
\end{thm}

\parit{Proof.}
Since $K$ is aspherical, any continuous map $\partial\mathbb{D}^n\to K$ for $n\ge 3$
is hull-homotopic;
that is, it can be extended to a map $\mathbb{D}^n\:\to K$.

It makes it possible to extend $f$ to $\spc{W}^3$, $\spc{W}^4$, and so on.
Therefore $f$ can be extended to whole $\spc{W}$.

The only-if part of the second part of lemma is trivial;
it remains to show the if part.

Sine $\spc{W}$ is connected, we can assume that $p$ forms the only 0-cell in $\spc{W}$;
otherwise we can collapse a maximal subtree of the 1-skeleton in $\spc{W}$ to $p$.
Therefore, $\spc{W}^1$ is formed by loops that generate $\pi_1(\spc{W},p)$.

By assumption, the restrictions of $\phi_0$ and $\phi_1$ to $\spc{W}^1$ are homotopic.
In other words the homotopy $\Phi\:[0,1]\times \spc{W}$ is defined on the 2-skeleton of $[0,1]\times \spc{W}$.
It remains to apply the first part of the lemma to the product $[0,1]\times \spc{W}$.
\qeds



\begin{thm}{Lemma}\label{lem:sys-homotopy}
Suppose $\gamma_0,\gamma_1$ are two paths between points in a Riemannian space $\spc{M}$ such that $\dist{\gamma_0(t)}{\gamma_1(t)}{\spc{M}}<r$ for any $t\in[0,1]$.
Let $\alpha$ be a shortest path from $\gamma_0(0)$ to $\gamma_1(0)$ and $\beta$ be a shortest path from $\gamma_0(1)$ to $\gamma_1(1)$. 
If $2\cdot r<\sys\spc{M}$, then there is a homotopy $\gamma_t$ from
$\gamma_0$ to $\gamma_1$ such that $\alpha(t)\equiv \gamma_t(0)$ and $\beta(t)\equiv \gamma_t(1)$.
\end{thm}

\parit{Proof.}
Set $s=\sys\spc{M}$; 
since $2\cdot r<s$, we have that $\eps=\tfrac1{10}(s-2\cdot r)>0$.

\begin{wrapfigure}{o}{34mm}
\vskip-0mm
\centering
\includegraphics{mppics/pic-1405}
\end{wrapfigure}

Note that we can assume that $\gamma_0$ and $\gamma_1$ are rectifiable;
if not we can homotopy each into a broken geodesic line kipping the assumptions true. 

Choose a fine partition $0\z=t_0\z<t_1\z<\z\dots\z<t_n=1$.
Consider a sequence of shortest paths $\alpha_i$ from $\gamma_0(t_i)$ to $\gamma_1(t_i)$.
We can assume that $\alpha_0=\alpha$, $\alpha_n=\beta$, and each arc $\gamma_j|_{[t_{i-1},t_i]}$ has length smaller than $\eps$.
Therefore, every quadrilateral formed by concatenation  of $\alpha_{i-1}$, $\gamma_1|_{[t_{i-1},t_i]}$, reversed $\alpha_i$, and reversed arc $\gamma_0|_{[t_{i-1},t_i]}$ has length smaller than $s$.
It follows that this curve is contractible.
Applying this observation for each quadrilateral, we get the statement.
\qeds


\parit{Proof of \ref{thm:sys<width}.}
Let $\spc{N}$ be the nerve of a covering $\{V_i\}$ of $\spc{M}$ and $\bm{\psi}\:\spc{M}\to\spc{N}$ be the map provided by \ref{prop:space->nerve}.
As usual, we denote by $v_i$ the vertex of $\spc{N}$ that corresponds to $V_i$.
Observe that $\dim\spc{N}<n$;
therefore, $\bm{\psi}$ kills the fundamental class of $\spc{M}$.

Let us construct a continuous map  $f\:\spc{N}\to  \spc{M}$ such that
$f\circ\bm{\psi}$ is homotopic to the identity map on $\spc{M}$.
Note that once $f$ is constructed, the theorem is proved.
Indeed, since $\bm{\psi}$ kills the fundamental class $[\spc{M}]$ of $\spc{M}$, so does $f\circ\bm{\psi}$.
Therefore, $[\spc{M}]=0$ --- a contradiction.

Set $R=\width \spc{M}$ and $s=\sys\spc{M}$.
Assume we choose $\{V_i\}$ as in the definition of width (\ref{def:width}).
For each $i$ choose a point $p_i\in \spc{M}$ such that $V_i\subset \oBall(p_i,R)$.

Set $f(v_i)=p_i$ for each $i$.
It defines the map $f$ on the 0-skeleton $\spc{N}^0$ of the nerve $\spc{N}$.
Further, $f$ will be defined step by step on the skeletons $\spc{N}^1,\spc{N}^2, \dots$ of $\spc{N}$.

Let us map each edge $[v_iv_j]$ in $\spc{N}$ to a shortest path $[p_ip_j]$.
It defines $f$ on $\spc{N}^1$.
Note that image of each edge is shorter than $2\cdot R$.

Suppose $[v_iv_jv_k]$ is a triangle in $\spc{N}$.
Note that perimeter of the triangle $[p_ip_jp_k]$ can not exceed $6\cdot R$.
Since $6\cdot R<s$, the contour of $[p_ip_jp_k]$ is contractible.
Therefore, we can extend $f$ to each triangle of~$\spc{N}$.
It defines the map $f$ on $\spc{N}^2$.

Finally, since $\spc{M}$ is aspherical, by \ref{lem:aspherical-homotopy}, the map $f$ can be extended to $\spc{N}^3$, $\spc{N}^4$ and so on.

It remains to show that $f\circ\bm{\psi}$ is homotopic to the identity map.
Choose a CW structure on $\spc{M}$ with sufficiently small cells, so that each cell lies in one of $V_i$.
Note that $\bm{\psi}$ is homotopic to a map $\bm{\psi}_1$ that sends $\spc{M}^k$ to $\spc{N}^k$ for any $k$.
Moreover, we may assume that (1) if a 0-cell $x$ of $\spc{M}$ maps to a $v_i$, then $x\in V_i$ and (2) each 1-cell  of $\spc{M}$ maps to an edge or a vertex of $\spc{N}$.
Choose a 1-cell $e$ in $\spc{M}$; by the construction, $f\circ\bm{\psi}_1$ maps $e$ to a shortest path $[p_ip_j]$ and $e$ lies $\oBall(p_i,R)$.
Observe that $[p_ip_j]$ is shorter than $2\cdot R$.
It follows that the distance between points on $[p_ip_j]$ and $e$ can not exceed $3\cdot R$.
Choose a shortest path $\alpha_i$ from every 0 cell $x_i$  of $\spc{M}$ to $p_j=f\circ\bm{\psi}_1(x_i)$.
It defines a homotopy on $\spc{M}^0$.
Since $6\cdot R<s$, \ref{lem:sys-homotopy} implies that this homotopy can be extended to $\spc{M}^1$.
By \ref{lem:aspherical-homotopy}, it can be extended to whole $\spc{M}$.
\qeds

\begin{thm}{Exercise}\label{ex:sys<width}
Analyze the proof of \ref{thm:sys<width} and improve its inequality to 
 \[\sys\spc{M}\le 4 \cdot \width \spc{M}.\]
\end{thm}

\begin{thm}{Exercise}\label{ex:fillrad-inj}
Modify the proof of \ref{thm:sys<width} to prove the following:

Suppose that $\spc{M}$ is a closed $n$-dimensional Riemannian manifold with \index{injectivity radius}\emph{injectivity radius} at least $r$; that is, if $\dist{p}{q}{\spc{M}}<r$, then a shortest path $[pq]_{\spc{M}}$ is uniquely defined.
Show that
\[\width\spc{M}\ge \tfrac{r}{2\cdot(n+1)}.\]

Use \ref{thm:width<vol} to conclude that $\vol\spc{M}\ge \eps_n \cdot r^n$
for some $\eps_n>0$ that depends only on $n$.
\end{thm} 

The second statement in the exercise is a theorem of Marcel Berger~\cite{berger-n};
an inequality with optimal constant (with equality for round sphere) was obtained by Marcel Berger and Jerry Kazdan \cite{berger-kazdan}. 


\section{Essential manifolds}

To generalize \ref{thm:sys<width} further, we need the following definition.

\begin{thm}{Definition}\label{def:essential}
A closed manifold $M$ is called \index{essential manifold}\emph{essential} if it admits a continuous map $\iota\:M\to K$ to an aspherical CW-complex $K$ such that $\iota$ sends the fundamental class of $M$ to a nonzero homology class in $K$.
\end{thm}

Note that any closed aspherical manifold is essential --- in this case one can take $\iota$ to be the identity map on $M$.

The real projective space $\RP^n$ provides an interesting example of an essential manifold which is not aspherical.
Indeed, the infinite dimensional projective space $\RP^\infty$ is aspherical and for the natural embedding $\RP^n\hookrightarrow\RP^\infty$ the image $\RP^n$ does not bound in $\RP^\infty$.
The following exercise provides more examples of that type:

\begin{thm}{Exercise}\label{ex:connected-sum-essential}
Show that the connected sum of an essential manifold with any closed manifold is essential.
\end{thm}

\begin{thm}{Exercise}\label{ex:product-essential}
Show that the product of two essential manifolds is essential.
\end{thm}

Assume that the manifold $M$ is essential and $\iota \:M\to K$ as in the definition.
Following the proof of \ref{thm:sys<width}, we can homotope the map 
$f\circ\bm{\psi}\:M\to M$ to the identity on the 2-skeleton of $M$;
further since $K$ is aspherical, we can homotope the composition
$\iota\z\circ f\circ\bm{\psi}$ to  $\iota$. 
Existence of this extension implies that $\iota$ kills the fundamental class of $M$ --- a contradiction.
So, taking \ref{ex:sys<width} into account, we proved the following generalization of \ref{thm:sys<width}:

\begin{thm}{Theorem}\label{thm:sys<width++}
Suppose $\spc{M}$ is an essential Riemannian space.
Then 
\[\sys\spc{M}\le 4 \cdot \width \spc{M}.\]
\end{thm}

As a corollary from \ref{thm:sys<width++} and \ref{thm:width<vol} we get the so-called \index{systolic inequality!Gromov's systolic inequality}\emph{Gromov's systolic inequality}:

\begin{thm}{Theorem}\label{thm:sys+}
Suppose $\spc{M}$ is an essential $n$-dimensional Riemannian space.
Then 
\[\sys\spc{M}\le 4 \cdot n\cdot \sqrt[n]{\vol\spc{M}}.\]
\end{thm}


\section{Remarks}

Theorem \ref{thm:sys+} was proved originally by Mikhael Gromov \cite{gromov-1983} with a worse constant.
The given proof is a result of a sequence of simplifications given by Larry Guth \cite{guth},
Panos Papasoglu \cite{papasoglu},
Alexander Nabutovsky and Roman Karasev \cite{nabutovsky}.

The calculations could be done better; namely we could get
\[\width\spc{P}\le c_n\cdot \sqrt[n]{\vol\spc{P}},\]
where
$c_n=\sqrt[n]{n!/2}= \tfrac ne+o(n)$ \cite{nabutovsky}.
As a result, we may get a stronger statement in \ref{thm:sys+}:
\[\sys\spc{M}\le 4 \cdot c_n\cdot \sqrt[n]{\vol\spc{M}}.\]

For any nonessential oriented manifold $M$ there is a metric with fixed volume and arbitrary small systole.
This statement is proved by Ivan Babenko \cite{babenko}.

A wide open conjecture says that for any $n$-dimensional essential manifold we have
\[\frac{\sys\spc{M}}{\sqrt[n]{\vol\spc{M}}}\le\frac{\sys\RP^n}{\sqrt[n]{\vol\RP^n}},\eqlbl{eq:RPn}\]
where we assume that the $n$-dimensional real projective space $\RP^n$ is equipped with a canonical metric.
In other words, the ratio in the right-hand side of \ref{eq:RPn} is the optimal constant in the Gromov's systolic inequality; this  ratio grows as $O(\sqrt n)$.
(The ratio for $n$-dimensional flat torus grows as $O(\sqrt n)$ as well.)
