\chapter{Volume bounds} 


\section{Riemannian metrics}

We are going to consider mostly Riemannian spaces;
that is smooth manifolds with metric defined by a metric tensor.
These are specially nice length metrics on manifolds.
However most of the statements we are going to discuss have counterpart for general length metrics on manifolds.

Let $M$ be a smooth manifold.
A \emph{metric tensor} on $M$ is a choice of positive definite quadratic forms $g_p$ on each tangent space $\T_pM$ that depends smoothly on the point $p$.
That is, if we fix a local coordinates on $M$ and write $g$ in this coordinates, then each component of $g$ is a smooth function. 

A Riemannian manifold is a smooth manifold $M$ with a choice \emph{metric tensor} $g$ on it.

The metric tensor $g$ can be used to define length of curves and volume of regions in $M$.

\parbf{Lengths and distances.}
If $\gamma\:[a,b]\to M$ is a piecewise smooth curve then 
\[\length_g\gamma=\int_a^b\sqrt{g(\gamma'(t),\gamma'(t))}\cdot dt.\]
Further we can define a metric on $M$ as least lower bound to lengths of piecewise smooth curves connecting two given points;
the described distance between points $x$ and $y$ will be denoted by $\dist{x}{y}{g}$ or $\distfun_x(y)_g$.
The distance function from a point $x$ will be denoted by $(\distfun_x)_g$ or $\distfun_x$ if the choice of $g$ is evident.

The following claim states that the constructed metric remembers everything about the metric tensor and the underlying smooth manifold;
this claim requires a proof, but we will assume that it is obvious.

\begin{thm}{Claim}
Let $(M,g)$ be a Riemannian manifold.
Then the metric $(x,y)\mapsto \dist{x}{y}{g}$ defines a length metric. Moreover this metric completely determines the metric tensor $g$.
\end{thm}

\parbf{Volume.}
If a region $R$ is covered by one chart $\iota\:U\to M$,
then its volume can be defined as an integral 
\[\vol R
\df
\int_{\iota^{-1}(R)}\sqrt{\det{g}}.\]
In the general case we subdivide $R$ into (a countable collection of) regions $R_1,R_2\dots$ and define
\[\vol R\df \vol R_1+\vol R_2+\dots\]

\section{Besikovitch inequality}

\begin{thm}{Theorem}\label{thm:besikovitch}
Let $g$ be a metric tensor on a unit $n$-dimensional cube $\square^n$.
Suppose that the $g$-distances between the opposite faces of $\square^n$ are at leat $1$; that is, any piecewise smooth curve that connects opposite faces has $g$-length at least $1$.
Then $\vol(\square^n, g)\ge 1$.
\end{thm}

\parit{Proof.}
We will consider the case $n=2$; the other cases are proved the same way.

Denote by $A$, $A'$, and $B$, $B'$ the opposite faces of the square~$\square$.
Consider two function 
\begin{align*}
f_A(x)&\df\min\{\,\distfun_A(x)_g,1\,\},
\\
f_B(x)&\df\min\{\,\distfun_B(x)_g,1\,\}.
\end{align*}
Define $f\:\square\to\square$ as a map with coordinate funcions $f_A$ and $f_B$;
that is, $f(x)\df(f_A(x), f_B(x))$.

Observe that $f$ maps each face to itself.
Indeed, 
\[x\in A \quad\Longrightarrow\quad \distfun_A(x)_g=0 \quad\Longrightarrow\quad f_A(x)=0 \quad\Longrightarrow\quad f(x)\in A.\]
Similarly if $x\in B$, then $f(x)\in B$.
Further, 
\[x\in A'
\quad\Longrightarrow\quad 
\distfun_A(x)_g\ge 1 
\quad\Longrightarrow\quad 
f_A(x)=1 
\quad\Longrightarrow\quad 
f(x)\in A'.\]
Similarly if $x\in B'$, then $f(x)\in B'$.

Therefore 
\[f_t(x)= t\cdot x + (1-t)\cdot f(x)\]
defines a homotopy of maps of pair of spaces $(\square,\partial \square)$ from $f$ to the identity map.
It follows that degree of $f$ is $1$; that is, $f$ sends the fundamental class of $(\square,\partial \square)$ to itself.
In particular $f$ is onto.

Suppose that Jacobian  matrix $\Jac_pf$ of $f$ is defined at $p\in \square$.
Choose an orthonormal basis in $\T_p$ with respect to $g$ and the standard basis in the target $\square$.
Observe that the differentials $d_pf_A$ and $d_pf_B$ written in these basises are the rows of $\Jac_pf$.
Evidently $|d_pf_A|\le 1$ and $|d_pf_B|\le 1$.
Since the determinant of a matrix is the volume of the parallelepiped spanned on its rows, we get 
\[|\det(\Jac_pf)|\le |d_pf_A|\cdot|d_pf_B|\le 1.\]
Since $f\:\square\to\square$ is a Lipschitz onto map, the \emph{area formula} implies that 
\[\vol(\square,g)\ge \vol\square=1.\]
\qedsf

The following generalization can be proved along the same lines.

\begin{thm}{Theorem}\label{thm:besikovitch+}
Let $(M,g)$ be Riemannian manifold and its boundary admits a degree 1 map $\partial M\to\partial\square^n$. 
Suppose $d_1,\dots, d_n$ the distances between the the inverse images of pairs of opposite faces of $\square^n$ in $\partial M$.
Then 
\[\vol(M,g)\ge d_1\cdots d_n.\]
\end{thm}

\begin{thm}{Exercise}\label{ex:besikovitch=}
Suppose that we have equality in \ref{thm:besikovitch}.
Show that $(\square^n,g)$ is isometric to $\square^n$.
\end{thm}

\begin{thm}{Exercise}\label{ex:hexagon}
Suppose $g$ is a metric tensor on a regular hexagon $\varhexagon
   $ such that $g$-distances between the opposite sides are at least $1$.
Is there a positive lower bound on $\area(\varhexagon,g)$?
\end{thm}

\begin{thm}{Exercise}\label{ex:gadograph}
Let $V$ be a compact set in $\EE^d$ bounded by a hypersurface $\Sigma$.
Suppose $g$ is a Riemannian metric on $V$ such that 
\[\dist{p}{q}{g}\ge\dist{p}{q}{\EE^d}\]
for any two points $p,q\in \Sigma$.
Show that
\[\vol(V,g)\ge \vol(V)_{\EE^d}.\]
 
\end{thm}

\begin{thm}{Exercise}\label{ex:involution-of-sphere}
Suppose that sphere with Riemannian matric $(\mathbb{S}^2,g)$ admits an involution $\iota$ such that $\dist{x}{\iota(x)}{g}\ge 1$.

Show that $\area(\mathbb{S}^2,g)\ge \tfrac1{1000}$;
try to show that $\area(\mathbb{S}^2,g)\ge \tfrac12$ or $\area(\mathbb{S}^2,g)\ge 1$.
\end{thm}

Christopher Croke conjectured that the optimal bound for this exercise is $\tfrac4\pi$ and the round sphere is the only space that achieves this \cite[see Conjecture 0.3 in][]{croke}.

\begin{thm}{Advanced exercise}\label{ex:involution-of-3sphere}
Construct a metric $g$ on $\mathbb{S}^3$ with arbitrary small $\vol(\mathbb{S}^3,g)$ and such that it admits an involution $\iota$ such that $\dist{x}{\iota(x)}{g}\ge 1$.
\end{thm}

\section{Systolic inequlaity}

Let $\spc{M}$ be a compact Riemannian manifold.
The \emph{systole} of $\spc{M}$ (brifly $\sys\spc{M}$) is defined to be the least length of a noncontractible closed curve in $\spc{M}$.

Let $\Lambda$ be a set of smooth closed $n$-dimensional manifolds.
We say that a systolic inequality holds for $\Lambda$ if there is a constant $c$ such that for any $M\in \Lambda$ and any metric tenor $g$ on $M$ we have
\[[\sys(M,g)]^n\le c\cdot \vol(M,g).\]

\begin{thm}{Exercise}\label{ex:sysT2}
Use \ref{thm:besikovitch} to show that systolic inequality holds for the 2-torus $\TT^2$.
\end{thm}

\begin{thm}{Exercise}\label{ex:sysRP2}
Use \ref{thm:besikovitch} to show that systolic inequality holds for the real projective palane $\RP^2$.
\end{thm}

\begin{thm}{Exercise}\label{ex:sysSg}
Use \ref{thm:besikovitch+} to show that systolic inequality holds for the set of all closed surfaces of positive genus.
\end{thm}

\parbf{Remarks.}
The optimal constants in the systolic inequality are known in the following three cases:
\begin{itemize}
\item For real projective plane $\RP^2$ the constant is $\tfrac\pi2$ --- the equality holds for a quotient of a round sphere by isometric involution. The statement was prove by Pao Ming Pu \cite{pu}.\label{page:pu}
\item For torus $\TT^2$ the constant is $\tfrac2{\sqrt{3}}$ --- the equality holds for a flat torus obtained from a regular hexagon by identifying opposite sides; this is the so called \emph{Loewner's torus inequality}.
\item For the Klein bottle $\RP^2\#\RP^2$  the constant is $\tfrac\pi{2\cdot\sqrt2}$ --- the equality holds for certain nonsmooth metrics \cite{bavard}.
\end{itemize}
The proofs of these results use the so called \emph{uniformization theorem}   available in the 2-dimensional case only.
These proofs are beautiful, but they too far from metric geometry.
A good survey on the subject is written by Christopher Croke and Mikhail Katz \cite{croke-katz}.

\begin{thm}{Exercise}\label{ex:sysS2xS1}
Show that systolic inequality does \emph{not} hold for $\mathbb{S}^2\times\mathbb{S}^1$.
\end{thm}


\begin{thm}{Therorem}\label{thm:sys(torus)}
Systolic intequality holds for the $n$-dimensional torus $\TT^n$. 
\end{thm}

The proof of this theorem and its generalization will take most of the remaining lectures.
In the following section we introduce a key notion in the proof.

\section{Filling radius}

The following definition was introduced by Mikhael Gromov \cite{gromov-1983}.

Let $\spc{M}$ be a closed $n$-dimensional Reimannian manifold.
Applying Kuratowski embedding (\ref{lem:kuratowski}) $x\mapsto \distfun_x$, we may think that $\spc{M}$ as a subset of $\ell^\infty(\spc{M})$ --- the space of functions on $\spc{M}$ equipped with the metric induced by the sup-norm.

Define the \emph{filling radius} of $\spc{M}$ (briefly $\FillRad\spc{M}$) as the least upper bound on values $r>0$ such that $\spc{M}$ bounds in its $r$-neighborhood in $\ell^\infty(\spc{M})$.
In other words, if $\iota_r$ denotes inclusion of $\spc{M}$ in its $r$-neighborhood $B_r(\spc{M})\subset \ell^\infty(\spc{M})$,
then 
\[\FillRad\spc{M}\df\inf\set{r>0}{(\iota_r)_*[\spc{M}]=0\in H_n(B_r(\spc{M}))},\]
where $[\spc{M}]$ denotes the fundamental class of $\spc{M}$.

We assume that the homologies are taken with coefficients in $\ZZ_2$.
In this case $[\spc{M}]\ne0\in H_n(\spc{M})$.
If we choose coefficients $\ZZ$, then it does not hold for nonorientable manifolds.


\begin{thm}{Exercise}\label{ex:fillrad<diam/2}
Show that the inequality
\[\FillRad \spc{M}\le \tfrac12\cdot\diam \spc{M}\]
holds for any compact Riemannian manifold $\spc{M}$.
\end{thm}

\parbf{Remark.}
The optimal bound for the above exercise was found by Mikhail Katz \cite{katz}.
Namely he proved that
\[\FillRad \spc{M}\le \tfrac13\cdot\diam \spc{M}\]
and equality holds if $\spc{M}$ is real projective space with canonical metric.
The proof is beautiful, elementary, and very readable.

\medskip

The following theorem is the main ingredient in the proof of \ref{thm:sys(torus)}.
This theorem will be the main subject of the following lecture.

\begin{thm}{Theorem}\label{thm:FillRad<vol}
Given an integer $n>0$, there is a constant $c(n)$ such that inequality
\[(\FillRad \spc{M})^n\le c(n)\cdot \vol \spc{M}\]
holds for any compact $n$-dimensional Riemannian manifold $\spc{M}$.
\end{thm}

In the following section we show why this theorem is related to \ref{thm:FillRad<vol}.

\section{Filling radius bounds systole}

\begin{thm}{Theorem}\label{thm:sys<FillRad}
Suppose $\spc{T}= (\TT^n,g)$ is a Riemnnian manifold on $n$-dimensional torus $\TT^n$.
Then 
\[\sys\spc{T}\le 6 \cdot \FillRad \spc{T}.\]
\end{thm}

Note that \ref{thm:sys<FillRad} and \ref{thm:FillRad<vol}  imply \ref{thm:sys(torus)}.

\parit{Proof.}
As usual we consider $\spc{T}$ as a subspace in $\ell^\infty(\spc{T})$.

Set $s=\sys\spc{T}$ and choose $R>\FillRad\spc{T}$.
Arguing by contradiction, assume $6\cdot R< s$;
so $\eps=\tfrac1{100}\cdot(s-6\cdot R)>0$.

Choose a simplicial complex $W$ and a map $\sigma\:W\to \ell^\infty(\spc{T})$ such that the restriction $\sigma|_{\partial W}$
represents the fundamental class $[\spc{T}]$ of $\spc{T}$
and $\sigma(W)\subset B_{R}(\spc{T})$.


Passing to barycentric subdivision few times, we may assume that the $\sigma$-image of any simplex in $W$ has diameter less than $\eps$.
We may perturb the map slightly to ensure that each edge $e$ of $W$ is mapped to a geodesic and still $\sigma|_{\partial W}$
represents the fundamental class $[\spc{T}]$ of $\spc{T}$.

Let us construct a continuous map
$f\:W\to  \spc{T}$ which agrees with $\sigma$ on $\partial W$.
Once it is done we get that the fundamental class of $\spc{T}$ vanish in $ H_n(\spc{T})$ --- a contradiction.

Set $f(x)=\sigma(x)$ for every $x\in \partial W$;
on the remaining part of $W$ we will construct $f$ recurcevely on the skeletons $W^0$, $W^1$, $W^2$ and so on.

For every vertex $v$, set $f(v)$ to be the closest point in $\spc{T}$ to $\sigma(v)$.
Note that if $v\in\partial W$, then $f(v)=\sigma(v)$.
This way we defined $f$ on~$W^0$.

Let $e$ be an edge in $W$ between vertexes $v$ and $w$.
Note that 
\begin{align*}
\dist{f(v)}{f(w)}{}
&\le\dist{f(v)}{\sigma(v)}{}
+\dist{\sigma(v)}{\sigma(w)}{}
+\dist{\sigma(w)}{f(w)}{}\le
\\
&\le R+\eps +R<
\\
&<\tfrac s3.
\end{align*}
Map $e$ to a shortest path $[f(v)\,f(w)]$ in $\spc{T}$;
if $e$ is an edge in $\partial W$ then no need to change $f$ on it.
This extends $f$ to $W^1$ such that each edge is mapped to a geodesic of length less that $\tfrac s3$.

Now for each triangle $uvw$ in $W$, the closed curve formed by $f$-images of its sides has length less than $s$.
That is, it is shorter than any noncontractible closed curve,
and therefore it is null-homotopic in $\spc{T}$.
Hence we can extend $f$ to the $W^2$.

Finally, since $\spc{T}$ is aspherical, there is no obstruction to extending $f$ to the rest of $W$.
\qeds

Observe that we use only that $\spc{T}$ is aspherical closed manifold.
That is, we actually proved the following more general theorem.

\begin{thm}{Theorem}\label{thm:sys<FillRad+}
Suppose $\spc{M}$ is an asherical $n$-dimensional Riemnnian space.
Then 
\[\sys\spc{M}\le 6 \cdot \FillRad \spc{M}.\]
\end{thm}

This statement will be generalized yet further in the following section.

\begin{thm}{Exercise}\label{ex:fillrad-inj}
Modify the proof of \ref{thm:sys<FillRad} to prove the following:

Suppose that $\spc{M}$ is a closed $n$-dimensional Reimannian manifold with \emph{injectivity radius} at least $r$; that is, if $\dist{p}{q}{\spc{M}}<r$, then there is geodesic $[pq]_{\spc{M}}$ is uniquely defined.
Show that
\[\FillRad\spc{M}\ge \tfrac{r}{n+1}.\]
 
\end{thm}

Note that this exercise together with bound on filling radius in \ref{thm:FillRad<vol} imply that lower bound on injectivity radius implies a lower bound on volume.
This statement was proved first by Marcel Berger \cite{berger} and it is not at all trivial.

\section{Essential manifolds}

Note that the real projective space $\RP^n$ is not aspherical.
Therefore \ref{thm:sys<FillRad+} is not applicable for $\RP^n$.
The goal of this section is to extend \ref{thm:sys<FillRad+} to $\RP^n$ and many other spaces.

...

