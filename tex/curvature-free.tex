\chapter{Besicovitch inequality} 

As it will be indicated in Section~\ref{sec:hausdorff-measure},
most of the statements of this and the following lecture have counterparts for general length metrics on manifolds.

For simplicity, we discuss mostly Riemannian spaces --- these are specially nice and smooth length metrics on manifolds.
These spaces are also most important in applications.

\section{Riemannian spaces}

Let $M$ be a smooth manifold.
A \index{metric tensor}\emph{metric tensor} on $M$ is a choice of positive definite quadratic forms $g_p$ on each tangent space $\T_pM$.
Usually one assumes that the metric tensor depends smoothly on the point $p$,
but for our purposes continuity will always be sufficient;
that is, one can fix a local coordinates at every point of $M$ so that the components of $g$ are continuous functions.
(Few times we will need piecewise continuous metric tensors.)

A Riemannian manifold $(M,g)$ is a smooth manifold $M$ with a choice of metric tensor $g$ on it.


If $\gamma\:[a,b]\to M$ is a Lipschitz curve, then its \index{length}\emph{$g$-length} is defined by
\[\length_g\gamma=\int_a^b\sqrt{g(\gamma'(t),\gamma'(t))}\cdot dt.\]
Let us define a metric on $M$ as greatest lower bound to lengths of Lipschitz curves connecting two given points;
the described distance between points $x$ and $y$ will be denoted by $\dist{x}{y}{g}$ or $\distfun_x(y)_g$.

When we say \emph{Riemannian manifold} we might mean a smooth manifold with a metric tensor $(M,g)$ or the corresponding metric space $\spc{M}$. 
It is not hard to show that the metric on the Riemannian manifold {}\emph{remembers} everything about the Riemannian manifold;
it is a part of the so-called Myers--Steenrod theorem \cite{myers-steenrod}.

The following observation will be used to extend a few results from Euclidean space to Riemannian manifolds.

\begin{thm}{Observation}\label{obs:lip-chart}
For any point $p$ in a Riemannian manifold $\spc{M}$ and any $\eps>0$ there is a $e^{\mp\eps}$-bi-Lipschitz bijection \index{chart}\emph{(chart)} $s\:W\to V$ from an open subset $W$ of the $n$-dimensional Euclidean space to some neighborhood $V\ni p$.
\end{thm}

\parit{Proof.}
Choose $p\in U$ and a chart $s\:U\to \spc{M}$ that covers $p$.
Note that there is a linear transformation $L$ such that for the metric tensor in the chart $s\circ L_i$ is coincides with the standard Euclidean tensor at the point $q=L^{-1}(p)$.

Observe that the restriction of $s\circ L$ to a small neighborhood of $q$ is $e^{\mp\eps}$-bi-Lipschitz.
Since $p$ is arbitrary, the observation follows.
\qeds

\section{Volume and Hausdorff measure}

Let $\spc{M}=(M,g)$ be an $n$-dimensional Riemannian manifold.
If a region $R\subset M$ is covered by one chart $\iota\:U\to \spc{M}$,
then its volume is defined as 
\[\vol_n R
\df
\int_{\iota^{-1}(R)}\sqrt{\det{g}}.\]
In the general case we subdivide $R$ into a countable collection of regions $R_1,R_2\dots$ such that each region $R_i$ is covered by a chart $\iota_i\:U_i\to \spc{M}$ and define
\[\vol_n R\df \vol_n R_1+\vol_n R_2+\dots\]
The chain rule for multiple integrals implies that the right-hand side does not depend on the choice of subdivision and the choice of charts.

Similarly, we define integral along $\spc{M}$.
Any Borel function $u\:\spc{M}\to \RR$, can be presented as a sum $u_1+u_2+\cdots$ such that the support of each function $u_i$ can be covered by one chart $\iota_i\:U_i\to \spc{M}$
and set 
\[\int_{p\in\spc{M}} u(p)
\df
\sum_i\left[\int_{x\in U_i} u_i\circ s(x)\cdot\sqrt{\det{g}}\right]
\]

Let $\spc{X}$ be a metric space and $R\subset \spc{X}$.
The \index{Hausdorff measure}\emph{$\alpha$-dimensional Hausdorff measure} of $R$ is defined as 
$$\haus_\alpha R
\df
\lim_{\eps\to0}
\,
\inf
\set{\sum_{n\in\NN}(\diam A_n)^\alpha}
{\begin{aligned}
&\diam A_n<\eps\ \text{for}
\\
&\text{for each}\ n,\text{all}\  A_n
\\
&\text{are closed, and} 
\\
& \bigcup_{n\in\NN}A_n\supset R.
\end{aligned}
}.$$
For properties of Hausdorff measure we refer to the classical book of  Herbert Federer \cite{federer};
in particular, $\haus_\alpha$ is indeed a measure and $\haus_\alpha$-measurable sets include all Borel sets.

The following observation follows from \ref{obs:lip-chart};
we will use it to extend some results in geometric measure theory to Riemannian manifolds.

\begin{thm}{Observation}\label{obs:lipcart+}
Suppose that a Borel set $R$ in an $n$-dimensional Riemannian manifold $\spc{M}$ is subdivided into a countable collection of subsets $R_i$ such that each $R_i$ is covered by an $e^{\mp\eps}$-bi-Lipschitz charts
$s_i\:W_i\to \spc{M}$.
Then
\begin{align*}
\vol R&\lege e^{\pm n\cdot\eps}\cdot\sum_i\vol[s_i^{-1}(R_i)]
\intertext{and}
\haus_n R&\lege e^{\pm n\cdot\eps}\cdot\sum_i\haus_n[s_i^{-1}(R_i)]
\end{align*}

\end{thm}

According to Haar's theorem, %ref???
a measure on $n$-dimensional Euclidean space that is invariant with respect to parallel translations is proportional to volume.
Observe that 
\begin{itemize}
\item A ball in $n$-dimensional Euclidean space of diameter $1$ has unit Hausdorff measure.
\item A unit cube in $n$-dimensional Euclidean space has volume.
\end{itemize}
Therefore, for any Borel region $R\subset \RR^n$, we have 
\[\vol_n R=\tfrac{\omega_n}{2^n}\cdot\haus_n R,\eqlbl{eq:vol/mu}\]
where $\omega_n$ denotes the volume of a unit ball in the $n$-dimensional Euclidean space.

Applying \ref{eq:vol/mu} together with \ref{obs:lipcart+}, we get that the inequality 
\[\vol_n R\lege e^{\pm2\cdot n\cdot \eps}\cdot\tfrac{\omega_n}{2^n}\cdot\haus_n R\]
holds for any $\eps>0$.
Since $\eps>0$ is arbitrary, we get the following:

\begin{thm}{Proposition}
The identity \ref{eq:vol/mu} holds for any Borel region $R$ in an $n$-dimensional Riemannian manifold. 
\end{thm}

This proposition can be used to define volume of any dimension $k$ by setting
\[\vol_k\df\tfrac{\omega_k}{2^k}\cdot\haus_k.\]
If $A$ is a subset of $k$-dimensional submanifold in, then the two definitions of $\vol_kA$ agree; but the latter definition works for a wider class of sets. 

\section{Area and coarea formulas}

Suppose that $f\:\spc{M}\to\spc{N}$ is a Lipschitz map between $n$-dimensional Riemannian manifolds $\spc{M}$ and $\spc{N}$.
Then by Rademacher's theorem %ref???
the differential $d_p f\:\T_p\spc{M}\to\T_{f(p)}\spc{N}$ is defined at almost all $p\in \spc{M}$.

The differential is a linear map, defines the Jacobian matrix $\Jac_pf$ in orthonormal frames of $\T_p$ and $\T_{f(p)}\spc{N}$.
The determinant of $\Jac_pf$ will be denoted by $\jac_p$;
its absolute value does not depend on the choice of frames.

The identity in the following proposition is called \index{area formula}\emph{area formula}.

\begin{thm}{Proposition}
Let $f\:\spc{M}\to\spc{N}$ be a Lipschitz map between $n$-dimensional Riemannian manifolds $\spc{M}$.
Then for  any Borel function $u\:\spc{M}\z\to \RR$ the following equality holds:
\[\int_{p\in \spc{M}} u(p)\cdot |\jac_pf|=\int_{q\in \spc{N}}\sum_{p\in f^{-1}(q)} u(p).\]

\end{thm}

\parit{Proof.}
If $\spc{M}$ and $\spc{N}$ are isometric $n$-dimensional Euclidean space, then the statement follows from the standard area formula;
see \cite[3.2.3]{federer}.

Note that Jacobian of a $e^{\mp\eps}$-bi-Lipschitz map between $n$-dimensional Riemannian manifolds (if defined) has determinant in the range $e^{\mp n\cdot\eps}$.
By \ref{obs:lipcart+}, we get the following approximate version of the area formula 
\[\int_{p\in \spc{M}} u(p)\cdot |\jac_pf|
\lege e^{\pm 3\cdot n\cdot \eps}\int_{q\in \spc{N}}\sum_{p\in f^{-1}(q)} u(p)\]
holds if $u\ge0$.

Since $\eps>0$ is arbitrary, we get that the area formula holds if $u\ge 0$.
Finally, since both sides of the area formula are linear in $u$, it holds for any $u$.
\qeds

The following inequality is called \index{area inequality}\emph{area inequality}.

\begin{thm}{Corollary}\label{cor:area-inequality}
Let $f\:\spc{M}\to\spc{N}$ be a locally Lipschitz map between $n$-dimensional Riemannian manifolds.
Then 
\[\int_{p\in A} |\jac_p f|\ge \vol[f(A)]\]
for any Borel subset $A\subset M$.

In particular, if $|\jac_p f|\le 1$ almost everywhere in $A$, then 
\[\vol A \ge \vol[f(A)].\]
\end{thm}

\parit{Proof.} Choose $u$ to be the characteristic function of $A$ and apply the area formula.
\qeds

Suppose that $f\:\spc{M}\to\RR$ is a Lipschitz function defined on an $n$-dimensional Riemannian manifold $\spc{M}$.
Then by Rademacher's theorem, the differential $d_pf\:\T_p\spc{M}\to\RR$  and the gradient 
$\nabla_pf\in\T_p\spc{M}$ are defined at almost all $p\in \spc{M}$.

The identity in the following proposition is a partial case of the so-called \index{coarea formula}\emph{coarea formula}.


\begin{thm}{Proposition}\label{prop:coarea}
Let $f\:\spc{M}\to\RR$ be a Lipschitz function defined on an $n$-dimensional Riemannian manifold $\spc{M}$.
Suppose that the level sets $L_x\df f^{-1}(x)$ are equipped with $(n-1)$-dimensional volume $\vol_{n-1}\df\tfrac{\omega_{n-1}}{2^{n-1}}\cdot \haus_{n-1}$.
Then for any Borel function $u\:\spc{M}\to \RR$ the following equality holds
\[\int_{p\in \spc{M}} u(p)\cdot |\nabla_pf|=\int_{x\in \RR}dx\cdot \int_{p\in L_x} u(p).\]
\end{thm}

\parit{Instead of proof.}
If $\spc{M}$ is isometric to Euclidean space, then the statement follows from \cite[3.2.12]{federer}.
The reduction to the Euclidean space is done the same way as in the proof of the area formula.
\qeds

\begin{thm}{Corollary}\label{cor:coarea}
Let $\spc{M}$, $f$, and $L_x$ be as in \ref{prop:coarea}.

Suppose that $f$ is 1-Lipschitz.
Then for any Borel subset $A\subset M$ we have
\[\vol_n A\ge \int_{x\in\RR} \vol_{n-1}[A\cap L_x]\cdot dx.\]
\end{thm}

This inequality is a partial case of the so-called  \index{coarea inequality}\emph{coarea inequality};
its right-hand side is called \index{coarea of function}\emph{coarea of the restriction $f|_A$}. 

\parit{Proof.} Choose $u$ to be the characteristic function of $A$ and apply the coarea formula.
\qeds


\section{Besicovitch inequality}

\begin{thm}{Theorem}\label{thm:besikovitch}
Let $g$ be a metric tensor on a unit $n$-dimensional cube $\square$.
Suppose that the $g$-distances between the opposite faces of $\square$ are at least $1$; that is, any Lipschitz curve that connects opposite faces has $g$-length at least $1$.
Then $\vol(\square, g)\ge 1$.
\end{thm}

This theorem was proved by Abram Besicovitch \cite{besicovitch}.

\parit{Proof.}
We will consider the case $n=2$; the other cases are proved the same way.

\begin{wrapfigure}{r}{30mm}
\vskip-0mm
\centering
\includegraphics{mppics/pic-1320}
\end{wrapfigure}

Denote by $A$, $A'$, and $B$, $B'$ the opposite faces of the square~$\square$.
Consider two function 
\begin{align*}
f_A(x)&\df\min\{\,\distfun_A(x)_g,1\,\},
\\
f_B(x)&\df\min\{\,\distfun_B(x)_g,1\,\}.
\end{align*}
Let $f\:\square\to\square$ be the map with coordinate functions $f_A$ and $f_B$;
that is, $f(x)\df(f_A(x), f_B(x))$.

Observe that $f$ maps each face to itself.
Indeed, 
\[x\in A \quad\Longrightarrow\quad \distfun_A(x)_g=0 \quad\Longrightarrow\quad f_A(x)=0 \quad\Longrightarrow\quad f(x)\in A.\]
Similarly, if $x\in B$, then $f(x)\in B$.
Further, 
\[x\in A'
\quad\Longrightarrow\quad 
\distfun_A(x)_g\ge 1 
\quad\Longrightarrow\quad 
f_A(x)=1 
\quad\Longrightarrow\quad 
f(x)\in A'.\]
Similarly, if $x\in B'$, then $f(x)\in B'$.

Therefore 
\[f_t(x)= t\cdot x + (1-t)\cdot f(x)\]
defines a homotopy of maps of the pair of spaces $(\square,\partial \square)$ from $f$ to the identity map.
It follows that degree of $f$ is $1$; that is, $f$ sends the fundamental class of $(\square,\partial \square)$ to itself.
In particular $f$ is onto.

Suppose that Jacobian  matrix $\Jac_pf$ of $f$ is defined at $p\in \square$.
Choose an orthonormal frame in $\T_p$ with respect to $g$ and the standard frame in the target $\square$.
Observe that the differentials $d_pf_A$ and $d_pf_B$ written in these frames are the rows of $\Jac_pf$.
Evidently $|d_pf_A|\le 1$ and $|d_pf_B|\le 1$.
Since the determinant of a matrix is the volume of the parallelepiped spanned on its rows, we get 
\[|\jac_p f|\le |d_pf_A|\cdot|d_pf_B|\le 1.\]
Since $f\:\square\to\square$ is a Lipschitz onto map, the {}area inequality \ref{cor:area-inequality} implies that 
\[\vol(\square,g)\ge \vol\square=1.\]
\qedsf


\begin{thm}{Theorem}\label{thm:besikovitch+}
Let $(M,g)$ be an $n$-dimensional Riemannian manifold.
Suppose that there is a degree 1 map from its boundary $\partial M$ to the surface of $n$-dimensional cube $\square$;
denote by $d_1,\dots, d_n$ the distances between the inverse images of pairs of opposite faces of $\square$ in $\partial M$.
Then 
\[\vol(M,g)\ge d_1\cdots d_n.\]

Moreover, in the case of equality, $\vol(M,g)$ is isometric to the product $[0,d_1]\times\dots\times[0,d_n]$.
\end{thm}

The first part of the stated generalization can be proved along the same lines as \ref{thm:besikovitch}.

\begin{thm}{Exercise}\label{ex:besikovitch=}
Prove the second part of \ref{thm:besikovitch+}.
\end{thm}


\begin{thm}{Exercise}\label{ex:hexagon}
Suppose $g$ is a metric tensor on a regular hexagon $\textbf{\hexagon}$ such that $g$-distances between the opposite sides are at least $1$.
Is there a positive lower bound on $\area(\textbf{\hexagon},g)$?
\end{thm}

\begin{thm}{Exercise}\label{ex:cylinder}
Let $g$ be a Riemannian metric on the cylinder $\mathbb{S}^1\z\times [0,1]$.
Suppose that 
\begin{itemize}
\item 
$g$-distance between pairs of points on the opposite boundary circles $\mathbb{S}^1\times\{0\}$ and $\mathbb{S}^1\times\{1\}$ is at least 1, and 
\item
any curve $\gamma$ in $\mathbb{S}^1\times [0,1]$ that is homotopic to $\mathbb{S}^1\times\{0\}$ has $g$-length at least $1$.
\end{itemize}

\begin{subthm}{ex:cylinder:besicovitch}
Use Besicovitch inequality to show that
\[\area(\mathbb{S}^1\times [0,1],g)\ge \tfrac12.\]

\end{subthm}

\begin{subthm}{ex:cylinder:coarea}
 Modify the proof of Besicovitch inequality using coarea inequality (\ref{cor:coarea}) to show that \[\area(\mathbb{S}^1\times [0,1],g)\ge 1.\]
 
\end{subthm}

\end{thm}


\begin{thm}{Exercise}\label{ex:gadograph}
Let $V$ be a compact set in the $n$-dimensional Euclidean space $\EE^n$ bounded by a hypersurface $\Sigma$.
Suppose $g$ is a Riemannian metric on $V$ such that 
\[\dist{p}{q}{g}\ge\dist{p}{q}{\EE^n}\]
for any two points $p,q\in \Sigma$.
Show that
\[\vol_n(V,g)\ge \vol_n(V)_{\EE^n}.\]
 
\end{thm}

\begin{thm}{Exercise}\label{ex:involution-of-sphere}
Suppose that sphere with Riemannian metric $(\mathbb{S}^2,g)$ admits an involution $\iota$ such that $\dist{x}{\iota(x)}{g}\ge 1$.

Show that 
\[\area(\mathbb{S}^2,g)\ge \tfrac1{1000}.\]
Try to show that 
\[\area(\mathbb{S}^2,g)\ge \tfrac12\quad\text{or}\quad \area(\mathbb{S}^2,g)\ge 1.\]
\end{thm}

\begin{thm}{Advanced exercise}\label{ex:involution-of-3sphere}
Construct a metric tensor $g$ on $\mathbb{S}^3$ such that (1) $\vol(\mathbb{S}^3,g)$ arbitrary small and (2) there is an involution $\iota\:\mathbb{S}^3\z\to \mathbb{S}^3$ such that $\dist{x}{\iota(x)}{g}\ge 1$ for any $x\in \mathbb{S}^3$.
\end{thm}

\begin{thm}{Exercise}\label{ex:GH-vol}
Let $g_1,g_2,\dots$, and $g_\infty$ be metrics on a fixed compact manifold $M$.
Suppose that $\distfun_{g_n}$ uniformly converge to $\distfun_{g_\infty}$ as functions on $M\times M\to\RR$.
Show that 
\[\liminf_{n\to\infty}\vol(M,g_n)\ge \vol(M,g_\infty).\]

Show that the inequality might be strict.
\end{thm}

\section{Systolic inequality}

Let $\spc{M}$ be a compact Riemannian manifold.
The \index{systole}\emph{systole} of $\spc{M}$ (briefly $\sys\spc{M}$) is defined to be the least length of a noncontractible closed curve in $\spc{M}$.

Let $\Lambda$ be a set of closed smooth $n$-dimensional manifolds.
We say that a \index{systolic inequality}\emph{systolic inequality} holds for $\Lambda$ if there is a constant $c$ such that for any $M\in \Lambda$ and any metric tenor $g$ on $M$ we have
\[\sys(M,g)\le c\cdot \sqrt[n]{\vol(M,g)}.\]

\begin{thm}{Exercise}\label{ex:sysT2}
Use \ref{thm:besikovitch} or \ref{ex:cylinder} to show that systolic inequality holds for any Riemannian metric on the 2-torus $\TT^2$.
\end{thm}

\begin{thm}{Exercise}\label{ex:sysRP2}
Use \ref{thm:besikovitch} to show that systolic inequality holds for any Riemannian metric on  the real projective plane $\RP^2$.
\end{thm}

\begin{thm}{Exercise}\label{ex:sysSg}
Use \ref{thm:besikovitch+} to show that systolic inequality holds for any Riemannian metric on any closed surfaces of positive genus.
\end{thm}

\begin{thm}{Exercise}\label{ex:sysS2xS1}
Show that no systolic inequality holds for Riemannian metrics on $\mathbb{S}^2\times\mathbb{S}^1$.
\end{thm}

In the following lecture we will show that systolic inequality holds for many manifolds, in particular for torus of arbitrary dimension.

\section{Generalization}\label{sec:hausdorff-measure}

The following proposition follows immediately from the definitions.

\begin{thm}{Proposition}\label{prop:bilip-measure}
Let $\spc{X}$ and $\spc{Y}$ be metric spaces, $A\subset \spc{X}$
and
 $f\: \spc{X}\to \spc{Y}$ be a $\Lip$-Lipschitz map. 
Then 
\[\haus_\alpha [f(A)]\le \Lip^\alpha\cdot\haus_\alpha\, A\]
for any $\alpha$.
\end{thm}

The following exercise provides a weak analog of the Besicovitch inequality that works for arbitrary metric spaces.

\begin{thm}{Exercise}\label{ex:besikovitch++}
Let $M$ be manifold with boundary and $\rho$ is a pseudometric on $M$.
Suppose $\partial M$ admits a degree 1 map to the surface of the $n$-dimensional cube $\square$;
denote by $d_1,\dots, d_n$ the $\rho$-distances between the inverse images of pairs of opposite faces of $\square$ in $M$.
Then 
\[\haus_n(M,\rho)\ge d_1\cdots d_n.\]
\end{thm}


Recall that $\tfrac{\omega_n}{2^n}\cdot\haus_n=\vol_n$ for $n$-dimensional Riemannian manifolds.
Note that $\tfrac{\omega_n}{2^n}<1$ if $n\ge 2$.
Therefore, the conclusions in \ref{ex:besikovitch++} are weaker than in \ref{thm:besikovitch+}; the assumptions are weaker as well.

It is straightforward to define systolic inequality on $n$-dimensional manifolds with the Hausdorff measure $\haus_n$ instead of volume and prove analogs of the exercises \ref{ex:sysT2}--\ref{ex:sysS2xS1}.

\begin{thm}{Exercise}\label{ex:2top-discs}
Let $\spc{X}$ be a contractible metric space with zero $(n\z+1)$-dimensional Hausdorff measure.
Assume that $\Delta_1,\Delta_2\subset X$ are two embedded $n$-disks having the same boundary.
Show that $\Delta_1=\Delta_2$.
\end{thm}

\section{Remarks}

The optimal constants in the systolic inequality are known only in the following three cases:
\begin{itemize}
\item For real projective plane $\RP^2$ the constant is $\tfrac\pi2$ --- the equality holds for a quotient of a round sphere by isometric involution. The statement was proved by Pao Ming Pu \cite{pu}.\label{page:pu}
\item For torus $\TT^2$ the constant is $\tfrac2{\sqrt{3}}$ --- the equality holds for a flat torus obtained from a regular hexagon by identifying opposite sides; this is the so-called \index{Loewner's torus inequality}\emph{Loewner's torus inequality}.
\item For the Klein bottle $\RP^2\#\RP^2$  the constant is $\tfrac\pi{2\cdot\sqrt2}$ --- the equality holds for a certain nonsmooth metric.
The statement was proved by Christophe Bavard \cite{bavard}.
\end{itemize}
The proofs of these results use the so-called {}\emph{uniformization theorem}   available in the 2-dimensional case only.
These proofs are beautiful, but they are too far from metric geometry.
A good survey on the subject is written by Christopher Croke and Mikhail Katz \cite{croke-katz}.

An analog of Exercise \ref{ex:GH-vol} with Hausdorff measure instead of volume does not hold for general metrics on a manifold.
In fact there is a nondecreasing sequence of metric tensors $g_n$ on $M$, such that (1) $\vol(M,g_n)<1$ for any $n$ and (2) $\distfun_{g_n}$ converges to a metric on $M$ with arbitrary large Hausdorff measure of any given dimension; such an example was constructed by Dmitri Burago, Sergei Ivanov, and David Shoenthal \cite{burago-ivanov-shoenthal}.
