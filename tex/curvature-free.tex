\chapter{Besikovitch inequality} 


\section{Riemannian spaces}

Riemannian spaces are smooth manifolds with metric defined by a metric tensor.
These are specially nice length metrics on manifolds.
However most of the statements we are going to discuss have counterpart for general length metrics on manifolds.

Let $M$ be a smooth manifold.
A \index{metric tensor}\emph{metric tensor} on $M$ is a choice of positive definite quadratic forms $g_p$ on each tangent space $\T_pM$ that depends smoothly on the point $p$.
That is, if we fix a local coordinates on $M$ and write $g$ in this coordinates, then each component of $g$ is a smooth function. 

A Riemannian manifold $(M,g)$ is a smooth manifold $M$ with a choice metric tensor $g$ on it.
The metric tensor $g$ can be used to define length of curves, distances, and volume of regions in $M$.
In particular any Riemannian manifold is a metric space which might be called \index{Riemannian space}\emph{Riemannian space}.


\parbf{Lengths and distances.}
If $\gamma\:[a,b]\to M$ is a piecewise smooth curve then 
\[\length_g\gamma=\int_a^b\sqrt{g(\gamma'(t),\gamma'(t))}\cdot dt.\]
Further we can define a metric on $M$ as least lower bound to lengths of piecewise smooth curves connecting two given points;
the described distance between points $x$ and $y$ will be denoted by $\dist{x}{y}{g}$ or $\distfun_x(y)_g$.
The distance function from a point $x$ will be denoted by $(\distfun_x)_g$ or $\distfun_x$ if the choice of $g$ is evident.

The following claim follows from Myers--Steenrod theorem \cite{myers-steenrod};
it shows that the metric on the Reimannian manifold {}\emph{remembers} everything about the Reimannian manifold.

\begin{thm}{Claim}
Let $(M,g)$ be a Riemannian manifold.
Then the metric $(x,y)\mapsto \dist{x}{y}{g}$ defines a length metric. Moreover this metric completely determines the metric tensor $g$.
\end{thm}

\parbf{Volume.}
If a region $R$ is covered by one chart $\iota\:U\to M$,
then its volume can be defined as an integral 
\[\vol R
\df
\int_{\iota^{-1}(R)}\sqrt{\det{g}}.\]
In the general case we subdivide $R$ into (a countable collection of) regions $R_1,R_2\dots$ and define
\[\vol R\df \vol R_1+\vol R_2+\dots\]

\section{Besikovitch inequality}

\begin{thm}{Theorem}\label{thm:besikovitch}
Let $g$ be a metric tensor on a unit $n$-dimensional cube $\square^n$.
Suppose that the $g$-distances between the opposite faces of $\square^n$ are at leat $1$; that is, any piecewise smooth curve that connects opposite faces has $g$-length at least $1$.
Then $\vol(\square^n, g)\ge 1$.
\end{thm}

\parit{Proof.}
We will consider the case $n=2$; the other cases are proved the same way.

Denote by $A$, $A'$, and $B$, $B'$ the opposite faces of the square~$\square$.
Consider two function 
\begin{align*}
f_A(x)&\df\min\{\,\distfun_A(x)_g,1\,\},
\\
f_B(x)&\df\min\{\,\distfun_B(x)_g,1\,\}.
\end{align*}
Define $f\:\square\to\square$ as a map with coordinate funcions $f_A$ and $f_B$;
that is, $f(x)\df(f_A(x), f_B(x))$.

Observe that $f$ maps each face to itself.
Indeed, 
\[x\in A \quad\Longrightarrow\quad \distfun_A(x)_g=0 \quad\Longrightarrow\quad f_A(x)=0 \quad\Longrightarrow\quad f(x)\in A.\]
Similarly if $x\in B$, then $f(x)\in B$.
Further, 
\[x\in A'
\quad\Longrightarrow\quad 
\distfun_A(x)_g\ge 1 
\quad\Longrightarrow\quad 
f_A(x)=1 
\quad\Longrightarrow\quad 
f(x)\in A'.\]
Similarly if $x\in B'$, then $f(x)\in B'$.

Therefore 
\[f_t(x)= t\cdot x + (1-t)\cdot f(x)\]
defines a homotopy of maps of pair of spaces $(\square,\partial \square)$ from $f$ to the identity map.
It follows that degree of $f$ is $1$; that is, $f$ sends the fundamental class of $(\square,\partial \square)$ to itself.
In particular $f$ is onto.

Suppose that Jacobian  matrix $\Jac_pf$ of $f$ is defined at $p\in \square$.
Choose an orthonormal basis in $\T_p$ with respect to $g$ and the standard basis in the target $\square$.
Observe that the differentials $d_pf_A$ and $d_pf_B$ written in these basises are the rows of $\Jac_pf$.
Evidently $|d_pf_A|\le 1$ and $|d_pf_B|\le 1$.
Since the determinant of a matrix is the volume of the parallelepiped spanned on its rows, we get 
\[|\det(\Jac_pf)|\le |d_pf_A|\cdot|d_pf_B|\le 1.\]
Since $f\:\square\to\square$ is a Lipschitz onto map, the {}\emph{area formula} implies that 
\[\vol(\square,g)\ge \vol\square=1.\]
\qedsf


\begin{thm}{Theorem}\label{thm:besikovitch+}
Let $(M,g)$ be Riemannian manifold and its boundary admits a degree 1 map $\partial M\to\partial\square^n$. 
Suppose $d_1,\dots, d_n$ the distances between the the inverse images of pairs of opposite faces of $\square^n$ in $\partial M$.
Then 
\[\vol(M,g)\ge d_1\cdots d_n.\]

Moreover, in the case of equality, $(\square^n,g)$ is isometric to the product $[0,d_1]\times\dots\times[0,d_n]$.
\end{thm}

The first part of the stated generalization can be proved along the same lines as \ref{thm:besikovitch}.

\begin{thm}{Exercise}\label{ex:besikovitch=}
Prove the second part of \ref{thm:besikovitch+}.
\end{thm}

\begin{thm}{Exercise}\label{ex:hexagon}
Suppose $g$ is a metric tensor on a regular hexagon $\varhexagon
   $ such that $g$-distances between the opposite sides are at least $1$.
Is there a positive lower bound on $\area(\varhexagon,g)$?
\end{thm}

\begin{thm}{Exercise}\label{ex:gadograph}
Let $V$ be a compact set in $\EE^d$ bounded by a hypersurface $\Sigma$.
Suppose $g$ is a Riemannian metric on $V$ such that 
\[\dist{p}{q}{g}\ge\dist{p}{q}{\EE^d}\]
for any two points $p,q\in \Sigma$.
Show that
\[\vol(V,g)\ge \vol(V)_{\EE^d}.\]
 
\end{thm}

\begin{thm}{Exercise}\label{ex:involution-of-sphere}
Suppose that sphere with Riemannian matric $(\mathbb{S}^2,g)$ admits an involution $\iota$ such that $\dist{x}{\iota(x)}{g}\ge 1$.

Show that $\area(\mathbb{S}^2,g)\ge \tfrac1{1000}$;
try to show that $\area(\mathbb{S}^2,g)\ge \tfrac12$ or $\area(\mathbb{S}^2,g)\ge 1$.
\end{thm}

\begin{thm}{Advanced exercise}\label{ex:involution-of-3sphere}
Construct a metric $g$ on $\mathbb{S}^3$ with arbitrary small $\vol(\mathbb{S}^3,g)$ and such that it admits an involution $\iota$ such that $\dist{x}{\iota(x)}{g}\ge 1$.
\end{thm}

\begin{thm}{Exercise}\label{ex:GH-vol}
Suppose a sequence of Riemannian spaces $\spc{M}_n$ \index{stable convergence}\emph{stably converges} in the sense of Gromov--Hausdorff  to a Riemannian spaces $\spc{M}_\infty$ as $n\to\infty$;
that is, the corresponding Hausdorff approximations can be chousen to be homeomorphisms.
Show that 
\[\liminf_{n\to\infty}\vol\spc{M}_n\ge \vol\spc{M}_\infty.\]

Show that the statement does not hold if we do not assume that the convergence is stable.
\end{thm}

\begin{thm}{Exercise}
Assume a metric $g$ on $\RR^m$ coincides with the Euclidean metric outside of a bounded set $K$;
assume further that any geodesic that enters $K$ exits $K$ the same way the Euclidean geodesic would have done. 
Show that $g$ is flat; that is, $(\RR^m,g)$ is isometric to the Euclidean space. 
\end{thm}

\section{Hausdorff measure}

Hausdorff measure is one of the most standard analogs of volume that can be defined on subset of general to metric space.

Let $\spc{X}$ be a metric space.
A function $\mu$ with values in $[0,\infty]$,
defined on all subsets of $\spc{X}$
is called \index{outer measure}\emph{outer measure} if
\begin{itemize}
\item $\mu\,\emptyset=0$;
\item If $A\subset B\subset \spc{X}$, then $\mu\, A\le \mu\, B$; 
\item For any sequence $A_1, A_2,\dots$ of subsets of $\spc{X}$ we have
$$\mu\left(\bigcup_n A_n\right) \le \sum_n \mu\, A_n.$$
\end{itemize}

A subset $E\subset \spc{X}$ is called \index{$\mu$-measurable}\emph{$\mu$-measurable} if 
$$\mu\, A = \mu(A \cap E) + \mu(A \backslash E)$$
for every subset $A\subset\spc{X}$.

The following is a classical lemma in measure theory \cite[2.1.3 and 2.3.2(9)]{federer}.

\begin{thm}{Carath\'eodory's lemma}\label{lem:caratheodory}
Let  $\mu$ be an outer measure on a metric space $\spc{X}$.
Then the $\mu$-measurable sets form a sigma-algebra.

Moreover, if 
\[\mu(A\cup B)=\mu\, A+\mu\, B
\eqlbl{eq:caratheodory}\]
for any two sets $A$ and $B$ 
such that there $\dist{a}{b}{}>\eps$ for any $a\in A$, $b\in B$, and fixed $\eps>0$, then any Borel set in $\spc{X}$ is $\mu$-measurable.
\end{thm}

\parbf{Carath\'eodory's construction.}\index{Carath\'eodory's construction}
Fix a function $\rho$, that returns a value in $[0,\infty]$
for any closed subset of $\spc{X}$.
Define outer measure $\mu_\rho$ of set $W$ in $\spc{X}$ in the following way:
$$\mu_\rho W
\df
\lim_{\eps\to0}
\,
\inf
\set{\sum_{n\in\NN}\rho A_n}
{\begin{aligned}
&\bigcup_{n\in\NN}A_n\supset W, \text{all}\  
A_n
\ \text{are closed,}\ 
\\
&
\text{and}\ \diam A_n<\eps\ \text{for each}\ n.
 \end{aligned}
}.$$
Note that
the value of the infimum above is non-decreasing in $\eps$;
in particular the limit is defined.

An outer measure $\mu$ on $\spc{X}$ is called \index{Borel regular measure}\emph{Borel regular} if any Borel set in $\spc{X}$ is $\mu$-measurable and for any set $A\subset \spc{X}$ there is a Borel set $B$ such that $A\subset B$ and $\mu\, A=\mu\, B$. 

From Carath\'eodory's lemma (\ref{lem:caratheodory}), we get the following.

\begin{thm}{Corollary}
The Carath\'eodory's construction always produce a Borel regular outer measure.
\end{thm}

The Carath\'eodory's construction can be applied to different choices of the function $\rho$.
One of the most popular choice is $\rho A=(\diam A)^\alpha$.
It produces the so called \index{Hausdorff measure}\emph{$\alpha$-dimensional Hausdorff measure} $\mu_\alpha$.
If we need to emphasize that the measure is defined on the space $\spc{X}$, we use $\spc{X}$ as the index.
For example, we may write $\mu_\alpha(A)_\spc{X}$ for the $\alpha$-dimensional Hausdorff measure of set $A$ in $\spc{X}$.

The following proposition trivially follows from the definitions

\begin{thm}{Proposition}\label{prop:bilip-measure}
Let $\spc{X}$ and $\spc{Y}$ be metric spaces, $A\subset \spc{X}$
and
 $f\: \spc{X}\to \spc{Y}$ be a $\Lip$-Lipschitz map. 
Then 
\[\mu_\alpha [f(A)]\le \Lip^\alpha\cdot\mu_\alpha\, A.\]

\end{thm}

The following exercise proves an weak analog of the Besikovitch inequality that works for arbitrary metric spaces.

\begin{thm}{Exercise}\label{ex:besikovitch++}
Let $M$ be manifold with boundary and $\rho$ is a pseudometric on $M$.
Suppose $\partial M$ admits a degree 1 map $\partial M\to\partial\square^n$. 
Suppose $d_1,\dots, d_n$ the $\rho$-distances between the the inverse images of pairs of opposite faces of $\square^n$ in $M$.
Then 
\[\mu_n(M,\rho)\ge d_1\cdots d_n.\]
\end{thm}

\begin{thm}{Exercise}\label{ex:huas/vol}
Let $\spc{M}$ an $n$-dimensional Riemannian manifold.
Show that 
\[\omega_n\cdot\mu_n\,A=\vol_n A\]
for any Borel set $A\subset \spc{M}$, where $\omega_n$ denotes the volume of a ball in the $n$-dimensional Euclidean space with diameter 1.
\end{thm}

Note that $\omega_n<1$ for $n\ge 2$.
Therefore by \ref{ex:huas/vol}, the conclusions in \ref{ex:besikovitch++} (as well as its assumptions) are weaker than in \ref{thm:besikovitch+}.

\begin{thm}{Exercise}\label{}
Let $X$ be a contractible metric space with zero $(n+1)$-dimensional Hausdorff measure.
Assume that $\Delta_1,\Delta_2\subset X$ are two embedded $n$-disks having the same boundary.
Show that $\Delta_1=\Delta_2$.
\end{thm}

\section{Systolic inequality}

Let $\spc{M}$ be a compact Riemannian space.
The \index{systole}\emph{systole} of $\spc{M}$ (briefly $\sys\spc{M}$) is defined to be the least length of a noncontractible closed curve in $\spc{M}$.

Let $\Lambda$ be a set of closed $n$-dimensional manifolds.
We say that a systolic inequality holds for $\Lambda$ if there is a constant $c$ such that for any $M\in \Lambda$ and any metric tenor $g$ on $M$ we have
\[\sys(M,g)\le c\cdot \sqrt[n]{\vol(M,g)}.\]

\begin{thm}{Exercise}\label{ex:sysT2}
Use \ref{thm:besikovitch} to show that systolic inequality holds for the 2-torus $\TT^2$.
\end{thm}

\begin{thm}{Exercise}\label{ex:sysRP2}
Use \ref{thm:besikovitch} to show that systolic inequality holds for the real projective palane $\RP^2$.
\end{thm}

\begin{thm}{Exercise}\label{ex:sysSg}
Use \ref{thm:besikovitch+} to show that systolic inequality holds for the set of all closed surfaces of positive genus.
\end{thm}

\begin{thm}{Exercise}\label{ex:sysS2xS1}
Show that no systolic inequality holds for $\mathbb{S}^2\times\mathbb{S}^1$.
\end{thm}

In the following lecture we will show that systolic inequality holds for many manifolds, in particular for torus of arbitrary dimension.


\section{Remarks}


The optimal constants in the systolic inequality are known only in the following three cases:
\begin{itemize}
\item For real projective plane $\RP^2$ the constant is $\tfrac\pi2$ --- the equality holds for a quotient of a round sphere by isometric involution. The statement was proved by Pao Ming Pu \cite{pu}.\label{page:pu}
\item For torus $\TT^2$ the constant is $\tfrac2{\sqrt{3}}$ --- the equality holds for a flat torus obtained from a regular hexagon by identifying opposite sides; this is the so called \index{Loewner's torus inequality}\emph{Loewner's torus inequality}.
\item For the Klein bottle $\RP^2\#\RP^2$  the constant is $\tfrac\pi{2\cdot\sqrt2}$ --- the equality holds for certain nonsmooth metrics \cite{bavard}.
\end{itemize}
The proofs of these results use the so called {}\emph{uniformization theorem}   available in the 2-dimensional case only.
These proofs are beautiful, but they too far from metric geometry.
A good survey on the subject is written by Christopher Croke and Mikhail Katz \cite{croke-katz}.



