Recall that a sequence of Borel measures $\mu_n$ \index{weak convergence}\emph{weakly converges} to a measure $\mu$ if for any continuous function $f$ with compact support we have
\[\int_{\spc{X}} f\cdot \mu_n\to\int_{\spc{X}} f\cdot \mu\]
as $n\to \infty$.
The weak convergence defines a topology on the space ...
This space comes with topology defined by weak convergence;
that is ...

















\section{Almost isometries}\label{sec:alm-isom}

\begin{thm}{Definition} Let $\spc{X}$ and $\spc{Y}$ be metric spaces and $\eps>0$. 
A  map\footnote{possibly noncontinuous} $f\: \spc{X} \z\to \spc{Y}$ is called an \index{almost isometry}\emph{$\eps$-isometry} 
if $f(\spc{X})$ is an $\eps$-net in $\spc{Y}$ and
\[\bigl|\dist{x}{x'}{\spc{X}}-\dist{f(x)}{f(x')}{\spc{Y}}\bigr|<\eps.\]
for any $x,x'\in \spc{X}$.
\end{thm}

\begin{thm}{Exercise}\label{ex:eps-isom}
Let $\spc{X}$ and $\spc{Y}$ be compact metric spaces.

\begin{subthm}{ex:eps-isom:GH>isom}
If $\dist{\spc{X}}{\spc{Y}}{\GH}<\eps$, then there is a $2\cdot\eps$-isometry $f\:\spc{X}\to\spc{Y}$.
\end{subthm}

\begin{subthm}{ex:eps-isom:isom>GH}
If there is an $\eps$-isometry $f\:\spc{X}\to\spc{Y}$, then $\dist{\spc{X}}{\spc{Y}}{\GH}<\eps$.
\end{subthm}

\end{thm}

\parit{Proof of the ``only if''-part in \ref{GH-2}.}
\label{page:GH-2-proof}
Let $\spc{X}$ and $\spc{Y}$ be compact metric spaces.
Suppose that $\dist{\spc{X}}{\spc{Y}}{\GH}<\eps$ for any $\eps>0$;
we need to show that there is an isometry $\spc{X}\to\spc{Y}$.

By \ref{ex:eps-isom:GH>isom}, for each positive integer $n$, we can choose a $\tfrac1n$-isometry $f_n\:\spc{X}\to\spc{Y}$.

Since $\spc{X}$ is compact, 
we can choose a countable dense set
$S$ in~$\spc{X}$.
Applying the diagonal procedure if necessary, we can assume that for every $x \in S$ the sequence $f_n(x)$ 
converges in $\spc{Y}$. 
Consider the pointwise limit map  $f_\infty \: S \to \spc{Y}$,
 $$f_\infty(x) \df \lim_{n\to\infty} f_n (x)$$ for every $x \in S$. 
Since $$|f_n (x)- f_n (x')|_{\spc{Y}}\lg |x- x'|_\spc{X} \pm\tfrac1n,$$ 
we have 
$$|f_\infty(x)-f_\infty (x')|_{\spc{Y}} 
= \lim_{n\to\infty} |f_n(x)-f_n (x')|_{\spc{Y}} 
= |x -x'|_\spc{X}$$ for all
$x, x' \in S$; 
that is, the map $f_\infty\:S\to \spc{Y}$ is distance-preserving. 
Therefore, $f_\infty$ can be extended to a distance-preserving map from the whole $\spc{X}$ to $\spc{Y}$.

The latter can be done by setting 
$$f_\infty(x)=\lim_{n\to\infty} f_\infty(x_n)$$ 
for some sequence $x_n$ of points  in $S$
that converges to $x$ in $\spc{X}$.
Indeed, if $x_n\to x$, then the sequence $x_n$ is Cauchy.
Since $f_\infty$ is distance-preserving, $y_n=f_\infty(x_n)$ is also a Cauchy sequence in $\spc{Y}$;
therefore it converges.
It remains to observe that this construction does not depend on the choice of the sequence $x_n$.

This way we obtain a distance-preserving map $f_\infty\:\spc{X}\to \spc{Y}$. 
It remains to show that $f_\infty$ is surjective; that is, $f_\infty(\spc{X})=\spc{Y}$.

The same argument produces a distance-preserving map $g_\infty\:\spc{Y}\z\to \spc{X}$.
If $f_\infty$ is not surjective, then neither is the composition $f_\infty\z\circ g_\infty\:\spc{Y}\to \spc{Y}$.
So $f_\infty \z\circ g_\infty$ is a distance-preserving map from a compact space to itself which is not an isometry.
The latter contradicts \ref{ex:non-contracting-map}. 
\qeds
















\parit{Proof.}
Suppose $\dist{\spc{X}}{\spc{Y}}{\GH}<r$, let $\spc{X}'$, $\spc{Y}'$, and $\spc{Z}$ are as in \ref{def:GH}.
Let us identify $\spc{X}$ with $\spc{X}'$ and  $\spc{Y}$ with $\spc{Y}'$ by isometries.
Observe that  
\[f(x,y)=\dist{x}{y}{\spc{Z}}\]
defines an appropriate function $f\:\spc{X}\times\spc{Y}\to\RR$ such that 
$a_f,b_f<r$.
It follows that 
\[\dist{\spc{X}}{\spc{Y}}{\GH}\ge\inf_f\{a_f,b_f\}.\]

Now suppose that $f\:\spc{X}\times\spc{Y}\to\RR$ is an appropriate function such that 
$a_f,b_f<r$.
Note that the following defines a semimetric on $\spc{X}\sqcup\spc{Y}$
\[\dist{x}{y}{\spc{X}\sqcup\spc{Y}}=
\begin{cases}
\dist{x}{y}{\spc{X}}&\text{if\ } x,y\in \spc{X},
\\
\dist{x}{y}{\spc{Y}}&\text{if\ } x,y\in \spc{Y},
\\
f(x,y)&\text{if\ } x\in \spc{X}\ \text{and}\ y\in \spc{Y}.
\end{cases}
\]
The corresponding metric space contains isometric copies of $\spc{X}$ and $\spc{Y}$.
It is straightforward to see that the right-hand side in \ref{eq:GH-afbf} is the Hausdorff distance between these copies.
Hence 
\[\dist{\spc{X}}{\spc{Y}}{\GH}\le\inf_f\{a_f,b_f\}.\qedsin\]

















\parbf{\ref{ex:wasserstein}.}
Choose a finite $\eps$-net $F\subset\spc{X}$.
Show that the space $P_F$ of probability measures with support in $F$ is a compact net in $\Wass_1\spc{X}$.
Observe that $\Wass_1\spc{X}$ is complete; 
by \ref{ex:compact-net}, it follows that $\Wass_1\spc{X}$ is compact.

Show that 

Choose an integer $n$.
Consider the set of probability measures $P_n$ of the form 
\[\tfrac1n\cdot\sum_{i=1}^n\delta_{x_i},\]
where $\delta_{x_i}$ denotes the delta-measure supported at $x_i\in\spc{X}$. 

Show that $P_n$ is a compact subset of $\Wass_1\spc{X}$.
Moreover, for any $\eps>0$ there is $n$ such that $P_n$ is 





















\section{Metrics on disjoint union}

Definition~\ref{def:GH} deals with a huge class of metric spaces,
namely, all metric spaces $\spc{Z}$ that contain subspaces isometric to $\spc{X}$ and $\spc{Y}$.
The following proposition shows that it is possible to reduce this class to metrics on the disjoint unions of $\spc{X}$ and $\spc{Y}$. 

\begin{thm}{Proposition}\label{prop:GH=X+Y}
The Gromov--Hausdorff distance between two compact metric spaces $\spc{X}$
and $\spc{Y}$ is the greatest lower bound of $r>0$ such that there exists a metric
$|{*}-{*}|_{\spc{W}}$ on the disjoint union $\spc{W}=\spc{X}\sqcup \spc{Y}$ 
such that $|\spc{X}-\spc{Y}|_{\Haus\spc{W}} < r$
and 
the restrictions of $|{*}-{*}|_{\spc{W}}$ to $\spc{X}$ and $\spc{Y}$
coincide with $|{*}-{*}|_\spc{X}$ and $|{*}-{*}|_{\spc{Y}}$ 
respectively. 
\end{thm}

\parit{Proof.}
Let $r>0$, $\spc{X}$, $\spc{Y}$, $\spc{X}'$, $\spc{Y}'$, and $\spc{Z}$ be as in Definition~\ref{def:GH}.

If  $\spc{X}'$ and $\spc{Y}'$ are disjoint, then we can identify $\spc{X}$ with  $\spc{X}'$ and $\spc{Y}$ with $\spc{Y}'$ and take $\spc{W}$ to be the subspace in $\spc{Z}$ formed by the union $\spc{X}'\cup\spc{Y}'$.

To do the general case,
choose isometries $f\: \spc{X} \to \spc{X}'$ and
$g\: \spc{Y} \z\to \spc{Y}'$, then define the metric by
\begin{align*}
|x-y|_{\spc{W}} &= |f (x)\z- g(y)|_{\spc{Z}}+\eps,
\\
|x-x'|_{\spc{W}} &= |x- x'|_{\spc{X}},
\\
|y-y'|_{\spc{W}} &= |y- y'|_{\spc{Y}},
\end{align*}
for any $x,x' \in \spc{X}$, $y,y' \in \spc{Y}$, and small fixed $\eps>0$.
We need to add $\eps$ to ensure that $|x-y|_{\spc{W}} > 0$ for any $x\in \spc{X}$ and $y\in \spc{Y}$;
so $|x-y|_{\spc{W}}$ is indeed a metric.

Since $\eps$ is small,
this yields a metric on $\spc{W}=\spc{X}\sqcup \spc{Y}$ for which
$|\spc{X}- \spc{Y}|_{\Haus\spc{W}} \z< r$.
\qeds





















\begin{thm}{Exercise}\label{ex:alm-isom}
\begin{subthm}{ex:alm-isom:compositon}
Let $f\:\spc{X}\to \spc{Y}$ and $g\:\spc{Y}\to \spc{Z}$ be two $\eps$-isometries.
Show that $g\circ f\: \spc{X}\to \spc{Z}$ is a $(3\cdot\eps)$-isometry.
\end{subthm}

\begin{subthm}{ex:alm-isom:inverse}
Assume $f\: \spc{X} \z\to \spc{Y}$ is an $\eps$-isometry.
Show that there is a $(3\cdot\eps)$-isometry 
$g\: \spc{Y}\to \spc{X}$.
\end{subthm}

\begin{subthm}{ex:GH=>eps-isom}
 Assume $|\spc{X}-\spc{Y}|_{\GH}<\eps$, show that there is a $(2\cdot\eps)$-isometry 
$f\: \spc{X}\to \spc{Y}$.
\end{subthm}
\end{thm}

\begin{thm}{Proposition}\label{prop:alm-isom=>GH}
Suppose that $f\: \spc{X}\to \spc{Y}$ be an $\eps$-isometry.
Then 
\[|\spc{X}-\spc{Y}|_{\GH}\le 2\cdot \eps.\]
\end{thm}

\parit{Proof.} Consider the set $\spc{W}=\spc{X}\sqcup \spc{Y}$.
Note that the following defines a metric on $\spc{W}$:
\begin{align*}
|x-x'|_{\spc{W}}&=|x-x'|_{\spc{X}},
\\
|y-y'|_{\spc{W}}&=|y-y'|_{\spc{Y}},
\\
|x-y|_{\spc{W}}&=\eps+\inf_{x'\in \spc{X}}\{|x-x'|_{\spc{X}}+|f(x')-y|_{\spc{Y}}\}
\end{align*}
for any $x,x'\in \spc{X}$ and $y,y'\in \spc{Y}$.

Since $f(\spc{X})$ is an $\eps$-net in $\spc{Y}$,
for any $y\in \spc{Y}$ there is $x\in \spc{X}$ such that $|f(x)-y|_{\spc{Y}}\z\le\eps$;
therefore $|x-y|_{\spc{W}}\le 2\cdot\eps$.
On the other hand for any $x\in \spc{X}$, we have $|x-y|_{\spc{W}}\le\eps$
for $y=f(x)\in \spc{Y}$.

It follows that $|\spc{X}-\spc{Y}|_{\Haus\spc{W}}\le 2\cdot\eps$.
\qedsf























\parit{Proof of \ref{GH-4}.}
Choose arbitrary $a,b \in \mathbb{R}$ such that
$$a>|\spc{X}-\spc{Y}|_{\GH}
\quad\text{and}\quad
b>|\spc{Y}-\spc{Z}|_{\GH}.$$
Choose two metrics on $\spc{U}=\spc{X}\sqcup \spc{Y}$ and $\spc{V}=\spc{Y}\sqcup \spc{Z}$ as in \ref{prop:GH=X+Y};
that is,
$|\spc{X}\z-\spc{Y}|_{\Haus\spc{U}}<a$ and $|\spc{Y}-\spc{Z}|_{\Haus\spc{V}}<b$ 
and the inclusions $\spc{X}\hookrightarrow \spc{U}$, $\spc{Y}\hookrightarrow \spc{U}$, $\spc{Y}\hookrightarrow \spc{V}$ and $\spc{Z}\hookrightarrow \spc{V}$ are distance-preserving.

Let $\spc{W}$ be the gluing of $\spc{U}$ and $\spc{V}$ along $\spc{X}$.
Observe that the $a$-neighborhood of $\spc{X}$ in $\spc{W}$ contains $\spc{Y}$ and $b$-neighborhood of $\spc{Y}$ contains $\spc{Z}$.
Therefore, $(a+b)$-neighborhood of $\spc{X}$ in $\spc{W}$ contains $\spc{Z}$.
The same way we can show that $(a+b)$-neighborhood of $\spc{Z}$ in $\spc{W}$ contains $\spc{X}$.
It follows that 
$$a+b\ge|\spc{X}-\spc{Y}|_{\GH}.$$
Whence the statement follows.
\qeds






















\parbf{Exercise~\ref{ex:fillrad<diam/2}.} 
As usual we consider $\spc{M}$ as a subset of $\ell^\infty(\spc{M})$;
more precisely we identify $\spc{M}$ with its image under the map  $x\mapsto \distfun_x$.
Set 
\[R=\tfrac12\cdot\diam\spc{M}.\]

Consider point $p$ in $\ell^\infty(\spc{M})$ that corresponds to the constant function $p(x)=R$.
Since $0\le \distfun_x(z)\le  2\cdot R$ and $0=\distfun_x(x)$, we get that
\[\sup\set{|p(z)-\distfun_x(z)|}{z\in\spc{M}}=R.\]
In other words, the point $p\in \ell^\infty(\spc{M})$ lies on distance $R$ from any point on $\spc{M}\subset \ell^\infty(\spc{M})$.

The linear homotopy $h_t\:\spc{M}\to \ell^\infty(\spc{M})$ defined by
\[h_t(x)=(1-t)\cdot x+t\cdot p\]
contracts $\spc{M}$ to the point $p$.
Note that $\dist{h_t(x)}{x}{\ell^\infty(\spc{M})}\le R$ for any $t\in[0,1]$.
In particular the fundamental class of $\spc{M}$ bounds in the closed $R$-neighborhood of $\spc{M}$ in $\ell^\infty(\spc{M})$.
Whence the statement follows.

%\parbf{Exercise~\ref{ex:fillrad-inj}.} 
%Arguing by contradiction, assume $(n+1)\cdot R<r$ for some $R>\width\spc{M}$.

%Choose a covering $\{V_i\}$ of $\spc{M}$ as in the definition of width (\ref{def:width}), so the dimension of the nerve $\spc{N}$ of the covering is at most $n-1$ and $V_i\subset \oBall(p_i,R)$ for each $i$.
%Suppose $\psi\:\spc{M}\to\spc{N}$ is a map provided by \ref{prop:space->nerve}.

%As usual we denote by $v_i$ the vertex in $\spc{N}$ that corresponds to $V_i$.
%Let us construct a continuous map $f\:\spc{N}\to \spc{M}$ such that $f\circ \psi$ is homotopic to the identity --- note that once it is done the problem is solved.



%Let $\sigma\:W\to \ell^\infty(\spc{M})$ be a map of a simplicial complex as in the proof of \ref{thm:sys<FillRad}.
%That is, the restriction $\sigma|_{\partial W}$ represents the fundamental class $[\spc{T}]$ of $\spc{T}$, the image $\sigma(W)\subset B_{R}(\spc{T})$, and $\sigma$-image of any simplex in $W$ has diameter less then $\eps>0$.

%We may assume that vertexes of $W$ can be divided into $n+2$ levels, say $0,1,\dots,n+1$, so that each simplex has vertexes of different levels.
%This can be achieved by applying barycentric subdivision once.
%Indeed every vertex in the barycentric subdivision corresponds to a simplex in the original triangulation; so we can define level as the dimension of the corresponding simplex.

%Let us map each vertex $v$ of $W$ to a point in $\spc{M}$ closest to $\sigma(v)$; it defines the map on the 0-skeleton $W^0$.

%To extend the map to the higher skeletons we will apply the following {}\emph{cone construction}.
%Suppose the map is defined on the base $\triangle_v$ of a simplex $\triangle$ and the opposite vertex $v$.
%Note that every point $x$ of $\triangle$ lies on a line segment $[vy]$ with $y\in \triangle_v$, say $x=(1-t)\cdot v+t\cdot y$.
%If $f(\triangle_v)$ lies on the distance less than $r$ from $f(v)$, then $f(v)$ is connected to $f(y)$ by a uniqe geodesic path $\gamma_y\:[0,1]\to \spc{M}$ that depends continuously on $y$.
%In this case we can set $f(x)=\gamma_y(t)$.

%We have to arrange the construction in such a way that the simplexes will fit together and to make sure that the image of tha base is $r$-far from the vertex.

%To do so let us apply the cone construction to the edges between 1-level vertex $v_1$ to the 0-level base $v_0$.
%This way we mapped the edge $[v_1v_0]$ to a geodesic path shorter than $2\cdot R'$.
%(If $2\cdot R'<r$, then it is unique, but it is not yet important.)
%It defines the map on all edges of $W$ between vertexes on level $0$ and $1$.

%Now let us extend the map to each triangle with vertex $v_2$ on level 2 and base edge $[v_1v_0]$ with vertexes of level 1 and 0.
%The $f$image of the base has length less than $2\cdot R'$  and the distance from $f(v_2)$ to $f(v_1)$ and $f(v_0)$ is also less than $2\cdot R'$.
%Therefore for every $y\in [v_1v_0]$, we have $\dist{f(v)}{f(y)}{\spc{M}}<3\cdot R'$.
%So if $3\cdot R'<r$, then we can apply cone construction to extend the map to each triangle with level of the vertexes $0,1,$ and $2$.

%Further let us extend the map to each triangle with vertex $v_3$ on level 3 and base edge $\triangle$ with vertexes $v_0,v_1,v_2$ of level 0,1, and 2.
%Any point $x$ in the base lies in a triangle with vertexes $v_3,v_2$ and a point $z$ on the edge $[v_0v_1]$.
%From above we have that $\dist{f(v_3)}{f(z)}{\spc{M}}<3\cdot R'$, $\dist{f(v_2)}{f(z)}{\spc{M}}<3\cdot R'$, and $\dist{f(v_3)}{f(v_2)}{\spc{M}}<2\cdot R'$,
%By triangle inequality $\dist{f(v_3)}{f(x)}{\spc{M}}<4\cdot R'$.
%Therefore, if $4\cdot R'<r$ we can extend $f$ to each triangle with the levels of vertexes $0,1,$ and $2$.

%Continuing this way we get that we can extend the map to whole $W$ if  $(n+1)\cdot R'<r$.
%In this case the fundamental class of $\spc{M}$ bounds in $\spc{M}$  which is not possible.
%That is, $(n+1)\cdot R'\ge r$; since $\eps>0$ and $R>\FillRad \spc{M}$ are arbitrary, we get
%\[\FillRad \spc{M}\ge r.\]
















Let us describe a more conceptual way to define essential manifolds.

Suppose $\pi$ is a group,
then we say that a connected topological space $\spc{K}$ is \emph{$K(\pi,1)$} (or, more precicely, \emph{Eilenberg--MacLane space of type $K(\pi,1)$})
if $\pi_1(\spc{K})=\pi$ and $\pi_n(\spc{K})=0$ for $n\ne 1$;
here $\pi_n(\spc{K})$ denotes the $n$-th homotopy group of $\spc{K}$.

A CW-complex $\spc{K}$ that is a $K(\pi,1)$ space can be constructed the following way.
Suppose the group $\pi$ is given via set of generators and a set of relations. 
\begin{itemize}
\item Start with the wedge sum of circles, one for each generator.
We obtain the 1-skeleton $\spc{K}^1$ of the complex; its fundamental group is free.

\item Attach a disc for to kill each relation in the fundamental group of the space.
This way we obtain a 2-dimensional CW-complex $\spc{K}^2$ with fundamental group $\pi$.

\item If $\pi_2(\spc{K}^2)\ne0$, chose a set of generators in $\pi_2(\spc{K}^2)$ and attach a $3$-disc to each a 2-spheroid in $\spc{K}^2$ for each generator.
This way we obtain a 3-dimensional CW-complex $\spc{K}^3$ such that  $\pi_1\spc{K}^3=\pi$ and $\pi_2\spc{K}^3=0$.


\item Do the same to kill $\pi_3(\spc{K}^2)\ne0$ and continue.
The process might terminate after finite number of steps, but typically it goes forever.
\end{itemize}
At the end you get $K(\pi,1)$ CW-complex  $\spc{K}=\spc{K}^\infty$.
It proves the existence.
The proof of the following observation is an exercise in topology.

\begin{thm}{Observations}

\begin{subthm}{}
The group $\pi$ defines a $K(\pi,1)$ space up to a weak homotopy equivalence.
In particular, the homologies of a $K(\pi,1)$ space are completely determined by $\pi$.
\end{subthm}


\begin{subthm}{}
Suppose $\spc{K}$ is a $K(\pi,1)$ space and $\spc{L}$ is a connected finite CW-complex.
Then any homomorphism $\pi_1\spc{L}\to\pi_1\spc{K}$ is induced by a continuous map $\phi\:\spc{L}\to\spc{K}$.
Moreover, $\phi$ is uniquely defined up to homotopy equivalence.
\end{subthm}

\end{thm}

The following proposition provides an alternative definition of essential manifold, it follows from the observations above.

\begin{thm}{Proposition}
Suppose $\spc{M}$ is a closed manifold, 
$\spc{K}$ is a $K(\pi,1)$ space and a map $\iota\:\spc{M}\to \spc{K}$ induces an isomorphism of fundamental groups.
Then $\spc{M}$ is essential if and only if $\iota$ sends the fundamental class of $\spc{M}$ to a nonzero homology class in $\spc{K}$.
\end{thm}















Note that in this case the balls $B(p_i,W)$, $B(p_j,W)$, and $B(p_k,W)$
have a common point, say $z$.
Choose geodesics $[zp_i]$, $[z_pj]$ and $[zp_k]$.
Observe that perimeter of each triangle $[p_ip_jz]$, $[p_jp_kz]$, and $[p_kp_iz]$ is smaller than $s$.
Therefore the contour of each traingle is contractible.
It follows that contour of the triangle $[p_ip_jp_k]$ is contractible.













\section{Filling radius}

The following definition was introduced by Mikhael Gromov \cite{gromov-1983}.

Let $\spc{M}$ be a closed $n$-dimensional Reimannian manifold.
Applying Kuratowski embedding (\ref{lem:kuratowski}) $x\mapsto \distfun_x$, we may think that $\spc{M}$ as a subset of $\ell^\infty(\spc{M})$ --- the space of functions on $\spc{M}$ equipped with the metric induced by the sup-norm.

Define the \emph{filling radius} of $\spc{M}$ (briefly $\FillRad\spc{M}$) as the least upper bound on values $r>0$ such that $\spc{M}$ bounds in its $r$-neighborhood in $\ell^\infty(\spc{M})$.
In other words, if $\iota_r$ denotes inclusion of $\spc{M}$ in its $r$-neighborhood $B_r(\spc{M})\subset \ell^\infty(\spc{M})$,
then 
\[\FillRad\spc{M}\df\inf\set{r>0}{(\iota_r)_*[\spc{M}]=0\in H_n(B_r(\spc{M}))},\]
where $[\spc{M}]$ denotes the fundamental class of $\spc{M}$.

We assume that the homologies are taken with coefficients in $\ZZ_2$.
In this case $[\spc{M}]\ne0\in H_n(\spc{M})$.
If we choose coefficients $\ZZ$, then it does not hold for nonorientable manifolds.


\begin{thm}{Exercise}\label{ex:fillrad<diam/2}
Show that the inequality
\[\FillRad \spc{M}\le \tfrac12\cdot\diam \spc{M}\]
holds for any compact Riemannian manifold $\spc{M}$.
\end{thm}

\parbf{Remark.}
The optimal bound for the above exercise was found by Mikhail Katz \cite{katz}.
Namely he proved that
\[\FillRad \spc{M}\le \tfrac13\cdot\diam \spc{M}\]
and equality holds if $\spc{M}$ is real projective space with canonical metric.
The proof is beautiful, elementary, and very readable.

\medskip

The following theorem is the main ingredient in the proof of \ref{thm:sys(torus)+}.
This theorem will be the main subject of the following lecture.

\begin{thm}{Theorem}\label{thm:FillRad<vol}
Given an integer $n>0$, there is a constant $c_n$ such that inequality
\[\FillRad \spc{M}\le c_n\cdot \sqrt[n]{\vol \spc{M}}\]
holds for any compact $n$-dimensional Riemannian manifold $\spc{M}$.
\end{thm}

In the following section we show why this theorem is related to \ref{thm:FillRad<vol}.





















that is affine on each simplex and $f\:v_i\mapsto p_i$ for each $i$.
Observe that $\dist{x}{f\circ\psi(x)}{\ell^\infty( \spc{T})}< R$.
It follows since (1) $f\circ\psi$ maps $x$ to the (linear) convex hull of $p_i$ such that $\dist{x}{p_i}{}<R$ and (2) the $R$-ball around $x$ is convex in $\ell^\infty( \spc{T})$.

The linear homotopy $s_t(x)=(1-t)\cdot x+t\cdot f\circ\psi(x)$ deforms the identity map on $M$ to $f\circ\psi$.
Since $\dist{x}{f\circ\psi(x)}{\ell^\infty( \spc{T})}< R$,
this homotopy does not escapes from $R$-neighborhood of $\spc{T}$.
Since $\psi$ kills the fundamental class of $\spc{T}$ so does $f\circ\psi$.
Hence $\FillRad \spc{T}<R$.


As usual we consider $\spc{T}$ as a subspace in $\ell^\infty(\spc{T})$.




Choose a simplicial complex $W$ and a map $\sigma\:W\to \ell^\infty(\spc{T})$ such that the restriction $\sigma|_{\partial W}$
represents the fundamental class $[\spc{T}]$ of $\spc{T}$
and $\sigma(W)\subset B_{R}(\spc{T})$.


Passing to barycentric subdivision few times, we may assume that the $\sigma$-image of any simplex in $W$ has diameter less than $\eps$.
We may perturb the map slightly to ensure that each edge $e$ of $W$ is mapped to a geodesic and still $\sigma|_{\partial W}$
represents the fundamental class $[\spc{T}]$ of $\spc{T}$.

Let us construct a continuous map
$f\:W\to  \spc{T}$ which agrees with $\sigma$ on $\partial W$.
Once it is done we get that the fundamental class of $\spc{T}$ vanish in $ H_n(\spc{T})$ --- a contradiction.














\begin{thm}{Exercise}\label{ex:FillRad<width}
Show that for any closed Riemannian manifold $\spc{M}$ we have
\[\FillRad \spc{M}\le 100\cdot \width\spc{M};\]
try to show that in fact
\[\FillRad \spc{M}\le \width\spc{M}.\]

\end{thm}


\parbf{Exercise~\ref{ex:FillRad<width}.}
Suppose $n=\dim \spc{M}$.
Let $\spc{N}$ be the nerve of a covering $\{V_i\}$ of $\spc{M}$ and $\psi\:\spc{M}\to\spc{N}$ be the map provided by \ref{prop:space->nerve}.

Assume we chose $\{V_i\}$ as in the definition of width (\ref{def:width}).
For each $i$ choose a point $p_i\in \spc{M}$ such that $V_i\subset \oBall(p_i,R)$.
Observe that in this case $\dim\spc{N}<n$;
therefore $\psi$ maps the fundamental class $[\spc{M}]$ to zero in $H_n(\spc{N})$.

As usual we think that $\spc{M}$ is isometrically embedded into $\ell^\infty( \spc{M})$, and we assume that $v_i$ is a vertex of $\spc{N}$ that corresponds to $V_i$.
Consider the map $f\:\spc{N}\to \ell^\infty( \spc{M})$ that is affine on each simplex and $f\:v_i\mapsto p_i$ for each $i$.
Observe that $\dist{x}{f\circ\psi(x)}{\ell^\infty( \spc{M})}< R$.
It follows since (1) $f\circ\psi$ maps $x$ to the (linear) convex hull of $p_i$ such that $\dist{x}{p_i}{}<R$ and (2) the $R$-ball around $x$ is convex in $\ell^\infty( \spc{M})$.

The linear homotopy $s_t(x)=(1-t)\cdot x+t\cdot f\circ\psi(x)$ deforms the identity map on $M$ to $f\circ\psi$.
Since $\dist{x}{f\circ\psi(x)}{\ell^\infty( \spc{M})}< R$,
this homotopy does not escapes from $R$-neighborhood of $\spc{M}$.
Since $\psi$ kills the fundamental class of $\spc{M}$ so does $f\circ\psi$.
Hence $\FillRad \spc{M}<R$.











\parit{Proof.}
Set $n+1=\dim \spc{M}$.
Choose $R>\width\spc{M}$.
Let $\psi\:\spc{M}\to \spc{N}$ be a map to an $n$-dimensional simpleicial complex $\spc{N}$ provided by \ref{prop:width=suprad(inv)}.

Given $x\in \spc{M}$ set $[x]=\psi^{-1}\{\psi(x)\}$.
Given $x\in \spc{M}$, consider the function $f_x\;\spc{M}\to\RR$ defined by
\[f_x(y)=\distfun_{[x]}(y)-R.\]
Note that $\diam [x]\le 2\cdot R$; therefore
\[\distfun_{[x]}(y)+2\cdot R\ge\distfun_x(x)\ge \distfun_{[x]}(y);\]
moreover if $x=y$, then the second inequality is an equality.
Whence \[\sup\set{|f_x(y)-\distfun_x(y)|}{y\in\spc{X}}=R\] 
for any $x\in \spc{M}$.
In other words, the map $f\:x\mapsto f_x\in\ell^\infty(\spc{M})$ moves each point by distance $R$.
In particular the homotopy $h_t\:\spc{M}\to\ell^\infty(\spc{M})$ 
defined by $h_t(x)=(1-t)\cdot x+ t\cdot f_x$ deforms $\spc{M}$ into a 
Since $f$ is factorized thru $\spc{N}$



Set $n=\dim \spc{M}$.
Choose $R>\width\spc{M}$.

Suppose $\{V_1,\dots, V_k\}$ is an open cover of $\spc{M}$ that meets the definition of width;
that is, the multiplicity of the cover is at most $n$ and for each $V_i$ there is a point $p_i\in  V_i$ such that $V_i\subset \oBall(p_i,R)$.

Further for each simplex $\triangle$ in $\spc{N}$ with the vertexes $v_{i_1},\dots,v_{i_m}$ choose a point 

Let $\spc{N}$ be the nerve of the covering;
denote by $v_i$ the vertex of $\spc{N}$ that corresponds to $V_i$.
Since multiplicity of $\{V_i\}$ is at most $n$, we have that $\dim\spc{N}\le n-1$. 
Let $\bm{\psi}\:\spc{M}\to\spc{N}$ be the map provides by \ref{prop:space->nerve}.

Consider the map $g\:\spc{N}\to\ell^\infty(\spc{M})$ that sends $v_i$ to $p_i$ and affine on each simplex.
Consider the linear homotopy \[h_t(x)\z=(1-t)\cdot x+ t\cdot g\circ \bm{\psi}(x).\]
Observe that $h_t(x)$ lies in a $R$-neighborhood of $\spc{M}$ in $\ell^\infty(\spc{M})$.
Indeed, if $\psi_i(x)>0$, then $\dist{p_i}{x}{}<R$.
In particular, if $n(x)$ denotes the number of functions $\psi_i$ such that $\psi_i(x)>0$, then  
\[\sum_{\set{i}{\psi_i(x)>0}} \dist{p_i}{x}{}\le n(x)\cdot R.\]
That is average distance from $x$ to $p_i$ such that  $\psi_i(x)>0$ is at most $R$.
Therefore the distance from $x$ to one of the points $p_i$ is at most $R$.
In particular 
\qeds


















\section{Essential manifolds}




\chapter{Other metrics on manifolds}

\section{Finsler metrics}

\section{Sub-Riemannian metrics}

\section{Hausdorff measure}

\section{On semicontinuity of volume}

\section{Lipschitz homeomorphism}




















\begin{thm}{Area formula}\label{thm:area-formula}
Let $\bm{f}\:K\to\RR^n$ be a smooth map defined on a compact set $K\subset \RR^n$.
Then for any function $h\:K\to \RR$
\[\int_K h(\bm{x})\cdot \jac_{\bm{x}}\bm{f}\cdot d\bm{x}=\int_{\bm{f}(K)} H_K(\bm{y})\cdot d\bm{y},\]
where 
\[H_K(\bm{y})=\sum_{\substack{\bm{x}\in K \\ \bm{f}(\bm{x})=\bm{y}}}h(\bm{x}).\]
(The integrals are understood in the sense of Lebesgue.)
\end{thm}

Let us sketch the proof of area formula using  Sard's lemma (\ref{lem:sard}) and the substitution rule (\ref{thm:mult-substitution}).

\parit{Sketch of proof.}
Denote by $S\subset K$ the set of critical points of $\bm{f}$; that is, $\bm{x}\in S$ if $\jac_{\bm{x}}\bm{f}=0$.
By Sard's lemma,
\emph{$\bm{f}(S)$ has vanishing measure}.
Note that 
\[\int_S h(\bm{x})\cdot \jac_{\bm{x}}\bm{f}\cdot d\bm{x}=0\]
since $\jac_{\bm{x}}\bm{f}=0$
and
\[\int_{\bm{f}(S)} H_S(\bm{y})\cdot d\bm{y}\]
since $\bm{f}(S)$ has vanishing measure.
In particular,
\[\int_S h(\bm{x})\cdot \jac_{\bm{x}}\bm{f}\cdot d\bm{x}=\int_{\bm{f}(S)} H_S(\bm{y})\cdot d\bm{y};\]
that is the area formula holds for $S$.

It remains to prove that 
\[\int_{K\setminus S} h(\bm{x})\cdot \jac_{\bm{x}}\bm{f}\cdot d\bm{x}=\int_{\bm{f}(K\setminus S)} H_{K\setminus S}(\bm{y})\cdot d\bm{y}.\]
Since $\jac_{\bm{x}}\bm{f}\ne 0$ for any $\bm{x}\in K\setminus S$, by inverse function theorem, the restriction of $\bm{f}$ to a neighborhood $U\ni\bm{x}$ has a smooth inverse.
Therefore for any compact set $K'\subset U$ we have that
\[\int_{K'}h(\bm{x})\cdot \jac_{\bm{x}}\bm{f}\cdot d\bm{x}=\int_{\bm{f}(K_1)} h(\bm{f}^{-1}(\bm{y}))\cdot d\bm{y}.\]
It remains to subdivide $K_1$ into a countable collection of subsets of that type and sum up the corresponding formulas.\qeds

















\newpage
\pagestyle{empty}

\noindent Suppose that $\rho$ is a positive continuous function on a complete metric space $\spc{X}$.
Show that for any $\eps>0$ there is a point $x\in \spc{X}$ such that 
\[\rho(x)<(1+\eps)\cdot\rho(y)\]
for any point $y\in \oBall(x,\rho(x))$.

\vskip 50mm

\noindent Let $\spc{K}$  be a compact metric space and
\[f\:\spc{K}\z\to \spc{K}\] 
be a distance-nondecreasing map.
Prove that $f$ is an isometry.

\vskip 50mm

\noindent 
Suppose $(\spc{X},\dist{*}{*}{})$ is a complete metric space.
Show that $(\spc{X},\yetdist{*}{*}{})$ is complete.

\newpage
\pagestyle{empty}

\noindent Show that any compact metric space is isometric $\spc{K}$ to a subspace of a compact geodesic space. 

\vskip 50mm

\noindent Let $\spc{C}$ be a subspace of $\mathcal{H}(\RR^2)$ formed by all compact convex subsets in $\RR^2$.
Show that perimeter and area are continuous on~$\spc{C}$.

That is, if a sequence of convex compact plane sets $X_n$ converges to $X_\infty$ in the sense of Hausdorff, then 
\[\perim X_n\to \perim X_\infty\quad\text{and}\quad\area X_n\to\area X_\infty\]
as $n\to\infty$.
(If the set degenerates to a line segment of length $\ell$, then its perimeter is defined as $2\cdot \ell$.)

\vskip 50mm

\noindent Let $\spc{Q}(C,D)$ be the set of all the compact metric spaces with diameter at most $D$ that admit a $C$-doubling measure.
Show that $\spc{Q}(C,D)$ is totally bounded.

\newpage
\pagestyle{empty}

\noindent
\begin{subthm}{pr:under:if}Let $\spc{Y}$ be a compact metric space.
Show that the set of all spaces $\spc{X}$ such that $\spc{X}\le\spc{Y}$
is uniformly totally bounded.
\end{subthm}

\begin{subthm}{pr:under:only-if}
Show that for any uniformly totally bounded set $\spc{Q}\subset\spc{M}$ there is a compact space $\spc{Y}$
such that $\spc{X}\le\spc{Y}$ for any $\spc{X}$ in $\spc{Q}$.
\end{subthm}

\stepcounter{thm}

\vskip 50mm

\noindent \begin{subthm}{}
Show that a sequence of compact simply connected length spaces cannot converge to a circle.
\end{subthm}

\begin{subthm}{ex:GH-SC:nonsc-limit}
Construct a sequence of compact simply connected length spaces that converges to a compact nonsimply connected space.
\end{subthm}

\stepcounter{thm}

\vskip 50mm

\noindent \begin{subthm}{ex:sphere-to-ball:2}
Show that a sequence of lenght metrics on the 2-sphere cannot converge to a the unit disc.
\end{subthm}

\begin{subthm}{ex:sphere-to-ball:3}
Construct a sequence of lenght metrics on the 3-sphere that converges to a unit 3-ball.
\end{subthm}

\stepcounter{thm}

\newpage
\pagestyle{empty}

\noindent For any point $x\in \spc{X}$, consider the constant sequence $x_n=x$
and set $\iota(x)=\lim_{n\to\omega}x_n\in\spc{X}^\omega$.

\begin{subthm}{ex:ultrapower:a}
Show that $\iota\:\spc{X}\to\spc{X}^\omega$ is distance-preserving embedding.
\end{subthm}

\begin{subthm}{ex:ultrapower:compact} 
Show that $\iota$ is onto if and only if $\spc{X}$ compact.
\end{subthm}

\stepcounter{thm}


\vskip 50mm

\noindent Assume $\spc{X}$ is a complete length space 
and $p,q\in\spc{X}$ cannot be joined by a geodesic in $\spc{X}$.  
Then there are at least two distinct geodesics between $p$ and $q$ 
in the ultrapower $\spc{X}^\omega$.

\vskip 50mm

\noindent Show that an ultralimit of metric trees is a metric tree. 

\newpage
\pagestyle{empty}

\noindent Show that any two distinct points in an Urysohn space can be jointed by infinite number of geodesics.

\vskip 50mm

\noindent 
Let $K$ be a compact set in a separable space $\spc{S}$.
Then any distance-preserving map from $K$ to an Urysohn space can be extended to 
a distance-preserving map on whole $\spc{S}$.


\vskip 50mm

\noindent Let $S$ be a sphere of radius $\tfrac d2$ in $\spc{U}_d$;
that is, 
\[S=\set{x\in \spc{U}_d}{\dist{p}{x}{\spc{U}_d}=\tfrac d2}\]
for some point $p\in \spc{U}_d$.
Show that $S$ is isometric to $\spc{U}_d$.

Use it to show that $\spc{U}_d$ is not countable-set homogeneous;
that is, there is an distance-preserving map from a countable subset of $\spc{U}_d$ to $\spc{U}_d$ that cannot be extended to an isometry of $\spc{U}_d$.

\newpage
\pagestyle{empty}

\noindent Let $r$ and $s$ be two extremal functions of a metric space $\spc{X}$.
Suppose that $r\ge s-c$ for some constant $c$.
Show that $c\ge 0$ and $r\le s+c$.

\vskip 50mm

\noindent Suppose that a metric space $\spc{X}$ satisfies the following property:

For any subspace $\spc{A}$ in $\spc{X}$ and any other metric space $\spc{Y}$, any short map $f\:\spc{A}\to \spc{Y}$ can be extended to a short map $F\:\spc{X}\to \spc{Y}$.

Show that $\spc{X}$ is an ultrametric space;
that is, the following strong version of triangle inequality
\[\dist{x}{z}{\spc{X}}
\le
\max\{\,\dist{x}{y}{\spc{X}},\dist{y}{z}{\spc{X}}\,\}\]
holds for any three points $x,y,z\in \spc{X}$.

\vskip 50mm

\noindent Show that the following spaces are injective:
\begin{subthm}{ex:injective-spaces:R}
the real line;
\end{subthm}

\begin{subthm}{ex:injective-spaces:tree}
complete metric tree;
\end{subthm}

\begin{subthm}{ex:injective-spaces:ell-infty}
plane with the metric induced by $\ell^\infty$-norm.
\end{subthm}

\stepcounter{thm}

\newpage
\pagestyle{empty}

\noindent Suppose that $\spc{X}$ is
\begin{subthm}{ex:tripod+square:tripod} 
a metric space with exactly tree points $a,b,c$ such that 
\[\dist{a}{b}{\spc{X}}=\dist{b}{c}{\spc{X}}=\dist{c}{a}{\spc{X}}=1.\]
\end{subthm}
\begin{subthm}{ex:tripod+square:square}
 a metric space with exactly four points $p,q,x,y$ such that 
\[\dist{p}{x}{\spc{X}}=\dist{p}{y}{\spc{X}}=\dist{q}{x}{\spc{X}}=\dist{q}{x}{\spc{X}}=1\]
and
\[\dist{p}{q}{\spc{X}}=\dist{x}{y}{\spc{X}}=2.\]
\end{subthm}
Describe the set of all extremal functions on $\spc{X}$ and the metric space $\Inj \spc{X}$ in each case.

\vskip 50mm

\noindent Let $\spc{X}$ be a compact space.
Show that for any two points $f,g\in\Inj \spc{X}$ there are points $p,q\in \spc{X}$
such that 
\[\dist{p}{f}{\Inj\spc{X}}+\dist{f}{g}{\Inj\spc{X}}+\dist{g}{q}{\Inj\spc{X}}=\dist{p}{q}{\Inj\spc{X}}.\]
























\begin{thm}{Uniqueness of geodesics}\label{thm:cat-unique}
In a proper length $\CAT(0)$ space, pairs of points are joined by unique geodesics, and these geodesics depend continuously on their endpoint pairs.

Analogously, in a proper length $\CAT(1)$ space, pairs of points at distance less than $\pi$ are joined by unique geodesics, and these geodesics depend continuously on their endpoint pairs.
\end{thm}

\parit{Proof.} 
Given 4 points $p^1,p^2,q^1,q^2$ in a proper length $\CAT(0)$ space $\spc{U}$, 
consider two triangles $\trig{p^1}{q^1}{p^2}$ and $\trig{p^2}{q^2}{q^1}$.
Since both of these triangles are thin, we get 
\begin{align*}
\dist{\geodpath_{[p^1q^1]}(t)}{\geodpath_{[p^2q^1]}(t)}{\spc{U}}
&\le (1-t)\cdot \dist{p^1}{p^2}{\spc{U}},
\\
\dist{\geodpath_{[p^2q^1]}(t)}{\geodpath_{[p^2q^2]}(t)}{\spc{U}}
&\le t\cdot \dist{q^1}{q^2}{\spc{U}}.
\intertext{By the triangle inequality,}
\dist{\geodpath_{[p^1q^1]}(t)}{\geodpath_{[p^2q^2]}(t)}{\spc{U}}&\le \max\{\dist{p^1}{p^2}{\spc{U}},\dist{q^1}{q^2}{\spc{U}}\}.
\end{align*}

This implies continuity and uniqueness in the $\CAT(0)$ case.  
 
The $\CAT(1)$ case is done in essentially the same way.
\qeds

Adding the first two inequalities of the preceding proof gives the following:

\begin{thm}{Proposition}
Suppose $p^1,p^2,q^1,q^2$ are points in a proper length $\CAT(0)$ space~$\spc{U}$.
Then 
\[\dist{\geodpath_{[p^1q^1]}(t)}{\geodpath_{[p^2q^2]}(t)}{\spc{U}}\]
is a convex function.
\end{thm}

\begin{thm}{Corollary}\label{cor:dist-convex}
Let $K$ be a closed convex subset in a proper length $\CAT(0)$ space~$\spc{U}$.
Then $\dist{K}{}{}\:\spc{U}\to\RR$ is \index{convex function}\emph{convex};
that is, the function $t\mapsto\dist{K}{}{}\circ\gamma$ is convex for any geodesic $\gamma$ in $\spc{U}$.

In particular, $\dist{p}{}{}$ is convex for any point $p$ in~$\spc{U}$.
\end{thm}


\begin{thm}{Corollary}\label{cor:contractible-cat}
Any proper length $\CAT(0)$ space is contractible.

Analogously, any proper length $\CAT(1)$ space with diameter $<\pi$ is contractible.
\end{thm}

\parit{Proof.} Let $\spc{U}$ be a proper length $\CAT(0)$ space.
Fix a point $p\in \spc{U}$.

For each point $x$ consider the geodesic path $\gamma_x\:[0,1]\to \spc{U}$ from $p$ to~$x$.
Consider the one parameter family of maps 
$h_t\:x\mapsto \gamma_x(t)$ for $t\in [0,1]$.
By uniqueness of geodesics (\ref{thm:cat-unique}), the map 
$(t,x)\mapsto h_t(x)$ is continuous. The family $h_t$ is called a \index{geodesic homotopy}\emph{geodesic homotopy}.

It remains to note that $h_1(x)=x$ and $h_0(x)=p$ for any~$x$.

The proof of the $\CAT(1)$ case is identical.
\qeds

\begin{thm}{Proposition}\label{cor:loc-geod-are-min}
Suppose $\spc{U}$ is a proper length $\CAT(0)$ space.  
Then any local geodesic in $\spc{U}$ is a geodesic.

Analogously, if $\spc{U}$ is a proper length $\CAT(1)$ space, then any local geodesic in $\spc{U}$ which is shorter than $\pi$ is a geodesic.
\end{thm}

\begin{wrapfigure}{r}{21mm}
\begin{lpic}[t(-0mm),b(0mm),r(0mm),l(0mm)]{pics/local-geod(1)}
\lbl[t]{2.5,1;$\gamma(0)$}
\lbl[b]{10,14;$\gamma(a)$}
\lbl[t]{19,8;$\gamma(b)$}
\end{lpic}
\end{wrapfigure}

\parit{Proof.}
Suppose $\gamma\:[0,\ell]\to\spc{U}$ is a local geodesic that is not a geodesic.
Choose $a$ to be the maximal value 
such that $\gamma$ is a geodesic on $[0,a]$.
Further choose $b>a$ so that $\gamma$ is a geodesic on $[a,b]$.

Since the triangle $\trig{\gamma(0)}{\gamma(a)}{\gamma(b)}$ is thin and 
$\dist{\gamma(0)}{\gamma(b)}{}<b$ we have
\[\dist{\gamma(a-\eps)}{\gamma(a+\eps)}{}<2\cdot\eps\]
for all small~$\eps>0$.
That is, $\gamma$ is not length-minimizing on the interval $[a-\eps,a+\eps]$ for any $\eps>0$,
a contradiction.

The spherical case is done in the same way.
\qeds


\begin{thm}{Exercise}\label{ex:geod-CBA}
Assume $\spc{U}$ is a proper length $\CAT(\kappa)$ space
 with extendable geodesics;
that is, any geodesic is an arc in a local geodesic $\RR\to \spc{U}$.

Show that the space of geodesic directions at any point in $\spc{U}$ is complete.

Does the statement remain true if $\spc{U}$ is complete, but not required to be proper?
\end{thm}

Now let us formulate the main result of this section.


\begin{wrapfigure}[6]{r}{28mm}
\begin{lpic}[t(-4mm),b(6mm),r(0mm),l(0mm)]{pics/lem_alex1(1)}
\lbl[lb]{10,23;$y$}
\lbl[rt]{1.5,.5;$p$}
\lbl[bl]{25,7.5;$x$}
\lbl[lb]{17,15;$z$}
\end{lpic}
\end{wrapfigure}

\begin{thm}{Inheritance lemma}
\label{lem:inherit-angle} 
Assume that a triangle $\trig p x y$ 
in a metric space is \index{decomposed triangle}\emph{decomposed} 
into two triangles $\trig p x z$ and $\trig p y z$;
that is, $\trig p x z$ and $\trig p y z$ have a common side $[p z]$, and the sides $[x z]$ and $[z y]$ together form the side $[x y]$ of $\trig p x y$.

If both triangles $\trig p x z$ and $\trig p y z$ are thin, 
then the triangle $\trig p x y$ is also thin.

Analogously, if $\trig p x y$ has perimeter $<2\cdot\pi$ and both triangles $\trig p x z$ and $\trig p y z$ are spherically thin, then triangle $\trig p x y$ is spherically thin.
\end{thm} 


\begin{wrapfigure}{r}{32mm}
\begin{lpic}[t(-4mm),b(0mm),r(0mm),l(0mm)]{pics/cat-monoton-ineq(1)}
\lbl[b]{14,23;$\dot z$}
\lbl[t]{10,.5;$\dot p$}
\lbl[r]{1,14;$\dot x$}
\lbl[l]{30.5,14;$\dot y$}
\lbl[tl]{13,13;$\dot w$}
\end{lpic}
\end{wrapfigure}

\parit{Proof.}
Construct  the model triangles $\trig{\dot p}{\dot x}{\dot z}\z=\modtrig(p x z)_{\EE^2}$ 
and $\trig {\dot p} {\dot y} {\dot z}\z=\modtrig(p y z)_{\EE^2}$ so that $\dot x$ and $\dot y$ lie on opposite sides of $[\dot p\dot z]$.

Let us show that 
\[\angk{z}{p}{x}+\angk{z}{p}{y}
\ge
\pi.
\eqlbl{eq:<+<>=pi}\]
Suppose the contrary, that is
\[\angk{z}{p}{x}+\angk{z}{p}{y}
<
\pi.\]
Then for some point $\dot w\in[\dot p\dot z]$, we have \[\dist{\dot x}{\dot w}{}+\dist{\dot w}{\dot y}{}
<
\dist{\dot x}{\dot z}{}+\dist{\dot z}{\dot y}{}=\dist{x}{y}{}.\]
Let $w\in[p z]$ correspond to $\dot w$; that is, $\dist{z}{w}{}=\dist{\dot z}{\dot w}{}$. 
Since $\trig p x z$ and $\trig p y z$ are thin, we have 
\[\dist{x}{w}{}+\dist{w}{y}{}<\dist{x}{y}{},\]
contradicting the triangle inequality. 

Denote by $\dot D$ the union of two solid triangles $\trig {\dot p}{\dot x}{\dot z}$ and $\trig {\dot p} {\dot y} {\dot z}$.
Further, denote by $\tilde D$ the solid triangle $\trig{\tilde  p}{\tilde  x}{\tilde  y}=\modtrig(p x y)_{\EE^2}$.
By \ref{eq:<+<>=pi}, there is a short map $F\:\tilde D\to \dot D$ that sends 
\begin{align*}
\tilde p&\mapsto \dot p,
&
\tilde x&\mapsto \dot x,
&
\tilde z&\mapsto \dot z,
&
\tilde y&\mapsto \dot y.
\end{align*}
\qedsf

\begin{thm}{Exercise}\label{ex:short-map}
Use Alexandrov's lemma (\ref{lem:alex}) to prove the last statement. 
\end{thm}


By assumption, the natural maps $\trig {\dot p} {\dot x} {\dot z}\to\trig p x z$ and $\trig {\dot p} {\dot y} {\dot z}\to\trig p y z$ are short.  
By composition,  the natural map from $\trig{\tilde  p}{\tilde  x}{\tilde  y}$ to $\trig p y z$ is short, as claimed.

The spherical case is done along the same lines.
\qeds

\begin{thm}{Exercise}\label{ex:convex-balls}
Show that any ball in a proper length $\CAT(0)$ space is a convex set.

Analogously, show that any ball of radius $R<\tfrac\pi2$ in a proper length $\CAT(1)$ space  is a convex set.
\end{thm}

Recall that a set $A$ in a metric space $\spc{U}$ is called locally convex if for any point $p\in A$ there is an open neighborhood $\spc{U}\ni p$ such that any geodesic in $\spc{U}$ with  ends in $A$ lies in~$A$. 

\begin{thm}{Exercise}\label{ex:locally-convex}
Let $\spc{U}$ be a proper length $\CAT(0)$ space.
Show that any closed, connected, locally convex set in $\spc{U}$ is convex.
\end{thm}

\begin{thm}{Exercise}\label{ex:closest-point}
Let  $\spc{U}$ be a proper length $\CAT(0)$ space 
and $K\subset \spc{U}$ be a closed convex set.
Show that: 

\begin{subthm}{ex:closest-point:a}
For each point $p\in \spc{U}$ there is unique point $p^*\in K$ that minimizes the distance $\dist{p}{p^*}{}$.
\end{subthm}

\begin{subthm}{}
The closest-point projection $p\mapsto p^*$ defined by \ref{SHORT.ex:closest-point:a} is short. 
\end{subthm}

\end{thm}




















\begin{thm}{Advanced exercise}\label{ex:urysohn-contractable}
 Show that the space $\spc{U}$ is contactable.
\end{thm}


\parbf{Advanced exercise~\ref{ex:urysohn-contractable}.}
Note that points in the space $\spc{X}_\infty$ constructed in the proof of \ref{prop:univeral-separable} can be multiplied number $t\in [0,1]$ --- simply multiply each function by factor $t$.
That defines a map 
\[\lambda_t\:\spc{X}_\infty\to \spc{X}_\infty\]
that scales all distances by factor $t$.
The map $\lambda_t$ can be extended to the completion of $\spc{X}_\infty$, which is isometic to $\spc{U}_d$ (or $\spc{U}$).

Observe that 
the map $\lambda_1$ is the identity  and $\lambda_0$ maps whole space to a single point, say $x_0$ --- that is the only point of $\spc{X}_0$.
Further note that the map $(t,p)\mapsto \lambda_t(p)$ is continuous ---  in particular $\spc{U}_d$ and $\spc{U}$ are contractible.\qeds

Source: \cite[(d) on page 82]{gromov-2007}.

Observe that for any point $p\in \spc{U}_d$ the curve $t\mapsto \lambda_t(p)$ is a geodesic path from $p$ to $x_0$.








Note that $\spc{M}$ --- the space of compact metric spaces can be treated as a space of compact subsets in $\spc{U}$ up to congruence.
Namely two subsets $A$ and $A'$ are called \emph{congruent} (briefly $A\cong A'$) if there is isometry of the ambient space $\spc{U}$ that maps $A$ to $A'$.
Let us define distance between congruence classes of two compact subsets $A$ and $B$ as 
\[\inf\set{\dist{A'}{B}{\spc{H}(\spc{U})}}{A'\cong A}.\]


By \ref{prop:sep-in-urys}, any compact metric spaces $\spc{K}$ admits a distance-preserving map $f\:\spc{K}\to\spc{U}$.
Moreover by \ref{thm:compact-homogeneous} any two such maps $f_1$ and $f_2$ differ by isometry of $\spc{U}$;
that is, there is an isometry $\iota\:\spc{U}\to\spc{U}$ such that $f_2=\iota\circ f_1$.
In particular $f_1(\spc{K})\cong f_2(\spc{K})$.










\section{Ultratangent space} 

Recall that we assume that $\omega$ is a once for all fixed choice of a nonprinciple ultrafilter.

For a metric space $\spc{X}$ and a positive real number $\lambda$,
we will denote by $\lambda\cdot\spc{X}$ its \emph{$\lambda$-blowup}\index{blowup},
which is a metric space with the same underlying set as $\spc{X}$ and the metric multiplied by $\lambda$.
The tautological bijection $\spc{X}\to \lambda\cdot\spc{X}$ will be denoted as $x\mapsto x^\lambda$, 
so 
\[\dist{x^\lambda}{y^\lambda}{}
=
\lambda\cdot\dist[{{}}]{x}{y}{}\] 
for any $x,y\in \spc{X}$.

The $\omega$-blowup $\omega\cdot\spc{X}$ of $\spc{X}$ is defined as the $\omega$-limit
of the $n$-blowups $n\cdot\spc{X}$; that is,
\[\omega\cdot\spc{X}
\df
\lim_{n\to\omega} n\cdot\spc{X}.\]

Given a point $x\in \spc{X}$ we can consider the sequence $x^n\in n\cdot\spc{X}$;
it corresponds to a point $x^\omega\in \omega\cdot\spc{X}$.
Note that if $x\ne y$, then 
\[\dist{x^\omega}{y^\omega}{\omega\cdot\spc{X}}=\infty;\]
that is, 
$x^\omega$ and $y^\omega$ 
belong to different metric components of $\omega\cdot\spc{X}$.

The metric component of $x^\omega$ in $\omega\cdot\spc{X}$ is called ultratangent space of $\spc{X}$ at $x$ and it is denoted as $\T^\omega_x\spc{X}$.

Equivalently, ultratangent space $\T^\omega_x\spc{X}$ can be defined the following way.
Consider all the sequences of points $x_n\in \spc{X}$ such that
the sequence $\ell_n=n\cdot\dist{x}{x_n}{\spc{X}}$ is bounded.
Define the pseudodistance between two such sequences as 
\[\dist{(x_n)}{(y_n)}{}
=
\lim_{n\to\omega}n\cdot\dist{x_n}{y_n}{\spc{X}}.\]
Then $\T^\omega_x\spc{X}$ is the corresponding metric space.

Tangent space as well as ultratangent space, 
generalize the notion of tangent space of Riemannian manifold.
In the simplest cases these two notions define the same space.
In general, they are different and both useful ---
often lack of a property in one is compensated by the other.

It is clear from the definition that tangent space has cone structure.
On the other hand, in general, ultratangent space does not have a cone structure; 
the Hilbert's cube $\prod_{n=1}^\infty[0,2^{-n}]$ is an example --- it is $\Alex{0}$ as well as $\CAT{0}$.

The next theorem shows that the tangent space $\T_p$ can be (and often will be) considered as a subset of  $\T^\omega_p$.

\begin{thm}{Theorem}\label{thm:tangent-ultratangent}
\label{thm:T-in-T^w} 
Let $\spc{X}$ be a metric space with defined angles.
Then for any $p\in \spc{L}$, there is an distance-preserving map 
\[\iota:\T_p\hookrightarrow \T^\omega_p\] 
such that for any geodesic $\gamma$ starting at $p$
we have 
\[\gamma^+(0)\mapsto \lim_{n\to\omega}[\gamma(\tfrac1n)]^n.\]

\end{thm}

\parit{Proof.}
Given $v\in \T'_p$ 
choose a geodesic $\gamma$ that starts at $p$ such that $\gamma^+(0)\z=v$.
Set $v^n=[\gamma(\tfrac1n)]^n\in n\cdot \spc{X}$ and 
\[v^\omega=\lim_{n\to\omega}v^n.\]

Note that the value $v^\omega\in\T^\omega_p$ does not depend on choice of $\gamma$;
that is, if $\gamma_1$ is another geodesic starting at $p$ such that $\gamma_1^+(0)=v$,
then 
\[\lim_{n\to\omega}v^n=\lim_{n\to\omega}v_1^n,\]
where $v_1^n=[\gamma_1(\tfrac1n)]^n\in n\cdot \spc{X}$.
The latter follows since
\[\dist{\gamma(t)}{\gamma_1(t)}{\spc{X}}=o(t)\]
and therefore $\dist{v^n}{v_1^n}{n\cdot \spc{X}}\to 0$ s $n\to\infty$.



Set $\iota(v)=v^\omega$.
Since angles between geodesics in $\spc{X}$ are defined, for any $v,w\in \T_p'$ we have
$n\cdot\dist[{{}}]{v_n}{w_n}{}\to\dist{v}{w}{}$.
Thus $\dist{v_\omega}{w_\omega}{}=\dist{v}{w}{}$; that is, $\iota$ is a global isometry of $\T_p'$.

Since $\T_p'$ is dense in $\T_p$,
we can extend $\iota$ to a global isometry $\T_p\to \T^\omega_p$.
\qeds

{\sloppy

\section[Gromov--Hausdorff and ultralimits]{Gromov--Hausdorff convergence and ultralimits}

}

\begin{thm}{Theorem}\label{thm:ultra-GH}
Assume $\spc{X}_n$ is a sequence of complete spaces. 
Let $\spc{X}_n\to \spc{X}_\omega$ as $n\to\omega$,
and $\spc{Y}_n\subset \spc{X}_n$ 
be a sequence of subsets such that $\spc{Y}_n\GHto\spc{Y}_\infty$. 
Then there is a distance-preserving map 
$\iota:\spc{Y}_\infty\to \spc{X}_\omega$.

Moreover:

\begin{subthm}{thm:ultra-GH:a}
If $\spc{X}_n\GHto \spc{X}_\infty$ 
and $\spc{X}_\infty$ is compact, then 
$\spc{X}_\infty$ is isometric to $\spc{X}_\omega$.
\end{subthm}

\begin{subthm}{thm:ultra-GH:b}
If $\spc{X}_n\GHto \spc{X}_\infty$ 
and $\spc{X}_\infty$ is proper, then 
$\spc{X}_\infty$ is isometric to a metric component of $\spc{X}_\omega$.
\end{subthm}

\end{thm}

\parit{Proof.} 
For each point $y_\infty\in \spc{Y}_\infty$ 
choose a lifting $y_n\in \spc{Y}_n$.
Pass to the $\omega$-limit $y_\omega\in \spc{X}_\omega$ of $(y_n)$.
Clearly for any $y_\infty,z_\infty\in \spc{Y}_\infty$, 
we have 
\[\dist{y_\infty}{z_\infty}{\spc{Y}_\infty}=\dist{y_\omega}{z_\omega}{\spc{X}_\omega};\] 
that is, the map $y_\infty\mapsto y_\omega$ gives a distance-preserving map $\iota:\spc{Y}_\infty\to \spc{X}_\omega$. 


\parit{\ref{SHORT.thm:ultra-GH:a}$+$\ref{SHORT.thm:ultra-GH:b}.}
Fix $x_\omega\in \spc{X}_\omega$.
Choose a sequence $x_n\in \spc{X}_n$ 
such that $x_n\to x_\omega$ as $n\to\omega$. 

Denote by $\bm{X}=\spc{X}_\infty\sqcup\spc{X}_1\sqcup\spc{X}_2\sqcup\dots$ the common space for the convergence $\spc{X}_n\GHto \spc{X}_\infty$;
as in the definition of Gromov--Hausdorff convergence.
Consider the sequence $(x_n)$ 
as a sequence of points in~$\bm{X}$.

If the $\omega$-limit $x_\infty$ of $(x_n)$ exists, 
it must lie in $\spc{X}_\infty$. 

The point $x_\infty$, if defined, does not depend on the choice of $(x_n)$.
Indeed, if $y_n\in\spc{X}_n$ is another sequence such that $y_n\to x_\omega$ as $n\to\omega$, then 
\[
\dist{y_\infty}{x_\infty}{}=\lim_{n\to\omega}\dist{y_n}{x_n}{}=0;
\]
that is, $x_\infty=y_\infty$.


This way we obtain a map $\nu\:x_\omega\to x_\infty$;
it is defined on a subset of $\Dom\nu \subset\spc{X}_\omega$.
By construction of $\iota$, 
we get  $\iota\circ\nu(x_\omega)=x_\omega$ for any $x_\omega\in \Dom\nu$.

Finally note that if $\spc{X}_\infty$ is compact, then $\nu$ is defined on all of $\spc{X}_\omega$;
this proves \ref{SHORT.thm:ultra-GH:a}.

If $\spc{X}_\infty$ is proper, choose any point $z_\infty\in \spc{X}_\infty$
and set $z_\omega=\iota(z_\infty)$.
For any point $x_\omega\in \spc{X}_\omega$ at finite distance from $z_\omega$,
for the sequence $x_n$ 
as above we have that $\dist{z_n}{x_n}{}$ is bounded for $\omega$-almost all $n$.
Since $\spc{X}_\infty$ is proper, $\nu(x_\omega)$ is defined;
in other words $\nu$ is defined on the metric component of $z_\omega$.
Hence \ref{SHORT.thm:ultra-GH:b} follows.
\qeds

\begin{thm}{Corollary} 
\label{cor:ulara-geod}
The $\omega$-limit of a sequence of complete length spaces is geodesic.
\end{thm}

\parit{Proof.} Given two points $x_\omega,y_\omega\in \spc{X}_\omega$, find two bounded sequences of points $x_n,y_n\in \spc{X}_n$, $x_n\to x_\omega$, $y_n\to y_\omega$ as $n\to\omega$.
Consider a sequence of paths  $\gamma_n\:[0,1]\to \spc{X}_n$ from $_n$ to $y_n$
 such that 
\[\length\gamma_n\le \dist{x_n}{y_n}{}+\tfrac{1}{n}.\]
Apply Theorem~\ref{thm:ultra-GH} 
for the images $\spc{Y}_n=\gamma_n([0,1])\subset \spc{X}_n$.
\qeds

\section{Ultralimits of sets}

Let $\spc{X}_n$ be a sequence of metric spaces and $\spc{X}_n\to \spc{X}_\omega$
as $n\to \omega$.

For a sequence of sets $\Omega_n\subset \spc{X}_n$,
consider the maximal set $\Omega_\omega\subset \spc{X}_\omega$ such that 
for any $x_\omega\in\Omega_\omega$ and any sequence $x_n\in\spc{X}_n$ such that $x_n\to x_\omega$ as $n\to \omega$, we have $x_n\in\Omega_n$ for $\omega$-almost all $n$.

The set $\Omega_\omega$ is called the  \emph{open $\omega$-limit} of $\Omega_n$;
we could also write  $\Omega_n\to \Omega_\omega$ as $n\to\omega$ or $\Omega_\omega=\lim_{n\to\omega}\Omega_n$. 

{\sloppy

Applying Observation~\ref{obs:ultralimit-is-complete} to the sequence of complements $\spc{X}_n\setminus \Omega_n$, we see that $\Omega_\omega$ is open for any sequence $\Omega_n$.
The definition can be applied for arbitrary sequences of sets, but  
open $\omega$-convergence  will be applied here only for sequences of open sets.

}

\section{Ultralimits of functions}

Recall that a family of submaps between metric spaces $\{f_\alpha\: \spc{X}\to\spc{Y}\}_{\alpha\in\mathcal A}$ is called \emph{equicontinuous} if for any $\eps>0$ there is $\delta>0$ such that for any $p,q\in\spc{X}$ with $\dist{p}{q}{}<\delta$ and any $\alpha\in\mathcal A$ it holds that $\dist{f(p)}{f(q)}{}<\eps$.

Let $f_n\:\spc{X}_n\to\RR$ be a sequence of subfunctions.

Set $\Omega_n=\Dom f_n$.
Consider the open $\omega$-limit set $\Omega_\omega\subset \spc{X}_\omega$ of $\Omega_n$.

Assume there is a subfunction $f_\omega\:\spc{X}_\omega\to\RR$
that satisfies the following conditions: 
(1) $\Dom f_\omega=\Omega_\omega$, (2) if $x_n\to x_\omega\in \Omega_\omega$ for a sequence of points $x_n\in\spc{X}_n$, then $f_n(x_n)\to f_\omega(x_\omega)$ as $n\to\omega$.
In this case 
the subfunction $f_\omega\:\spc{X}_\omega\to\RR$ 
is said to be the 
$\omega$-limit of $f_n\:\spc{X}_n\to\RR$.

The following lemma gives a mild condition on a sequence of functions $f_n$
guaranteeing the existence of its $\omega$-limit.

\begin{thm}{Lemma}
Let $\spc{X}_n$ be a sequence of metric spaces
and $f_n\:\spc{X}_n\to\RR$ be a sequence of subfunctions.

Assume for any positive integer $k$, there is an open set $\Omega_n(k)\subset \Dom f_n$
such that the restrictions $f_n|_{\Omega_n(k)}$ are uniformly bounded and continuous
and the open $\omega$-limit of $\Omega_n(n)$ coincides with the open $\omega$-limit of $\Dom f_n$.
Then the $\omega$-limit of $f_n$ is defined.

In particular, if the $f_n$ are uniformly bounded and continuous, then the $\omega$-limit is defined.
\end{thm}

The proof is straightforward.

{\sloppy

\begin{thm}{Exercise}\label{ex:nonconvex-limit}
Construct a sequence of compact length spaces 
$\spc{X}_n$  
with a converging sequence of $\Lip$-Lipschitz concave functions $f_n\:\spc{X}_n\to\RR$ such that
the $\omega$-limit $\spc{X}_\omega$ of $\spc{X}_n$ is compact
and the $\omega$-limit $f_\omega\:\spc{X}_\omega\to\RR$ of $f_n$ is not concave.
\end{thm}

}

If $f\:\spc{X}\to\RR$ is a subfunction, 
the $\omega$-limit of the constant sequence $f_n=f$ is called the $\omega$-power of $f$ and denoted by $f^\omega$.
So
\[f^\omega\:\spc{X}\to\RR,\ \ f^\omega(x_\omega)=\lim_{n\to\omega} f(x_n).\]

Recall that we treat $\spc{X}$ as a subset of its $\omega$-power $\spc{X}^\omega$.
Note that $\Dom f=\spc{X}\cap \Dom f^\omega$.
Moreover, 
if $\oBall(x,\eps)_{\spc{X}}\subset \Dom f$
then $\oBall(x,\eps)_{\spc{X}^\omega}\subset \Dom f^\omega$.


\parbf{Ultradifferential.}
Given a function $f\:\spc{L}\to\RR$, consider sequence of functions $f_n\:n\cdot\spc{L}\to\RR$, defined by 
\[f_n(x^n)=n\cdot(f(x)-f(p)),\]
here $x^n\in n\cdot\spc{L}$ is the point corresopnding to $x\in\spc{L}$.
While $n\cdot(\spc{L},p)\to(\T^\omega,\0)$ as $n\to\omega$, 
functions $f_n$ converge to $\omega$-differential of $f$ at $p$.
It will be denoted by $\dd_p^\omega f$;
\[\dd_p^\omega f\:\T_p^\omega\to\RR,\ \ \dd_p^\omega f=\lim_{n\to\omega} f_n.\] 

Clearly, the $\omega$-differential $\dd_p^\omega f$ of a locally Lipschitz subfunction $f$ is defined at each point $p\in \Dom f$.
















\section{Comments} 

Given two metric spaces $\spc{X}$ and $\spc{Y}$, we will write $\spc{X}\preccurlyeq \spc{Y}$ if there is a noncontracting map $f\:\spc{X}\to \spc{Y}$;
that is, if 
$$ |x-x'|_{\spc{X}}\le|f(x)-f(x')|_{\spc{Y}}$$
for any $x,x'\in \spc{X}$.

Further, given $\eps>0$, we will write $\spc{X}\preccurlyeq \spc{Y}+\eps$
if there is a map $f\:\spc{X}\to \spc{Y}$ such that 
$$|x-x'|_{\spc{X}}\le|f(x)-f(x')|_{\spc{Y}}+\eps$$
for any $x,x'\in \spc{X}$.

Define 
$$\dist[\star]{\spc{X}}{\spc{Y}}{\spc{M}}=\inf\set{\eps}{\spc{X}\preccurlyeq \spc{Y}+\eps
\quad\text{and}\quad
\spc{Y}\preccurlyeq \spc{X}+\eps}$$
It turns out that $\dist[\star]{*}{*}{\spc{M}}$ is a different metric on the set of isometry classes of compact metric spaces; that is, in general $\dist[\star]{\spc{X}}{\spc{Y}}{\spc{M}}\not=|\spc{X}-\spc{Y}|_{\spc{M}}$. 
However, these two metrics define the same topology on $\spc{M}$.
More precicely:

\begin{thm}{Proposition}\label{GH-po}
For any sequence of compact metric spaces $(\spc{X}_n)$ and a compact metric space $\spc{X}_\infty$,
we have
$$|\spc{X}_n-\spc{X}_\infty|_{\spc{M}}\to 0
\quad\iff\quad
\dist[\star]{\spc{X}_n}{\spc{X}_\infty}{\spc{M}}\to 0$$ 
as $n\to\infty$.
\end{thm}

We will not give a proof of this proposition. 
Likely, we will not use it further in the lectures, 
but it might help you to build intuition for Gromov--Hausdorff convergence.
If you want to prove it yourself look in the proof of Theorem~\ref{thm:GH-is-a-metric} 
and try to modify it using ideas from the proof of Problem~\ref{pr:non-contracting=>isometry}.

The Gromov--Hausdorff distance can be defined for arbitrary pair of metric space.
Therefore it is natural to ask why we only consider compact metric spaces.
First note the Gromov--Hausdorff distance from any metric space $\spc{X}$ 
to its completion $\bar {\spc{X}}$ is zero.
Therefore if you want to end up in a metric space, it is better to consider only complete metric spaces.

Further, the distance between one-point-space and a metric spce with infinite diameter is infinite.
Therefore one has to either consider only bounded metric spaces (that is, the spaces with finite diameter)
or relux the definition of metric space which allow metric to take infinite value.

Finally, the class of isometry classes of all bounded complete metric spaces forms a class, but not a set.
Hence again we have two choices: either relux the definition of metric space so its points will form a class, or restrict further the class of spaces for which the isometry classes will form a set.

\begin{thm}{Exercise}
Prove that isometry classes of compact metric spaces form a set. 
\end{thm}

\begin{thm}{Exercise}\label{pr:GH1}
Let $\spc{X}=\{x,y,z\}$ be a three point subset of Euclidean plane with distances
$$|x-y|=|y-z|=|z-x|=1.$$
\begin{enumerate}[(i)]
\item Find the minimal Hausdorff distance from $\spc{X}$ to a one-point subset of the plane.
\item Find the Gromov--Hausdorff distance from $\spc{X}$ to the one-point metric space. 
\end{enumerate}
\end{thm}

\begin{thm}{Exercise}\label{pr:GH2}
Let $\spc{X}$ and $\spc{Y}$ be a compact metric spaces which have isometric $\eps$-nets.
Show that 
$$|\spc{X}-\spc{Y}|_{\spc{M}}\le 2\cdot\eps.$$
Is it always true that 
$$|\spc{X}-\spc{Y}|_{\spc{M}}\le \eps?$$
\end{thm}




\begin{thm}{Exercise}\label{pr:GH3}
Define the \emph{radius of a metric space}\index{radius of a metric space} $\spc{X}$ as 
$$\rad \spc{X}=\inf_x\left\{\sup_y\{|x-y|_{\spc{X}}\}\right\}.$$
Equivalently, 
$$\rad \spc{X}=\inf\set{R>0}{\text{there is}\ x\in \spc{X}\  \text{such that}\ B_R(x)\supset \spc{X}}.$$
 
\begin{enumerate}[(i)]
\item Show that for any compact metric space $\spc{X}$ we have
$$\tfrac12\cdot\diam \spc{X}\le \rad \spc{X}\le \diam \spc{X}.$$
\item Show that for any compact metric spaces $\spc{X},\spc{Y}$ we have
$$|\rad \spc{X}-\rad \spc{Y}|\le 2\cdot |\spc{X}-\spc{Y}|_{\spc{M}}.$$
\end{enumerate}
\end{thm}

\begin{thm}{Exercise}\label{pr:F-X}
Let $\spc{X}$ be a metric space.
If two compact sets $A, B$ in $\spc{X}$ are isometric,
we will write $A\iso B$. 
Set
$$d(A,B)=\inf \set{|A'-B'|_{\mathcal{H}(\spc{X})}}{A'\iso A \ \text{and}\ B'\iso B}.$$
Note that if $\spc{X}=\ell^\infty$, then according to Proposition~\ref{prop:GH-with-fixed-Z}, 
$d$ is a metric on $\mathcal{H}(\spc{X})/\iso$ (that is, on the ``$\iso$''-equivalecne classes of $\mathcal{H}(\spc{X})$).

Show that it does not hold for arbitrary metric space $\spc{X}$.
Understand the reason why it holds for $\spc{X}=\ell^\infty$.
\end{thm}


\begin{thm}{Exercise}\label{pr:GH-variation}
Consider the pairs $(\spc{X},A)$, where $\spc{X}$ is a compact metric space and $A$ is a closed subset in $\spc{X}$.
Two such pairs, say $(\spc{X},A)$ and $(\spc{X}',A')$ will be called equivalent (briefly $(\spc{X},A)\sim(\spc{X}',A')$)
if there is an isometry $\iota\:\spc{X}\to \spc{X}'$ such that $\iota(A)=A'$.

Modify the definition of Gromov--Hausdorff metric to construct a natural metric on the set of $\sim$-equivalence classes of the pairs $(\spc{X},A)$.
\end{thm}

Here we introduce so called Gromov--Hausdorff convergence for metric spaces.
This convergence was introduced by Gromov around 1980, published in \cite{gromov-1981}.
Very soon this notion began to be used in all branches of geometry.
In fact today I have difficulty to understand 
how one could do geometry without this type of convergence.%
(Some types of convergences of metric spaces was considered before Gromov,
but they had lack of generality;
each type of convergence was desined to solve one particular problem.)


\begin{thm}{Exercise}\label{ex:euclid-isom}
\begin{subthm}{}
Let $\spc{X},\spc{Y}$ be two compact sets in the Euclidean plane $\RR^2$.
Show that $\spc{X}$ is isometric to $\spc{Y}$ if and only if there is an motrio $\iota\:\RR^2\to \RR^2$
that sends $\spc{X}$ to $\spc{Y}$.
\end{subthm}

\begin{subthm}{}
Find two isometric subsets $\spc{X},\spc{Y}$ of $\ell^\infty$
such that there is no isometry $\iota\:\ell^\infty\to \ell^\infty$ 
that sends $\spc{X}$ to $\spc{Y}$.
\end{subthm}
\end{thm}
