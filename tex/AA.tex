\section{Gromov--Hausdorff metric is a metric}

\begin{thm}{Theorem}\label{thm:GH-is-a-metric}
The set of isometry classes of compact metric spaces equipped with Gromov--Hausdorff metric forms a metric space which will be denoted further as $\spc{M}$.
\end{thm}



Let $\spc{X}$, $\spc{Y}$ and $\spc{Z}$ be arbitrary  compact metric spaces.
We need to check the following:
\begin{enumerate}[{\it (i)}]
\item\label{GH-1} $|\spc{X}-\spc{Y}|_{\spc{M}}\ge 0$;
\item\label{GH-2} $|\spc{X}-\spc{Y}|_{\spc{M}}=0$ if and only if $\spc{X}$ is isometric to $\spc{Y}$;
\item\label{GH-3} $|\spc{X}-\spc{Y}|_{\spc{M}}=|\spc{Y}-\spc{X}|_{\spc{M}}$;
\item\label{GH-4} $|\spc{X}-\spc{Y}|_{\spc{M}}+|\spc{Y}-\spc{Z}|_{\spc{M}}\ge |\spc{X}-\spc{Z}|_{\spc{M}}$.
\end{enumerate}


Note that {\it (\ref{GH-1})}, {\it(\ref{GH-3})} and ``if''-part of {\it(\ref{GH-2})} follow directly from Definition \ref{def:GH}.

\parit{Proof of (\ref{GH-4}).}
Choose arbitrary $a,b \in \mathbb{R}$ such that
$$a>|\spc{X}-\spc{Y}|_{\spc{M}}\ \ \text{and}\ \  b>|\spc{Y}-\spc{Z}|_{\spc{M}}.$$
Choose two metrics on $\spc{U}=\spc{X}\sqcup \spc{Y}$ and $\spc{V}=\spc{Y}\sqcup \spc{Z}$ so that
$|\spc{X}-\spc{Y}|_{\mathcal{H}(\spc{U})}<a$ and $|\spc{Y}-\spc{Z}|_{\mathcal{H}(\spc{V})}<b$ 
and the inclusions $\spc{X}\hookrightarrow \spc{U}$, $\spc{Y}\hookrightarrow \spc{U}$, $\spc{Y}\hookrightarrow \spc{V}$ and $\spc{Z}\hookrightarrow \spc{V}$ are distance preserving.

Consider the metric on $\spc{W}=\spc{X}\sqcup \spc{Z}$ 
so that inclusions $\spc{X}\hookrightarrow \spc{W}$ and $\spc{Z}\hookrightarrow \spc{W}$ are distance preserving
and 
$$|x-z|_{\spc{W}}=\inf_{y\in \spc{Y}}\{|x-y|_{\spc{U}}+|y-z|_{\spc{V}}\}.$$
Note that $|{*}-{*}|_{\spc{W}}$ is indeed a metric and 
$$|\spc{X}-\spc{Z}|_{\mathcal{H}(\spc{W})}<a+b.$$
Property {\it (\ref{GH-4})} follows since the last inequality holds for any $a>|\spc{X}\z-\spc{Y}|_{\spc{M}}$ and $b>|\spc{Y}-\spc{Z}|_{\spc{M}}$.
\qeds

\parit{Proof of ``only if''-part of (\ref{GH-2}).}
According to Exercise~\ref{ex:GH=>eps-isom},
for any sequence $\eps_n\to0+$ there is a sequence of $\eps_n$-isometries 
$f_n\:\spc{X}\to \spc{Y}$.

Since $\spc{X}$ is compact, 
we can choose a countable dense set
$S$ in $\spc{X}$.
Use a diagonal procedure if necessary, to pass to a subsequence of $(f_n)$
such that for every $x \in S$ the sequence $(f_n(x))$ 
converges in $\spc{Y}$. 
Consider the pointwise limit map  $f_\infty \: S \to \spc{Y}$ defined by
 $$f_\infty(x) = \lim_{n\to\infty} f_n (x)$$ for every $x \in S$. 
Since $$|f_n (x)- f_n (x')|_{\spc{Y}}\lege |x- x'|_\spc{X} \pm\eps_n,$$ 
we have 
$$|f_\infty(x)-f_\infty (x')|_{\spc{Y}} 
= \lim_{n\to\infty} |f_n(x)-f_n (x')|_{\spc{Y}} 
= |x -x'|_\spc{X}$$ for all
$x, x' \in S$; 
that is, $f_\infty\:S\to \spc{Y}$ is a distance-preserving map. 
Therefore $f_\infty$ can be extended to a distance-preserving map from all of $\spc{X}$ to $\spc{Y}$.
The later is done by setting 
$$f_\infty(x)=\lim_{n\to\infty} f_\infty(x_n)$$ 
for some (and therefore any) sequence of points $(x_n)$ in $S$
which converges to $x$ in $\spc{X}$.
(Note that if $x_n\to x$ then $(x_n)$ is Cauchy.
Since $f_\infty$ is distance preserving, $y_n=f_\infty(x_n)$ is also a Cauchy sequence in $\spc{Y}$;
therefore it converges.)

This way we obtain a distance preserving map $f_\infty\:\spc{X}\to \spc{Y}$. 
It remains to show that $f_\infty$ is surjective; that is, $f_\infty(\spc{X})=\spc{Y}$.

Note that in the same way we can obtain a distance preserving map $g_\infty\:\spc{Y}\z\to \spc{X}$.
If $f_\infty$ is not surjective, then neither is $f_\infty\circ g_\infty\:\spc{Y}\to \spc{Y}$.
So $f_\infty \circ g_\infty$ is a distance preserving map from a compact space to itself which is not an isometry.
The later contradicts Exercise~\ref{ex:non-contracting-map}. 
\qeds

\begin{thm}{Exercise}\label{ex:d_GH-and-diam}
 Let $\spc{X}$ and $\spc{Y}$ be two compact metric spaces.
Prove that 
$$|\diam \spc{X} - \diam \spc{Y} |\le 2\cdot |\spc{X}-\spc{Y}|_{\spc{M}}.$$
In other words, $\diam\:\spc{M}\to\RR$ is a $2$-Lipschitz function.
\end{thm}

\begin{thm}{Exercise}\label{ex:pack-GH}
Suppose $\spc{X}_n$ is a sequence of compact metric spaces which converges to a compact metric space $\spc{X}_\infty$
in the sense of Gromov--Hausdorff.
Show that for any $\eps>0$
$$\pack_\eps \spc{X}_n\ge\pack_\eps \spc{X}_\infty$$ 
for all large $n$.
In particular, $\pack_\eps$ is a lower semicontinuous function on $\spc{M}$.
\end{thm}


\section{Gromov--Hausdorff convegence}

The Gromov--Hausdorff metric defines Gromov--Hausdorff convegence
and this is the only thing it is good for.
In other words in all applications, we use only topology on $\spc{M}$
and we do not care about particular value of Gromov--Hausdorff distance between spaces.

In order to determine that a given sequence of metric spaces $(\spc{X}_n)$ converges in the Gromov--Hausdorff sense to $\spc{X}_\infty$, it is sufficient to estimate distances $|\spc{X}_n-\spc{X}_\infty|_{\spc{M}}$ and  check if $|\spc{X}_n-\spc{X}_\infty|_{\spc{M}}\to 0$.
This problem turns to be simpler than finding Gromov--Hausdorff distance between a particular pair of spaces.
The proposition below gives one way to do this.

\begin{thm}{Proposition}\label{prop:GH-e-isom}
A sequence of compact metric spaces $(\spc{X}_n)$ converges to  $\spc{X}_\infty$ in the sense of Gromov--Hausdorff if and only if there is a sequence $\eps_n\to0+$
and an $\eps_n$-isometry $f_n\:\spc{X}_n\to \spc{X}_\infty$ for each $n$.
\end{thm}

\parit{Proof.} Follows from Propsition~\ref{prop:alm-isom=>GH} and Exercise~\ref{ex:GH=>eps-isom}
\qeds


\section{Gromov's compactness theorem}

The following theorem is analogous to Blaschke's compactness theorems (\ref{thm:compact+Hausdorff}).

\begin{thm}{Gromov's compactness theorem}\label{thm:gromov-compactness}%
Let $\spc{Q}$ be a closed subset of $\spc{M}$.
Then $\spc{Q}$ is compact if and only if there is a sequence of positive numbers $\eps_1,\eps_2,\dots$ such that $\eps_n\to 0$ and 
$$\pack_{\eps_n}\spc{X}\le n\eqlbl{eq:pack<n}$$
for any space $\spc{X}$ in $\spc{Q}$.
\end{thm}

\begin{thm}{Exercise}\label{pack<n;n>1}
Show that the conclusion of the theorem does not hold
if the inequality \ref{eq:pack<n} holds only for $n\ge 2$.
\end{thm}




\begin{thm}{Lemma}
$\spc{M}$ is complete.
\end{thm}

\parit{Proof.}
Let $(\spc{X}_n)$ be a Cauchy sequence in $\spc{M}$.
Passing to a subsequence if necessary, 
we can assume that $|\spc{X}_n-\spc{X}_{n+1}|_{\spc{M}}<\tfrac1{2^n}$ for each $n$.
In particular, for each $n$ one can equip $\spc{W}_n=\spc{X}_n \sqcup \spc{X}_{n+1}$ with a metric such that
inclusions $\spc{X}_n\hookrightarrow \spc{W}_n$ and $\spc{X}_{n+1}\hookrightarrow \spc{W}_n$ are distance preserving
and $$|\spc{X}_n-\spc{X}_{n+1}|_{\mathcal{H}(\spc{W}_n)}\z<\tfrac1{2^n}$$
for each $n$.

Set $\spc{W}$ to be the disjoint union of all $\spc{X}_n$.
Let us equip $\spc{W}$ with a metric defined the following way:
\begin{itemize}
\item for any fixed $n$ and any two points $x_n,x_n'\in \spc{X}_n$ set
$$|x_n-x_n'|_{\spc{W}}=|x_n-x_n'|_{\spc{X}_n}$$
\item for any positive integers $m>n$ and any two points $x_n\in \spc{X}_n$ and $x_m\in \spc{X}_m$ set
$$|x_n-x_m|_{\spc{W}}=\inf\left\{\sum_{i=n}^{m-1}|x_i-x_{i+1}|_{\spc{W}_i}\right\},$$
where the infimum is taken for all sequences $x_i\in \spc{X}_i$.
\end{itemize}

\begin{thm}{Exercise}
Check that this indeed defines a metric on $\spc{W}$.
\end{thm}

Let $\bar{\spc{W}}$ be the completion of $\spc{W}$.
Note that $|\spc{X}_m-\spc{X}_n|<\tfrac1{2^{n-1}}$ if $m>n$.
Therefore the union of $\spc{X}_1\cup \spc{X}_2\cup\dots\cup \spc{X}_n$ forms a $\tfrac1{2^{n-1}}$-net in $\bar{\spc{W}}$.
Since each $\spc{X}_i$ is compact, we get that $\bar{\spc{W}}$ admits a compact $\eps$-net for any $\eps>0$.
According to Problem~\ref{pr:compact-net}, $\bar{\spc{W}}$ is compact.

According to Blaschke's compactness theorem (\ref{thm:compact+Hausdorff}),
we can pass to a subsequence of $(\spc{X}_n)$ which converge in $\mathcal{H}(\bar{\spc{W}})$ and therefore in $\spc{M}$.
\qeds

\parit{Proof of \ref{thm:gromov-compactness}; ``only if'' part.}
If there is no sequence $\eps_n\to0$ as described in the problem, then for a fixed fixed $\delta>0$
there is a sequence of spaces $\spc{X}_n\in\spc{Q}$ such that $$\pack_\delta \spc{X}_n\to\infty\ \ \text{as}\ \  n\to\infty.$$
Since $\spc{Q}$ is compact, 
this sequence has a partial limit say $\spc{X}_\infty\in\spc{Q}$.
It is easy to see that $\pack_{\delta/10} \spc{X}_\infty=\infty$;
the later contradicts Theorem~\ref{thm:finite_pack=compact}.

\parit{``If'' part.}
Let us fix the sequence $\eps_n\to 0$ as in the problem and consider the set $\hat{\spc{Q}}$ of all (isometry classes of all) metric spaces $\spc{X}$ such that
$\pack_{\eps_n} \spc{X}\le n$ for any $n$. 
According to Exercise~\ref{ex:pack-GH}, $\hat{\spc{Q}}$ is closed in $\spc{M}$.
Clearly $\spc{Q}\subset\hat{\spc{Q}}$.
Therefore it is sufficient to prove that $\hat{\spc{Q}}$ is compact.

Note that $\diam \spc{X}\le \eps_1$ for any $\spc{X}\in \hat{\spc{Q}}$.
Given positive integer $n$ consider set of all metric spaces $\spc{W}_n$
with number of points at most $n$ and diameter $\le \eps_1$.
Note that $\spc{W}_n$ is compact for each $n$.
Further a maximal $\eps_n$-packing of any $\spc{X}\in\hat{\spc{Q}}$ forms a subspace from $\spc{W}_n$.
Therefore $\spc{W}_n\cap\hat{\spc{Q}}$ is a comapct $\eps_n$-net in  $\hat{\spc{Q}}$.
Problem~\ref{pr:compact-net} implies that $\hat{\spc{Q}}$ is compact.
\qeds



\section{Comments} 

Given two metric spaces $\spc{X}$ and $\spc{Y}$, we will write $\spc{X}\preccurlyeq \spc{Y}$ if there is a noncontracting map $f\:\spc{X}\to \spc{Y}$;
that is, if 
$$ |x-x'|_{\spc{X}}\le|f(x)-f(x')|_{\spc{Y}}$$
for any $x,x'\in \spc{X}$.

Further, given $\eps>0$, we will write $\spc{X}\preccurlyeq \spc{Y}+\eps$
if there is a map $f\:\spc{X}\to \spc{Y}$ such that 
$$|x-x'|_{\spc{X}}\le|f(x)-f(x')|_{\spc{Y}}+\eps$$
for any $x,x'\in \spc{X}$.

Define 
$$\dist[\star]{\spc{X}}{\spc{Y}}{\spc{M}}=\inf\set{\eps}{\spc{X}\preccurlyeq \spc{Y}+\eps\ \ \text{and}\ \ \spc{Y}\preccurlyeq \spc{X}+\eps}$$
It turns out that $\dist[\star]{*}{*}{\spc{M}}$ is a different metric on the set of isometry classes of compact metric spaces; that is, in general $\dist[\star]{\spc{X}}{\spc{Y}}{\spc{M}}\not=|\spc{X}-\spc{Y}|_{\spc{M}}$. 
However, these two metrics define the same topology on $\spc{M}$.
More precicely:

\begin{thm}{Proposition}\label{GH-po}
For any sequence of compact metric spaces $(\spc{X}_n)$ and a compact metric space $\spc{X}_\infty$,
we have
$$|\spc{X}_n-\spc{X}_\infty|_{\spc{M}}\to 0 \ \ \ \Leftrightarrow\ \ \ \dist[\star]{\spc{X}_n}{\spc{X}_\infty}{\spc{M}}\to 0$$ 
as $n\to\infty$.
\end{thm}

We will not give a proof of this proposition. 
Likely, we will not use it further in the lectures, 
but it might help you to build intuition for Gromov--Hausdorff convergence.
If you want to prove it yourself look in the proof of Theorem~\ref{thm:GH-is-a-metric} 
and try to modify it using ideas from the proof of Problem~\ref{pr:non-contracting=>isometry}.

The Gromov--Hausdorff distance can be defined for arbitrary pair of metric space.
Therefore it is natural to ask why we only consider compact metric spaces.
First note the Gromov--Hausdorff distance from any metric space $\spc{X}$ 
to its completion $\bar {\spc{X}}$ is zero.
Therefore if you want to end up in a metric space, it is better to consider only complete metric spaces.

Further, the distance between one-point-space and a metric spce with infinite diameter is infinite.
Therefore one has to either consider only bounded metric spaces (that is, the spaces with finite diameter)
or relux the definition of metric space which allow metric to take infinite value.

Finally, the class of isometry classes of all bounded complete metric spaces forms a class, but not a set.
Hence again we have two choices: either relux the definition of metric space so its points will form a class, or restrict further the class of spaces for which the isometry classes will form a set.

\begin{thm}{Exercise}
Prove that isometry classes of compact metric spaces form a set. 
\end{thm}

\begin{thm}{Exercise}\label{pr:GH1}
Let $\spc{X}=\{x,y,z\}$ be a three point subset of Euclidean plane with distances
$$|x-y|=|y-z|=|z-x|=1.$$
\begin{enumerate}[(i)]
\item Find the minimal Hausdorff distance from $\spc{X}$ to a one-point subset of the plane.
\item Find the Gromov--Hausdorff distance from $\spc{X}$ to the one-point metric space. 
\end{enumerate}
\end{thm}

\begin{thm}{Exercise}\label{pr:GH2}
Let $\spc{X}$ and $\spc{Y}$ be a compact metric spaces which have isometric $\eps$-nets.
Show that 
$$|\spc{X}-\spc{Y}|_{\spc{M}}\le 2\cdot\eps.$$
Is it always true that 
$$|\spc{X}-\spc{Y}|_{\spc{M}}\le \eps?$$
\end{thm}




\begin{thm}{Exercise}\label{pr:GH3}
Define the \emph{radius of a metric space}\index{radius of a metric space} $\spc{X}$ as 
$$\rad \spc{X}=\inf_x\left\{\sup_y\{|x-y|_{\spc{X}}\}\right\}.$$
Equivalently, 
$$\rad \spc{X}=\inf\set{R>0}{\text{there is}\ x\in \spc{X}\  \text{such that}\ B_R(x)\supset \spc{X}}.$$
 
\begin{enumerate}[(i)]
\item Show that for any compact metric space $\spc{X}$ we have
$$\tfrac12\cdot\diam \spc{X}\le \rad \spc{X}\le \diam \spc{X}.$$
\item Show that for any compact metric spaces $\spc{X},\spc{Y}$ we have
$$|\rad \spc{X}-\rad \spc{Y}|\le 2\cdot |\spc{X}-\spc{Y}|_{\spc{M}}.$$
\end{enumerate}
\end{thm}

\begin{thm}{Exercise}\label{pr:F-X}
Let $\spc{X}$ be a metric space.
If two compact sets $A, B$ in $\spc{X}$ are isometric,
we will write $A\iso B$. 
Set
$$d(A,B)=\inf \set{|A'-B'|_{\mathcal{H}(\spc{X})}}{A'\iso A \ \text{and}\ B'\iso B}.$$
Note that if $\spc{X}=\ell^\infty$ then according to Proposition~\ref{prop:GH-with-fixed-Z}, 
$d$ is a metric on $\mathcal{H}(\spc{X})/\iso$ (that is, on the ``$\iso$''-equivalecne classes of $\mathcal{H}(\spc{X})$).

Show that it does not hold for arbitrary metric space $\spc{X}$.
Understand the reason why it holds for $\spc{X}=\ell^\infty$.
\end{thm}

\begin{thm}{Exercise}\label{pr:under}
Let $\spc{X}$ be a comapact metric space.
Denote by $\Under(\spc{X})$ the set of all isometry classes of metric spaces $\spc{Y}$ 
which admit a distance non-contracting map $\spc{Y}\to \spc{X}$.

\begin{subthm}{pr:under:if}
Show that $\Under(\spc{X})$ forms a compact set in $\spc{M}$.
\end{subthm}

\begin{subthm}{pr:under:only-if}
Show that for any compact set $K$ in $\spc{M}$ there is a compact space $\spc{X}$
such that  $\Under(\spc{X})\supset K$.
\end{subthm}

\end{thm}

\begin{thm}{Exercise}\label{pr:GH-variation}
Consider the pairs $(\spc{X},A)$, where $\spc{X}$ is a compact metric space and $A$ is a closed subset in $\spc{X}$.
Two such pairs, say $(\spc{X},A)$ and $(\spc{X}',A')$ will be called equivalent (briefly $(\spc{X},A)\sim(\spc{X}',A')$)
if there is an isometry $\iota\:\spc{X}\to \spc{X}'$ such that $\iota(A)=A'$.

Modify the definition of Gromov--Hausdorff metric to construct a natural metric on the set of $\sim$-equivalence classes of the pairs $(\spc{X},A)$.
\end{thm}

Here we introduce so called Gromov--Hausdorff convergence for metric spaces.
This convergence was introduced by Gromov around 1980, published in \cite{gromov-1981}.
Very soon this notion began to be used in all branches of geometry.
In fact today I have difficulty to understand 
how one could do geometry without this type of convergence.%
(Some types of convergences of metric spaces was considered before Gromov,
but they had lack of generality;
each type of convergence was desined to solve one particular problem.)


\begin{thm}{Exercise}\label{ex:euclid-isom}
\begin{subthm}{}
Let $\spc{X},\spc{Y}$ be two compact sets in the Euclidean plane $\RR^2$.
Show that $\spc{X}$ is isometric to $\spc{Y}$ if and only if there is an motrio $\iota\:\RR^2\to \RR^2$
that sends $\spc{X}$ to $\spc{Y}$.
\end{subthm}

\begin{subthm}{}
Find two isometric subsets $\spc{X},\spc{Y}$ of $\ell^\infty$
such that there is no isometry $\iota\:\ell^\infty\to \ell^\infty$ 
that sends $\spc{X}$ to $\spc{Y}$.
\end{subthm}
\end{thm}
