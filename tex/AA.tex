
\begin{thm}{Uniqueness of geodesics}\label{thm:cat-unique}
In a proper length $\CAT(0)$ space, pairs of points are joined by unique geodesics, and these geodesics depend continuously on their endpoint pairs.

Analogously, in a proper length $\CAT(1)$ space, pairs of points at distance less than $\pi$ are joined by unique geodesics, and these geodesics depend continuously on their endpoint pairs.
\end{thm}

\parit{Proof.} 
Given 4 points $p^1,p^2,q^1,q^2$ in a proper length $\CAT(0)$ space $\spc{U}$, 
consider two triangles $\trig{p^1}{q^1}{p^2}$ and $\trig{p^2}{q^2}{q^1}$.
Since both of these triangles are thin, we get 
\begin{align*}
\dist{\geodpath_{[p^1q^1]}(t)}{\geodpath_{[p^2q^1]}(t)}{\spc{U}}
&\le (1-t)\cdot \dist{p^1}{p^2}{\spc{U}},
\\
\dist{\geodpath_{[p^2q^1]}(t)}{\geodpath_{[p^2q^2]}(t)}{\spc{U}}
&\le t\cdot \dist{q^1}{q^2}{\spc{U}}.
\intertext{By the triangle inequality,}
\dist{\geodpath_{[p^1q^1]}(t)}{\geodpath_{[p^2q^2]}(t)}{\spc{U}}&\le \max\{\dist{p^1}{p^2}{\spc{U}},\dist{q^1}{q^2}{\spc{U}}\}.
\end{align*}

This implies continuity and uniqueness in the $\CAT(0)$ case.  
 
The $\CAT(1)$ case is done in essentially the same way.
\qeds

Adding the first two inequalities of the preceding proof gives the following:

\begin{thm}{Proposition}
Suppose $p^1,p^2,q^1,q^2$ are points in a proper length $\CAT(0)$ space~$\spc{U}$.
Then 
\[\dist{\geodpath_{[p^1q^1]}(t)}{\geodpath_{[p^2q^2]}(t)}{\spc{U}}\]
is a convex function.
\end{thm}

\begin{thm}{Corollary}\label{cor:dist-convex}
Let $K$ be a closed convex subset in a proper length $\CAT(0)$ space~$\spc{U}$.
Then $\dist{K}{}{}\:\spc{U}\to\RR$ is \index{convex function}\emph{convex};
that is, the function $t\mapsto\dist{K}{}{}\circ\gamma$ is convex for any geodesic $\gamma$ in $\spc{U}$.

In particular, $\dist{p}{}{}$ is convex for any point $p$ in~$\spc{U}$.
\end{thm}


\begin{thm}{Corollary}\label{cor:contractible-cat}
Any proper length $\CAT(0)$ space is contractible.

Analogously, any proper length $\CAT(1)$ space with diameter $<\pi$ is contractible.
\end{thm}

\parit{Proof.} Let $\spc{U}$ be a proper length $\CAT(0)$ space.
Fix a point $p\in \spc{U}$.

For each point $x$ consider the geodesic path $\gamma_x\:[0,1]\to \spc{U}$ from $p$ to~$x$.
Consider the one parameter family of maps 
$h_t\:x\mapsto \gamma_x(t)$ for $t\in [0,1]$.
By uniqueness of geodesics (\ref{thm:cat-unique}), the map 
$(t,x)\mapsto h_t(x)$ is continuous. The family $h_t$ is called a \index{geodesic homotopy}\emph{geodesic homotopy}.

It remains to note that $h_1(x)=x$ and $h_0(x)=p$ for any~$x$.

The proof of the $\CAT(1)$ case is identical.
\qeds

\begin{thm}{Proposition}\label{cor:loc-geod-are-min}
Suppose $\spc{U}$ is a proper length $\CAT(0)$ space.  
Then any local geodesic in $\spc{U}$ is a geodesic.

Analogously, if $\spc{U}$ is a proper length $\CAT(1)$ space, then any local geodesic in $\spc{U}$ which is shorter than $\pi$ is a geodesic.
\end{thm}

\begin{wrapfigure}{r}{21mm}
\begin{lpic}[t(-0mm),b(0mm),r(0mm),l(0mm)]{pics/local-geod(1)}
\lbl[t]{2.5,1;$\gamma(0)$}
\lbl[b]{10,14;$\gamma(a)$}
\lbl[t]{19,8;$\gamma(b)$}
\end{lpic}
\end{wrapfigure}

\parit{Proof.}
Suppose $\gamma\:[0,\ell]\to\spc{U}$ is a local geodesic that is not a geodesic.
Choose $a$ to be the maximal value 
such that $\gamma$ is a geodesic on $[0,a]$.
Further choose $b>a$ so that $\gamma$ is a geodesic on $[a,b]$.

Since the triangle $\trig{\gamma(0)}{\gamma(a)}{\gamma(b)}$ is thin and 
$\dist{\gamma(0)}{\gamma(b)}{}<b$ we have
\[\dist{\gamma(a-\eps)}{\gamma(a+\eps)}{}<2\cdot\eps\]
for all small~$\eps>0$.
That is, $\gamma$ is not length-minimizing on the interval $[a-\eps,a+\eps]$ for any $\eps>0$,
a contradiction.

The spherical case is done in the same way.
\qeds


\begin{thm}{Exercise}\label{ex:geod-CBA}
Assume $\spc{U}$ is a proper length $\CAT(\kappa)$ space
 with extendable geodesics;
that is, any geodesic is an arc in a local geodesic $\RR\to \spc{U}$.

Show that the space of geodesic directions at any point in $\spc{U}$ is complete.

Does the statement remain true if $\spc{U}$ is complete, but not required to be proper?
\end{thm}

Now let us formulate the main result of this section.


\begin{wrapfigure}[6]{r}{28mm}
\begin{lpic}[t(-4mm),b(6mm),r(0mm),l(0mm)]{pics/lem_alex1(1)}
\lbl[lb]{10,23;$y$}
\lbl[rt]{1.5,.5;$p$}
\lbl[bl]{25,7.5;$x$}
\lbl[lb]{17,15;$z$}
\end{lpic}
\end{wrapfigure}

\begin{thm}{Inheritance lemma}
\label{lem:inherit-angle} 
Assume that a triangle $\trig p x y$ 
in a metric space is \index{decomposed triangle}\emph{decomposed} 
into two triangles $\trig p x z$ and $\trig p y z$;
that is, $\trig p x z$ and $\trig p y z$ have a common side $[p z]$, and the sides $[x z]$ and $[z y]$ together form the side $[x y]$ of $\trig p x y$.

If both triangles $\trig p x z$ and $\trig p y z$ are thin, 
then the triangle $\trig p x y$ is also thin.

Analogously, if $\trig p x y$ has perimeter $<2\cdot\pi$ and both triangles $\trig p x z$ and $\trig p y z$ are spherically thin, then triangle $\trig p x y$ is spherically thin.
\end{thm} 


\begin{wrapfigure}{r}{32mm}
\begin{lpic}[t(-4mm),b(0mm),r(0mm),l(0mm)]{pics/cat-monoton-ineq(1)}
\lbl[b]{14,23;$\dot z$}
\lbl[t]{10,.5;$\dot p$}
\lbl[r]{1,14;$\dot x$}
\lbl[l]{30.5,14;$\dot y$}
\lbl[tl]{13,13;$\dot w$}
\end{lpic}
\end{wrapfigure}

\parit{Proof.}
Construct  the model triangles $\trig{\dot p}{\dot x}{\dot z}\z=\modtrig(p x z)_{\EE^2}$ 
and $\trig {\dot p} {\dot y} {\dot z}\z=\modtrig(p y z)_{\EE^2}$ so that $\dot x$ and $\dot y$ lie on opposite sides of $[\dot p\dot z]$.

Let us show that 
\[\angk{z}{p}{x}+\angk{z}{p}{y}
\ge
\pi.
\eqlbl{eq:<+<>=pi}\]
Suppose the contrary, that is
\[\angk{z}{p}{x}+\angk{z}{p}{y}
<
\pi.\]
Then for some point $\dot w\in[\dot p\dot z]$, we have \[\dist{\dot x}{\dot w}{}+\dist{\dot w}{\dot y}{}
<
\dist{\dot x}{\dot z}{}+\dist{\dot z}{\dot y}{}=\dist{x}{y}{}.\]
Let $w\in[p z]$ correspond to $\dot w$; that is, $\dist{z}{w}{}=\dist{\dot z}{\dot w}{}$. 
Since $\trig p x z$ and $\trig p y z$ are thin, we have 
\[\dist{x}{w}{}+\dist{w}{y}{}<\dist{x}{y}{},\]
contradicting the triangle inequality. 

Denote by $\dot D$ the union of two solid triangles $\trig {\dot p}{\dot x}{\dot z}$ and $\trig {\dot p} {\dot y} {\dot z}$.
Further, denote by $\tilde D$ the solid triangle $\trig{\tilde  p}{\tilde  x}{\tilde  y}=\modtrig(p x y)_{\EE^2}$.
By \ref{eq:<+<>=pi}, there is a short map $F\:\tilde D\to \dot D$ that sends 
\begin{align*}
\tilde p&\mapsto \dot p,
&
\tilde x&\mapsto \dot x,
&
\tilde z&\mapsto \dot z,
&
\tilde y&\mapsto \dot y.
\end{align*}
\qedsf

\begin{thm}{Exercise}\label{ex:short-map}
Use Alexandrov's lemma (\ref{lem:alex}) to prove the last statement. 
\end{thm}


By assumption, the natural maps $\trig {\dot p} {\dot x} {\dot z}\to\trig p x z$ and $\trig {\dot p} {\dot y} {\dot z}\to\trig p y z$ are short.  
By composition,  the natural map from $\trig{\tilde  p}{\tilde  x}{\tilde  y}$ to $\trig p y z$ is short, as claimed.

The spherical case is done along the same lines.
\qeds

\begin{thm}{Exercise}\label{ex:convex-balls}
Show that any ball in a proper length $\CAT(0)$ space is a convex set.

Analogously, show that any ball of radius $R<\tfrac\pi2$ in a proper length $\CAT(1)$ space  is a convex set.
\end{thm}

Recall that a set $A$ in a metric space $\spc{U}$ is called locally convex if for any point $p\in A$ there is an open neighborhood $\spc{U}\ni p$ such that any geodesic in $\spc{U}$ with  ends in $A$ lies in~$A$. 

\begin{thm}{Exercise}\label{ex:locally-convex}
Let $\spc{U}$ be a proper length $\CAT(0)$ space.
Show that any closed, connected, locally convex set in $\spc{U}$ is convex.
\end{thm}

\begin{thm}{Exercise}\label{ex:closest-point}
Let  $\spc{U}$ be a proper length $\CAT(0)$ space 
and $K\subset \spc{U}$ be a closed convex set.
Show that: 

\begin{subthm}{ex:closest-point:a}
For each point $p\in \spc{U}$ there is unique point $p^*\in K$ that minimizes the distance $\dist{p}{p^*}{}$.
\end{subthm}

\begin{subthm}{}
The closest-point projection $p\mapsto p^*$ defined by (\ref{SHORT.ex:closest-point:a}) is short. 
\end{subthm}

\end{thm}




















\begin{thm}{Advanced exercise}\label{ex:urysohn-contractable}
 Show that the space $\spc{U}$ is contactable.
\end{thm}


\parbf{Advanced exercise~\ref{ex:urysohn-contractable}.}
Note that points in the space $\spc{X}_\infty$ constructed in the proof of \ref{prop:univeral-separable} can be multiplied number $t\in [0,1]$ --- simply multiply each function by factor $t$.
That defines a map 
\[\lambda_t\:\spc{X}_\infty\to \spc{X}_\infty\]
that scales all distances by factor $t$.
The map $\lambda_t$ can be extended to the completion of $\spc{X}_\infty$, which is isometic to $\spc{U}_d$ (or $\spc{U}$).

Observe that 
the map $\lambda_1$ is the identity  and $\lambda_0$ maps whole space to a single point, say $x_0$ --- that is the only point of $\spc{X}_0$.
Further note that the map $(t,p)\mapsto \lambda_t(p)$ is continuous ---  in particular $\spc{U}_d$ and $\spc{U}$ are contractible.\qeds

Source: \cite[(d) on page 82]{gromov-2007}.

Observe that for any point $p\in \spc{U}_d$ the curve $t\mapsto \lambda_t(p)$ is a geodesic path from $p$ to $x_0$.








Note that $\spc{M}$ --- the space of compact metric spaces can be treated as a space of compact subsets in $\spc{U}$ up to congruence.
Namely two subsets $A$ and $A'$ are called \emph{congruent} (briefly $A\cong A'$) if there is isometry of the ambient space $\spc{U}$ that maps $A$ to $A'$.
Let us define distance between congruence classes of two compact subsets $A$ and $B$ as 
\[\inf\set{\dist{A'}{B}{\spc{H}(\spc{U})}}{A'\cong A}.\]


By \ref{prop:sep-in-urys}, any compact metric spaces $\spc{K}$ admits a distance preserving map $f\:\spc{K}\to\spc{U}$.
Moreover by \ref{thm:compact-homogeneous} any two such maps $f_1$ and $f_2$ differ by isometry of $\spc{U}$;
that is, there is an isometry $\iota\:\spc{U}\to\spc{U}$ such that $f_2=\iota\circ f_1$.
In particular $f_1(\spc{K})\cong f_2(\spc{K})$.










\section{Ultratangent space} 

Recall that we assume that $\omega$ is a once for all fixed choice of a nonprinciple ultrafilter.

For a metric space $\spc{X}$ and a positive real number $\lambda$,
we will denote by $\lambda\cdot\spc{X}$ its \emph{$\lambda$-blowup}\index{blowup},
which is a metric space with the same underlying set as $\spc{X}$ and the metric multiplied by $\lambda$.
The tautological bijection $\spc{X}\to \lambda\cdot\spc{X}$ will be denoted as $x\mapsto x^\lambda$, 
so 
\[\dist{x^\lambda}{y^\lambda}{}
=
\lambda\cdot\dist[{{}}]{x}{y}{}\] 
for any $x,y\in \spc{X}$.

The $\omega$-blowup $\omega\cdot\spc{X}$ of $\spc{X}$ is defined as the $\omega$-limit
of the $n$-blowups $n\cdot\spc{X}$; that is,
\[\omega\cdot\spc{X}
\df
\lim_{n\to\omega} n\cdot\spc{X}.\]

Given a point $x\in \spc{X}$ we can consider the sequence $x^n\in n\cdot\spc{X}$;
it corresponds to a point $x^\omega\in \omega\cdot\spc{X}$.
Note that if $x\ne y$, then 
\[\dist{x^\omega}{y^\omega}{\omega\cdot\spc{X}}=\infty;\]
that is, 
$x^\omega$ and $y^\omega$ 
belong to different metric components of $\omega\cdot\spc{X}$.

The metric component of $x^\omega$ in $\omega\cdot\spc{X}$ is called ultratangent space of $\spc{X}$ at $x$ and it is denoted as $\T^\omega_x\spc{X}$.

Equivalently, ultratangent space $\T^\omega_x\spc{X}$ can be defined the following way.
Consider all the sequences of points $x_n\in \spc{X}$ such that
the sequence $\ell_n=n\cdot\dist{x}{x_n}{\spc{X}}$ is bounded.
Define the pseudodistance between two such sequences as 
\[\dist{(x_n)}{(y_n)}{}
=
\lim_{n\to\omega}n\cdot\dist{x_n}{y_n}{\spc{X}}.\]
Then $\T^\omega_x\spc{X}$ is the corresponding metric space.

Tangent space as well as ultratangent space, 
generalize the notion of tangent space of Riemannian manifold.
In the simplest cases these two notions define the same space.
In general, they are different and both useful ---
often lack of a property in one is compensated by the other.

It is clear from the definition that tangent space has cone structure.
On the other hand, in general, ultratangent space does not have a cone structure; 
the Hilbert's cube $\prod_{n=1}^\infty[0,2^{-n}]$ is an example --- it is $\Alex{0}$ as well as $\CAT{0}$.

The next theorem shows that the tangent space $\T_p$ can be (and often will be) considered as a subset of  $\T^\omega_p$.

\begin{thm}{Theorem}\label{thm:tangent-ultratangent}
\label{thm:T-in-T^w} 
Let $\spc{X}$ be a metric space with defined angles.
Then for any $p\in \spc{L}$, there is an distance preserving map 
\[\iota:\T_p\hookrightarrow \T^\omega_p\] 
such that for any geodesic $\gamma$ starting at $p$
we have 
\[\gamma^+(0)\mapsto \lim_{n\to\omega}[\gamma(\tfrac1n)]^n.\]

\end{thm}

\parit{Proof.}
Given $v\in \T'_p$ 
choose a geodesic $\gamma$ that starts at $p$ such that $\gamma^+(0)\z=v$.
Set $v^n=[\gamma(\tfrac1n)]^n\in n\cdot \spc{X}$ and 
\[v^\omega=\lim_{n\to\omega}v^n.\]

Note that the value $v^\omega\in\T^\omega_p$ does not depend on choice of $\gamma$;
that is, if $\gamma_1$ is an other geodesic starting at $p$ such that $\gamma_1^+(0)=v$,
then 
\[\lim_{n\to\omega}v^n=\lim_{n\to\omega}v_1^n,\]
where $v_1^n=[\gamma_1(\tfrac1n)]^n\in n\cdot \spc{X}$.
The latter follows since
\[\dist{\gamma(t)}{\gamma_1(t)}{\spc{X}}=o(t)\]
and therefore $\dist{v^n}{v_1^n}{n\cdot \spc{X}}\to 0$ s $n\to\infty$.



Set $\iota(v)=v^\omega$.
Since angles between geodesics in $\spc{X}$ are defined, for any $v,w\in \T_p'$ we have
$n\cdot\dist[{{}}]{v_n}{w_n}{}\to\dist{v}{w}{}$.
Thus $\dist{v_\omega}{w_\omega}{}=\dist{v}{w}{}$; that is, $\iota$ is a global isometry of $\T_p'$.

Since $\T_p'$ is dense in $\T_p$,
we can extend $\iota$ to a global isometry $\T_p\to \T^\omega_p$.
\qeds

{\sloppy

\section[Gromov--Hausdorff and ultralimits]{Gromov--Hausdorff convergence and ultralimits}

}

\begin{thm}{Theorem}\label{thm:ultra-GH}
Assume $\spc{X}_n$ is a sequence of complete spaces. 
Let $\spc{X}_n\to \spc{X}_\omega$ as $n\to\omega$,
and $\spc{Y}_n\subset \spc{X}_n$ 
be a sequence of subsets such that $\spc{Y}_n\GHto\spc{Y}_\infty$. 
Then there is a distance preserving map 
$\iota:\spc{Y}_\infty\to \spc{X}_\omega$.

Moreover:

\begin{subthm}{thm:ultra-GH:a}
If $\spc{X}_n\GHto \spc{X}_\infty$ 
and $\spc{X}_\infty$ is compact, then 
$\spc{X}_\infty$ is isometric to $\spc{X}_\omega$.
\end{subthm}

\begin{subthm}{thm:ultra-GH:b}
If $\spc{X}_n\GHto \spc{X}_\infty$ 
and $\spc{X}_\infty$ is proper, then 
$\spc{X}_\infty$ is isometric to a metric component of $\spc{X}_\omega$.
\end{subthm}

\end{thm}

\parit{Proof.} 
For each point $y_\infty\in \spc{Y}_\infty$ 
choose a lifting $y_n\in \spc{Y}_n$.
Pass to the $\omega$-limit $y_\omega\in \spc{X}_\omega$ of $(y_n)$.
Clearly for any $y_\infty,z_\infty\in \spc{Y}_\infty$, 
we have 
\[\dist{y_\infty}{z_\infty}{\spc{Y}_\infty}=\dist{y_\omega}{z_\omega}{\spc{X}_\omega};\] 
that is, the map $y_\infty\mapsto y_\omega$ gives a distance preserving map $\iota:\spc{Y}_\infty\to \spc{X}_\omega$. 


\parit{(\ref{SHORT.thm:ultra-GH:a})$+$(\ref{SHORT.thm:ultra-GH:b}).}
Fix $x_\omega\in \spc{X}_\omega$.
Choose a sequence $x_n\in \spc{X}_n$ 
such that $x_n\to x_\omega$ as $n\to\omega$. 

Denote by $\bm{X}=\spc{X}_\infty\sqcup\spc{X}_1\sqcup\spc{X}_2\sqcup\dots$ the common space for the convergence $\spc{X}_n\GHto \spc{X}_\infty$;
as in the definition of Gromov--Hausdorff convergence.
Consider the sequence $(x_n)$ 
as a sequence of points in~$\bm{X}$.

If the $\omega$-limit $x_\infty$ of $(x_n)$ exists, 
it must lie in $\spc{X}_\infty$. 

The point $x_\infty$, if defined, does not depend on the choice of $(x_n)$.
Indeed, if $y_n\in\spc{X}_n$ is an other sequence such that $y_n\to x_\omega$ as $n\to\omega$, then 
\[
\dist{y_\infty}{x_\infty}{}=\lim_{n\to\omega}\dist{y_n}{x_n}{}=0;
\]
that is, $x_\infty=y_\infty$.


In this way we obtain a map $\nu\:x_\omega\to x_\infty$;
it is defined on a subset of $\Dom\nu \subset\spc{X}_\omega$.
By construction of $\iota$, 
we get  $\iota\circ\nu(x_\omega)=x_\omega$ for any $x_\omega\in \Dom\nu$.

Finally note that if $\spc{X}_\infty$ is compact, then $\nu$ is defined on all of $\spc{X}_\omega$;
this proves (\ref{SHORT.thm:ultra-GH:a}).

If $\spc{X}_\infty$ is proper, choose any point $z_\infty\in \spc{X}_\infty$
and set $z_\omega=\iota(z_\infty)$.
For any point $x_\omega\in \spc{X}_\omega$ at finite distance from $z_\omega$,
for the sequence $x_n$ 
as above we have that $\dist{z_n}{x_n}{}$ is bounded for $\omega$-almost all $n$.
Since $\spc{X}_\infty$ is proper, $\nu(x_\omega)$ is defined;
in other words $\nu$ is defined on the metric component of $z_\omega$.
Hence (\ref{SHORT.thm:ultra-GH:b}) follows.
\qeds

\begin{thm}{Corollary} 
\label{cor:ulara-geod}
The $\omega$-limit of a sequence of complete length spaces is geodesic.
\end{thm}

\parit{Proof.} Given two points $x_\omega,y_\omega\in \spc{X}_\omega$, find two bounded sequences of points $x_n,y_n\in \spc{X}_n$, $x_n\to x_\omega$, $y_n\to y_\omega$ as $n\to\omega$.
Consider a sequence of paths  $\gamma_n\:[0,1]\to \spc{X}_n$ from $_n$ to $y_n$
 such that 
\[\length\gamma_n\le \dist{x_n}{y_n}{}+\tfrac{1}{n}.\]
Apply Theorem~\ref{thm:ultra-GH} 
for the images $\spc{Y}_n=\gamma_n([0,1])\subset \spc{X}_n$.
\qeds

\section{Ultralimits of sets}

Let $\spc{X}_n$ be a sequence of metric spaces and $\spc{X}_n\to \spc{X}_\omega$
as $n\to \omega$.

For a sequence of sets $\Omega_n\subset \spc{X}_n$,
consider the maximal set $\Omega_\omega\subset \spc{X}_\omega$ such that 
for any $x_\omega\in\Omega_\omega$ and any sequence $x_n\in\spc{X}_n$ such that $x_n\to x_\omega$ as $n\to \omega$, we have $x_n\in\Omega_n$ for $\omega$-almost all $n$.

The set $\Omega_\omega$ is called the  \emph{open $\omega$-limit} of $\Omega_n$;
we could also write  $\Omega_n\to \Omega_\omega$ as $n\to\omega$ or $\Omega_\omega=\lim_{n\to\omega}\Omega_n$. 

{\sloppy

Applying Observation~\ref{obs:ultralimit-is-complete} to the sequence of complements $\spc{X}_n\backslash \Omega_n$, we see that $\Omega_\omega$ is open for any sequence $\Omega_n$.
The definition can be applied for arbitrary sequences of sets, but  
open $\omega$-convergence  will be applied here only for sequences of open sets.

}

\section{Ultralimits of functions}

Recall that a family of submaps between metric spaces $\{f_\alpha\: \spc{X}\to\spc{Y}\}_{\alpha\in\mathcal A}$ is called \emph{equicontinuous} if for any $\eps>0$ there is $\delta>0$ such that for any $p,q\in\spc{X}$ with $\dist{p}{q}{}<\delta$ and any $\alpha\in\mathcal A$ it holds that $\dist{f(p)}{f(q)}{}<\eps$.

Let $f_n\:\spc{X}_n\to\RR$ be a sequence of subfunctions.

Set $\Omega_n=\Dom f_n$.
Consider the open $\omega$-limit set $\Omega_\omega\subset \spc{X}_\omega$ of $\Omega_n$.

Assume there is a subfunction $f_\omega\:\spc{X}_\omega\to\RR$
that satisfies the following conditions: 
(1) $\Dom f_\omega=\Omega_\omega$, (2) if $x_n\to x_\omega\in \Omega_\omega$ for a sequence of points $x_n\in\spc{X}_n$, then $f_n(x_n)\to f_\omega(x_\omega)$ as $n\to\omega$.
In this case 
the subfunction $f_\omega\:\spc{X}_\omega\to\RR$ 
is said to be the 
$\omega$-limit of $f_n\:\spc{X}_n\to\RR$.

The following lemma gives a mild condition on a sequence of functions $f_n$
guaranteeing the existence of its $\omega$-limit.

\begin{thm}{Lemma}
Let $\spc{X}_n$ be a sequence of metric spaces
and $f_n\:\spc{X}_n\to\RR$ be a sequence of subfunctions.

Assume for any positive integer $k$, there is an open set $\Omega_n(k)\subset \Dom f_n$
such that the restrictions $f_n|_{\Omega_n(k)}$ are uniformly bounded and continuous
and the open $\omega$-limit of $\Omega_n(n)$ coincides with the open $\omega$-limit of $\Dom f_n$.
Then the $\omega$-limit of $f_n$ is defined.

In particular, if the $f_n$ are uniformly bounded and continuous, then the $\omega$-limit is defined.
\end{thm}

The proof is straightforward.

{\sloppy

\begin{thm}{Exercise}\label{ex:nonconvex-limit}
Construct a sequence of compact length spaces 
$\spc{X}_n$  
with a converging sequence of $\Lip$-Lipschitz concave functions $f_n\:\spc{X}_n\to\RR$ such that
the $\omega$-limit $\spc{X}_\omega$ of $\spc{X}_n$ is compact
and the $\omega$-limit $f_\omega\:\spc{X}_\omega\to\RR$ of $f_n$ is not concave.
\end{thm}

}

If $f\:\spc{X}\to\RR$ is a subfunction, 
the $\omega$-limit of the constant sequence $f_n=f$ is called the $\omega$-power of $f$ and denoted by $f^\omega$.
So
\[f^\omega\:\spc{X}\to\RR,\ \ f^\omega(x_\omega)=\lim_{n\to\omega} f(x_n).\]

Recall that we treat $\spc{X}$ as a subset of its $\omega$-power $\spc{X}^\omega$.
Note that $\Dom f=\spc{X}\cap \Dom f^\omega$.
Moreover, 
if $\oBall(x,\eps)_{\spc{X}}\subset \Dom f$
then $\oBall(x,\eps)_{\spc{X}^\omega}\subset \Dom f^\omega$.


\parbf{Ultradifferential.}
Given a function $f\:\spc{L}\to\RR$, consider sequence of functions $f_n\:n\cdot\spc{L}\to\RR$, defined by 
\[f_n(x^n)=n\cdot(f(x)-f(p)),\]
here $x^n\in n\cdot\spc{L}$ is the point corresopnding to $x\in\spc{L}$.
While $n\cdot(\spc{L},p)\to(\T^\omega,\0)$ as $n\to\omega$, 
functions $f_n$ converge to $\omega$-differential of $f$ at $p$.
It will be denoted by $\dd_p^\omega f$;
\[\dd_p^\omega f\:\T_p^\omega\to\RR,\ \ \dd_p^\omega f=\lim_{n\to\omega} f_n.\] 

Clearly, the $\omega$-differential $\dd_p^\omega f$ of a locally Lipschitz subfunction $f$ is defined at each point $p\in \Dom f$.
















\section{Comments} 

Given two metric spaces $\spc{X}$ and $\spc{Y}$, we will write $\spc{X}\preccurlyeq \spc{Y}$ if there is a noncontracting map $f\:\spc{X}\to \spc{Y}$;
that is, if 
$$ |x-x'|_{\spc{X}}\le|f(x)-f(x')|_{\spc{Y}}$$
for any $x,x'\in \spc{X}$.

Further, given $\eps>0$, we will write $\spc{X}\preccurlyeq \spc{Y}+\eps$
if there is a map $f\:\spc{X}\to \spc{Y}$ such that 
$$|x-x'|_{\spc{X}}\le|f(x)-f(x')|_{\spc{Y}}+\eps$$
for any $x,x'\in \spc{X}$.

Define 
$$\dist[\star]{\spc{X}}{\spc{Y}}{\spc{M}}=\inf\set{\eps}{\spc{X}\preccurlyeq \spc{Y}+\eps
\quad\text{and}\quad
\spc{Y}\preccurlyeq \spc{X}+\eps}$$
It turns out that $\dist[\star]{*}{*}{\spc{M}}$ is a different metric on the set of isometry classes of compact metric spaces; that is, in general $\dist[\star]{\spc{X}}{\spc{Y}}{\spc{M}}\not=|\spc{X}-\spc{Y}|_{\spc{M}}$. 
However, these two metrics define the same topology on $\spc{M}$.
More precicely:

\begin{thm}{Proposition}\label{GH-po}
For any sequence of compact metric spaces $(\spc{X}_n)$ and a compact metric space $\spc{X}_\infty$,
we have
$$|\spc{X}_n-\spc{X}_\infty|_{\spc{M}}\to 0
\quad\iff\quad
\dist[\star]{\spc{X}_n}{\spc{X}_\infty}{\spc{M}}\to 0$$ 
as $n\to\infty$.
\end{thm}

We will not give a proof of this proposition. 
Likely, we will not use it further in the lectures, 
but it might help you to build intuition for Gromov--Hausdorff convergence.
If you want to prove it yourself look in the proof of Theorem~\ref{thm:GH-is-a-metric} 
and try to modify it using ideas from the proof of Problem~\ref{pr:non-contracting=>isometry}.

The Gromov--Hausdorff distance can be defined for arbitrary pair of metric space.
Therefore it is natural to ask why we only consider compact metric spaces.
First note the Gromov--Hausdorff distance from any metric space $\spc{X}$ 
to its completion $\bar {\spc{X}}$ is zero.
Therefore if you want to end up in a metric space, it is better to consider only complete metric spaces.

Further, the distance between one-point-space and a metric spce with infinite diameter is infinite.
Therefore one has to either consider only bounded metric spaces (that is, the spaces with finite diameter)
or relux the definition of metric space which allow metric to take infinite value.

Finally, the class of isometry classes of all bounded complete metric spaces forms a class, but not a set.
Hence again we have two choices: either relux the definition of metric space so its points will form a class, or restrict further the class of spaces for which the isometry classes will form a set.

\begin{thm}{Exercise}
Prove that isometry classes of compact metric spaces form a set. 
\end{thm}

\begin{thm}{Exercise}\label{pr:GH1}
Let $\spc{X}=\{x,y,z\}$ be a three point subset of Euclidean plane with distances
$$|x-y|=|y-z|=|z-x|=1.$$
\begin{enumerate}[(i)]
\item Find the minimal Hausdorff distance from $\spc{X}$ to a one-point subset of the plane.
\item Find the Gromov--Hausdorff distance from $\spc{X}$ to the one-point metric space. 
\end{enumerate}
\end{thm}

\begin{thm}{Exercise}\label{pr:GH2}
Let $\spc{X}$ and $\spc{Y}$ be a compact metric spaces which have isometric $\eps$-nets.
Show that 
$$|\spc{X}-\spc{Y}|_{\spc{M}}\le 2\cdot\eps.$$
Is it always true that 
$$|\spc{X}-\spc{Y}|_{\spc{M}}\le \eps?$$
\end{thm}




\begin{thm}{Exercise}\label{pr:GH3}
Define the \emph{radius of a metric space}\index{radius of a metric space} $\spc{X}$ as 
$$\rad \spc{X}=\inf_x\left\{\sup_y\{|x-y|_{\spc{X}}\}\right\}.$$
Equivalently, 
$$\rad \spc{X}=\inf\set{R>0}{\text{there is}\ x\in \spc{X}\  \text{such that}\ B_R(x)\supset \spc{X}}.$$
 
\begin{enumerate}[(i)]
\item Show that for any compact metric space $\spc{X}$ we have
$$\tfrac12\cdot\diam \spc{X}\le \rad \spc{X}\le \diam \spc{X}.$$
\item Show that for any compact metric spaces $\spc{X},\spc{Y}$ we have
$$|\rad \spc{X}-\rad \spc{Y}|\le 2\cdot |\spc{X}-\spc{Y}|_{\spc{M}}.$$
\end{enumerate}
\end{thm}

\begin{thm}{Exercise}\label{pr:F-X}
Let $\spc{X}$ be a metric space.
If two compact sets $A, B$ in $\spc{X}$ are isometric,
we will write $A\iso B$. 
Set
$$d(A,B)=\inf \set{|A'-B'|_{\mathcal{H}(\spc{X})}}{A'\iso A \ \text{and}\ B'\iso B}.$$
Note that if $\spc{X}=\ell^\infty$, then according to Proposition~\ref{prop:GH-with-fixed-Z}, 
$d$ is a metric on $\mathcal{H}(\spc{X})/\iso$ (that is, on the ``$\iso$''-equivalecne classes of $\mathcal{H}(\spc{X})$).

Show that it does not hold for arbitrary metric space $\spc{X}$.
Understand the reason why it holds for $\spc{X}=\ell^\infty$.
\end{thm}


\begin{thm}{Exercise}\label{pr:GH-variation}
Consider the pairs $(\spc{X},A)$, where $\spc{X}$ is a compact metric space and $A$ is a closed subset in $\spc{X}$.
Two such pairs, say $(\spc{X},A)$ and $(\spc{X}',A')$ will be called equivalent (briefly $(\spc{X},A)\sim(\spc{X}',A')$)
if there is an isometry $\iota\:\spc{X}\to \spc{X}'$ such that $\iota(A)=A'$.

Modify the definition of Gromov--Hausdorff metric to construct a natural metric on the set of $\sim$-equivalence classes of the pairs $(\spc{X},A)$.
\end{thm}

Here we introduce so called Gromov--Hausdorff convergence for metric spaces.
This convergence was introduced by Gromov around 1980, published in \cite{gromov-1981}.
Very soon this notion began to be used in all branches of geometry.
In fact today I have difficulty to understand 
how one could do geometry without this type of convergence.%
(Some types of convergences of metric spaces was considered before Gromov,
but they had lack of generality;
each type of convergence was desined to solve one particular problem.)


\begin{thm}{Exercise}\label{ex:euclid-isom}
\begin{subthm}{}
Let $\spc{X},\spc{Y}$ be two compact sets in the Euclidean plane $\RR^2$.
Show that $\spc{X}$ is isometric to $\spc{Y}$ if and only if there is an motrio $\iota\:\RR^2\to \RR^2$
that sends $\spc{X}$ to $\spc{Y}$.
\end{subthm}

\begin{subthm}{}
Find two isometric subsets $\spc{X},\spc{Y}$ of $\ell^\infty$
such that there is no isometry $\iota\:\ell^\infty\to \ell^\infty$ 
that sends $\spc{X}$ to $\spc{Y}$.
\end{subthm}
\end{thm}
