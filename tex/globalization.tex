\chapter{Globalization}\label{chapter:globalization}

This lecture is nearly a copy of \cite[Sections 3.1--3.3]{alexander-kapovitch-petrunin-2019};
here we introduce locally $\CAT(0)$ spaces and prove the globalization theorem that provides a sufficient condition for locally $\CAT(0)$ spaces to be globally $\CAT(0)$.

\section{Locally CAT spaces}

We say that a space $\spc{U}$ is \index{locally $\CAT(\kappa)$ space}\emph{locally $\CAT(0)$} (or {}\emph{locally $\CAT(1)$}) if
a small closed ball centered at any point $p$ in $\spc{U}$ is $\CAT(0)$ (or $\CAT(1)$, respectively).

For example, the circle $\mathbb{S}^1=\RR/\ZZ$ is locally isometric to $\RR$, and so $\mathbb{S}^1$ is locally $\CAT(0)$.
On the other hand, $\mathbb{S}^1$ is not $\CAT(0)$, since closed local geodesics in $\mathbb{S}^1$ are not geodesics, so $\mathbb{S}^1$ does not satisfy Proposition~\ref{cor:loc-geod-are-min}.

If $\spc{U}$ is a proper length space, then it is locally $\CAT(0)$ (or locally $\CAT(1)$) 
if and only if 
each point $p\in \spc{U}$ admits an open neighborhood $\Omega$ that is geodesic and such that any triangle in $\Omega$ is thin (or spherically thin, respectively).

\section{Space of local geodesic paths}\label{sec:geod-space}

In this section we will study behavior of local geodesics in locally $\CAT(\kappa)$  spaces.  
The results will be used in the proof of the globalization theorem (\ref{thm:hadamard-cartan}).

Recall that a \index{path}\emph{path} is a curve parametrized by $[0,1]$.
The space of paths in a metric space $\spc{U}$ comes with the natural metric
\[\dist{\alpha}{\beta}{}
=
\sup\set{\dist{\alpha(t)}{\beta(t)}{\spc{U}}}{t\in[0,1]}.
\eqlbl{eq:dist-between-paths}
\]

\begin{thm}{Proposition}\label{prop:geo-complete}
Let $\spc{U}$ be a proper length, locally $\CAT(\kappa)$ space.

Assume $\gamma_n\:[0,1]\to\spc{U}$ is a sequence of local geodesic paths converging to a path $\gamma_\infty\:[0,1]\to\spc{U}$.
Then $\gamma_\infty$ is a local geodesic path.
Moreover 
\[\length\gamma_n\to\length\gamma_\infty\]
as $n\to\infty$.
\end{thm}

\parit{Proof; $\CAT(0)$ case.} 
Fix $t\in[0,1]$.  
Let $R>0$ be sufficiently small, so that $\cBall[\gamma_\infty(t),R]$ forms a proper length $\CAT(0)$ space.
%"value" looks weird here. value usually means value of a function. V.

Assume that  a local geodesic $\sigma$  is shorter than $R/2$ and intersects the ball $\oBall(\gamma_\infty(t),R/2)$.
Then $\sigma$ cannot leave the ball $\cBall[\gamma_\infty(t),R]$.
Hence, by Proposition~\ref{cor:loc-geod-are-min}, $\sigma$ is a geodesic.  
In particular, for all sufficiently large $n$, any arc of $\gamma_n$ of length $R/2$ or less containing $\gamma_n(t)$ is a geodesic.

Since $\spc{B}=\cBall[\gamma_\infty(t),R]$ is a proper length $\CAT(0)$ space, by \ref{ex:CAT-geodesic},
geodesic segments in $\spc{B}$ depend uniquely on their endpoint pairs.  
Thus there is a subinterval $\II$ of $[0,1]$,
that  contains a neighborhood of $t$ in $[0,1]$
and such that the arc $\gamma_n|_\II$ is minimizing for all large~$n$.
It follows that $\gamma_\infty|_\II$ is a geodesic,
and therefore $\gamma_\infty$ is a local geodesic.

The $\CAT(1)$ case is done in the same way, but one has to assume in addition that $R<\pi$.
\qeds

The following lemma and its proof were suggested to us by Alexander Lytchak.  
This lemma allows  a local geodesic path 
to be moved continuously so that its endpoints follow given trajectories.
This statement was originally proved by Stephanie Alexander and Richard Bishop \cite{alexander-bishop-1990} using a different method.

\begin{thm}{Patchwork along a curve}
\label{lem:patch}
Let $\spc{U}$ be a proper length, locally $\CAT(0)$ space, 
and $\gamma\:[0,1]\to\spc{U}$ be a 
 path.

Then there is a proper length  $\CAT(0)$ space   $\spc{N}$,
an open set $\hat\Omega\subset \spc{N}$,
and a  
 path $\hat\gamma\:[0,1]\to\hat\Omega$,
such that there is an open locally isometric immersion 
$\Phi\:\hat\Omega\looparrowright\spc{U}$ satisfying
$\Phi\circ\hat\gamma=\gamma$.

If $\length\gamma<\pi$,
then the same holds in the $\CAT(1)$ case.
Namely we assume that $\spc{U}$ is a proper length, 
locally $\CAT(1)$ space and construct a proper length $\CAT(1)$ space $\spc{N}$ with the same property as above.
\end{thm}

\parit{Proof.} 
Fix $r>0$ so that for each $t\in[0,1]$,
the closed ball
$\cBall[\gamma(t),r]$ forms a proper length $\CAT(\kappa)$ space.

Choose a partition $0\z=t^0<t^1<\dots<t^n\z=1$ such that 
\[\oBall(\gamma(t^i),r)\supset \gamma([t^{i-1},t^i])\] for all $n>i>0$.
Set $\spc{B}^i=\cBall[\gamma(t^i),r]$.

\begin{figure}[h!]
\vskip-0mm
\centering
\includegraphics{mppics/pic-910}
\end{figure}

Consider the disjoint union $\bigsqcup_i\spc{B}^i=\set{(i,x)}{x\in\spc{B}^i}$ with the minimal equivalence relation $\sim$ such that $(i,x)\sim(i-1,x)$ for all~$i$.
Let  $\spc{N}$ be the space obtained by gluing the $\spc{B}^i$ along~$\sim$.

Note that $A^i=\spc{B}^i\cap\spc{B}^{i-1}$ is convex in $\spc{B}^i$ and in $\spc{B}^{i-1}$.
Applying the Reshetnyak gluing theorem (\ref{thm:gluing}) $n$ times, 
we conclude that $\spc{N}$ is a proper length $\CAT(0)$ space.

For $t\in[t^{i-1},t^i]$, define $\hat\gamma(t)$ as the equivalence class of $(i,\gamma(t))$ in~$\spc{N}$.
Let $\hat\Omega$ be the $\eps$-neighborhood of $\hat\gamma$ in $\spc{N}$, where $\eps>0$ is chosen so that $\oBall(\gamma(t),\eps)\subset\spc{B}^i$ for all $t\in[t^{i-1},t^i]$.

Define $\Phi\:\hat\Omega\to\spc{U}$
by sending the equivalence class of $(i,x)$ to~$x$.
It is straightforward to check that $\Phi$, 
$\hat\gamma$ and $\hat\Omega\subset\spc{N}$ satisfy the conclusion of  the lemma.

The $\CAT(1)$ case is proved in the same way.
\qeds

The following two corollaries follow from:
(1) patchwork (\ref{lem:patch}),
(2) Proposition \ref{cor:loc-geod-are-min}, which states that local geodesics are geodesics in any $\CAT(0)$ space, 
and (3) Proposition~\ref{ex:CAT-geodesic} on uniqueness of geodesics.

\begin{thm}{Corollary}\label{cor:discrete-paths}
If $\spc{U}$ is a proper length, locally $\CAT(0)$ space, then for any pair of points $p,q\in\spc{U}$, the space of all local geodesic paths from $p$ to $q$ is discrete;
that is, for any local geodesic path $\gamma$ connecting $p$ to $q$, there is $\eps>0$ such that for any other local geodesic path $\delta$ from $p$ to $q$ we have
$\dist{\gamma(t)}{\delta(t)}{\spc{U}}>\eps$ for some $t\in[0,1]$.

Analogously, if $\spc{U}$ is a proper length, locally $\CAT(1)$ space, then for any pair of points $p,q\in\spc{U}$,  the space of all local geodesic paths shorter than $\pi$ from $p$ to $q$ is discrete.
\end{thm}

\begin{thm}{Corollary}\label{cor:path-geod}
If $\spc{U}$ is a proper length, locally $\CAT(0)$ space, then 
for any path $\alpha$ there is a choice of   local geodesic path $\gamma_\alpha$  connecting the ends of $\alpha$ such that the map $\alpha\mapsto\gamma_\alpha$ is continuous, and if $\alpha$ is a local geodesic path then $\gamma_\alpha=\alpha$. 

Analogously, if $\spc{U}$ is a proper length, locally $\CAT(1)$ space, then 
for any path $\alpha$ shorter than $\pi$,  
there is a choice of  local geodesic path $\gamma_\alpha$ shorter than $\pi$ connecting the ends of $\alpha$ such that the map $\alpha\mapsto\gamma_\alpha$ is continuous, and if $\alpha$ is a local geodesic path then $\gamma_\alpha=\alpha$.
\end{thm}

\parit{Proof of \ref{cor:path-geod}.} 
We do the $\CAT(0)$ case;
the $\CAT(1)$ case is analogous.

Consider the maximal interval $\II\subset[0,1]$ containing $0$
such that there is a continuous one-parameter family of 
local geodesic paths $\gamma_t$ for $t\in \II$ connecting $\alpha(0)$ to $\alpha(t)$, with $\gamma_t(0)=\gamma_0(t)=\alpha(0)$ for any~$t$. 

By Proposition~\ref{prop:geo-complete}, $\II$ is closed,
so we may assume $\II=[0,s]$ for some $s\in [0,1]$.

Applying  patchwork (\ref{lem:patch}) to  $\gamma_{s}$, 
we find that $\II$ is also open in $[0,1]$. 
Hence $\II=[0,1]$.
Set $\gamma_\alpha=\gamma_1$.

By construction,  if $\alpha$ is a local geodesic path, then $\gamma_\alpha=\alpha$. 

Moreover, from Corollary \ref{cor:discrete-paths},
the construction $\alpha\mapsto \gamma_\alpha$ produces close results for sufficiently close paths in the metric defined by \ref{eq:dist-between-paths};
that is, the map  $\alpha\mapsto \gamma_\alpha$ is continuous.
\qeds

Given a path $\alpha\:[0,1]\to\spc{U}$,
we denote by $\bar\alpha$ the same path traveled in the opposite direction;
that is,
\[\bar\alpha(t)=\alpha(1-t).\]
The \index{product of paths}\emph{product} of two paths  will be denoted with ``$*$'';
if two paths $\alpha$ and $\beta$ connect the same pair of points, then the product $\bar\alpha*\beta$ is a closed curve.

\begin{thm}{Exercise}\label{ex:null-homotopic}
Assume $\spc{U}$ is a proper length, locally $\CAT(1)$ space. 
Consider the construction $\alpha\mapsto\gamma_\alpha$ provided by Corollary~\ref{cor:path-geod}.

Assume that $\alpha$ and $\beta$ are two paths connecting the same pair of points in $\spc{U}$, where 
each is shorter than $\pi$ 
and the product  
$\bar\alpha*\beta$ is null-homotopic in the class of closed curves shorter than $2\cdot\pi$.
Show that $\gamma_\alpha=\gamma_\beta$.
\end{thm}

\section{Globalization}\label{sec:Hadamard--Cartan}



\begin{thm}{Globalization theorem}
\label{thm:hadamard-cartan}
If a proper length, locally $\CAT(0)$ space is simply connected, then it 
is $\CAT(0)$.

Analogously, suppose $\spc{U}$ is a proper length, locally $\CAT(1)$ space
such that any closed curve $\gamma\:\mathbb{S}^1\to \spc{U}$ shorter than $2\cdot\pi$
is null-homotopic in the class of closed curves shorter than $2\cdot\pi$.
Then $\spc{U}$ is $\CAT(1)$.
\end{thm}


The surface on the diagram 
is an example of a simply connected space that  is locally $\CAT(1)$ but not $\CAT(1)$.
\begin{figure}[h!]
\vskip0mm
\centering
\includegraphics{mppics/pic-930}
\end{figure}
To contract the marked curve one has to increase its length to $2\cdot\pi$ or more;
in particular the surface does not satisfy the assumption of the globalization theorem.


The proof of the globalization theorem relies on the following theorem, 
which is essentially  \cite[Satz 9]{alexandrov-1957}.  

\begin{thm}{Patchwork globalization theorem}\label{thm:alex-patch}
A proper length, locally $\CAT(0)$ space $\spc{U}$ is $\CAT(0)$
if and only if all pairs of points in $\spc{U}$  are joined by unique geodesics, and these geodesics depend continuously on their endpoint pairs.

Analogously, a proper length, locally $\CAT(1)$ space $\spc{U}$ is $\CAT(1)$ 
if and only if all pairs of points in $\spc{U}$ at distance less than $\pi$ are joined by unique geodesics, and these geodesics depend continuously on their endpoint pairs.
\end{thm}

The proof uses a thin-triangle decomposition with the inheritance lemma (\ref{lem:inherit-angle}) and the following construction:

\begin{thm}{Line-of-sight map} \label{def:sight}
Let  $p$ be a point and $\alpha$ be a curve of finite length in  a length space~$\spc{X}$. 
Let $\mathring\alpha:[0,1]\to\spc{U}$ be the constant-speed parametrization of~$\alpha$.  
If   $\gamma_t\:[0,1]\to\spc{U}$ is a geodesic path from $p$ to $\mathring\alpha(t)$, we say 
\[
[0,1]\times[0,1]\to\spc{U}\:(t,s)\mapsto\gamma_t(s)
\]
is a \index{line-of-sight map}\emph{line-of-sight map from $p$ to $\alpha$}.  
\end{thm}

\parit{Proof of the patchwork globalization theorem (\ref{thm:alex-patch}).} 
Note that the implication ``only if'' follows from \ref{ex:CAT-geodesic} and \ref{ex:convex-dist}; it remains to prove the ``if'' part.

Fix a triangle $\trig p x y$  in $\spc{U}$. 
We need to show that $\trig p x y$ is thin.

By the assumptions, the line-of-sight map  $(t,s)\mapsto\gamma_t(s)$ from $p$ to   $[x y]$ is uniquely defined and continuous.    


\begin{figure}[h!]
\vskip0mm
\centering
\includegraphics{mppics/pic-950}
\end{figure}

Fix  a partition \[0\z=t^0\z<t^1\z<\z\dots\z<t^N=1,\] 
and set $x^{i,j}=\gamma_{t^i}(t^j)$. 
Since the line-of-sight map is continuous and $\spc{U}$ is locally $\CAT(0)$, we may assume that the triangles 
\[\trig{x^{i,j}}{x^{i,j+1}}{x^{i+1,j+1}}\quad\text{and}\quad\trig{x^{i,j}}{x^{i+1,j}}{x^{i+1,j+1}}\] 
are thin for each pair $i$,~$j$.

Now we show that the thin property propagates to $\trig p x y$ by repeated application of the inheritance lemma (\ref{lem:inherit-angle}):
\begin{itemize}
\item 
For fixed $i$, 
sequentially applying the lemma shows that the triangles 
$\trig{p}{x^{i,1}}{x^{i+1,2}}$, 
$\trig{p}{x^{i,2}}{x^{i+1,2}}$, 
$\trig{p}{x^{i,2}}{x^{i+1,3}}$,
and so on are thin. 
\end{itemize}
In particular, for each $i$, the long triangle $\trig{p}{x^{i,N}}{x^{i+1,N}}$ is thin.
\begin{itemize} 
\item 
By the same lemma the  triangles $\trig{p}{x^{0,N}}{x^{2,N}}$, $\trig{p}{x^{0,N}}{x^{3,N}}$, and so on, are thin. 
\end{itemize}
In particular, $\trig p x y=\trig{p}{x^{0,N}}{x^{N,N}}$ is thin.
\qeds

\parit{Proof of the globalization theorem; $\CAT(0)$ case.}
Let $\spc{U}$ be a proper length, locally $\CAT(0)$ space that is simply connected.
Given a path $\alpha$ in $\spc{U}$, 
denote by $\gamma_\alpha$ the local geodesic path provided by Corollary \ref{cor:path-geod}.
Since the map $\alpha\mapsto\gamma_\alpha$ is continuous, by Corollary~\ref{cor:discrete-paths}
we have $\gamma_\alpha=\gamma_\beta$ for any pair of  paths $\alpha$ and $\beta$  homotopic relative to the ends.

Since $\spc{U}$ is simply connected, any pair of paths with common ends are homotopic.  In particular, if $\alpha$ and $\beta$ are local geodesics from $p$ to $q$, then $\alpha =\gamma_\alpha=\gamma_\beta=\beta$ by Corollary \ref{cor:path-geod}.
It follows that any two points $p,q\in\spc{U}$ are joined by a unique local geodesic that depends continuously on $(p,q)$.

Since $\spc{U}$ is geodesic, it remains to apply the patchwork globalization theorem (\ref{thm:alex-patch}).

\parit{$\CAT(1)$ case.}
The proof goes along the same lines, 
but one needs to use Exercise~\ref{ex:null-homotopic}. \qeds

\begin{thm}{Corollary}\label{cor:closed-geod-cat} 
Any compact length, locally $\CAT(0)$ space that contains no closed local geodesics is $\CAT(0)$. 
 
Analogously, any compact length, locally $\CAT(1)$ space that  contains no closed local geodesics shorter than $2\cdot\pi$ is $\CAT(1)$.
\end{thm}

\parit{Proof.}
By the globalization theorem (\ref{thm:hadamard-cartan}), we need to show that the space is simply connected.
Assume the contrary. 
Fix a nontrivial homotopy class of closed curves.

Denote by $\ell$ the exact lower bound for the lengths of curves in the class.
Note that $\ell>0$;
otherwise there would be a closed noncontractible curve in a $\CAT(0)$ neighborhood of some point, contradicting \ref{ex:contractible}.

Since the space is compact, the class contains a length-minimizing curve, 
which must be a closed local geodesic. 

The $\CAT(1)$ case is analogous, one only has to consider a homotopy class of closed curves shorter than $2\cdot\pi$.
\qeds

\begin{thm}{Exercise}\label{ex:geod-circle}
Prove that any compact length, locally $\CAT(0)$ space $\spc{X}$ that is not $\CAT(0)$ contains a \index{geodesic circle}\emph{geodesic circle};
that is, a simple closed curve $\gamma$ such that 
for any two points $p,q\in\gamma$, one of the arcs of $\gamma$ with endpoints $p$ and $q$ is a  geodesic.

Formulate and prove the analogous statement for $\CAT(1)$ spaces.
\end{thm}

\begin{thm}{Advanced exercise}\label{ex:branching-cover} 
Let $\spc{U}$ be a proper length $\CAT(0)$ space.
Assume $\tilde{\spc{U}}\to \spc{U}$ is a metric  double cover branching along a geodesic.
(For example 3-dimensional Euclidean space admits a double cover branching along a line.)

Show that $\tilde{\spc{U}}$ is $\CAT(0)$.
\end{thm}

\parit{Hint:} Apply the globalization theorem (\ref{thm:hadamard-cartan}) and that an $r$-neighborhood of convex set is convex (\ref{ex:convex-nbhd}).


\section{Remarks}

Riemannian manifolds with nonpositive sectional curvature are locally $\CAT(0)$.
The original formulation of the 
\index{globalization theorem}\emph{globalization theorem}, or 
\index{Hadamard--Cartan theorem}\emph{Hadamard--Cartan theorem}, states that if $M$ is a complete Riemannian manifold with sectional curvature at most $0$,  
then the exponential map at any point $p\in M$ is a covering;
in particular it implies that the universal cover of $M$ is diffeomorphic to the Euclidean space of the same dimension.

In this generality, this theorem appeared in the lectures of Elie Cartan~\cite{cartan}.
This theorem was proved for surfaces in Euclidean $3$-space 
by Hans von Mangoldt \cite{mangoldt}
and a few years later independently for two-dimensional Riemannian manifolds by Jacques Hadamard \cite{hadamard}.

Formulations for metric spaces of different generality were proved by 
Herbert Busemann \cite{busemann-CBA},
Willi Rinow \cite{rinow},
Mikhael Gromov \cite[p.~119]{gromov-1987}. 
A detailed proof of Gromov's statement was given by Werner Ballmann \cite{ballmann-1995} when $\spc{U}$ is proper,
and by the Stephanie Alexander and Richard Bishop \cite{alexander-bishop-1990} in more generality.

For proper $\CAT(1)$ spaces, the globalization theorem was proved by Brian Bowditch~\cite{bowditch}.

The globalization theorem holds for complete length spaces (not necessary proper spaces) \cite{alexander-kapovitch-petrunin-2025}.


The patchwork globalization (\ref{thm:alex-patch}) is proved by Alexandrov \cite[Satz 9]{alexandrov-1957}.
For proper spaces one can remove the continuous dependence from the formulation; it follows from uniqueness.
For complete spaces the later is not true \cite[Chapter I, Exercise 3.14]{bridson-haefliger}.

For spaces with curvature bounded below globalization requires no additional condition.
Namely the following theorem holds \cite[see][and the references therein]{alexander-kapovitch-petrunin-2025}.

\begin{thm}{Globalization theorem}\label{thm:cbb-globalization}
Any complete length locally $\CBB(\kappa)$ space is $\CBB(\kappa)$.
\end{thm}




