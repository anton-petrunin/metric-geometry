\chapter{Injective spaces}\label{chap:injective}


Injective hull is a useful construction that provides a canonical choice of a specially nice (injective) space that includes a given metric space. 
This construction is similar to the convex hull in euclidean space.
The following exercise gives a bridge from the latter to the former.

\begin{thm}{Advanced exercise}\label{ex:conv-short}
Show that $A\subset \RR^n$ is a closed convex set if and only if for any  $B\subset \RR^n$ any short map $B\to A$ can be extended to a short map $\RR^n\to A$.
\end{thm}

\section{Definition}

\begin{thm}{Definition}\label{def:injective}
A metric space $\spc{Y}$ is called \index{injective space}\emph{injective} if, for any metric space $\spc{X}$ and any of its subspace $\spc{A}$,
any short map $f\:\spc{A}\to \spc{Y}$ can be extended to a short map $F\:\spc{X}\to \spc{Y}$;
that is, $f=F|_{\spc{A}}$.
\end{thm}

\begin{thm}{Exercise}\label{ex:inj=complete-geodesic-contractible}
Show that any injective space is 
\begin{multicols}{3}

\begin{subthm}{ex:inj=complete-geodesic-contractible:complete}
complete,
\end{subthm}

\begin{subthm}{ex:inj=complete-geodesic-contractible:geodesic}
geodesic, and
\end{subthm}

\begin{subthm}{ex:inj=complete-geodesic-contractible:contractible}
contractible.
\end{subthm}

\end{multicols}

\end{thm}

\begin{thm}{Exercise}\label{ex:bicombing}
Show that for any injective space $\spc{Y}$ there is a map $m\:\spc{Y}\times\spc{Y}\to\spc{Y}$ (the \index{midpoint map}\emph{midpoint map}) such that the inequality
\[2\cdot \dist{p}{m(x,y)}{\spc{Y}}\le\dist{p}{x}{\spc{Y}}+\dist{p}{y}{\spc{Y}}\]
holds for any $p,x,y\in \spc{Y}$.
\end{thm}

\begin{thm}{Exercise}\label{ex:injective-spaces}
Show that the following spaces are injective:
\begin{subthm}{ex:injective-spaces:R}
the real line;
\end{subthm}

\begin{subthm}{ex:injective-spaces:tree}
complete metric tree;
\end{subthm}

\begin{subthm}{ex:injective-spaces:ell-infty}
The space $\ell^\infty(\spc{S})$ for any set $\spc{S}$ (defined in \ref{lem:kuratowski}).
In particular, the coordinate plane with the metric induced by the $\ell^\infty$-norm.
\end{subthm}

\end{thm}

\begin{thm}{Exercise}\label{ex:extr-ball}
Let $\spc{Y}$ be an injective space.

\begin{subthm}{ex:extr-ball:one}
Show that any closed ball in $\spc{Y}$ is injective.
\end{subthm}

\begin{subthm}{ex:extr-ball:many}
Show that the intersection of an arbitrary collection of closed balls in $\spc{Y}$ is injective.
\end{subthm}

\end{thm}

\begin{thm}{Advanced exercise}\label{ex:extr-fixed}
Let $\spc{Y}$ be a bounded injective space.
Show that any short map $s\:\spc{Y}\to\spc{Y}$ has a fixed point. 
\end{thm}


\section{Admissible and extremal functions}

Let $\spc{X}$ be a metric space.
A function $r\:\spc{X}\to(-\infty,\infty]$ is called \label{page:admissible function}\index{admissible function}\emph{admissible} if the following inequality
\[r(x)+r(y)\ge \dist{x}{y}{\spc{X}}\eqlbl{eq:admissible}\]
holds for any $x,y\in \spc{X}$.

\begin{thm}{Observation}\label{obs:admissible}

\begin{subthm}{obs:admissible:nonnegative}
Any admissible function is nonnegative.
\end{subthm}

\begin{subthm}{obs:admissible:balls}
If $\spc{X}$ is a geodesic space, then a function $r\:\spc{X}\to\RR$ is admissible if and only if 
\[\cBall[x,r(x)]\cap\cBall[y,r(y)]\ne \emptyset\]
for any $x,y\in \spc{X}$.
\end{subthm}
 
\end{thm}

\parit{Proof; \ref{SHORT.obs:admissible:nonnegative}.} Apply \ref{eq:admissible} for $x=y$.

\parit{\ref{SHORT.obs:admissible:balls}.} Apply the triangle inequality and the existence of a geodesic $[xy]$.
\qeds

A minimal admissible function will be called \label{page:extremal function}\index{extremal function}\emph{extremal}.
More precisely, an admissible function $r\:\spc{X}\to\RR$ is extremal 
if for any admissible function $s\:\spc{X}\to\RR$ we have
\[s\le r\quad\Longrightarrow\quad s=r.\]

Applying Zorn's lemma, we get the following.

\begin{thm}{Observation}\label{obs:extremal:below}
For any admissible function $s$ there is an extremal function $r$ such that $r\le s$.
\end{thm}

\begin{thm}{Lemma}\label{lem:+-c}
Let $r$ be an extremal function and $s$ an admissible function on a metric space $\spc{X}$.
Suppose that $r\ge s-c$ for some constant~$c$.
Then $r\le s+c$; in particular, $c\ge 0$.
\end{thm}

\parit{Proof.}
Note that if $c<0$, then $r>s$.
The latter is impossible since $r$ is extremal and $s$ is admissible.

Observe that the function $\bar r=\min\{\,r,s+c\,\}$ is admissible.
Indeed, choose $x,y\in \spc{X}$.
If $\bar r(x)=r(x)$ and $\bar r(y)=r(y)$, then 
\[\bar r(x)+\bar r(y)=r(x)+ r(y)\ge \dist{x}{y}{}.\]
Further, if $\bar r(x)=s(x)+c$, then 
\begin{align*}
\bar r(x)+\bar r(y)&\ge [s(x)+c]+ [s(y)-c]= 
\\
&=s(x)+s(y) \ge 
\\
&\ge\dist{x}{y}{}.
\end{align*}

Since $r$ is extremal, we have $r=\bar r$;
that is, $r\le s+c$.
\qeds

\begin{thm}{Observations}\label{obs:extremal}
Let $\spc{X}$ be a metric space.

\begin{subthm}{obs:extremal:distfun}
For any point $p\in\spc{X}$ the distance function $r\z=\distfun_p$ is extremal.
\end{subthm}

\begin{subthm}{lem:extremal-lipschitz}
Any extremal function $r$ on $\spc{X}$ is \index{1-Lipschitz function}\emph{1-Lipschitz};
that is,
\[|r(p)-r(q)|\le \dist{p}{q}{}\]
for any $p,q\in\spc{X}$.
In other words, any extremal function is an extension function [see \ref{sec:Extension property}].
\end{subthm}

\begin{subthm}{lem:opposite}
An admissible function $r$ on $\spc{X}$ is extremal if and only if
for any point $p\in\spc{X}$ and any $\delta>0$, there is a point $q\in \spc{X}$
such that 
\[r(p)+r(q)<\dist{p}{q}{\spc{X}}+\delta.\]
\end{subthm}

\begin{subthm}{lem:opposite-compact}
Suppose $\spc{X}$ is compact.
Then an admissible function $r$ on $\spc{X}$ is extremal if and only if
for any point $p\in\spc{X}$ there is a point $q\in \spc{X}$
such that 
\[r(p)+r(q)=\dist{p}{q}{\spc{X}}.\]
\end{subthm}

\end{thm}

\parit{Proof; \ref{SHORT.obs:extremal:distfun}.}
By the triangle inequality, \ref{eq:admissible} holds;
that is, $r=\distfun_p$ is an admissible function.

Further, if $s\le r$ is another admissible function, then $s(p)=0$ and \ref{eq:admissible} implies that $s(x)\z\ge\dist{p}{x}{}$.
Whence $s=r$.

\parit{\ref{SHORT.lem:extremal-lipschitz}.}
By \ref{SHORT.obs:extremal:distfun}, $\distfun_p$ is admissible.
Since $r$ is admissible, we have that
\[r\ge \distfun_p-r(p).\]
Since $r$ is extremal, \ref{lem:+-c} implies that
\[r\le \distfun_p+r(p),\]
or, equivalently,
\[r(q)-r(p)\le \dist{p}{q}{}\]
for any $p,q\in\spc{X}$.
Whence the statement follows.

\parit{\ref{SHORT.lem:opposite}.}
Assume $r$ is extremal.
Arguing by contradiction, assume there is $\delta>0$ such that
\[r(q)\ge \distfun_p(q)-r(p)+\delta\]
for any $q$.
By \ref{SHORT.obs:extremal:distfun}, $\distfun_p$ is extremal; in particular, admissible.
Therefore \ref{lem:+-c} implies that
\[r(q)\le \distfun_p(q)+r(p)-\delta\]
for any $q$.
Taking $q=p$, we get $r(p)\le r(p)-\delta$, a contradiction.

Now suppose $r$ is not extremal; that is, there is an admissible function $s\le r$ such that $r(p)-s(p)=\delta>0$ for some $p$.
Then, for any $q$, we have
\[r(p)+r(q)\ge s(p)+s(q)+\delta\ge \dist{p}{q}{\spc{X}}+\delta\]
--- a contradiction.

\parit{\ref{SHORT.lem:opposite-compact}.}
The if part follows from \ref{SHORT.lem:opposite}.

Denote by $q_n$ the point provided by \ref{SHORT.lem:opposite} for $\delta=\tfrac1n$.
Let $q$ be a partial limit of $q_n$. 
Then 
\[r(p)+r(q)\le\dist{p}{q}{\spc{X}}.\]
Since $r$ is admissible, the opposite inequality holds;
whence the only-if part follows.
\qeds

\begin{thm}{Exercise}\label{ex:circle}
Consider the unit circle 
\[\mathbb{S}^1=\set{(x,y)}{x^2+y^2=1}\]
in the plane with induced length metric.
Show that $r\:\mathbb{S}^1\to\RR$ is extremal if and only if it is 1-Lipschitz and 
\[r(p)+r(-p)=\pi\] for any $p\in\mathbb{S}^1$.
\end{thm}

\begin{thm}{Exercise}\label{ex:retraction}
Given a real-valued function $s$ on a metric space $\spc{X}$,
consider the function
\[s^*(x)=\sup\set{\dist{x}{y}{\spc{X}}-s(y)}{y\in \spc{X}}\]
Show that the function $\tfrac12\cdot(s+s^*)$ is admissible for any $s$.
\end{thm}

\section{Equivalent conditions}

\begin{thm}{Theorem}\label{thm:injective=hyperconvex}
For any metric space $\spc{Y}$ the following conditions are equivalent:

\begin{subthm}{thm:injective=hyperconvex:injective}
$\spc{Y}$ is injective
\end{subthm}


\begin{subthm}{thm:injective=hyperconvex:extremal}
If $r\:\spc{Y}\to\RR$ is an extremal function, then there is a point $p\in \spc{Y}$ such that 
\[\dist{p}{x}{}= r(x)\]
for any $x\in \spc{Y}$.
\end{subthm}

\begin{subthm}{thm:injective=hyperconvex:balls}
$\spc{Y}$ is \index{hyperconvex space}\emph{hyperconvex};
that is, if $\set{\cBall[x_\alpha,r_\alpha]}{\alpha\in\IndexSet}$ is a family of closed balls in $\spc{Y}$ such that 
 \[r_\alpha+r_\beta\ge \dist{x_\alpha}{x_\beta}{}\]
 for any $\alpha,\beta\in \IndexSet$, then all the balls in the family $\{\cBall[x_\alpha,r_\alpha]\}_{\alpha\in\IndexSet}$ have a common point.
\end{subthm}

\end{thm}

\parit{Proof.} We will prove implications 
\ref{SHORT.thm:injective=hyperconvex:injective}$\Rightarrow$\ref{SHORT.thm:injective=hyperconvex:extremal}$\Rightarrow$\ref{SHORT.thm:injective=hyperconvex:balls}$\Rightarrow$\ref{SHORT.thm:injective=hyperconvex:injective}.

\parit{\ref{SHORT.thm:injective=hyperconvex:injective}$\Rightarrow$\ref{SHORT.thm:injective=hyperconvex:extremal}.}
By \ref{lem:extremal-lipschitz}, $r$ is an extension function.
Applying the definition of injective space to a one-point extension of $\spc{Y}$, we get a point $p\in \spc{Y}$ such that 
\[\dist{p}{x}{}=\distfun_p(x)\le r(x)\]
for any $x\in \spc{Y}$.
By \ref{obs:extremal:distfun}, the distance function $\distfun_p$ is extremal.
Since  $r$ is extremal, we get $\distfun_p= r$.


\parit{\ref{SHORT.thm:injective=hyperconvex:extremal}$\Rightarrow$\ref{SHORT.thm:injective=hyperconvex:balls}.}
By \ref{obs:admissible:balls}, part \ref{SHORT.thm:injective=hyperconvex:balls} is equivalent to the following statement:
\begin{itemize}
 \item If $r\:\spc{Y}\to\RR$ is an admissible function, then there is a point $p\in \spc{Y}$ such that 
\[\dist{p}{x}{}\le r(x)\eqlbl{eq:|p-x|=<r(x)}\]
for any $x\in \spc{Y}$.
\end{itemize}
Indeed, set $r(x)\df\inf\set{r_\alpha}{x_\alpha=x}$.
(If $x_\alpha\ne x$ for any $\alpha$, then $r(x)=\infty$.)
The condition in \ref{SHORT.thm:injective=hyperconvex:balls} implies that $r$ is admissible.
It remains to observe that $p\in \cBall[x_\alpha,r_\alpha]$ for every $\alpha$ if and only if \ref{eq:|p-x|=<r(x)} holds.

By \ref{obs:extremal:below}, for any admissible function $r$ there is an extremal function $\bar r\le r$;
hence \ref{SHORT.thm:injective=hyperconvex:extremal}$\Rightarrow$\ref{SHORT.thm:injective=hyperconvex:balls}.

\parit{\ref{SHORT.thm:injective=hyperconvex:balls}$\Rightarrow$\ref{SHORT.thm:injective=hyperconvex:injective}.}
Arguing by contradiction, suppose $\spc{Y}$ is not injective;
that is, there is a metric space $\spc{X}$ with a subset $\spc{A}$
such that a short map $f\:\spc{A}\to \spc{Y}$ cannot be extended to a short map $F\:\spc{X}\to \spc{Y}$.
By Zorn's lemma, we may assume that $\spc{A}$ is a maximal subset; that is, the domain of $f$ cannot be enlarged by a single point.%
\footnote{In this case, $\spc{A}$ must be closed, but we will not use it.}

Fix a point $p$ in the complement $\spc{X}\setminus \spc{A}$.
To extend $f$ to $p$, we need to choose $f(p)$ in the intersection of the balls 
$\cBall[f(x),r(x)]$, where $r(x)=\dist{p}{x}{}$.
Therefore, this intersection for all $x\in \spc{A}$ has to be empty.

Since $f$ is short, we have that 
\begin{align*}
r(x)+r(y)&\ge \dist{x}{y}{\spc{X}}\ge
\\
&\ge \dist{f(x)}{f(y)}{\spc{Y}}.
\end{align*}
By \ref{SHORT.thm:injective=hyperconvex:balls} the balls 
$\cBall[f(x),r(x)]$ have a common point --- a contradiction. 
\qeds

\begin{thm}{Exercise}\label{ex:one-point-gluing}
Suppose a length space $\spc{W}$ has two subspaces $\spc{X}$ and $\spc{Y}$ such that $\spc{X}\cup\spc{Y}=\spc{W}$ and $\spc{X}\cap\spc{Y}$ is a one-point set.
Assume $\spc{X}$ and $\spc{Y}$ are injective.
Show that  $\spc{W}$ is injective
\end{thm}

\begin{thm}{Exercise}\label{ex:Rm-ell-infty}
Show that an $m$-dimensional normed space is injective if and only if it is isometric to $\RR^m$ with $\ell^\infty$-norm; that is,
\[|(x_1,\dots,x_m)|=\max_i\{\,|x_i|\,\}.\]
\end{thm}

A metric space $\spc{Y}$ is called \index{finitely hyperconvex}\emph{finitely hyperconvex} or \index{countably hyperconvex}\emph{countably hyperconvex} if the condition in \ref{thm:injective=hyperconvex:balls} holds only for any finite or respectively countable family of balls.

\begin{thm}{Exercise}\label{ex:compact-hyperconvex}
Show that any proper finitely hyperconvex metric space is hyperconvex.
\end{thm}


\begin{thm}{Exercise}\label{ex:urysohn-hyperconvex}
Show that the $d$-Urysohn space is finitely hyperconvex, but not countably hyperconvex.
Conclude that the $d$-Urysohn space is not injective.

Try to do the same for the Urysohn space.
\end{thm}

\begin{thm}{Exercise}\label{ex:almost-hyperconvex}
Let $\spc{Y}$ be a complete metric space.
Suppose $\spc{Y}$ is \index{almost hyperconvex}\emph{almost hyperconvex},
that is, for any $\eps>0$ any family of closed balls $\set{\cBall[x_\alpha,r_\alpha+\eps]}{\alpha\in\IndexSet}$ has a common point if 
$r_\alpha+r_\beta\ge \dist{x_\alpha}{x_\beta}{}$ for all $\alpha,\beta\in \IndexSet$.
Show that $\spc{Y}$ is hyperconvex.
\end{thm}


\section{Space of extremal functions}
\label{sec:extremal-functions}

Let $\spc{X}$ be a metric space.
Consider the space $\Inj \spc{X}$ of extremal functions on $\spc{X}$ equipped with sup-norm; \label{page:InjX}
that is,
\[\dist{f}{g}{\Inj \spc{X}}\df\sup\set{|f(x)-g(x)|}{x\in \spc{X}}.\]

Recall that by \ref{obs:extremal:distfun}, any distance function is extremal.
It follows that the map $x\mapsto \distfun_x$ produces a distance-preserving embedding $\spc{X}\hookrightarrow\Inj \spc{X}$.
So we can (and will) treat $\spc{X}$ as a subspace of $\Inj \spc{X}$,
or, equivalently, $\Inj \spc{X}$ as an extension of $\spc{X}$.
In particular, from now on, a point $x\in\spc{X}$ can refer to the function $\distfun_x\:\spc{X}\to\RR$ and the other way around.

Since any extremal function is 1-Lipschitz, for any $f\in \Inj \spc{X}$ and $p\in \spc{X}$, we have that
$f(x)\le f(p)+\distfun_p(x)$.
By \ref{lem:+-c}, we also get $f(x)\ge -f(p)+\distfun_p(x)$.
Therefore
\[
\begin{aligned}
\dist{f}{p}{\Inj \spc{X}}&=\sup\set{|f(x)-\distfun_p(x)|}{x\in \spc{X}}=
\\
&=f(p).
\end{aligned}
\eqlbl{eq:f(p)=|f-p|}
\]
In particular, the statement in \ref{lem:opposite} can be written as 
\[\dist{f}{p}{\Inj\spc{X}}+\dist{f}{q}{\Inj\spc{X}}<\dist{p}{q}{\Inj\spc{X}}+\delta.\]

\begin{thm}{Exercise}\label{ex:Inj(compact)}
Show that $\Inj\spc{X}$ is compact if and only if so is $\spc{X}$.
\end{thm}

\begin{thm}{Exercise}\label{ex:tripod+square}
Describe the set of all extremal functions on a metric space $\spc{X}$ and the metric space $\Inj \spc{X}$ in each of the following cases:

\begin{subthm}{ex:tripod+square:2}
$\spc{X}$ is a metric space with exactly two points $v,w$ on distance 1 from each other.
\end{subthm}


\begin{subthm}{ex:tripod+square:tripod} 
$\spc{X}$ is a metric space with exactly three points $a,b,c$ such that 
\[\dist{a}{b}{\spc{X}}=\dist{b}{c}{\spc{X}}=\dist{c}{a}{\spc{X}}=1.\]
\end{subthm}

\begin{subthm}{ex:tripod+square:square}
$\spc{X}$ is  a metric space with exactly four points $p,q,x,y$ such that 
\[\dist{p}{x}{\spc{X}}=\dist{p}{y}{\spc{X}}=\dist{q}{x}{\spc{X}}=\dist{q}{y}{\spc{X}}=1\]
and
\[\dist{p}{q}{\spc{X}}=\dist{x}{y}{\spc{X}}=2.\]
\end{subthm}

\end{thm}

\begin{thm}{Exercise}\label{ex:kur-inj}
Assume $\spc{X}$ is a compact metric space.
Recall that the map $x\mapsto \distfun_x$ gives an isometric embedding $\spc{X}\hookrightarrow\ell^\infty(\spc{X})$; so we can think that $\spc{X}$ is a subset of $\ell^\infty(\spc{X})$.

Given two points $x,y\in \spc{X}$, denote by $G_{x,y}$ the union of all geodesics from $x$ to $y$ in $\ell^\infty(\spc{X})$.
Show that $\Inj\spc{X}$ is isometric to
\[G=\bigcap_{x\in \spc{X}}\left(\bigcup_{y\in \spc{X}}G_{x,y}\right).\]

\end{thm}


\begin{thm}{Proposition}\label{prop:InjX-is-injective}
$\Inj\spc{X}$ is injective for any metric space $\spc{X}$. 
\end{thm}

\begin{thm}{Lemma}\label{lem:r|X-extremal}
Let $\spc{X}$ be a metric space.
Then 
\[\sigma\in \Inj(\Inj \spc{X})
\quad\Longrightarrow\quad
\sigma|_\spc{X}\in \Inj \spc{X}.\]
\end{thm}

In other words, if $\sigma$ is an extremal function on $\Inj \spc{X}$,
then the restriction of $\sigma$ to $\spc{X}$ is an extremal function on $\spc{X}$.

\parit{Proof.}
Arguing by contradiction, suppose that there is an admissible function $s\:\spc{X}\to \RR$ such that $s(x)\le \sigma(x)$ for any $x\in\spc{X}$ and $s(p)\z< \sigma(p)$ for some point $p\in\spc{X}$.
Consider another function $\bar \sigma\:\Inj \spc{X}\z\to\RR$ such that $\bar \sigma(f)\df \sigma(f)$ if $f\ne p$ and $\bar \sigma(p)\df s(p)$.

Let us show that $\bar \sigma$ is admissible; that is, 
\[\dist{f}{g}{\Inj \spc{X}}\le\bar \sigma(f)+\bar \sigma(g)
\eqlbl{r-admissible}\]
for any $f,g\in \Inj \spc{X}$.

Since $\sigma$ is admissible and $\bar \sigma= \sigma$ on $(\Inj \spc{X})\setminus \{p\}$, it is sufficient to prove \ref{r-admissible} assuming $f\ne g=p$.
By \ref{eq:f(p)=|f-p|}, we have $\dist{f}{p}{\Inj \spc{X}}=f(p)$.
Therefore, \ref{r-admissible} boils down to the following inequality
\[\sigma(f)+s(p)\ge f(p).\eqlbl{eq:r(f)+s(p)>=f(p)}\]
for any $f\in\Inj \spc{X}$.

Fix small $\delta>0$. 
Let $q\in\spc{X}$ be the point provided by \ref{lem:opposite}.
Then
\begin{align*}
\sigma(f)+s(p)&\ge [\sigma(f)-\sigma(q)]+[\sigma(q)+s(p)]\ge
\intertext{since $\sigma$ is 1-Lipschitz, and $\sigma(q)\ge s(q)$, we can continue}
&\ge -\dist{q}{f}{\Inj \spc{X}}+[s(q)+s(p)]\ge
\intertext{by \ref{eq:f(p)=|f-p|} and since $s$ is admissible}
&\ge -f(q)+\dist{p}{q}{}>
\intertext{and by \ref{lem:opposite}}
&> f(p)-\delta.
\end{align*}
Since $\delta>0$ is arbitrary, \ref{eq:r(f)+s(p)>=f(p)} and \ref{r-admissible} follow.

Summarizing: the function $\bar \sigma$ is admissible, $\bar \sigma\le \sigma$ and $\bar \sigma(p)<\sigma(p)$;
that is, $\sigma$ is not extremal --- a contradiction.
\qeds

\parit{Proof of \ref{prop:InjX-is-injective}.}
Choose a function $\sigma\in\Inj(\Inj\spc{X})$.
By \ref{lem:r|X-extremal}, $s\z\df \sigma|_{\spc{X}}\in \Inj\spc{X}$;
that is, $s$ is extremal.
By \ref{thm:injective=hyperconvex:extremal},
it is sufficient to show that  
\[\sigma(f)\ge\dist{s}{f}{\Inj\spc{X}}
\eqlbl{eq:r(f)>=| r-f|}\]
for any $f\in\Inj\spc{X}$.

Since $\sigma$ is $1$-Lipschitz (\ref{lem:extremal-lipschitz}) we have that
\[
s(x)-f(x)=\sigma(x)-\dist{f}{x}{\Inj \spc{X}}\le \sigma(f).
\]
for any $x\in\spc{X}$.
By \ref{lem:+-c},
$
s(x)-f(x)\ge -\sigma(f)
$
for any $x\in\spc{X}$.
Whence \ref{eq:r(f)>=| r-f|} follows.
\qeds

\begin{thm}{Exercise}\label{ex:4-on-a-line}
Let $\spc{X}$ be a compact metric space.
Show that for any two points $f,g\in\Inj \spc{X}$ lie on a geodesic $[pq]$ with $p,q\in \spc{X}$.
\end{thm}

A metric space $\spc{X}$ is called \index{$\delta$-hyperbolic}\emph{$\delta$-hyperbolic} if 
\[\dist{p}{q}{}+\dist{x}{y}{}\le
\max\{\,\dist{p}{x}{}+\dist{q}{y}{},
\,
\dist{p}{y}{}+\dist{q}{x}{}\,\}+2\cdot\delta\]
for any $p,q,x,y\in \spc{X}$.

\begin{thm}{Advanced exercise}\label{ex:delta-hyp}
Show that $\Inj \spc{X}$ is $\delta$-hyperbolic if and only if so is $\spc{X}$.
\end{thm}


\section{Injective envelope}

An extension $\spc{E}$ of a metric space $\spc{X}$ will be called its \index{injective envelope}\emph{injective envelope} if $\spc{E}$ is an injective space, and there is no proper injective subspace of $\spc{E}$ that contains $\spc{X}$.

Two injective envelopes $e\:\spc{X}\hookrightarrow \spc{E}$ and $f\:\spc{X}\hookrightarrow \spc{F}$ are called  equivalent if there is an isometry $\iota\: \spc{E}\to\spc{F}$ such that $f=\iota\circ e$.

\begin{thm}{Theorem}\label{thm:inj-envelope}
For any metric space $\spc{X}$, its extension $\Inj\spc{X}$ is an injective envelope.

Moreover, any other injective envelope of $\spc{X}$ is equivalent to $\Inj\spc{X}$.
\end{thm}

\parit{Proof.} 
Suppose $S\subset \Inj\spc{X}$ is an injective subspace containing $\spc{X}$.
Since $S$ is injective, there is a short map $w\:\Inj\spc{X}\to S$ that fixes all points in $\spc{X}$.

Suppose that $w\:f\mapsto f'$; observe that $f(x)\ge f'(x)$ for any $x\in \spc{X}$.
Since $f$ is extremal, $f=f'$;
that is, $w$ is the identity map, and therefore $S=\Inj\spc{X}$.

Assume we have another injective envelope $e\:\spc{X}\hookrightarrow \spc{E}$.
Then there are short maps $v\:\spc{E}\to \Inj\spc{X}$ and $w\:\Inj\spc{X}\to \spc{E}$ such that $x=v\circ e(x)$ and $e(x)=w(x)$ for any $x\in\spc{X}$.
From above, the composition $v\circ w$ is the identity on $\Inj\spc{X}$.
In particular, $w$ is distance-preserving.

The composition $w\circ v\:\spc{E}\to \spc{E}$ is a short map that fixes points in $e(\spc{X})$.
Since $e\:\spc{X}\hookrightarrow \spc{E}$ is an injective envelope, the composition $w\circ v$ and therefore $w$ are onto.
Whence $w$ is an isometry.
\qeds

\begin{thm}{Exercise}\label{ex:inj-envelope}
Suppose $e\:\spc{X}\hookrightarrow \spc{E}$ and $f\:\spc{X}\hookrightarrow \spc{F}$ are two injective envelopes of $\spc{X}$.
Show that there is a unique isometry $\iota\:\spc{E}\to \spc{F}$ such that $\iota\circ e=f$.
\end{thm}

\begin{thm}{Exercise}\label{ex:d-p-inclusion}
Suppose $\spc{X}$ is a subspace of a metric space $\spc{U}$.
Show that the inclusion $\spc{X}\hookrightarrow\spc{U}$ can be extended to a distance-preserving inclusion $\Inj\spc{X}\hookrightarrow\Inj\spc{U}$.
\end{thm}

\begin{thm}{Exercise}\label{ex:hemisphere-inj}
Consider the hemisphere 
\begin{align*}
\mathbb{S}^2_+&=\set{(x,y,z)\in\RR^3}{x^2+y^2+z^2=1,\quad z\ge0}
\intertext{and its boundary}
\mathbb{S}^1&=\set{(x,y,z)\in\RR^3}{x^2+y^2+z^2=1,\quad z=0};
\end{align*}
 both with induced length metrics.
 
Show that there is unique isometric embedding $\iota\:\mathbb{S}^2_+\hookrightarrow\Inj\mathbb{S}^1$ such that $\iota(u)=u$ for any $u\in \mathbb{S}^1$.
\end{thm}


\section{Remarks}

Injective spaces were introduced by Nachman Aronszajn and Prom Panitchpakdi \cite{aronszajn-panitchpakdi}.
The injective envelope was introduced by John Isbell \cite{isbell}; it is also known as \index{tight span}\emph{tight span} and \index{hyperconvex hull}\emph{hyperconvex hull}.

It was observed by John Isbell \cite{isbell2} that \textit{if $\spc{X}$ is a Banach space, then its injective hull $\Inj\spc{X}$ has a natural structure of Banach space} (which is unique by the Mazur--Ulam theorem).
Moreover, $\spc{X}$ is a linear subspace of $\Inj\spc{X}$.
 
Let us mention that a metric space $\spc{X}$ is called \index{convex space}\emph{convex} if for any pair of points $x_1,x_2\in \spc{X}$ and any $r_1,r_2\ge 0$ we have 
\[r_1+r_2\ge \dist{x_1}{x_2}{\spc{X}}\qquad\Longrightarrow\qquad\cBall[x,r_1]_\spc{X}\cap \cBall[y,r_2]_\spc{X}\ne\emptyset;\]
in other words, a pair of balls intersect if the triangle inequality does not forbid it.
Clearly, hyperconvexity (\ref{thm:injective=hyperconvex:balls}) is stronger than convexity.
Note that \textit{any geodesic space is convex}.
The converse does not hold in general, but by Menger's lemma (\ref{lem:mid>geod:geod}) \textit{any complete convex space is geodesic}.

More generally, a metric space $\spc{X}$ is called \index{$n$-hyperconvex space}\emph{$n$-hyperconvex} if the condition in \ref{thm:injective=hyperconvex:balls} holds only for families with at most $n$ balls; so \textit{convex means $2$-hyperconvex}.

The following striking result was proved by Benjamin Miesch and Maël Pavón \cite{miesch-pavon2016}.

\begin{thm}{Theorem}
Any complete $4$-hyperconvex space is finitely hyperconvex.
\end{thm}

So, by \ref{ex:compact-hyperconvex}, it follows that \textit{any proper $4$-hyperconvex space is hyperconvex}.

\begin{thm}{Exercise}\label{ex:3-4-hypreconvex}
Show that $\ell^1$ is $3$- but not $4$-hyperconvex.
\end{thm}
 

Recall that if the following inequality
\[\dist{x}{z}{\spc{X}}
\le
\max\{\,\dist{x}{y}{\spc{X}},\dist{y}{z}{\spc{X}}\,\}\]
holds for any three points $x,y,z$ in a metric space $\spc{X}$,
then $\spc{X}$ is called an \index{ultrametric space}\emph{ultrametric space}.
In some sense, ultrametric spaces are dual to injective spaces.

\begin{thm}{Exercise}\label{ex:ultrametric}
Suppose that a metric space $\spc{X}$ satisfies the following property:
For any subspace $\spc{A}$ in $\spc{X}$ and any other metric space $\spc{Y}$, any short map $f\:\spc{A}\to \spc{Y}$ can be extended to a short map $F\:\spc{X}\to \spc{Y}$.

Show that $\spc{X}$ is an ultrametric space.
\end{thm}

A subspace $\spc{S}$ of a metric space $\spc{X}$ is called its \index{short retract}\emph{short retract} if there is a short map $\spc{X}\to \spc{S}$ that is the identity on $\spc{S}$.

\begin{thm}{Exercise}\label{ex:ultrametric-converse}
Show that any compact subspace $\spc{K}$ of an ultrametric space $\spc{X}$ is its short retract.

Construct an example of a complete ultrametric space $\spc{X}$ with a closed subspace $\spc{Q}$ that is not its short retract.
\end{thm}

The following exercise gives a sufficient condition for the existence of a short extension.

\begin{thm}{Exercise}\label{ex:petrunin-stadler}
Let $f\:A\z\to \spc{K}$ be a short map from a subset $A$ in a metric space $\spc{X}$ to compact metric space $\spc{K}$.
Assume that for any finite set $F\subset \spc{X}$ there is a short map $F\to \spc{K}$ that agrees with $f$ on $F\cap A$.
Show that there is a short map $\spc{X}\to \spc{K}$ that agrees with $f$ on $A$.
\end{thm}
