\chapter{Volume bounds filling radius}

This chapter 
is devoted to a proof of \ref{thm:FillRad<vol};
that is, we will show that \emph{Riemannian manifolds with small volume have small filling radius}.
This theorem was proved originally by Mikhael Gromov \cite{gromov-1983}.
We follow closely a simplified proof given by Alexander Nabutovsky, which is based on a sequence of other simplifications and improvements; see \cite{nabutovsky} and the references there in.

\section{Nerves and partition of unity}

Let $\{V_1,\dots,V_k\}$ be a finite open cover of a compact metric space $\spc{X}$.
Consider the abstract simplicial complex $\spc{N}$, with one vertex $v_i$ for each set $V_i$ such that a simplex with vertexes $v_{i_1},\dots, v_{i_k}$ is included in $\spc{N}$ if 
the intersection $V_{i_1}\cap\dots\cap V_{i_m}$ is nonempty.
We obtain a simplicial complex $\spc{N}$ called the \index{nerve}\emph{nerve of the covering $\{V_i\}$}.

Note that $\spc{N}$ is a finite simplicial complex and it has dimension at most $n$ if and only if the covering $\{V_1,\dots,V_k\}$ has multiplicity is at most $n+1$;
that is, at most $n+1$ different sets $V_{i_1},\dots, V_{i_{n+1}}$ have a nonempty intersection.
The nerve $\spc{N}$ is a subcomplex of a simplex with the vertixes $v_1,\dots,v_k\}$.

\begin{thm}{Proposition}\label{thm:part-unit}
 Let $\{V_1,\dots,V_k\}$ is a finite open covering of a compact metric space ${\spc{X}}$.
Then there are Lipschitz functions $\psi_i\:{\spc{X}}\to[0,1]$ such that
if $\psi_i(x)>0$ then $x\in V_i$ and
$$\sum_i\psi_i(x)=1$$
for any $x\in {\spc{X}}$.
\end{thm}

A collection of functions $\psi_i$ with above properies is called 
a \emph{partition of unity subordinate to the open cover}\index{partition of unity} $\{V_1,\dots,V_k\}$.

\parit{Proof.}
Consider the functions $\phi_i\:{\spc{X}}\to\RR$ defined as
$$\phi_i(x)=\distfun_{{\spc{X}}\backslash V_i} x.$$
Note $\phi_i$ is $1$-Lipschitz
for any $i$
and $\phi_i(x)>0$ if and only if $x\in V_i$.
In particular, 
$$\sum_i\phi_i(x)>0\ \ \text{for any}\ \ x\in {\spc{X}}.$$

Set 
$$\psi_k(x)=\frac{\phi_k(x)}{\sum_i\phi_i(x)}.$$
It remains to note that by construction the functions $\psi_i$ meet the conditions in the proposition.
\qedsf


Note that in the above proof for any point $x\in {\spc{X}}$,
the set
$$\set{v_i}{\psi_i(x)>0}$$
describe vertexes of a simplices in the nerve.
Therefore 
$$\bm{\psi}\:x\mapsto \psi_1(x)\cdot v_1+\psi_2(x)\cdot v_2+\dots+\psi_k(x)\cdot v_n.$$
can be thought of as a Lipschitz map from ${\spc{X}}$ to the nerve $\spc{N}$ of $\{V_i\}$;
where the point $x$ is mapped to the point with barycentric coordinates $\psi_i(x)$.
In other words we proved the following:

\begin{thm}{Proposition}\label{prop:space->nerve}
Let $\spc{N}$ be a nerve of an open covering $\{V_1,\dots,V_k\}$ of a compact metric space $\spc{X}$.
Denote by $v_i$ the vertex of $\spc{N}$ that corresponds to $V_i$.

Then there is a Lipschitz map from $\bm{\psi}\:\spc{X}\to\spc{N}$ such that $\bm{\psi}(V_i)\z\subset\Star_{v_i}$ for every $i$.
\end{thm}


\section{Width}

Suppose $A$ is a subset of a metric space $\spc{X}$.
The radius of $A$ (briefly $\rad A$) is defined as the least upper bound on the values $R>0$ such that $\oBall(x,R)\supset A$ for some $x\in \spc{X}$.

\begin{thm}{Definition}\label{def:width}
Let $\spc{X}$ be a metric space.
The $n$-width of $\spc{X}$ (briefly $\width_n\spc{X}$) is defined as least upper bound on values $R>0$ such that $\spc{X}$ admits a finite open covering $\{V_i\}$ with multiplicity at most $n+1$ and $\rad V_i< R$ for any $i$.
\end{thm}

\parit{Remarks.}
\begin{itemize}
\item Observe that if $\spc{X}$ is connected, then 
\[\width_0\spc{X}=\rad\spc{X}.\]
\item 
Usually width is defined using diameter instead of radius, but the result differ at most twice.
Namely if $r$ is the radius width and $d$ --- the diameter width of the same dimension, then 
$r\le d\le 2\cdot r$.

\item The definition of width reminds the definition of Lebesgue covering dimension.
In fact one says that a space has \emph{macroscopic dimesion} $\le n$ on the space $R$ if it admits an open cover as in the definiton.
\end{itemize}

\begin{thm}{Exercise}
Suppose $\spc{X}$ be a simply connected metric space such that any closed curve $\gamma$ in $\spc{X}$ can be contracted in its $R$-neighborhood.
Show that $\spc{X}$ is has macroscopic dimension at most 1 on scale $100\cdot R$.

Proove a quasiconverse; that is, if a simply connected metric space $\spc{X}$ has macroscopic dimension at most 1 on scale $R$, then any closed curve $\gamma$ in $\spc{X}$ can be contracted in its $100\cdot R$-neighborhood.
\end{thm}


The following proposition provides an equivalent definition.

\begin{thm}{Proposition}\label{prop:width=suprad(inv)}
Suppose $\spc{X}$ is a compact metric space.
Then $\width_n\spc{X}<R$ if and only if there is a finite $n$-dimensional somplicial complex $\spc{S}$ and a continuous map $\bm{\psi}\:\spc{X}\to \spc{N}$
such that $\rad[\bm{\psi}^{-1}(s)]\z<R$
for any $s\in \spc{N}$.
\end{thm}

\parit{Proof; ``only if'' part.}
Suppose $\width_n\spc{X}<R$.
Consider a covering $\{V_1,\dots,V_k\}$ of $\spc{X}$ guaranteed by the definition of width.
Let $\spc{N}$ be its nerve and $\bm{\psi}\:\spc{X}\to \spc{N}$ be the map provided by \ref{prop:space->nerve}.

Note that if $x\in \spc{N}$ lies in a symplex with a vertex $v_i$,
then $\bm{\psi}^{-1}(x)\subset V_i$;
in particulr $\bm{\psi}^{-1}(x)$ can be covered by a ball of radius $R$ in $\spc{X}$.

\parit{``If'' part.}
Choose $x\in \spc{N}$.
Since the inverse image $\bm{\psi}^{-1}(x)$ is compact, $\bm{\psi}$ is continuous, and $\rad[\bm{\psi}^{-1}(x)]<R$,
here is a neighborhood $U\ni x$ such that the  $\rad[\bm{\psi}^{-1}(U)]<R$.

It follows that there is a finite cover $\{U_i\}$ of $\spc{N}$ such that $\bm{\psi}^{-1}(U_i)\subset\spc{X}$ has radius smaller than $R$ for each $i$.
Since $\spc{N}$ has dimension $n$, we can inscribe%
\footnote{Recall that a covering $\{W_i\}$ is inscribed in the covering $\{U_i\}$ if for every $W_i$ is a subset of some $U_j$.} 
in $\{U_i\}$ an finite open cover $\{W_i\}$ with multiplicity at most $n+1$.
It remains to observe that $V_i=\bm{\psi}(W_i)$ defines a finite open cover of $\spc{X}$ with radius less than $R$ and multiplicity at most $n+1$. 
\qeds

Further we will apply the notion of width to compact Riemannian polyhedrons;
If $n$ is the dimension of a compact Riemannian polyhedron $\spc{P}$, then 
we suppose that
\[\width\spc{P}\df\width_{n-1}\spc{P}.\]

\begin{thm}{Exercise}
Show that for any closed Riemannian manifold $\spc{M}$ we have
\[\FillRad \spc{M}\le 100\cdot \width\spc{M};\]
try to show that in fact
\[\FillRad \spc{M}\le \width\spc{M}.\]

\end{thm}




\section{Volume bounds width}

A \emph{Riemannian polyhedron} is defined as a finite connected simplicial complex with a metric tensor on each simplex such that the restriction of the metric on each simplex to a subsymplex coinsides with the metric on the subsmplex.
The dimension of Riemannian polyhedron is defined as the largest dimension it its triangulation.
For Riemannian polhedron one can define length of curves and volume the same way as for Riemannian manifolds.

Let $\spc{P}$ be a Riemnnian polyhedron of dimension $n$.
Let us define volume profile of $\spc{P}$ as a function $\VolPro_{\spc{P}}\:\RR_+\to\RR_+$ defined by 
\[\VolPro_{\spc{P}}(r)\df \sup\set{\vol \oBall(p,r)}{p\in\spc{P}}.\]
Note that $\VolPro_{\spc{P}}$ is a nondecreasing function and $\VolPro_{\spc{P}}(r)\z\to\vol\spc{P}$ as $r\to\infty$.

\begin{thm}{Exercise}
Suppose $\spc{M}$ be a 1-dimensional Riemannian polhedron.
Suppose $\VolPro_{\spc{P}}(r_0)<r_0$ for some $r_0>0$.
Show that 
\[\diam \spc{P}<r_0.\]
Note tha it is equivalent to $\width \spc{P}<r_0$.
\end{thm}


An $(n-1)$-dimensional subpolyhedron $\spc{Q}\subset\spc{P}$ is called $R$-separating if each
connected component of its complement $\spc{P}\backslash \spc{Q}$ can be covered by a metric ball of radius $R$.

\begin{thm}{Lemma}
Let $\spc{P}$ be an $n$-dimensional Riemannian polyhedron.
Then given $R>0$ and $\eps>0$ there is a $R$-separating subpolyhedron $\spc{Q}\subset\spc{P}$ such that for any $r_0<r_1\le R$ we have
\[\VolPro_{\spc{Q}}(r_0)< \tfrac1{r_1-r_0}\cdot \VolPro_{\spc{P}}(r_1)+\eps.\]

\end{thm}

\begin{thm}{Lemma}
Let $\spc{Q}$ be a $R$-separating subpolyhedron in an $n$-dimensional Riemannian polyhedron $\spc{P}$.
Suppose $\width\spc{Q}\le R$.
Then $\width\spc{P}\le R$
\end{thm}

\parit{Proof.}
Start with an open covering $\{V_1,\dots,V_k\}$ of $\spc{Q}$ of multiplicity $\le n$ with radiuses of the sets in the intrinsic metric $\le R$.
Convert $\{V_1,\dots,V_k\}$ into an an open covering of
a small neighbourhood of $\spc{Q}$ in $\spc{P}$ without increasing the multiplicity.
Finally, add all the components of $\spc{P}\backslash \spc{Q}$ to the covering;
it increases the multiplicity by 1.
\qeds

The following technical statement will be used without a proof.

\begin{thm}{Claim}
Let $\spc{P}$ be a Reimannian polyhedron and $f\:\spc{P}\to \RR$ be a 1-Lipschitz function.
Then for any $\eps>0$ there is a  1-Lipschitz function $f_\eps\:\spc{P}\to \RR$ that is smooth on each simplex of the triangulation and $\eps$-close to $f$.
\end{thm}

\begin{thm}{Sard's theorem}
Let $\spc{P}$ be an $n$-dimensional Reimannian polyhedron and $f\:\spc{P}\to \RR$ be a function that is smooth on each simplex.
Then for almost all values $a$ each component of the inverse image $f^{-1}(a)$ is a equipped with the induced metric is a Reimannian polyhedron.
\end{thm}


\begin{thm}{Coarea inequality}
Let $\spc{P}$ be an $n$-dimensional Reimannian polyhedron and $f\:\spc{P}\to \RR$ be a 1-Lipschitz function that is smooth on each simplex.
Then 
\[\vol_n (f^{-1}[a,b]) \le \int_a^b\vol_{n-1}[f^{-1}(x)]\cdot dx.\]
\end{thm}

