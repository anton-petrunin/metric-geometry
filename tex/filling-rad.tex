\chapter{Volume bounds filling radius}

This chapter 
is devoted to a proof of \ref{thm:FillRad<vol};
that is, we will show that \emph{Riemannian manifolds with small volume have small filling radius}.
Note that once it is proved, \ref{thm:sys<FillRad} implies \ref{thm:sys(torus)}.
Moreover \ref{thm:sys<FillRad++} implies the following:

\begin{thm}{Therorem}\label{thm:sys(torus)}
Systolic intequality holds for any essential manifold. 
\end{thm}

This theorem was proved originally by Mikhael Gromov \cite{gromov-1983}.
We follow closely a simplified proof given by Alexander Nabutovsky, which is based on a sequence of other simplifications and improvements; see \cite{nabutovsky} and the references therein.

\section{Nerves and partition of unity}

Let $\{V_1,\dots,V_k\}$ be a finite open cover of a compact metric space $\spc{X}$.
Consider an abstract simplicial complex $\spc{N}$, with one vertex $v_i$ for each set $V_i$ such that a simplex with vertexes $v_{i_1},\dots, v_{i_m}$ is included in $\spc{N}$ if 
the intersection $V_{i_1}\cap\dots\cap V_{i_m}$ is nonempty.
We obtain a simplicial complex $\spc{N}$ called the \index{nerve}\emph{nerve of the covering $\{V_i\}$}.

Note that $\spc{N}$ is a finite simplicial complex and it has dimension at most $n$ if and only if the covering $\{V_1,\dots,V_k\}$ has multiplicity is at most $n+1$;
that is, at most $n+1$ different sets $V_{i_1},\dots, V_{i_{n+1}}$ have a nonempty intersection.
The nerve $\spc{N}$ is a subcomplex of a simplex with the vertixes $\{v_1,\dots,v_k\}$.

\begin{thm}{Proposition}\label{thm:part-unit}
 Let $\{V_1,\dots,V_k\}$ is a finite open covering of a compact metric space ${\spc{X}}$.
Then there are Lipschitz functions $\psi_i\:{\spc{X}}\to[0,1]$ such that
if $\psi_i(x)>0$ then $x\in V_i$ and
$$\sum_i\psi_i(x)=1$$
for any $x\in {\spc{X}}$.
\end{thm}

A collection of functions $\psi_i$ with above properies is called 
a \emph{partition of unity subordinate to the open cover}\index{partition of unity} $\{V_1,\dots,V_k\}$.

\parit{Proof.}
Consider the functions $\phi_i\:{\spc{X}}\to\RR$ defined as
$$\phi_i(x)=\distfun_{({\spc{X}}\backslash V_i)} x.$$
Note $\phi_i$ is $1$-Lipschitz
for any $i$
and $\phi_i(x)>0$ if and only if $x\in V_i$.
In particular, 
$$\sum_i\phi_i(x)>0\ \ \text{for any}\ \ x\in {\spc{X}}.$$

Set 
$$\psi_k(x)=\frac{\phi_k(x)}{\sum_i\phi_i(x)}.$$
It remains to note that by construction the functions $\psi_i$ meet the conditions in the proposition.
\qedsf


Note that in the above proof for any point $x\in {\spc{X}}$,
the set
$$\set{v_i}{\psi_i(x)>0}$$
describe vertexes of a simplices in the nerve.
Therefore 
$$\psi\:x\mapsto \psi_1(x)\cdot v_1+\psi_2(x)\cdot v_2+\dots+\psi_k(x)\cdot v_n.$$
can be thought of as a Lipschitz map from ${\spc{X}}$ to the nerve $\spc{N}$ of $\{V_i\}$;
where the point $x$ is mapped to the point with barycentric coordinates $\psi_i(x)$.
In other words we proved the following:

\begin{thm}{Proposition}\label{prop:space->nerve}
Let $\spc{N}$ be a nerve of an open covering $\{V_1,\z\dots,V_k\}$ of a compact metric space $\spc{X}$.
Denote by $v_i$ the vertex of $\spc{N}$ that corresponds to $V_i$.

Then there is a Lipschitz map from $\psi\:\spc{X}\to\spc{N}$ such that $\psi(V_i)\z\subset\Star_{v_i}$ for every $i$.
\end{thm}


\section{Width}

Suppose $A$ is a subset of a metric space $\spc{X}$.
The radius of $A$ (briefly $\rad A$) is defined as the least upper bound on the values $R>0$ such that $\oBall(x,R)\supset A$ for some $x\in \spc{X}$.

\begin{thm}{Definition}\label{def:width}
Let $\spc{X}$ be a metric space.
The $n$-th width of $\spc{X}$ (briefly $\width_n\spc{X}$) is defined as least upper bound on values $R>0$ such that $\spc{X}$ admits a finite open covering $\{V_i\}$ with multiplicity at most $n+1$ and $\rad V_i< R$ for each $i$.
\end{thm}

\parit{Remarks.}
\begin{itemize}
\item Observe that if $\spc{X}$ is connected, then 
\[\width_0\spc{X}=\rad\spc{X}.\]
\item 
Usually width is defined using diameter instead of radius, but the result differ at most twice.
Namely if $r$ is an $n$-th radius-width and $d$ --- $n$-th diameter-width of the same dimension, then 
$r\le d\le 2\cdot r$.

\item The definition of width reminds the definition of Lebesgue covering dimension.
In fact one says that a space has \emph{macroscopic dimesion} $\le n$ on the space $R$ if it admits an open cover as in the definiton.
\end{itemize}

\begin{thm}{Exercise}\label{ex:macrodimension}
Suppose $\spc{X}$ be a metric space such that any closed curve $\gamma$ in $\spc{X}$ can be contracted in its $R$-neighborhood.
Show that $\spc{X}$ is has macroscopic dimension at most 1 on scale $100\cdot R$.

What about quasiconverse? That is, suppose a simply connected metric space $\spc{X}$ has macroscopic dimension at most 1 on scale $R$, is it true that any closed curve $\gamma$ in $\spc{X}$ can be contracted in its $100\cdot R$-neighborhood?
\end{thm}


The following proposition provides an equivalent definition;
we will not use it, but it provides a good reason for the name width.

\begin{thm}{Proposition}\label{prop:width=suprad(inv)}
Suppose $\spc{X}$ is a compact metric space.
Then $\width_n\spc{X}<R$ if and only if there is a finite $n$-dimensional somplicial complex $\spc{S}$ and a continuous map $\psi\:\spc{X}\to \spc{N}$
such that $\rad[\psi^{-1}(s)]\z<R$
for any $s\in \spc{N}$.
\end{thm}

\parit{Proof; ``only if'' part.}
Suppose $\width_n\spc{X}<R$.
Consider a covering $\{V_1,\dots,V_k\}$ of $\spc{X}$ guaranteed by the definition of width.
Let $\spc{N}$ be its nerve and $\psi\:\spc{X}\to \spc{N}$ be the map provided by \ref{prop:space->nerve}.

Note that if $x\in \spc{N}$ lies in a symplex with a vertex $v_i$,
then $\psi^{-1}(x)\subset V_i$;
in particulr $\psi^{-1}(x)$ can be covered by a ball of radius $R$ in $\spc{X}$.

\parit{``If'' part.}
Choose $x\in \spc{N}$.
Since the inverse image $\psi^{-1}(x)$ is compact, $\psi$ is continuous, and $\rad[\psi^{-1}(x)]<R$,
here is a neighborhood $U\ni x$ such that the  $\rad[\psi^{-1}(U)]<R$.

It follows that there is a finite cover $\{U_i\}$ of $\spc{N}$ such that $\psi^{-1}(U_i)\subset\spc{X}$ has radius smaller than $R$ for each $i$.
Since $\spc{N}$ has dimension $n$, we can inscribe%
\footnote{Recall that a covering $\{W_i\}$ is inscribed in the covering $\{U_i\}$ if for every $W_i$ is a subset of some $U_j$.} 
in $\{U_i\}$ an finite open cover $\{W_i\}$ with multiplicity at most $n+1$.
It remains to observe that $V_i=\psi(W_i)$ defines a finite open cover of $\spc{X}$ with radius less than $R$ and multiplicity at most $n+1$. 
\qeds

Further we will apply the notion of width to compact Riemannian polyhedrons;
If $n$ is the dimension of a compact Riemannian polyhedron $\spc{P}$, then 
we suppose that
\[\width\spc{P}\df\width_{n-1}\spc{P}.\]

\begin{thm}{Exercise}\label{ex:FillRad<width}
Show that for any closed Riemannian manifold $\spc{M}$ we have
\[\FillRad \spc{M}\le 100\cdot \width\spc{M};\]
try to show that in fact
\[\FillRad \spc{M}\le \width\spc{M}.\]

\end{thm}




\section{Volume profile}

A \emph{Riemannian polyhedron} is defined as a finite connected simplicial complex with a metric tensor on each simplex such that the restriction of the metric on each simplex to a subsymplex coinsides with the metric on the subsmplex.
The dimension of Riemannian polyhedron is defined as the largest dimension it its triangulation.
For Riemannian polhedron one can define length of curves and volume the same way as for Riemannian manifolds.

Let $\spc{P}$ be a Riemnnian polyhedron of dimension $n$.
Let us define volume profile of $\spc{P}$ as a function $\VolPro_{\spc{P}}\:\RR_+\to\RR_+$ defined by 
\[\VolPro_{\spc{P}}(r)\df \sup\set{\vol \oBall(p,r)}{p\in\spc{P}}.\]
Note that $\VolPro_{\spc{P}}$ is a nondecreasing function and $\VolPro_{\spc{P}}(r)\z\to\vol\spc{P}$ as $r\to\infty$.

\begin{thm}{Theorem}\label{thm:width<volpro}
There is a constant $c_n>0$ such that the following holds true:

If $\spc{P}$ is an $n$-dimensional Reimannian polyhedron such that 
\[r> c_n\cdot \sqrt[n]{\VolPro_{\spc{P}}(r)}\] 
for some $r>0$, then 
\[\width\spc{P}\le  r.\]
\end{thm}

Since $\VolPro_{\spc{P}}(r)\le \vol\spc{P}$ for any $r$,
Theorem \ref{thm:width<volpro} implies the following:

\begin{thm}{Theorem}\label{thm:width<vol}
There is a constant $c_n>0$ such that 
\[\width\spc{P}\le c_n\cdot \sqrt[n]{\vol\spc{P}}\] 
for any  $n$-dimensional Reimannian polyhedron $\spc{P}$.
\end{thm}

Together with \ref{ex:FillRad<width}, the last theorem implies \ref{thm:FillRad<vol} which is the goal of this lecture.

\section{Proof}

In the proof of \ref{thm:width<volpro}, we will use the following three technical statements,
the proofs are omitted, but they are not hard. 

\begin{thm}{Smoothing procedure}
Let $\spc{P}$ be a Reimannian polyhedron and $f\:\spc{P}\to \RR$ be a 1-Lipschitz function.
Then for any $\delta>0$ there is a  1-Lipschitz function $\tilde f\:\spc{P}\to \RR$ that is smooth on each simplex of the triangulation and $\delta$-close to $f$.
\end{thm}

\begin{thm}{Sard's theorem}
Let $\spc{P}$ be an $n$-dimensional Reimannian polyhedron and $f\:\spc{P}\to \RR$ be a function that is smooth on each simplex.
Then for almost all values $a$ each component of the inverse image $f^{-1}(a)$ is a equipped with the induced metric is a Reimannian polyhedron.
\end{thm}


\begin{thm}{Coarea inequality}
Let $\spc{P}$ be an $n$-dimensional Reimannian polyhedron and $f\:\spc{P}\to \RR$ be a 1-Lipschitz function that is smooth on each simplex.
Then 
\[\vol_n (f^{-1}[a,b]) \le \int_a^b\vol_{n-1}(f^{-1}\{x\})\cdot dx.\]
\end{thm}

Theorem \ref{thm:width<volpro} will be proved by induction on the dimension of $\spc{P}$;
the following exercise provides a base for the induction.
Note that $\spc{P}$ is connected by definition and if it is 1-dimensional, then 
\[\width\spc{P}=\width_0\spc{P}=\rad\spc{P}.\]

\begin{thm}{Exercise}\label{ex:1D-case}
Suppose $\spc{P}$ be a 1-dimensional Riemannian polhedron.
Suppose $\VolPro_{\spc{P}}(r)<r$ for some $r>0$.
Show that 
\[\width \spc{P}<r.\]

\end{thm}


An $(n-1)$-dimensional subpolyhedron $\spc{Q}\subset\spc{P}$ is called $R$-separating if each
connected component of the complement $\spc{P}\backslash \spc{Q}$ has radius smaller than $R$.

\begin{thm}{Lemma}\label{lem:separating}
Let $\spc{P}$ be an $n$-dimensional Riemannian polyhedron.
Then given $R>0$ and $\eps>0$ there is a $R$-separating subpolyhedron $\spc{Q}\subset\spc{P}$ such that for any $r_0<r_1\le R$ we have
\[\VolPro_{\spc{Q}}(r_0)< \tfrac1{r_1-r_0}\cdot \VolPro_{\spc{P}}(r_1)+\eps.\]

\end{thm}

\parit{Proof.}
Choose small $\delta>0$.
Applying the smoothing procedure, we can exchange each distance function $\distfun_p$ on $\spc{P}$ by $\delta$-close smooth 1-Lipschitz function, which will be denoted by $\widetilde \distfun_p$.

By Sard's theorem, almost all level sets $\tilde S_c(p)$ defined by $\widetilde \distfun_p=c$ are smooth Riemannian polyhedrons of dimension $n-1$.

Since $\delta$ is small, the coarea inequality implies that 
for some  $c\z\in(r_0+\delta, r_1-\delta)$ we have
\begin{align*}
\vol_{n-1}\tilde S_c(p)&\le \tfrac1{r_1-r_0-2\cdot\delta}\cdot\vol_n[\oBall(p,r_1)]<
\\
&<\tfrac1{r_1-r_0}\cdot \VolPro_{\spc{P}}(r_1)+\tfrac\eps2.
\end{align*}

Now suppose $\spc{Q}$ is an $R$-separating subpolyhedron in $\spc{P}$ with almost minimal volume, say its volume is at most $\tfrac\eps2$-far from the greatest lower bound.
Note that cutting from $\spc{Q}$ everything inside $\tilde S_c$ and adding $\tilde S_c$ keeps it to be $R$-separating subpolyhedron.
It follows that
\[\vol_{n-1}[\spc{Q}\cap \oBall(p,r_0)_{\spc{P}}]-\tfrac\eps2\le \vol_{n-1}S_c.\]
Therefore 
\[\vol_{n-1}[\spc{Q}\cap \oBall(p,r_0)_{\spc{P}}]\le\tfrac1{r_1-r_0}\cdot \VolPro_{\spc{P}}(r_1)+\eps\eqlbl{eq:volQ<ProP}\]
Recall that $\spc{Q}$ is equipped with the induced length metric;
therefore $\dist{p}{q}{\spc{Q}}\ge \dist{p}{q}{\spc{P}}$ for any $p,q\in \spc{Q}$;
in particular, 
\[\oBall(p,r_0)_{\spc{Q}}\subset \spc{Q}\cap \oBall(p,r_0)_{\spc{P}}.\]
Hence \ref{eq:volQ<ProP} implies the lemma.
\qeds

\begin{thm}{Lemma}\label{lem:separating-width}
Let $\spc{Q}$ be a $R$-separating subpolyhedron in an $n$-dimensional Riemannian polyhedron $\spc{P}$.
Suppose $\width\spc{Q}\le R$.
Then $\width\spc{P}\le R$
\end{thm}

\parit{Proof.}
Start with an open covering $\{V_1,\dots,V_k\}$ of $\spc{Q}$ of multiplicity $\le n$ with radiuses of the sets in the intrinsic metric $\le R$.

Note that $\{V_1,\dots,V_k\}$ can be converted into an an open covering of
a small neighbourhood of $\spc{Q}$ in $\spc{P}$ without increasing the multiplicity.
This is can be done by setting 
\[V_i'=\bigcup_{x\in V_i}\oBall(x,r_x),\]
where $r_x=\tfrac1{10}\cdot\inf\set{\dist{x}{y}{}}{y\in \spc{Q}\backslash V_i}$.

Finally, add all the components of $\spc{P}\backslash \spc{Q}$ to the covering;
it increases the multiplicity by 1.
The statement follows since $\dim \spc{P}= \dim \spc{Q}\z+1$.
\qeds

\parit{Proof of \ref{thm:width<volpro}.}
We apply induction on the dimension $n=\dim\spc{P}$;
the base case $n=1$ is provided by \ref{ex:1D-case},  for $c_1=1$.

Suppose that the constant $c_{n-1}$ is known, choose sufficiently small $c_n$
\[c_n>2\cdot c_{n-1}.\]

Assume $c_n\cdot \sqrt[n]{\VolPro\spc{P}(r)}< r$.
Fix small $\eps>0$.
By taking $r_0=\tfrac r2$ and $r_1=r$ in \ref{lem:separating}, we have an $r$-separating subpolhedron $\spc{Q}$ in $\spc{P}$ such that 
\begin{align*}
\VolPro_\spc{Q}(r_0) &< \tfrac 1 {r_0}\cdot \VolPro_\spc{P}(r)+\eps<
\\
&<\tfrac 1 {r_0}\cdot \left(\frac{2\cdot r_0}{c_n}\right)^n+\eps=
\\
&=\left(\frac2{c_n}\right)^n\cdot r_0^{n-1}+\eps<
\\
&<\left(\frac1{c_{n-1}}\right)^{n-1}\cdot r_0^{n-1};
\end{align*}
that is, $c_{n-1}\cdot \sqrt[n-1]{\VolPro\spc{Q}(r_0)}< r_0$.
By the induction hypothesis 
\[\width\spc{Q}\le r_0<r.\]

Applying \ref{lem:separating-width}, we get $\width\spc{P}<r$
\qeds

