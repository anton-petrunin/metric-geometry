\chapter{Volume bounds filling radius}

This lecture is devoted to a proof of \ref{thm:FillRad<vol};
that is, we will show that {}\emph{Riemannian manifolds with small volume have small filling radius}.
Note that once it is proved, \ref{thm:sys<FillRad} implies \ref{thm:sys(torus)}.
Moreover \ref{thm:sys<FillRad++} implies the following:

\begin{thm}{Therorem}\label{thm:sys(torus)+}
A systolic intequality holds for any essential manifold. 
\end{thm}







\section{Remarks}

%The notion of nerve was introduced by Pavel Alexandrov \cite{alexandrov-1928}.



In the end of \cite{nabutovsky} you can find a direct proof of the inequality 
\[\sys \spc{M}\le 4\cdot \width\spc{M}\]
for essential manifold $\spc{M}$.
The proof was suggested by Roman Karasev;
it makes possible to prove a systolic inequality without using filling radius.
In particular we get the systolic inequality 
\[\sys \spc{M}\le 4\cdot n\cdot\sqrt[n]{\vol\spc{M}}\]
holds for any $n$-dimensional essential Riemannian manifold.
The $n$-dimensional real projective space with canonical metric has systole $\pi$ and volume $v_n={(n+1)\cdot\pi^\frac{n+1}{2}}/[{2\cdot \Gamma(\tfrac{n+3}{2})]}$, therefore the optimal constant 
cannot be smaller than $c_n=\pi/\sqrt[n]{v_n}$ which grows as $\sqrt n$.
It is expected that $c_n$ is optimal constant, but this problem is a wide open.
