\chapter{Ultralimits}

Ultralimits provide a very general way to pass to a limit.
This procedure works for \textit{any} sequence of metric spaces, its result reminds limit in the sense of Gromov--Hausdorff, but has some strange features; for example, the limit of a constant sequence of spaces $\spc{X}_n=\spc{X}$ is \textit{not} $\spc{X}$ (see \ref{ex:ultrapower:compact}).

In geometry, ultralimits are used only as a canonical way to pass to a convergent subsequence.
It is useful in the proofs where one needs to repeat ``pass to convergent subsequence'' too many times.

This lecture is based on the introductory part of the paper by Bruce Kleiner and Bernhard Leeb \cite{kleiner-leeb}.

\section{Faces of ultrafilters}

Recall that $\NN$ denotes the set of natural numbers, $\NN=\{1,2,\dots\}$

\begin{thm}{Definition}
A finitely additive measure $\omega$ 
on $\NN$ 
is called an \index{ultrafilter}\emph{ultrafilter} if it satisfies the following condition:
\begin{subthm}{}
$\omega(\NN)=1$ and 
$\omega(S)=0$ or $1$ for any subset $S\subset \NN$.
\end{subthm}
An ultrafilter $\omega$ is called 
\emph{nonprincipal}\index{ultrafilter!nonprincipal ultrafilter}\index{nonprincipal ultrafilter} if in addition 
\begin{subthm}{}
$\omega(F)=0$ for any finite subset $F\subset \NN$.
\end{subthm}
\end{thm}

If $\omega(S)=0$ for some subset $S\subset \NN$,
we say that $S$ is \index{$\omega$-small}\emph{$\omega$-small}. 
If $\omega(S)=1$, we say that $S$ contains \index{$\omega$-almost all}\emph{$\omega$-almost all} elements of $\NN$.

\begin{thm}{Advanced exercise}\label{ex:ultrakatetov}
Let $\omega$ be an ultrafilter on $\NN$ and $f\:\NN\z\to \NN$.
Suppose that $\omega(S)\le \omega(f^{-1}(S))$ for any set $S\subset \NN$.
Show that $f(n)=n$ for $\omega$-almost all $n\in\NN$.
\end{thm}


\parbf{Classical definition.}
More commonly, a nonprincipal ultrafilter is defined as a collection, say $\mathfrak{F}$, of sets in $\NN$ such that
\begin{enumerate}
\item\label{filter:supset} if $P\in \mathfrak{F}$ and $Q\supset P$, then $Q\in \mathfrak{F}$,
\item\label{filter:cap} if $P, Q\in \mathfrak{F}$, then $P\cap Q\in \mathfrak{F}$,
\item\label{filter:ultra} for any subset $P\subset\NN$, either $P$ or its complement is an element of $\mathfrak{F}$.
\item\label{filter:non-prin} if $F\subset \NN $ is finite, then $F\notin \mathfrak{F}$.
\end{enumerate}
Setting $P\in\mathfrak{F}\Leftrightarrow\omega(P)=1$ makes these two definitions equivalent.

A nonempty collection of sets $\mathfrak{F}$ that does not include the empty set and satisfies only conditions \ref{filter:supset} and \ref{filter:cap} is called a \index{filter}\emph{filter}; 
if in addition $\mathfrak{F}$ satisfies condition \ref{filter:ultra} it is called an \index{ultrafilter}\emph{ultrafilter}.
From Zorn's lemma, it follows that every filter contains an ultrafilter.
Thus there is an ultrafilter $\mathfrak{F}$ contained in the filter of all complements of finite sets; clearly, this ultrafilter $\mathfrak{F}$ is nonprincipal.


\parbf{Stone--\v{C}ech compactification.}
Given a set $S\subset \NN$, consider subset $\Omega_S$ of all ultrafilters $\omega$ such that $\omega(S)=1$.
It is straightforward to check that the sets $\Omega_S$ for all $S\subset \NN$ form a topology on the set of ultrafilters on $\NN$. 
The obtained space is called \index{Stone--\v{C}ech compactification}\emph{Stone--\v{C}ech compactification} of $\NN$;
it is usually denoted as $\beta\NN$.

Let $\omega_n$ denotes the principal ultrafilter such that $\omega_n(\{n\})=1$; that is, $\omega_n(S)=1$ if and only if $n\in S$.
Note that $n\mapsto\omega_n$ defines a natural embedding $\NN\hookrightarrow\beta\NN$. 
Using the described embedding, we can (and will) consider $\NN$ as a subset of $\beta\NN$.

The space $\beta\NN$ is the maximal compact Hausdorff space that contains $\NN$  as an everywhere dense subset.
More precisely, for any compact Hausdorff space $\spc{X}$ 
and a map $f\:\NN\to \spc{X}$ there is a unique continuous map $\bar f\:\beta\NN\to X$ such that the restriction $\bar f|_\NN$ coincides with $f$. 

\section{Ultralimits of points}
\label{ultralimits}\index{ultralimit}

Let us fix a nonprincipal  ultrafilter $\omega$ once and for all.

Assume $x_n$ is a sequence of points in a metric space $\spc{X}$. 
Let us define the \index{$\omega$-limit}\emph{$\omega$-limit} of a sequence $x_1,x_2,\dots$ as the point $x_\omega$ 
such that for any $\eps>0$, point $x_n$ lie in $\oBall(x_\omega,\eps)$ for $\omega$-almost all $n$; 
that is,
\[\omega\set{n\in\NN}{\dist{x_\omega}{x_n}{}<\eps}=1.\]
In this case, we will write 
\[x_\omega=\lim_{n\to\omega} x_n
\ \ \text{or}\ \ 
x_n\to x_\omega\ \text{as}\ n\to\omega.\]

For example, if $\omega_n$ is the \textit{principal} ultrafilter such that $\omega_n\{n\}=1$ for some $n\in\NN$, then
$x_{\omega_n}=x_n$.

The sequence $x_n$ can be regarded as a map $\NN\to\spc{X}$ defined by $n\mapsto x_n$.
If $\spc{X}$ is compact, then the map $\NN\to\spc{X}$ can be extended to a continuous map $\beta\NN\to\spc{X}$ from the Stone--\v{C}ech compactification $\beta\NN$ of $\NN$.
Then the $\omega$-limit $x_\omega$ can be regarded as the image of $\omega$.

Note that the $\omega$-limits of a sequence and its subsequence may differ.
For example, sequence $y_n=-(-1)^n$ is a subsequence of $x_n=(-1)^n$, but for any ultrafilter $\omega$, we have
\[\lim_{n\to\omega}x_n
\ne
\lim_{n\to\omega}y_n.\] 

\begin{thm}{Proposition}\label{prop:ultra/partial}
Let $\omega$ be a nonprincipal ultrafilter.
Assume $x_n$ is a sequence of points in a metric space $\spc{X}$
and $x_n\to  x_\omega$ as $n\to\omega$.
Then $x_\omega$ is a partial limit of the sequence $x_n$;
that is, there is a subsequence $(x_n)_{n\in S}$ that converges to $x_\omega$ in the usual sense.
\end{thm}

\parit{Proof.}
Given $\eps>0$, 
set $S_\eps=\set{n\in\NN}{\dist{x_n}{x_\omega}{}<\eps}$.

Note that $\omega(S_\eps)=1$ for any $\eps>0$.
Since $\omega$ is nonprincipal, the set $S_\eps$ is infinite.
Therefore, we can choose an increasing sequence $n_k$
such that $n_k\in S_{\frac1k}$ for each $k\in \NN$.
Clearly $x_{n_k}\to x_\omega$ as $k\to\infty$.
\qeds

The following proposition 
is analogous to the statement that any sequence in a compact metric space 
has a convergent subsequence;
it can be proved the same way.

\begin{thm}{Proposition}\label{prop:ultra/compact}
Let $\spc{X}$ be a compact metric space.
Then
any sequence $x_n$ of points in $\spc{X}$ has a unique $\omega$-limit $x_\omega$.

In particular, a bounded sequence of real numbers has a unique $\omega$-limit.
\end{thm}

The following lemma is an ultralimit analog of the Cauchy convergence test.

\begin{thm}{Lemma}\label{lem:X-X^w}
Let $x_n$ be a sequence of points in a complete space~$\spc{X}$. 
Assume for each subsequence $y_n$ of $x_n$, 
the $\omega$-limit 
\[y_\omega=\lim_{n\to\omega}y_{n}\in \spc{X}\]
is defined and does not depend on the choice of subsequence, then the sequence $x_n$ converges in the usual sense.
\end{thm}

\parit{Proof.} If $x_n$ is not a Cauchy sequence, then for some $\eps>0$, there is a subsequence $y_n$ of $x_n$ such that $\dist{x_n}{y_n}{}\ge\eps$ for all $n$.

It follows that $\dist{x_\omega}{y_\omega}{}\ge \eps$ --- a contradiction.\qeds

\begin{thm}{Exercise}\label{ex:linear}
Denote by $S$ the space of bounded sequences of real numbers.
Show that there is a linear functional $L\:S\to\RR$ such that
for any sequence $\bm{s}=(s_1,s_2,\dots)\in S$ the image $L(\bm{s})$ is a partial limit of $s_1,s_2,\dots$
\end{thm}

\begin{thm}{Exercise}\label{ex:ultrakatetov+}
Suppose that $f\:\NN\to\NN$ is a map such that 
\[\lim_{n\to\omega}x_n=\lim_{n\to\omega}x_{f(n)}\]
for any bounded sequence $x_n$ of real numbers.
Show that $f(n)=n$ for $\omega$-almost all $n\in\NN$.
\end{thm}

\section{An illustration}

\begin{thm}{Claim}
Let $\spc{X}$ and $\spc{Y}$ be compact spaces.
Suppose that for every $n\in\NN$ there is a $\tfrac1n$-isometry $f_n\:\spc{X}\to \spc{Y}$.
Then there is an isometry $\spc{X}\to \spc{Y}$.
\end{thm}

We give a proof of this claim only as an illustration for ulralimits.

\parit{Proof.}
Consider the $\omega$-limit $f_\omega$ of~$f_n$;
according to \ref{prop:ultra/compact}, $f_\omega$ is defined.
Since 
\[|f_n(x)-f_n(x')|\lege |x-x'|\pm\tfrac1n\]
we get that 
\[|f_\omega(x)-f_\omega(x')|= |x-x'|\]
for any $x,x'\in \spc{X}$;
that is, $f_\omega$ is distance-preserving.

Further, since $f_n$ is a $\tfrac1n$-isometry,
for any $y\in \spc{Y}$ there is a sequence $x_n\in \spc{X}$ such that $|f_n(x_n)-y|\le \tfrac1n$.
Therefore,
\[f_\omega(x_\omega)=y,\]
where $x_\omega$ is the $\omega$-limit of $x_n$;
that is, $f_\omega$ is onto.

It follows that $f_\omega\:\spc{X}\to\spc{Y}$ is an isometry.
\qeds

\section{Ultralimits of spaces}\label{sec:Ultralimit of spaces}

Recall that $\omega$ denotes a nonprincipal ultrafilter on the set of natural numbers.

Let $\spc{X}_n$ be a sequence of metric spaces.
Consider all sequences of points $x_n\in \spc{X}_n$.
On the set of all such sequences,
define a pseudometric by
\[\dist{(x_n)}{(y_n)}{}
=
\lim_{n\to\omega} \dist{x_n}{y_n}{\spc{X}_n}.
\eqlbl{eq:olim-dist}\]
Note that the $\omega$-limit on the right-hand side is always defined 
and takes a value in $[0,\infty]$. 
(The $\omega$-convergence to $\infty$ is defined analogously to the usual convergence to $\infty$).

Set $\spc{X}_\omega$ to be the corresponding metric space; 
that is, the underlying set of $\spc{X}_\omega$ is formed by classes of equivalence of sequences of points $x_n\in\spc{X}_n$ 
defined by 
\[(x_n)\sim(y_n)
\ \Leftrightarrow\ 
\lim_{n\to\omega} \dist{x_n}{y_n}{}=0\]
and the distance is defined by \ref{eq:olim-dist}.

The space $\spc{X}_\omega$ is called the \index{$\omega$-limit space}\emph{$\omega$-limit} of $\spc{X}_n$.
Typically  $\spc{X}_\omega$ will denote the  
$\omega$-limit of sequence $\spc{X}_n$;
we may also write  
\[\spc{X}_n\to\spc{X}_\omega\ \ \text{as}\ \  n\to\omega\ \ \text{or}\ \ \spc{X}_\omega=\lim_{n\to\omega}\spc{X}_n.\]

Given a sequence $x_n\in \spc{X}_n$,
we will denote by $x_\omega$ its equivalence class which is a point in $\spc{X}_\omega$;
it can be written as
\[x_n\to x_\omega \ \ \text{as}\ \  n\to\omega,\ \ \text{or}\ \ x_\omega=\lim_{n\to\omega} x_n.\]

\begin{thm}{Observation}\label{obs:ultralimit-is-complete}
The $\omega$-limit of any sequence of metric spaces is complete. 
\end{thm}

We will repeat the proof of \ref{ex:complete-completion} using a slightly different language.

\parit{Proof.}
Let $\spc{X}_n$ be a sequence of metric spaces and $\spc{X}_n\to\spc{X}_\omega$ as $n\to\omega$.
Choose a Cauchy sequence $x_1,x_2,\dots{}\in\spc{X}_\omega$.
Passing to a subsequence, we can assume that $\dist{x_k}{x_{m}}{\spc{X}_\omega}<\tfrac1{k}$ if $k<m$.

Choose a double sequence $x_{n,m}\in \spc{X}_n$ such that for any fixed $m$ we have $x_{n,m}\to x_m$ as $n\to\omega$.
Note that for any $k<m$ the inequality $\dist{x_{n,k}}{x_{n,m}}{}<\tfrac1{k}$ holds for $\omega$-almost all $n$.
It follows that we can choose a nested sequence of sets 
\[\NN= S_1\supset S_2\supset\dots\] 
such that 
\begin{itemize}
\item $\omega(S_m)=1$ for each $m$, 
\item $\bigcap_m S_m=\emptyset$, and
\item $\dist{x_{n,k}}{x_{n,l}}{}<\tfrac1{k}$ for $k<l\le m$ and $n\in S_m$.
\end{itemize}

Consider the sequence $y_n=x_{n,m(n)}$, where $m(n)$ is the largest value such that $n\in S_{m(n)}$.
Denote by $y_\omega\in \spc{X}_\omega$ the $\omega$-limit of $y_n$.

Observe that $|y_m-x_{n,m}|<\tfrac1{m}$ for $\omega$-almost all $n$.
It follows that $|x_m-y_\omega|\le \tfrac1{m}$ for any $m$.
Therefore, $x_n\to y_\omega$ as $n\to \infty$.
That is, any Cauchy sequence in $\spc{X}_\omega$ converges.
\qeds

\begin{thm}{Observation}\label{obs:ultralimit-is-geodesic}
The $\omega$-limit of any sequence of length spaces is geodesic. 
\end{thm}

\parit{Proof.}
If $\spc{X}_n$ is a sequence of length spaces, then for any sequence of pairs $x_n, y_n\in X_n$ there is a sequence of $\tfrac1n$-midpoints $z_n$.

Let $x_n\to x_\omega$, $y_n\to y_\omega$ and $z_n\to z_\omega$ as $n\to \omega$.
Note that $z_\omega$ is a midpoint of $x_\omega$ and $y_\omega$ in $\spc{X}_\omega$.

By Observation~\ref{obs:ultralimit-is-complete}, $\spc{X}_\omega$ is complete.
Applying Lemma~\ref{lem:mid>geod} we get the statement.
\qeds


\begin{thm}{Exercise}\label{ex:lim(tree)}
Show that an ultralimit of metric trees is a metric tree. 
\end{thm}

\begin{thm}{Exercise}\label{ex:ultracompact}
Suppose that $\spc{X}_\infty$ and $\spc{X}_1,\spc{X}_2,\dots$ are compact metric spaces.
Assume $\spc{X}_n\GHto\spc{X}_\infty$.
Show that $\spc{X}_\omega\iso\spc{X}_\infty$.
\end{thm}


\section{Ultrapower}

If all the metric spaces in the sequence are identical $\spc{X}_n=\spc{X}$, 
its $\omega$-limit 
$\lim_{n\to\omega}\spc{X}_n$
is denoted by $\spc{X}^\omega$
and called $\omega$-power of $\spc{X}$.



\begin{thm}{Exercise}\label{ex:ultrapower}
For any point $x\in \spc{X}$, consider the constant sequence $x_n=x$
and set $\iota(x)=\lim_{n\to\omega}x_n\in\spc{X}^\omega$.

\begin{subthm}{ex:ultrapower:a}
Show that $\iota\:\spc{X}\to\spc{X}^\omega$ is distance-preserving embedding. (So we can and will consider $\spc{X}$ as a subset of $\spc{X}^\omega$.)
\end{subthm}

\begin{subthm}{ex:ultrapower:compact} 
Show that $\iota$ is onto if and only if $\spc{X}$ is compact.
\end{subthm}

\begin{subthm}{ex:ultrapower:proper} 
Show that if $\spc{X}$ is proper, then $\iota(\spc{X})$ forms a metric component of $\spc{X}^\omega$; that is, a subset of $\spc{X}^\omega$ that lies at a finite distance from a given point.
\end{subthm}

\end{thm}

Note that \ref{SHORT.ex:ultrapower:compact} implies that the inclusion $\spc{X}\hookrightarrow\spc{X}^\omega$ is not onto if the space $\spc{X}$ is not compact.
However, the spaces $\spc{X}$ and $\spc{X}^\omega$ might be isometric; here is an example:

\begin{thm}{Exercise}\label{ex:isom-ultrapower}
Let $\spc{X}$ be an infinite countable set with discrete metric;
that is $\dist{x}{y}{\spc{X}}=1$ if $x\ne y$.
Show that 

\begin{subthm}{ex:isom-ultrapower:no}
$\spc{X}^\omega$ is not isometric to $\spc{X}$.
\end{subthm}

\begin{subthm}{ex:isom-ultrapower:yes}
$\spc{X}^\omega$ is  isometric to $(\spc{X}^\omega)^\omega$.
\end{subthm}

\end{thm}

\begin{thm}{Exercise}\label{ex:ultrapower(ultrapower)}
Given a nonprincipal ultrafilter $\omega$, construct an ultrafilter $\omega_1$ such that 
\[\spc{X}^{\omega_1}\iso(\spc{X}^\omega)^\omega\]
for any metric space~$\spc{X}$.

\end{thm}


\begin{thm}{Observation}\label{obs:ultrapower-is-geodesic}
Let $\spc{X}$ be a complete metric space. 
Then $\spc{X}^\omega$ is geodesic space if and only if $\spc{X}$ is a length space.
\end{thm}

\parit{Proof.}
The ``if''-part follows from \ref{obs:ultralimit-is-geodesic}; it remains to prove the ``only-if''-part

Assume $\spc{X}^\omega$ is geodesic space.
Then any pair of points $x,y\in \spc{X}$ has a midpoint $z_\omega\in\spc{X}^\omega$.
Fix a sequence of points $z_n\in  \spc{X}$ such that $z_n\to z_\omega$ as $n\to \omega$.

Note that 
$\dist{x}{z_n}{\spc{X}}\to \tfrac12\cdot \dist{x}{y}{\spc{X}}$
and 
$\dist{y}{z_n}{\spc{X}}\to \tfrac12\cdot \dist{x}{y}{\spc{X}}$
as 
$n\to\omega$.
In particular, for any $\eps>0$, the point $z_n$ is an $\eps$-midpoint of $x$ and $y$ for $\omega$-almost all $n$.
It remains to apply \ref{lem:mid>geod}.
\qeds

\begin{thm}{Exercise}\label{ex:two-geodesics-in-ultrapower}
Assume $\spc{X}$ is a complete length space 
and $p,q\in\spc{X}$ cannot be joined by a geodesic in $\spc{X}$.  
Then there are at least two distinct geodesics between $p$ and $q$ 
in the ultrapower $\spc{X}^\omega$.
\end{thm}

\begin{thm}{Exercise}\label{ex:notproper-limit}
Construct a proper metric space $\spc{X}$ such that $\spc{X}^\omega$ is not proper;
that is, there is a point $p\in \spc{X}^\omega$ and $R<\infty$ such that the closed ball $\cBall[p,R]_{\spc{X}^\omega}$ is not compact.
\end{thm}

\section{Tangent and asymptotic spaces}
\label{sec:tan+asymptotic}

Choose a space $\spc{X}$ and a sequence $\lambda_n$ of positive numbers.
Consider the sequence of \index{rescaled space}\emph{rescalings} $\spc{X}_n=\lambda_n\cdot\spc{X}=(\spc{X},\lambda_n\cdot\dist{*}{*}{\spc{X}})$.

Choose a point $p\in \spc{X}$ and denote by $p_n$ the corresponding point in $\spc{X}_n$.
Consider the $\omega$-limit $\spc{X}_\omega$ of $\spc{X}_n$ (one may denote it by $\lambda_\omega\cdot \spc{X}$);
set $p_\omega$ to be the $\omega$-limit of $p_n$.

If $\lambda_n\to \infty$ as $n\to\omega$, then the metric component of $p_\omega$ in $\spc{X}_\omega$ is called \index{$\lambda_\omega$-tangent space}\emph{$\lambda_\omega$-tangent space} at $p$ and denoted by $\T_p^{\lambda_\omega}\spc{X}$ (or $\T_p^{\omega}\spc{X}$ if $\lambda_n=n$).\label{page:ultratangent space}

If $\lambda_n\to 0$ as $n\to\omega$, then the metric component of $p_\omega$ is called \index{$\lambda_\omega$-asymptotic space}\emph{$\lambda_\omega$-asymptotic space}%
\footnote{Often it is called an {}\emph{asymptotic cone} despite that it is not a cone in general; this name is used since in good cases it has a cone structure.} and denoted by $\Asym\spc{X}$ or $\Asym^{\lambda_\omega}\spc{X}$.
Note that the space $\Asym\spc{X}$ and its point $p_\omega$ does not depend on the choice of $p\in \spc{X}$.

The following exercise states that the constructions above depend on the sequence $\lambda_n$ and a nonprincipal ultrafilter $\omega$.

\begin{thm}{Exercise}\label{ex:ultraT}
Construct a metric space $\spc{X}$ with a point $p$ such that the tangent space
$\T_p^{\lambda_\omega}\spc{X}$ depends on the sequence $\lambda_n$ and/or ultrafilter~$\omega$.
\end{thm}

For nice spaces, different choices may give the same space.

\begin{thm}{Exercise}\label{ex:Asym(Lob)}
Let $\spc{T}=\Asym\spc{L}$, where $\spc{L}$ is the Lobachevsky plane, or Lobachevsky space, or 3-regular%
\footnote{that is, the degree of any vertex is 3.}
metric tree with unit edge length (choose your favorite space from these three).

\begin{subthm}{ex:Asym(Lob):metric-tree}
Show that $\spc{T}$ is a complete metric tree.
\end{subthm}

\begin{subthm}{ex:Asym(Lob):homogeneous}
Show that $\spc{T}$ is one-point-homogeneous; that is, given two points $s,t\in \spc{T}$ there is an isometry of $\spc{T}$ that maps $s$ to $t$.
\end{subthm}

\begin{subthm}{ex:Asym(Lob):continuum}
Show that $\spc{T}$ has \index{degree}\emph{continuum degree} at any point;
that is, for any point $t\in \spc{T}$ the set of connected components of the complement $\spc{T}\setminus\{t\}$ has cardinality continuum.
\end{subthm}

\end{thm}




\section{Remarks}

A nonprincipal ultrafilter $\omega$ is called 
\emph{selective}\index{ultrafilter!selective ultrafilter}\index{selective ultrafilter} if for any partition of $\NN$ into sets $\{C_\alpha\}_{\alpha\in\IndexSet}$ such that $\omega(C_\alpha)\z=0$ for each $\alpha$, 
there is a set $S\subset \NN$ such that $\omega(S)=1$ and $S\cap C_\alpha$ is a one-point set for each $\alpha\in\IndexSet$.

The existence of a selective ultrafilter follows from the continuum hypothesis \cite{rudin}.

If needed, we may assume that the chosen ultrafilter $\omega$ is selective.
In this case \textit{the subsequence $(x_n)_{n\in S}$ in \ref{prop:ultra/partial} can be chosen so that $\omega(S)=1$}.

