\documentclass[twoside]{book}

\usepackage{lectures}
\usepackage[colorlinks=true,
citecolor=black,
linkcolor=black,
anchorcolor=black,
filecolor=black,
menucolor=black,
urlcolor=black,
pdftitle={Metric geometry on manifolds: two lectures},
pdfsubject={Geometry},
pdfauthor={Anton Petrunin}
]{hyperref}
\makeindex

\begin{document}
%\pagestyle{empty}
 
\title{Metric geometry on manifolds:
\\ two lectures}
\author{Anton Petrunin}
\date{}
\maketitle

We discuss Besicovitch inequality, width, and systole of manifolds.
Prerequisite includes  
basic measure theory, and
definitions of smooth manifolds,
degree of map, 
CW-complexes,
and related notions.

This is a polished version of two final lectures of a graduate course given at Penn State, Spring 2020.
The complete lectures can be found on the author's website;
it includes an introduction to metric geometry \cite{petrunin2020pure}
and elements of Alexandrov geometry based on \cite{alexander-kapovitch-petrunin-2019}.

\parbf{Acknowledgments.}
I want to thank Alexander Lytchak and Alexander Nabutovsky for help.

\thispagestyle{empty}
\tableofcontents
\thispagestyle{empty}

%%%%%%%%%%%%%%%%%%%%%%%%%%%%
%\addtocounter{chapter}{-1}
\chapter{Homework assignments}


It is better to think about all the problems, but you do not have to solve \emph{all} of them.
If a problem is solved, you do not have to write its solutions, but try sketch it.

\section{Due Tue Jan 21}

Exercises: \sout{\ref{ex:almost-min},} \ref{ex:non-contracting-map}, \ref{ex:no-geod}, \sout{\ref{ex:compact=>complete},} \ref{exercise from BH}, \ref{ex:Hausdorff-bry}.

\section{Due Tue Jan 28}

Exercises: \ref{ex:almost-min},  \ref{ex:compact=>complete}, \ref{ex:Huas-perimeter-area}, \ref{ex:GH-po}, \ref{pr:doubling}, \ref{pr:under:if}.

\section{Due Tue Feb 4}
Exercises: 
\ref{ex:compact-length}, 
\ref{pr:under:only-if}, 
\sout{\ref{ex:GH-SC},}
\sout{\ref{ex:sphere-to-ball},}
\ref{ex:ultrapower}, 
\ref{ex:two-geodesics-in-ultrapower}.

\section{Due Tue Feb 11}

Finish exercises \ref{ex:compact-length} , \ref{pr:under:only-if}, \ref{ex:GH-SC}, \ref{ex:sphere-to-ball}.

\noindent
Exercises: \ref{ex:lim(tree)}, \ref{ex:Asym(Lob)}, \ref{ex:geodesics-urysohn}, \ref{ex:sphere-in-urysohn}.

\section{Due Tue Feb 18}

Exercises: \ref{ex:compact-extension}, \ref{ex:+-c}, \ref{ex:ultrametric}, \ref{ex:injective-spaces}, \ref{ex:tripod+square}, \ref{ex:4-on-a-line}.

\noindent Write down a solution of at least one of the exercises.

\section{Due Tue Feb 25}

Finish Exercise \ref{ex:tripod+square:square}.
Prepare questions for review on Tuesday.

\section{Due Tue Mar 3}

Exercises: \ref{ex:sba-2+2-short}, \ref{ex:(3+1)-expanding}, \ref{ex:CAT+CBB}, \ref{ex:product-CBB}, \sout{\ref{ex:CBB-geodesic},} \ref{ex:fat-triangle}.

\noindent Write down a solution of at least one of the exercises.

\section{Due Tue Mar 17}

Exercises: \ref{ex:tringle-inq-angles},
\ref{ex:CBB-geodesic},
\ref{ex:convex-dist},
\ref{ex:reshetnyak-doubling},
\ref{ex:supporting-planes},
\ref{ex:centrally-simmetric-walls}.

\noindent Write down a solution of at least one of the exercises.

\section{Due Tue Mar 24}

Exercises: 
\ref{ex:contractible},
\ref{ex:convex-nbhd},
\ref{ex:closest-point},
\ref{cor:balls:dim=1},
\ref{ex:null-homotopic},
\sout{\ref{ex:branching-cover}.}

 Write down as many solutions as you can; email it to Zetian Yan (zxy5156) + cc to me (aqp6).

\section{Due Tue Mar 31}

Exercises: 
\ref{ex:branching-cover},
\ref{ex:tan(CAT)isCAT},
\ref{ex:tan(CAT)is-length},
%\ref{ex:product-cone},
\ref{ex:unique-geod=CAT},
\ref{ex:flag>=pi/2},
\ref{ex:tree}.

Write down as many solutions as you can; email it to Zetian Yan (zxy5156) + cc to me (aqp6).

\section{Due Tue Apr 7}

Exercises: 
\ref{ex:CAT-mnfld=>ext.geod},
\ref{ex:locally-convex},
\ref{ex:geod-circle},
\ref{ex:flag-aspherical},
\ref{ex:example-pi_infty-new},
\ref{ex:cube-infty=>cube-2}.

Write down as many solutions as you can; email it to Zetian Yan (zxy5156) + cc to me (aqp6).

\section{Due Tue Apr 14}

Exercises: 
\ref{ex:geod-CBA},
%\ref{prop:two-hull-open},
%\ref{ex:chopping-triangle},
\ref{ex:concave-triangle},
\ref{ex:two-planes},
\ref{ex:hemisphere},
\ref{ex:inner-support},
\ref{ex:convex+saddle+broken=>PL}.

Write down as many solutions as you can; email it to Zetian Yan (zxy5156) + cc to me (aqp6).

\section*{Remark}
Each working day I will check email before 15:00 and will appear online if you ask (it is easy for me --- do not hesitate to ask).
We will meet regular hours online (as we did before).

%%%%%%%%%%%%%%%%%%%%%%%%%%%%

\chapter{Besicovitch inequality} 

We will focus on Riemannian spaces --- these are specially nice length metrics on manifolds.
These spaces are also most important in applications.

As it will be indicated in Section~\ref{sec:hausdorff-measure},
most of the statements of this and the following lecture have counterparts for general length metrics on manifolds.

\section{Riemannian spaces}

Let $M$ be a smooth connected manifold.
A \index{metric tensor}\emph{metric tensor} on $M$ is a choice of positive definite quadratic forms $g_p$ on each tangent space $\T_pM$ that depends continuously on the point;
that is, in any local coordinates of $M$ the components of $g$ are continuous functions.

A \index{Riemannian!manifold}\emph{Riemannian manifold} $(M,g)$ is a smooth manifold $M$ with a choice of metric tensor $g$ on it.

The \index{length}\emph{$g$-length} of a Lipschitz curve $\gamma\:[a,b]\to M$  is defined by
\[\length_g\gamma=\int_a^b\sqrt{g(\gamma'(t),\gamma'(t))}\cdot dt.\]
The $g$-length induces a metric metric on $M$; it is defined as the greatest lower bound to lengths of Lipschitz curves connecting two given points;
the distance between a pair of points $x,y\in M$ will be denoted by 
\[\dist{x}{y}{g}\quad\text{or}\quad\distfun_x(y)_g.\]
The corresponding metric space $\spc{M}$ will be called \index{Riemannian!space}\emph{Riemannian}.

The following exercise implies that \textit{isometry between Riemannian spaces might be not induced by a diffeomorphism}.

\begin{thm}{Exercise}\label{ex:non-differentiable}
Construct a continuous Riemannian metric $g$ on $\RR^2$ such that the corresponding Riemannian space admits an isometry to the Euclidean palne but the induced map $\iota\:\RR^2\to\RR^2$ is not differentiable at some point.
\end{thm}

The exercise above shows that in general the smooth structure is not uniquely defined on Riemannian space.
Therefore in general case one has to distinguish between Riemannian manifold and the corresponding Riemannian space altho there is almost no difference.%
\footnote{In fact a straightforward smoothing procedure shows that isometry between Riemannian spaces can be approximated by diffeomorphisms between underlying manifolds; in particular these manifolds are diffeomorphic.
Also, if the metric tensor is smooth, then it is not hard to show that Riemannian space {}\emph{remembers} everything about the Riemannian manifold, in particular the smooth structure;
it is a part of the so-called Myers--Steenrod theorem \cite{myers-steenrod}.} 

The following observation states the key property of Riemannian spaces;
it will be used to extend results from Euclidean space to Riemannian spaces.

\begin{thm}{Observation}\label{obs:lip-chart}
For any point $p$ in a Riemannian space $\spc{M}$ and any $\eps>0$ there is a $e^{\mp\eps}$-bilipschitz chart $s\:W\to V$ from an open subset $W$ of the $n$-dimensional Euclidean space to some neighborhood $V\ni p$.
\end{thm}

\parit{Proof.}
Choose a chart $s\:U\to \spc{M}$ that covers $p$.
Note that there is a linear transformation $L$ such that for the metric tensor in the chart $s\circ L$ is coincides with the standard Euclidean tensor at the point $x=(s\circ L)^{-1}(p)$.

Since the metric tensor is continuous, the restriction of $s\circ L$ to a small neighborhood of $x$ is $e^{\mp\eps}$-bilipschitz.
\qeds

\section{Volume and Hausdorff measure}\label{sec:vol-haus}

Let $(M,g)$ be an $n$-dimensional Riemannian manifold.
If a Borel set $R\subset M$ is covered by one chart $\iota\:U\to M$,
then its \index{volume}\emph{volume} (briefly, $\vol R$ or $\vol_n R$) is defined by 
\[\vol R
\df
\int_{\iota^{-1}(R)}\sqrt{\det{g}}.\]
In the general case we can subdivide $R$ into a countable collection of regions $R_1,R_2\dots$ such that each region $R_i$ is covered by one chart $\iota_i\:U_i\to M$ and define
\[\vol R\df \vol R_1+\vol R_2+\dots\]
The chain rule for multiple integrals implies that the right-hand side does not depend on the choice of subdivision and the choice of charts.

Similarly, we define integral along $(M,g)$.
Any Borel function $u\:M\z\to \RR$, can be presented as a sum $u_1+u_2+\cdots$ such that the support of each function $u_i$ can be covered by one chart $\iota_i\:U_i\to M$
and set 
\[\int_{p\in\spc{M}} u(p)
\df
\sum_i\left[\int_{x\in U_i} u_i\circ s(x)\cdot\sqrt{\det{g}}\right].
\]
In particular
\[\vol R=\int_{p\in R} 1.\]

Let $\spc{X}$ be a metric space and $R\subset \spc{X}$.
The \index{Hausdorff measure}\emph{$\alpha$-dimensional Hausdorff measure} of $R$ is defined by 
$$\haus_\alpha R
\df
\lim_{\eps\to0}
\,
\inf
\set{\sum_{n\in\NN}(\diam A_n)^\alpha}
{\begin{aligned}
&\diam A_n<\eps\ \text{for}
\\
&\text{for each}\ n,\text{all}\  A_n
\\
&\text{are closed, and} 
\\
& \bigcup_{n\in\NN}A_n\supset R.
\end{aligned}
}.$$
For properties of Hausdorff measure we refer to the classical book of  Herbert Federer \cite{federer};
in particular, $\haus_\alpha$ is indeed a measure and $\haus_\alpha$-measurable sets include all Borel sets.

The following observation follows from \ref{obs:lip-chart} and Rademacher's theorem:

\begin{thm}{Observation}\label{obs:lipcart+}
Suppose that a Borel set $R$ in an $n$-dimensional Riemannian space $\spc{M}$ is subdivided into a countable collection of subsets $R_i$ such that each $R_i$ is covered by an $e^{\mp\eps}$-bilipschitz charts
$s_i$.
Then
\begin{align*}
\vol_n R&\lege e^{\pm n\cdot\eps}\cdot\sum_i\vol_n[s_i^{-1}(R_i)]
\intertext{and}
\haus_n R&\lege e^{\pm n\cdot\eps}\cdot\sum_i\haus_n[s_i^{-1}(R_i)]
\end{align*}

\end{thm}

According to \index{Haar's theorem}\emph{Haar's theorem}, 
a measure on $n$-dimensional Euclidean space that is invariant with respect to parallel translations is proportional to volume.
Observe that 
\begin{itemize}
\item A ball in $n$-dimensional Euclidean space of diameter $1$ has unit Hausdorff measure.
\item A unit cube in $n$-dimensional Euclidean space has unit volume.
\end{itemize}
Therefore, for any Borel region $R\subset \EE^n$, we have 
\[\vol_n R=\tfrac{\omega_n}{2^n}\cdot\haus_n R,\eqlbl{eq:vol/mu}\]
where $\omega_n$ denotes the volume of a unit ball in the $n$-dimensional Euclidean space.

Applying \ref{eq:vol/mu} together with \ref{obs:lipcart+}, we get that the inequalities
\[\vol_n R\lege e^{\pm2\cdot n\cdot \eps}\cdot\tfrac{\omega_n}{2^n}\cdot\haus_n R\]
hold for any $\eps>0$.
Since $\eps>0$ is arbitrary, we get that \ref{eq:vol/mu} holds in $n$-dimensional Riemannian spaces.
More precisely:

\begin{thm}{Proposition}\label{prop:vol=haus}
The identity 
\[\vol_n R=\tfrac{\omega_n}{2^n}\cdot\haus_n R\]
holds for any Borel region $R$ in an $n$-dimensional Riemannian space. 
\end{thm}

Since the Hausdorff measure is defined in pure metric terms, the proposition gives another way to prove that the volume does not depend on the choice of chars and subdivision of $R$.

The identity in this proposition will be used to define volume of any dimension.
Namely, given an integer $k\ge 0$, the $k$-volume is defined by
\[\vol_k\df\tfrac{\omega_k}{2^k}\cdot\haus_k.\]
By \ref{prop:vol=haus}, if $A$ is a subset of $k$-dimensional submanifold $\spc{N}\subset \spc{M}$, then the two definitions of $\vol_kA$ agree; but the latter definition works for a wider class of sets. 

\begin{thm}{Exercise}\label{ex:volume-preserving+short}
Let $f\:\spc{M}\to \spc{N}$ be a short volume-preserving map between $n$-dimensional Riemannian spaces.
Prove the following statements and use them to conclude that $f$ is locally distance-preserving.

\begin{subthm}{ex:volume-preserving+short:injective}
$f$ is injective; 
that is, if $f(x)=f(y)$, then $x=y$.
\end{subthm}

\begin{subthm}{ex:volume-preserving+short:bi}
For any $c<1$, the map $f$ is locally $[c,1]$-bilipschitz;
that is, for any point in $\spc{M}$ there is a neighborhood $\Omega$ and $\eps>0$ such that the inequality 
\[c\le \frac{|f(x)-f(y)|_{\spc{N}}}{|x-y|_{\spc{M}}}\le 1 \]
holds for any pair of distinct points $x,y\in \Omega$.
\end{subthm}

\end{thm}


\section{Area and coarea formulas}

Suppose that $f\:\spc{M}\to\spc{N}$ is a Lipschitz map between $n$-dimensional Riemannian spaces $\spc{M}$ and $\spc{N}$.
Then by \index{Rademacher's theorem}\emph{Rademacher's theorem} 
the differential $d_p f\:\T_p\spc{M}\to\T_{f(p)}\spc{N}$ is defined at \index{almost all}\emph{almost all} $p\in \spc{M}$;
that is, the differential defined at all points $p\in\spc{M}$ with exception of a subset with vanishing volume.

The differential is a linear map; it defines the Jacobian matrix $\Jac_pf$ in orthonormal frames of $\T_p$ and $\T_{f(p)}\spc{N}$.
The determinant of $\Jac_pf$ will be denoted by $\jac_p$.
Note that the absolute value $|\jac_p|$ does not depend on the choice of the orthonormal frames.

The identity in the following proposition is called \index{area formula}\emph{area formula}.

\begin{thm}{Proposition}
Let $f\:\spc{M}\to\spc{N}$ be a Lipschitz map between $n$-dimensional Riemannian spaces $\spc{M}$.
Then for  any Borel function $u\:\spc{M}\z\to \RR$ the following equality holds:
\[\int_{p\in \spc{M}} u(p)\cdot |\jac_pf|=\int_{q\in \spc{N}}\sum_{p\in f^{-1}(q)} u(p).\]

\end{thm}

\parit{Proof.}
If $\spc{M}$ and $\spc{N}$ are isometric to the $n$-dimensional Euclidean space, then the statement follows from the standard area formula \cite[3.2.3]{federer}.

Note that Jacobian of a $e^{\mp\eps}$-bilipschitz map between $n$-dimensional Riemannian manifolds (if defined) has determinant in the range $e^{\mp n\cdot\eps}$.
Applying \ref{obs:lipcart+} and the area formula in $\EE^n$, we get the following approximate version of the needed identity for any $u\ge0$: 
\[\int_{p\in \spc{M}} u(p)\cdot |\jac_pf|
\lege e^{\pm 3\cdot n\cdot \eps}\int_{q\in \spc{N}}\sum_{p\in f^{-1}(q)} u(p).\]

Since $\eps>0$ is arbitrary, we get that the area formula holds if $u\ge 0$.
Finally, since both sides of the area formula are linear in $u$, it holds for any $u$.
\qeds

The following inequality is called \index{area inequality}\emph{area inequality}:

\begin{thm}{Corollary}\label{cor:area-inequality}
Let $f\:\spc{M}\to\spc{N}$ be a locally Lipschitz map between $n$-dimensional Riemannian spaces.
Then 
\[\int_{p\in A} |\jac_p f|\ge \vol[f(A)]\]
for any Borel subset $A\subset M$.

In particular, if $|\jac_p f|\le 1$ almost everywhere in $A$, then 
\[\vol A \ge \vol[f(A)].\]
\end{thm}

\parit{Proof.} Apply the area formula to the characteristic function of $A$.
\qeds

Suppose that $f\:\spc{M}\to\RR$ is a Lipschitz function defined on an $n$-dimensional Riemannian space $\spc{M}$.
Then by Rademacher's theorem, the differential $d_pf\:\T_p\spc{M}\to\RR$  and the gradient 
$\nabla_pf\in\T_p\spc{M}$ are defined at almost all $p\in \spc{M}$.

The identity in the following proposition is a partial case of the so-called \index{coarea formula}\emph{coarea formula}.
(The general coarea formula deals with the maps to the spaces of arbitrary dimension, not necessary $1$.)


\begin{thm}{Proposition}\label{prop:coarea}
Let $f\:\spc{M}\to\RR$ be a Lipschitz function defined on an $n$-dimensional Riemannian space $\spc{M}$.
Suppose that the level sets $L_x\df f^{-1}(x)$ are equipped with $(n-1)$-dimensional volume $\vol_{n-1}\z\df\tfrac{\omega_{n-1}}{2^{n-1}}\cdot \haus_{n-1}$.
Then for any Borel function $u\:\spc{M}\to \RR$ the following equality holds
\[\int_{p\in \spc{M}} u(p)\cdot |\nabla_pf|=\int_{-\infty}^{+\infty} \left(\,\int_{p\in L_x} u(p)\,\right)\cdot dx.\]
\end{thm}

The following corollary is a partial case of the so-called  \index{coarea inequality}\emph{coarea inequality};

\begin{thm}{Corollary}\label{cor:coarea}
Let $\spc{M}$, $f$, and $L_x$ be as in \ref{prop:coarea}.

Suppose that $f$ is 1-Lipschitz.
Then for any Borel subset $A\subset M$ we have
\[\vol_n A\ge \int_{x\in\RR} \vol_{n-1}[A\cap L_x]\cdot dx.\eqlbl{eq:coarea-inq}\]
\end{thm}

The right-hand side in \ref{eq:coarea-inq} is called \index{coarea}\emph{coarea of the restriction $f|_A$}. 


\parit{Instead of proof of \ref{prop:coarea} and \ref{cor:coarea}.}
If $\spc{M}$ is isometric to Euclidean space, then the statement follows from the standard coarea formula \cite[3.2.12]{federer}.
The reduction to the Euclidean space is done the same way as in the proof of the area formula.

To prove the corollary, choose $u$ to be the characteristic function of $A$ and apply the coarea formula.
\qeds


\section{Besicovitch inequality}

A closed connected region in a Riemannian manifold bounded by hypersurface will be called \index{Riemannian!manifold with boundary}\emph{Riemannian manifold with boundary}.
We always assume that the hypersurface can be realized locally as a graph of Lipschitz function in a suitable chart.
In this case one can define $g$-length, $g$-distance, and $g$-volume the same way as we did for usual Riemannian manifolds.

\begin{thm}{Exercise}\label{ex:compact-interior}
Suppose that $(M,g)$ is a compact Riemannian manifold with boundary. 
Observe that the interior $(M^\circ,g)$ of $(M,g)$ is a usual Riemannian manifold.
Show that the space of $(M,g)$ is isometric to the completion of the space of $(M^\circ,g)$.
\end{thm}
 

\begin{thm}{Theorem}\label{thm:besikovitch}
Let $g$ be a continuous metric tensor on a unit $n$-dimensional cube $\square$.
Suppose that the $g$-distances between the opposite faces of $\square$ are at least $1$; that is, any Lipschitz curve that connects opposite faces has $g$-length at least $1$.
Then \[\vol(\square, g)\ge 1.\]

\end{thm}

This is a partial case of the theorem proved by Abram Besicovitch \cite{besicovitch}.

\parit{Proof.}
We will consider the case $n=2$; the other cases are proved the same way.

\begin{wrapfigure}{r}{30mm}
\vskip-0mm
\centering
\includegraphics{mppics/pic-1320}
\end{wrapfigure}

Denote by $A$, $A'$, and $B$, $B'$ the opposite faces of the square~$\square$.
Consider two functions
\begin{align*}
f_A(x)&\df\min\{\,\distfun_A(x)_g,1\,\},
\\
f_B(x)&\df\min\{\,\distfun_B(x)_g,1\,\}.
\end{align*}
Let $\bm{f}\:\square\to\square$ be the map with coordinate functions $f_A$ and $f_B$;
that is, $\bm{f}(x)\df(f_A(x), f_B(x))$.

\begin{clm}{}\label{f:A->A}
The map $\bm{f}$ sends each face of $\square$ to itself.
\end{clm}


Indeed, 
\[x\in A \quad\Longrightarrow\quad \distfun_A(x)_g=0 \quad\Longrightarrow\quad f_A(x)=0 \quad\Longrightarrow\quad \bm{f}(x)\in A.\]
Similarly, if $x\in B$, then $\bm{f}(x)\in B$.
Further, 
\[x\in A'
\quad\Longrightarrow\quad 
\distfun_A(x)_g\ge 1 
\quad\Longrightarrow\quad 
f_A(x)=1 
\quad\Longrightarrow\quad 
\bm{f}(x)\in A'.\]
Similarly, if $x\in B'$, then $\bm{f}(x)\in B'$.

By \ref{f:A->A}, it follows 
\[\bm{f}_t(x)= t\cdot x + (1-t)\cdot \bm{f}(x)\]
defines a homotopy of maps of the pair of spaces $(\square,\partial \square)$ from $\bm{f}$ to the identity map;
that is, $(t,x)\mapsto \bm{f}_t(x)$ is a continuous map and if $x\in \partial \square$, then $\bm{f}_t(x)\in \partial \square$ for any $t\in [0,1]$.

It follows that $\deg\bm{f}=1$; that is, $\bm{f}$ sends the fundamental class of $(\square,\partial \square)$ to itself.%
\footnote{Here and further, we assume that homologies are taken with the coefficients in $\ZZ_2$, but you are welcome to play with other coefficients.}
In particular $\bm{f}$ is onto.

Suppose that Jacobian  matrix $\Jac_p\bm{f}$ of $\bm{f}$ is defined at $p\in \square$.
Choose an orthonormal frame in $\T_p$ with respect to $g$ and the standard frame in the target $\square$.
Observe that the differentials $d_pf_A$ and $d_pf_B$ written in these frames are the rows of $\Jac_p\bm{f}$.
Evidently $|d_pf_A|\le 1$ and $|d_pf_B|\le 1$.
Since the determinant of a matrix is the volume of the parallelepiped spanned on its rows, we get 
\[|\jac_p \bm{f}|\le |d_pf_A|\cdot|d_pf_B|\le 1.\]
Since $\bm{f}\:\square\to\square$ is a Lipschitz onto map, the area inequality (\ref{cor:area-inequality}) implies that 
\[\vol(\square,g)\ge \vol\square=1.\]
\qedsf

If the $g$-distances between the opposite sides are $d_1,\dots ,d_n$, then following the same lines  one get that 
$\vol (\square,g)\ge d_1\cdots d_n$.
Also note that in the proof we use topology of the $n$-cube only once, to show that the map $f$ has degree one.
Taking all this into account we get the following generalization of \ref{thm:besikovitch}:

\begin{thm}{Theorem}\label{thm:besikovitch+}
Let $(M,g)$ be an $n$-dimensional Riemannian manifold with coonected boundary $\partial M$.
Suppose that there is a degree 1 map $\partial M\to \partial\square$;
denote by $d_1,\dots, d_n$ the $g$-distances between the inverse images of pairs of opposite faces of $\square$ in $M$.
Then 
\[\vol(M,g)\ge d_1\cdots d_n.\]

\end{thm}

\begin{thm}{Exercise}\label{ex:besikovitch=}
Show that if equality holds in \ref{thm:besikovitch+},
then $(M,g)$ is isometric to the rectangle $[0,d_1]\times\dots\times[0, d_n]$.
\end{thm}



\begin{thm}{Exercise}\label{ex:hexagon}
Suppose $g$ is a metric tensor on a regular hexagon $\text{\rm\hexagon}$ such that $g$-distances between the opposite sides are at least $1$.
Is there a positive lower bound on $\area(\text{\rm\hexagon},g)$?
\end{thm}

\begin{thm}{Exercise}\label{ex:cylinder}
Let $g$ be a Riemannian metric on the cylinder $\mathbb{S}^1\z\times [0,1]$.
Suppose that 
\begin{itemize}
\item 
$g$-distance between pairs of points on the opposite boundary circles $\mathbb{S}^1\times\{0\}$ and $\mathbb{S}^1\times\{1\}$ is at least 1, and 
\item
any curve $\gamma$ in $\mathbb{S}^1\times [0,1]$ that is homotopic to $\mathbb{S}^1\times\{0\}$ has $g$-length at least $1$.
\end{itemize}

\begin{subthm}{ex:cylinder:besicovitch}
Use Besicovitch inequality to show that
\[\area(\mathbb{S}^1\times [0,1],g)\ge \tfrac12.\]

\end{subthm}

\begin{subthm}{ex:cylinder:coarea}
Modify the proof of Besicovitch inequality using coarea inequality (\ref{cor:coarea}) to prove the optimal bound  
\[\area(\mathbb{S}^1\times [0,1],g)\ge 1.\]
 
\end{subthm}

\end{thm}

\begin{thm}{Exercise}\label{ex:gadograph}

\begin{subthm}{ex:gadograph-besikovitch}
Generalize \ref{thm:besikovitch+} to noncontinuous metric tensor $g$ described the following way:
there are two Riemannian metric tensors $g_1$ and $g_2$ on $M$ and a subset $V\subset M$ bounded by a Lipschitz hypersurface $\Sigma$ such that 
$g=g_1$ at the points in $V$ and $g=g_2$ otherwise.
\end{subthm}



\begin{subthm}{ex:gadograph-gadograph}
Use part \ref{SHORT.ex:gadograph-besikovitch} to prove the following: 
Let $V$ be a compact set in the $n$-dimensional Euclidean space $\EE^n$ bounded by a Lipschitz hypersurface $\Sigma$.
Suppose $g$ is a Riemannian metric on $V$ such that 
\[\dist{p}{q}{g}\ge\dist{p}{q}{\EE^n}\]
for any two points $p,q\in \Sigma$.
Show that
\[\vol(V,g)\ge \vol(V)_{\EE^n}.\]
\end{subthm}

\end{thm}

\begin{thm}{Exercise}\label{ex:involution-of-sphere}
Suppose that sphere with Riemannian metric $(\mathbb{S}^2,g)$ admits an involution $\iota$ such that $\dist{x}{\iota(x)}{g}\ge 1$.

Show that 
\[\area(\mathbb{S}^2,g)\ge \tfrac1{1000}.\]
Try to show that 
\[\area(\mathbb{S}^2,g)\ge \tfrac12,
\quad \area(\mathbb{S}^2,g)\ge 1,
\quad\text{or}\quad\area(\mathbb{S}^2,g)\ge \tfrac4\pi\]

\end{thm}

\begin{thm}{Advanced exercise}\label{ex:involution-of-3sphere}
Construct a metric tensor $g$ on $\mathbb{S}^3$ such that (1) $\vol(\mathbb{S}^3,g)$ arbitrarily small and (2) there is an involution $\iota\:\mathbb{S}^3\z\to \mathbb{S}^3$ such that $\dist{x}{\iota(x)}{g}\ge 1$ for any $x\in \mathbb{S}^3$.
\end{thm}

\begin{thm}{Exercise}\label{ex:GH-vol}
Let $g_1,g_2,\dots$, and $g_\infty$ be metrics on a fixed compact manifold $M$.
Suppose that $\distfun_{g_n}$ uniformly converges to $\distfun_{g_\infty}$ as functions on $M\times M\to\RR$.
Show that 
\[\liminf_{n\to\infty}\vol(M,g_n)\ge \vol(M,g_\infty).\]

Show that the inequality might be strict.
\end{thm}

\section{Systolic inequality}

Let $\spc{M}$ be a compact Riemannian space.
The \index{systole}\emph{systole} of $\spc{M}$ (briefly $\sys\spc{M}$) is defined to be the least length of a noncontractible closed curve in $\spc{M}$.

Let $\Lambda$ be a class of closed $n$-dimensional Riemannian spaces.
We say that a \index{systolic inequality}\emph{systolic inequality} holds for $\Lambda$ if there is a constant $c$ such that 
\[\sys\spc{M}\le c\cdot \sqrt[n]{\vol\spc{M}}\]
for any $\spc{M}\in \Lambda$.

\begin{thm}{Exercise}\label{ex:sysT2}
Use \ref{thm:besikovitch} or \ref{ex:cylinder} to show that a systolic inequality holds for any Riemannian metric on the 2-torus $\TT^2$.
\end{thm}

\begin{thm}{Exercise}\label{ex:sysRP2}
Use \ref{thm:besikovitch} to show that a systolic inequality holds for any Riemannian metric on  the real projective plane $\RP^2$.
\end{thm}

\begin{thm}{Exercise}\label{ex:sysSg}
Use \ref{thm:besikovitch+} to show that systolic inequality holds for any Riemannian metric on any closed surfaces of positive genus.
\end{thm}

\begin{thm}{Exercise}\label{ex:sysS2xS1}
Show that no systolic inequality holds for Riemannian metrics on $\mathbb{S}^2\times\mathbb{S}^1$.
\end{thm}

In the following lecture we will show that systolic inequality holds for many manifolds, in particular for torus of arbitrary dimension.

\section{Generalization}\label{sec:hausdorff-measure}

The following proposition follows immediately from the definitions of Hausdorff measure (Section \ref{sec:vol-haus}).

\begin{thm}{Proposition}\label{prop:bilip-measure}
Let $\spc{X}$ and $\spc{Y}$ be metric spaces, $A\subset \spc{X}$
and
 $f\: \spc{X}\to \spc{Y}$ be a $\Lip$-Lipschitz map. 
Then 
\[\haus_\alpha [f(A)]\le \Lip^\alpha\cdot\haus_\alpha\, A\]
for any $\alpha$.
\end{thm}

The following exercise provides a weak analog of the Besicovitch inequality that works for arbitrary metrics.

\begin{thm}{Exercise}\label{ex:besikovitch++}
Let $M$ be manifold with boundary and $\rho$ is a semimetric on $M$.
Suppose $\partial M$ admits a degree 1 map to the surface of the $n$-dimensional cube $\square$;
denote by $d_1,\dots, d_n$ the $\rho$-distances between the inverse images of pairs of opposite faces of $\square$ in $M$.
Then 
\[\haus_n(M,\rho)\ge d_1\cdots d_n.\]
\end{thm}


Recall that in $n$-dimensional Riemannian spaces we have 
\[\tfrac{\omega_n}{2^n}\cdot\haus_n=\vol_n.\]
Note that $\tfrac{\omega_n}{2^n}<1$ if $n\ge 2$.
Therefore, the conclusion in \ref{ex:besikovitch++} is weaker than in \ref{thm:besikovitch+} (the assumptions are weaker as well).

One can redefine systolic inequality on $n$-dimensional manifolds using the Hausdorff measure $\haus_n$ instead of the volume.
It is straightforward to prove analogs of the exercises \ref{ex:sysT2}--\ref{ex:sysS2xS1} with this definition.

\begin{thm}{Exercise}\label{ex:2top-discs}
Suppose that two embedded $n$-disks $\Delta_1,\Delta_2$ in a metric space $\spc{X}$ have identical boundaries.
Assume that $\spc{X}$ is contractible and $\haus_{n+1}\spc{X}=0$.
Show that $\Delta_1=\Delta_2$.
\end{thm}

\section{Remarks}\label{sec:besicovitch-remarks}

The optimal constants in the systolic inequality are known only in the following three cases:
\begin{itemize}
\item For real projective plane $\RP^2$ the constant is $\sqrt{\pi/2}$ --- the equality holds for a quotient of a round sphere by isometric involution. The statement was proved by Pao Ming Pu \cite{pu}.\label{page:pu}
\item For torus $\TT^2$ the constant is $\sqrt{2}/\sqrt[4]{3}$ --- the equality holds for a flat torus obtained from a regular hexagon by identifying opposite sides; this is the so-called \index{Loewner's torus inequality}\emph{Loewner's torus inequality}.
\item For the Klein bottle $\RP^2\#\RP^2$  the constant is $\sqrt{\pi}/2^{3/4}$ --- the equality holds for a certain nonsmooth metric.
The statement was proved by Christophe Bavard \cite{bavard}.
\end{itemize}
The proofs of these results use the so-called {}\emph{uniformization theorem}   available in the 2-dimensional case only.
These proofs are beautiful, but they are too far from metric geometry.
A good survey on the subject is written by Christopher Croke and Mikhail Katz \cite{croke-katz}.

An analog of Exercise \ref{ex:GH-vol} with Hausdorff measure instead of volume does not hold for general metrics on a manifold.
In fact there is a nondecreasing sequence of metric tensors $g_n$ on $M$, such that (1) $\vol(M,g_n)<1$ for any $n$ and (2) $\distfun_{g_n}$ converges to a metric on $M$ with arbitrary large Hausdorff measure of any given dimension; such examples were constructed by Dmitri Burago, Sergei Ivanov, and David Shoenthal \cite{burago-ivanov-shoenthal}.

\chapter{Width and systole}

This lecture is based on the paper of Alexander Nabutovsky \cite{nabutovsky}.

\section{Partition of unity}

\begin{thm}{Proposition}\label{thm:part-unit}
 Let $\{V_i\}$ be a finite open covering of a compact metric space ${\spc{X}}$.
Then there are Lipschitz functions $\psi_i\:{\spc{X}}\z\to[0,1]$ such that (1) if $\psi_i(x)>0$, then $x\in V_i$ and (2) for any $x\in {\spc{X}}$ we have
$$\sum_i\psi_i(x)=1.$$

\end{thm}

A collection of functions $\{\psi_i\}$ that meets the conditions in \ref{thm:part-unit} is called 
a \index{partition of unity}\emph{partition of unity} subordinate to the covering $\{V_i\}$.

\parit{Proof.}
Denote by $\phi_i(x)$ the distance from $x$ to the complement of $V_i$;
that is,
$$\phi_i(x)=\distfun_{{\spc{X}}\setminus V_i}(x).$$
Note $\phi_i$ is $1$-Lipschitz
for any $i$
and $\phi_i(x)>0$ if and only if $x\in V_i$.
Since $\{V_i\}$ is a covering, we have that
$$\Phi(x)\df\sum_i\phi_i(x)>0\ \ \text{for any}\ \ x\in {\spc{X}}.$$
Since $\spc{X}$ is compact, $\Phi>\delta$ for some $\delta>0$.
It follows that $x\mapsto\tfrac1{\Phi(x)}$ is a bounded Lipschitz function. 

Set 
$$\psi_k(x)=\frac{\phi_k(x)}{\Phi(x)}.$$
Observe that by construction the functions $\psi_i$ meet the conditions in the proposition.
\qedsf

\section{Nerves}

Let $\{V_1,\dots,V_k\}$ be a finite open cover of a compact metric space $\spc{X}$.
Consider an abstract simplicial complex $\spc{N}$, with one vertex $v_i$ for each set $V_i$ such that a simplex with vertices $v_{i_1},\dots, v_{i_m}$ is included in $\spc{N}$ if 
the intersection $V_{i_1}\cap\dots\cap V_{i_m}$ is nonempty.
\begin{figure}[ht!]
\vskip-0mm
\centering
\includegraphics{mppics/pic-1402}
\end{figure}
The obtained simplicial complex $\spc{N}$ is called the \index{nerve}\emph{nerve} of the covering $\{V_i\}$.
Evidently $\spc{N}$ is a finite simplicial complex ---
it is a subcomplex of a simplex with the vertices $\{v_1,\dots,v_k\}$.

Note that the nerve $\spc{N}$ has dimension at most $n$ if and only if the covering $\{V_1,\dots,V_k\}$ has \index{multiplicity of covering}\emph{multiplicity} at most $n+1$;
that is, any point $x\in\spc{X}$ belongs to
at most $n+1$ sets of the covering.

Suppose $\{\psi_i\}$ is  
a partition of unity subordinate to the covering $\{V_1,\dots,V_k\}$.
Choose a point $x\in {\spc{X}}$.
Note that the set
$$\{v_{i_1},\dots,v_{i_n}\}=\set{v_i}{\psi_i(x)>0}$$
form vertices of a simplex in $\spc{N}$.
Therefore 
$$\bm{\psi}\:x\mapsto \psi_1(x)\cdot v_1+\psi_2(x)\cdot v_2+\dots+\psi_k(x)\cdot v_n.$$
describes a Lipschitz map from ${\spc{X}}$ to the nerve $\spc{N}$ of $\{V_i\}$.
In other words, $\bm{\psi}$ maps a point $x$ to the point in $\spc{N}$ with \index{barycentric coordinates}\emph{barycentric coordinates} $(\psi_1(x),\dots,\psi_k(x))$.

Recall that the \index{star}\emph{star} of a vertex $v_i$ (briefly $\Star_{v_i}$) is defined as the union of the interiors of all simplicies that have $v_i$ as a vertex.
Recall that $\psi_i(x)>0$ implies $x\in V_i$.
Therefore we get the following:

\begin{thm}{Proposition}\label{prop:space->nerve}
Let $\spc{N}$ be a nerve of an open covering $\{V_1,\z\dots,V_k\}$ of a compact metric space $\spc{X}$.
Denote by $v_i$ the vertex of $\spc{N}$ that corresponds to $V_i$.

Then there is a Lipschitz map $\bm{\psi}\:\spc{X}\to\spc{N}$ such that $\bm{\psi}(V_i)\z\subset\Star_{v_i}$ for every $i$.
\end{thm}


\section{Width}

Suppose $A$ is a subset of a metric space $\spc{X}$.
The radius of $A$ (briefly $\rad A$) is defined as the least upper bound on the values $R>0$ such that $\oBall(x,R)\supset A$ for some $x\in \spc{X}$.

\begin{thm}{Definition}\label{def:width}
Let $\spc{X}$ be a metric space.
The \index{width}\emph{$n$-th width} of $\spc{X}$ (briefly $\width_n\spc{X}$) is the least upper bound on values $R>0$ such that $\spc{X}$ admits a finite open covering $\{V_i\}$ with multiplicity at most $n+1$ and $\rad V_i< R$ for each $i$.
\end{thm}

\parbf{Remarks.}

\begin{itemize} 
\item Observe that 
\[\width_0\spc{X}\ge\width_1\spc{X}\ge\width_2\spc{X}\ge\dots\]
for any compact metric space $\spc{X}$.
Moreover, if $\spc{X}$ is connected, then 
\[\width_0\spc{X}=\rad\spc{X}.\]

\item Usually width is defined using diameter instead of radius, but the results differ at most twice.
Namely, if $r$ is the $n$-th radius-width and $d$ --- the $n$-th diameter-width, then 
$r\le d\le 2\cdot r$.

\item Note that \index{Lebesgue covering dimension}\emph{Lebesgue covering dimension} of $\spc{X}$ can be defined as the least number $n$ such that $\width_n\spc{X}=0$.

\item Another closely related notion is the so-called \index{macroscopic dimension}\emph{macroscopic dimension on scale $R$};
it is defined as the  least number $n$ such that $\width_n\spc{X}<R$.
\end{itemize}



\begin{thm}{Exercise}\label{ex:macrodimension}
Suppose $\spc{X}$ is a compact metric space such that any closed curve $\gamma$ in $\spc{X}$ can be contracted in its $R$-neighborhood.
Show that macroscopic dimension of $\spc{X}$ on scale $100\cdot R$ is at most 1.

What about quasiconverse? That is, suppose a simply connected compact metric space $\spc{X}$ has macroscopic dimension at most 1 on scale $R$, is it true that any closed curve $\gamma$ in $\spc{X}$ can be contracted in its $100\cdot R$-neighborhood?
\end{thm}


The following exercise gives a good reason for the choice of term \index{width}\emph{width}; it also can be used as an alternative definition.

\begin{thm}{Exercise}\label{ex:width=suprad(inv)}
Suppose $\spc{X}$ is a compact metric space.
Show that $\width_n\spc{X}<R$ if and only if there is a finite $n$-dimensional simplicial complex $\spc{N}$ and a continuous map $\bm{\psi}\:\spc{X}\to \spc{N}$
such that 
\[\rad[\bm{\psi}^{-1}(s)]<R\]
for any $s\in \spc{N}$.
\end{thm}

\section{Riemannian polyhedrons}

A \index{Riemannian!polyhedron}\emph{Riemannian polyhedron} is defined as a finite simplicial complex with a metric tensor on each simplex such that the restriction of the metric tensor to a subsimplex coincides with the metric on the subsimplex.

The {}\emph{dimension} of a Riemannian polyhedron is defined as the largest dimension in its triangulation.
For Riemannian polyhedrons one can define length of curves and volume the same way as for Riemannian manifolds.

The obtained metric space will be called \emph{Riemannian polyhedron} as well.
A \index{triangulation}\emph{triangulation} of Riemannian polyhedron  will always be assumed to have the above property on the metric tensor.

Further we will apply the notion of width only to compact Riemannian polyhedrons.
If $\spc{P}$ is an $n$-dimensional Riemannian polyhedron, then 
we suppose that
\[\width\spc{P}\df\width_{n-1}\spc{P}.\]


Suppose that $\spc{P}$ is an $n$-dimensional Riemannian polyhedron;
in this case we will use short cut $\vol$ for $\vol_n$.
Let us define \index{volume profile}\emph{volume profile} of $\spc{P}$ as a function 
returning largest volume of $r$-ball in~$\spc{P}$;
that is, the volume profile of $\spc{P}$ is a function $\VolPro_{\spc{P}}\:\RR_+\to\RR_+$ defined by 
\[\VolPro_{\spc{P}}(r)\df \sup\set{\vol \oBall(p,r)}{p\in\spc{P}}.\]
Note that 
$r\mapsto \VolPro_{\spc{P}}(r)$ is nondecreasing  and
\[\VolPro_{\spc{P}}(r)\le\vol\spc{P}\]
for any $r$.
Moreover, if $\spc{P}$ is connected, then the equality $\VolPro_{\spc{P}}(r)\z=\vol\spc{P}$ holds
for $r\ge \rad \spc{P}$.

Note that if $\spc{P}$ is a connected 1-dimensional Riemannian polyhedron, then 
\[\width\spc{P}=\width_0\spc{P}=\rad\spc{P}.\]

\begin{thm}{Exercise}\label{ex:1D-case}
Let $\spc{P}$ be a 1-dimensional Riemannian polyhedron.
Suppose that $\VolPro_{\spc{P}}(R)<R$ for some $R>0$.
Show that 
\[\width \spc{P}<R.\]
Try to show that $c=\tfrac 12$ is the optimal constant for which the following inequality holds: 
\[\width \spc{P}<c\cdot R.\]
\end{thm}

\section{Volume profile bounds width}

\begin{thm}{Theorem}\label{thm:width<volpro}
Let $\spc{P}$ be an $n$-dimensional Riemannian polyhedron. 
If the inequality 
\[R> n\cdot \sqrt[n]{\VolPro_{\spc{P}}(R)}\]
holds for some $R>0$, then 
\[\width\spc{P}\le  R.\]
\end{thm}

Since $\VolPro_{\spc{P}}(R)\le \vol\spc{P}$ for any $R>0$,
we get the following:

\begin{thm}{Corollary}\label{thm:width<vol}
For any $n$-dimensional Riemannian polyhedron $\spc{P}$, we have
\[\width\spc{P}\le n\cdot \sqrt[n]{\vol\spc{P}}.\]

\end{thm}

The proof of \ref{thm:width<volpro} will be given at the very end of this section,
after discussing {}\emph{separating polyhedrons}. 

Let us start three technical statements.
The first statement can be obtained by modifying a smoothing procedure for functions defined on Euclidean space. 

A function $f$ defined on a Riemannian polyhedron $\spc{P}$ is called \index{piecewise smooth}\emph{piecewise smooth} if there is a triangulation of $\spc{P}$ such that restriction of $f$ to every simplex is smooth.


\begin{thm}{Smoothing procedure}\label{smoothing-procedure}
Let $\spc{P}$ be a Riemannian polyhedron and $f\:\spc{P}\to \RR$ be a 1-Lipschitz function.
Then for any $\delta>0$ there is a piecewise smooth 1-Lipschitz function $\tilde f\:\spc{P}\to \RR$ such that 
\[|\tilde f(x)-f(x)|<\delta\]
for any $x\in  \spc{P}$.
\end{thm}

The following statement can be proved by applying the classical Sard's theorem to each simplex of a Riemannian polyhedron.

\begin{thm}{Sard's theorem}\label{sard}\index{Sard's theorem}
Let $\spc{P}$ be an $n$-dimensional Riemannian polyhedron and $f\:\spc{P}\to \RR$ be a piecewise smooth function.
Then for almost all values $a\in\RR$, the inverse image $f^{-1}\{a\}$  is a Riemannian polyhedron of dimension at most $n-1$ (we assume that $f^{-1}\{a\}$ is equipped with the induced length metric).
\end{thm}

The following statement can be proved by applying the coarea inequality (\ref{cor:coarea}) to the restriction of $f$ to each simplex of the polyhedron and summing up the results.

\begin{thm}{Coarea inequality}\index{coarea inequality}\label{poly-coarea}
Let $\spc{P}$ be an $n$-dimensional Riemannian polyhedron and $f\:\spc{P}\to \RR$ be a piecewise smooth 1-Lipschitz function.
Set $v\z=\vol_n (f^{-1}[r,R])$ and $a(t)=\vol_{n-1}(f^{-1}\{t\})$.
Then 
\[\int_r^Ra(t)\cdot dt\ge v .\]
In particular, there is a subset of positive measure $S\subset [r,R]$ such that the inequality 
\[a(t)\ge \frac v{R-r}\]
holds for any $t\in S$.
\end{thm}

\section*{Separating subpolyhedrons}

\begin{thm}{Definition}
Let $\spc{P}$ be an $n$-dimensional Riemannian polyhedron.
An $(n-1)$-dimensional subpolyhedron $\spc{Q}\subset\spc{P}$ is called \index{separating subpolyhedron}\emph{$R$-separating} if for each connected component $U$ of the complement $\spc{P}\setminus \spc{Q}$ we have 
\[\rad U<R.\]

\end{thm}



\begin{thm}{Lemma}\label{lem:separating}
Let $\spc{P}$ be an $n$-dimensional Riemannian polyhedron.
Then given $R>0$ and $\eps>0$ there is a $R$-separating subpolyhedron $\spc{Q}\subset\spc{P}$ such that for any $r_0<r_1\le R$ we have
\[\VolPro_{\spc{Q}}(r_0)< \tfrac1{r_1-r_0}\cdot \VolPro_{\spc{P}}(r_1)+\eps.\]

\end{thm}

The proof reminds the proof of the following statement about minimal surfaces: 
\textit{if a point $p$ lies on an compact area-minimizing surface $\Sigma$ and $\partial\Sigma \cap \oBall(p,r)=\emptyset$, then
\[\area(\Sigma\cap \oBall(p,r))\le \tfrac12\cdot \area\mathbb{S}^2\cdot r^2.\]
}


\parit{Proof.}
Choose a small $\delta>0$.
Applying the smoothing procedure (\ref{smoothing-procedure}), we can exchange each distance function $\distfun_p$ on $\spc{P}$ by $\delta$-close piecewise smooth 1-Lipschitz function, which will be denoted by $\widetilde \distfun_p$.

By Sard's theorem (\ref{sard}), for almost all values $c\z\in(r_0\z+\delta, r_1-\delta)$, the level set
\[\tilde S_c(p)=\set{x\in \spc{P}}{\widetilde \distfun_p(x)=c}\]
is a Riemannian polyhedron of dimension at most $n-1$.
Since $\delta$ is small, the coarea inequality (\ref{poly-coarea}) implies that $c$ can be chosen so that in addition the following inequality holds:
\begin{align*}
\vol_{n-1}\tilde S_c(p)&\le \tfrac1{r_1-r_0-2\cdot\delta}\cdot\vol_n[\oBall(p,r_1)]<
\\
&<\tfrac1{r_1-r_0}\cdot \VolPro_{\spc{P}}(r_1)+\tfrac\eps2.
\end{align*}

Suppose $\spc{Q}$ is an $R$-separating subpolyhedron in $\spc{P}$ with almost minimal volume;
say its volume is at most $\tfrac\eps2$-far from the greatest lower bound.
Note that cutting from $\spc{Q}$ everything inside $\tilde S_c(p)$ and adding $\tilde S_c(p)$ produces a $R$-separating subpolyhedron, say $\spc{Q}'$.%
\footnote{If $\dim\tilde S_c(p)<n-1$, then it might happen that $\dim\spc{Q}'<n-1$; so, by the definition, $\spc{Q}'$ is not separating.
It can be fixed by adding a tiny $(n-1)$-dimensional piece to $\spc{Q}'$.}

Since $\spc{Q}$ has almost minimal volume, we have
\[\vol_{n-1}[\spc{Q}\cap \oBall(p,r_0)_{\spc{P}}]-\tfrac\eps2\le \vol_{n-1}S_c(p).\]
Therefore 
\[\vol_{n-1}[\spc{Q}\cap \oBall(p,r_0)_{\spc{P}}]\le\tfrac1{r_1-r_0}\cdot \VolPro_{\spc{P}}(r_1)+\eps.
\eqlbl{eq:volQ<ProP}\]
Recall that $\spc{Q}$ is equipped with the induced length metric;
therefore $\dist{p}{q}{\spc{Q}}\ge \dist{p}{q}{\spc{P}}$ for any $p,q\in \spc{Q}$;
in particular, 
\[\oBall(p,r_0)_{\spc{Q}}\subset \spc{Q}\cap \oBall(p,r_0)_{\spc{P}}\]
for any $p\in \spc{Q}$ and $r_0\ge 0$.
Hence, \ref{eq:volQ<ProP} implies the lemma.
\qeds

\begin{thm}{Lemma}\label{lem:separating-width}
Let $\spc{Q}$ be an $R$-separating subpolyhedron in an $n$-dimensional Riemannian polyhedron $\spc{P}$.
Then 
\[\width\spc{Q}\le R
\quad\Longrightarrow\quad
\width\spc{P}\le R.\]
\end{thm}

\parit{Proof.}
Choose an open covering $\{V_1,\dots,V_k\}$ of $\spc{Q}$ as in the definition of width (\ref{def:width});
that is, it has multiplicity at most $n$ and $\rad V_i<R$ for any $i$. 

Note that $\{V_1,\dots,V_k\}$ can be converted into an open covering of
a small neighbourhood of $\spc{Q}$ in $\spc{P}$ without increasing the multiplicity.
This can be done by setting 
\[V_i'=\bigcup_{x\in V_i}\oBall(x,r_x),\]
where $r_x\df\tfrac1{10}\cdot\inf\set{\dist{x}{y}{}}{y\in \spc{Q}\setminus V_i}$.

By adding to  $\{V_i'\}$ all the components of $\spc{P}\setminus \spc{Q}$,
we increase the multiplicity by at most 1 and obtain a covering of $\spc{P}$.
The statement follows since $\dim \spc{P}= \dim \spc{Q}\z+1$.
\qeds

\section*{Proof assembling}

\parit{Proof of \ref{thm:width<volpro}.}
We apply induction on the dimension $n=\dim\spc{P}$.
The base case $n=1$ is given in \ref{ex:1D-case}.

Suppose that the  $(n-1)$-dimensional case is proved.
Consider an $n$-dimensional Riemannian polyhedron $\spc{P}$ and suppose
\[n\cdot \sqrt[n]{\VolPro\spc{P}(R)}< R\]
for some $R>0$.
Let $\spc{Q}$ be an $R$-separating subpolyhedron in $\spc{P}$ provided by \ref{lem:separating} for a small $\eps>0$.

Applying  \ref{lem:separating} for $r=\tfrac{n-1}n\cdot R$ and $R$, we have that 
\begin{align*}
\VolPro_\spc{Q}(r) &< \frac 1{R-r}\cdot \VolPro_\spc{P}(R)+\eps<
\\
&<\frac {n}{R}\cdot\left(\frac{R}{n}\right)^n=
\\
&=\left(\frac{r}{n-1}\right)^{n-1};
\end{align*}
that is, $(n-1)\cdot \sqrt[n-1]{\VolPro\spc{Q}(r)}< r$.
Since $\dim\spc{Q}= n-1$, by the induction hypothesis, we get that
\[\width\spc{Q}\le r<R.\]
It remains to apply \ref{lem:separating-width}.
\qeds





\section{Width bounds systole}

Recall that a topological space $K$ is called \index{aspherical space}\emph{aspherical} if any continuous map $\mathbb{S}^k\to K$ for $k\ge 2$ is null-homotopic.

\begin{thm}{Theorem}\label{thm:sys<width}
Suppose $\spc{M}$ is a compact aspherical $n$-dimensional Riemannian manifold.
Then 
\[\sys\spc{M}\le 6 \cdot \width \spc{M}.\]
\end{thm}

\begin{thm}{Lemma}\label{lem:aspherical-homotopy}
Let $K$ be an aspherical space and $\spc{W}$ a connected CW-complex.
Denote by $\spc{W}^k$ the k-skeleton of $\spc{W}$.
Then any continuous map $f\:\spc{W}^2\to K$ can be extended to a continuous map $\bar f\:\spc{W}\to K$

Moreover, if $p\in \spc{W}$ is a 0-cell and $q\in K$.
Then a continuous maps of pairs $\phi_0,\phi_1\:(\spc{W},p)\to(K,q)$ are homotopic if and only if $\phi_0$ and $\phi_1$ induce the same homomorphism on fundamental groups $\pi_1(\spc{W},p)\to\pi_1(K,q)$.
\end{thm}

\parit{Proof.}
Since $K$ is aspherical, any continuous map $\partial\mathbb{D}^n\to K$ for $n\ge 3$
is hull-homotopic;
that is, it can be extended to a map $\mathbb{D}^n\:\to K$.

It makes it possible to extend $f$ to $\spc{W}^3$, $\spc{W}^4$, and so on.
Therefore $f$ can be extended to whole $\spc{W}$.

The only-if part of the second part of lemma is trivial;
it remains to show the if part.

Sine $\spc{W}$ is connected, we can assume that $p$ forms the only 0-cell in $\spc{W}$;
otherwise, we can collapse a maximal subtree of the 1-skeleton in $\spc{W}$ to $p$.
Therefore, $\spc{W}^1$ is formed by loops that generate $\pi_1(\spc{W},p)$.

By assumption, the restrictions of $\phi_0$ and $\phi_1$ to $\spc{W}^1$ are homotopic.
In other words the homotopy $\Phi\:[0,1]\times \spc{W}$ is defined on the 2-skeleton of $[0,1]\times \spc{W}$.
It remains to apply the first part of the lemma to the product $[0,1]\times \spc{W}$.
\qeds



\begin{thm}{Lemma}\label{lem:sys-homotopy}
Suppose $\gamma_0,\gamma_1$ are two paths between points in a Riemannian space $\spc{M}$ such that $\dist{\gamma_0(t)}{\gamma_1(t)}{\spc{M}}<r$ for any $t\in[0,1]$.
Let $\alpha$ be a shortest path from $\gamma_0(0)$ to $\gamma_1(0)$ and $\beta$ be a shortest path from $\gamma_0(1)$ to $\gamma_1(1)$. 
If $2\cdot r<\sys\spc{M}$, then there is a homotopy $\gamma_t$ from
$\gamma_0$ to $\gamma_1$ such that $\alpha(t)\equiv \gamma_t(0)$ and $\beta(t)\equiv \gamma_t(1)$.
\end{thm}

\parit{Proof.}
Set $s=\sys\spc{M}$; 
since $2\cdot r<s$, we have that $\eps=\tfrac1{10}(s-2\cdot r)>0$.

\begin{wrapfigure}{o}{34mm}
\vskip-0mm
\centering
\includegraphics{mppics/pic-1405}
\end{wrapfigure}

Note that we can assume that $\gamma_0$ and $\gamma_1$ are rectifiable;
if not we can homotopy each into a broken geodesic line kipping the assumptions true. 

Choose a fine partition $0\z=t_0\z<t_1\z<\z\dots\z<t_n=1$.
Consider a sequence of shortest paths $\alpha_i$ from $\gamma_0(t_i)$ to $\gamma_1(t_i)$.
We can assume that $\alpha_0=\alpha$, $\alpha_n=\beta$, and each arc $\gamma_j|_{[t_{i-1},t_i]}$ has length smaller than $\eps$.
Therefore, every quadrilateral formed by concatenation  of $\alpha_{i-1}$, $\gamma_1|_{[t_{i-1},t_i]}$, reversed $\alpha_i$, and reversed arc $\gamma_0|_{[t_{i-1},t_i]}$ has length smaller than $s$.
It follows that this curve is contractible.
Applying this observation for each quadrilateral, we get the statement.
\qeds


\parit{Proof of \ref{thm:sys<width}.}
Let $\spc{N}$ be the nerve of a covering $\{V_i\}$ of $\spc{M}$ and $\bm{\psi}\:\spc{M}\to\spc{N}$ be the map provided by \ref{prop:space->nerve}.
As usual, we denote by $v_i$ the vertex of $\spc{N}$ that corresponds to $V_i$.
Observe that $\dim\spc{N}<n$;
therefore, $\bm{\psi}$ kills the fundamental class of $\spc{M}$.

Let us construct a continuous map  $f\:\spc{N}\to  \spc{M}$ such that
$f\circ\bm{\psi}$ is homotopic to the identity map on $\spc{M}$.
Note that once $f$ is constructed, the theorem is proved.
Indeed, since $\bm{\psi}$ kills the fundamental class $[\spc{M}]$ of $\spc{M}$, so does $f\circ\bm{\psi}$.
Therefore, $[\spc{M}]=0$ --- a contradiction.

Set $R=\width \spc{M}$ and $s=\sys\spc{M}$.
Assume we choose $\{V_i\}$ as in the definition of width (\ref{def:width}).
For each $i$ choose a point $p_i\in \spc{M}$ such that $V_i\subset \oBall(p_i,R)$.

Set $f(v_i)=p_i$ for each $i$.
It defines the map $f$ on the 0-skeleton $\spc{N}^0$ of the nerve $\spc{N}$.
Further, $f$ will be defined step by step on the skeletons $\spc{N}^1,\spc{N}^2, \dots$ of $\spc{N}$.

Let us map each edge $[v_iv_j]$ in $\spc{N}$ to a shortest path $[p_ip_j]$.
It defines $f$ on $\spc{N}^1$.
Note that image of each edge is shorter than $2\cdot R$.

Suppose $[v_iv_jv_k]$ is a triangle in $\spc{N}$.
Note that perimeter of the triangle $[p_ip_jp_k]$ can not exceed $6\cdot R$.
Since $6\cdot R<s$, the contour of $[p_ip_jp_k]$ is contractible.
Therefore, we can extend $f$ to each triangle of~$\spc{N}$.
It defines the map $f$ on $\spc{N}^2$.

Finally, since $\spc{M}$ is aspherical, by \ref{lem:aspherical-homotopy}, the map $f$ can be extended to $\spc{N}^3$, $\spc{N}^4$ and so on.

It remains to show that $f\circ\bm{\psi}$ is homotopic to the identity map.
Choose a CW structure on $\spc{M}$ with sufficiently small cells, so that each cell lies in one of $V_i$.
Note that $\bm{\psi}$ is homotopic to a map $\bm{\psi}_1$ that sends $\spc{M}^k$ to $\spc{N}^k$ for any $k$.
Moreover, we may assume that (1) if a 0-cell $x$ of $\spc{M}$ maps to a $v_i$, then $x\in V_i$ and (2) each 1-cell  of $\spc{M}$ maps to an edge or a vertex of $\spc{N}$.
Choose a 1-cell $e$ in $\spc{M}$; by the construction, $f\circ\bm{\psi}_1$ maps $e$ to a shortest path $[p_ip_j]$ and $e$ lies $\oBall(p_i,R)$.
Observe that $[p_ip_j]$ is shorter than $2\cdot R$.
It follows that the distance between points on $[p_ip_j]$ and $e$ can not exceed $3\cdot R$.
Choose a shortest path $\alpha_i$ from every 0 cell $x_i$  of $\spc{M}$ to $p_j=f\circ\bm{\psi}_1(x_i)$.
It defines a homotopy on $\spc{M}^0$.
Since $6\cdot R<s$, \ref{lem:sys-homotopy} implies that this homotopy can be extended to $\spc{M}^1$.
By \ref{lem:aspherical-homotopy}, it can be extended to whole $\spc{M}$.
\qeds

\begin{thm}{Exercise}\label{ex:sys<width}
Analyze the proof of \ref{thm:sys<width} and improve its inequality to 
 \[\sys\spc{M}\le 4 \cdot \width \spc{M}.\]
\end{thm}

\begin{thm}{Exercise}\label{ex:fillrad-inj}
Modify the proof of \ref{thm:sys<width} to prove the following:

Suppose that $\spc{M}$ is a closed $n$-dimensional Riemannian manifold with \index{injectivity radius}\emph{injectivity radius} at least $r$; that is, if $\dist{p}{q}{\spc{M}}<r$, then a shortest path $[pq]_{\spc{M}}$ is uniquely defined.
Show that
\[\width\spc{M}\ge \tfrac{r}{2\cdot(n+1)}.\]

Use \ref{thm:width<vol} to conclude that $\vol\spc{M}\ge \eps_n \cdot r^n$
for some $\eps_n>0$ that depends only on $n$.
\end{thm} 

The second statement in the exercise is a theorem of Marcel Berger~\cite{berger-n};
an inequality with optimal constant (with equality for round sphere) was obtained by Marcel Berger and Jerry Kazdan \cite{berger-kazdan}. 


\section{Essential manifolds}

To generalize \ref{thm:sys<width} further, we need the following definition.

\begin{thm}{Definition}\label{def:essential}
A closed manifold $M$ is called \index{essential manifold}\emph{essential} if it admits a continuous map $\iota\:M\to K$ to an aspherical CW-complex $K$ such that $\iota$ sends the fundamental class of $M$ to a nonzero homology class in $K$.
\end{thm}

Note that any closed aspherical manifold is essential --- in this case one can take $\iota$ to be the identity map on $M$.

The real projective space $\RP^n$ provides an interesting example of an essential manifold which is not aspherical.
Indeed, the infinite dimensional projective space $\RP^\infty$ is aspherical and for the natural embedding $\RP^n\hookrightarrow\RP^\infty$ the image $\RP^n$ does not bound in $\RP^\infty$.
The following exercise provides more examples of that type:

\begin{thm}{Exercise}\label{ex:connected-sum-essential}
Show that the connected sum of an essential manifold with any closed manifold is essential.
\end{thm}

\begin{thm}{Exercise}\label{ex:product-essential}
Show that the product of two essential manifolds is essential.
\end{thm}

Assume that the manifold $M$ is essential and $\iota \:M\to K$ as in the definition.
Following the proof of \ref{thm:sys<width}, we can homotope the map 
$f\circ\bm{\psi}\:M\to M$ to the identity on the 2-skeleton of $M$;
further since $K$ is aspherical, we can homotope the composition
$\iota\z\circ f\circ\bm{\psi}$ to  $\iota$. 
Existence of this extension implies that $\iota$ kills the fundamental class of $M$ --- a contradiction.
So, taking \ref{ex:sys<width} into account, we proved the following generalization of \ref{thm:sys<width}:

\begin{thm}{Theorem}\label{thm:sys<width++}
Suppose $\spc{M}$ is an essential Riemannian space.
Then 
\[\sys\spc{M}\le 4 \cdot \width \spc{M}.\]
\end{thm}

As a corollary from \ref{thm:sys<width++} and \ref{thm:width<vol} we get the so-called  Gromov's \index{systolic inequality}\emph{systolic inequality}:

\begin{thm}{Theorem}\label{thm:sys+}
Suppose $\spc{M}$ is an essential $n$-dimensional Riemannian space.
Then 
\[\sys\spc{M}\le 4 \cdot n\cdot \sqrt[n]{\vol\spc{M}}.\]
\end{thm}


\section{Remarks}

Theorem \ref{thm:sys+} was proved originally by Mikhael Gromov \cite{gromov-1983} with a worse constant.
The given proof is a result of a sequence of simplifications given by Larry Guth \cite{guth},
Panos Papasoglu \cite{papasoglu},
Alexander Nabutovsky and Roman Karasev \cite{nabutovsky}.

The calculations could be done better; namely we could get
\[\width\spc{P}\le c_n\cdot \sqrt[n]{\vol\spc{P}},\]
where
$c_n=\sqrt[n]{n!/2}= \tfrac ne+o(n)$ \cite{nabutovsky}.
As a result, we may get a stronger statement in \ref{thm:sys+}:
\[\sys\spc{M}\le 4 \cdot c_n\cdot \sqrt[n]{\vol\spc{M}}.\]

For any nonessential oriented manifold $M$ there is a metric with fixed volume and arbitrary small systole.
This statement is proved by Ivan Babenko \cite{babenko}.

A wide open conjecture says that for any $n$-dimensional essential manifold we have
\[\frac{\sys\spc{M}}{\sqrt[n]{\vol\spc{M}}}\le\frac{\sys\RP^n}{\sqrt[n]{\vol\RP^n}},\eqlbl{eq:RPn}\]
where we assume that the $n$-dimensional real projective space $\RP^n$ is equipped with a canonical metric.
In other words, the ratio in the right-hand side of \ref{eq:RPn} is the optimal constant in the Gromov's systolic inequality; this  ratio grows as $O(\sqrt n)$.
(The ratio for $n$-dimensional flat torus grows as $O(\sqrt n)$ as well.)

%\chapter{Volume bounds filling radius}

This chapter 
is devoted to a proof of \ref{thm:FillRad<vol};
that is, we will show that \emph{Riemannian manifolds with small volume have small filling radius}.
This theorem was proved originally by Mikhael Gromov \cite{gromov-1983}.
We follow closely a simplified proof given by Alexander Nabutovsky, which is based on a sequence of other simplifications and improvements; see \cite{nabutovsky} and the references there in.

\section{Nerves and partition of unity}

Let $\{V_1,\dots,V_k\}$ be a finite open cover of a compact metric space $\spc{X}$.
Consider the abstract simplicial complex $\spc{N}$, with one vertex $v_i$ for each set $V_i$ such that a simplex with vertexes $v_{i_1},\dots, v_{i_k}$ is included in $\spc{N}$ if 
the intersection $V_{i_1}\cap\dots\cap V_{i_m}$ is nonempty.
We obtain a simplicial complex $\spc{N}$ called the \index{nerve}\emph{nerve of the covering $\{V_i\}$}.

Note that $\spc{N}$ is a finite simplicial complex and it has dimension at most $n$ if and only if the covering $\{V_1,\dots,V_k\}$ has multiplicity is at most $n+1$;
that is, at most $n+1$ different sets $V_{i_1},\dots, V_{i_{n+1}}$ have a nonempty intersection.
The nerve $\spc{N}$ is a subcomplex of a simplex with the vertixes $v_1,\dots,v_k\}$.

\begin{thm}{Proposition}\label{thm:part-unit}
 Let $\{V_1,\dots,V_k\}$ is a finite open covering of a compact metric space ${\spc{X}}$.
Then there are Lipschitz functions $\psi_i\:{\spc{X}}\to[0,1]$ such that
if $\psi_i(x)>0$ then $x\in V_i$ and
$$\sum_i\psi_i(x)=1$$
for any $x\in {\spc{X}}$.
\end{thm}

A collection of functions $\psi_i$ with above properies is called 
a \emph{partition of unity subordinate to the open cover}\index{partition of unity} $\{V_1,\dots,V_k\}$.

\parit{Proof.}
Consider the functions $\phi_i\:{\spc{X}}\to\RR$ defined as
$$\phi_i(x)=\distfun_{{\spc{X}}\backslash V_i} x.$$
Note $\phi_i$ is $1$-Lipschitz
for any $i$
and $\phi_i(x)>0$ if and only if $x\in V_i$.
In particular, 
$$\sum_i\phi_i(x)>0\ \ \text{for any}\ \ x\in {\spc{X}}.$$

Set 
$$\psi_k(x)=\frac{\phi_k(x)}{\sum_i\phi_i(x)}.$$
It remains to note that by construction the functions $\psi_i$ meet the conditions in the proposition.
\qedsf


Note that in the above proof for any point $x\in {\spc{X}}$,
the set
$$\set{v_i}{\psi_i(x)>0}$$
describe vertexes of a simplices in the nerve.
Therefore 
$$\bm{\psi}\:x\mapsto \psi_1(x)\cdot v_1+\psi_2(x)\cdot v_2+\dots+\psi_k(x)\cdot v_n.$$
can be thought of as a Lipschitz map from ${\spc{X}}$ to the nerve $\spc{N}$ of $\{V_i\}$;
where the point $x$ is mapped to the point with barycentric coordinates $\psi_i(x)$.
In other words we proved the following:

\begin{thm}{Proposition}\label{prop:space->nerve}
Let $\spc{N}$ be a nerve of an open covering $\{V_1,\dots,V_k\}$ of a compact metric space $\spc{X}$.
Denote by $v_i$ the vertex of $\spc{N}$ that corresponds to $V_i$.

Then there is a Lipschitz map from $\bm{\psi}\:\spc{X}\to\spc{N}$ such that $\bm{\psi}(V_i)\z\subset\Star_{v_i}$ for every $i$.
\end{thm}


\section{Width}

Suppose $A$ is a subset of a metric space $\spc{X}$.
The radius of $A$ (briefly $\rad A$) is defined as the least upper bound on the values $R>0$ such that $\oBall(x,R)\supset A$ for some $x\in \spc{X}$.

\begin{thm}{Definition}\label{def:width}
Let $\spc{X}$ be a metric space.
The $n$-width of $\spc{X}$ (briefly $\width_n\spc{X}$) is defined as least upper bound on values $R>0$ such that $\spc{X}$ admits a finite open covering $\{V_i\}$ with multiplicity at most $n+1$ and $\rad V_i< R$ for any $i$.
\end{thm}

\parit{Remarks.}
\begin{itemize}
\item Observe that if $\spc{X}$ is connected, then 
\[\width_0\spc{X}=\rad\spc{X}.\]
\item 
Usually width is defined using diameter instead of radius, but the result differ at most twice.
Namely if $r$ is the radius width and $d$ --- the diameter width of the same dimension, then 
$r\le d\le 2\cdot r$.

\item The definition of width reminds the definition of Lebesgue covering dimension.
In fact one says that a space has \emph{macroscopic dimesion} $\le n$ on the space $R$ if it admits an open cover as in the definiton.
\end{itemize}

\begin{thm}{Exercise}
Suppose $\spc{X}$ be a simply connected metric space such that any closed curve $\gamma$ in $\spc{X}$ can be contracted in its $R$-neighborhood.
Show that $\spc{X}$ is has macroscopic dimension at most 1 on scale $100\cdot R$.

Proove a quasiconverse; that is, if a simply connected metric space $\spc{X}$ has macroscopic dimension at most 1 on scale $R$, then any closed curve $\gamma$ in $\spc{X}$ can be contracted in its $100\cdot R$-neighborhood.
\end{thm}


The following proposition provides an equivalent definition.

\begin{thm}{Proposition}\label{prop:width=suprad(inv)}
Suppose $\spc{X}$ is a compact metric space.
Then $\width_n\spc{X}<R$ if and only if there is a finite $n$-dimensional somplicial complex $\spc{S}$ and a continuous map $\bm{\psi}\:\spc{X}\to \spc{N}$
such that $\rad[\bm{\psi}^{-1}(s)]\z<R$
for any $s\in \spc{N}$.
\end{thm}

\parit{Proof; ``only if'' part.}
Suppose $\width_n\spc{X}<R$.
Consider a covering $\{V_1,\dots,V_k\}$ of $\spc{X}$ guaranteed by the definition of width.
Let $\spc{N}$ be its nerve and $\bm{\psi}\:\spc{X}\to \spc{N}$ be the map provided by \ref{prop:space->nerve}.

Note that if $x\in \spc{N}$ lies in a symplex with a vertex $v_i$,
then $\bm{\psi}^{-1}(x)\subset V_i$;
in particulr $\bm{\psi}^{-1}(x)$ can be covered by a ball of radius $R$ in $\spc{X}$.

\parit{``If'' part.}
Choose $x\in \spc{N}$.
Since the inverse image $\bm{\psi}^{-1}(x)$ is compact, $\bm{\psi}$ is continuous, and $\rad[\bm{\psi}^{-1}(x)]<R$,
here is a neighborhood $U\ni x$ such that the  $\rad[\bm{\psi}^{-1}(U)]<R$.

It follows that there is a finite cover $\{U_i\}$ of $\spc{N}$ such that $\bm{\psi}^{-1}(U_i)\subset\spc{X}$ has radius smaller than $R$ for each $i$.
Since $\spc{N}$ has dimension $n$, we can inscribe%
\footnote{Recall that a covering $\{W_i\}$ is inscribed in the covering $\{U_i\}$ if for every $W_i$ is a subset of some $U_j$.} 
in $\{U_i\}$ an finite open cover $\{W_i\}$ with multiplicity at most $n+1$.
It remains to observe that $V_i=\bm{\psi}(W_i)$ defines a finite open cover of $\spc{X}$ with radius less than $R$ and multiplicity at most $n+1$. 
\qeds

Further we will apply the notion of width to compact Riemannian polyhedrons;
If $n$ is the dimension of a compact Riemannian polyhedron $\spc{P}$, then 
we suppose that
\[\width\spc{P}\df\width_{n-1}\spc{P}.\]

\begin{thm}{Exercise}
Show that for any closed Riemannian manifold $\spc{M}$ we have
\[\FillRad \spc{M}\le 100\cdot \width\spc{M};\]
try to show that in fact
\[\FillRad \spc{M}\le \width\spc{M}.\]

\end{thm}




\section{Volume bounds width}

A \emph{Riemannian polyhedron} is defined as a finite connected simplicial complex with a metric tensor on each simplex such that the restriction of the metric on each simplex to a subsymplex coinsides with the metric on the subsmplex.
The dimension of Riemannian polyhedron is defined as the largest dimension it its triangulation.
For Riemannian polhedron one can define length of curves and volume the same way as for Riemannian manifolds.

Let $\spc{P}$ be a Riemnnian polyhedron of dimension $n$.
Let us define volume profile of $\spc{P}$ as a function $\VolPro_{\spc{P}}\:\RR_+\to\RR_+$ defined by 
\[\VolPro_{\spc{P}}(r)\df \sup\set{\vol \oBall(p,r)}{p\in\spc{P}}.\]
Note that $\VolPro_{\spc{P}}$ is a nondecreasing function and $\VolPro_{\spc{P}}(r)\z\to\vol\spc{P}$ as $r\to\infty$.

\begin{thm}{Exercise}
Suppose $\spc{M}$ be a 1-dimensional Riemannian polhedron.
Suppose $\VolPro_{\spc{P}}(r_0)<r_0$ for some $r_0>0$.
Show that 
\[\diam \spc{P}<r_0.\]
Note tha it is equivalent to $\width \spc{P}<r_0$.
\end{thm}


An $(n-1)$-dimensional subpolyhedron $\spc{Q}\subset\spc{P}$ is called $R$-separating if each
connected component of its complement $\spc{P}\backslash \spc{Q}$ can be covered by a metric ball of radius $R$.

\begin{thm}{Lemma}
Let $\spc{P}$ be an $n$-dimensional Riemannian polyhedron.
Then given $R>0$ and $\eps>0$ there is a $R$-separating subpolyhedron $\spc{Q}\subset\spc{P}$ such that for any $r_0<r_1\le R$ we have
\[\VolPro_{\spc{Q}}(r_0)< \tfrac1{r_1-r_0}\cdot \VolPro_{\spc{P}}(r_1)+\eps.\]

\end{thm}

\begin{thm}{Lemma}
Let $\spc{Q}$ be a $R$-separating subpolyhedron in an $n$-dimensional Riemannian polyhedron $\spc{P}$.
Suppose $\width\spc{Q}\le R$.
Then $\width\spc{P}\le R$
\end{thm}

\parit{Proof.}
Start with an open covering $\{V_1,\dots,V_k\}$ of $\spc{Q}$ of multiplicity $\le n$ with radiuses of the sets in the intrinsic metric $\le R$.
Convert $\{V_1,\dots,V_k\}$ into an an open covering of
a small neighbourhood of $\spc{Q}$ in $\spc{P}$ without increasing the multiplicity.
Finally, add all the components of $\spc{P}\backslash \spc{Q}$ to the covering;
it increases the multiplicity by 1.
\qeds

The following technical statement will be used without a proof.

\begin{thm}{Claim}
Let $\spc{P}$ be a Reimannian polyhedron and $f\:\spc{P}\to \RR$ be a 1-Lipschitz function.
Then for any $\eps>0$ there is a  1-Lipschitz function $f_\eps\:\spc{P}\to \RR$ that is smooth on each simplex of the triangulation and $\eps$-close to $f$.
\end{thm}

\begin{thm}{Sard's theorem}
Let $\spc{P}$ be an $n$-dimensional Reimannian polyhedron and $f\:\spc{P}\to \RR$ be a function that is smooth on each simplex.
Then for almost all values $a$ each component of the inverse image $f^{-1}(a)$ is a equipped with the induced metric is a Reimannian polyhedron.
\end{thm}


\begin{thm}{Coarea inequality}
Let $\spc{P}$ be an $n$-dimensional Reimannian polyhedron and $f\:\spc{P}\to \RR$ be a 1-Lipschitz function that is smooth on each simplex.
Then 
\[\vol_n (f^{-1}[a,b]) \le \int_a^b\vol_{n-1}[f^{-1}(x)]\cdot dx.\]
\end{thm}


%\chapter{Examples}



\section{On semicontinuity}

Recall that according to \ref{ex:GH-vol}, volume is semicontinuos on the space of Riemannian manifolds with respect to stable Gromov--Hausdorff convergence.
Analogous statement for $n$-dimensional Hausdorff measure on a $n$-dimensional manifolds does not hold.

\begin{thm}{Claim}
 
\end{thm}

First let us show that for any $\alpha>0$, the $\alpha$-dimensional Hausdorff measure is not semicontinuous in the space of all compact metric spaces.

Choose a decreasing sequence $\eps_n\to 0$.
Consider the space $\spc{C}$ of infinite binary sequences with distance between two sequences $\bm{a}=(a_0,a_1,\dots)$ and $\bm{b}=(b_0,b_1,\dots)$ defined by 
\[\dist{\bm{a}}{\bm{b}}{\spc{C}}=\eps_n,\]
where $n$ is the minimal index such that $a_n\ne b_n$.
Note that $\spc{C}$ is homeomorphic to the Cantor set and 
given $\alpha>0$,
the sequence $\eps_n$ can be chosen so that its $\alpha$-dimensional Hausdorff measure is infinite.

Note that $\spc{C}$ is a Hausdorff limit of its subsets $\spc{C}_n$ formed by sequences that constantly zero starting from $n$-th element.
The sets $\spc{C}_n$ is finite in particular its $\alpha$-dimensional Hausdorff measure vanish for $\alpha>0$.
This example shows that for any $\alpha>0$, the $\alpha$-dimensional Hausdorff measure is not semicontinuous in the space of all compact metric spaces.

An analogous example can be produced comapct length spaces.
To do this consider a metric binary rooted tree $\spc{T}$ in which edges connecting level $n-1$ to the level $n$ of length $\eps_{n-1}-\eps_n$.
Note that the completion $\bar{\spc{T}}$ of $\spc{T}$ has a subset (its crown) isometric to $\spc{C}$.
Note further that $\bar{\spc{T}}$ is a Hausdorff limit of its subsets $\spc{T}_n$ --- the subtrees up to level $n$.
Note that $\spc{T}_n$ is can be covered by a finite line segments, in particular it has finite $1$-dimensional Hausdorff measure and therefore vanishing $\alpha$-dimensional Hausdorff for any $\alpha>1$.
Since the limit $\bar{\spc{T}}$ contains $\spc{C}$, we can choose a sequence $\eps_n$ so that $\mu_\alpha\spc{C}$ is arbitrary large (or even infinite).
It shows that for any $\alpha>1$, the $\alpha$-dimensional Hausdorff measure is not semicontinuous in the space of all compact length spaces.

This construction can be modified further to obtain an increasing sequence of metric tensors $g_n$ on a disc $\DD$ such that (1) $\vol(\DD,g_n)<1$ for each $n$, (2) the induced metrics $\dist{*}{*}{g_n}$ converge to a metric $\rho$ on $\DD$, and given any Cantor space $\spc{C}$ as described above (3) there is a bilipschitz map $\spc{C}\to(\DD,\rho)$.
Note that the last condition implies that $\mu_2(\DD,rho)$ can be made arbitrary large, or infinite.
Therefore for any $\alpha\ge 2$, the $\alpha$-dimensional Hausdorff measure is not semicontinuous in the space of all compact length spaces homeomorphic to a manifold and equipped with stable convergence.

Now we want to extend nonsemicontinuity even further.
Note that the tree $\bar{\spc{T}}$ admits a length-preserving embedding to the Euclidean space; we may assume that all 



\section{Sub-Riemannian metrics}

Choose a metric space $\spc{X}$.
Note that the function $\alpha\mapsto \mu_\alpha(A)_\spc{X}$ is nondecreasing;
moreover there is a critical value $\alpha_0\in[0,\infty]$ such that $\mu_\alpha(A)_\spc{X}=0$ if $\alpha<\alpha_0$ and $\mu_\alpha(A)_\spc{X}=\infty$ if $\alpha>\alpha_0$.
This value is called \index{Hausdorff dimension}\emph{Hausdorff dimension} of $\spc{X}$, or briefly $\alpha_0=\dim_H\spc{X}$.

The following statement is classical, a proof can be found in .

\begin{thm}{Theorem}
The Hausdorff dimension of any metric space can not be smaller than its Lebesgue covering dimension.
In particular, if a metric space $\spc{X}$ is homeomorphic to an $n$-dimensional manifold, then $\dim_H\spc{X}\ge n$.
 
\end{thm}

Note that the construction described in the previous section can be used to produce a metric on manifold of dimension $n\ge 2$ with arbitrary Hausdorff dimension $\alpha\ge n$.

In this section we will discuss another interesting source of such examples.



%
\begin{thm}{Uniqueness of geodesics}\label{thm:cat-unique}
In a proper length $\CAT(0)$ space, pairs of points are joined by unique geodesics, and these geodesics depend continuously on their endpoint pairs.

Analogously, in a proper length $\CAT(1)$ space, pairs of points at distance less than $\pi$ are joined by unique geodesics, and these geodesics depend continuously on their endpoint pairs.
\end{thm}

\parit{Proof.} 
Given 4 points $p^1,p^2,q^1,q^2$ in a proper length $\CAT(0)$ space $\spc{U}$, 
consider two triangles $\trig{p^1}{q^1}{p^2}$ and $\trig{p^2}{q^2}{q^1}$.
Since both of these triangles are thin, we get 
\begin{align*}
\dist{\geodpath_{[p^1q^1]}(t)}{\geodpath_{[p^2q^1]}(t)}{\spc{U}}
&\le (1-t)\cdot \dist{p^1}{p^2}{\spc{U}},
\\
\dist{\geodpath_{[p^2q^1]}(t)}{\geodpath_{[p^2q^2]}(t)}{\spc{U}}
&\le t\cdot \dist{q^1}{q^2}{\spc{U}}.
\intertext{By the triangle inequality,}
\dist{\geodpath_{[p^1q^1]}(t)}{\geodpath_{[p^2q^2]}(t)}{\spc{U}}&\le \max\{\dist{p^1}{p^2}{\spc{U}},\dist{q^1}{q^2}{\spc{U}}\}.
\end{align*}

This implies continuity and uniqueness in the $\CAT(0)$ case.  
 
The $\CAT(1)$ case is done in essentially the same way.
\qeds

Adding the first two inequalities of the preceding proof gives the following:

\begin{thm}{Proposition}
Suppose $p^1,p^2,q^1,q^2$ are points in a proper length $\CAT(0)$ space~$\spc{U}$.
Then 
\[\dist{\geodpath_{[p^1q^1]}(t)}{\geodpath_{[p^2q^2]}(t)}{\spc{U}}\]
is a convex function.
\end{thm}

\begin{thm}{Corollary}\label{cor:dist-convex}
Let $K$ be a closed convex subset in a proper length $\CAT(0)$ space~$\spc{U}$.
Then $\dist{K}{}{}\:\spc{U}\to\RR$ is \index{convex function}\emph{convex};
that is, the function $t\mapsto\dist{K}{}{}\circ\gamma$ is convex for any geodesic $\gamma$ in $\spc{U}$.

In particular, $\dist{p}{}{}$ is convex for any point $p$ in~$\spc{U}$.
\end{thm}


\begin{thm}{Corollary}\label{cor:contractible-cat}
Any proper length $\CAT(0)$ space is contractible.

Analogously, any proper length $\CAT(1)$ space with diameter $<\pi$ is contractible.
\end{thm}

\parit{Proof.} Let $\spc{U}$ be a proper length $\CAT(0)$ space.
Fix a point $p\in \spc{U}$.

For each point $x$ consider the geodesic path $\gamma_x\:[0,1]\to \spc{U}$ from $p$ to~$x$.
Consider the one parameter family of maps 
$h_t\:x\mapsto \gamma_x(t)$ for $t\in [0,1]$.
By uniqueness of geodesics (\ref{thm:cat-unique}), the map 
$(t,x)\mapsto h_t(x)$ is continuous. The family $h_t$ is called a \index{geodesic homotopy}\emph{geodesic homotopy}.

It remains to note that $h_1(x)=x$ and $h_0(x)=p$ for any~$x$.

The proof of the $\CAT(1)$ case is identical.
\qeds

\begin{thm}{Proposition}\label{cor:loc-geod-are-min}
Suppose $\spc{U}$ is a proper length $\CAT(0)$ space.  
Then any local geodesic in $\spc{U}$ is a geodesic.

Analogously, if $\spc{U}$ is a proper length $\CAT(1)$ space, then any local geodesic in $\spc{U}$ which is shorter than $\pi$ is a geodesic.
\end{thm}

\begin{wrapfigure}{r}{21mm}
\begin{lpic}[t(-0mm),b(0mm),r(0mm),l(0mm)]{pics/local-geod(1)}
\lbl[t]{2.5,1;$\gamma(0)$}
\lbl[b]{10,14;$\gamma(a)$}
\lbl[t]{19,8;$\gamma(b)$}
\end{lpic}
\end{wrapfigure}

\parit{Proof.}
Suppose $\gamma\:[0,\ell]\to\spc{U}$ is a local geodesic that is not a geodesic.
Choose $a$ to be the maximal value 
such that $\gamma$ is a geodesic on $[0,a]$.
Further choose $b>a$ so that $\gamma$ is a geodesic on $[a,b]$.

Since the triangle $\trig{\gamma(0)}{\gamma(a)}{\gamma(b)}$ is thin and 
$\dist{\gamma(0)}{\gamma(b)}{}<b$ we have
\[\dist{\gamma(a-\eps)}{\gamma(a+\eps)}{}<2\cdot\eps\]
for all small~$\eps>0$.
That is, $\gamma$ is not length-minimizing on the interval $[a-\eps,a+\eps]$ for any $\eps>0$,
a contradiction.

The spherical case is done in the same way.
\qeds


\begin{thm}{Exercise}\label{ex:geod-CBA}
Assume $\spc{U}$ is a proper length $\CAT(\kappa)$ space
 with extendable geodesics;
that is, any geodesic is an arc in a local geodesic $\RR\to \spc{U}$.

Show that the space of geodesic directions at any point in $\spc{U}$ is complete.

Does the statement remain true if $\spc{U}$ is complete, but not required to be proper?
\end{thm}

Now let us formulate the main result of this section.


\begin{wrapfigure}[6]{r}{28mm}
\begin{lpic}[t(-4mm),b(6mm),r(0mm),l(0mm)]{pics/lem_alex1(1)}
\lbl[lb]{10,23;$y$}
\lbl[rt]{1.5,.5;$p$}
\lbl[bl]{25,7.5;$x$}
\lbl[lb]{17,15;$z$}
\end{lpic}
\end{wrapfigure}

\begin{thm}{Inheritance lemma}
\label{lem:inherit-angle} 
Assume that a triangle $\trig p x y$ 
in a metric space is \index{decomposed triangle}\emph{decomposed} 
into two triangles $\trig p x z$ and $\trig p y z$;
that is, $\trig p x z$ and $\trig p y z$ have a common side $[p z]$, and the sides $[x z]$ and $[z y]$ together form the side $[x y]$ of $\trig p x y$.

If both triangles $\trig p x z$ and $\trig p y z$ are thin, 
then the triangle $\trig p x y$ is also thin.

Analogously, if $\trig p x y$ has perimeter $<2\cdot\pi$ and both triangles $\trig p x z$ and $\trig p y z$ are spherically thin, then triangle $\trig p x y$ is spherically thin.
\end{thm} 


\begin{wrapfigure}{r}{32mm}
\begin{lpic}[t(-4mm),b(0mm),r(0mm),l(0mm)]{pics/cat-monoton-ineq(1)}
\lbl[b]{14,23;$\dot z$}
\lbl[t]{10,.5;$\dot p$}
\lbl[r]{1,14;$\dot x$}
\lbl[l]{30.5,14;$\dot y$}
\lbl[tl]{13,13;$\dot w$}
\end{lpic}
\end{wrapfigure}

\parit{Proof.}
Construct  the model triangles $\trig{\dot p}{\dot x}{\dot z}\z=\modtrig(p x z)_{\EE^2}$ 
and $\trig {\dot p} {\dot y} {\dot z}\z=\modtrig(p y z)_{\EE^2}$ so that $\dot x$ and $\dot y$ lie on opposite sides of $[\dot p\dot z]$.

Let us show that 
\[\angk{z}{p}{x}+\angk{z}{p}{y}
\ge
\pi.
\eqlbl{eq:<+<>=pi}\]
Suppose the contrary, that is
\[\angk{z}{p}{x}+\angk{z}{p}{y}
<
\pi.\]
Then for some point $\dot w\in[\dot p\dot z]$, we have \[\dist{\dot x}{\dot w}{}+\dist{\dot w}{\dot y}{}
<
\dist{\dot x}{\dot z}{}+\dist{\dot z}{\dot y}{}=\dist{x}{y}{}.\]
Let $w\in[p z]$ correspond to $\dot w$; that is, $\dist{z}{w}{}=\dist{\dot z}{\dot w}{}$. 
Since $\trig p x z$ and $\trig p y z$ are thin, we have 
\[\dist{x}{w}{}+\dist{w}{y}{}<\dist{x}{y}{},\]
contradicting the triangle inequality. 

Denote by $\dot D$ the union of two solid triangles $\trig {\dot p}{\dot x}{\dot z}$ and $\trig {\dot p} {\dot y} {\dot z}$.
Further, denote by $\tilde D$ the solid triangle $\trig{\tilde  p}{\tilde  x}{\tilde  y}=\modtrig(p x y)_{\EE^2}$.
By \ref{eq:<+<>=pi}, there is a short map $F\:\tilde D\to \dot D$ that sends 
\begin{align*}
\tilde p&\mapsto \dot p,
&
\tilde x&\mapsto \dot x,
&
\tilde z&\mapsto \dot z,
&
\tilde y&\mapsto \dot y.
\end{align*}
\qedsf

\begin{thm}{Exercise}\label{ex:short-map}
Use Alexandrov's lemma (\ref{lem:alex}) to prove the last statement. 
\end{thm}


By assumption, the natural maps $\trig {\dot p} {\dot x} {\dot z}\to\trig p x z$ and $\trig {\dot p} {\dot y} {\dot z}\to\trig p y z$ are short.  
By composition,  the natural map from $\trig{\tilde  p}{\tilde  x}{\tilde  y}$ to $\trig p y z$ is short, as claimed.

The spherical case is done along the same lines.
\qeds

\begin{thm}{Exercise}\label{ex:convex-balls}
Show that any ball in a proper length $\CAT(0)$ space is a convex set.

Analogously, show that any ball of radius $R<\tfrac\pi2$ in a proper length $\CAT(1)$ space  is a convex set.
\end{thm}

Recall that a set $A$ in a metric space $\spc{U}$ is called locally convex if for any point $p\in A$ there is an open neighborhood $\spc{U}\ni p$ such that any geodesic in $\spc{U}$ with  ends in $A$ lies in~$A$. 

\begin{thm}{Exercise}\label{ex:locally-convex}
Let $\spc{U}$ be a proper length $\CAT(0)$ space.
Show that any closed, connected, locally convex set in $\spc{U}$ is convex.
\end{thm}

\begin{thm}{Exercise}\label{ex:closest-point}
Let  $\spc{U}$ be a proper length $\CAT(0)$ space 
and $K\subset \spc{U}$ be a closed convex set.
Show that: 

\begin{subthm}{ex:closest-point:a}
For each point $p\in \spc{U}$ there is unique point $p^*\in K$ that minimizes the distance $\dist{p}{p^*}{}$.
\end{subthm}

\begin{subthm}{}
The closest-point projection $p\mapsto p^*$ defined by (\ref{SHORT.ex:closest-point:a}) is short. 
\end{subthm}

\end{thm}




















\begin{thm}{Advanced exercise}\label{ex:urysohn-contractable}
 Show that the space $\spc{U}$ is contactable.
\end{thm}


\parbf{Advanced exercise~\ref{ex:urysohn-contractable}.}
Note that points in the space $\spc{X}_\infty$ constructed in the proof of \ref{prop:univeral-separable} can be multiplied number $t\in [0,1]$ --- simply multiply each function by factor $t$.
That defines a map 
\[\lambda_t\:\spc{X}_\infty\to \spc{X}_\infty\]
that scales all distances by factor $t$.
The map $\lambda_t$ can be extended to the completion of $\spc{X}_\infty$, which is isometic to $\spc{U}_d$ (or $\spc{U}$).

Observe that 
the map $\lambda_1$ is the identity  and $\lambda_0$ maps whole space to a single point, say $x_0$ --- that is the only point of $\spc{X}_0$.
Further note that the map $(t,p)\mapsto \lambda_t(p)$ is continuous ---  in particular $\spc{U}_d$ and $\spc{U}$ are contractible.\qeds

Source: \cite[(d) on page 82]{gromov-2007}.

Observe that for any point $p\in \spc{U}_d$ the curve $t\mapsto \lambda_t(p)$ is a geodesic path from $p$ to $x_0$.








Note that $\spc{M}$ --- the space of compact metric spaces can be treated as a space of compact subsets in $\spc{U}$ up to congruence.
Namely two subsets $A$ and $A'$ are called \emph{congruent} (briefly $A\cong A'$) if there is isometry of the ambient space $\spc{U}$ that maps $A$ to $A'$.
Let us define distance between congruence classes of two compact subsets $A$ and $B$ as 
\[\inf\set{\dist{A'}{B}{\spc{H}(\spc{U})}}{A'\cong A}.\]


By \ref{prop:sep-in-urys}, any compact metric spaces $\spc{K}$ admits a distance preserving map $f\:\spc{K}\to\spc{U}$.
Moreover by \ref{thm:compact-homogeneous} any two such maps $f_1$ and $f_2$ differ by isometry of $\spc{U}$;
that is, there is an isometry $\iota\:\spc{U}\to\spc{U}$ such that $f_2=\iota\circ f_1$.
In particular $f_1(\spc{K})\cong f_2(\spc{K})$.










\section{Ultratangent space} 

Recall that we assume that $\omega$ is a once for all fixed choice of a nonprinciple ultrafilter.

For a metric space $\spc{X}$ and a positive real number $\lambda$,
we will denote by $\lambda\cdot\spc{X}$ its \emph{$\lambda$-blowup}\index{blowup},
which is a metric space with the same underlying set as $\spc{X}$ and the metric multiplied by $\lambda$.
The tautological bijection $\spc{X}\to \lambda\cdot\spc{X}$ will be denoted as $x\mapsto x^\lambda$, 
so 
\[\dist{x^\lambda}{y^\lambda}{}
=
\lambda\cdot\dist[{{}}]{x}{y}{}\] 
for any $x,y\in \spc{X}$.

The $\omega$-blowup $\omega\cdot\spc{X}$ of $\spc{X}$ is defined as the $\omega$-limit
of the $n$-blowups $n\cdot\spc{X}$; that is,
\[\omega\cdot\spc{X}
\df
\lim_{n\to\omega} n\cdot\spc{X}.\]

Given a point $x\in \spc{X}$ we can consider the sequence $x^n\in n\cdot\spc{X}$;
it corresponds to a point $x^\omega\in \omega\cdot\spc{X}$.
Note that if $x\ne y$, then 
\[\dist{x^\omega}{y^\omega}{\omega\cdot\spc{X}}=\infty;\]
that is, 
$x^\omega$ and $y^\omega$ 
belong to different metric components of $\omega\cdot\spc{X}$.

The metric component of $x^\omega$ in $\omega\cdot\spc{X}$ is called ultratangent space of $\spc{X}$ at $x$ and it is denoted as $\T^\omega_x\spc{X}$.

Equivalently, ultratangent space $\T^\omega_x\spc{X}$ can be defined the following way.
Consider all the sequences of points $x_n\in \spc{X}$ such that
the sequence $\ell_n=n\cdot\dist{x}{x_n}{\spc{X}}$ is bounded.
Define the pseudodistance between two such sequences as 
\[\dist{(x_n)}{(y_n)}{}
=
\lim_{n\to\omega}n\cdot\dist{x_n}{y_n}{\spc{X}}.\]
Then $\T^\omega_x\spc{X}$ is the corresponding metric space.

Tangent space as well as ultratangent space, 
generalize the notion of tangent space of Riemannian manifold.
In the simplest cases these two notions define the same space.
In general, they are different and both useful ---
often lack of a property in one is compensated by the other.

It is clear from the definition that tangent space has cone structure.
On the other hand, in general, ultratangent space does not have a cone structure; 
the Hilbert's cube $\prod_{n=1}^\infty[0,2^{-n}]$ is an example --- it is $\Alex{0}$ as well as $\CAT{0}$.

The next theorem shows that the tangent space $\T_p$ can be (and often will be) considered as a subset of  $\T^\omega_p$.

\begin{thm}{Theorem}\label{thm:tangent-ultratangent}
\label{thm:T-in-T^w} 
Let $\spc{X}$ be a metric space with defined angles.
Then for any $p\in \spc{L}$, there is an distance preserving map 
\[\iota:\T_p\hookrightarrow \T^\omega_p\] 
such that for any geodesic $\gamma$ starting at $p$
we have 
\[\gamma^+(0)\mapsto \lim_{n\to\omega}[\gamma(\tfrac1n)]^n.\]

\end{thm}

\parit{Proof.}
Given $v\in \T'_p$ 
choose a geodesic $\gamma$ that starts at $p$ such that $\gamma^+(0)\z=v$.
Set $v^n=[\gamma(\tfrac1n)]^n\in n\cdot \spc{X}$ and 
\[v^\omega=\lim_{n\to\omega}v^n.\]

Note that the value $v^\omega\in\T^\omega_p$ does not depend on choice of $\gamma$;
that is, if $\gamma_1$ is an other geodesic starting at $p$ such that $\gamma_1^+(0)=v$,
then 
\[\lim_{n\to\omega}v^n=\lim_{n\to\omega}v_1^n,\]
where $v_1^n=[\gamma_1(\tfrac1n)]^n\in n\cdot \spc{X}$.
The latter follows since
\[\dist{\gamma(t)}{\gamma_1(t)}{\spc{X}}=o(t)\]
and therefore $\dist{v^n}{v_1^n}{n\cdot \spc{X}}\to 0$ s $n\to\infty$.



Set $\iota(v)=v^\omega$.
Since angles between geodesics in $\spc{X}$ are defined, for any $v,w\in \T_p'$ we have
$n\cdot\dist[{{}}]{v_n}{w_n}{}\to\dist{v}{w}{}$.
Thus $\dist{v_\omega}{w_\omega}{}=\dist{v}{w}{}$; that is, $\iota$ is a global isometry of $\T_p'$.

Since $\T_p'$ is dense in $\T_p$,
we can extend $\iota$ to a global isometry $\T_p\to \T^\omega_p$.
\qeds

{\sloppy

\section[Gromov--Hausdorff and ultralimits]{Gromov--Hausdorff convergence and ultralimits}

}

\begin{thm}{Theorem}\label{thm:ultra-GH}
Assume $\spc{X}_n$ is a sequence of complete spaces. 
Let $\spc{X}_n\to \spc{X}_\omega$ as $n\to\omega$,
and $\spc{Y}_n\subset \spc{X}_n$ 
be a sequence of subsets such that $\spc{Y}_n\GHto\spc{Y}_\infty$. 
Then there is a distance preserving map 
$\iota:\spc{Y}_\infty\to \spc{X}_\omega$.

Moreover:

\begin{subthm}{thm:ultra-GH:a}
If $\spc{X}_n\GHto \spc{X}_\infty$ 
and $\spc{X}_\infty$ is compact, then 
$\spc{X}_\infty$ is isometric to $\spc{X}_\omega$.
\end{subthm}

\begin{subthm}{thm:ultra-GH:b}
If $\spc{X}_n\GHto \spc{X}_\infty$ 
and $\spc{X}_\infty$ is proper, then 
$\spc{X}_\infty$ is isometric to a metric component of $\spc{X}_\omega$.
\end{subthm}

\end{thm}

\parit{Proof.} 
For each point $y_\infty\in \spc{Y}_\infty$ 
choose a lifting $y_n\in \spc{Y}_n$.
Pass to the $\omega$-limit $y_\omega\in \spc{X}_\omega$ of $(y_n)$.
Clearly for any $y_\infty,z_\infty\in \spc{Y}_\infty$, 
we have 
\[\dist{y_\infty}{z_\infty}{\spc{Y}_\infty}=\dist{y_\omega}{z_\omega}{\spc{X}_\omega};\] 
that is, the map $y_\infty\mapsto y_\omega$ gives a distance preserving map $\iota:\spc{Y}_\infty\to \spc{X}_\omega$. 


\parit{(\ref{SHORT.thm:ultra-GH:a})$+$(\ref{SHORT.thm:ultra-GH:b}).}
Fix $x_\omega\in \spc{X}_\omega$.
Choose a sequence $x_n\in \spc{X}_n$ 
such that $x_n\to x_\omega$ as $n\to\omega$. 

Denote by $\bm{X}=\spc{X}_\infty\sqcup\spc{X}_1\sqcup\spc{X}_2\sqcup\dots$ the common space for the convergence $\spc{X}_n\GHto \spc{X}_\infty$;
as in the definition of Gromov--Hausdorff convergence.
Consider the sequence $(x_n)$ 
as a sequence of points in~$\bm{X}$.

If the $\omega$-limit $x_\infty$ of $(x_n)$ exists, 
it must lie in $\spc{X}_\infty$. 

The point $x_\infty$, if defined, does not depend on the choice of $(x_n)$.
Indeed, if $y_n\in\spc{X}_n$ is an other sequence such that $y_n\to x_\omega$ as $n\to\omega$, then 
\[
\dist{y_\infty}{x_\infty}{}=\lim_{n\to\omega}\dist{y_n}{x_n}{}=0;
\]
that is, $x_\infty=y_\infty$.


In this way we obtain a map $\nu\:x_\omega\to x_\infty$;
it is defined on a subset of $\Dom\nu \subset\spc{X}_\omega$.
By construction of $\iota$, 
we get  $\iota\circ\nu(x_\omega)=x_\omega$ for any $x_\omega\in \Dom\nu$.

Finally note that if $\spc{X}_\infty$ is compact, then $\nu$ is defined on all of $\spc{X}_\omega$;
this proves (\ref{SHORT.thm:ultra-GH:a}).

If $\spc{X}_\infty$ is proper, choose any point $z_\infty\in \spc{X}_\infty$
and set $z_\omega=\iota(z_\infty)$.
For any point $x_\omega\in \spc{X}_\omega$ at finite distance from $z_\omega$,
for the sequence $x_n$ 
as above we have that $\dist{z_n}{x_n}{}$ is bounded for $\omega$-almost all $n$.
Since $\spc{X}_\infty$ is proper, $\nu(x_\omega)$ is defined;
in other words $\nu$ is defined on the metric component of $z_\omega$.
Hence (\ref{SHORT.thm:ultra-GH:b}) follows.
\qeds

\begin{thm}{Corollary} 
\label{cor:ulara-geod}
The $\omega$-limit of a sequence of complete length spaces is geodesic.
\end{thm}

\parit{Proof.} Given two points $x_\omega,y_\omega\in \spc{X}_\omega$, find two bounded sequences of points $x_n,y_n\in \spc{X}_n$, $x_n\to x_\omega$, $y_n\to y_\omega$ as $n\to\omega$.
Consider a sequence of paths  $\gamma_n\:[0,1]\to \spc{X}_n$ from $_n$ to $y_n$
 such that 
\[\length\gamma_n\le \dist{x_n}{y_n}{}+\tfrac{1}{n}.\]
Apply Theorem~\ref{thm:ultra-GH} 
for the images $\spc{Y}_n=\gamma_n([0,1])\subset \spc{X}_n$.
\qeds

\section{Ultralimits of sets}

Let $\spc{X}_n$ be a sequence of metric spaces and $\spc{X}_n\to \spc{X}_\omega$
as $n\to \omega$.

For a sequence of sets $\Omega_n\subset \spc{X}_n$,
consider the maximal set $\Omega_\omega\subset \spc{X}_\omega$ such that 
for any $x_\omega\in\Omega_\omega$ and any sequence $x_n\in\spc{X}_n$ such that $x_n\to x_\omega$ as $n\to \omega$, we have $x_n\in\Omega_n$ for $\omega$-almost all $n$.

The set $\Omega_\omega$ is called the  \emph{open $\omega$-limit} of $\Omega_n$;
we could also write  $\Omega_n\to \Omega_\omega$ as $n\to\omega$ or $\Omega_\omega=\lim_{n\to\omega}\Omega_n$. 

{\sloppy

Applying Observation~\ref{obs:ultralimit-is-complete} to the sequence of complements $\spc{X}_n\backslash \Omega_n$, we see that $\Omega_\omega$ is open for any sequence $\Omega_n$.
The definition can be applied for arbitrary sequences of sets, but  
open $\omega$-convergence  will be applied here only for sequences of open sets.

}

\section{Ultralimits of functions}

Recall that a family of submaps between metric spaces $\{f_\alpha\: \spc{X}\to\spc{Y}\}_{\alpha\in\mathcal A}$ is called \emph{equicontinuous} if for any $\eps>0$ there is $\delta>0$ such that for any $p,q\in\spc{X}$ with $\dist{p}{q}{}<\delta$ and any $\alpha\in\mathcal A$ it holds that $\dist{f(p)}{f(q)}{}<\eps$.

Let $f_n\:\spc{X}_n\to\RR$ be a sequence of subfunctions.

Set $\Omega_n=\Dom f_n$.
Consider the open $\omega$-limit set $\Omega_\omega\subset \spc{X}_\omega$ of $\Omega_n$.

Assume there is a subfunction $f_\omega\:\spc{X}_\omega\to\RR$
that satisfies the following conditions: 
(1) $\Dom f_\omega=\Omega_\omega$, (2) if $x_n\to x_\omega\in \Omega_\omega$ for a sequence of points $x_n\in\spc{X}_n$, then $f_n(x_n)\to f_\omega(x_\omega)$ as $n\to\omega$.
In this case 
the subfunction $f_\omega\:\spc{X}_\omega\to\RR$ 
is said to be the 
$\omega$-limit of $f_n\:\spc{X}_n\to\RR$.

The following lemma gives a mild condition on a sequence of functions $f_n$
guaranteeing the existence of its $\omega$-limit.

\begin{thm}{Lemma}
Let $\spc{X}_n$ be a sequence of metric spaces
and $f_n\:\spc{X}_n\to\RR$ be a sequence of subfunctions.

Assume for any positive integer $k$, there is an open set $\Omega_n(k)\subset \Dom f_n$
such that the restrictions $f_n|_{\Omega_n(k)}$ are uniformly bounded and continuous
and the open $\omega$-limit of $\Omega_n(n)$ coincides with the open $\omega$-limit of $\Dom f_n$.
Then the $\omega$-limit of $f_n$ is defined.

In particular, if the $f_n$ are uniformly bounded and continuous, then the $\omega$-limit is defined.
\end{thm}

The proof is straightforward.

{\sloppy

\begin{thm}{Exercise}\label{ex:nonconvex-limit}
Construct a sequence of compact length spaces 
$\spc{X}_n$  
with a converging sequence of $\Lip$-Lipschitz concave functions $f_n\:\spc{X}_n\to\RR$ such that
the $\omega$-limit $\spc{X}_\omega$ of $\spc{X}_n$ is compact
and the $\omega$-limit $f_\omega\:\spc{X}_\omega\to\RR$ of $f_n$ is not concave.
\end{thm}

}

If $f\:\spc{X}\to\RR$ is a subfunction, 
the $\omega$-limit of the constant sequence $f_n=f$ is called the $\omega$-power of $f$ and denoted by $f^\omega$.
So
\[f^\omega\:\spc{X}\to\RR,\ \ f^\omega(x_\omega)=\lim_{n\to\omega} f(x_n).\]

Recall that we treat $\spc{X}$ as a subset of its $\omega$-power $\spc{X}^\omega$.
Note that $\Dom f=\spc{X}\cap \Dom f^\omega$.
Moreover, 
if $\oBall(x,\eps)_{\spc{X}}\subset \Dom f$
then $\oBall(x,\eps)_{\spc{X}^\omega}\subset \Dom f^\omega$.


\parbf{Ultradifferential.}
Given a function $f\:\spc{L}\to\RR$, consider sequence of functions $f_n\:n\cdot\spc{L}\to\RR$, defined by 
\[f_n(x^n)=n\cdot(f(x)-f(p)),\]
here $x^n\in n\cdot\spc{L}$ is the point corresopnding to $x\in\spc{L}$.
While $n\cdot(\spc{L},p)\to(\T^\omega,\0)$ as $n\to\omega$, 
functions $f_n$ converge to $\omega$-differential of $f$ at $p$.
It will be denoted by $\dd_p^\omega f$;
\[\dd_p^\omega f\:\T_p^\omega\to\RR,\ \ \dd_p^\omega f=\lim_{n\to\omega} f_n.\] 

Clearly, the $\omega$-differential $\dd_p^\omega f$ of a locally Lipschitz subfunction $f$ is defined at each point $p\in \Dom f$.
















\section{Comments} 

Given two metric spaces $\spc{X}$ and $\spc{Y}$, we will write $\spc{X}\preccurlyeq \spc{Y}$ if there is a noncontracting map $f\:\spc{X}\to \spc{Y}$;
that is, if 
$$ |x-x'|_{\spc{X}}\le|f(x)-f(x')|_{\spc{Y}}$$
for any $x,x'\in \spc{X}$.

Further, given $\eps>0$, we will write $\spc{X}\preccurlyeq \spc{Y}+\eps$
if there is a map $f\:\spc{X}\to \spc{Y}$ such that 
$$|x-x'|_{\spc{X}}\le|f(x)-f(x')|_{\spc{Y}}+\eps$$
for any $x,x'\in \spc{X}$.

Define 
$$\dist[\star]{\spc{X}}{\spc{Y}}{\spc{M}}=\inf\set{\eps}{\spc{X}\preccurlyeq \spc{Y}+\eps
\quad\text{and}\quad
\spc{Y}\preccurlyeq \spc{X}+\eps}$$
It turns out that $\dist[\star]{*}{*}{\spc{M}}$ is a different metric on the set of isometry classes of compact metric spaces; that is, in general $\dist[\star]{\spc{X}}{\spc{Y}}{\spc{M}}\not=|\spc{X}-\spc{Y}|_{\spc{M}}$. 
However, these two metrics define the same topology on $\spc{M}$.
More precicely:

\begin{thm}{Proposition}\label{GH-po}
For any sequence of compact metric spaces $(\spc{X}_n)$ and a compact metric space $\spc{X}_\infty$,
we have
$$|\spc{X}_n-\spc{X}_\infty|_{\spc{M}}\to 0
\quad\iff\quad
\dist[\star]{\spc{X}_n}{\spc{X}_\infty}{\spc{M}}\to 0$$ 
as $n\to\infty$.
\end{thm}

We will not give a proof of this proposition. 
Likely, we will not use it further in the lectures, 
but it might help you to build intuition for Gromov--Hausdorff convergence.
If you want to prove it yourself look in the proof of Theorem~\ref{thm:GH-is-a-metric} 
and try to modify it using ideas from the proof of Problem~\ref{pr:non-contracting=>isometry}.

The Gromov--Hausdorff distance can be defined for arbitrary pair of metric space.
Therefore it is natural to ask why we only consider compact metric spaces.
First note the Gromov--Hausdorff distance from any metric space $\spc{X}$ 
to its completion $\bar {\spc{X}}$ is zero.
Therefore if you want to end up in a metric space, it is better to consider only complete metric spaces.

Further, the distance between one-point-space and a metric spce with infinite diameter is infinite.
Therefore one has to either consider only bounded metric spaces (that is, the spaces with finite diameter)
or relux the definition of metric space which allow metric to take infinite value.

Finally, the class of isometry classes of all bounded complete metric spaces forms a class, but not a set.
Hence again we have two choices: either relux the definition of metric space so its points will form a class, or restrict further the class of spaces for which the isometry classes will form a set.

\begin{thm}{Exercise}
Prove that isometry classes of compact metric spaces form a set. 
\end{thm}

\begin{thm}{Exercise}\label{pr:GH1}
Let $\spc{X}=\{x,y,z\}$ be a three point subset of Euclidean plane with distances
$$|x-y|=|y-z|=|z-x|=1.$$
\begin{enumerate}[(i)]
\item Find the minimal Hausdorff distance from $\spc{X}$ to a one-point subset of the plane.
\item Find the Gromov--Hausdorff distance from $\spc{X}$ to the one-point metric space. 
\end{enumerate}
\end{thm}

\begin{thm}{Exercise}\label{pr:GH2}
Let $\spc{X}$ and $\spc{Y}$ be a compact metric spaces which have isometric $\eps$-nets.
Show that 
$$|\spc{X}-\spc{Y}|_{\spc{M}}\le 2\cdot\eps.$$
Is it always true that 
$$|\spc{X}-\spc{Y}|_{\spc{M}}\le \eps?$$
\end{thm}




\begin{thm}{Exercise}\label{pr:GH3}
Define the \emph{radius of a metric space}\index{radius of a metric space} $\spc{X}$ as 
$$\rad \spc{X}=\inf_x\left\{\sup_y\{|x-y|_{\spc{X}}\}\right\}.$$
Equivalently, 
$$\rad \spc{X}=\inf\set{R>0}{\text{there is}\ x\in \spc{X}\  \text{such that}\ B_R(x)\supset \spc{X}}.$$
 
\begin{enumerate}[(i)]
\item Show that for any compact metric space $\spc{X}$ we have
$$\tfrac12\cdot\diam \spc{X}\le \rad \spc{X}\le \diam \spc{X}.$$
\item Show that for any compact metric spaces $\spc{X},\spc{Y}$ we have
$$|\rad \spc{X}-\rad \spc{Y}|\le 2\cdot |\spc{X}-\spc{Y}|_{\spc{M}}.$$
\end{enumerate}
\end{thm}

\begin{thm}{Exercise}\label{pr:F-X}
Let $\spc{X}$ be a metric space.
If two compact sets $A, B$ in $\spc{X}$ are isometric,
we will write $A\iso B$. 
Set
$$d(A,B)=\inf \set{|A'-B'|_{\mathcal{H}(\spc{X})}}{A'\iso A \ \text{and}\ B'\iso B}.$$
Note that if $\spc{X}=\ell^\infty$, then according to Proposition~\ref{prop:GH-with-fixed-Z}, 
$d$ is a metric on $\mathcal{H}(\spc{X})/\iso$ (that is, on the ``$\iso$''-equivalecne classes of $\mathcal{H}(\spc{X})$).

Show that it does not hold for arbitrary metric space $\spc{X}$.
Understand the reason why it holds for $\spc{X}=\ell^\infty$.
\end{thm}


\begin{thm}{Exercise}\label{pr:GH-variation}
Consider the pairs $(\spc{X},A)$, where $\spc{X}$ is a compact metric space and $A$ is a closed subset in $\spc{X}$.
Two such pairs, say $(\spc{X},A)$ and $(\spc{X}',A')$ will be called equivalent (briefly $(\spc{X},A)\sim(\spc{X}',A')$)
if there is an isometry $\iota\:\spc{X}\to \spc{X}'$ such that $\iota(A)=A'$.

Modify the definition of Gromov--Hausdorff metric to construct a natural metric on the set of $\sim$-equivalence classes of the pairs $(\spc{X},A)$.
\end{thm}

Here we introduce so called Gromov--Hausdorff convergence for metric spaces.
This convergence was introduced by Gromov around 1980, published in \cite{gromov-1981}.
Very soon this notion began to be used in all branches of geometry.
In fact today I have difficulty to understand 
how one could do geometry without this type of convergence.%
(Some types of convergences of metric spaces was considered before Gromov,
but they had lack of generality;
each type of convergence was desined to solve one particular problem.)


\begin{thm}{Exercise}\label{ex:euclid-isom}
\begin{subthm}{}
Let $\spc{X},\spc{Y}$ be two compact sets in the Euclidean plane $\RR^2$.
Show that $\spc{X}$ is isometric to $\spc{Y}$ if and only if there is an motrio $\iota\:\RR^2\to \RR^2$
that sends $\spc{X}$ to $\spc{Y}$.
\end{subthm}

\begin{subthm}{}
Find two isometric subsets $\spc{X},\spc{Y}$ of $\ell^\infty$
such that there is no isometry $\iota\:\ell^\infty\to \ell^\infty$ 
that sends $\spc{X}$ to $\spc{Y}$.
\end{subthm}
\end{thm}

\backmatter

\newgeometry{top=0.9in, bottom=0.9in,inner=0.5in, outer=0.5in}
\chapter{Semisolutions}
{

\footnotesize
\begin{multicols}{2}
\parbf{\ref{ex:besikovitch=}.}
Let us use the same notation as in the proof of \ref{thm:besikovitch}.

Consider the map $s\:x\mapsto(\distfun_A(x),\distfun_B(x))$.
From the proof of \ref{thm:besikovitch} we get that $\Im s\supset \square$.
Observe that in the case of equality we have that $\Im s= \square$.
Indeed,
the same argument shows that 
\[\vol(s^{-1}(\square),g)\ge \vol\square=1.\]
The set $s^{-1}(\RR^1\backslash \square)$ is an open subset of $\square$.
If it is nonempty, then it has positive volume.
In this case
\[\vol(\square,g)>\vol(s^{-1}(\square),g)\ge 1\]
--- a contradiction.

Summarizing above discussion, there is a geodesic path of $g$-length $1$ connecting a point on one face of cube to the opposite face.

Moreover, for any pair of opposite faces and a point $p\in\square$, there is a geodesic path of $g$-length $1$ from one face to the other that pass thru $p$.
The latter can be shown by cutting $\square$ into two rectangles by a level surface of $\distfun_A$ thru $p$,
applying the above statement to both rectangles and taking the concatination of the obtained geodesic paths with end at $p$.
(The level surface might cut a rectangle with some topology, so have to apply \ref{thm:besikovitch+} instead of \ref{thm:besikovitch}).

Let $\gamma$ be such geodesic path from $A$ to $A'$.
Observe that $\gamma'(t)\z=\nabla_{\gamma(t)}\distfun_A$.
Therefore $\distfun_A$ is differentiable at every point $p\in \square$.
It follows that the map $s$ is differentiable.

Further checking the equality case in each inequality in the proof of \ref{thm:besikovitch}, we get that $s$ is a bijection and the equalities
\[|d_{p}\distfun_A|= 1,\quad|d_{p}\distfun_B|=1,\quad \text{and}\quad \langle d_{p}\distfun_A,d_{p}\distfun_B\rangle= 0\]
hold for almost all $p\in\square$.
Since $d_{p}\distfun_A$ and $d_{p}\distfun_B$ are well defined, we get that the equalities hold everywhere.
That is $s$ is an isometry.

\begin{wrapfigure}{r}{45 mm}
\vskip-4mm
\centering
\includegraphics{mppics/pic-27}
\end{wrapfigure}

\parbf{\ref{ex:hexagon}.}
Consider the hexagon with flat matric and curved sides shown on the diagram.
Observe that its area can be made arbitrary small while keeping the distances from the opposite sides at least 1.

\parbf{\ref{ex:gadograph}.}
Without loss of generality, we may assume that $V$ lies in a unit cube $\square$.
Consider a noncontinuous metric tensor $\bar g$ on $\square$ that coincides with $g$ inside $V$ and with the canonical flat metric tensor outside of $V$.

Observe that the $\bar g$-distances between opposite faces of $\square$ are at least 1.
Indeed this is true for the Euclidean metric and the assumption $\dist{p}{q}{g}\ge\dist{p}{q}{\EE^d}$  guarantees that one cannot make a shortcut in~$V$.
Therefore the $\bar g$-distances between every pair of opposite faces is at least as large as 1 which is the Euclidean distance.

This metric tensor $\bar g$ is not continuous at $\Sigma$, but the same argument as in \ref{thm:besikovitch} can be applied to show that $\vol(\square,\bar g)\ge \vol\square$.
Whence the statement follows.


\parbf{\ref{ex:involution-of-sphere}.}
Let $x\in \mathbb{S}^2$ be a point that minimize the distance $|x-x'|_g$.
Consider a minimizing geodesic $\gamma$ from $x$ to $x'$.
We can assume that 
\[|x-x'|_g=\length \gamma=1.\]

Let $\gamma'$ be the antipodal arc to $\gamma$.
Note that $\gamma'$ intersects $\gamma$ only at the common endpoints $x$ and $x'$.
Indeed, if $p'=q$ for some $p,q\in\gamma$, then $|p-q|\ge 1$.
Since $\length \gamma=1$, the points $p$ and $q$ must be the ends of $\gamma$.

It follows that $\gamma$ together with $\gamma'$ forms a closed simple curve in $\mathbb{S}^2$
that divides the sphere into two disks $D$ and $D'$.

Let us divide $\gamma$ into two equal arcs $\gamma_1$ and $\gamma_2$; each of length $\tfrac12$.
Suppose that $p,q\in\gamma_1$, then 
\begin{align*}
|p-q'|_g&\ge |q-q'|_g-|p-q|_g\ge
\\
&\ge 1-\tfrac12=\tfrac12.
\end{align*}
That is, the minimal distance from $\gamma_1$ to $\gamma_1'$ is at least~$\tfrac12$.
The same way we get that the minimal distance from $\gamma_2$ to $\gamma_2'$ is at least~$\tfrac12$.
By Besicovitch inequality, we get that 
\[\area(D,g)\ge \tfrac14\quad\text{and}\quad \area(D',g)\ge \tfrac14.\]
Therefore 
\[\area(\mathbb{S}^2,g)\ge\tfrac12.\]

\parit{A better estimate.}
Let us indicate how to improve the obtained bound to
\[\area(\mathbb{S}^2,g)\ge1.\]

Suppose $x$, $x'$, $\gamma$ and $\gamma'$ are as above.
Consider the function
\[f(z)=\min_t \{\,|\gamma'(t)-z|_g+t\,\}.\]
Observe that $f$ is 1-Lipschitz.

Show that two points $\gamma'(c)$ and $\gamma(1-c)$ lie on one connected component of the level set $L_c=\set{z\in\mathbb{S}^2}{f(z)=c}$;
in particular 
\[\length L_c\ge 2\cdot|\gamma'(c)-\gamma(1-c)|_g.\]
By the triangle inequality, we have that
\begin{align*}
|\gamma'(c)-\gamma(1-c)|_g&\ge 1-|\gamma(c)-\gamma(1-c)|_g=
\\
&=1-|1-2\cdot c|.
\end{align*}

It remains to apply the coarea formula
\[\area(\mathbb{S}^2,g)\ge \int\limits_0^1\length L_c\cdot dc.\]

\parit{Remarks.}
The bound $\tfrac12$ was proved by Marcel Berger \cite{berger}. 
Christopher Croke conjectured that the optimal bound is $\tfrac4\pi$ and the round sphere is the only space that achieves this \cite[Conjecture 0.3 in][]{croke}.

\begin{wrapfigure}{r}{20 mm}
\vskip-0mm
\centering
\includegraphics{mppics/pic-1305}
\end{wrapfigure}

\parbf{\ref{ex:involution-of-3sphere}.}
Given $\eps>0$, construct a disk $\Delta$ in the plane with 
\[\length\partial \Delta<10\ \ \text{and}\ \ \area \Delta<\eps\]
that admits an continuous involution $\iota$ such that 
\[|\iota(x)-x|\ge 1\]
for any $x\in\partial \Delta$.

An example of $\Delta$ can be guessed from the picture;
the invoultion $\iota$ makes a length preserving half turn of its boundary $\partial \Delta$.


Take the product $\Delta\times \Delta\subset \RR^4$;
it is homeomorphic to the 4-ball.
Note that 
$$\vol_3[\partial(\Delta\times \Delta)]=2\cdot\area \Delta\cdot\length \partial \Delta<20\cdot\eps.$$
The boundary $\partial(\Delta\times \Delta)$ is homeomorphic to $\mathbb{S}^3$
and the restriction of the involution $(x,y)\z\mapsto (\iota(x),\iota(y))$ has the needed property.

All we have to do now is to smooth $\partial(\Delta\times \Delta)$ a little bit.

\parit{Remark.} This example is given by Christopher Croke \cite{croke}.
Note that according to \ref{thm:sys+}, 
the involution $\iota$ cannot be made isometric.

\parbf{\ref{ex:GH-vol}.}
Note that if $\spc{M}_\infty$ is $e^{\pm\eps}$-bilipschitz to a cube, then applying Besicovitch inequality, we get that 
\[\liminf_{n\to\infty} \vol \spc{M}_n\ge e^{-n\cdot \eps}\cdot\vol \spc{M}_\infty.\]

Applying Vitali covering theorem, given $\eps>0$, we can cover whole volume of $\spc{M}_\infty$ by $e^{\pm\eps}$-bilipschitz cubes.
Applying the above observation and summing up the results, we get that 
\[\liminf_{n\to\infty} \vol \spc{M}_n\ge e^{-n\cdot \eps}\cdot\vol \spc{M}_\infty.\]
The statement follows since $\eps$ is arbitrary positive number.

\parit{Remark.} A more general result was obtaind by Sergei Ivanov~\cite{ivanov-1997}.
Note that the statement does not hold without stability of the convergence. In fact any compact metric space can be approximated by Riemannian surface with arbitrary small area.

\parbf{\ref{ex:sysT2}.}
Set $s=\sys(\TT^2,g)$.

Cut $\TT^2$ along a shortest closed noncontractible curve $\gamma_1$.
We get an anulus with a Riemnnian metric on it $(N,g)$.
Denote by $A$ and $A'$ the two components of its boundary.

Assume that $\gamma_2$ is a shortest path that runs from $A$ to $A'$ in $(N,g)$.
The image of $\gamma_2$ in $\TT^2$ connects two points in $\gamma_1$;
further we will use the same notation for $\gamma_2$ and its image in $\TT^2$.
Connect $\gamma_2(0)$ to $\gamma_2(1)$ by a shorter arc in $\gamma_1$.
Note that the obtained closed curve is noncontractible in $\TT^2$.
Therefore its length is at least $s$.
The arc of $\gamma_1$ has length at most half of $\length\gamma_1$.
Whence $\length \gamma_2\ge \tfrac s2$.
In particular the distance from $A$ to $A'$ in $(N,g)$ is at least $\tfrac s2$.

\begin{wrapfigure}{r}{45 mm}
\vskip-4mm
\centering
\includegraphics{mppics/pic-23}
\end{wrapfigure}

Let us cut $(N,g)$ by $\gamma_2$, we obtain a square $(\square,g)$ with Riemnnian metric on it.
Let us keep the notation $A$ and $A'$ for the pair of opposite sides in $(\square,g)$ that correspond to $A$ and $A'$ in $(N,g)$.
From above we have that distance from $A$ to $A'$ is at least $\tfrac s2$.

Denote by $B$ and $B'$ the remaining pair of opposite sides $(\square,g)$.
Suppose that $\gamma_3$ is a path connecting these sides.
Map it the curves $\gamma_i$ back to the torus and let us keep for them the same notation.
The path $\gamma_3$ connects two points on $\gamma_2$.
Since $\gamma_2$ is shortest, the arc of $\gamma_2$ between this pair of points cannot be longer than $\gamma_3$.
This arc together with $\gamma_3$ forms a closed noncontractible curve, so its length has to be at least $s$.
It follows that $\length\gamma_3\ge \tfrac s2$.
That is distance from $B$ to $B'$ in  $(\square,g)$ is at least $\tfrac s2$.

Applying Besikovitch inequality, we get that 
\[\area(\TT^2,g)=\area(\square,g)\ge \tfrac14\cdot s^2.\]

\parit{Remark.}
Alternatively one may notice that any curve in $(N,g)$ that is bordant to $A$ has length at least $\tfrac s2$.
Therefore the level sets defined by $\distfun_A(x)_{(N,g)}=t$ have length at least $\tfrac s2$ if $0\le t\le \tfrac s2$.
Applying coarea fromula we get that
\[\area(\TT^2,g)=\area(N,g)\ge \tfrac12\cdot s^2.\]
This estimate is twice better then the one above, but it is still far from the optimal bound $\tfrac2{\sqrt{3}}\cdot s^2$ in proved by Loewner inequality

\begin{wrapfigure}{r}{44 mm}
\vskip-4mm
\centering
\includegraphics{mppics/pic-25}
\end{wrapfigure}

\parbf{\ref{ex:sysRP2}.}
Set $s\z=\sys (\RP^2,g)$.
Cut $(\RP^2,g)$ along a shortest noncontractible curve $\gamma$.
We obtain $(\DD^2,g)$ --- a disc with metric tensor which we still denote by $g$.
Divide $\gamma$ into two equal arcs $\alpha$ and $\beta$.
Denote by $A$ and $A'$ the two connected components of the inverse image of $\alpha$.
Similarly denote by $B$ and $B'$ the two connected components of the inverse image of $\beta$.

Let $\gamma_1$ be a path from $A$ to $A'$;
map it to $\RP^2$ and keep the same notation for it.
Note that $\gamma_1$ together with a subarc of $\alpha$ forms a closed noncontractible curve in $\RP^2$.
Since $\length\alpha=\tfrac s2$, we have that $\length\gamma_1\ge \tfrac s2$.
It follows that the distance between $A$ and $A'$ in $(\DD^2,g)$ is at least $\tfrac s2$.
The same way we show that the distance between $B$ and $B'$ in $(\DD^2,g)$ is at least $\tfrac s2$.

Note that $(\DD^2,g)$ can be paraneterized by a square with sides $A$, $B$, $A'$ and $B'$ and apply \ref{thm:besikovitch} to show that 
\[\area(\RP^2,g)=\area(\DD^2,g)\ge \tfrac14\cdot s^2.\]

\parit{Remark.}
For the optimal constant was found by Pao Ming Pu see the discussion on page \pageref{page:pu}.
His proof shows that any Riemannian metric on the disc with the boundary globally isometric to a unit circle with angle metric has area at least as large as the unit hemisphere.
It is expected that the same inequality holds for any compact surface bounded by a single curve (not necessary a disc);
this is the so called the {}\emph{filling area conjecture} mentioned in \cite[5.5.B$'$(e$'$)]{gromov-1983}.

\parbf{\ref{ex:sysSg}.} Cut the surface along a shortest noncontractible curve $\gamma$. 
We might get a surface with one or two components of the boundary.
In these two cases repeat the arguments in \ref{ex:sysRP2} or \ref{ex:sysT2} using \ref{thm:besikovitch+} instead of \ref{thm:besikovitch}.


\parbf{\ref{ex:sysS2xS1}.} Consider the product of small 2-sphere with a unit circle.

\parbf{\ref{ex:macrodimension}.}
The following claim resembles Besikovitch inequality;
it is key to the proof:
\begin{itemize}
 \item[$({*})$] Let $a$ be a positive real number.
 Assume that a closed curve $\gamma$ in a metric space $\spc{X}$ can be sudivided into 4 arcs $\alpha$, $\beta$, $\alpha'$, and $\beta'$ in such a way that 
 \begin{itemize}
 \item $|x-x'|>a$ for any $x\in\alpha$ and $x'\in \alpha'$
 and
 \item $|y-y'|>a$ for any $y\in\beta$ and $y'\in \beta'$.
 \end{itemize}
 Then $\gamma$ is not contractable in its $\tfrac a2$-neighborhood.
\end{itemize}

To prove $({*})$, consider two functions defined on $\spc{X}$ as follows:
\begin{align*}
w_1(x)&=\min \{\,a,\distfun_{\alpha}(x)\,\}
\\
w_2(x)&=\min \{\,a,\distfun_{\beta}(x)\,\}
\end{align*}
and the map $\bm{w}\:\spc{X}\to [0,a]\times[0,a]$, defined by
\[\bm{w}\:x\mapsto(w_1(x),w_2(x)).\]

Note that 
\begin{align*}
\bm{w}(\alpha)&=0\times [0,a],
&
\bm{w}(\beta)&=[0,a]\times 0,
\\
\bm{w}(\alpha')&=a\times [0,a],
&
\bm{w}(\beta')&=[0,a]\times a,
\end{align*} 
Therefore, the composition $\bm{w}\circ\gamma$ is a degree 1 map 
\[\mathbb{S}^1\to \partial([0,a]\times[0,a]).\] 
It follows that if $h\:\DD\to \spc{X}$ shrinks $\gamma$, then there is a point $z\in\DD$ such that 
$\bm{w}\circ h(z)=(\tfrac a2,\tfrac a2)$.
Therefore $h(z)$ lies at distance at least $\tfrac a2$ from $\alpha$, $\beta$, $\alpha'$, $\beta'$
and therefore from $\gamma$.
Hence the claim $({*})$ follows.

\medskip

Coming back to the problem, let $\{W_i\}$ be an open covering of the real line with multiplicity $2$ and $\rad W_i<R$ for each $i$;
for example one may take $W_i=((i-\tfrac23)\cdot R,(i+\tfrac23)\cdot R)$.

Choose a point $p\in \spc{X}$.
Denote by $\{V_j\}$ the connected components of $\distfun_p^{-1}(W_i)$ for all $i$.
Note that $\{V_j\}$ is an open finite cover of $\spc{X}$ with multiplicity at most 2.
It remains to show that $\rad V_j<100\cdot R$ for each $j$.

\begin{wrapfigure}{o}{31 mm}
\vskip-2mm
\centering
\includegraphics{mppics/pic-1310}
\end{wrapfigure}

Aarguing by contradiction assume there is a pair of points  $x,y\in V_i$ 
such that $|x\z-y|_{\spc{X}}\ge 100\cdot R$.
Connect $x$ to $y$ with a curve $\tau$ in $V_j$.
Consider the closed curve $\sigma$ formed by $\tau$ and two geodesics $[px]$, $[py]$.


Note that $|p-x|>40$.
Therefore there is a point $m$ on $[px]$ such that $|m-x|=20$.

By the triangle inequality, the subsdivision of $\sigma$ into the arcs $[pm]$, $[mx]$, $\tau$ and $[yp]$ satisfy the conditions of the claim $({*})$ for $a=10\cdot R$.
Hence the statement follows.

\parit{The quasiconverse} does not hold.
As an example take a surface that looks like a long cylinder with two hats,
it is a smooth surface diffeomorphic to a sphere.
\begin{figure}[h!]
\vskip0mm
\centering
\includegraphics{mppics/pic-1315}
\end{figure}
Assuming the cylinder is thin, it has macroscopic dimension 1 at a given scale.
However a circle formed by a section of cylinder around its midpoint by a plane parallel to the base is a circle that cannot be contracted in its small neighborhood.

\parit{Sourse:} \cite[Appendix 1(E$_{2}$)]{gromov-1983}.

\parbf{\ref{ex:width=suprad(inv)},} \textit{``only if'' part.}
Suppose $\width_n\spc{X}<R$.
Consider a covering $\{V_1,\dots,V_k\}$ of $\spc{X}$ guaranteed by the definition of width.
Let $\spc{N}$ be its nerve and $\psi\:\spc{X}\to \spc{N}$ be the map provided by \ref{prop:space->nerve}.

Since the multiplicity of the covering is at most $n+1$, we ahve $\dim \spc{N}\le n$.

Note that if $x\in \spc{N}$ lies in a star of a vertex $v_i$,
then $\psi^{-1}\{x\}\z\subset V_i$;
in particular $\rad[\psi^{-1}\{x\}]<R$.

\parit{``If'' part.}
Choose $x\in \spc{N}$.
Since the inverse image $\psi^{-1}\{x\}$ is compact, $\psi$ is continuous, and $\rad[\psi^{-1}\{x\}]<R$,
there is a neighborhood $U\ni x$ such that the  $\rad[\psi^{-1}(U)]<R$.

Since $\spc{X}$ is compact,  there is a finite cover $\{U_i\}$ of $\spc{N}$ such that $\psi^{-1}(U_i)\subset\spc{X}$ has radius smaller than $R$ for each $i$.
Since $\spc{N}$ has dimension $n$, we can inscribe%
\footnote{Recall that a covering $\{W_i\}$ is inscribed in the covering $\{U_i\}$ if for every $W_i$ is a subset of some $U_j$.} 
in $\{U_i\}$ a finite open cover $\{W_i\}$ with multiplicity at most $n+1$.
It remains to observe that $V_i=\psi^{-1}(W_i)$ defines a finite open cover of $\spc{X}$ with radius less than $R$ and multiplicity at most $n+1$. 


\parbf{\ref{ex:1D-case}.}
Assume that $\spc{P}$ is connected.

Let us show that $\diam\spc{P}<R$.
If this is not the case, then there are points $p,q\in\spc{P}$ on distance $R$ from each other.
Let $\gamma$ be a geodesic from $p$ to $q$.
Clearly $\length\gamma\ge R$ and $\gamma$ lies in $\oBall(p,R)$ except for the endpoint $q$.
Therefore $\length[\oBall(p,R)_{\spc{P}}]\ge R$.
Since $\VolPro_{\spc{P}}(R)\z\ge \length[\oBall(p,R)_{\spc{P}}]$,
the latter contradicts $\VolPro_{\spc{P}}(R)<R$.

In general case, we get that each connected component of $\spc{P}$ has radius smaller that $R$.
Whence the width of $\spc{P}$ is smaller that $R$.

\parit{Second part.} Again, we can assume that $\spc{P}$ is connected.

The examples of line segment or a circle show that the constant $c=\tfrac12$ cannot be improved.
It remains to show that the inequality holds with $c=\tfrac12$.

Choose $p\in\spc{P}$ such that the value
\[\rho(p)=\max\set{\dist{p}{q}{\spc{P}}}{q\in\spc{P}}\]
is minimal.
Suppose $\rho(p)\ge\tfrac 12\cdot R$.
Observe that there is a point $x\in \spc{P}\backslash\{p\}$ that lies on any shortest path starting from $p$ and length $\ge\tfrac 12\cdot R$.
Otherwise for any $r\in(0,\tfrac 12\cdot R)$ there would be at least two points on distance $r$ from $p$;
by coarea inequality we get that the total length of $\spc{P}\cap \oBall(p,\tfrac 12\cdot R)$ is at least $R$ --- a contradiction.

Moving $p$ toward to $x$ reduce $\rho(p)$ which contradicts the choice of~$p$.

\parbf{\ref{ex:connected-sum-essential}.}
Suppose $M$ is an essential manifold and $N$ is arbitrary closed manifold.
Observe that shrinking $N$ to a point produces a map $f\:N\#M\to M$ of degree 1; that is, the fundamental class of $N\#M$ maps to the fundamental class of $M$.

Since $M$ is essential, there is an aspherical space $K$ and a map $\iota\:M\to K$ that sends fundamental class of $M$ to nonzero homology class in $K$.
From above, the composition $\iota\circ f\:N\#M\to K$ sends fundamental class of $N\#M$ to the same homology class in $K$.

\parit{Remark.} Note that we only used that there is a map $N\#M\to K$ of degree 1. If essential manifold is defined using homologies with integer coefficients, then existence of map of nonzero degree is sufficient.


\end{multicols}
}

\newgeometry{top=0.9in, bottom=0.9in,left=0.9in, right=0.9in, paperwidth=6in, paperheight=9in}

{\small\sloppy
\RequirePackage{snapshot}
\documentclass[twoside]{book}

\usepackage{lectures}
\usepackage[colorlinks=true,
citecolor=black,
linkcolor=black,
anchorcolor=black,
filecolor=black,
menucolor=black,
urlcolor=black,
pdftitle={Metric geometry on manifolds: two lectures},
pdfsubject={Geometry},
pdfauthor={Anton Petrunin}
]{hyperref}
\makeindex

\begin{document}
 
\title{Metric geometry on manifolds:
\\ two lectures}
\author{Anton Petrunin}
\date{}
\maketitle

We discuss Besikovitch inequality, width, and systole of manifolds.

We assume that students familiar with the smooth manifolds, degree of map, CW-complexes and related notions.

These are two final lectures of a graduate course given at Penn State, Spring 2020.
The complete lectures can be found on the authors website;
it includes an introduction to metric geometry \cite{petrunin2020pure}
and elements of Alexandrov geometry based on \cite{alexander-kapovitch-petrunin-2019}.

\thispagestyle{empty}
\tableofcontents
\thispagestyle{empty}

%%%%%%%%%%%%%%%%%%%%%%%%%%%%
%\addtocounter{chapter}{-1}
\chapter{Homework assignments}


It is better to think about all the problems, but you do not have to solve \emph{all} of them.
If a problem is solved, you do not have to write its solutions, but try sketch it.

\section{Due Tue Jan 21}

Exercises: \sout{\ref{ex:almost-min},} \ref{ex:non-contracting-map}, \ref{ex:no-geod}, \sout{\ref{ex:compact=>complete},} \ref{exercise from BH}, \ref{ex:Hausdorff-bry}.

\section{Due Tue Jan 28}

Exercises: \ref{ex:almost-min},  \ref{ex:compact=>complete}, \ref{ex:Huas-perimeter-area}, \ref{ex:GH-po}, \ref{pr:doubling}, \ref{pr:under:if}.

\section{Due Tue Feb 4}
Exercises: 
\ref{ex:compact-length}, 
\ref{pr:under:only-if}, 
\sout{\ref{ex:GH-SC},}
\sout{\ref{ex:sphere-to-ball},}
\ref{ex:ultrapower}, 
\ref{ex:two-geodesics-in-ultrapower}.

\section{Due Tue Feb 11}

Finish exercises \ref{ex:compact-length} , \ref{pr:under:only-if}, \ref{ex:GH-SC}, \ref{ex:sphere-to-ball}.

\noindent
Exercises: \ref{ex:lim(tree)}, \ref{ex:Asym(Lob)}, \ref{ex:geodesics-urysohn}, \ref{ex:sphere-in-urysohn}.

\section{Due Tue Feb 18}

Exercises: \ref{ex:compact-extension}, \ref{ex:+-c}, \ref{ex:ultrametric}, \ref{ex:injective-spaces}, \ref{ex:tripod+square}, \ref{ex:4-on-a-line}.

\noindent Write down a solution of at least one of the exercises.

\section{Due Tue Feb 25}

Finish Exercise \ref{ex:tripod+square:square}.
Prepare questions for review on Tuesday.

\section{Due Tue Mar 3}

Exercises: \ref{ex:sba-2+2-short}, \ref{ex:(3+1)-expanding}, \ref{ex:CAT+CBB}, \ref{ex:product-CBB}, \sout{\ref{ex:CBB-geodesic},} \ref{ex:fat-triangle}.

\noindent Write down a solution of at least one of the exercises.

\section{Due Tue Mar 17}

Exercises: \ref{ex:tringle-inq-angles},
\ref{ex:CBB-geodesic},
\ref{ex:convex-dist},
\ref{ex:reshetnyak-doubling},
\ref{ex:supporting-planes},
\ref{ex:centrally-simmetric-walls}.

\noindent Write down a solution of at least one of the exercises.

\section{Due Tue Mar 24}

Exercises: 
\ref{ex:contractible},
\ref{ex:convex-nbhd},
\ref{ex:closest-point},
\ref{cor:balls:dim=1},
\ref{ex:null-homotopic},
\sout{\ref{ex:branching-cover}.}

 Write down as many solutions as you can; email it to Zetian Yan (zxy5156) + cc to me (aqp6).

\section{Due Tue Mar 31}

Exercises: 
\ref{ex:branching-cover},
\ref{ex:tan(CAT)isCAT},
\ref{ex:tan(CAT)is-length},
%\ref{ex:product-cone},
\ref{ex:unique-geod=CAT},
\ref{ex:flag>=pi/2},
\ref{ex:tree}.

Write down as many solutions as you can; email it to Zetian Yan (zxy5156) + cc to me (aqp6).

\section{Due Tue Apr 7}

Exercises: 
\ref{ex:CAT-mnfld=>ext.geod},
\ref{ex:locally-convex},
\ref{ex:geod-circle},
\ref{ex:flag-aspherical},
\ref{ex:example-pi_infty-new},
\ref{ex:cube-infty=>cube-2}.

Write down as many solutions as you can; email it to Zetian Yan (zxy5156) + cc to me (aqp6).

\section{Due Tue Apr 14}

Exercises: 
\ref{ex:geod-CBA},
%\ref{prop:two-hull-open},
%\ref{ex:chopping-triangle},
\ref{ex:concave-triangle},
\ref{ex:two-planes},
\ref{ex:hemisphere},
\ref{ex:inner-support},
\ref{ex:convex+saddle+broken=>PL}.

Write down as many solutions as you can; email it to Zetian Yan (zxy5156) + cc to me (aqp6).

\section*{Remark}
Each working day I will check email before 15:00 and will appear online if you ask (it is easy for me --- do not hesitate to ask).
We will meet regular hours online (as we did before).

%%%%%%%%%%%%%%%%%%%%%%%%%%%%

\chapter{Besicovitch inequality} 

We will focus on Riemannian spaces --- these are specially nice length metrics on manifolds.
These spaces are also most important in applications.

As it will be indicated in Section~\ref{sec:hausdorff-measure},
most of the statements of this and the following lecture have counterparts for general length metrics on manifolds.

\section{Riemannian spaces}

Let $M$ be a smooth connected manifold.
A \index{metric tensor}\emph{metric tensor} on $M$ is a choice of positive definite quadratic forms $g_p$ on each tangent space $\T_pM$ that depends continuously on the point;
that is, in any local coordinates of $M$ the components of $g$ are continuous functions.

A \index{Riemannian!manifold}\emph{Riemannian manifold} $(M,g)$ is a smooth manifold $M$ with a choice of metric tensor $g$ on it.

The \index{length}\emph{$g$-length} of a Lipschitz curve $\gamma\:[a,b]\to M$  is defined by
\[\length_g\gamma=\int_a^b\sqrt{g(\gamma'(t),\gamma'(t))}\cdot dt.\]
The $g$-length induces a metric metric on $M$; it is defined as the greatest lower bound to lengths of Lipschitz curves connecting two given points;
the distance between a pair of points $x,y\in M$ will be denoted by 
\[\dist{x}{y}{g}\quad\text{or}\quad\distfun_x(y)_g.\]
The corresponding metric space $\spc{M}$ will be called \index{Riemannian!space}\emph{Riemannian}.

The following exercise implies that \textit{isometry between Riemannian spaces might be not induced by a diffeomorphism}.

\begin{thm}{Exercise}\label{ex:non-differentiable}
Construct a continuous Riemannian metric $g$ on $\RR^2$ such that the corresponding Riemannian space admits an isometry to the Euclidean palne but the induced map $\iota\:\RR^2\to\RR^2$ is not differentiable at some point.
\end{thm}

The exercise above shows that in general the smooth structure is not uniquely defined on Riemannian space.
Therefore in general case one has to distinguish between Riemannian manifold and the corresponding Riemannian space altho there is almost no difference.%
\footnote{In fact a straightforward smoothing procedure shows that isometry between Riemannian spaces can be approximated by diffeomorphisms between underlying manifolds; in particular these manifolds are diffeomorphic.
Also, if the metric tensor is smooth, then it is not hard to show that Riemannian space {}\emph{remembers} everything about the Riemannian manifold, in particular the smooth structure;
it is a part of the so-called Myers--Steenrod theorem \cite{myers-steenrod}.} 

The following observation states the key property of Riemannian spaces;
it will be used to extend results from Euclidean space to Riemannian spaces.

\begin{thm}{Observation}\label{obs:lip-chart}
For any point $p$ in a Riemannian space $\spc{M}$ and any $\eps>0$ there is a $e^{\mp\eps}$-bilipschitz chart $s\:W\to V$ from an open subset $W$ of the $n$-dimensional Euclidean space to some neighborhood $V\ni p$.
\end{thm}

\parit{Proof.}
Choose a chart $s\:U\to \spc{M}$ that covers $p$.
Note that there is a linear transformation $L$ such that for the metric tensor in the chart $s\circ L$ is coincides with the standard Euclidean tensor at the point $x=(s\circ L)^{-1}(p)$.

Since the metric tensor is continuous, the restriction of $s\circ L$ to a small neighborhood of $x$ is $e^{\mp\eps}$-bilipschitz.
\qeds

\section{Volume and Hausdorff measure}\label{sec:vol-haus}

Let $(M,g)$ be an $n$-dimensional Riemannian manifold.
If a Borel set $R\subset M$ is covered by one chart $\iota\:U\to M$,
then its \index{volume}\emph{volume} (briefly, $\vol R$ or $\vol_n R$) is defined by 
\[\vol R
\df
\int_{\iota^{-1}(R)}\sqrt{\det{g}}.\]
In the general case we can subdivide $R$ into a countable collection of regions $R_1,R_2\dots$ such that each region $R_i$ is covered by one chart $\iota_i\:U_i\to M$ and define
\[\vol R\df \vol R_1+\vol R_2+\dots\]
The chain rule for multiple integrals implies that the right-hand side does not depend on the choice of subdivision and the choice of charts.

Similarly, we define integral along $(M,g)$.
Any Borel function $u\:M\z\to \RR$, can be presented as a sum $u_1+u_2+\cdots$ such that the support of each function $u_i$ can be covered by one chart $\iota_i\:U_i\to M$
and set 
\[\int_{p\in\spc{M}} u(p)
\df
\sum_i\left[\int_{x\in U_i} u_i\circ s(x)\cdot\sqrt{\det{g}}\right].
\]
In particular
\[\vol R=\int_{p\in R} 1.\]

Let $\spc{X}$ be a metric space and $R\subset \spc{X}$.
The \index{Hausdorff measure}\emph{$\alpha$-dimensional Hausdorff measure} of $R$ is defined by 
$$\haus_\alpha R
\df
\lim_{\eps\to0}
\,
\inf
\set{\sum_{n\in\NN}(\diam A_n)^\alpha}
{\begin{aligned}
&\diam A_n<\eps\ \text{for}
\\
&\text{for each}\ n,\text{all}\  A_n
\\
&\text{are closed, and} 
\\
& \bigcup_{n\in\NN}A_n\supset R.
\end{aligned}
}.$$
For properties of Hausdorff measure we refer to the classical book of  Herbert Federer \cite{federer};
in particular, $\haus_\alpha$ is indeed a measure and $\haus_\alpha$-measurable sets include all Borel sets.

The following observation follows from \ref{obs:lip-chart} and Rademacher's theorem:

\begin{thm}{Observation}\label{obs:lipcart+}
Suppose that a Borel set $R$ in an $n$-dimensional Riemannian space $\spc{M}$ is subdivided into a countable collection of subsets $R_i$ such that each $R_i$ is covered by an $e^{\mp\eps}$-bilipschitz charts
$s_i$.
Then
\begin{align*}
\vol_n R&\lege e^{\pm n\cdot\eps}\cdot\sum_i\vol_n[s_i^{-1}(R_i)]
\intertext{and}
\haus_n R&\lege e^{\pm n\cdot\eps}\cdot\sum_i\haus_n[s_i^{-1}(R_i)]
\end{align*}

\end{thm}

According to \index{Haar's theorem}\emph{Haar's theorem}, 
a measure on $n$-dimensional Euclidean space that is invariant with respect to parallel translations is proportional to volume.
Observe that 
\begin{itemize}
\item A ball in $n$-dimensional Euclidean space of diameter $1$ has unit Hausdorff measure.
\item A unit cube in $n$-dimensional Euclidean space has unit volume.
\end{itemize}
Therefore, for any Borel region $R\subset \EE^n$, we have 
\[\vol_n R=\tfrac{\omega_n}{2^n}\cdot\haus_n R,\eqlbl{eq:vol/mu}\]
where $\omega_n$ denotes the volume of a unit ball in the $n$-dimensional Euclidean space.

Applying \ref{eq:vol/mu} together with \ref{obs:lipcart+}, we get that the inequalities
\[\vol_n R\lege e^{\pm2\cdot n\cdot \eps}\cdot\tfrac{\omega_n}{2^n}\cdot\haus_n R\]
hold for any $\eps>0$.
Since $\eps>0$ is arbitrary, we get that \ref{eq:vol/mu} holds in $n$-dimensional Riemannian spaces.
More precisely:

\begin{thm}{Proposition}\label{prop:vol=haus}
The identity 
\[\vol_n R=\tfrac{\omega_n}{2^n}\cdot\haus_n R\]
holds for any Borel region $R$ in an $n$-dimensional Riemannian space. 
\end{thm}

Since the Hausdorff measure is defined in pure metric terms, the proposition gives another way to prove that the volume does not depend on the choice of chars and subdivision of $R$.

The identity in this proposition will be used to define volume of any dimension.
Namely, given an integer $k\ge 0$, the $k$-volume is defined by
\[\vol_k\df\tfrac{\omega_k}{2^k}\cdot\haus_k.\]
By \ref{prop:vol=haus}, if $A$ is a subset of $k$-dimensional submanifold $\spc{N}\subset \spc{M}$, then the two definitions of $\vol_kA$ agree; but the latter definition works for a wider class of sets. 

\begin{thm}{Exercise}\label{ex:volume-preserving+short}
Let $f\:\spc{M}\to \spc{N}$ be a short volume-preserving map between $n$-dimensional Riemannian spaces.
Prove the following statements and use them to conclude that $f$ is locally distance-preserving.

\begin{subthm}{ex:volume-preserving+short:injective}
$f$ is injective; 
that is, if $f(x)=f(y)$, then $x=y$.
\end{subthm}

\begin{subthm}{ex:volume-preserving+short:bi}
For any $c<1$, the map $f$ is locally $[c,1]$-bilipschitz;
that is, for any point in $\spc{M}$ there is a neighborhood $\Omega$ and $\eps>0$ such that the inequality 
\[c\le \frac{|f(x)-f(y)|_{\spc{N}}}{|x-y|_{\spc{M}}}\le 1 \]
holds for any pair of distinct points $x,y\in \Omega$.
\end{subthm}

\end{thm}


\section{Area and coarea formulas}

Suppose that $f\:\spc{M}\to\spc{N}$ is a Lipschitz map between $n$-dimensional Riemannian spaces $\spc{M}$ and $\spc{N}$.
Then by \index{Rademacher's theorem}\emph{Rademacher's theorem} 
the differential $d_p f\:\T_p\spc{M}\to\T_{f(p)}\spc{N}$ is defined at \index{almost all}\emph{almost all} $p\in \spc{M}$;
that is, the differential defined at all points $p\in\spc{M}$ with exception of a subset with vanishing volume.

The differential is a linear map; it defines the Jacobian matrix $\Jac_pf$ in orthonormal frames of $\T_p$ and $\T_{f(p)}\spc{N}$.
The determinant of $\Jac_pf$ will be denoted by $\jac_p$.
Note that the absolute value $|\jac_p|$ does not depend on the choice of the orthonormal frames.

The identity in the following proposition is called \index{area formula}\emph{area formula}.

\begin{thm}{Proposition}
Let $f\:\spc{M}\to\spc{N}$ be a Lipschitz map between $n$-dimensional Riemannian spaces $\spc{M}$.
Then for  any Borel function $u\:\spc{M}\z\to \RR$ the following equality holds:
\[\int_{p\in \spc{M}} u(p)\cdot |\jac_pf|=\int_{q\in \spc{N}}\sum_{p\in f^{-1}(q)} u(p).\]

\end{thm}

\parit{Proof.}
If $\spc{M}$ and $\spc{N}$ are isometric to the $n$-dimensional Euclidean space, then the statement follows from the standard area formula \cite[3.2.3]{federer}.

Note that Jacobian of a $e^{\mp\eps}$-bilipschitz map between $n$-dimensional Riemannian manifolds (if defined) has determinant in the range $e^{\mp n\cdot\eps}$.
Applying \ref{obs:lipcart+} and the area formula in $\EE^n$, we get the following approximate version of the needed identity for any $u\ge0$: 
\[\int_{p\in \spc{M}} u(p)\cdot |\jac_pf|
\lege e^{\pm 3\cdot n\cdot \eps}\int_{q\in \spc{N}}\sum_{p\in f^{-1}(q)} u(p).\]

Since $\eps>0$ is arbitrary, we get that the area formula holds if $u\ge 0$.
Finally, since both sides of the area formula are linear in $u$, it holds for any $u$.
\qeds

The following inequality is called \index{area inequality}\emph{area inequality}:

\begin{thm}{Corollary}\label{cor:area-inequality}
Let $f\:\spc{M}\to\spc{N}$ be a locally Lipschitz map between $n$-dimensional Riemannian spaces.
Then 
\[\int_{p\in A} |\jac_p f|\ge \vol[f(A)]\]
for any Borel subset $A\subset M$.

In particular, if $|\jac_p f|\le 1$ almost everywhere in $A$, then 
\[\vol A \ge \vol[f(A)].\]
\end{thm}

\parit{Proof.} Apply the area formula to the characteristic function of $A$.
\qeds

Suppose that $f\:\spc{M}\to\RR$ is a Lipschitz function defined on an $n$-dimensional Riemannian space $\spc{M}$.
Then by Rademacher's theorem, the differential $d_pf\:\T_p\spc{M}\to\RR$  and the gradient 
$\nabla_pf\in\T_p\spc{M}$ are defined at almost all $p\in \spc{M}$.

The identity in the following proposition is a partial case of the so-called \index{coarea formula}\emph{coarea formula}.
(The general coarea formula deals with the maps to the spaces of arbitrary dimension, not necessary $1$.)


\begin{thm}{Proposition}\label{prop:coarea}
Let $f\:\spc{M}\to\RR$ be a Lipschitz function defined on an $n$-dimensional Riemannian space $\spc{M}$.
Suppose that the level sets $L_x\df f^{-1}(x)$ are equipped with $(n-1)$-dimensional volume $\vol_{n-1}\z\df\tfrac{\omega_{n-1}}{2^{n-1}}\cdot \haus_{n-1}$.
Then for any Borel function $u\:\spc{M}\to \RR$ the following equality holds
\[\int_{p\in \spc{M}} u(p)\cdot |\nabla_pf|=\int_{-\infty}^{+\infty} \left(\,\int_{p\in L_x} u(p)\,\right)\cdot dx.\]
\end{thm}

The following corollary is a partial case of the so-called  \index{coarea inequality}\emph{coarea inequality};

\begin{thm}{Corollary}\label{cor:coarea}
Let $\spc{M}$, $f$, and $L_x$ be as in \ref{prop:coarea}.

Suppose that $f$ is 1-Lipschitz.
Then for any Borel subset $A\subset M$ we have
\[\vol_n A\ge \int_{x\in\RR} \vol_{n-1}[A\cap L_x]\cdot dx.\eqlbl{eq:coarea-inq}\]
\end{thm}

The right-hand side in \ref{eq:coarea-inq} is called \index{coarea}\emph{coarea of the restriction $f|_A$}. 


\parit{Instead of proof of \ref{prop:coarea} and \ref{cor:coarea}.}
If $\spc{M}$ is isometric to Euclidean space, then the statement follows from the standard coarea formula \cite[3.2.12]{federer}.
The reduction to the Euclidean space is done the same way as in the proof of the area formula.

To prove the corollary, choose $u$ to be the characteristic function of $A$ and apply the coarea formula.
\qeds


\section{Besicovitch inequality}

A closed connected region in a Riemannian manifold bounded by hypersurface will be called \index{Riemannian!manifold with boundary}\emph{Riemannian manifold with boundary}.
We always assume that the hypersurface can be realized locally as a graph of Lipschitz function in a suitable chart.
In this case one can define $g$-length, $g$-distance, and $g$-volume the same way as we did for usual Riemannian manifolds.

\begin{thm}{Exercise}\label{ex:compact-interior}
Suppose that $(M,g)$ is a compact Riemannian manifold with boundary. 
Observe that the interior $(M^\circ,g)$ of $(M,g)$ is a usual Riemannian manifold.
Show that the space of $(M,g)$ is isometric to the completion of the space of $(M^\circ,g)$.
\end{thm}
 

\begin{thm}{Theorem}\label{thm:besikovitch}
Let $g$ be a continuous metric tensor on a unit $n$-dimensional cube $\square$.
Suppose that the $g$-distances between the opposite faces of $\square$ are at least $1$; that is, any Lipschitz curve that connects opposite faces has $g$-length at least $1$.
Then \[\vol(\square, g)\ge 1.\]

\end{thm}

This is a partial case of the theorem proved by Abram Besicovitch \cite{besicovitch}.

\parit{Proof.}
We will consider the case $n=2$; the other cases are proved the same way.

\begin{wrapfigure}{r}{30mm}
\vskip-0mm
\centering
\includegraphics{mppics/pic-1320}
\end{wrapfigure}

Denote by $A$, $A'$, and $B$, $B'$ the opposite faces of the square~$\square$.
Consider two functions
\begin{align*}
f_A(x)&\df\min\{\,\distfun_A(x)_g,1\,\},
\\
f_B(x)&\df\min\{\,\distfun_B(x)_g,1\,\}.
\end{align*}
Let $\bm{f}\:\square\to\square$ be the map with coordinate functions $f_A$ and $f_B$;
that is, $\bm{f}(x)\df(f_A(x), f_B(x))$.

\begin{clm}{}\label{f:A->A}
The map $\bm{f}$ sends each face of $\square$ to itself.
\end{clm}


Indeed, 
\[x\in A \quad\Longrightarrow\quad \distfun_A(x)_g=0 \quad\Longrightarrow\quad f_A(x)=0 \quad\Longrightarrow\quad \bm{f}(x)\in A.\]
Similarly, if $x\in B$, then $\bm{f}(x)\in B$.
Further, 
\[x\in A'
\quad\Longrightarrow\quad 
\distfun_A(x)_g\ge 1 
\quad\Longrightarrow\quad 
f_A(x)=1 
\quad\Longrightarrow\quad 
\bm{f}(x)\in A'.\]
Similarly, if $x\in B'$, then $\bm{f}(x)\in B'$.

By \ref{f:A->A}, it follows 
\[\bm{f}_t(x)= t\cdot x + (1-t)\cdot \bm{f}(x)\]
defines a homotopy of maps of the pair of spaces $(\square,\partial \square)$ from $\bm{f}$ to the identity map;
that is, $(t,x)\mapsto \bm{f}_t(x)$ is a continuous map and if $x\in \partial \square$, then $\bm{f}_t(x)\in \partial \square$ for any $t\in [0,1]$.

It follows that $\deg\bm{f}=1$; that is, $\bm{f}$ sends the fundamental class of $(\square,\partial \square)$ to itself.%
\footnote{Here and further, we assume that homologies are taken with the coefficients in $\ZZ_2$, but you are welcome to play with other coefficients.}
In particular $\bm{f}$ is onto.

Suppose that Jacobian  matrix $\Jac_p\bm{f}$ of $\bm{f}$ is defined at $p\in \square$.
Choose an orthonormal frame in $\T_p$ with respect to $g$ and the standard frame in the target $\square$.
Observe that the differentials $d_pf_A$ and $d_pf_B$ written in these frames are the rows of $\Jac_p\bm{f}$.
Evidently $|d_pf_A|\le 1$ and $|d_pf_B|\le 1$.
Since the determinant of a matrix is the volume of the parallelepiped spanned on its rows, we get 
\[|\jac_p \bm{f}|\le |d_pf_A|\cdot|d_pf_B|\le 1.\]
Since $\bm{f}\:\square\to\square$ is a Lipschitz onto map, the area inequality (\ref{cor:area-inequality}) implies that 
\[\vol(\square,g)\ge \vol\square=1.\]
\qedsf

If the $g$-distances between the opposite sides are $d_1,\dots ,d_n$, then following the same lines  one get that 
$\vol (\square,g)\ge d_1\cdots d_n$.
Also note that in the proof we use topology of the $n$-cube only once, to show that the map $f$ has degree one.
Taking all this into account we get the following generalization of \ref{thm:besikovitch}:

\begin{thm}{Theorem}\label{thm:besikovitch+}
Let $(M,g)$ be an $n$-dimensional Riemannian manifold with coonected boundary $\partial M$.
Suppose that there is a degree 1 map $\partial M\to \partial\square$;
denote by $d_1,\dots, d_n$ the $g$-distances between the inverse images of pairs of opposite faces of $\square$ in $M$.
Then 
\[\vol(M,g)\ge d_1\cdots d_n.\]

\end{thm}

\begin{thm}{Exercise}\label{ex:besikovitch=}
Show that if equality holds in \ref{thm:besikovitch+},
then $(M,g)$ is isometric to the rectangle $[0,d_1]\times\dots\times[0, d_n]$.
\end{thm}



\begin{thm}{Exercise}\label{ex:hexagon}
Suppose $g$ is a metric tensor on a regular hexagon $\text{\rm\hexagon}$ such that $g$-distances between the opposite sides are at least $1$.
Is there a positive lower bound on $\area(\text{\rm\hexagon},g)$?
\end{thm}

\begin{thm}{Exercise}\label{ex:cylinder}
Let $g$ be a Riemannian metric on the cylinder $\mathbb{S}^1\z\times [0,1]$.
Suppose that 
\begin{itemize}
\item 
$g$-distance between pairs of points on the opposite boundary circles $\mathbb{S}^1\times\{0\}$ and $\mathbb{S}^1\times\{1\}$ is at least 1, and 
\item
any curve $\gamma$ in $\mathbb{S}^1\times [0,1]$ that is homotopic to $\mathbb{S}^1\times\{0\}$ has $g$-length at least $1$.
\end{itemize}

\begin{subthm}{ex:cylinder:besicovitch}
Use Besicovitch inequality to show that
\[\area(\mathbb{S}^1\times [0,1],g)\ge \tfrac12.\]

\end{subthm}

\begin{subthm}{ex:cylinder:coarea}
Modify the proof of Besicovitch inequality using coarea inequality (\ref{cor:coarea}) to prove the optimal bound  
\[\area(\mathbb{S}^1\times [0,1],g)\ge 1.\]
 
\end{subthm}

\end{thm}

\begin{thm}{Exercise}\label{ex:gadograph}

\begin{subthm}{ex:gadograph-besikovitch}
Generalize \ref{thm:besikovitch+} to noncontinuous metric tensor $g$ described the following way:
there are two Riemannian metric tensors $g_1$ and $g_2$ on $M$ and a subset $V\subset M$ bounded by a Lipschitz hypersurface $\Sigma$ such that 
$g=g_1$ at the points in $V$ and $g=g_2$ otherwise.
\end{subthm}



\begin{subthm}{ex:gadograph-gadograph}
Use part \ref{SHORT.ex:gadograph-besikovitch} to prove the following: 
Let $V$ be a compact set in the $n$-dimensional Euclidean space $\EE^n$ bounded by a Lipschitz hypersurface $\Sigma$.
Suppose $g$ is a Riemannian metric on $V$ such that 
\[\dist{p}{q}{g}\ge\dist{p}{q}{\EE^n}\]
for any two points $p,q\in \Sigma$.
Show that
\[\vol(V,g)\ge \vol(V)_{\EE^n}.\]
\end{subthm}

\end{thm}

\begin{thm}{Exercise}\label{ex:involution-of-sphere}
Suppose that sphere with Riemannian metric $(\mathbb{S}^2,g)$ admits an involution $\iota$ such that $\dist{x}{\iota(x)}{g}\ge 1$.

Show that 
\[\area(\mathbb{S}^2,g)\ge \tfrac1{1000}.\]
Try to show that 
\[\area(\mathbb{S}^2,g)\ge \tfrac12,
\quad \area(\mathbb{S}^2,g)\ge 1,
\quad\text{or}\quad\area(\mathbb{S}^2,g)\ge \tfrac4\pi\]

\end{thm}

\begin{thm}{Advanced exercise}\label{ex:involution-of-3sphere}
Construct a metric tensor $g$ on $\mathbb{S}^3$ such that (1) $\vol(\mathbb{S}^3,g)$ arbitrarily small and (2) there is an involution $\iota\:\mathbb{S}^3\z\to \mathbb{S}^3$ such that $\dist{x}{\iota(x)}{g}\ge 1$ for any $x\in \mathbb{S}^3$.
\end{thm}

\begin{thm}{Exercise}\label{ex:GH-vol}
Let $g_1,g_2,\dots$, and $g_\infty$ be metrics on a fixed compact manifold $M$.
Suppose that $\distfun_{g_n}$ uniformly converges to $\distfun_{g_\infty}$ as functions on $M\times M\to\RR$.
Show that 
\[\liminf_{n\to\infty}\vol(M,g_n)\ge \vol(M,g_\infty).\]

Show that the inequality might be strict.
\end{thm}

\section{Systolic inequality}

Let $\spc{M}$ be a compact Riemannian space.
The \index{systole}\emph{systole} of $\spc{M}$ (briefly $\sys\spc{M}$) is defined to be the least length of a noncontractible closed curve in $\spc{M}$.

Let $\Lambda$ be a class of closed $n$-dimensional Riemannian spaces.
We say that a \index{systolic inequality}\emph{systolic inequality} holds for $\Lambda$ if there is a constant $c$ such that 
\[\sys\spc{M}\le c\cdot \sqrt[n]{\vol\spc{M}}\]
for any $\spc{M}\in \Lambda$.

\begin{thm}{Exercise}\label{ex:sysT2}
Use \ref{thm:besikovitch} or \ref{ex:cylinder} to show that a systolic inequality holds for any Riemannian metric on the 2-torus $\TT^2$.
\end{thm}

\begin{thm}{Exercise}\label{ex:sysRP2}
Use \ref{thm:besikovitch} to show that a systolic inequality holds for any Riemannian metric on  the real projective plane $\RP^2$.
\end{thm}

\begin{thm}{Exercise}\label{ex:sysSg}
Use \ref{thm:besikovitch+} to show that systolic inequality holds for any Riemannian metric on any closed surfaces of positive genus.
\end{thm}

\begin{thm}{Exercise}\label{ex:sysS2xS1}
Show that no systolic inequality holds for Riemannian metrics on $\mathbb{S}^2\times\mathbb{S}^1$.
\end{thm}

In the following lecture we will show that systolic inequality holds for many manifolds, in particular for torus of arbitrary dimension.

\section{Generalization}\label{sec:hausdorff-measure}

The following proposition follows immediately from the definitions of Hausdorff measure (Section \ref{sec:vol-haus}).

\begin{thm}{Proposition}\label{prop:bilip-measure}
Let $\spc{X}$ and $\spc{Y}$ be metric spaces, $A\subset \spc{X}$
and
 $f\: \spc{X}\to \spc{Y}$ be a $\Lip$-Lipschitz map. 
Then 
\[\haus_\alpha [f(A)]\le \Lip^\alpha\cdot\haus_\alpha\, A\]
for any $\alpha$.
\end{thm}

The following exercise provides a weak analog of the Besicovitch inequality that works for arbitrary metrics.

\begin{thm}{Exercise}\label{ex:besikovitch++}
Let $M$ be manifold with boundary and $\rho$ is a semimetric on $M$.
Suppose $\partial M$ admits a degree 1 map to the surface of the $n$-dimensional cube $\square$;
denote by $d_1,\dots, d_n$ the $\rho$-distances between the inverse images of pairs of opposite faces of $\square$ in $M$.
Then 
\[\haus_n(M,\rho)\ge d_1\cdots d_n.\]
\end{thm}


Recall that in $n$-dimensional Riemannian spaces we have 
\[\tfrac{\omega_n}{2^n}\cdot\haus_n=\vol_n.\]
Note that $\tfrac{\omega_n}{2^n}<1$ if $n\ge 2$.
Therefore, the conclusion in \ref{ex:besikovitch++} is weaker than in \ref{thm:besikovitch+} (the assumptions are weaker as well).

One can redefine systolic inequality on $n$-dimensional manifolds using the Hausdorff measure $\haus_n$ instead of the volume.
It is straightforward to prove analogs of the exercises \ref{ex:sysT2}--\ref{ex:sysS2xS1} with this definition.

\begin{thm}{Exercise}\label{ex:2top-discs}
Suppose that two embedded $n$-disks $\Delta_1,\Delta_2$ in a metric space $\spc{X}$ have identical boundaries.
Assume that $\spc{X}$ is contractible and $\haus_{n+1}\spc{X}=0$.
Show that $\Delta_1=\Delta_2$.
\end{thm}

\section{Remarks}\label{sec:besicovitch-remarks}

The optimal constants in the systolic inequality are known only in the following three cases:
\begin{itemize}
\item For real projective plane $\RP^2$ the constant is $\sqrt{\pi/2}$ --- the equality holds for a quotient of a round sphere by isometric involution. The statement was proved by Pao Ming Pu \cite{pu}.\label{page:pu}
\item For torus $\TT^2$ the constant is $\sqrt{2}/\sqrt[4]{3}$ --- the equality holds for a flat torus obtained from a regular hexagon by identifying opposite sides; this is the so-called \index{Loewner's torus inequality}\emph{Loewner's torus inequality}.
\item For the Klein bottle $\RP^2\#\RP^2$  the constant is $\sqrt{\pi}/2^{3/4}$ --- the equality holds for a certain nonsmooth metric.
The statement was proved by Christophe Bavard \cite{bavard}.
\end{itemize}
The proofs of these results use the so-called {}\emph{uniformization theorem}   available in the 2-dimensional case only.
These proofs are beautiful, but they are too far from metric geometry.
A good survey on the subject is written by Christopher Croke and Mikhail Katz \cite{croke-katz}.

An analog of Exercise \ref{ex:GH-vol} with Hausdorff measure instead of volume does not hold for general metrics on a manifold.
In fact there is a nondecreasing sequence of metric tensors $g_n$ on $M$, such that (1) $\vol(M,g_n)<1$ for any $n$ and (2) $\distfun_{g_n}$ converges to a metric on $M$ with arbitrary large Hausdorff measure of any given dimension; such examples were constructed by Dmitri Burago, Sergei Ivanov, and David Shoenthal \cite{burago-ivanov-shoenthal}.

\chapter{Width and systole}

This lecture is based on the paper of Alexander Nabutovsky \cite{nabutovsky}.

\section{Partition of unity}

\begin{thm}{Proposition}\label{thm:part-unit}
 Let $\{V_i\}$ be a finite open covering of a compact metric space ${\spc{X}}$.
Then there are Lipschitz functions $\psi_i\:{\spc{X}}\z\to[0,1]$ such that (1) if $\psi_i(x)>0$, then $x\in V_i$ and (2) for any $x\in {\spc{X}}$ we have
$$\sum_i\psi_i(x)=1.$$

\end{thm}

A collection of functions $\{\psi_i\}$ that meets the conditions in \ref{thm:part-unit} is called 
a \index{partition of unity}\emph{partition of unity} subordinate to the covering $\{V_i\}$.

\parit{Proof.}
Denote by $\phi_i(x)$ the distance from $x$ to the complement of $V_i$;
that is,
$$\phi_i(x)=\distfun_{{\spc{X}}\setminus V_i}(x).$$
Note $\phi_i$ is $1$-Lipschitz
for any $i$
and $\phi_i(x)>0$ if and only if $x\in V_i$.
Since $\{V_i\}$ is a covering, we have that
$$\Phi(x)\df\sum_i\phi_i(x)>0\ \ \text{for any}\ \ x\in {\spc{X}}.$$
Since $\spc{X}$ is compact, $\Phi>\delta$ for some $\delta>0$.
It follows that $x\mapsto\tfrac1{\Phi(x)}$ is a bounded Lipschitz function. 

Set 
$$\psi_k(x)=\frac{\phi_k(x)}{\Phi(x)}.$$
Observe that by construction the functions $\psi_i$ meet the conditions in the proposition.
\qedsf

\section{Nerves}

Let $\{V_1,\dots,V_k\}$ be a finite open cover of a compact metric space $\spc{X}$.
Consider an abstract simplicial complex $\spc{N}$, with one vertex $v_i$ for each set $V_i$ such that a simplex with vertices $v_{i_1},\dots, v_{i_m}$ is included in $\spc{N}$ if 
the intersection $V_{i_1}\cap\dots\cap V_{i_m}$ is nonempty.
\begin{figure}[ht!]
\vskip-0mm
\centering
\includegraphics{mppics/pic-1402}
\end{figure}
The obtained simplicial complex $\spc{N}$ is called the \index{nerve}\emph{nerve} of the covering $\{V_i\}$.
Evidently $\spc{N}$ is a finite simplicial complex ---
it is a subcomplex of a simplex with the vertices $\{v_1,\dots,v_k\}$.

Note that the nerve $\spc{N}$ has dimension at most $n$ if and only if the covering $\{V_1,\dots,V_k\}$ has \index{multiplicity of covering}\emph{multiplicity} at most $n+1$;
that is, any point $x\in\spc{X}$ belongs to
at most $n+1$ sets of the covering.

Suppose $\{\psi_i\}$ is  
a partition of unity subordinate to the covering $\{V_1,\dots,V_k\}$.
Choose a point $x\in {\spc{X}}$.
Note that the set
$$\{v_{i_1},\dots,v_{i_n}\}=\set{v_i}{\psi_i(x)>0}$$
form vertices of a simplex in $\spc{N}$.
Therefore 
$$\bm{\psi}\:x\mapsto \psi_1(x)\cdot v_1+\psi_2(x)\cdot v_2+\dots+\psi_k(x)\cdot v_n.$$
describes a Lipschitz map from ${\spc{X}}$ to the nerve $\spc{N}$ of $\{V_i\}$.
In other words, $\bm{\psi}$ maps a point $x$ to the point in $\spc{N}$ with \index{barycentric coordinates}\emph{barycentric coordinates} $(\psi_1(x),\dots,\psi_k(x))$.

Recall that the \index{star}\emph{star} of a vertex $v_i$ (briefly $\Star_{v_i}$) is defined as the union of the interiors of all simplicies that have $v_i$ as a vertex.
Recall that $\psi_i(x)>0$ implies $x\in V_i$.
Therefore we get the following:

\begin{thm}{Proposition}\label{prop:space->nerve}
Let $\spc{N}$ be a nerve of an open covering $\{V_1,\z\dots,V_k\}$ of a compact metric space $\spc{X}$.
Denote by $v_i$ the vertex of $\spc{N}$ that corresponds to $V_i$.

Then there is a Lipschitz map $\bm{\psi}\:\spc{X}\to\spc{N}$ such that $\bm{\psi}(V_i)\z\subset\Star_{v_i}$ for every $i$.
\end{thm}


\section{Width}

Suppose $A$ is a subset of a metric space $\spc{X}$.
The radius of $A$ (briefly $\rad A$) is defined as the least upper bound on the values $R>0$ such that $\oBall(x,R)\supset A$ for some $x\in \spc{X}$.

\begin{thm}{Definition}\label{def:width}
Let $\spc{X}$ be a metric space.
The \index{width}\emph{$n$-th width} of $\spc{X}$ (briefly $\width_n\spc{X}$) is the least upper bound on values $R>0$ such that $\spc{X}$ admits a finite open covering $\{V_i\}$ with multiplicity at most $n+1$ and $\rad V_i< R$ for each $i$.
\end{thm}

\parbf{Remarks.}

\begin{itemize} 
\item Observe that 
\[\width_0\spc{X}\ge\width_1\spc{X}\ge\width_2\spc{X}\ge\dots\]
for any compact metric space $\spc{X}$.
Moreover, if $\spc{X}$ is connected, then 
\[\width_0\spc{X}=\rad\spc{X}.\]

\item Usually width is defined using diameter instead of radius, but the results differ at most twice.
Namely, if $r$ is the $n$-th radius-width and $d$ --- the $n$-th diameter-width, then 
$r\le d\le 2\cdot r$.

\item Note that \index{Lebesgue covering dimension}\emph{Lebesgue covering dimension} of $\spc{X}$ can be defined as the least number $n$ such that $\width_n\spc{X}=0$.

\item Another closely related notion is the so-called \index{macroscopic dimension}\emph{macroscopic dimension on scale $R$};
it is defined as the  least number $n$ such that $\width_n\spc{X}<R$.
\end{itemize}



\begin{thm}{Exercise}\label{ex:macrodimension}
Suppose $\spc{X}$ is a compact metric space such that any closed curve $\gamma$ in $\spc{X}$ can be contracted in its $R$-neighborhood.
Show that macroscopic dimension of $\spc{X}$ on scale $100\cdot R$ is at most 1.

What about quasiconverse? That is, suppose a simply connected compact metric space $\spc{X}$ has macroscopic dimension at most 1 on scale $R$, is it true that any closed curve $\gamma$ in $\spc{X}$ can be contracted in its $100\cdot R$-neighborhood?
\end{thm}


The following exercise gives a good reason for the choice of term \index{width}\emph{width}; it also can be used as an alternative definition.

\begin{thm}{Exercise}\label{ex:width=suprad(inv)}
Suppose $\spc{X}$ is a compact metric space.
Show that $\width_n\spc{X}<R$ if and only if there is a finite $n$-dimensional simplicial complex $\spc{N}$ and a continuous map $\bm{\psi}\:\spc{X}\to \spc{N}$
such that 
\[\rad[\bm{\psi}^{-1}(s)]<R\]
for any $s\in \spc{N}$.
\end{thm}

\section{Riemannian polyhedrons}

A \index{Riemannian!polyhedron}\emph{Riemannian polyhedron} is defined as a finite simplicial complex with a metric tensor on each simplex such that the restriction of the metric tensor to a subsimplex coincides with the metric on the subsimplex.

The {}\emph{dimension} of a Riemannian polyhedron is defined as the largest dimension in its triangulation.
For Riemannian polyhedrons one can define length of curves and volume the same way as for Riemannian manifolds.

The obtained metric space will be called \emph{Riemannian polyhedron} as well.
A \index{triangulation}\emph{triangulation} of Riemannian polyhedron  will always be assumed to have the above property on the metric tensor.

Further we will apply the notion of width only to compact Riemannian polyhedrons.
If $\spc{P}$ is an $n$-dimensional Riemannian polyhedron, then 
we suppose that
\[\width\spc{P}\df\width_{n-1}\spc{P}.\]


Suppose that $\spc{P}$ is an $n$-dimensional Riemannian polyhedron;
in this case we will use short cut $\vol$ for $\vol_n$.
Let us define \index{volume profile}\emph{volume profile} of $\spc{P}$ as a function 
returning largest volume of $r$-ball in~$\spc{P}$;
that is, the volume profile of $\spc{P}$ is a function $\VolPro_{\spc{P}}\:\RR_+\to\RR_+$ defined by 
\[\VolPro_{\spc{P}}(r)\df \sup\set{\vol \oBall(p,r)}{p\in\spc{P}}.\]
Note that 
$r\mapsto \VolPro_{\spc{P}}(r)$ is nondecreasing  and
\[\VolPro_{\spc{P}}(r)\le\vol\spc{P}\]
for any $r$.
Moreover, if $\spc{P}$ is connected, then the equality $\VolPro_{\spc{P}}(r)\z=\vol\spc{P}$ holds
for $r\ge \rad \spc{P}$.

Note that if $\spc{P}$ is a connected 1-dimensional Riemannian polyhedron, then 
\[\width\spc{P}=\width_0\spc{P}=\rad\spc{P}.\]

\begin{thm}{Exercise}\label{ex:1D-case}
Let $\spc{P}$ be a 1-dimensional Riemannian polyhedron.
Suppose that $\VolPro_{\spc{P}}(R)<R$ for some $R>0$.
Show that 
\[\width \spc{P}<R.\]
Try to show that $c=\tfrac 12$ is the optimal constant for which the following inequality holds: 
\[\width \spc{P}<c\cdot R.\]
\end{thm}

\section{Volume profile bounds width}

\begin{thm}{Theorem}\label{thm:width<volpro}
Let $\spc{P}$ be an $n$-dimensional Riemannian polyhedron. 
If the inequality 
\[R> n\cdot \sqrt[n]{\VolPro_{\spc{P}}(R)}\]
holds for some $R>0$, then 
\[\width\spc{P}\le  R.\]
\end{thm}

Since $\VolPro_{\spc{P}}(R)\le \vol\spc{P}$ for any $R>0$,
we get the following:

\begin{thm}{Corollary}\label{thm:width<vol}
For any $n$-dimensional Riemannian polyhedron $\spc{P}$, we have
\[\width\spc{P}\le n\cdot \sqrt[n]{\vol\spc{P}}.\]

\end{thm}

The proof of \ref{thm:width<volpro} will be given at the very end of this section,
after discussing {}\emph{separating polyhedrons}. 

Let us start three technical statements.
The first statement can be obtained by modifying a smoothing procedure for functions defined on Euclidean space. 

A function $f$ defined on a Riemannian polyhedron $\spc{P}$ is called \index{piecewise smooth}\emph{piecewise smooth} if there is a triangulation of $\spc{P}$ such that restriction of $f$ to every simplex is smooth.


\begin{thm}{Smoothing procedure}\label{smoothing-procedure}
Let $\spc{P}$ be a Riemannian polyhedron and $f\:\spc{P}\to \RR$ be a 1-Lipschitz function.
Then for any $\delta>0$ there is a piecewise smooth 1-Lipschitz function $\tilde f\:\spc{P}\to \RR$ such that 
\[|\tilde f(x)-f(x)|<\delta\]
for any $x\in  \spc{P}$.
\end{thm}

The following statement can be proved by applying the classical Sard's theorem to each simplex of a Riemannian polyhedron.

\begin{thm}{Sard's theorem}\label{sard}\index{Sard's theorem}
Let $\spc{P}$ be an $n$-dimensional Riemannian polyhedron and $f\:\spc{P}\to \RR$ be a piecewise smooth function.
Then for almost all values $a\in\RR$, the inverse image $f^{-1}\{a\}$  is a Riemannian polyhedron of dimension at most $n-1$ (we assume that $f^{-1}\{a\}$ is equipped with the induced length metric).
\end{thm}

The following statement can be proved by applying the coarea inequality (\ref{cor:coarea}) to the restriction of $f$ to each simplex of the polyhedron and summing up the results.

\begin{thm}{Coarea inequality}\index{coarea inequality}\label{poly-coarea}
Let $\spc{P}$ be an $n$-dimensional Riemannian polyhedron and $f\:\spc{P}\to \RR$ be a piecewise smooth 1-Lipschitz function.
Set $v\z=\vol_n (f^{-1}[r,R])$ and $a(t)=\vol_{n-1}(f^{-1}\{t\})$.
Then 
\[\int_r^Ra(t)\cdot dt\ge v .\]
In particular, there is a subset of positive measure $S\subset [r,R]$ such that the inequality 
\[a(t)\ge \frac v{R-r}\]
holds for any $t\in S$.
\end{thm}

\section*{Separating subpolyhedrons}

\begin{thm}{Definition}
Let $\spc{P}$ be an $n$-dimensional Riemannian polyhedron.
An $(n-1)$-dimensional subpolyhedron $\spc{Q}\subset\spc{P}$ is called \index{separating subpolyhedron}\emph{$R$-separating} if for each connected component $U$ of the complement $\spc{P}\setminus \spc{Q}$ we have 
\[\rad U<R.\]

\end{thm}



\begin{thm}{Lemma}\label{lem:separating}
Let $\spc{P}$ be an $n$-dimensional Riemannian polyhedron.
Then given $R>0$ and $\eps>0$ there is a $R$-separating subpolyhedron $\spc{Q}\subset\spc{P}$ such that for any $r_0<r_1\le R$ we have
\[\VolPro_{\spc{Q}}(r_0)< \tfrac1{r_1-r_0}\cdot \VolPro_{\spc{P}}(r_1)+\eps.\]

\end{thm}

The proof reminds the proof of the following statement about minimal surfaces: 
\textit{if a point $p$ lies on an compact area-minimizing surface $\Sigma$ and $\partial\Sigma \cap \oBall(p,r)=\emptyset$, then
\[\area(\Sigma\cap \oBall(p,r))\le \tfrac12\cdot \area\mathbb{S}^2\cdot r^2.\]
}


\parit{Proof.}
Choose a small $\delta>0$.
Applying the smoothing procedure (\ref{smoothing-procedure}), we can exchange each distance function $\distfun_p$ on $\spc{P}$ by $\delta$-close piecewise smooth 1-Lipschitz function, which will be denoted by $\widetilde \distfun_p$.

By Sard's theorem (\ref{sard}), for almost all values $c\z\in(r_0\z+\delta, r_1-\delta)$, the level set
\[\tilde S_c(p)=\set{x\in \spc{P}}{\widetilde \distfun_p(x)=c}\]
is a Riemannian polyhedron of dimension at most $n-1$.
Since $\delta$ is small, the coarea inequality (\ref{poly-coarea}) implies that $c$ can be chosen so that in addition the following inequality holds:
\begin{align*}
\vol_{n-1}\tilde S_c(p)&\le \tfrac1{r_1-r_0-2\cdot\delta}\cdot\vol_n[\oBall(p,r_1)]<
\\
&<\tfrac1{r_1-r_0}\cdot \VolPro_{\spc{P}}(r_1)+\tfrac\eps2.
\end{align*}

Suppose $\spc{Q}$ is an $R$-separating subpolyhedron in $\spc{P}$ with almost minimal volume;
say its volume is at most $\tfrac\eps2$-far from the greatest lower bound.
Note that cutting from $\spc{Q}$ everything inside $\tilde S_c(p)$ and adding $\tilde S_c(p)$ produces a $R$-separating subpolyhedron, say $\spc{Q}'$.%
\footnote{If $\dim\tilde S_c(p)<n-1$, then it might happen that $\dim\spc{Q}'<n-1$; so, by the definition, $\spc{Q}'$ is not separating.
It can be fixed by adding a tiny $(n-1)$-dimensional piece to $\spc{Q}'$.}

Since $\spc{Q}$ has almost minimal volume, we have
\[\vol_{n-1}[\spc{Q}\cap \oBall(p,r_0)_{\spc{P}}]-\tfrac\eps2\le \vol_{n-1}S_c(p).\]
Therefore 
\[\vol_{n-1}[\spc{Q}\cap \oBall(p,r_0)_{\spc{P}}]\le\tfrac1{r_1-r_0}\cdot \VolPro_{\spc{P}}(r_1)+\eps.
\eqlbl{eq:volQ<ProP}\]
Recall that $\spc{Q}$ is equipped with the induced length metric;
therefore $\dist{p}{q}{\spc{Q}}\ge \dist{p}{q}{\spc{P}}$ for any $p,q\in \spc{Q}$;
in particular, 
\[\oBall(p,r_0)_{\spc{Q}}\subset \spc{Q}\cap \oBall(p,r_0)_{\spc{P}}\]
for any $p\in \spc{Q}$ and $r_0\ge 0$.
Hence, \ref{eq:volQ<ProP} implies the lemma.
\qeds

\begin{thm}{Lemma}\label{lem:separating-width}
Let $\spc{Q}$ be an $R$-separating subpolyhedron in an $n$-dimensional Riemannian polyhedron $\spc{P}$.
Then 
\[\width\spc{Q}\le R
\quad\Longrightarrow\quad
\width\spc{P}\le R.\]
\end{thm}

\parit{Proof.}
Choose an open covering $\{V_1,\dots,V_k\}$ of $\spc{Q}$ as in the definition of width (\ref{def:width});
that is, it has multiplicity at most $n$ and $\rad V_i<R$ for any $i$. 

Note that $\{V_1,\dots,V_k\}$ can be converted into an open covering of
a small neighbourhood of $\spc{Q}$ in $\spc{P}$ without increasing the multiplicity.
This can be done by setting 
\[V_i'=\bigcup_{x\in V_i}\oBall(x,r_x),\]
where $r_x\df\tfrac1{10}\cdot\inf\set{\dist{x}{y}{}}{y\in \spc{Q}\setminus V_i}$.

By adding to  $\{V_i'\}$ all the components of $\spc{P}\setminus \spc{Q}$,
we increase the multiplicity by at most 1 and obtain a covering of $\spc{P}$.
The statement follows since $\dim \spc{P}= \dim \spc{Q}\z+1$.
\qeds

\section*{Proof assembling}

\parit{Proof of \ref{thm:width<volpro}.}
We apply induction on the dimension $n=\dim\spc{P}$.
The base case $n=1$ is given in \ref{ex:1D-case}.

Suppose that the  $(n-1)$-dimensional case is proved.
Consider an $n$-dimensional Riemannian polyhedron $\spc{P}$ and suppose
\[n\cdot \sqrt[n]{\VolPro\spc{P}(R)}< R\]
for some $R>0$.
Let $\spc{Q}$ be an $R$-separating subpolyhedron in $\spc{P}$ provided by \ref{lem:separating} for a small $\eps>0$.

Applying  \ref{lem:separating} for $r=\tfrac{n-1}n\cdot R$ and $R$, we have that 
\begin{align*}
\VolPro_\spc{Q}(r) &< \frac 1{R-r}\cdot \VolPro_\spc{P}(R)+\eps<
\\
&<\frac {n}{R}\cdot\left(\frac{R}{n}\right)^n=
\\
&=\left(\frac{r}{n-1}\right)^{n-1};
\end{align*}
that is, $(n-1)\cdot \sqrt[n-1]{\VolPro\spc{Q}(r)}< r$.
Since $\dim\spc{Q}= n-1$, by the induction hypothesis, we get that
\[\width\spc{Q}\le r<R.\]
It remains to apply \ref{lem:separating-width}.
\qeds





\section{Width bounds systole}

Recall that a topological space $K$ is called \index{aspherical space}\emph{aspherical} if any continuous map $\mathbb{S}^k\to K$ for $k\ge 2$ is null-homotopic.

\begin{thm}{Theorem}\label{thm:sys<width}
Suppose $\spc{M}$ is a compact aspherical $n$-dimensional Riemannian manifold.
Then 
\[\sys\spc{M}\le 6 \cdot \width \spc{M}.\]
\end{thm}

\begin{thm}{Lemma}\label{lem:aspherical-homotopy}
Let $K$ be an aspherical space and $\spc{W}$ a connected CW-complex.
Denote by $\spc{W}^k$ the k-skeleton of $\spc{W}$.
Then any continuous map $f\:\spc{W}^2\to K$ can be extended to a continuous map $\bar f\:\spc{W}\to K$

Moreover, if $p\in \spc{W}$ is a 0-cell and $q\in K$.
Then a continuous maps of pairs $\phi_0,\phi_1\:(\spc{W},p)\to(K,q)$ are homotopic if and only if $\phi_0$ and $\phi_1$ induce the same homomorphism on fundamental groups $\pi_1(\spc{W},p)\to\pi_1(K,q)$.
\end{thm}

\parit{Proof.}
Since $K$ is aspherical, any continuous map $\partial\mathbb{D}^n\to K$ for $n\ge 3$
is hull-homotopic;
that is, it can be extended to a map $\mathbb{D}^n\:\to K$.

It makes it possible to extend $f$ to $\spc{W}^3$, $\spc{W}^4$, and so on.
Therefore $f$ can be extended to whole $\spc{W}$.

The only-if part of the second part of lemma is trivial;
it remains to show the if part.

Sine $\spc{W}$ is connected, we can assume that $p$ forms the only 0-cell in $\spc{W}$;
otherwise, we can collapse a maximal subtree of the 1-skeleton in $\spc{W}$ to $p$.
Therefore, $\spc{W}^1$ is formed by loops that generate $\pi_1(\spc{W},p)$.

By assumption, the restrictions of $\phi_0$ and $\phi_1$ to $\spc{W}^1$ are homotopic.
In other words the homotopy $\Phi\:[0,1]\times \spc{W}$ is defined on the 2-skeleton of $[0,1]\times \spc{W}$.
It remains to apply the first part of the lemma to the product $[0,1]\times \spc{W}$.
\qeds



\begin{thm}{Lemma}\label{lem:sys-homotopy}
Suppose $\gamma_0,\gamma_1$ are two paths between points in a Riemannian space $\spc{M}$ such that $\dist{\gamma_0(t)}{\gamma_1(t)}{\spc{M}}<r$ for any $t\in[0,1]$.
Let $\alpha$ be a shortest path from $\gamma_0(0)$ to $\gamma_1(0)$ and $\beta$ be a shortest path from $\gamma_0(1)$ to $\gamma_1(1)$. 
If $2\cdot r<\sys\spc{M}$, then there is a homotopy $\gamma_t$ from
$\gamma_0$ to $\gamma_1$ such that $\alpha(t)\equiv \gamma_t(0)$ and $\beta(t)\equiv \gamma_t(1)$.
\end{thm}

\parit{Proof.}
Set $s=\sys\spc{M}$; 
since $2\cdot r<s$, we have that $\eps=\tfrac1{10}(s-2\cdot r)>0$.

\begin{wrapfigure}{o}{34mm}
\vskip-0mm
\centering
\includegraphics{mppics/pic-1405}
\end{wrapfigure}

Note that we can assume that $\gamma_0$ and $\gamma_1$ are rectifiable;
if not we can homotopy each into a broken geodesic line kipping the assumptions true. 

Choose a fine partition $0\z=t_0\z<t_1\z<\z\dots\z<t_n=1$.
Consider a sequence of shortest paths $\alpha_i$ from $\gamma_0(t_i)$ to $\gamma_1(t_i)$.
We can assume that $\alpha_0=\alpha$, $\alpha_n=\beta$, and each arc $\gamma_j|_{[t_{i-1},t_i]}$ has length smaller than $\eps$.
Therefore, every quadrilateral formed by concatenation  of $\alpha_{i-1}$, $\gamma_1|_{[t_{i-1},t_i]}$, reversed $\alpha_i$, and reversed arc $\gamma_0|_{[t_{i-1},t_i]}$ has length smaller than $s$.
It follows that this curve is contractible.
Applying this observation for each quadrilateral, we get the statement.
\qeds


\parit{Proof of \ref{thm:sys<width}.}
Let $\spc{N}$ be the nerve of a covering $\{V_i\}$ of $\spc{M}$ and $\bm{\psi}\:\spc{M}\to\spc{N}$ be the map provided by \ref{prop:space->nerve}.
As usual, we denote by $v_i$ the vertex of $\spc{N}$ that corresponds to $V_i$.
Observe that $\dim\spc{N}<n$;
therefore, $\bm{\psi}$ kills the fundamental class of $\spc{M}$.

Let us construct a continuous map  $f\:\spc{N}\to  \spc{M}$ such that
$f\circ\bm{\psi}$ is homotopic to the identity map on $\spc{M}$.
Note that once $f$ is constructed, the theorem is proved.
Indeed, since $\bm{\psi}$ kills the fundamental class $[\spc{M}]$ of $\spc{M}$, so does $f\circ\bm{\psi}$.
Therefore, $[\spc{M}]=0$ --- a contradiction.

Set $R=\width \spc{M}$ and $s=\sys\spc{M}$.
Assume we choose $\{V_i\}$ as in the definition of width (\ref{def:width}).
For each $i$ choose a point $p_i\in \spc{M}$ such that $V_i\subset \oBall(p_i,R)$.

Set $f(v_i)=p_i$ for each $i$.
It defines the map $f$ on the 0-skeleton $\spc{N}^0$ of the nerve $\spc{N}$.
Further, $f$ will be defined step by step on the skeletons $\spc{N}^1,\spc{N}^2, \dots$ of $\spc{N}$.

Let us map each edge $[v_iv_j]$ in $\spc{N}$ to a shortest path $[p_ip_j]$.
It defines $f$ on $\spc{N}^1$.
Note that image of each edge is shorter than $2\cdot R$.

Suppose $[v_iv_jv_k]$ is a triangle in $\spc{N}$.
Note that perimeter of the triangle $[p_ip_jp_k]$ can not exceed $6\cdot R$.
Since $6\cdot R<s$, the contour of $[p_ip_jp_k]$ is contractible.
Therefore, we can extend $f$ to each triangle of~$\spc{N}$.
It defines the map $f$ on $\spc{N}^2$.

Finally, since $\spc{M}$ is aspherical, by \ref{lem:aspherical-homotopy}, the map $f$ can be extended to $\spc{N}^3$, $\spc{N}^4$ and so on.

It remains to show that $f\circ\bm{\psi}$ is homotopic to the identity map.
Choose a CW structure on $\spc{M}$ with sufficiently small cells, so that each cell lies in one of $V_i$.
Note that $\bm{\psi}$ is homotopic to a map $\bm{\psi}_1$ that sends $\spc{M}^k$ to $\spc{N}^k$ for any $k$.
Moreover, we may assume that (1) if a 0-cell $x$ of $\spc{M}$ maps to a $v_i$, then $x\in V_i$ and (2) each 1-cell  of $\spc{M}$ maps to an edge or a vertex of $\spc{N}$.
Choose a 1-cell $e$ in $\spc{M}$; by the construction, $f\circ\bm{\psi}_1$ maps $e$ to a shortest path $[p_ip_j]$ and $e$ lies $\oBall(p_i,R)$.
Observe that $[p_ip_j]$ is shorter than $2\cdot R$.
It follows that the distance between points on $[p_ip_j]$ and $e$ can not exceed $3\cdot R$.
Choose a shortest path $\alpha_i$ from every 0 cell $x_i$  of $\spc{M}$ to $p_j=f\circ\bm{\psi}_1(x_i)$.
It defines a homotopy on $\spc{M}^0$.
Since $6\cdot R<s$, \ref{lem:sys-homotopy} implies that this homotopy can be extended to $\spc{M}^1$.
By \ref{lem:aspherical-homotopy}, it can be extended to whole $\spc{M}$.
\qeds

\begin{thm}{Exercise}\label{ex:sys<width}
Analyze the proof of \ref{thm:sys<width} and improve its inequality to 
 \[\sys\spc{M}\le 4 \cdot \width \spc{M}.\]
\end{thm}

\begin{thm}{Exercise}\label{ex:fillrad-inj}
Modify the proof of \ref{thm:sys<width} to prove the following:

Suppose that $\spc{M}$ is a closed $n$-dimensional Riemannian manifold with \index{injectivity radius}\emph{injectivity radius} at least $r$; that is, if $\dist{p}{q}{\spc{M}}<r$, then a shortest path $[pq]_{\spc{M}}$ is uniquely defined.
Show that
\[\width\spc{M}\ge \tfrac{r}{2\cdot(n+1)}.\]

Use \ref{thm:width<vol} to conclude that $\vol\spc{M}\ge \eps_n \cdot r^n$
for some $\eps_n>0$ that depends only on $n$.
\end{thm} 

The second statement in the exercise is a theorem of Marcel Berger~\cite{berger-n};
an inequality with optimal constant (with equality for round sphere) was obtained by Marcel Berger and Jerry Kazdan \cite{berger-kazdan}. 


\section{Essential manifolds}

To generalize \ref{thm:sys<width} further, we need the following definition.

\begin{thm}{Definition}\label{def:essential}
A closed manifold $M$ is called \index{essential manifold}\emph{essential} if it admits a continuous map $\iota\:M\to K$ to an aspherical CW-complex $K$ such that $\iota$ sends the fundamental class of $M$ to a nonzero homology class in $K$.
\end{thm}

Note that any closed aspherical manifold is essential --- in this case one can take $\iota$ to be the identity map on $M$.

The real projective space $\RP^n$ provides an interesting example of an essential manifold which is not aspherical.
Indeed, the infinite dimensional projective space $\RP^\infty$ is aspherical and for the natural embedding $\RP^n\hookrightarrow\RP^\infty$ the image $\RP^n$ does not bound in $\RP^\infty$.
The following exercise provides more examples of that type:

\begin{thm}{Exercise}\label{ex:connected-sum-essential}
Show that the connected sum of an essential manifold with any closed manifold is essential.
\end{thm}

\begin{thm}{Exercise}\label{ex:product-essential}
Show that the product of two essential manifolds is essential.
\end{thm}

Assume that the manifold $M$ is essential and $\iota \:M\to K$ as in the definition.
Following the proof of \ref{thm:sys<width}, we can homotope the map 
$f\circ\bm{\psi}\:M\to M$ to the identity on the 2-skeleton of $M$;
further since $K$ is aspherical, we can homotope the composition
$\iota\z\circ f\circ\bm{\psi}$ to  $\iota$. 
Existence of this extension implies that $\iota$ kills the fundamental class of $M$ --- a contradiction.
So, taking \ref{ex:sys<width} into account, we proved the following generalization of \ref{thm:sys<width}:

\begin{thm}{Theorem}\label{thm:sys<width++}
Suppose $\spc{M}$ is an essential Riemannian space.
Then 
\[\sys\spc{M}\le 4 \cdot \width \spc{M}.\]
\end{thm}

As a corollary from \ref{thm:sys<width++} and \ref{thm:width<vol} we get the so-called  Gromov's \index{systolic inequality}\emph{systolic inequality}:

\begin{thm}{Theorem}\label{thm:sys+}
Suppose $\spc{M}$ is an essential $n$-dimensional Riemannian space.
Then 
\[\sys\spc{M}\le 4 \cdot n\cdot \sqrt[n]{\vol\spc{M}}.\]
\end{thm}


\section{Remarks}

Theorem \ref{thm:sys+} was proved originally by Mikhael Gromov \cite{gromov-1983} with a worse constant.
The given proof is a result of a sequence of simplifications given by Larry Guth \cite{guth},
Panos Papasoglu \cite{papasoglu},
Alexander Nabutovsky and Roman Karasev \cite{nabutovsky}.

The calculations could be done better; namely we could get
\[\width\spc{P}\le c_n\cdot \sqrt[n]{\vol\spc{P}},\]
where
$c_n=\sqrt[n]{n!/2}= \tfrac ne+o(n)$ \cite{nabutovsky}.
As a result, we may get a stronger statement in \ref{thm:sys+}:
\[\sys\spc{M}\le 4 \cdot c_n\cdot \sqrt[n]{\vol\spc{M}}.\]

For any nonessential oriented manifold $M$ there is a metric with fixed volume and arbitrary small systole.
This statement is proved by Ivan Babenko \cite{babenko}.

A wide open conjecture says that for any $n$-dimensional essential manifold we have
\[\frac{\sys\spc{M}}{\sqrt[n]{\vol\spc{M}}}\le\frac{\sys\RP^n}{\sqrt[n]{\vol\RP^n}},\eqlbl{eq:RPn}\]
where we assume that the $n$-dimensional real projective space $\RP^n$ is equipped with a canonical metric.
In other words, the ratio in the right-hand side of \ref{eq:RPn} is the optimal constant in the Gromov's systolic inequality; this  ratio grows as $O(\sqrt n)$.
(The ratio for $n$-dimensional flat torus grows as $O(\sqrt n)$ as well.)

%\chapter{Volume bounds filling radius}

This chapter 
is devoted to a proof of \ref{thm:FillRad<vol};
that is, we will show that \emph{Riemannian manifolds with small volume have small filling radius}.
This theorem was proved originally by Mikhael Gromov \cite{gromov-1983}.
We follow closely a simplified proof given by Alexander Nabutovsky, which is based on a sequence of other simplifications and improvements; see \cite{nabutovsky} and the references there in.

\section{Nerves and partition of unity}

Let $\{V_1,\dots,V_k\}$ be a finite open cover of a compact metric space $\spc{X}$.
Consider the abstract simplicial complex $\spc{N}$, with one vertex $v_i$ for each set $V_i$ such that a simplex with vertexes $v_{i_1},\dots, v_{i_k}$ is included in $\spc{N}$ if 
the intersection $V_{i_1}\cap\dots\cap V_{i_m}$ is nonempty.
We obtain a simplicial complex $\spc{N}$ called the \index{nerve}\emph{nerve of the covering $\{V_i\}$}.

Note that $\spc{N}$ is a finite simplicial complex and it has dimension at most $n$ if and only if the covering $\{V_1,\dots,V_k\}$ has multiplicity is at most $n+1$;
that is, at most $n+1$ different sets $V_{i_1},\dots, V_{i_{n+1}}$ have a nonempty intersection.
The nerve $\spc{N}$ is a subcomplex of a simplex with the vertixes $v_1,\dots,v_k\}$.

\begin{thm}{Proposition}\label{thm:part-unit}
 Let $\{V_1,\dots,V_k\}$ is a finite open covering of a compact metric space ${\spc{X}}$.
Then there are Lipschitz functions $\psi_i\:{\spc{X}}\to[0,1]$ such that
if $\psi_i(x)>0$ then $x\in V_i$ and
$$\sum_i\psi_i(x)=1$$
for any $x\in {\spc{X}}$.
\end{thm}

A collection of functions $\psi_i$ with above properies is called 
a \emph{partition of unity subordinate to the open cover}\index{partition of unity} $\{V_1,\dots,V_k\}$.

\parit{Proof.}
Consider the functions $\phi_i\:{\spc{X}}\to\RR$ defined as
$$\phi_i(x)=\distfun_{{\spc{X}}\backslash V_i} x.$$
Note $\phi_i$ is $1$-Lipschitz
for any $i$
and $\phi_i(x)>0$ if and only if $x\in V_i$.
In particular, 
$$\sum_i\phi_i(x)>0\ \ \text{for any}\ \ x\in {\spc{X}}.$$

Set 
$$\psi_k(x)=\frac{\phi_k(x)}{\sum_i\phi_i(x)}.$$
It remains to note that by construction the functions $\psi_i$ meet the conditions in the proposition.
\qedsf


Note that in the above proof for any point $x\in {\spc{X}}$,
the set
$$\set{v_i}{\psi_i(x)>0}$$
describe vertexes of a simplices in the nerve.
Therefore 
$$\bm{\psi}\:x\mapsto \psi_1(x)\cdot v_1+\psi_2(x)\cdot v_2+\dots+\psi_k(x)\cdot v_n.$$
can be thought of as a Lipschitz map from ${\spc{X}}$ to the nerve $\spc{N}$ of $\{V_i\}$;
where the point $x$ is mapped to the point with barycentric coordinates $\psi_i(x)$.
In other words we proved the following:

\begin{thm}{Proposition}\label{prop:space->nerve}
Let $\spc{N}$ be a nerve of an open covering $\{V_1,\dots,V_k\}$ of a compact metric space $\spc{X}$.
Denote by $v_i$ the vertex of $\spc{N}$ that corresponds to $V_i$.

Then there is a Lipschitz map from $\bm{\psi}\:\spc{X}\to\spc{N}$ such that $\bm{\psi}(V_i)\z\subset\Star_{v_i}$ for every $i$.
\end{thm}


\section{Width}

Suppose $A$ is a subset of a metric space $\spc{X}$.
The radius of $A$ (briefly $\rad A$) is defined as the least upper bound on the values $R>0$ such that $\oBall(x,R)\supset A$ for some $x\in \spc{X}$.

\begin{thm}{Definition}\label{def:width}
Let $\spc{X}$ be a metric space.
The $n$-width of $\spc{X}$ (briefly $\width_n\spc{X}$) is defined as least upper bound on values $R>0$ such that $\spc{X}$ admits a finite open covering $\{V_i\}$ with multiplicity at most $n+1$ and $\rad V_i< R$ for any $i$.
\end{thm}

\parit{Remarks.}
\begin{itemize}
\item Observe that if $\spc{X}$ is connected, then 
\[\width_0\spc{X}=\rad\spc{X}.\]
\item 
Usually width is defined using diameter instead of radius, but the result differ at most twice.
Namely if $r$ is the radius width and $d$ --- the diameter width of the same dimension, then 
$r\le d\le 2\cdot r$.

\item The definition of width reminds the definition of Lebesgue covering dimension.
In fact one says that a space has \emph{macroscopic dimesion} $\le n$ on the space $R$ if it admits an open cover as in the definiton.
\end{itemize}

\begin{thm}{Exercise}
Suppose $\spc{X}$ be a simply connected metric space such that any closed curve $\gamma$ in $\spc{X}$ can be contracted in its $R$-neighborhood.
Show that $\spc{X}$ is has macroscopic dimension at most 1 on scale $100\cdot R$.

Proove a quasiconverse; that is, if a simply connected metric space $\spc{X}$ has macroscopic dimension at most 1 on scale $R$, then any closed curve $\gamma$ in $\spc{X}$ can be contracted in its $100\cdot R$-neighborhood.
\end{thm}


The following proposition provides an equivalent definition.

\begin{thm}{Proposition}\label{prop:width=suprad(inv)}
Suppose $\spc{X}$ is a compact metric space.
Then $\width_n\spc{X}<R$ if and only if there is a finite $n$-dimensional somplicial complex $\spc{S}$ and a continuous map $\bm{\psi}\:\spc{X}\to \spc{N}$
such that $\rad[\bm{\psi}^{-1}(s)]\z<R$
for any $s\in \spc{N}$.
\end{thm}

\parit{Proof; ``only if'' part.}
Suppose $\width_n\spc{X}<R$.
Consider a covering $\{V_1,\dots,V_k\}$ of $\spc{X}$ guaranteed by the definition of width.
Let $\spc{N}$ be its nerve and $\bm{\psi}\:\spc{X}\to \spc{N}$ be the map provided by \ref{prop:space->nerve}.

Note that if $x\in \spc{N}$ lies in a symplex with a vertex $v_i$,
then $\bm{\psi}^{-1}(x)\subset V_i$;
in particulr $\bm{\psi}^{-1}(x)$ can be covered by a ball of radius $R$ in $\spc{X}$.

\parit{``If'' part.}
Choose $x\in \spc{N}$.
Since the inverse image $\bm{\psi}^{-1}(x)$ is compact, $\bm{\psi}$ is continuous, and $\rad[\bm{\psi}^{-1}(x)]<R$,
here is a neighborhood $U\ni x$ such that the  $\rad[\bm{\psi}^{-1}(U)]<R$.

It follows that there is a finite cover $\{U_i\}$ of $\spc{N}$ such that $\bm{\psi}^{-1}(U_i)\subset\spc{X}$ has radius smaller than $R$ for each $i$.
Since $\spc{N}$ has dimension $n$, we can inscribe%
\footnote{Recall that a covering $\{W_i\}$ is inscribed in the covering $\{U_i\}$ if for every $W_i$ is a subset of some $U_j$.} 
in $\{U_i\}$ an finite open cover $\{W_i\}$ with multiplicity at most $n+1$.
It remains to observe that $V_i=\bm{\psi}(W_i)$ defines a finite open cover of $\spc{X}$ with radius less than $R$ and multiplicity at most $n+1$. 
\qeds

Further we will apply the notion of width to compact Riemannian polyhedrons;
If $n$ is the dimension of a compact Riemannian polyhedron $\spc{P}$, then 
we suppose that
\[\width\spc{P}\df\width_{n-1}\spc{P}.\]

\begin{thm}{Exercise}
Show that for any closed Riemannian manifold $\spc{M}$ we have
\[\FillRad \spc{M}\le 100\cdot \width\spc{M};\]
try to show that in fact
\[\FillRad \spc{M}\le \width\spc{M}.\]

\end{thm}




\section{Volume bounds width}

A \emph{Riemannian polyhedron} is defined as a finite connected simplicial complex with a metric tensor on each simplex such that the restriction of the metric on each simplex to a subsymplex coinsides with the metric on the subsmplex.
The dimension of Riemannian polyhedron is defined as the largest dimension it its triangulation.
For Riemannian polhedron one can define length of curves and volume the same way as for Riemannian manifolds.

Let $\spc{P}$ be a Riemnnian polyhedron of dimension $n$.
Let us define volume profile of $\spc{P}$ as a function $\VolPro_{\spc{P}}\:\RR_+\to\RR_+$ defined by 
\[\VolPro_{\spc{P}}(r)\df \sup\set{\vol \oBall(p,r)}{p\in\spc{P}}.\]
Note that $\VolPro_{\spc{P}}$ is a nondecreasing function and $\VolPro_{\spc{P}}(r)\z\to\vol\spc{P}$ as $r\to\infty$.

\begin{thm}{Exercise}
Suppose $\spc{M}$ be a 1-dimensional Riemannian polhedron.
Suppose $\VolPro_{\spc{P}}(r_0)<r_0$ for some $r_0>0$.
Show that 
\[\diam \spc{P}<r_0.\]
Note tha it is equivalent to $\width \spc{P}<r_0$.
\end{thm}


An $(n-1)$-dimensional subpolyhedron $\spc{Q}\subset\spc{P}$ is called $R$-separating if each
connected component of its complement $\spc{P}\backslash \spc{Q}$ can be covered by a metric ball of radius $R$.

\begin{thm}{Lemma}
Let $\spc{P}$ be an $n$-dimensional Riemannian polyhedron.
Then given $R>0$ and $\eps>0$ there is a $R$-separating subpolyhedron $\spc{Q}\subset\spc{P}$ such that for any $r_0<r_1\le R$ we have
\[\VolPro_{\spc{Q}}(r_0)< \tfrac1{r_1-r_0}\cdot \VolPro_{\spc{P}}(r_1)+\eps.\]

\end{thm}

\begin{thm}{Lemma}
Let $\spc{Q}$ be a $R$-separating subpolyhedron in an $n$-dimensional Riemannian polyhedron $\spc{P}$.
Suppose $\width\spc{Q}\le R$.
Then $\width\spc{P}\le R$
\end{thm}

\parit{Proof.}
Start with an open covering $\{V_1,\dots,V_k\}$ of $\spc{Q}$ of multiplicity $\le n$ with radiuses of the sets in the intrinsic metric $\le R$.
Convert $\{V_1,\dots,V_k\}$ into an an open covering of
a small neighbourhood of $\spc{Q}$ in $\spc{P}$ without increasing the multiplicity.
Finally, add all the components of $\spc{P}\backslash \spc{Q}$ to the covering;
it increases the multiplicity by 1.
\qeds

The following technical statement will be used without a proof.

\begin{thm}{Claim}
Let $\spc{P}$ be a Reimannian polyhedron and $f\:\spc{P}\to \RR$ be a 1-Lipschitz function.
Then for any $\eps>0$ there is a  1-Lipschitz function $f_\eps\:\spc{P}\to \RR$ that is smooth on each simplex of the triangulation and $\eps$-close to $f$.
\end{thm}

\begin{thm}{Sard's theorem}
Let $\spc{P}$ be an $n$-dimensional Reimannian polyhedron and $f\:\spc{P}\to \RR$ be a function that is smooth on each simplex.
Then for almost all values $a$ each component of the inverse image $f^{-1}(a)$ is a equipped with the induced metric is a Reimannian polyhedron.
\end{thm}


\begin{thm}{Coarea inequality}
Let $\spc{P}$ be an $n$-dimensional Reimannian polyhedron and $f\:\spc{P}\to \RR$ be a 1-Lipschitz function that is smooth on each simplex.
Then 
\[\vol_n (f^{-1}[a,b]) \le \int_a^b\vol_{n-1}[f^{-1}(x)]\cdot dx.\]
\end{thm}


%\chapter{Examples}



\section{On semicontinuity}

Recall that according to \ref{ex:GH-vol}, volume is semicontinuos on the space of Riemannian manifolds with respect to stable Gromov--Hausdorff convergence.
Analogous statement for $n$-dimensional Hausdorff measure on a $n$-dimensional manifolds does not hold.

\begin{thm}{Claim}
 
\end{thm}

First let us show that for any $\alpha>0$, the $\alpha$-dimensional Hausdorff measure is not semicontinuous in the space of all compact metric spaces.

Choose a decreasing sequence $\eps_n\to 0$.
Consider the space $\spc{C}$ of infinite binary sequences with distance between two sequences $\bm{a}=(a_0,a_1,\dots)$ and $\bm{b}=(b_0,b_1,\dots)$ defined by 
\[\dist{\bm{a}}{\bm{b}}{\spc{C}}=\eps_n,\]
where $n$ is the minimal index such that $a_n\ne b_n$.
Note that $\spc{C}$ is homeomorphic to the Cantor set and 
given $\alpha>0$,
the sequence $\eps_n$ can be chosen so that its $\alpha$-dimensional Hausdorff measure is infinite.

Note that $\spc{C}$ is a Hausdorff limit of its subsets $\spc{C}_n$ formed by sequences that constantly zero starting from $n$-th element.
The sets $\spc{C}_n$ is finite in particular its $\alpha$-dimensional Hausdorff measure vanish for $\alpha>0$.
This example shows that for any $\alpha>0$, the $\alpha$-dimensional Hausdorff measure is not semicontinuous in the space of all compact metric spaces.

An analogous example can be produced comapct length spaces.
To do this consider a metric binary rooted tree $\spc{T}$ in which edges connecting level $n-1$ to the level $n$ of length $\eps_{n-1}-\eps_n$.
Note that the completion $\bar{\spc{T}}$ of $\spc{T}$ has a subset (its crown) isometric to $\spc{C}$.
Note further that $\bar{\spc{T}}$ is a Hausdorff limit of its subsets $\spc{T}_n$ --- the subtrees up to level $n$.
Note that $\spc{T}_n$ is can be covered by a finite line segments, in particular it has finite $1$-dimensional Hausdorff measure and therefore vanishing $\alpha$-dimensional Hausdorff for any $\alpha>1$.
Since the limit $\bar{\spc{T}}$ contains $\spc{C}$, we can choose a sequence $\eps_n$ so that $\mu_\alpha\spc{C}$ is arbitrary large (or even infinite).
It shows that for any $\alpha>1$, the $\alpha$-dimensional Hausdorff measure is not semicontinuous in the space of all compact length spaces.

This construction can be modified further to obtain an increasing sequence of metric tensors $g_n$ on a disc $\DD$ such that (1) $\vol(\DD,g_n)<1$ for each $n$, (2) the induced metrics $\dist{*}{*}{g_n}$ converge to a metric $\rho$ on $\DD$, and given any Cantor space $\spc{C}$ as described above (3) there is a bilipschitz map $\spc{C}\to(\DD,\rho)$.
Note that the last condition implies that $\mu_2(\DD,rho)$ can be made arbitrary large, or infinite.
Therefore for any $\alpha\ge 2$, the $\alpha$-dimensional Hausdorff measure is not semicontinuous in the space of all compact length spaces homeomorphic to a manifold and equipped with stable convergence.

Now we want to extend nonsemicontinuity even further.
Note that the tree $\bar{\spc{T}}$ admits a length-preserving embedding to the Euclidean space; we may assume that all 



\section{Sub-Riemannian metrics}

Choose a metric space $\spc{X}$.
Note that the function $\alpha\mapsto \mu_\alpha(A)_\spc{X}$ is nondecreasing;
moreover there is a critical value $\alpha_0\in[0,\infty]$ such that $\mu_\alpha(A)_\spc{X}=0$ if $\alpha<\alpha_0$ and $\mu_\alpha(A)_\spc{X}=\infty$ if $\alpha>\alpha_0$.
This value is called \index{Hausdorff dimension}\emph{Hausdorff dimension} of $\spc{X}$, or briefly $\alpha_0=\dim_H\spc{X}$.

The following statement is classical, a proof can be found in .

\begin{thm}{Theorem}
The Hausdorff dimension of any metric space can not be smaller than its Lebesgue covering dimension.
In particular, if a metric space $\spc{X}$ is homeomorphic to an $n$-dimensional manifold, then $\dim_H\spc{X}\ge n$.
 
\end{thm}

Note that the construction described in the previous section can be used to produce a metric on manifold of dimension $n\ge 2$ with arbitrary Hausdorff dimension $\alpha\ge n$.

In this section we will discuss another interesting source of such examples.



%
\begin{thm}{Uniqueness of geodesics}\label{thm:cat-unique}
In a proper length $\CAT(0)$ space, pairs of points are joined by unique geodesics, and these geodesics depend continuously on their endpoint pairs.

Analogously, in a proper length $\CAT(1)$ space, pairs of points at distance less than $\pi$ are joined by unique geodesics, and these geodesics depend continuously on their endpoint pairs.
\end{thm}

\parit{Proof.} 
Given 4 points $p^1,p^2,q^1,q^2$ in a proper length $\CAT(0)$ space $\spc{U}$, 
consider two triangles $\trig{p^1}{q^1}{p^2}$ and $\trig{p^2}{q^2}{q^1}$.
Since both of these triangles are thin, we get 
\begin{align*}
\dist{\geodpath_{[p^1q^1]}(t)}{\geodpath_{[p^2q^1]}(t)}{\spc{U}}
&\le (1-t)\cdot \dist{p^1}{p^2}{\spc{U}},
\\
\dist{\geodpath_{[p^2q^1]}(t)}{\geodpath_{[p^2q^2]}(t)}{\spc{U}}
&\le t\cdot \dist{q^1}{q^2}{\spc{U}}.
\intertext{By the triangle inequality,}
\dist{\geodpath_{[p^1q^1]}(t)}{\geodpath_{[p^2q^2]}(t)}{\spc{U}}&\le \max\{\dist{p^1}{p^2}{\spc{U}},\dist{q^1}{q^2}{\spc{U}}\}.
\end{align*}

This implies continuity and uniqueness in the $\CAT(0)$ case.  
 
The $\CAT(1)$ case is done in essentially the same way.
\qeds

Adding the first two inequalities of the preceding proof gives the following:

\begin{thm}{Proposition}
Suppose $p^1,p^2,q^1,q^2$ are points in a proper length $\CAT(0)$ space~$\spc{U}$.
Then 
\[\dist{\geodpath_{[p^1q^1]}(t)}{\geodpath_{[p^2q^2]}(t)}{\spc{U}}\]
is a convex function.
\end{thm}

\begin{thm}{Corollary}\label{cor:dist-convex}
Let $K$ be a closed convex subset in a proper length $\CAT(0)$ space~$\spc{U}$.
Then $\dist{K}{}{}\:\spc{U}\to\RR$ is \index{convex function}\emph{convex};
that is, the function $t\mapsto\dist{K}{}{}\circ\gamma$ is convex for any geodesic $\gamma$ in $\spc{U}$.

In particular, $\dist{p}{}{}$ is convex for any point $p$ in~$\spc{U}$.
\end{thm}


\begin{thm}{Corollary}\label{cor:contractible-cat}
Any proper length $\CAT(0)$ space is contractible.

Analogously, any proper length $\CAT(1)$ space with diameter $<\pi$ is contractible.
\end{thm}

\parit{Proof.} Let $\spc{U}$ be a proper length $\CAT(0)$ space.
Fix a point $p\in \spc{U}$.

For each point $x$ consider the geodesic path $\gamma_x\:[0,1]\to \spc{U}$ from $p$ to~$x$.
Consider the one parameter family of maps 
$h_t\:x\mapsto \gamma_x(t)$ for $t\in [0,1]$.
By uniqueness of geodesics (\ref{thm:cat-unique}), the map 
$(t,x)\mapsto h_t(x)$ is continuous. The family $h_t$ is called a \index{geodesic homotopy}\emph{geodesic homotopy}.

It remains to note that $h_1(x)=x$ and $h_0(x)=p$ for any~$x$.

The proof of the $\CAT(1)$ case is identical.
\qeds

\begin{thm}{Proposition}\label{cor:loc-geod-are-min}
Suppose $\spc{U}$ is a proper length $\CAT(0)$ space.  
Then any local geodesic in $\spc{U}$ is a geodesic.

Analogously, if $\spc{U}$ is a proper length $\CAT(1)$ space, then any local geodesic in $\spc{U}$ which is shorter than $\pi$ is a geodesic.
\end{thm}

\begin{wrapfigure}{r}{21mm}
\begin{lpic}[t(-0mm),b(0mm),r(0mm),l(0mm)]{pics/local-geod(1)}
\lbl[t]{2.5,1;$\gamma(0)$}
\lbl[b]{10,14;$\gamma(a)$}
\lbl[t]{19,8;$\gamma(b)$}
\end{lpic}
\end{wrapfigure}

\parit{Proof.}
Suppose $\gamma\:[0,\ell]\to\spc{U}$ is a local geodesic that is not a geodesic.
Choose $a$ to be the maximal value 
such that $\gamma$ is a geodesic on $[0,a]$.
Further choose $b>a$ so that $\gamma$ is a geodesic on $[a,b]$.

Since the triangle $\trig{\gamma(0)}{\gamma(a)}{\gamma(b)}$ is thin and 
$\dist{\gamma(0)}{\gamma(b)}{}<b$ we have
\[\dist{\gamma(a-\eps)}{\gamma(a+\eps)}{}<2\cdot\eps\]
for all small~$\eps>0$.
That is, $\gamma$ is not length-minimizing on the interval $[a-\eps,a+\eps]$ for any $\eps>0$,
a contradiction.

The spherical case is done in the same way.
\qeds


\begin{thm}{Exercise}\label{ex:geod-CBA}
Assume $\spc{U}$ is a proper length $\CAT(\kappa)$ space
 with extendable geodesics;
that is, any geodesic is an arc in a local geodesic $\RR\to \spc{U}$.

Show that the space of geodesic directions at any point in $\spc{U}$ is complete.

Does the statement remain true if $\spc{U}$ is complete, but not required to be proper?
\end{thm}

Now let us formulate the main result of this section.


\begin{wrapfigure}[6]{r}{28mm}
\begin{lpic}[t(-4mm),b(6mm),r(0mm),l(0mm)]{pics/lem_alex1(1)}
\lbl[lb]{10,23;$y$}
\lbl[rt]{1.5,.5;$p$}
\lbl[bl]{25,7.5;$x$}
\lbl[lb]{17,15;$z$}
\end{lpic}
\end{wrapfigure}

\begin{thm}{Inheritance lemma}
\label{lem:inherit-angle} 
Assume that a triangle $\trig p x y$ 
in a metric space is \index{decomposed triangle}\emph{decomposed} 
into two triangles $\trig p x z$ and $\trig p y z$;
that is, $\trig p x z$ and $\trig p y z$ have a common side $[p z]$, and the sides $[x z]$ and $[z y]$ together form the side $[x y]$ of $\trig p x y$.

If both triangles $\trig p x z$ and $\trig p y z$ are thin, 
then the triangle $\trig p x y$ is also thin.

Analogously, if $\trig p x y$ has perimeter $<2\cdot\pi$ and both triangles $\trig p x z$ and $\trig p y z$ are spherically thin, then triangle $\trig p x y$ is spherically thin.
\end{thm} 


\begin{wrapfigure}{r}{32mm}
\begin{lpic}[t(-4mm),b(0mm),r(0mm),l(0mm)]{pics/cat-monoton-ineq(1)}
\lbl[b]{14,23;$\dot z$}
\lbl[t]{10,.5;$\dot p$}
\lbl[r]{1,14;$\dot x$}
\lbl[l]{30.5,14;$\dot y$}
\lbl[tl]{13,13;$\dot w$}
\end{lpic}
\end{wrapfigure}

\parit{Proof.}
Construct  the model triangles $\trig{\dot p}{\dot x}{\dot z}\z=\modtrig(p x z)_{\EE^2}$ 
and $\trig {\dot p} {\dot y} {\dot z}\z=\modtrig(p y z)_{\EE^2}$ so that $\dot x$ and $\dot y$ lie on opposite sides of $[\dot p\dot z]$.

Let us show that 
\[\angk{z}{p}{x}+\angk{z}{p}{y}
\ge
\pi.
\eqlbl{eq:<+<>=pi}\]
Suppose the contrary, that is
\[\angk{z}{p}{x}+\angk{z}{p}{y}
<
\pi.\]
Then for some point $\dot w\in[\dot p\dot z]$, we have \[\dist{\dot x}{\dot w}{}+\dist{\dot w}{\dot y}{}
<
\dist{\dot x}{\dot z}{}+\dist{\dot z}{\dot y}{}=\dist{x}{y}{}.\]
Let $w\in[p z]$ correspond to $\dot w$; that is, $\dist{z}{w}{}=\dist{\dot z}{\dot w}{}$. 
Since $\trig p x z$ and $\trig p y z$ are thin, we have 
\[\dist{x}{w}{}+\dist{w}{y}{}<\dist{x}{y}{},\]
contradicting the triangle inequality. 

Denote by $\dot D$ the union of two solid triangles $\trig {\dot p}{\dot x}{\dot z}$ and $\trig {\dot p} {\dot y} {\dot z}$.
Further, denote by $\tilde D$ the solid triangle $\trig{\tilde  p}{\tilde  x}{\tilde  y}=\modtrig(p x y)_{\EE^2}$.
By \ref{eq:<+<>=pi}, there is a short map $F\:\tilde D\to \dot D$ that sends 
\begin{align*}
\tilde p&\mapsto \dot p,
&
\tilde x&\mapsto \dot x,
&
\tilde z&\mapsto \dot z,
&
\tilde y&\mapsto \dot y.
\end{align*}
\qedsf

\begin{thm}{Exercise}\label{ex:short-map}
Use Alexandrov's lemma (\ref{lem:alex}) to prove the last statement. 
\end{thm}


By assumption, the natural maps $\trig {\dot p} {\dot x} {\dot z}\to\trig p x z$ and $\trig {\dot p} {\dot y} {\dot z}\to\trig p y z$ are short.  
By composition,  the natural map from $\trig{\tilde  p}{\tilde  x}{\tilde  y}$ to $\trig p y z$ is short, as claimed.

The spherical case is done along the same lines.
\qeds

\begin{thm}{Exercise}\label{ex:convex-balls}
Show that any ball in a proper length $\CAT(0)$ space is a convex set.

Analogously, show that any ball of radius $R<\tfrac\pi2$ in a proper length $\CAT(1)$ space  is a convex set.
\end{thm}

Recall that a set $A$ in a metric space $\spc{U}$ is called locally convex if for any point $p\in A$ there is an open neighborhood $\spc{U}\ni p$ such that any geodesic in $\spc{U}$ with  ends in $A$ lies in~$A$. 

\begin{thm}{Exercise}\label{ex:locally-convex}
Let $\spc{U}$ be a proper length $\CAT(0)$ space.
Show that any closed, connected, locally convex set in $\spc{U}$ is convex.
\end{thm}

\begin{thm}{Exercise}\label{ex:closest-point}
Let  $\spc{U}$ be a proper length $\CAT(0)$ space 
and $K\subset \spc{U}$ be a closed convex set.
Show that: 

\begin{subthm}{ex:closest-point:a}
For each point $p\in \spc{U}$ there is unique point $p^*\in K$ that minimizes the distance $\dist{p}{p^*}{}$.
\end{subthm}

\begin{subthm}{}
The closest-point projection $p\mapsto p^*$ defined by (\ref{SHORT.ex:closest-point:a}) is short. 
\end{subthm}

\end{thm}




















\begin{thm}{Advanced exercise}\label{ex:urysohn-contractable}
 Show that the space $\spc{U}$ is contactable.
\end{thm}


\parbf{Advanced exercise~\ref{ex:urysohn-contractable}.}
Note that points in the space $\spc{X}_\infty$ constructed in the proof of \ref{prop:univeral-separable} can be multiplied number $t\in [0,1]$ --- simply multiply each function by factor $t$.
That defines a map 
\[\lambda_t\:\spc{X}_\infty\to \spc{X}_\infty\]
that scales all distances by factor $t$.
The map $\lambda_t$ can be extended to the completion of $\spc{X}_\infty$, which is isometic to $\spc{U}_d$ (or $\spc{U}$).

Observe that 
the map $\lambda_1$ is the identity  and $\lambda_0$ maps whole space to a single point, say $x_0$ --- that is the only point of $\spc{X}_0$.
Further note that the map $(t,p)\mapsto \lambda_t(p)$ is continuous ---  in particular $\spc{U}_d$ and $\spc{U}$ are contractible.\qeds

Source: \cite[(d) on page 82]{gromov-2007}.

Observe that for any point $p\in \spc{U}_d$ the curve $t\mapsto \lambda_t(p)$ is a geodesic path from $p$ to $x_0$.








Note that $\spc{M}$ --- the space of compact metric spaces can be treated as a space of compact subsets in $\spc{U}$ up to congruence.
Namely two subsets $A$ and $A'$ are called \emph{congruent} (briefly $A\cong A'$) if there is isometry of the ambient space $\spc{U}$ that maps $A$ to $A'$.
Let us define distance between congruence classes of two compact subsets $A$ and $B$ as 
\[\inf\set{\dist{A'}{B}{\spc{H}(\spc{U})}}{A'\cong A}.\]


By \ref{prop:sep-in-urys}, any compact metric spaces $\spc{K}$ admits a distance preserving map $f\:\spc{K}\to\spc{U}$.
Moreover by \ref{thm:compact-homogeneous} any two such maps $f_1$ and $f_2$ differ by isometry of $\spc{U}$;
that is, there is an isometry $\iota\:\spc{U}\to\spc{U}$ such that $f_2=\iota\circ f_1$.
In particular $f_1(\spc{K})\cong f_2(\spc{K})$.










\section{Ultratangent space} 

Recall that we assume that $\omega$ is a once for all fixed choice of a nonprinciple ultrafilter.

For a metric space $\spc{X}$ and a positive real number $\lambda$,
we will denote by $\lambda\cdot\spc{X}$ its \emph{$\lambda$-blowup}\index{blowup},
which is a metric space with the same underlying set as $\spc{X}$ and the metric multiplied by $\lambda$.
The tautological bijection $\spc{X}\to \lambda\cdot\spc{X}$ will be denoted as $x\mapsto x^\lambda$, 
so 
\[\dist{x^\lambda}{y^\lambda}{}
=
\lambda\cdot\dist[{{}}]{x}{y}{}\] 
for any $x,y\in \spc{X}$.

The $\omega$-blowup $\omega\cdot\spc{X}$ of $\spc{X}$ is defined as the $\omega$-limit
of the $n$-blowups $n\cdot\spc{X}$; that is,
\[\omega\cdot\spc{X}
\df
\lim_{n\to\omega} n\cdot\spc{X}.\]

Given a point $x\in \spc{X}$ we can consider the sequence $x^n\in n\cdot\spc{X}$;
it corresponds to a point $x^\omega\in \omega\cdot\spc{X}$.
Note that if $x\ne y$, then 
\[\dist{x^\omega}{y^\omega}{\omega\cdot\spc{X}}=\infty;\]
that is, 
$x^\omega$ and $y^\omega$ 
belong to different metric components of $\omega\cdot\spc{X}$.

The metric component of $x^\omega$ in $\omega\cdot\spc{X}$ is called ultratangent space of $\spc{X}$ at $x$ and it is denoted as $\T^\omega_x\spc{X}$.

Equivalently, ultratangent space $\T^\omega_x\spc{X}$ can be defined the following way.
Consider all the sequences of points $x_n\in \spc{X}$ such that
the sequence $\ell_n=n\cdot\dist{x}{x_n}{\spc{X}}$ is bounded.
Define the pseudodistance between two such sequences as 
\[\dist{(x_n)}{(y_n)}{}
=
\lim_{n\to\omega}n\cdot\dist{x_n}{y_n}{\spc{X}}.\]
Then $\T^\omega_x\spc{X}$ is the corresponding metric space.

Tangent space as well as ultratangent space, 
generalize the notion of tangent space of Riemannian manifold.
In the simplest cases these two notions define the same space.
In general, they are different and both useful ---
often lack of a property in one is compensated by the other.

It is clear from the definition that tangent space has cone structure.
On the other hand, in general, ultratangent space does not have a cone structure; 
the Hilbert's cube $\prod_{n=1}^\infty[0,2^{-n}]$ is an example --- it is $\Alex{0}$ as well as $\CAT{0}$.

The next theorem shows that the tangent space $\T_p$ can be (and often will be) considered as a subset of  $\T^\omega_p$.

\begin{thm}{Theorem}\label{thm:tangent-ultratangent}
\label{thm:T-in-T^w} 
Let $\spc{X}$ be a metric space with defined angles.
Then for any $p\in \spc{L}$, there is an distance preserving map 
\[\iota:\T_p\hookrightarrow \T^\omega_p\] 
such that for any geodesic $\gamma$ starting at $p$
we have 
\[\gamma^+(0)\mapsto \lim_{n\to\omega}[\gamma(\tfrac1n)]^n.\]

\end{thm}

\parit{Proof.}
Given $v\in \T'_p$ 
choose a geodesic $\gamma$ that starts at $p$ such that $\gamma^+(0)\z=v$.
Set $v^n=[\gamma(\tfrac1n)]^n\in n\cdot \spc{X}$ and 
\[v^\omega=\lim_{n\to\omega}v^n.\]

Note that the value $v^\omega\in\T^\omega_p$ does not depend on choice of $\gamma$;
that is, if $\gamma_1$ is an other geodesic starting at $p$ such that $\gamma_1^+(0)=v$,
then 
\[\lim_{n\to\omega}v^n=\lim_{n\to\omega}v_1^n,\]
where $v_1^n=[\gamma_1(\tfrac1n)]^n\in n\cdot \spc{X}$.
The latter follows since
\[\dist{\gamma(t)}{\gamma_1(t)}{\spc{X}}=o(t)\]
and therefore $\dist{v^n}{v_1^n}{n\cdot \spc{X}}\to 0$ s $n\to\infty$.



Set $\iota(v)=v^\omega$.
Since angles between geodesics in $\spc{X}$ are defined, for any $v,w\in \T_p'$ we have
$n\cdot\dist[{{}}]{v_n}{w_n}{}\to\dist{v}{w}{}$.
Thus $\dist{v_\omega}{w_\omega}{}=\dist{v}{w}{}$; that is, $\iota$ is a global isometry of $\T_p'$.

Since $\T_p'$ is dense in $\T_p$,
we can extend $\iota$ to a global isometry $\T_p\to \T^\omega_p$.
\qeds

{\sloppy

\section[Gromov--Hausdorff and ultralimits]{Gromov--Hausdorff convergence and ultralimits}

}

\begin{thm}{Theorem}\label{thm:ultra-GH}
Assume $\spc{X}_n$ is a sequence of complete spaces. 
Let $\spc{X}_n\to \spc{X}_\omega$ as $n\to\omega$,
and $\spc{Y}_n\subset \spc{X}_n$ 
be a sequence of subsets such that $\spc{Y}_n\GHto\spc{Y}_\infty$. 
Then there is a distance preserving map 
$\iota:\spc{Y}_\infty\to \spc{X}_\omega$.

Moreover:

\begin{subthm}{thm:ultra-GH:a}
If $\spc{X}_n\GHto \spc{X}_\infty$ 
and $\spc{X}_\infty$ is compact, then 
$\spc{X}_\infty$ is isometric to $\spc{X}_\omega$.
\end{subthm}

\begin{subthm}{thm:ultra-GH:b}
If $\spc{X}_n\GHto \spc{X}_\infty$ 
and $\spc{X}_\infty$ is proper, then 
$\spc{X}_\infty$ is isometric to a metric component of $\spc{X}_\omega$.
\end{subthm}

\end{thm}

\parit{Proof.} 
For each point $y_\infty\in \spc{Y}_\infty$ 
choose a lifting $y_n\in \spc{Y}_n$.
Pass to the $\omega$-limit $y_\omega\in \spc{X}_\omega$ of $(y_n)$.
Clearly for any $y_\infty,z_\infty\in \spc{Y}_\infty$, 
we have 
\[\dist{y_\infty}{z_\infty}{\spc{Y}_\infty}=\dist{y_\omega}{z_\omega}{\spc{X}_\omega};\] 
that is, the map $y_\infty\mapsto y_\omega$ gives a distance preserving map $\iota:\spc{Y}_\infty\to \spc{X}_\omega$. 


\parit{(\ref{SHORT.thm:ultra-GH:a})$+$(\ref{SHORT.thm:ultra-GH:b}).}
Fix $x_\omega\in \spc{X}_\omega$.
Choose a sequence $x_n\in \spc{X}_n$ 
such that $x_n\to x_\omega$ as $n\to\omega$. 

Denote by $\bm{X}=\spc{X}_\infty\sqcup\spc{X}_1\sqcup\spc{X}_2\sqcup\dots$ the common space for the convergence $\spc{X}_n\GHto \spc{X}_\infty$;
as in the definition of Gromov--Hausdorff convergence.
Consider the sequence $(x_n)$ 
as a sequence of points in~$\bm{X}$.

If the $\omega$-limit $x_\infty$ of $(x_n)$ exists, 
it must lie in $\spc{X}_\infty$. 

The point $x_\infty$, if defined, does not depend on the choice of $(x_n)$.
Indeed, if $y_n\in\spc{X}_n$ is an other sequence such that $y_n\to x_\omega$ as $n\to\omega$, then 
\[
\dist{y_\infty}{x_\infty}{}=\lim_{n\to\omega}\dist{y_n}{x_n}{}=0;
\]
that is, $x_\infty=y_\infty$.


In this way we obtain a map $\nu\:x_\omega\to x_\infty$;
it is defined on a subset of $\Dom\nu \subset\spc{X}_\omega$.
By construction of $\iota$, 
we get  $\iota\circ\nu(x_\omega)=x_\omega$ for any $x_\omega\in \Dom\nu$.

Finally note that if $\spc{X}_\infty$ is compact, then $\nu$ is defined on all of $\spc{X}_\omega$;
this proves (\ref{SHORT.thm:ultra-GH:a}).

If $\spc{X}_\infty$ is proper, choose any point $z_\infty\in \spc{X}_\infty$
and set $z_\omega=\iota(z_\infty)$.
For any point $x_\omega\in \spc{X}_\omega$ at finite distance from $z_\omega$,
for the sequence $x_n$ 
as above we have that $\dist{z_n}{x_n}{}$ is bounded for $\omega$-almost all $n$.
Since $\spc{X}_\infty$ is proper, $\nu(x_\omega)$ is defined;
in other words $\nu$ is defined on the metric component of $z_\omega$.
Hence (\ref{SHORT.thm:ultra-GH:b}) follows.
\qeds

\begin{thm}{Corollary} 
\label{cor:ulara-geod}
The $\omega$-limit of a sequence of complete length spaces is geodesic.
\end{thm}

\parit{Proof.} Given two points $x_\omega,y_\omega\in \spc{X}_\omega$, find two bounded sequences of points $x_n,y_n\in \spc{X}_n$, $x_n\to x_\omega$, $y_n\to y_\omega$ as $n\to\omega$.
Consider a sequence of paths  $\gamma_n\:[0,1]\to \spc{X}_n$ from $_n$ to $y_n$
 such that 
\[\length\gamma_n\le \dist{x_n}{y_n}{}+\tfrac{1}{n}.\]
Apply Theorem~\ref{thm:ultra-GH} 
for the images $\spc{Y}_n=\gamma_n([0,1])\subset \spc{X}_n$.
\qeds

\section{Ultralimits of sets}

Let $\spc{X}_n$ be a sequence of metric spaces and $\spc{X}_n\to \spc{X}_\omega$
as $n\to \omega$.

For a sequence of sets $\Omega_n\subset \spc{X}_n$,
consider the maximal set $\Omega_\omega\subset \spc{X}_\omega$ such that 
for any $x_\omega\in\Omega_\omega$ and any sequence $x_n\in\spc{X}_n$ such that $x_n\to x_\omega$ as $n\to \omega$, we have $x_n\in\Omega_n$ for $\omega$-almost all $n$.

The set $\Omega_\omega$ is called the  \emph{open $\omega$-limit} of $\Omega_n$;
we could also write  $\Omega_n\to \Omega_\omega$ as $n\to\omega$ or $\Omega_\omega=\lim_{n\to\omega}\Omega_n$. 

{\sloppy

Applying Observation~\ref{obs:ultralimit-is-complete} to the sequence of complements $\spc{X}_n\backslash \Omega_n$, we see that $\Omega_\omega$ is open for any sequence $\Omega_n$.
The definition can be applied for arbitrary sequences of sets, but  
open $\omega$-convergence  will be applied here only for sequences of open sets.

}

\section{Ultralimits of functions}

Recall that a family of submaps between metric spaces $\{f_\alpha\: \spc{X}\to\spc{Y}\}_{\alpha\in\mathcal A}$ is called \emph{equicontinuous} if for any $\eps>0$ there is $\delta>0$ such that for any $p,q\in\spc{X}$ with $\dist{p}{q}{}<\delta$ and any $\alpha\in\mathcal A$ it holds that $\dist{f(p)}{f(q)}{}<\eps$.

Let $f_n\:\spc{X}_n\to\RR$ be a sequence of subfunctions.

Set $\Omega_n=\Dom f_n$.
Consider the open $\omega$-limit set $\Omega_\omega\subset \spc{X}_\omega$ of $\Omega_n$.

Assume there is a subfunction $f_\omega\:\spc{X}_\omega\to\RR$
that satisfies the following conditions: 
(1) $\Dom f_\omega=\Omega_\omega$, (2) if $x_n\to x_\omega\in \Omega_\omega$ for a sequence of points $x_n\in\spc{X}_n$, then $f_n(x_n)\to f_\omega(x_\omega)$ as $n\to\omega$.
In this case 
the subfunction $f_\omega\:\spc{X}_\omega\to\RR$ 
is said to be the 
$\omega$-limit of $f_n\:\spc{X}_n\to\RR$.

The following lemma gives a mild condition on a sequence of functions $f_n$
guaranteeing the existence of its $\omega$-limit.

\begin{thm}{Lemma}
Let $\spc{X}_n$ be a sequence of metric spaces
and $f_n\:\spc{X}_n\to\RR$ be a sequence of subfunctions.

Assume for any positive integer $k$, there is an open set $\Omega_n(k)\subset \Dom f_n$
such that the restrictions $f_n|_{\Omega_n(k)}$ are uniformly bounded and continuous
and the open $\omega$-limit of $\Omega_n(n)$ coincides with the open $\omega$-limit of $\Dom f_n$.
Then the $\omega$-limit of $f_n$ is defined.

In particular, if the $f_n$ are uniformly bounded and continuous, then the $\omega$-limit is defined.
\end{thm}

The proof is straightforward.

{\sloppy

\begin{thm}{Exercise}\label{ex:nonconvex-limit}
Construct a sequence of compact length spaces 
$\spc{X}_n$  
with a converging sequence of $\Lip$-Lipschitz concave functions $f_n\:\spc{X}_n\to\RR$ such that
the $\omega$-limit $\spc{X}_\omega$ of $\spc{X}_n$ is compact
and the $\omega$-limit $f_\omega\:\spc{X}_\omega\to\RR$ of $f_n$ is not concave.
\end{thm}

}

If $f\:\spc{X}\to\RR$ is a subfunction, 
the $\omega$-limit of the constant sequence $f_n=f$ is called the $\omega$-power of $f$ and denoted by $f^\omega$.
So
\[f^\omega\:\spc{X}\to\RR,\ \ f^\omega(x_\omega)=\lim_{n\to\omega} f(x_n).\]

Recall that we treat $\spc{X}$ as a subset of its $\omega$-power $\spc{X}^\omega$.
Note that $\Dom f=\spc{X}\cap \Dom f^\omega$.
Moreover, 
if $\oBall(x,\eps)_{\spc{X}}\subset \Dom f$
then $\oBall(x,\eps)_{\spc{X}^\omega}\subset \Dom f^\omega$.


\parbf{Ultradifferential.}
Given a function $f\:\spc{L}\to\RR$, consider sequence of functions $f_n\:n\cdot\spc{L}\to\RR$, defined by 
\[f_n(x^n)=n\cdot(f(x)-f(p)),\]
here $x^n\in n\cdot\spc{L}$ is the point corresopnding to $x\in\spc{L}$.
While $n\cdot(\spc{L},p)\to(\T^\omega,\0)$ as $n\to\omega$, 
functions $f_n$ converge to $\omega$-differential of $f$ at $p$.
It will be denoted by $\dd_p^\omega f$;
\[\dd_p^\omega f\:\T_p^\omega\to\RR,\ \ \dd_p^\omega f=\lim_{n\to\omega} f_n.\] 

Clearly, the $\omega$-differential $\dd_p^\omega f$ of a locally Lipschitz subfunction $f$ is defined at each point $p\in \Dom f$.
















\section{Comments} 

Given two metric spaces $\spc{X}$ and $\spc{Y}$, we will write $\spc{X}\preccurlyeq \spc{Y}$ if there is a noncontracting map $f\:\spc{X}\to \spc{Y}$;
that is, if 
$$ |x-x'|_{\spc{X}}\le|f(x)-f(x')|_{\spc{Y}}$$
for any $x,x'\in \spc{X}$.

Further, given $\eps>0$, we will write $\spc{X}\preccurlyeq \spc{Y}+\eps$
if there is a map $f\:\spc{X}\to \spc{Y}$ such that 
$$|x-x'|_{\spc{X}}\le|f(x)-f(x')|_{\spc{Y}}+\eps$$
for any $x,x'\in \spc{X}$.

Define 
$$\dist[\star]{\spc{X}}{\spc{Y}}{\spc{M}}=\inf\set{\eps}{\spc{X}\preccurlyeq \spc{Y}+\eps
\quad\text{and}\quad
\spc{Y}\preccurlyeq \spc{X}+\eps}$$
It turns out that $\dist[\star]{*}{*}{\spc{M}}$ is a different metric on the set of isometry classes of compact metric spaces; that is, in general $\dist[\star]{\spc{X}}{\spc{Y}}{\spc{M}}\not=|\spc{X}-\spc{Y}|_{\spc{M}}$. 
However, these two metrics define the same topology on $\spc{M}$.
More precicely:

\begin{thm}{Proposition}\label{GH-po}
For any sequence of compact metric spaces $(\spc{X}_n)$ and a compact metric space $\spc{X}_\infty$,
we have
$$|\spc{X}_n-\spc{X}_\infty|_{\spc{M}}\to 0
\quad\iff\quad
\dist[\star]{\spc{X}_n}{\spc{X}_\infty}{\spc{M}}\to 0$$ 
as $n\to\infty$.
\end{thm}

We will not give a proof of this proposition. 
Likely, we will not use it further in the lectures, 
but it might help you to build intuition for Gromov--Hausdorff convergence.
If you want to prove it yourself look in the proof of Theorem~\ref{thm:GH-is-a-metric} 
and try to modify it using ideas from the proof of Problem~\ref{pr:non-contracting=>isometry}.

The Gromov--Hausdorff distance can be defined for arbitrary pair of metric space.
Therefore it is natural to ask why we only consider compact metric spaces.
First note the Gromov--Hausdorff distance from any metric space $\spc{X}$ 
to its completion $\bar {\spc{X}}$ is zero.
Therefore if you want to end up in a metric space, it is better to consider only complete metric spaces.

Further, the distance between one-point-space and a metric spce with infinite diameter is infinite.
Therefore one has to either consider only bounded metric spaces (that is, the spaces with finite diameter)
or relux the definition of metric space which allow metric to take infinite value.

Finally, the class of isometry classes of all bounded complete metric spaces forms a class, but not a set.
Hence again we have two choices: either relux the definition of metric space so its points will form a class, or restrict further the class of spaces for which the isometry classes will form a set.

\begin{thm}{Exercise}
Prove that isometry classes of compact metric spaces form a set. 
\end{thm}

\begin{thm}{Exercise}\label{pr:GH1}
Let $\spc{X}=\{x,y,z\}$ be a three point subset of Euclidean plane with distances
$$|x-y|=|y-z|=|z-x|=1.$$
\begin{enumerate}[(i)]
\item Find the minimal Hausdorff distance from $\spc{X}$ to a one-point subset of the plane.
\item Find the Gromov--Hausdorff distance from $\spc{X}$ to the one-point metric space. 
\end{enumerate}
\end{thm}

\begin{thm}{Exercise}\label{pr:GH2}
Let $\spc{X}$ and $\spc{Y}$ be a compact metric spaces which have isometric $\eps$-nets.
Show that 
$$|\spc{X}-\spc{Y}|_{\spc{M}}\le 2\cdot\eps.$$
Is it always true that 
$$|\spc{X}-\spc{Y}|_{\spc{M}}\le \eps?$$
\end{thm}




\begin{thm}{Exercise}\label{pr:GH3}
Define the \emph{radius of a metric space}\index{radius of a metric space} $\spc{X}$ as 
$$\rad \spc{X}=\inf_x\left\{\sup_y\{|x-y|_{\spc{X}}\}\right\}.$$
Equivalently, 
$$\rad \spc{X}=\inf\set{R>0}{\text{there is}\ x\in \spc{X}\  \text{such that}\ B_R(x)\supset \spc{X}}.$$
 
\begin{enumerate}[(i)]
\item Show that for any compact metric space $\spc{X}$ we have
$$\tfrac12\cdot\diam \spc{X}\le \rad \spc{X}\le \diam \spc{X}.$$
\item Show that for any compact metric spaces $\spc{X},\spc{Y}$ we have
$$|\rad \spc{X}-\rad \spc{Y}|\le 2\cdot |\spc{X}-\spc{Y}|_{\spc{M}}.$$
\end{enumerate}
\end{thm}

\begin{thm}{Exercise}\label{pr:F-X}
Let $\spc{X}$ be a metric space.
If two compact sets $A, B$ in $\spc{X}$ are isometric,
we will write $A\iso B$. 
Set
$$d(A,B)=\inf \set{|A'-B'|_{\mathcal{H}(\spc{X})}}{A'\iso A \ \text{and}\ B'\iso B}.$$
Note that if $\spc{X}=\ell^\infty$, then according to Proposition~\ref{prop:GH-with-fixed-Z}, 
$d$ is a metric on $\mathcal{H}(\spc{X})/\iso$ (that is, on the ``$\iso$''-equivalecne classes of $\mathcal{H}(\spc{X})$).

Show that it does not hold for arbitrary metric space $\spc{X}$.
Understand the reason why it holds for $\spc{X}=\ell^\infty$.
\end{thm}


\begin{thm}{Exercise}\label{pr:GH-variation}
Consider the pairs $(\spc{X},A)$, where $\spc{X}$ is a compact metric space and $A$ is a closed subset in $\spc{X}$.
Two such pairs, say $(\spc{X},A)$ and $(\spc{X}',A')$ will be called equivalent (briefly $(\spc{X},A)\sim(\spc{X}',A')$)
if there is an isometry $\iota\:\spc{X}\to \spc{X}'$ such that $\iota(A)=A'$.

Modify the definition of Gromov--Hausdorff metric to construct a natural metric on the set of $\sim$-equivalence classes of the pairs $(\spc{X},A)$.
\end{thm}

Here we introduce so called Gromov--Hausdorff convergence for metric spaces.
This convergence was introduced by Gromov around 1980, published in \cite{gromov-1981}.
Very soon this notion began to be used in all branches of geometry.
In fact today I have difficulty to understand 
how one could do geometry without this type of convergence.%
(Some types of convergences of metric spaces was considered before Gromov,
but they had lack of generality;
each type of convergence was desined to solve one particular problem.)


\begin{thm}{Exercise}\label{ex:euclid-isom}
\begin{subthm}{}
Let $\spc{X},\spc{Y}$ be two compact sets in the Euclidean plane $\RR^2$.
Show that $\spc{X}$ is isometric to $\spc{Y}$ if and only if there is an motrio $\iota\:\RR^2\to \RR^2$
that sends $\spc{X}$ to $\spc{Y}$.
\end{subthm}

\begin{subthm}{}
Find two isometric subsets $\spc{X},\spc{Y}$ of $\ell^\infty$
such that there is no isometry $\iota\:\ell^\infty\to \ell^\infty$ 
that sends $\spc{X}$ to $\spc{Y}$.
\end{subthm}
\end{thm}

\appendix
\chapter{Semisolutions}
\parbf{\ref{ex:besikovitch=}.}
Let us use the same notation as in the proof of \ref{thm:besikovitch}.

Consider the map $s\:x\mapsto(\distfun_A(x),\distfun_B(x))$.
From the proof of \ref{thm:besikovitch} we get that $\Im s\supset \square$.
Observe that in the case of equality we have that $\Im s= \square$.
Indeed,
the same argument shows that 
\[\vol(s^{-1}(\square),g)\ge \vol\square=1.\]
The set $s^{-1}(\RR^1\backslash \square)$ is an open subset of $\square$.
If it is nonempty, then it has positive volume.
In this case
\[\vol(\square,g)>\vol(s^{-1}(\square),g)\ge 1\]
--- a contradiction.

Summarizing above discussion, there is a geodesic path of $g$-length $1$ connecting a point on one face of cube to the opposite face.

Moreover, for any pair of opposite faces and a point $p\in\square$, there is a geodesic path of $g$-length $1$ from one face to the other that pass thru $p$.
The latter can be shown by cutting $\square$ into two rectangles by a level surface of $\distfun_A$ thru $p$,
applying the above statement to both rectangles and taking the concatination of the obtained geodesic paths with end at $p$.
(The level surface might cut a rectangle with some topology, so have to apply \ref{thm:besikovitch+} instead of \ref{thm:besikovitch}).

Let $\gamma$ be such geodesic path from $A$ to $A'$.
Observe that $\gamma'(t)\z=\nabla_{\gamma(t)}\distfun_A$.
Therefore $\distfun_A$ is differentiable at every point $p\in \square$.
It follows that the map $s$ is differentiable.

Further checking the equality case in each inequality in the proof of \ref{thm:besikovitch}, we get that $s$ is a bijection and the equalities
\[|d_{p}\distfun_A|= 1,\quad|d_{p}\distfun_B|=1,\quad \text{and}\quad \langle d_{p}\distfun_A,d_{p}\distfun_B\rangle= 0\]
hold for almost all $p\in\square$.
Since $d_{p}\distfun_A$ and $d_{p}\distfun_B$ are well defined, we get that the equalities hold everywhere.
That is $s$ is an isometry.

\begin{wrapfigure}{r}{45 mm}
\vskip-4mm
\centering
\includegraphics{mppics/pic-27}
\end{wrapfigure}

\parbf{\ref{ex:hexagon}.}
Consider the hexagon with flat matric and curved sides shown on the diagram.
Observe that its area can be made arbitrary small while keeping the distances from the opposite sides at least 1.

\parbf{\ref{ex:gadograph}.}
Without loss of generality, we may assume that $V$ lies in a unit cube $\square$.
Consider a noncontinuous metric tensor $\bar g$ on $\square$ that coincides with $g$ inside $V$ and with the canonical flat metric tensor outside of $V$.

Observe that the $\bar g$-distances between opposite faces of $\square$ are at least 1.
Indeed this is true for the Euclidean metric and the assumption $\dist{p}{q}{g}\ge\dist{p}{q}{\EE^d}$  guarantees that one cannot make a shortcut in~$V$.
Therefore the $\bar g$-distances between every pair of opposite faces is at least as large as 1 which is the Euclidean distance.

This metric tensor $\bar g$ is not continuous at $\Sigma$, but the same argument as in \ref{thm:besikovitch} can be applied to show that $\vol(\square,\bar g)\ge \vol\square$.
Whence the statement follows.


\parbf{\ref{ex:involution-of-sphere}.}
Let $x\in \mathbb{S}^2$ be a point that minimize the distance $|x-x'|_g$.
Consider a minimizing geodesic $\gamma$ from $x$ to $x'$.
We can assume that 
\[|x-x'|_g=\length \gamma=1.\]

Let $\gamma'$ be the antipodal arc to $\gamma$.
Note that $\gamma'$ intersects $\gamma$ only at the common endpoints $x$ and $x'$.
Indeed, if $p'=q$ for some $p,q\in\gamma$, then $|p-q|\ge 1$.
Since $\length \gamma=1$, the points $p$ and $q$ must be the ends of $\gamma$.

It follows that $\gamma$ together with $\gamma'$ forms a closed simple curve in $\mathbb{S}^2$
that divides the sphere into two disks $D$ and $D'$.

Let us divide $\gamma$ into two equal arcs $\gamma_1$ and $\gamma_2$; each of length $\tfrac12$.
Suppose that $p,q\in\gamma_1$, then 
\begin{align*}
|p-q'|_g&\ge |q-q'|_g-|p-q|_g\ge
\\
&\ge 1-\tfrac12=\tfrac12.
\end{align*}
That is, the minimal distance from $\gamma_1$ to $\gamma_1'$ is at least~$\tfrac12$.
The same way we get that the minimal distance from $\gamma_2$ to $\gamma_2'$ is at least~$\tfrac12$.
By Besicovitch inequality, we get that 
\[\area(D,g)\ge \tfrac14\quad\text{and}\quad \area(D',g)\ge \tfrac14.\]
Therefore 
\[\area(\mathbb{S}^2,g)\ge\tfrac12.\]

\parit{A better estimate.}
Let us indicate how to improve the obtained bound to
\[\area(\mathbb{S}^2,g)\ge1.\]

Suppose $x$, $x'$, $\gamma$ and $\gamma'$ are as above.
Consider the function
\[f(z)=\min_t \{\,|\gamma'(t)-z|_g+t\,\}.\]
Observe that $f$ is 1-Lipschitz.

Show that two points $\gamma'(c)$ and $\gamma(1-c)$ lie on one connected component of the level set $L_c=\set{z\in\mathbb{S}^2}{f(z)=c}$;
in particular 
\[\length L_c\ge 2\cdot|\gamma'(c)-\gamma(1-c)|_g.\]
By the triangle inequality, we have that
\begin{align*}
|\gamma'(c)-\gamma(1-c)|_g&\ge 1-|\gamma(c)-\gamma(1-c)|_g=
\\
&=1-|1-2\cdot c|.
\end{align*}

It remains to apply the coarea formula
\[\area(\mathbb{S}^2,g)\ge \int\limits_0^1\length L_c\cdot dc.\]

\parit{Remarks.}
The bound $\tfrac12$ was proved by Marcel Berger \cite{berger}. 
Christopher Croke conjectured that the optimal bound is $\tfrac4\pi$ and the round sphere is the only space that achieves this \cite[Conjecture 0.3 in][]{croke}.

\begin{wrapfigure}{r}{20 mm}
\vskip-0mm
\centering
\includegraphics{mppics/pic-1305}
\end{wrapfigure}

\parbf{\ref{ex:involution-of-3sphere}.}
Given $\eps>0$, construct a disk $\Delta$ in the plane with 
\[\length\partial \Delta<10\ \ \text{and}\ \ \area \Delta<\eps\]
that admits an continuous involution $\iota$ such that 
\[|\iota(x)-x|\ge 1\]
for any $x\in\partial \Delta$.

An example of $\Delta$ can be guessed from the picture;
the invoultion $\iota$ makes a length preserving half turn of its boundary $\partial \Delta$.


Take the product $\Delta\times \Delta\subset \RR^4$;
it is homeomorphic to the 4-ball.
Note that 
$$\vol_3[\partial(\Delta\times \Delta)]=2\cdot\area \Delta\cdot\length \partial \Delta<20\cdot\eps.$$
The boundary $\partial(\Delta\times \Delta)$ is homeomorphic to $\mathbb{S}^3$
and the restriction of the involution $(x,y)\z\mapsto (\iota(x),\iota(y))$ has the needed property.

All we have to do now is to smooth $\partial(\Delta\times \Delta)$ a little bit.

\parit{Remark.} This example is given by Christopher Croke \cite{croke}.
Note that according to \ref{thm:sys+}, 
the involution $\iota$ cannot be made isometric.

\parbf{\ref{ex:GH-vol}.}
Note that if $\spc{M}_\infty$ is $e^{\pm\eps}$-bilipschitz to a cube, then applying Besicovitch inequality, we get that 
\[\liminf_{n\to\infty} \vol \spc{M}_n\ge e^{-n\cdot \eps}\cdot\vol \spc{M}_\infty.\]

Applying Vitali covering theorem, given $\eps>0$, we can cover whole volume of $\spc{M}_\infty$ by $e^{\pm\eps}$-bilipschitz cubes.
Applying the above observation and summing up the results, we get that 
\[\liminf_{n\to\infty} \vol \spc{M}_n\ge e^{-n\cdot \eps}\cdot\vol \spc{M}_\infty.\]
The statement follows since $\eps$ is arbitrary positive number.

\parit{Remark.} A more general result was obtaind by Sergei Ivanov~\cite{ivanov-1997}.
Note that the statement does not hold without stability of the convergence. In fact any compact metric space can be approximated by Riemannian surface with arbitrary small area.

\parbf{\ref{ex:sysT2}.}
Set $s=\sys(\TT^2,g)$.

Cut $\TT^2$ along a shortest closed noncontractible curve $\gamma_1$.
We get an anulus with a Riemnnian metric on it $(N,g)$.
Denote by $A$ and $A'$ the two components of its boundary.

Assume that $\gamma_2$ is a shortest path that runs from $A$ to $A'$ in $(N,g)$.
The image of $\gamma_2$ in $\TT^2$ connects two points in $\gamma_1$;
further we will use the same notation for $\gamma_2$ and its image in $\TT^2$.
Connect $\gamma_2(0)$ to $\gamma_2(1)$ by a shorter arc in $\gamma_1$.
Note that the obtained closed curve is noncontractible in $\TT^2$.
Therefore its length is at least $s$.
The arc of $\gamma_1$ has length at most half of $\length\gamma_1$.
Whence $\length \gamma_2\ge \tfrac s2$.
In particular the distance from $A$ to $A'$ in $(N,g)$ is at least $\tfrac s2$.

\begin{wrapfigure}{r}{45 mm}
\vskip-4mm
\centering
\includegraphics{mppics/pic-23}
\end{wrapfigure}

Let us cut $(N,g)$ by $\gamma_2$, we obtain a square $(\square,g)$ with Riemnnian metric on it.
Let us keep the notation $A$ and $A'$ for the pair of opposite sides in $(\square,g)$ that correspond to $A$ and $A'$ in $(N,g)$.
From above we have that distance from $A$ to $A'$ is at least $\tfrac s2$.

Denote by $B$ and $B'$ the remaining pair of opposite sides $(\square,g)$.
Suppose that $\gamma_3$ is a path connecting these sides.
Map it the curves $\gamma_i$ back to the torus and let us keep for them the same notation.
The path $\gamma_3$ connects two points on $\gamma_2$.
Since $\gamma_2$ is shortest, the arc of $\gamma_2$ between this pair of points cannot be longer than $\gamma_3$.
This arc together with $\gamma_3$ forms a closed noncontractible curve, so its length has to be at least $s$.
It follows that $\length\gamma_3\ge \tfrac s2$.
That is distance from $B$ to $B'$ in  $(\square,g)$ is at least $\tfrac s2$.

Applying Besikovitch inequality, we get that 
\[\area(\TT^2,g)=\area(\square,g)\ge \tfrac14\cdot s^2.\]

\parit{Remark.}
Alternatively one may notice that any curve in $(N,g)$ that is bordant to $A$ has length at least $\tfrac s2$.
Therefore the level sets defined by $\distfun_A(x)_{(N,g)}=t$ have length at least $\tfrac s2$ if $0\le t\le \tfrac s2$.
Applying coarea fromula we get that
\[\area(\TT^2,g)=\area(N,g)\ge \tfrac12\cdot s^2.\]
This estimate is twice better then the one above, but it is still far from the optimal bound $\tfrac2{\sqrt{3}}\cdot s^2$ in proved by Loewner inequality

\begin{wrapfigure}{r}{44 mm}
\vskip-4mm
\centering
\includegraphics{mppics/pic-25}
\end{wrapfigure}

\parbf{\ref{ex:sysRP2}.}
Set $s\z=\sys (\RP^2,g)$.
Cut $(\RP^2,g)$ along a shortest noncontractible curve $\gamma$.
We obtain $(\DD^2,g)$ --- a disc with metric tensor which we still denote by $g$.
Divide $\gamma$ into two equal arcs $\alpha$ and $\beta$.
Denote by $A$ and $A'$ the two connected components of the inverse image of $\alpha$.
Similarly denote by $B$ and $B'$ the two connected components of the inverse image of $\beta$.

Let $\gamma_1$ be a path from $A$ to $A'$;
map it to $\RP^2$ and keep the same notation for it.
Note that $\gamma_1$ together with a subarc of $\alpha$ forms a closed noncontractible curve in $\RP^2$.
Since $\length\alpha=\tfrac s2$, we have that $\length\gamma_1\ge \tfrac s2$.
It follows that the distance between $A$ and $A'$ in $(\DD^2,g)$ is at least $\tfrac s2$.
The same way we show that the distance between $B$ and $B'$ in $(\DD^2,g)$ is at least $\tfrac s2$.

Note that $(\DD^2,g)$ can be paraneterized by a square with sides $A$, $B$, $A'$ and $B'$ and apply \ref{thm:besikovitch} to show that 
\[\area(\RP^2,g)=\area(\DD^2,g)\ge \tfrac14\cdot s^2.\]

\parit{Remark.}
For the optimal constant was found by Pao Ming Pu see the discussion on page \pageref{page:pu}.
His proof shows that any Riemannian metric on the disc with the boundary globally isometric to a unit circle with angle metric has area at least as large as the unit hemisphere.
It is expected that the same inequality holds for any compact surface bounded by a single curve (not necessary a disc);
this is the so called the {}\emph{filling area conjecture} mentioned in \cite[5.5.B$'$(e$'$)]{gromov-1983}.

\parbf{\ref{ex:sysSg}.} Cut the surface along a shortest noncontractible curve $\gamma$. 
We might get a surface with one or two components of the boundary.
In these two cases repeat the arguments in \ref{ex:sysRP2} or \ref{ex:sysT2} using \ref{thm:besikovitch+} instead of \ref{thm:besikovitch}.


\parbf{\ref{ex:sysS2xS1}.} Consider the product of small 2-sphere with a unit circle.

\parbf{\ref{ex:macrodimension}.}
The following claim resembles Besikovitch inequality;
it is key to the proof:
\begin{itemize}
 \item[$({*})$] Let $a$ be a positive real number.
 Assume that a closed curve $\gamma$ in a metric space $\spc{X}$ can be sudivided into 4 arcs $\alpha$, $\beta$, $\alpha'$, and $\beta'$ in such a way that 
 \begin{itemize}
 \item $|x-x'|>a$ for any $x\in\alpha$ and $x'\in \alpha'$
 and
 \item $|y-y'|>a$ for any $y\in\beta$ and $y'\in \beta'$.
 \end{itemize}
 Then $\gamma$ is not contractable in its $\tfrac a2$-neighborhood.
\end{itemize}

To prove $({*})$, consider two functions defined on $\spc{X}$ as follows:
\begin{align*}
w_1(x)&=\min \{\,a,\distfun_{\alpha}(x)\,\}
\\
w_2(x)&=\min \{\,a,\distfun_{\beta}(x)\,\}
\end{align*}
and the map $\bm{w}\:\spc{X}\to [0,a]\times[0,a]$, defined by
\[\bm{w}\:x\mapsto(w_1(x),w_2(x)).\]

Note that 
\begin{align*}
\bm{w}(\alpha)&=0\times [0,a],
&
\bm{w}(\beta)&=[0,a]\times 0,
\\
\bm{w}(\alpha')&=a\times [0,a],
&
\bm{w}(\beta')&=[0,a]\times a,
\end{align*} 
Therefore, the composition $\bm{w}\circ\gamma$ is a degree 1 map 
\[\mathbb{S}^1\to \partial([0,a]\times[0,a]).\] 
It follows that if $h\:\DD\to \spc{X}$ shrinks $\gamma$, then there is a point $z\in\DD$ such that 
$\bm{w}\circ h(z)=(\tfrac a2,\tfrac a2)$.
Therefore $h(z)$ lies at distance at least $\tfrac a2$ from $\alpha$, $\beta$, $\alpha'$, $\beta'$
and therefore from $\gamma$.
Hence the claim $({*})$ follows.

\medskip

Coming back to the problem, let $\{W_i\}$ be an open covering of the real line with multiplicity $2$ and $\rad W_i<R$ for each $i$;
for example one may take $W_i=((i-\tfrac23)\cdot R,(i+\tfrac23)\cdot R)$.

Choose a point $p\in \spc{X}$.
Denote by $\{V_j\}$ the connected components of $\distfun_p^{-1}(W_i)$ for all $i$.
Note that $\{V_j\}$ is an open finite cover of $\spc{X}$ with multiplicity at most 2.
It remains to show that $\rad V_j<100\cdot R$ for each $j$.

\begin{wrapfigure}{o}{31 mm}
\vskip-2mm
\centering
\includegraphics{mppics/pic-1310}
\end{wrapfigure}

Aarguing by contradiction assume there is a pair of points  $x,y\in V_i$ 
such that $|x\z-y|_{\spc{X}}\ge 100\cdot R$.
Connect $x$ to $y$ with a curve $\tau$ in $V_j$.
Consider the closed curve $\sigma$ formed by $\tau$ and two geodesics $[px]$, $[py]$.


Note that $|p-x|>40$.
Therefore there is a point $m$ on $[px]$ such that $|m-x|=20$.

By the triangle inequality, the subsdivision of $\sigma$ into the arcs $[pm]$, $[mx]$, $\tau$ and $[yp]$ satisfy the conditions of the claim $({*})$ for $a=10\cdot R$.
Hence the statement follows.

\parit{The quasiconverse} does not hold.
As an example take a surface that looks like a long cylinder with two hats,
it is a smooth surface diffeomorphic to a sphere.
\begin{figure}[h!]
\vskip0mm
\centering
\includegraphics{mppics/pic-1315}
\end{figure}
Assuming the cylinder is thin, it has macroscopic dimension 1 at a given scale.
However a circle formed by a section of cylinder around its midpoint by a plane parallel to the base is a circle that cannot be contracted in its small neighborhood.

\parit{Sourse:} \cite[Appendix 1(E$_{2}$)]{gromov-1983}.

\parbf{\ref{ex:width=suprad(inv)},} \textit{``only if'' part.}
Suppose $\width_n\spc{X}<R$.
Consider a covering $\{V_1,\dots,V_k\}$ of $\spc{X}$ guaranteed by the definition of width.
Let $\spc{N}$ be its nerve and $\psi\:\spc{X}\to \spc{N}$ be the map provided by \ref{prop:space->nerve}.

Since the multiplicity of the covering is at most $n+1$, we ahve $\dim \spc{N}\le n$.

Note that if $x\in \spc{N}$ lies in a star of a vertex $v_i$,
then $\psi^{-1}\{x\}\z\subset V_i$;
in particular $\rad[\psi^{-1}\{x\}]<R$.

\parit{``If'' part.}
Choose $x\in \spc{N}$.
Since the inverse image $\psi^{-1}\{x\}$ is compact, $\psi$ is continuous, and $\rad[\psi^{-1}\{x\}]<R$,
there is a neighborhood $U\ni x$ such that the  $\rad[\psi^{-1}(U)]<R$.

Since $\spc{X}$ is compact,  there is a finite cover $\{U_i\}$ of $\spc{N}$ such that $\psi^{-1}(U_i)\subset\spc{X}$ has radius smaller than $R$ for each $i$.
Since $\spc{N}$ has dimension $n$, we can inscribe%
\footnote{Recall that a covering $\{W_i\}$ is inscribed in the covering $\{U_i\}$ if for every $W_i$ is a subset of some $U_j$.} 
in $\{U_i\}$ a finite open cover $\{W_i\}$ with multiplicity at most $n+1$.
It remains to observe that $V_i=\psi^{-1}(W_i)$ defines a finite open cover of $\spc{X}$ with radius less than $R$ and multiplicity at most $n+1$. 


\parbf{\ref{ex:1D-case}.}
Assume that $\spc{P}$ is connected.

Let us show that $\diam\spc{P}<R$.
If this is not the case, then there are points $p,q\in\spc{P}$ on distance $R$ from each other.
Let $\gamma$ be a geodesic from $p$ to $q$.
Clearly $\length\gamma\ge R$ and $\gamma$ lies in $\oBall(p,R)$ except for the endpoint $q$.
Therefore $\length[\oBall(p,R)_{\spc{P}}]\ge R$.
Since $\VolPro_{\spc{P}}(R)\z\ge \length[\oBall(p,R)_{\spc{P}}]$,
the latter contradicts $\VolPro_{\spc{P}}(R)<R$.

In general case, we get that each connected component of $\spc{P}$ has radius smaller that $R$.
Whence the width of $\spc{P}$ is smaller that $R$.

\parit{Second part.} Again, we can assume that $\spc{P}$ is connected.

The examples of line segment or a circle show that the constant $c=\tfrac12$ cannot be improved.
It remains to show that the inequality holds with $c=\tfrac12$.

Choose $p\in\spc{P}$ such that the value
\[\rho(p)=\max\set{\dist{p}{q}{\spc{P}}}{q\in\spc{P}}\]
is minimal.
Suppose $\rho(p)\ge\tfrac 12\cdot R$.
Observe that there is a point $x\in \spc{P}\backslash\{p\}$ that lies on any shortest path starting from $p$ and length $\ge\tfrac 12\cdot R$.
Otherwise for any $r\in(0,\tfrac 12\cdot R)$ there would be at least two points on distance $r$ from $p$;
by coarea inequality we get that the total length of $\spc{P}\cap \oBall(p,\tfrac 12\cdot R)$ is at least $R$ --- a contradiction.

Moving $p$ toward to $x$ reduce $\rho(p)$ which contradicts the choice of~$p$.

\parbf{\ref{ex:connected-sum-essential}.}
Suppose $M$ is an essential manifold and $N$ is arbitrary closed manifold.
Observe that shrinking $N$ to a point produces a map $f\:N\#M\to M$ of degree 1; that is, the fundamental class of $N\#M$ maps to the fundamental class of $M$.

Since $M$ is essential, there is an aspherical space $K$ and a map $\iota\:M\to K$ that sends fundamental class of $M$ to nonzero homology class in $K$.
From above, the composition $\iota\circ f\:N\#M\to K$ sends fundamental class of $N\#M$ to the same homology class in $K$.

\parit{Remark.} Note that we only used that there is a map $N\#M\to K$ of degree 1. If essential manifold is defined using homologies with integer coefficients, then existence of map of nonzero degree is sufficient.


%\chapter{Midterm}\label{chap:midterm}

An oral exam, Th, Feb 27 in class.

\bigskip

\noi 
One theoretical questions from the following list:

\begin{enumerate}
\item 
Semicontinuity of length.
\item
Length spaces and Hopf--Rinow theorem.
\item
Fréchet lemma and Kuratowski embedding.
\item
Hausdorff convergence and Blaschke selection theorem.
\item
Gromov--Hausdorff metric, why it is a metric, almost isometries.
\item
Uniformly totally bonded families and Gromov selection theorem.
\item
Ultralimits and ultrapower of spaces.
\item
Urysohn space.
\item
Injective spaces and injective envelop.
\end{enumerate}

\bigskip

\noi One exercise from the following list:
\\
\ref{ex:almost-min},
\ref{ex:non-contracting-map},
\ref{ex:compact=>complete},
\ref{ex:compact-length},
\\
\ref{ex:Huas-perimeter-area},
\\
\ref{pr:doubling},
\ref{pr:under},
\ref{ex:GH-SC},
\ref{ex:sphere-to-ball},
\\
\ref{ex:ultrapower}, 
\ref{ex:two-geodesics-in-ultrapower},
\ref{ex:lim(tree)},
\\
\ref{ex:geodesics-urysohn},
\ref{ex:sphere-in-urysohn},
\ref{ex:compact-extension},
\\
\ref{ex:+-c},
\ref{ex:ultrametric},
\ref{ex:injective-spaces},
\ref{ex:tripod+square},
\ref{ex:4-on-a-line}.

\bigskip

\noi One more problem for a perfect score.

%%%%%%%%%%%%%%%%%%%%%%%%%%%%
{\small\sloppy
\RequirePackage{snapshot}
\documentclass[twoside]{book}

\usepackage{lectures}
\usepackage[colorlinks=true,
citecolor=black,
linkcolor=black,
anchorcolor=black,
filecolor=black,
menucolor=black,
urlcolor=black,
pdftitle={Metric geometry on manifolds: two lectures},
pdfsubject={Geometry},
pdfauthor={Anton Petrunin}
]{hyperref}
\makeindex

\begin{document}
 
\title{Metric geometry on manifolds:
\\ two lectures}
\author{Anton Petrunin}
\date{}
\maketitle

We discuss Besikovitch inequality, width, and systole of manifolds.

We assume that students familiar with the smooth manifolds, degree of map, CW-complexes and related notions.

These are two final lectures of a graduate course given at Penn State, Spring 2020.
The complete lectures can be found on the authors website;
it includes an introduction to metric geometry \cite{petrunin2020pure}
and elements of Alexandrov geometry based on \cite{alexander-kapovitch-petrunin-2019}.

\thispagestyle{empty}
\tableofcontents
\thispagestyle{empty}

%%%%%%%%%%%%%%%%%%%%%%%%%%%%
%\addtocounter{chapter}{-1}
\chapter{Homework assignments}


It is better to think about all the problems, but you do not have to solve \emph{all} of them.
If a problem is solved, you do not have to write its solutions, but try sketch it.

\section{Due Tue Jan 21}

Exercises: \sout{\ref{ex:almost-min},} \ref{ex:non-contracting-map}, \ref{ex:no-geod}, \sout{\ref{ex:compact=>complete},} \ref{exercise from BH}, \ref{ex:Hausdorff-bry}.

\section{Due Tue Jan 28}

Exercises: \ref{ex:almost-min},  \ref{ex:compact=>complete}, \ref{ex:Huas-perimeter-area}, \ref{ex:GH-po}, \ref{pr:doubling}, \ref{pr:under:if}.

\section{Due Tue Feb 4}
Exercises: 
\ref{ex:compact-length}, 
\ref{pr:under:only-if}, 
\sout{\ref{ex:GH-SC},}
\sout{\ref{ex:sphere-to-ball},}
\ref{ex:ultrapower}, 
\ref{ex:two-geodesics-in-ultrapower}.

\section{Due Tue Feb 11}

Finish exercises \ref{ex:compact-length} , \ref{pr:under:only-if}, \ref{ex:GH-SC}, \ref{ex:sphere-to-ball}.

\noindent
Exercises: \ref{ex:lim(tree)}, \ref{ex:Asym(Lob)}, \ref{ex:geodesics-urysohn}, \ref{ex:sphere-in-urysohn}.

\section{Due Tue Feb 18}

Exercises: \ref{ex:compact-extension}, \ref{ex:+-c}, \ref{ex:ultrametric}, \ref{ex:injective-spaces}, \ref{ex:tripod+square}, \ref{ex:4-on-a-line}.

\noindent Write down a solution of at least one of the exercises.

\section{Due Tue Feb 25}

Finish Exercise \ref{ex:tripod+square:square}.
Prepare questions for review on Tuesday.

\section{Due Tue Mar 3}

Exercises: \ref{ex:sba-2+2-short}, \ref{ex:(3+1)-expanding}, \ref{ex:CAT+CBB}, \ref{ex:product-CBB}, \sout{\ref{ex:CBB-geodesic},} \ref{ex:fat-triangle}.

\noindent Write down a solution of at least one of the exercises.

\section{Due Tue Mar 17}

Exercises: \ref{ex:tringle-inq-angles},
\ref{ex:CBB-geodesic},
\ref{ex:convex-dist},
\ref{ex:reshetnyak-doubling},
\ref{ex:supporting-planes},
\ref{ex:centrally-simmetric-walls}.

\noindent Write down a solution of at least one of the exercises.

\section{Due Tue Mar 24}

Exercises: 
\ref{ex:contractible},
\ref{ex:convex-nbhd},
\ref{ex:closest-point},
\ref{cor:balls:dim=1},
\ref{ex:null-homotopic},
\sout{\ref{ex:branching-cover}.}

 Write down as many solutions as you can; email it to Zetian Yan (zxy5156) + cc to me (aqp6).

\section{Due Tue Mar 31}

Exercises: 
\ref{ex:branching-cover},
\ref{ex:tan(CAT)isCAT},
\ref{ex:tan(CAT)is-length},
%\ref{ex:product-cone},
\ref{ex:unique-geod=CAT},
\ref{ex:flag>=pi/2},
\ref{ex:tree}.

Write down as many solutions as you can; email it to Zetian Yan (zxy5156) + cc to me (aqp6).

\section{Due Tue Apr 7}

Exercises: 
\ref{ex:CAT-mnfld=>ext.geod},
\ref{ex:locally-convex},
\ref{ex:geod-circle},
\ref{ex:flag-aspherical},
\ref{ex:example-pi_infty-new},
\ref{ex:cube-infty=>cube-2}.

Write down as many solutions as you can; email it to Zetian Yan (zxy5156) + cc to me (aqp6).

\section{Due Tue Apr 14}

Exercises: 
\ref{ex:geod-CBA},
%\ref{prop:two-hull-open},
%\ref{ex:chopping-triangle},
\ref{ex:concave-triangle},
\ref{ex:two-planes},
\ref{ex:hemisphere},
\ref{ex:inner-support},
\ref{ex:convex+saddle+broken=>PL}.

Write down as many solutions as you can; email it to Zetian Yan (zxy5156) + cc to me (aqp6).

\section*{Remark}
Each working day I will check email before 15:00 and will appear online if you ask (it is easy for me --- do not hesitate to ask).
We will meet regular hours online (as we did before).

%%%%%%%%%%%%%%%%%%%%%%%%%%%%

\chapter{Besicovitch inequality} 

We will focus on Riemannian spaces --- these are specially nice length metrics on manifolds.
These spaces are also most important in applications.

As it will be indicated in Section~\ref{sec:hausdorff-measure},
most of the statements of this and the following lecture have counterparts for general length metrics on manifolds.

\section{Riemannian spaces}

Let $M$ be a smooth connected manifold.
A \index{metric tensor}\emph{metric tensor} on $M$ is a choice of positive definite quadratic forms $g_p$ on each tangent space $\T_pM$ that depends continuously on the point;
that is, in any local coordinates of $M$ the components of $g$ are continuous functions.

A \index{Riemannian!manifold}\emph{Riemannian manifold} $(M,g)$ is a smooth manifold $M$ with a choice of metric tensor $g$ on it.

The \index{length}\emph{$g$-length} of a Lipschitz curve $\gamma\:[a,b]\to M$  is defined by
\[\length_g\gamma=\int_a^b\sqrt{g(\gamma'(t),\gamma'(t))}\cdot dt.\]
The $g$-length induces a metric metric on $M$; it is defined as the greatest lower bound to lengths of Lipschitz curves connecting two given points;
the distance between a pair of points $x,y\in M$ will be denoted by 
\[\dist{x}{y}{g}\quad\text{or}\quad\distfun_x(y)_g.\]
The corresponding metric space $\spc{M}$ will be called \index{Riemannian!space}\emph{Riemannian}.

The following exercise implies that \textit{isometry between Riemannian spaces might be not induced by a diffeomorphism}.

\begin{thm}{Exercise}\label{ex:non-differentiable}
Construct a continuous Riemannian metric $g$ on $\RR^2$ such that the corresponding Riemannian space admits an isometry to the Euclidean palne but the induced map $\iota\:\RR^2\to\RR^2$ is not differentiable at some point.
\end{thm}

The exercise above shows that in general the smooth structure is not uniquely defined on Riemannian space.
Therefore in general case one has to distinguish between Riemannian manifold and the corresponding Riemannian space altho there is almost no difference.%
\footnote{In fact a straightforward smoothing procedure shows that isometry between Riemannian spaces can be approximated by diffeomorphisms between underlying manifolds; in particular these manifolds are diffeomorphic.
Also, if the metric tensor is smooth, then it is not hard to show that Riemannian space {}\emph{remembers} everything about the Riemannian manifold, in particular the smooth structure;
it is a part of the so-called Myers--Steenrod theorem \cite{myers-steenrod}.} 

The following observation states the key property of Riemannian spaces;
it will be used to extend results from Euclidean space to Riemannian spaces.

\begin{thm}{Observation}\label{obs:lip-chart}
For any point $p$ in a Riemannian space $\spc{M}$ and any $\eps>0$ there is a $e^{\mp\eps}$-bilipschitz chart $s\:W\to V$ from an open subset $W$ of the $n$-dimensional Euclidean space to some neighborhood $V\ni p$.
\end{thm}

\parit{Proof.}
Choose a chart $s\:U\to \spc{M}$ that covers $p$.
Note that there is a linear transformation $L$ such that for the metric tensor in the chart $s\circ L$ is coincides with the standard Euclidean tensor at the point $x=(s\circ L)^{-1}(p)$.

Since the metric tensor is continuous, the restriction of $s\circ L$ to a small neighborhood of $x$ is $e^{\mp\eps}$-bilipschitz.
\qeds

\section{Volume and Hausdorff measure}\label{sec:vol-haus}

Let $(M,g)$ be an $n$-dimensional Riemannian manifold.
If a Borel set $R\subset M$ is covered by one chart $\iota\:U\to M$,
then its \index{volume}\emph{volume} (briefly, $\vol R$ or $\vol_n R$) is defined by 
\[\vol R
\df
\int_{\iota^{-1}(R)}\sqrt{\det{g}}.\]
In the general case we can subdivide $R$ into a countable collection of regions $R_1,R_2\dots$ such that each region $R_i$ is covered by one chart $\iota_i\:U_i\to M$ and define
\[\vol R\df \vol R_1+\vol R_2+\dots\]
The chain rule for multiple integrals implies that the right-hand side does not depend on the choice of subdivision and the choice of charts.

Similarly, we define integral along $(M,g)$.
Any Borel function $u\:M\z\to \RR$, can be presented as a sum $u_1+u_2+\cdots$ such that the support of each function $u_i$ can be covered by one chart $\iota_i\:U_i\to M$
and set 
\[\int_{p\in\spc{M}} u(p)
\df
\sum_i\left[\int_{x\in U_i} u_i\circ s(x)\cdot\sqrt{\det{g}}\right].
\]
In particular
\[\vol R=\int_{p\in R} 1.\]

Let $\spc{X}$ be a metric space and $R\subset \spc{X}$.
The \index{Hausdorff measure}\emph{$\alpha$-dimensional Hausdorff measure} of $R$ is defined by 
$$\haus_\alpha R
\df
\lim_{\eps\to0}
\,
\inf
\set{\sum_{n\in\NN}(\diam A_n)^\alpha}
{\begin{aligned}
&\diam A_n<\eps\ \text{for}
\\
&\text{for each}\ n,\text{all}\  A_n
\\
&\text{are closed, and} 
\\
& \bigcup_{n\in\NN}A_n\supset R.
\end{aligned}
}.$$
For properties of Hausdorff measure we refer to the classical book of  Herbert Federer \cite{federer};
in particular, $\haus_\alpha$ is indeed a measure and $\haus_\alpha$-measurable sets include all Borel sets.

The following observation follows from \ref{obs:lip-chart} and Rademacher's theorem:

\begin{thm}{Observation}\label{obs:lipcart+}
Suppose that a Borel set $R$ in an $n$-dimensional Riemannian space $\spc{M}$ is subdivided into a countable collection of subsets $R_i$ such that each $R_i$ is covered by an $e^{\mp\eps}$-bilipschitz charts
$s_i$.
Then
\begin{align*}
\vol_n R&\lege e^{\pm n\cdot\eps}\cdot\sum_i\vol_n[s_i^{-1}(R_i)]
\intertext{and}
\haus_n R&\lege e^{\pm n\cdot\eps}\cdot\sum_i\haus_n[s_i^{-1}(R_i)]
\end{align*}

\end{thm}

According to \index{Haar's theorem}\emph{Haar's theorem}, 
a measure on $n$-dimensional Euclidean space that is invariant with respect to parallel translations is proportional to volume.
Observe that 
\begin{itemize}
\item A ball in $n$-dimensional Euclidean space of diameter $1$ has unit Hausdorff measure.
\item A unit cube in $n$-dimensional Euclidean space has unit volume.
\end{itemize}
Therefore, for any Borel region $R\subset \EE^n$, we have 
\[\vol_n R=\tfrac{\omega_n}{2^n}\cdot\haus_n R,\eqlbl{eq:vol/mu}\]
where $\omega_n$ denotes the volume of a unit ball in the $n$-dimensional Euclidean space.

Applying \ref{eq:vol/mu} together with \ref{obs:lipcart+}, we get that the inequalities
\[\vol_n R\lege e^{\pm2\cdot n\cdot \eps}\cdot\tfrac{\omega_n}{2^n}\cdot\haus_n R\]
hold for any $\eps>0$.
Since $\eps>0$ is arbitrary, we get that \ref{eq:vol/mu} holds in $n$-dimensional Riemannian spaces.
More precisely:

\begin{thm}{Proposition}\label{prop:vol=haus}
The identity 
\[\vol_n R=\tfrac{\omega_n}{2^n}\cdot\haus_n R\]
holds for any Borel region $R$ in an $n$-dimensional Riemannian space. 
\end{thm}

Since the Hausdorff measure is defined in pure metric terms, the proposition gives another way to prove that the volume does not depend on the choice of chars and subdivision of $R$.

The identity in this proposition will be used to define volume of any dimension.
Namely, given an integer $k\ge 0$, the $k$-volume is defined by
\[\vol_k\df\tfrac{\omega_k}{2^k}\cdot\haus_k.\]
By \ref{prop:vol=haus}, if $A$ is a subset of $k$-dimensional submanifold $\spc{N}\subset \spc{M}$, then the two definitions of $\vol_kA$ agree; but the latter definition works for a wider class of sets. 

\begin{thm}{Exercise}\label{ex:volume-preserving+short}
Let $f\:\spc{M}\to \spc{N}$ be a short volume-preserving map between $n$-dimensional Riemannian spaces.
Prove the following statements and use them to conclude that $f$ is locally distance-preserving.

\begin{subthm}{ex:volume-preserving+short:injective}
$f$ is injective; 
that is, if $f(x)=f(y)$, then $x=y$.
\end{subthm}

\begin{subthm}{ex:volume-preserving+short:bi}
For any $c<1$, the map $f$ is locally $[c,1]$-bilipschitz;
that is, for any point in $\spc{M}$ there is a neighborhood $\Omega$ and $\eps>0$ such that the inequality 
\[c\le \frac{|f(x)-f(y)|_{\spc{N}}}{|x-y|_{\spc{M}}}\le 1 \]
holds for any pair of distinct points $x,y\in \Omega$.
\end{subthm}

\end{thm}


\section{Area and coarea formulas}

Suppose that $f\:\spc{M}\to\spc{N}$ is a Lipschitz map between $n$-dimensional Riemannian spaces $\spc{M}$ and $\spc{N}$.
Then by \index{Rademacher's theorem}\emph{Rademacher's theorem} 
the differential $d_p f\:\T_p\spc{M}\to\T_{f(p)}\spc{N}$ is defined at \index{almost all}\emph{almost all} $p\in \spc{M}$;
that is, the differential defined at all points $p\in\spc{M}$ with exception of a subset with vanishing volume.

The differential is a linear map; it defines the Jacobian matrix $\Jac_pf$ in orthonormal frames of $\T_p$ and $\T_{f(p)}\spc{N}$.
The determinant of $\Jac_pf$ will be denoted by $\jac_p$.
Note that the absolute value $|\jac_p|$ does not depend on the choice of the orthonormal frames.

The identity in the following proposition is called \index{area formula}\emph{area formula}.

\begin{thm}{Proposition}
Let $f\:\spc{M}\to\spc{N}$ be a Lipschitz map between $n$-dimensional Riemannian spaces $\spc{M}$.
Then for  any Borel function $u\:\spc{M}\z\to \RR$ the following equality holds:
\[\int_{p\in \spc{M}} u(p)\cdot |\jac_pf|=\int_{q\in \spc{N}}\sum_{p\in f^{-1}(q)} u(p).\]

\end{thm}

\parit{Proof.}
If $\spc{M}$ and $\spc{N}$ are isometric to the $n$-dimensional Euclidean space, then the statement follows from the standard area formula \cite[3.2.3]{federer}.

Note that Jacobian of a $e^{\mp\eps}$-bilipschitz map between $n$-dimensional Riemannian manifolds (if defined) has determinant in the range $e^{\mp n\cdot\eps}$.
Applying \ref{obs:lipcart+} and the area formula in $\EE^n$, we get the following approximate version of the needed identity for any $u\ge0$: 
\[\int_{p\in \spc{M}} u(p)\cdot |\jac_pf|
\lege e^{\pm 3\cdot n\cdot \eps}\int_{q\in \spc{N}}\sum_{p\in f^{-1}(q)} u(p).\]

Since $\eps>0$ is arbitrary, we get that the area formula holds if $u\ge 0$.
Finally, since both sides of the area formula are linear in $u$, it holds for any $u$.
\qeds

The following inequality is called \index{area inequality}\emph{area inequality}:

\begin{thm}{Corollary}\label{cor:area-inequality}
Let $f\:\spc{M}\to\spc{N}$ be a locally Lipschitz map between $n$-dimensional Riemannian spaces.
Then 
\[\int_{p\in A} |\jac_p f|\ge \vol[f(A)]\]
for any Borel subset $A\subset M$.

In particular, if $|\jac_p f|\le 1$ almost everywhere in $A$, then 
\[\vol A \ge \vol[f(A)].\]
\end{thm}

\parit{Proof.} Apply the area formula to the characteristic function of $A$.
\qeds

Suppose that $f\:\spc{M}\to\RR$ is a Lipschitz function defined on an $n$-dimensional Riemannian space $\spc{M}$.
Then by Rademacher's theorem, the differential $d_pf\:\T_p\spc{M}\to\RR$  and the gradient 
$\nabla_pf\in\T_p\spc{M}$ are defined at almost all $p\in \spc{M}$.

The identity in the following proposition is a partial case of the so-called \index{coarea formula}\emph{coarea formula}.
(The general coarea formula deals with the maps to the spaces of arbitrary dimension, not necessary $1$.)


\begin{thm}{Proposition}\label{prop:coarea}
Let $f\:\spc{M}\to\RR$ be a Lipschitz function defined on an $n$-dimensional Riemannian space $\spc{M}$.
Suppose that the level sets $L_x\df f^{-1}(x)$ are equipped with $(n-1)$-dimensional volume $\vol_{n-1}\z\df\tfrac{\omega_{n-1}}{2^{n-1}}\cdot \haus_{n-1}$.
Then for any Borel function $u\:\spc{M}\to \RR$ the following equality holds
\[\int_{p\in \spc{M}} u(p)\cdot |\nabla_pf|=\int_{-\infty}^{+\infty} \left(\,\int_{p\in L_x} u(p)\,\right)\cdot dx.\]
\end{thm}

The following corollary is a partial case of the so-called  \index{coarea inequality}\emph{coarea inequality};

\begin{thm}{Corollary}\label{cor:coarea}
Let $\spc{M}$, $f$, and $L_x$ be as in \ref{prop:coarea}.

Suppose that $f$ is 1-Lipschitz.
Then for any Borel subset $A\subset M$ we have
\[\vol_n A\ge \int_{x\in\RR} \vol_{n-1}[A\cap L_x]\cdot dx.\eqlbl{eq:coarea-inq}\]
\end{thm}

The right-hand side in \ref{eq:coarea-inq} is called \index{coarea}\emph{coarea of the restriction $f|_A$}. 


\parit{Instead of proof of \ref{prop:coarea} and \ref{cor:coarea}.}
If $\spc{M}$ is isometric to Euclidean space, then the statement follows from the standard coarea formula \cite[3.2.12]{federer}.
The reduction to the Euclidean space is done the same way as in the proof of the area formula.

To prove the corollary, choose $u$ to be the characteristic function of $A$ and apply the coarea formula.
\qeds


\section{Besicovitch inequality}

A closed connected region in a Riemannian manifold bounded by hypersurface will be called \index{Riemannian!manifold with boundary}\emph{Riemannian manifold with boundary}.
We always assume that the hypersurface can be realized locally as a graph of Lipschitz function in a suitable chart.
In this case one can define $g$-length, $g$-distance, and $g$-volume the same way as we did for usual Riemannian manifolds.

\begin{thm}{Exercise}\label{ex:compact-interior}
Suppose that $(M,g)$ is a compact Riemannian manifold with boundary. 
Observe that the interior $(M^\circ,g)$ of $(M,g)$ is a usual Riemannian manifold.
Show that the space of $(M,g)$ is isometric to the completion of the space of $(M^\circ,g)$.
\end{thm}
 

\begin{thm}{Theorem}\label{thm:besikovitch}
Let $g$ be a continuous metric tensor on a unit $n$-dimensional cube $\square$.
Suppose that the $g$-distances between the opposite faces of $\square$ are at least $1$; that is, any Lipschitz curve that connects opposite faces has $g$-length at least $1$.
Then \[\vol(\square, g)\ge 1.\]

\end{thm}

This is a partial case of the theorem proved by Abram Besicovitch \cite{besicovitch}.

\parit{Proof.}
We will consider the case $n=2$; the other cases are proved the same way.

\begin{wrapfigure}{r}{30mm}
\vskip-0mm
\centering
\includegraphics{mppics/pic-1320}
\end{wrapfigure}

Denote by $A$, $A'$, and $B$, $B'$ the opposite faces of the square~$\square$.
Consider two functions
\begin{align*}
f_A(x)&\df\min\{\,\distfun_A(x)_g,1\,\},
\\
f_B(x)&\df\min\{\,\distfun_B(x)_g,1\,\}.
\end{align*}
Let $\bm{f}\:\square\to\square$ be the map with coordinate functions $f_A$ and $f_B$;
that is, $\bm{f}(x)\df(f_A(x), f_B(x))$.

\begin{clm}{}\label{f:A->A}
The map $\bm{f}$ sends each face of $\square$ to itself.
\end{clm}


Indeed, 
\[x\in A \quad\Longrightarrow\quad \distfun_A(x)_g=0 \quad\Longrightarrow\quad f_A(x)=0 \quad\Longrightarrow\quad \bm{f}(x)\in A.\]
Similarly, if $x\in B$, then $\bm{f}(x)\in B$.
Further, 
\[x\in A'
\quad\Longrightarrow\quad 
\distfun_A(x)_g\ge 1 
\quad\Longrightarrow\quad 
f_A(x)=1 
\quad\Longrightarrow\quad 
\bm{f}(x)\in A'.\]
Similarly, if $x\in B'$, then $\bm{f}(x)\in B'$.

By \ref{f:A->A}, it follows 
\[\bm{f}_t(x)= t\cdot x + (1-t)\cdot \bm{f}(x)\]
defines a homotopy of maps of the pair of spaces $(\square,\partial \square)$ from $\bm{f}$ to the identity map;
that is, $(t,x)\mapsto \bm{f}_t(x)$ is a continuous map and if $x\in \partial \square$, then $\bm{f}_t(x)\in \partial \square$ for any $t\in [0,1]$.

It follows that $\deg\bm{f}=1$; that is, $\bm{f}$ sends the fundamental class of $(\square,\partial \square)$ to itself.%
\footnote{Here and further, we assume that homologies are taken with the coefficients in $\ZZ_2$, but you are welcome to play with other coefficients.}
In particular $\bm{f}$ is onto.

Suppose that Jacobian  matrix $\Jac_p\bm{f}$ of $\bm{f}$ is defined at $p\in \square$.
Choose an orthonormal frame in $\T_p$ with respect to $g$ and the standard frame in the target $\square$.
Observe that the differentials $d_pf_A$ and $d_pf_B$ written in these frames are the rows of $\Jac_p\bm{f}$.
Evidently $|d_pf_A|\le 1$ and $|d_pf_B|\le 1$.
Since the determinant of a matrix is the volume of the parallelepiped spanned on its rows, we get 
\[|\jac_p \bm{f}|\le |d_pf_A|\cdot|d_pf_B|\le 1.\]
Since $\bm{f}\:\square\to\square$ is a Lipschitz onto map, the area inequality (\ref{cor:area-inequality}) implies that 
\[\vol(\square,g)\ge \vol\square=1.\]
\qedsf

If the $g$-distances between the opposite sides are $d_1,\dots ,d_n$, then following the same lines  one get that 
$\vol (\square,g)\ge d_1\cdots d_n$.
Also note that in the proof we use topology of the $n$-cube only once, to show that the map $f$ has degree one.
Taking all this into account we get the following generalization of \ref{thm:besikovitch}:

\begin{thm}{Theorem}\label{thm:besikovitch+}
Let $(M,g)$ be an $n$-dimensional Riemannian manifold with coonected boundary $\partial M$.
Suppose that there is a degree 1 map $\partial M\to \partial\square$;
denote by $d_1,\dots, d_n$ the $g$-distances between the inverse images of pairs of opposite faces of $\square$ in $M$.
Then 
\[\vol(M,g)\ge d_1\cdots d_n.\]

\end{thm}

\begin{thm}{Exercise}\label{ex:besikovitch=}
Show that if equality holds in \ref{thm:besikovitch+},
then $(M,g)$ is isometric to the rectangle $[0,d_1]\times\dots\times[0, d_n]$.
\end{thm}



\begin{thm}{Exercise}\label{ex:hexagon}
Suppose $g$ is a metric tensor on a regular hexagon $\text{\rm\hexagon}$ such that $g$-distances between the opposite sides are at least $1$.
Is there a positive lower bound on $\area(\text{\rm\hexagon},g)$?
\end{thm}

\begin{thm}{Exercise}\label{ex:cylinder}
Let $g$ be a Riemannian metric on the cylinder $\mathbb{S}^1\z\times [0,1]$.
Suppose that 
\begin{itemize}
\item 
$g$-distance between pairs of points on the opposite boundary circles $\mathbb{S}^1\times\{0\}$ and $\mathbb{S}^1\times\{1\}$ is at least 1, and 
\item
any curve $\gamma$ in $\mathbb{S}^1\times [0,1]$ that is homotopic to $\mathbb{S}^1\times\{0\}$ has $g$-length at least $1$.
\end{itemize}

\begin{subthm}{ex:cylinder:besicovitch}
Use Besicovitch inequality to show that
\[\area(\mathbb{S}^1\times [0,1],g)\ge \tfrac12.\]

\end{subthm}

\begin{subthm}{ex:cylinder:coarea}
Modify the proof of Besicovitch inequality using coarea inequality (\ref{cor:coarea}) to prove the optimal bound  
\[\area(\mathbb{S}^1\times [0,1],g)\ge 1.\]
 
\end{subthm}

\end{thm}

\begin{thm}{Exercise}\label{ex:gadograph}

\begin{subthm}{ex:gadograph-besikovitch}
Generalize \ref{thm:besikovitch+} to noncontinuous metric tensor $g$ described the following way:
there are two Riemannian metric tensors $g_1$ and $g_2$ on $M$ and a subset $V\subset M$ bounded by a Lipschitz hypersurface $\Sigma$ such that 
$g=g_1$ at the points in $V$ and $g=g_2$ otherwise.
\end{subthm}



\begin{subthm}{ex:gadograph-gadograph}
Use part \ref{SHORT.ex:gadograph-besikovitch} to prove the following: 
Let $V$ be a compact set in the $n$-dimensional Euclidean space $\EE^n$ bounded by a Lipschitz hypersurface $\Sigma$.
Suppose $g$ is a Riemannian metric on $V$ such that 
\[\dist{p}{q}{g}\ge\dist{p}{q}{\EE^n}\]
for any two points $p,q\in \Sigma$.
Show that
\[\vol(V,g)\ge \vol(V)_{\EE^n}.\]
\end{subthm}

\end{thm}

\begin{thm}{Exercise}\label{ex:involution-of-sphere}
Suppose that sphere with Riemannian metric $(\mathbb{S}^2,g)$ admits an involution $\iota$ such that $\dist{x}{\iota(x)}{g}\ge 1$.

Show that 
\[\area(\mathbb{S}^2,g)\ge \tfrac1{1000}.\]
Try to show that 
\[\area(\mathbb{S}^2,g)\ge \tfrac12,
\quad \area(\mathbb{S}^2,g)\ge 1,
\quad\text{or}\quad\area(\mathbb{S}^2,g)\ge \tfrac4\pi\]

\end{thm}

\begin{thm}{Advanced exercise}\label{ex:involution-of-3sphere}
Construct a metric tensor $g$ on $\mathbb{S}^3$ such that (1) $\vol(\mathbb{S}^3,g)$ arbitrarily small and (2) there is an involution $\iota\:\mathbb{S}^3\z\to \mathbb{S}^3$ such that $\dist{x}{\iota(x)}{g}\ge 1$ for any $x\in \mathbb{S}^3$.
\end{thm}

\begin{thm}{Exercise}\label{ex:GH-vol}
Let $g_1,g_2,\dots$, and $g_\infty$ be metrics on a fixed compact manifold $M$.
Suppose that $\distfun_{g_n}$ uniformly converges to $\distfun_{g_\infty}$ as functions on $M\times M\to\RR$.
Show that 
\[\liminf_{n\to\infty}\vol(M,g_n)\ge \vol(M,g_\infty).\]

Show that the inequality might be strict.
\end{thm}

\section{Systolic inequality}

Let $\spc{M}$ be a compact Riemannian space.
The \index{systole}\emph{systole} of $\spc{M}$ (briefly $\sys\spc{M}$) is defined to be the least length of a noncontractible closed curve in $\spc{M}$.

Let $\Lambda$ be a class of closed $n$-dimensional Riemannian spaces.
We say that a \index{systolic inequality}\emph{systolic inequality} holds for $\Lambda$ if there is a constant $c$ such that 
\[\sys\spc{M}\le c\cdot \sqrt[n]{\vol\spc{M}}\]
for any $\spc{M}\in \Lambda$.

\begin{thm}{Exercise}\label{ex:sysT2}
Use \ref{thm:besikovitch} or \ref{ex:cylinder} to show that a systolic inequality holds for any Riemannian metric on the 2-torus $\TT^2$.
\end{thm}

\begin{thm}{Exercise}\label{ex:sysRP2}
Use \ref{thm:besikovitch} to show that a systolic inequality holds for any Riemannian metric on  the real projective plane $\RP^2$.
\end{thm}

\begin{thm}{Exercise}\label{ex:sysSg}
Use \ref{thm:besikovitch+} to show that systolic inequality holds for any Riemannian metric on any closed surfaces of positive genus.
\end{thm}

\begin{thm}{Exercise}\label{ex:sysS2xS1}
Show that no systolic inequality holds for Riemannian metrics on $\mathbb{S}^2\times\mathbb{S}^1$.
\end{thm}

In the following lecture we will show that systolic inequality holds for many manifolds, in particular for torus of arbitrary dimension.

\section{Generalization}\label{sec:hausdorff-measure}

The following proposition follows immediately from the definitions of Hausdorff measure (Section \ref{sec:vol-haus}).

\begin{thm}{Proposition}\label{prop:bilip-measure}
Let $\spc{X}$ and $\spc{Y}$ be metric spaces, $A\subset \spc{X}$
and
 $f\: \spc{X}\to \spc{Y}$ be a $\Lip$-Lipschitz map. 
Then 
\[\haus_\alpha [f(A)]\le \Lip^\alpha\cdot\haus_\alpha\, A\]
for any $\alpha$.
\end{thm}

The following exercise provides a weak analog of the Besicovitch inequality that works for arbitrary metrics.

\begin{thm}{Exercise}\label{ex:besikovitch++}
Let $M$ be manifold with boundary and $\rho$ is a semimetric on $M$.
Suppose $\partial M$ admits a degree 1 map to the surface of the $n$-dimensional cube $\square$;
denote by $d_1,\dots, d_n$ the $\rho$-distances between the inverse images of pairs of opposite faces of $\square$ in $M$.
Then 
\[\haus_n(M,\rho)\ge d_1\cdots d_n.\]
\end{thm}


Recall that in $n$-dimensional Riemannian spaces we have 
\[\tfrac{\omega_n}{2^n}\cdot\haus_n=\vol_n.\]
Note that $\tfrac{\omega_n}{2^n}<1$ if $n\ge 2$.
Therefore, the conclusion in \ref{ex:besikovitch++} is weaker than in \ref{thm:besikovitch+} (the assumptions are weaker as well).

One can redefine systolic inequality on $n$-dimensional manifolds using the Hausdorff measure $\haus_n$ instead of the volume.
It is straightforward to prove analogs of the exercises \ref{ex:sysT2}--\ref{ex:sysS2xS1} with this definition.

\begin{thm}{Exercise}\label{ex:2top-discs}
Suppose that two embedded $n$-disks $\Delta_1,\Delta_2$ in a metric space $\spc{X}$ have identical boundaries.
Assume that $\spc{X}$ is contractible and $\haus_{n+1}\spc{X}=0$.
Show that $\Delta_1=\Delta_2$.
\end{thm}

\section{Remarks}\label{sec:besicovitch-remarks}

The optimal constants in the systolic inequality are known only in the following three cases:
\begin{itemize}
\item For real projective plane $\RP^2$ the constant is $\sqrt{\pi/2}$ --- the equality holds for a quotient of a round sphere by isometric involution. The statement was proved by Pao Ming Pu \cite{pu}.\label{page:pu}
\item For torus $\TT^2$ the constant is $\sqrt{2}/\sqrt[4]{3}$ --- the equality holds for a flat torus obtained from a regular hexagon by identifying opposite sides; this is the so-called \index{Loewner's torus inequality}\emph{Loewner's torus inequality}.
\item For the Klein bottle $\RP^2\#\RP^2$  the constant is $\sqrt{\pi}/2^{3/4}$ --- the equality holds for a certain nonsmooth metric.
The statement was proved by Christophe Bavard \cite{bavard}.
\end{itemize}
The proofs of these results use the so-called {}\emph{uniformization theorem}   available in the 2-dimensional case only.
These proofs are beautiful, but they are too far from metric geometry.
A good survey on the subject is written by Christopher Croke and Mikhail Katz \cite{croke-katz}.

An analog of Exercise \ref{ex:GH-vol} with Hausdorff measure instead of volume does not hold for general metrics on a manifold.
In fact there is a nondecreasing sequence of metric tensors $g_n$ on $M$, such that (1) $\vol(M,g_n)<1$ for any $n$ and (2) $\distfun_{g_n}$ converges to a metric on $M$ with arbitrary large Hausdorff measure of any given dimension; such examples were constructed by Dmitri Burago, Sergei Ivanov, and David Shoenthal \cite{burago-ivanov-shoenthal}.

\chapter{Width and systole}

This lecture is based on the paper of Alexander Nabutovsky \cite{nabutovsky}.

\section{Partition of unity}

\begin{thm}{Proposition}\label{thm:part-unit}
 Let $\{V_i\}$ be a finite open covering of a compact metric space ${\spc{X}}$.
Then there are Lipschitz functions $\psi_i\:{\spc{X}}\z\to[0,1]$ such that (1) if $\psi_i(x)>0$, then $x\in V_i$ and (2) for any $x\in {\spc{X}}$ we have
$$\sum_i\psi_i(x)=1.$$

\end{thm}

A collection of functions $\{\psi_i\}$ that meets the conditions in \ref{thm:part-unit} is called 
a \index{partition of unity}\emph{partition of unity} subordinate to the covering $\{V_i\}$.

\parit{Proof.}
Denote by $\phi_i(x)$ the distance from $x$ to the complement of $V_i$;
that is,
$$\phi_i(x)=\distfun_{{\spc{X}}\setminus V_i}(x).$$
Note $\phi_i$ is $1$-Lipschitz
for any $i$
and $\phi_i(x)>0$ if and only if $x\in V_i$.
Since $\{V_i\}$ is a covering, we have that
$$\Phi(x)\df\sum_i\phi_i(x)>0\ \ \text{for any}\ \ x\in {\spc{X}}.$$
Since $\spc{X}$ is compact, $\Phi>\delta$ for some $\delta>0$.
It follows that $x\mapsto\tfrac1{\Phi(x)}$ is a bounded Lipschitz function. 

Set 
$$\psi_k(x)=\frac{\phi_k(x)}{\Phi(x)}.$$
Observe that by construction the functions $\psi_i$ meet the conditions in the proposition.
\qedsf

\section{Nerves}

Let $\{V_1,\dots,V_k\}$ be a finite open cover of a compact metric space $\spc{X}$.
Consider an abstract simplicial complex $\spc{N}$, with one vertex $v_i$ for each set $V_i$ such that a simplex with vertices $v_{i_1},\dots, v_{i_m}$ is included in $\spc{N}$ if 
the intersection $V_{i_1}\cap\dots\cap V_{i_m}$ is nonempty.
\begin{figure}[ht!]
\vskip-0mm
\centering
\includegraphics{mppics/pic-1402}
\end{figure}
The obtained simplicial complex $\spc{N}$ is called the \index{nerve}\emph{nerve} of the covering $\{V_i\}$.
Evidently $\spc{N}$ is a finite simplicial complex ---
it is a subcomplex of a simplex with the vertices $\{v_1,\dots,v_k\}$.

Note that the nerve $\spc{N}$ has dimension at most $n$ if and only if the covering $\{V_1,\dots,V_k\}$ has \index{multiplicity of covering}\emph{multiplicity} at most $n+1$;
that is, any point $x\in\spc{X}$ belongs to
at most $n+1$ sets of the covering.

Suppose $\{\psi_i\}$ is  
a partition of unity subordinate to the covering $\{V_1,\dots,V_k\}$.
Choose a point $x\in {\spc{X}}$.
Note that the set
$$\{v_{i_1},\dots,v_{i_n}\}=\set{v_i}{\psi_i(x)>0}$$
form vertices of a simplex in $\spc{N}$.
Therefore 
$$\bm{\psi}\:x\mapsto \psi_1(x)\cdot v_1+\psi_2(x)\cdot v_2+\dots+\psi_k(x)\cdot v_n.$$
describes a Lipschitz map from ${\spc{X}}$ to the nerve $\spc{N}$ of $\{V_i\}$.
In other words, $\bm{\psi}$ maps a point $x$ to the point in $\spc{N}$ with \index{barycentric coordinates}\emph{barycentric coordinates} $(\psi_1(x),\dots,\psi_k(x))$.

Recall that the \index{star}\emph{star} of a vertex $v_i$ (briefly $\Star_{v_i}$) is defined as the union of the interiors of all simplicies that have $v_i$ as a vertex.
Recall that $\psi_i(x)>0$ implies $x\in V_i$.
Therefore we get the following:

\begin{thm}{Proposition}\label{prop:space->nerve}
Let $\spc{N}$ be a nerve of an open covering $\{V_1,\z\dots,V_k\}$ of a compact metric space $\spc{X}$.
Denote by $v_i$ the vertex of $\spc{N}$ that corresponds to $V_i$.

Then there is a Lipschitz map $\bm{\psi}\:\spc{X}\to\spc{N}$ such that $\bm{\psi}(V_i)\z\subset\Star_{v_i}$ for every $i$.
\end{thm}


\section{Width}

Suppose $A$ is a subset of a metric space $\spc{X}$.
The radius of $A$ (briefly $\rad A$) is defined as the least upper bound on the values $R>0$ such that $\oBall(x,R)\supset A$ for some $x\in \spc{X}$.

\begin{thm}{Definition}\label{def:width}
Let $\spc{X}$ be a metric space.
The \index{width}\emph{$n$-th width} of $\spc{X}$ (briefly $\width_n\spc{X}$) is the least upper bound on values $R>0$ such that $\spc{X}$ admits a finite open covering $\{V_i\}$ with multiplicity at most $n+1$ and $\rad V_i< R$ for each $i$.
\end{thm}

\parbf{Remarks.}

\begin{itemize} 
\item Observe that 
\[\width_0\spc{X}\ge\width_1\spc{X}\ge\width_2\spc{X}\ge\dots\]
for any compact metric space $\spc{X}$.
Moreover, if $\spc{X}$ is connected, then 
\[\width_0\spc{X}=\rad\spc{X}.\]

\item Usually width is defined using diameter instead of radius, but the results differ at most twice.
Namely, if $r$ is the $n$-th radius-width and $d$ --- the $n$-th diameter-width, then 
$r\le d\le 2\cdot r$.

\item Note that \index{Lebesgue covering dimension}\emph{Lebesgue covering dimension} of $\spc{X}$ can be defined as the least number $n$ such that $\width_n\spc{X}=0$.

\item Another closely related notion is the so-called \index{macroscopic dimension}\emph{macroscopic dimension on scale $R$};
it is defined as the  least number $n$ such that $\width_n\spc{X}<R$.
\end{itemize}



\begin{thm}{Exercise}\label{ex:macrodimension}
Suppose $\spc{X}$ is a compact metric space such that any closed curve $\gamma$ in $\spc{X}$ can be contracted in its $R$-neighborhood.
Show that macroscopic dimension of $\spc{X}$ on scale $100\cdot R$ is at most 1.

What about quasiconverse? That is, suppose a simply connected compact metric space $\spc{X}$ has macroscopic dimension at most 1 on scale $R$, is it true that any closed curve $\gamma$ in $\spc{X}$ can be contracted in its $100\cdot R$-neighborhood?
\end{thm}


The following exercise gives a good reason for the choice of term \index{width}\emph{width}; it also can be used as an alternative definition.

\begin{thm}{Exercise}\label{ex:width=suprad(inv)}
Suppose $\spc{X}$ is a compact metric space.
Show that $\width_n\spc{X}<R$ if and only if there is a finite $n$-dimensional simplicial complex $\spc{N}$ and a continuous map $\bm{\psi}\:\spc{X}\to \spc{N}$
such that 
\[\rad[\bm{\psi}^{-1}(s)]<R\]
for any $s\in \spc{N}$.
\end{thm}

\section{Riemannian polyhedrons}

A \index{Riemannian!polyhedron}\emph{Riemannian polyhedron} is defined as a finite simplicial complex with a metric tensor on each simplex such that the restriction of the metric tensor to a subsimplex coincides with the metric on the subsimplex.

The {}\emph{dimension} of a Riemannian polyhedron is defined as the largest dimension in its triangulation.
For Riemannian polyhedrons one can define length of curves and volume the same way as for Riemannian manifolds.

The obtained metric space will be called \emph{Riemannian polyhedron} as well.
A \index{triangulation}\emph{triangulation} of Riemannian polyhedron  will always be assumed to have the above property on the metric tensor.

Further we will apply the notion of width only to compact Riemannian polyhedrons.
If $\spc{P}$ is an $n$-dimensional Riemannian polyhedron, then 
we suppose that
\[\width\spc{P}\df\width_{n-1}\spc{P}.\]


Suppose that $\spc{P}$ is an $n$-dimensional Riemannian polyhedron;
in this case we will use short cut $\vol$ for $\vol_n$.
Let us define \index{volume profile}\emph{volume profile} of $\spc{P}$ as a function 
returning largest volume of $r$-ball in~$\spc{P}$;
that is, the volume profile of $\spc{P}$ is a function $\VolPro_{\spc{P}}\:\RR_+\to\RR_+$ defined by 
\[\VolPro_{\spc{P}}(r)\df \sup\set{\vol \oBall(p,r)}{p\in\spc{P}}.\]
Note that 
$r\mapsto \VolPro_{\spc{P}}(r)$ is nondecreasing  and
\[\VolPro_{\spc{P}}(r)\le\vol\spc{P}\]
for any $r$.
Moreover, if $\spc{P}$ is connected, then the equality $\VolPro_{\spc{P}}(r)\z=\vol\spc{P}$ holds
for $r\ge \rad \spc{P}$.

Note that if $\spc{P}$ is a connected 1-dimensional Riemannian polyhedron, then 
\[\width\spc{P}=\width_0\spc{P}=\rad\spc{P}.\]

\begin{thm}{Exercise}\label{ex:1D-case}
Let $\spc{P}$ be a 1-dimensional Riemannian polyhedron.
Suppose that $\VolPro_{\spc{P}}(R)<R$ for some $R>0$.
Show that 
\[\width \spc{P}<R.\]
Try to show that $c=\tfrac 12$ is the optimal constant for which the following inequality holds: 
\[\width \spc{P}<c\cdot R.\]
\end{thm}

\section{Volume profile bounds width}

\begin{thm}{Theorem}\label{thm:width<volpro}
Let $\spc{P}$ be an $n$-dimensional Riemannian polyhedron. 
If the inequality 
\[R> n\cdot \sqrt[n]{\VolPro_{\spc{P}}(R)}\]
holds for some $R>0$, then 
\[\width\spc{P}\le  R.\]
\end{thm}

Since $\VolPro_{\spc{P}}(R)\le \vol\spc{P}$ for any $R>0$,
we get the following:

\begin{thm}{Corollary}\label{thm:width<vol}
For any $n$-dimensional Riemannian polyhedron $\spc{P}$, we have
\[\width\spc{P}\le n\cdot \sqrt[n]{\vol\spc{P}}.\]

\end{thm}

The proof of \ref{thm:width<volpro} will be given at the very end of this section,
after discussing {}\emph{separating polyhedrons}. 

Let us start three technical statements.
The first statement can be obtained by modifying a smoothing procedure for functions defined on Euclidean space. 

A function $f$ defined on a Riemannian polyhedron $\spc{P}$ is called \index{piecewise smooth}\emph{piecewise smooth} if there is a triangulation of $\spc{P}$ such that restriction of $f$ to every simplex is smooth.


\begin{thm}{Smoothing procedure}\label{smoothing-procedure}
Let $\spc{P}$ be a Riemannian polyhedron and $f\:\spc{P}\to \RR$ be a 1-Lipschitz function.
Then for any $\delta>0$ there is a piecewise smooth 1-Lipschitz function $\tilde f\:\spc{P}\to \RR$ such that 
\[|\tilde f(x)-f(x)|<\delta\]
for any $x\in  \spc{P}$.
\end{thm}

The following statement can be proved by applying the classical Sard's theorem to each simplex of a Riemannian polyhedron.

\begin{thm}{Sard's theorem}\label{sard}\index{Sard's theorem}
Let $\spc{P}$ be an $n$-dimensional Riemannian polyhedron and $f\:\spc{P}\to \RR$ be a piecewise smooth function.
Then for almost all values $a\in\RR$, the inverse image $f^{-1}\{a\}$  is a Riemannian polyhedron of dimension at most $n-1$ (we assume that $f^{-1}\{a\}$ is equipped with the induced length metric).
\end{thm}

The following statement can be proved by applying the coarea inequality (\ref{cor:coarea}) to the restriction of $f$ to each simplex of the polyhedron and summing up the results.

\begin{thm}{Coarea inequality}\index{coarea inequality}\label{poly-coarea}
Let $\spc{P}$ be an $n$-dimensional Riemannian polyhedron and $f\:\spc{P}\to \RR$ be a piecewise smooth 1-Lipschitz function.
Set $v\z=\vol_n (f^{-1}[r,R])$ and $a(t)=\vol_{n-1}(f^{-1}\{t\})$.
Then 
\[\int_r^Ra(t)\cdot dt\ge v .\]
In particular, there is a subset of positive measure $S\subset [r,R]$ such that the inequality 
\[a(t)\ge \frac v{R-r}\]
holds for any $t\in S$.
\end{thm}

\section*{Separating subpolyhedrons}

\begin{thm}{Definition}
Let $\spc{P}$ be an $n$-dimensional Riemannian polyhedron.
An $(n-1)$-dimensional subpolyhedron $\spc{Q}\subset\spc{P}$ is called \index{separating subpolyhedron}\emph{$R$-separating} if for each connected component $U$ of the complement $\spc{P}\setminus \spc{Q}$ we have 
\[\rad U<R.\]

\end{thm}



\begin{thm}{Lemma}\label{lem:separating}
Let $\spc{P}$ be an $n$-dimensional Riemannian polyhedron.
Then given $R>0$ and $\eps>0$ there is a $R$-separating subpolyhedron $\spc{Q}\subset\spc{P}$ such that for any $r_0<r_1\le R$ we have
\[\VolPro_{\spc{Q}}(r_0)< \tfrac1{r_1-r_0}\cdot \VolPro_{\spc{P}}(r_1)+\eps.\]

\end{thm}

The proof reminds the proof of the following statement about minimal surfaces: 
\textit{if a point $p$ lies on an compact area-minimizing surface $\Sigma$ and $\partial\Sigma \cap \oBall(p,r)=\emptyset$, then
\[\area(\Sigma\cap \oBall(p,r))\le \tfrac12\cdot \area\mathbb{S}^2\cdot r^2.\]
}


\parit{Proof.}
Choose a small $\delta>0$.
Applying the smoothing procedure (\ref{smoothing-procedure}), we can exchange each distance function $\distfun_p$ on $\spc{P}$ by $\delta$-close piecewise smooth 1-Lipschitz function, which will be denoted by $\widetilde \distfun_p$.

By Sard's theorem (\ref{sard}), for almost all values $c\z\in(r_0\z+\delta, r_1-\delta)$, the level set
\[\tilde S_c(p)=\set{x\in \spc{P}}{\widetilde \distfun_p(x)=c}\]
is a Riemannian polyhedron of dimension at most $n-1$.
Since $\delta$ is small, the coarea inequality (\ref{poly-coarea}) implies that $c$ can be chosen so that in addition the following inequality holds:
\begin{align*}
\vol_{n-1}\tilde S_c(p)&\le \tfrac1{r_1-r_0-2\cdot\delta}\cdot\vol_n[\oBall(p,r_1)]<
\\
&<\tfrac1{r_1-r_0}\cdot \VolPro_{\spc{P}}(r_1)+\tfrac\eps2.
\end{align*}

Suppose $\spc{Q}$ is an $R$-separating subpolyhedron in $\spc{P}$ with almost minimal volume;
say its volume is at most $\tfrac\eps2$-far from the greatest lower bound.
Note that cutting from $\spc{Q}$ everything inside $\tilde S_c(p)$ and adding $\tilde S_c(p)$ produces a $R$-separating subpolyhedron, say $\spc{Q}'$.%
\footnote{If $\dim\tilde S_c(p)<n-1$, then it might happen that $\dim\spc{Q}'<n-1$; so, by the definition, $\spc{Q}'$ is not separating.
It can be fixed by adding a tiny $(n-1)$-dimensional piece to $\spc{Q}'$.}

Since $\spc{Q}$ has almost minimal volume, we have
\[\vol_{n-1}[\spc{Q}\cap \oBall(p,r_0)_{\spc{P}}]-\tfrac\eps2\le \vol_{n-1}S_c(p).\]
Therefore 
\[\vol_{n-1}[\spc{Q}\cap \oBall(p,r_0)_{\spc{P}}]\le\tfrac1{r_1-r_0}\cdot \VolPro_{\spc{P}}(r_1)+\eps.
\eqlbl{eq:volQ<ProP}\]
Recall that $\spc{Q}$ is equipped with the induced length metric;
therefore $\dist{p}{q}{\spc{Q}}\ge \dist{p}{q}{\spc{P}}$ for any $p,q\in \spc{Q}$;
in particular, 
\[\oBall(p,r_0)_{\spc{Q}}\subset \spc{Q}\cap \oBall(p,r_0)_{\spc{P}}\]
for any $p\in \spc{Q}$ and $r_0\ge 0$.
Hence, \ref{eq:volQ<ProP} implies the lemma.
\qeds

\begin{thm}{Lemma}\label{lem:separating-width}
Let $\spc{Q}$ be an $R$-separating subpolyhedron in an $n$-dimensional Riemannian polyhedron $\spc{P}$.
Then 
\[\width\spc{Q}\le R
\quad\Longrightarrow\quad
\width\spc{P}\le R.\]
\end{thm}

\parit{Proof.}
Choose an open covering $\{V_1,\dots,V_k\}$ of $\spc{Q}$ as in the definition of width (\ref{def:width});
that is, it has multiplicity at most $n$ and $\rad V_i<R$ for any $i$. 

Note that $\{V_1,\dots,V_k\}$ can be converted into an open covering of
a small neighbourhood of $\spc{Q}$ in $\spc{P}$ without increasing the multiplicity.
This can be done by setting 
\[V_i'=\bigcup_{x\in V_i}\oBall(x,r_x),\]
where $r_x\df\tfrac1{10}\cdot\inf\set{\dist{x}{y}{}}{y\in \spc{Q}\setminus V_i}$.

By adding to  $\{V_i'\}$ all the components of $\spc{P}\setminus \spc{Q}$,
we increase the multiplicity by at most 1 and obtain a covering of $\spc{P}$.
The statement follows since $\dim \spc{P}= \dim \spc{Q}\z+1$.
\qeds

\section*{Proof assembling}

\parit{Proof of \ref{thm:width<volpro}.}
We apply induction on the dimension $n=\dim\spc{P}$.
The base case $n=1$ is given in \ref{ex:1D-case}.

Suppose that the  $(n-1)$-dimensional case is proved.
Consider an $n$-dimensional Riemannian polyhedron $\spc{P}$ and suppose
\[n\cdot \sqrt[n]{\VolPro\spc{P}(R)}< R\]
for some $R>0$.
Let $\spc{Q}$ be an $R$-separating subpolyhedron in $\spc{P}$ provided by \ref{lem:separating} for a small $\eps>0$.

Applying  \ref{lem:separating} for $r=\tfrac{n-1}n\cdot R$ and $R$, we have that 
\begin{align*}
\VolPro_\spc{Q}(r) &< \frac 1{R-r}\cdot \VolPro_\spc{P}(R)+\eps<
\\
&<\frac {n}{R}\cdot\left(\frac{R}{n}\right)^n=
\\
&=\left(\frac{r}{n-1}\right)^{n-1};
\end{align*}
that is, $(n-1)\cdot \sqrt[n-1]{\VolPro\spc{Q}(r)}< r$.
Since $\dim\spc{Q}= n-1$, by the induction hypothesis, we get that
\[\width\spc{Q}\le r<R.\]
It remains to apply \ref{lem:separating-width}.
\qeds





\section{Width bounds systole}

Recall that a topological space $K$ is called \index{aspherical space}\emph{aspherical} if any continuous map $\mathbb{S}^k\to K$ for $k\ge 2$ is null-homotopic.

\begin{thm}{Theorem}\label{thm:sys<width}
Suppose $\spc{M}$ is a compact aspherical $n$-dimensional Riemannian manifold.
Then 
\[\sys\spc{M}\le 6 \cdot \width \spc{M}.\]
\end{thm}

\begin{thm}{Lemma}\label{lem:aspherical-homotopy}
Let $K$ be an aspherical space and $\spc{W}$ a connected CW-complex.
Denote by $\spc{W}^k$ the k-skeleton of $\spc{W}$.
Then any continuous map $f\:\spc{W}^2\to K$ can be extended to a continuous map $\bar f\:\spc{W}\to K$

Moreover, if $p\in \spc{W}$ is a 0-cell and $q\in K$.
Then a continuous maps of pairs $\phi_0,\phi_1\:(\spc{W},p)\to(K,q)$ are homotopic if and only if $\phi_0$ and $\phi_1$ induce the same homomorphism on fundamental groups $\pi_1(\spc{W},p)\to\pi_1(K,q)$.
\end{thm}

\parit{Proof.}
Since $K$ is aspherical, any continuous map $\partial\mathbb{D}^n\to K$ for $n\ge 3$
is hull-homotopic;
that is, it can be extended to a map $\mathbb{D}^n\:\to K$.

It makes it possible to extend $f$ to $\spc{W}^3$, $\spc{W}^4$, and so on.
Therefore $f$ can be extended to whole $\spc{W}$.

The only-if part of the second part of lemma is trivial;
it remains to show the if part.

Sine $\spc{W}$ is connected, we can assume that $p$ forms the only 0-cell in $\spc{W}$;
otherwise, we can collapse a maximal subtree of the 1-skeleton in $\spc{W}$ to $p$.
Therefore, $\spc{W}^1$ is formed by loops that generate $\pi_1(\spc{W},p)$.

By assumption, the restrictions of $\phi_0$ and $\phi_1$ to $\spc{W}^1$ are homotopic.
In other words the homotopy $\Phi\:[0,1]\times \spc{W}$ is defined on the 2-skeleton of $[0,1]\times \spc{W}$.
It remains to apply the first part of the lemma to the product $[0,1]\times \spc{W}$.
\qeds



\begin{thm}{Lemma}\label{lem:sys-homotopy}
Suppose $\gamma_0,\gamma_1$ are two paths between points in a Riemannian space $\spc{M}$ such that $\dist{\gamma_0(t)}{\gamma_1(t)}{\spc{M}}<r$ for any $t\in[0,1]$.
Let $\alpha$ be a shortest path from $\gamma_0(0)$ to $\gamma_1(0)$ and $\beta$ be a shortest path from $\gamma_0(1)$ to $\gamma_1(1)$. 
If $2\cdot r<\sys\spc{M}$, then there is a homotopy $\gamma_t$ from
$\gamma_0$ to $\gamma_1$ such that $\alpha(t)\equiv \gamma_t(0)$ and $\beta(t)\equiv \gamma_t(1)$.
\end{thm}

\parit{Proof.}
Set $s=\sys\spc{M}$; 
since $2\cdot r<s$, we have that $\eps=\tfrac1{10}(s-2\cdot r)>0$.

\begin{wrapfigure}{o}{34mm}
\vskip-0mm
\centering
\includegraphics{mppics/pic-1405}
\end{wrapfigure}

Note that we can assume that $\gamma_0$ and $\gamma_1$ are rectifiable;
if not we can homotopy each into a broken geodesic line kipping the assumptions true. 

Choose a fine partition $0\z=t_0\z<t_1\z<\z\dots\z<t_n=1$.
Consider a sequence of shortest paths $\alpha_i$ from $\gamma_0(t_i)$ to $\gamma_1(t_i)$.
We can assume that $\alpha_0=\alpha$, $\alpha_n=\beta$, and each arc $\gamma_j|_{[t_{i-1},t_i]}$ has length smaller than $\eps$.
Therefore, every quadrilateral formed by concatenation  of $\alpha_{i-1}$, $\gamma_1|_{[t_{i-1},t_i]}$, reversed $\alpha_i$, and reversed arc $\gamma_0|_{[t_{i-1},t_i]}$ has length smaller than $s$.
It follows that this curve is contractible.
Applying this observation for each quadrilateral, we get the statement.
\qeds


\parit{Proof of \ref{thm:sys<width}.}
Let $\spc{N}$ be the nerve of a covering $\{V_i\}$ of $\spc{M}$ and $\bm{\psi}\:\spc{M}\to\spc{N}$ be the map provided by \ref{prop:space->nerve}.
As usual, we denote by $v_i$ the vertex of $\spc{N}$ that corresponds to $V_i$.
Observe that $\dim\spc{N}<n$;
therefore, $\bm{\psi}$ kills the fundamental class of $\spc{M}$.

Let us construct a continuous map  $f\:\spc{N}\to  \spc{M}$ such that
$f\circ\bm{\psi}$ is homotopic to the identity map on $\spc{M}$.
Note that once $f$ is constructed, the theorem is proved.
Indeed, since $\bm{\psi}$ kills the fundamental class $[\spc{M}]$ of $\spc{M}$, so does $f\circ\bm{\psi}$.
Therefore, $[\spc{M}]=0$ --- a contradiction.

Set $R=\width \spc{M}$ and $s=\sys\spc{M}$.
Assume we choose $\{V_i\}$ as in the definition of width (\ref{def:width}).
For each $i$ choose a point $p_i\in \spc{M}$ such that $V_i\subset \oBall(p_i,R)$.

Set $f(v_i)=p_i$ for each $i$.
It defines the map $f$ on the 0-skeleton $\spc{N}^0$ of the nerve $\spc{N}$.
Further, $f$ will be defined step by step on the skeletons $\spc{N}^1,\spc{N}^2, \dots$ of $\spc{N}$.

Let us map each edge $[v_iv_j]$ in $\spc{N}$ to a shortest path $[p_ip_j]$.
It defines $f$ on $\spc{N}^1$.
Note that image of each edge is shorter than $2\cdot R$.

Suppose $[v_iv_jv_k]$ is a triangle in $\spc{N}$.
Note that perimeter of the triangle $[p_ip_jp_k]$ can not exceed $6\cdot R$.
Since $6\cdot R<s$, the contour of $[p_ip_jp_k]$ is contractible.
Therefore, we can extend $f$ to each triangle of~$\spc{N}$.
It defines the map $f$ on $\spc{N}^2$.

Finally, since $\spc{M}$ is aspherical, by \ref{lem:aspherical-homotopy}, the map $f$ can be extended to $\spc{N}^3$, $\spc{N}^4$ and so on.

It remains to show that $f\circ\bm{\psi}$ is homotopic to the identity map.
Choose a CW structure on $\spc{M}$ with sufficiently small cells, so that each cell lies in one of $V_i$.
Note that $\bm{\psi}$ is homotopic to a map $\bm{\psi}_1$ that sends $\spc{M}^k$ to $\spc{N}^k$ for any $k$.
Moreover, we may assume that (1) if a 0-cell $x$ of $\spc{M}$ maps to a $v_i$, then $x\in V_i$ and (2) each 1-cell  of $\spc{M}$ maps to an edge or a vertex of $\spc{N}$.
Choose a 1-cell $e$ in $\spc{M}$; by the construction, $f\circ\bm{\psi}_1$ maps $e$ to a shortest path $[p_ip_j]$ and $e$ lies $\oBall(p_i,R)$.
Observe that $[p_ip_j]$ is shorter than $2\cdot R$.
It follows that the distance between points on $[p_ip_j]$ and $e$ can not exceed $3\cdot R$.
Choose a shortest path $\alpha_i$ from every 0 cell $x_i$  of $\spc{M}$ to $p_j=f\circ\bm{\psi}_1(x_i)$.
It defines a homotopy on $\spc{M}^0$.
Since $6\cdot R<s$, \ref{lem:sys-homotopy} implies that this homotopy can be extended to $\spc{M}^1$.
By \ref{lem:aspherical-homotopy}, it can be extended to whole $\spc{M}$.
\qeds

\begin{thm}{Exercise}\label{ex:sys<width}
Analyze the proof of \ref{thm:sys<width} and improve its inequality to 
 \[\sys\spc{M}\le 4 \cdot \width \spc{M}.\]
\end{thm}

\begin{thm}{Exercise}\label{ex:fillrad-inj}
Modify the proof of \ref{thm:sys<width} to prove the following:

Suppose that $\spc{M}$ is a closed $n$-dimensional Riemannian manifold with \index{injectivity radius}\emph{injectivity radius} at least $r$; that is, if $\dist{p}{q}{\spc{M}}<r$, then a shortest path $[pq]_{\spc{M}}$ is uniquely defined.
Show that
\[\width\spc{M}\ge \tfrac{r}{2\cdot(n+1)}.\]

Use \ref{thm:width<vol} to conclude that $\vol\spc{M}\ge \eps_n \cdot r^n$
for some $\eps_n>0$ that depends only on $n$.
\end{thm} 

The second statement in the exercise is a theorem of Marcel Berger~\cite{berger-n};
an inequality with optimal constant (with equality for round sphere) was obtained by Marcel Berger and Jerry Kazdan \cite{berger-kazdan}. 


\section{Essential manifolds}

To generalize \ref{thm:sys<width} further, we need the following definition.

\begin{thm}{Definition}\label{def:essential}
A closed manifold $M$ is called \index{essential manifold}\emph{essential} if it admits a continuous map $\iota\:M\to K$ to an aspherical CW-complex $K$ such that $\iota$ sends the fundamental class of $M$ to a nonzero homology class in $K$.
\end{thm}

Note that any closed aspherical manifold is essential --- in this case one can take $\iota$ to be the identity map on $M$.

The real projective space $\RP^n$ provides an interesting example of an essential manifold which is not aspherical.
Indeed, the infinite dimensional projective space $\RP^\infty$ is aspherical and for the natural embedding $\RP^n\hookrightarrow\RP^\infty$ the image $\RP^n$ does not bound in $\RP^\infty$.
The following exercise provides more examples of that type:

\begin{thm}{Exercise}\label{ex:connected-sum-essential}
Show that the connected sum of an essential manifold with any closed manifold is essential.
\end{thm}

\begin{thm}{Exercise}\label{ex:product-essential}
Show that the product of two essential manifolds is essential.
\end{thm}

Assume that the manifold $M$ is essential and $\iota \:M\to K$ as in the definition.
Following the proof of \ref{thm:sys<width}, we can homotope the map 
$f\circ\bm{\psi}\:M\to M$ to the identity on the 2-skeleton of $M$;
further since $K$ is aspherical, we can homotope the composition
$\iota\z\circ f\circ\bm{\psi}$ to  $\iota$. 
Existence of this extension implies that $\iota$ kills the fundamental class of $M$ --- a contradiction.
So, taking \ref{ex:sys<width} into account, we proved the following generalization of \ref{thm:sys<width}:

\begin{thm}{Theorem}\label{thm:sys<width++}
Suppose $\spc{M}$ is an essential Riemannian space.
Then 
\[\sys\spc{M}\le 4 \cdot \width \spc{M}.\]
\end{thm}

As a corollary from \ref{thm:sys<width++} and \ref{thm:width<vol} we get the so-called  Gromov's \index{systolic inequality}\emph{systolic inequality}:

\begin{thm}{Theorem}\label{thm:sys+}
Suppose $\spc{M}$ is an essential $n$-dimensional Riemannian space.
Then 
\[\sys\spc{M}\le 4 \cdot n\cdot \sqrt[n]{\vol\spc{M}}.\]
\end{thm}


\section{Remarks}

Theorem \ref{thm:sys+} was proved originally by Mikhael Gromov \cite{gromov-1983} with a worse constant.
The given proof is a result of a sequence of simplifications given by Larry Guth \cite{guth},
Panos Papasoglu \cite{papasoglu},
Alexander Nabutovsky and Roman Karasev \cite{nabutovsky}.

The calculations could be done better; namely we could get
\[\width\spc{P}\le c_n\cdot \sqrt[n]{\vol\spc{P}},\]
where
$c_n=\sqrt[n]{n!/2}= \tfrac ne+o(n)$ \cite{nabutovsky}.
As a result, we may get a stronger statement in \ref{thm:sys+}:
\[\sys\spc{M}\le 4 \cdot c_n\cdot \sqrt[n]{\vol\spc{M}}.\]

For any nonessential oriented manifold $M$ there is a metric with fixed volume and arbitrary small systole.
This statement is proved by Ivan Babenko \cite{babenko}.

A wide open conjecture says that for any $n$-dimensional essential manifold we have
\[\frac{\sys\spc{M}}{\sqrt[n]{\vol\spc{M}}}\le\frac{\sys\RP^n}{\sqrt[n]{\vol\RP^n}},\eqlbl{eq:RPn}\]
where we assume that the $n$-dimensional real projective space $\RP^n$ is equipped with a canonical metric.
In other words, the ratio in the right-hand side of \ref{eq:RPn} is the optimal constant in the Gromov's systolic inequality; this  ratio grows as $O(\sqrt n)$.
(The ratio for $n$-dimensional flat torus grows as $O(\sqrt n)$ as well.)

%\chapter{Volume bounds filling radius}

This chapter 
is devoted to a proof of \ref{thm:FillRad<vol};
that is, we will show that \emph{Riemannian manifolds with small volume have small filling radius}.
This theorem was proved originally by Mikhael Gromov \cite{gromov-1983}.
We follow closely a simplified proof given by Alexander Nabutovsky, which is based on a sequence of other simplifications and improvements; see \cite{nabutovsky} and the references there in.

\section{Nerves and partition of unity}

Let $\{V_1,\dots,V_k\}$ be a finite open cover of a compact metric space $\spc{X}$.
Consider the abstract simplicial complex $\spc{N}$, with one vertex $v_i$ for each set $V_i$ such that a simplex with vertexes $v_{i_1},\dots, v_{i_k}$ is included in $\spc{N}$ if 
the intersection $V_{i_1}\cap\dots\cap V_{i_m}$ is nonempty.
We obtain a simplicial complex $\spc{N}$ called the \index{nerve}\emph{nerve of the covering $\{V_i\}$}.

Note that $\spc{N}$ is a finite simplicial complex and it has dimension at most $n$ if and only if the covering $\{V_1,\dots,V_k\}$ has multiplicity is at most $n+1$;
that is, at most $n+1$ different sets $V_{i_1},\dots, V_{i_{n+1}}$ have a nonempty intersection.
The nerve $\spc{N}$ is a subcomplex of a simplex with the vertixes $v_1,\dots,v_k\}$.

\begin{thm}{Proposition}\label{thm:part-unit}
 Let $\{V_1,\dots,V_k\}$ is a finite open covering of a compact metric space ${\spc{X}}$.
Then there are Lipschitz functions $\psi_i\:{\spc{X}}\to[0,1]$ such that
if $\psi_i(x)>0$ then $x\in V_i$ and
$$\sum_i\psi_i(x)=1$$
for any $x\in {\spc{X}}$.
\end{thm}

A collection of functions $\psi_i$ with above properies is called 
a \emph{partition of unity subordinate to the open cover}\index{partition of unity} $\{V_1,\dots,V_k\}$.

\parit{Proof.}
Consider the functions $\phi_i\:{\spc{X}}\to\RR$ defined as
$$\phi_i(x)=\distfun_{{\spc{X}}\backslash V_i} x.$$
Note $\phi_i$ is $1$-Lipschitz
for any $i$
and $\phi_i(x)>0$ if and only if $x\in V_i$.
In particular, 
$$\sum_i\phi_i(x)>0\ \ \text{for any}\ \ x\in {\spc{X}}.$$

Set 
$$\psi_k(x)=\frac{\phi_k(x)}{\sum_i\phi_i(x)}.$$
It remains to note that by construction the functions $\psi_i$ meet the conditions in the proposition.
\qedsf


Note that in the above proof for any point $x\in {\spc{X}}$,
the set
$$\set{v_i}{\psi_i(x)>0}$$
describe vertexes of a simplices in the nerve.
Therefore 
$$\bm{\psi}\:x\mapsto \psi_1(x)\cdot v_1+\psi_2(x)\cdot v_2+\dots+\psi_k(x)\cdot v_n.$$
can be thought of as a Lipschitz map from ${\spc{X}}$ to the nerve $\spc{N}$ of $\{V_i\}$;
where the point $x$ is mapped to the point with barycentric coordinates $\psi_i(x)$.
In other words we proved the following:

\begin{thm}{Proposition}\label{prop:space->nerve}
Let $\spc{N}$ be a nerve of an open covering $\{V_1,\dots,V_k\}$ of a compact metric space $\spc{X}$.
Denote by $v_i$ the vertex of $\spc{N}$ that corresponds to $V_i$.

Then there is a Lipschitz map from $\bm{\psi}\:\spc{X}\to\spc{N}$ such that $\bm{\psi}(V_i)\z\subset\Star_{v_i}$ for every $i$.
\end{thm}


\section{Width}

Suppose $A$ is a subset of a metric space $\spc{X}$.
The radius of $A$ (briefly $\rad A$) is defined as the least upper bound on the values $R>0$ such that $\oBall(x,R)\supset A$ for some $x\in \spc{X}$.

\begin{thm}{Definition}\label{def:width}
Let $\spc{X}$ be a metric space.
The $n$-width of $\spc{X}$ (briefly $\width_n\spc{X}$) is defined as least upper bound on values $R>0$ such that $\spc{X}$ admits a finite open covering $\{V_i\}$ with multiplicity at most $n+1$ and $\rad V_i< R$ for any $i$.
\end{thm}

\parit{Remarks.}
\begin{itemize}
\item Observe that if $\spc{X}$ is connected, then 
\[\width_0\spc{X}=\rad\spc{X}.\]
\item 
Usually width is defined using diameter instead of radius, but the result differ at most twice.
Namely if $r$ is the radius width and $d$ --- the diameter width of the same dimension, then 
$r\le d\le 2\cdot r$.

\item The definition of width reminds the definition of Lebesgue covering dimension.
In fact one says that a space has \emph{macroscopic dimesion} $\le n$ on the space $R$ if it admits an open cover as in the definiton.
\end{itemize}

\begin{thm}{Exercise}
Suppose $\spc{X}$ be a simply connected metric space such that any closed curve $\gamma$ in $\spc{X}$ can be contracted in its $R$-neighborhood.
Show that $\spc{X}$ is has macroscopic dimension at most 1 on scale $100\cdot R$.

Proove a quasiconverse; that is, if a simply connected metric space $\spc{X}$ has macroscopic dimension at most 1 on scale $R$, then any closed curve $\gamma$ in $\spc{X}$ can be contracted in its $100\cdot R$-neighborhood.
\end{thm}


The following proposition provides an equivalent definition.

\begin{thm}{Proposition}\label{prop:width=suprad(inv)}
Suppose $\spc{X}$ is a compact metric space.
Then $\width_n\spc{X}<R$ if and only if there is a finite $n$-dimensional somplicial complex $\spc{S}$ and a continuous map $\bm{\psi}\:\spc{X}\to \spc{N}$
such that $\rad[\bm{\psi}^{-1}(s)]\z<R$
for any $s\in \spc{N}$.
\end{thm}

\parit{Proof; ``only if'' part.}
Suppose $\width_n\spc{X}<R$.
Consider a covering $\{V_1,\dots,V_k\}$ of $\spc{X}$ guaranteed by the definition of width.
Let $\spc{N}$ be its nerve and $\bm{\psi}\:\spc{X}\to \spc{N}$ be the map provided by \ref{prop:space->nerve}.

Note that if $x\in \spc{N}$ lies in a symplex with a vertex $v_i$,
then $\bm{\psi}^{-1}(x)\subset V_i$;
in particulr $\bm{\psi}^{-1}(x)$ can be covered by a ball of radius $R$ in $\spc{X}$.

\parit{``If'' part.}
Choose $x\in \spc{N}$.
Since the inverse image $\bm{\psi}^{-1}(x)$ is compact, $\bm{\psi}$ is continuous, and $\rad[\bm{\psi}^{-1}(x)]<R$,
here is a neighborhood $U\ni x$ such that the  $\rad[\bm{\psi}^{-1}(U)]<R$.

It follows that there is a finite cover $\{U_i\}$ of $\spc{N}$ such that $\bm{\psi}^{-1}(U_i)\subset\spc{X}$ has radius smaller than $R$ for each $i$.
Since $\spc{N}$ has dimension $n$, we can inscribe%
\footnote{Recall that a covering $\{W_i\}$ is inscribed in the covering $\{U_i\}$ if for every $W_i$ is a subset of some $U_j$.} 
in $\{U_i\}$ an finite open cover $\{W_i\}$ with multiplicity at most $n+1$.
It remains to observe that $V_i=\bm{\psi}(W_i)$ defines a finite open cover of $\spc{X}$ with radius less than $R$ and multiplicity at most $n+1$. 
\qeds

Further we will apply the notion of width to compact Riemannian polyhedrons;
If $n$ is the dimension of a compact Riemannian polyhedron $\spc{P}$, then 
we suppose that
\[\width\spc{P}\df\width_{n-1}\spc{P}.\]

\begin{thm}{Exercise}
Show that for any closed Riemannian manifold $\spc{M}$ we have
\[\FillRad \spc{M}\le 100\cdot \width\spc{M};\]
try to show that in fact
\[\FillRad \spc{M}\le \width\spc{M}.\]

\end{thm}




\section{Volume bounds width}

A \emph{Riemannian polyhedron} is defined as a finite connected simplicial complex with a metric tensor on each simplex such that the restriction of the metric on each simplex to a subsymplex coinsides with the metric on the subsmplex.
The dimension of Riemannian polyhedron is defined as the largest dimension it its triangulation.
For Riemannian polhedron one can define length of curves and volume the same way as for Riemannian manifolds.

Let $\spc{P}$ be a Riemnnian polyhedron of dimension $n$.
Let us define volume profile of $\spc{P}$ as a function $\VolPro_{\spc{P}}\:\RR_+\to\RR_+$ defined by 
\[\VolPro_{\spc{P}}(r)\df \sup\set{\vol \oBall(p,r)}{p\in\spc{P}}.\]
Note that $\VolPro_{\spc{P}}$ is a nondecreasing function and $\VolPro_{\spc{P}}(r)\z\to\vol\spc{P}$ as $r\to\infty$.

\begin{thm}{Exercise}
Suppose $\spc{M}$ be a 1-dimensional Riemannian polhedron.
Suppose $\VolPro_{\spc{P}}(r_0)<r_0$ for some $r_0>0$.
Show that 
\[\diam \spc{P}<r_0.\]
Note tha it is equivalent to $\width \spc{P}<r_0$.
\end{thm}


An $(n-1)$-dimensional subpolyhedron $\spc{Q}\subset\spc{P}$ is called $R$-separating if each
connected component of its complement $\spc{P}\backslash \spc{Q}$ can be covered by a metric ball of radius $R$.

\begin{thm}{Lemma}
Let $\spc{P}$ be an $n$-dimensional Riemannian polyhedron.
Then given $R>0$ and $\eps>0$ there is a $R$-separating subpolyhedron $\spc{Q}\subset\spc{P}$ such that for any $r_0<r_1\le R$ we have
\[\VolPro_{\spc{Q}}(r_0)< \tfrac1{r_1-r_0}\cdot \VolPro_{\spc{P}}(r_1)+\eps.\]

\end{thm}

\begin{thm}{Lemma}
Let $\spc{Q}$ be a $R$-separating subpolyhedron in an $n$-dimensional Riemannian polyhedron $\spc{P}$.
Suppose $\width\spc{Q}\le R$.
Then $\width\spc{P}\le R$
\end{thm}

\parit{Proof.}
Start with an open covering $\{V_1,\dots,V_k\}$ of $\spc{Q}$ of multiplicity $\le n$ with radiuses of the sets in the intrinsic metric $\le R$.
Convert $\{V_1,\dots,V_k\}$ into an an open covering of
a small neighbourhood of $\spc{Q}$ in $\spc{P}$ without increasing the multiplicity.
Finally, add all the components of $\spc{P}\backslash \spc{Q}$ to the covering;
it increases the multiplicity by 1.
\qeds

The following technical statement will be used without a proof.

\begin{thm}{Claim}
Let $\spc{P}$ be a Reimannian polyhedron and $f\:\spc{P}\to \RR$ be a 1-Lipschitz function.
Then for any $\eps>0$ there is a  1-Lipschitz function $f_\eps\:\spc{P}\to \RR$ that is smooth on each simplex of the triangulation and $\eps$-close to $f$.
\end{thm}

\begin{thm}{Sard's theorem}
Let $\spc{P}$ be an $n$-dimensional Reimannian polyhedron and $f\:\spc{P}\to \RR$ be a function that is smooth on each simplex.
Then for almost all values $a$ each component of the inverse image $f^{-1}(a)$ is a equipped with the induced metric is a Reimannian polyhedron.
\end{thm}


\begin{thm}{Coarea inequality}
Let $\spc{P}$ be an $n$-dimensional Reimannian polyhedron and $f\:\spc{P}\to \RR$ be a 1-Lipschitz function that is smooth on each simplex.
Then 
\[\vol_n (f^{-1}[a,b]) \le \int_a^b\vol_{n-1}[f^{-1}(x)]\cdot dx.\]
\end{thm}


%\chapter{Examples}



\section{On semicontinuity}

Recall that according to \ref{ex:GH-vol}, volume is semicontinuos on the space of Riemannian manifolds with respect to stable Gromov--Hausdorff convergence.
Analogous statement for $n$-dimensional Hausdorff measure on a $n$-dimensional manifolds does not hold.

\begin{thm}{Claim}
 
\end{thm}

First let us show that for any $\alpha>0$, the $\alpha$-dimensional Hausdorff measure is not semicontinuous in the space of all compact metric spaces.

Choose a decreasing sequence $\eps_n\to 0$.
Consider the space $\spc{C}$ of infinite binary sequences with distance between two sequences $\bm{a}=(a_0,a_1,\dots)$ and $\bm{b}=(b_0,b_1,\dots)$ defined by 
\[\dist{\bm{a}}{\bm{b}}{\spc{C}}=\eps_n,\]
where $n$ is the minimal index such that $a_n\ne b_n$.
Note that $\spc{C}$ is homeomorphic to the Cantor set and 
given $\alpha>0$,
the sequence $\eps_n$ can be chosen so that its $\alpha$-dimensional Hausdorff measure is infinite.

Note that $\spc{C}$ is a Hausdorff limit of its subsets $\spc{C}_n$ formed by sequences that constantly zero starting from $n$-th element.
The sets $\spc{C}_n$ is finite in particular its $\alpha$-dimensional Hausdorff measure vanish for $\alpha>0$.
This example shows that for any $\alpha>0$, the $\alpha$-dimensional Hausdorff measure is not semicontinuous in the space of all compact metric spaces.

An analogous example can be produced comapct length spaces.
To do this consider a metric binary rooted tree $\spc{T}$ in which edges connecting level $n-1$ to the level $n$ of length $\eps_{n-1}-\eps_n$.
Note that the completion $\bar{\spc{T}}$ of $\spc{T}$ has a subset (its crown) isometric to $\spc{C}$.
Note further that $\bar{\spc{T}}$ is a Hausdorff limit of its subsets $\spc{T}_n$ --- the subtrees up to level $n$.
Note that $\spc{T}_n$ is can be covered by a finite line segments, in particular it has finite $1$-dimensional Hausdorff measure and therefore vanishing $\alpha$-dimensional Hausdorff for any $\alpha>1$.
Since the limit $\bar{\spc{T}}$ contains $\spc{C}$, we can choose a sequence $\eps_n$ so that $\mu_\alpha\spc{C}$ is arbitrary large (or even infinite).
It shows that for any $\alpha>1$, the $\alpha$-dimensional Hausdorff measure is not semicontinuous in the space of all compact length spaces.

This construction can be modified further to obtain an increasing sequence of metric tensors $g_n$ on a disc $\DD$ such that (1) $\vol(\DD,g_n)<1$ for each $n$, (2) the induced metrics $\dist{*}{*}{g_n}$ converge to a metric $\rho$ on $\DD$, and given any Cantor space $\spc{C}$ as described above (3) there is a bilipschitz map $\spc{C}\to(\DD,\rho)$.
Note that the last condition implies that $\mu_2(\DD,rho)$ can be made arbitrary large, or infinite.
Therefore for any $\alpha\ge 2$, the $\alpha$-dimensional Hausdorff measure is not semicontinuous in the space of all compact length spaces homeomorphic to a manifold and equipped with stable convergence.

Now we want to extend nonsemicontinuity even further.
Note that the tree $\bar{\spc{T}}$ admits a length-preserving embedding to the Euclidean space; we may assume that all 



\section{Sub-Riemannian metrics}

Choose a metric space $\spc{X}$.
Note that the function $\alpha\mapsto \mu_\alpha(A)_\spc{X}$ is nondecreasing;
moreover there is a critical value $\alpha_0\in[0,\infty]$ such that $\mu_\alpha(A)_\spc{X}=0$ if $\alpha<\alpha_0$ and $\mu_\alpha(A)_\spc{X}=\infty$ if $\alpha>\alpha_0$.
This value is called \index{Hausdorff dimension}\emph{Hausdorff dimension} of $\spc{X}$, or briefly $\alpha_0=\dim_H\spc{X}$.

The following statement is classical, a proof can be found in .

\begin{thm}{Theorem}
The Hausdorff dimension of any metric space can not be smaller than its Lebesgue covering dimension.
In particular, if a metric space $\spc{X}$ is homeomorphic to an $n$-dimensional manifold, then $\dim_H\spc{X}\ge n$.
 
\end{thm}

Note that the construction described in the previous section can be used to produce a metric on manifold of dimension $n\ge 2$ with arbitrary Hausdorff dimension $\alpha\ge n$.

In this section we will discuss another interesting source of such examples.



%
\begin{thm}{Uniqueness of geodesics}\label{thm:cat-unique}
In a proper length $\CAT(0)$ space, pairs of points are joined by unique geodesics, and these geodesics depend continuously on their endpoint pairs.

Analogously, in a proper length $\CAT(1)$ space, pairs of points at distance less than $\pi$ are joined by unique geodesics, and these geodesics depend continuously on their endpoint pairs.
\end{thm}

\parit{Proof.} 
Given 4 points $p^1,p^2,q^1,q^2$ in a proper length $\CAT(0)$ space $\spc{U}$, 
consider two triangles $\trig{p^1}{q^1}{p^2}$ and $\trig{p^2}{q^2}{q^1}$.
Since both of these triangles are thin, we get 
\begin{align*}
\dist{\geodpath_{[p^1q^1]}(t)}{\geodpath_{[p^2q^1]}(t)}{\spc{U}}
&\le (1-t)\cdot \dist{p^1}{p^2}{\spc{U}},
\\
\dist{\geodpath_{[p^2q^1]}(t)}{\geodpath_{[p^2q^2]}(t)}{\spc{U}}
&\le t\cdot \dist{q^1}{q^2}{\spc{U}}.
\intertext{By the triangle inequality,}
\dist{\geodpath_{[p^1q^1]}(t)}{\geodpath_{[p^2q^2]}(t)}{\spc{U}}&\le \max\{\dist{p^1}{p^2}{\spc{U}},\dist{q^1}{q^2}{\spc{U}}\}.
\end{align*}

This implies continuity and uniqueness in the $\CAT(0)$ case.  
 
The $\CAT(1)$ case is done in essentially the same way.
\qeds

Adding the first two inequalities of the preceding proof gives the following:

\begin{thm}{Proposition}
Suppose $p^1,p^2,q^1,q^2$ are points in a proper length $\CAT(0)$ space~$\spc{U}$.
Then 
\[\dist{\geodpath_{[p^1q^1]}(t)}{\geodpath_{[p^2q^2]}(t)}{\spc{U}}\]
is a convex function.
\end{thm}

\begin{thm}{Corollary}\label{cor:dist-convex}
Let $K$ be a closed convex subset in a proper length $\CAT(0)$ space~$\spc{U}$.
Then $\dist{K}{}{}\:\spc{U}\to\RR$ is \index{convex function}\emph{convex};
that is, the function $t\mapsto\dist{K}{}{}\circ\gamma$ is convex for any geodesic $\gamma$ in $\spc{U}$.

In particular, $\dist{p}{}{}$ is convex for any point $p$ in~$\spc{U}$.
\end{thm}


\begin{thm}{Corollary}\label{cor:contractible-cat}
Any proper length $\CAT(0)$ space is contractible.

Analogously, any proper length $\CAT(1)$ space with diameter $<\pi$ is contractible.
\end{thm}

\parit{Proof.} Let $\spc{U}$ be a proper length $\CAT(0)$ space.
Fix a point $p\in \spc{U}$.

For each point $x$ consider the geodesic path $\gamma_x\:[0,1]\to \spc{U}$ from $p$ to~$x$.
Consider the one parameter family of maps 
$h_t\:x\mapsto \gamma_x(t)$ for $t\in [0,1]$.
By uniqueness of geodesics (\ref{thm:cat-unique}), the map 
$(t,x)\mapsto h_t(x)$ is continuous. The family $h_t$ is called a \index{geodesic homotopy}\emph{geodesic homotopy}.

It remains to note that $h_1(x)=x$ and $h_0(x)=p$ for any~$x$.

The proof of the $\CAT(1)$ case is identical.
\qeds

\begin{thm}{Proposition}\label{cor:loc-geod-are-min}
Suppose $\spc{U}$ is a proper length $\CAT(0)$ space.  
Then any local geodesic in $\spc{U}$ is a geodesic.

Analogously, if $\spc{U}$ is a proper length $\CAT(1)$ space, then any local geodesic in $\spc{U}$ which is shorter than $\pi$ is a geodesic.
\end{thm}

\begin{wrapfigure}{r}{21mm}
\begin{lpic}[t(-0mm),b(0mm),r(0mm),l(0mm)]{pics/local-geod(1)}
\lbl[t]{2.5,1;$\gamma(0)$}
\lbl[b]{10,14;$\gamma(a)$}
\lbl[t]{19,8;$\gamma(b)$}
\end{lpic}
\end{wrapfigure}

\parit{Proof.}
Suppose $\gamma\:[0,\ell]\to\spc{U}$ is a local geodesic that is not a geodesic.
Choose $a$ to be the maximal value 
such that $\gamma$ is a geodesic on $[0,a]$.
Further choose $b>a$ so that $\gamma$ is a geodesic on $[a,b]$.

Since the triangle $\trig{\gamma(0)}{\gamma(a)}{\gamma(b)}$ is thin and 
$\dist{\gamma(0)}{\gamma(b)}{}<b$ we have
\[\dist{\gamma(a-\eps)}{\gamma(a+\eps)}{}<2\cdot\eps\]
for all small~$\eps>0$.
That is, $\gamma$ is not length-minimizing on the interval $[a-\eps,a+\eps]$ for any $\eps>0$,
a contradiction.

The spherical case is done in the same way.
\qeds


\begin{thm}{Exercise}\label{ex:geod-CBA}
Assume $\spc{U}$ is a proper length $\CAT(\kappa)$ space
 with extendable geodesics;
that is, any geodesic is an arc in a local geodesic $\RR\to \spc{U}$.

Show that the space of geodesic directions at any point in $\spc{U}$ is complete.

Does the statement remain true if $\spc{U}$ is complete, but not required to be proper?
\end{thm}

Now let us formulate the main result of this section.


\begin{wrapfigure}[6]{r}{28mm}
\begin{lpic}[t(-4mm),b(6mm),r(0mm),l(0mm)]{pics/lem_alex1(1)}
\lbl[lb]{10,23;$y$}
\lbl[rt]{1.5,.5;$p$}
\lbl[bl]{25,7.5;$x$}
\lbl[lb]{17,15;$z$}
\end{lpic}
\end{wrapfigure}

\begin{thm}{Inheritance lemma}
\label{lem:inherit-angle} 
Assume that a triangle $\trig p x y$ 
in a metric space is \index{decomposed triangle}\emph{decomposed} 
into two triangles $\trig p x z$ and $\trig p y z$;
that is, $\trig p x z$ and $\trig p y z$ have a common side $[p z]$, and the sides $[x z]$ and $[z y]$ together form the side $[x y]$ of $\trig p x y$.

If both triangles $\trig p x z$ and $\trig p y z$ are thin, 
then the triangle $\trig p x y$ is also thin.

Analogously, if $\trig p x y$ has perimeter $<2\cdot\pi$ and both triangles $\trig p x z$ and $\trig p y z$ are spherically thin, then triangle $\trig p x y$ is spherically thin.
\end{thm} 


\begin{wrapfigure}{r}{32mm}
\begin{lpic}[t(-4mm),b(0mm),r(0mm),l(0mm)]{pics/cat-monoton-ineq(1)}
\lbl[b]{14,23;$\dot z$}
\lbl[t]{10,.5;$\dot p$}
\lbl[r]{1,14;$\dot x$}
\lbl[l]{30.5,14;$\dot y$}
\lbl[tl]{13,13;$\dot w$}
\end{lpic}
\end{wrapfigure}

\parit{Proof.}
Construct  the model triangles $\trig{\dot p}{\dot x}{\dot z}\z=\modtrig(p x z)_{\EE^2}$ 
and $\trig {\dot p} {\dot y} {\dot z}\z=\modtrig(p y z)_{\EE^2}$ so that $\dot x$ and $\dot y$ lie on opposite sides of $[\dot p\dot z]$.

Let us show that 
\[\angk{z}{p}{x}+\angk{z}{p}{y}
\ge
\pi.
\eqlbl{eq:<+<>=pi}\]
Suppose the contrary, that is
\[\angk{z}{p}{x}+\angk{z}{p}{y}
<
\pi.\]
Then for some point $\dot w\in[\dot p\dot z]$, we have \[\dist{\dot x}{\dot w}{}+\dist{\dot w}{\dot y}{}
<
\dist{\dot x}{\dot z}{}+\dist{\dot z}{\dot y}{}=\dist{x}{y}{}.\]
Let $w\in[p z]$ correspond to $\dot w$; that is, $\dist{z}{w}{}=\dist{\dot z}{\dot w}{}$. 
Since $\trig p x z$ and $\trig p y z$ are thin, we have 
\[\dist{x}{w}{}+\dist{w}{y}{}<\dist{x}{y}{},\]
contradicting the triangle inequality. 

Denote by $\dot D$ the union of two solid triangles $\trig {\dot p}{\dot x}{\dot z}$ and $\trig {\dot p} {\dot y} {\dot z}$.
Further, denote by $\tilde D$ the solid triangle $\trig{\tilde  p}{\tilde  x}{\tilde  y}=\modtrig(p x y)_{\EE^2}$.
By \ref{eq:<+<>=pi}, there is a short map $F\:\tilde D\to \dot D$ that sends 
\begin{align*}
\tilde p&\mapsto \dot p,
&
\tilde x&\mapsto \dot x,
&
\tilde z&\mapsto \dot z,
&
\tilde y&\mapsto \dot y.
\end{align*}
\qedsf

\begin{thm}{Exercise}\label{ex:short-map}
Use Alexandrov's lemma (\ref{lem:alex}) to prove the last statement. 
\end{thm}


By assumption, the natural maps $\trig {\dot p} {\dot x} {\dot z}\to\trig p x z$ and $\trig {\dot p} {\dot y} {\dot z}\to\trig p y z$ are short.  
By composition,  the natural map from $\trig{\tilde  p}{\tilde  x}{\tilde  y}$ to $\trig p y z$ is short, as claimed.

The spherical case is done along the same lines.
\qeds

\begin{thm}{Exercise}\label{ex:convex-balls}
Show that any ball in a proper length $\CAT(0)$ space is a convex set.

Analogously, show that any ball of radius $R<\tfrac\pi2$ in a proper length $\CAT(1)$ space  is a convex set.
\end{thm}

Recall that a set $A$ in a metric space $\spc{U}$ is called locally convex if for any point $p\in A$ there is an open neighborhood $\spc{U}\ni p$ such that any geodesic in $\spc{U}$ with  ends in $A$ lies in~$A$. 

\begin{thm}{Exercise}\label{ex:locally-convex}
Let $\spc{U}$ be a proper length $\CAT(0)$ space.
Show that any closed, connected, locally convex set in $\spc{U}$ is convex.
\end{thm}

\begin{thm}{Exercise}\label{ex:closest-point}
Let  $\spc{U}$ be a proper length $\CAT(0)$ space 
and $K\subset \spc{U}$ be a closed convex set.
Show that: 

\begin{subthm}{ex:closest-point:a}
For each point $p\in \spc{U}$ there is unique point $p^*\in K$ that minimizes the distance $\dist{p}{p^*}{}$.
\end{subthm}

\begin{subthm}{}
The closest-point projection $p\mapsto p^*$ defined by (\ref{SHORT.ex:closest-point:a}) is short. 
\end{subthm}

\end{thm}




















\begin{thm}{Advanced exercise}\label{ex:urysohn-contractable}
 Show that the space $\spc{U}$ is contactable.
\end{thm}


\parbf{Advanced exercise~\ref{ex:urysohn-contractable}.}
Note that points in the space $\spc{X}_\infty$ constructed in the proof of \ref{prop:univeral-separable} can be multiplied number $t\in [0,1]$ --- simply multiply each function by factor $t$.
That defines a map 
\[\lambda_t\:\spc{X}_\infty\to \spc{X}_\infty\]
that scales all distances by factor $t$.
The map $\lambda_t$ can be extended to the completion of $\spc{X}_\infty$, which is isometic to $\spc{U}_d$ (or $\spc{U}$).

Observe that 
the map $\lambda_1$ is the identity  and $\lambda_0$ maps whole space to a single point, say $x_0$ --- that is the only point of $\spc{X}_0$.
Further note that the map $(t,p)\mapsto \lambda_t(p)$ is continuous ---  in particular $\spc{U}_d$ and $\spc{U}$ are contractible.\qeds

Source: \cite[(d) on page 82]{gromov-2007}.

Observe that for any point $p\in \spc{U}_d$ the curve $t\mapsto \lambda_t(p)$ is a geodesic path from $p$ to $x_0$.








Note that $\spc{M}$ --- the space of compact metric spaces can be treated as a space of compact subsets in $\spc{U}$ up to congruence.
Namely two subsets $A$ and $A'$ are called \emph{congruent} (briefly $A\cong A'$) if there is isometry of the ambient space $\spc{U}$ that maps $A$ to $A'$.
Let us define distance between congruence classes of two compact subsets $A$ and $B$ as 
\[\inf\set{\dist{A'}{B}{\spc{H}(\spc{U})}}{A'\cong A}.\]


By \ref{prop:sep-in-urys}, any compact metric spaces $\spc{K}$ admits a distance preserving map $f\:\spc{K}\to\spc{U}$.
Moreover by \ref{thm:compact-homogeneous} any two such maps $f_1$ and $f_2$ differ by isometry of $\spc{U}$;
that is, there is an isometry $\iota\:\spc{U}\to\spc{U}$ such that $f_2=\iota\circ f_1$.
In particular $f_1(\spc{K})\cong f_2(\spc{K})$.










\section{Ultratangent space} 

Recall that we assume that $\omega$ is a once for all fixed choice of a nonprinciple ultrafilter.

For a metric space $\spc{X}$ and a positive real number $\lambda$,
we will denote by $\lambda\cdot\spc{X}$ its \emph{$\lambda$-blowup}\index{blowup},
which is a metric space with the same underlying set as $\spc{X}$ and the metric multiplied by $\lambda$.
The tautological bijection $\spc{X}\to \lambda\cdot\spc{X}$ will be denoted as $x\mapsto x^\lambda$, 
so 
\[\dist{x^\lambda}{y^\lambda}{}
=
\lambda\cdot\dist[{{}}]{x}{y}{}\] 
for any $x,y\in \spc{X}$.

The $\omega$-blowup $\omega\cdot\spc{X}$ of $\spc{X}$ is defined as the $\omega$-limit
of the $n$-blowups $n\cdot\spc{X}$; that is,
\[\omega\cdot\spc{X}
\df
\lim_{n\to\omega} n\cdot\spc{X}.\]

Given a point $x\in \spc{X}$ we can consider the sequence $x^n\in n\cdot\spc{X}$;
it corresponds to a point $x^\omega\in \omega\cdot\spc{X}$.
Note that if $x\ne y$, then 
\[\dist{x^\omega}{y^\omega}{\omega\cdot\spc{X}}=\infty;\]
that is, 
$x^\omega$ and $y^\omega$ 
belong to different metric components of $\omega\cdot\spc{X}$.

The metric component of $x^\omega$ in $\omega\cdot\spc{X}$ is called ultratangent space of $\spc{X}$ at $x$ and it is denoted as $\T^\omega_x\spc{X}$.

Equivalently, ultratangent space $\T^\omega_x\spc{X}$ can be defined the following way.
Consider all the sequences of points $x_n\in \spc{X}$ such that
the sequence $\ell_n=n\cdot\dist{x}{x_n}{\spc{X}}$ is bounded.
Define the pseudodistance between two such sequences as 
\[\dist{(x_n)}{(y_n)}{}
=
\lim_{n\to\omega}n\cdot\dist{x_n}{y_n}{\spc{X}}.\]
Then $\T^\omega_x\spc{X}$ is the corresponding metric space.

Tangent space as well as ultratangent space, 
generalize the notion of tangent space of Riemannian manifold.
In the simplest cases these two notions define the same space.
In general, they are different and both useful ---
often lack of a property in one is compensated by the other.

It is clear from the definition that tangent space has cone structure.
On the other hand, in general, ultratangent space does not have a cone structure; 
the Hilbert's cube $\prod_{n=1}^\infty[0,2^{-n}]$ is an example --- it is $\Alex{0}$ as well as $\CAT{0}$.

The next theorem shows that the tangent space $\T_p$ can be (and often will be) considered as a subset of  $\T^\omega_p$.

\begin{thm}{Theorem}\label{thm:tangent-ultratangent}
\label{thm:T-in-T^w} 
Let $\spc{X}$ be a metric space with defined angles.
Then for any $p\in \spc{L}$, there is an distance preserving map 
\[\iota:\T_p\hookrightarrow \T^\omega_p\] 
such that for any geodesic $\gamma$ starting at $p$
we have 
\[\gamma^+(0)\mapsto \lim_{n\to\omega}[\gamma(\tfrac1n)]^n.\]

\end{thm}

\parit{Proof.}
Given $v\in \T'_p$ 
choose a geodesic $\gamma$ that starts at $p$ such that $\gamma^+(0)\z=v$.
Set $v^n=[\gamma(\tfrac1n)]^n\in n\cdot \spc{X}$ and 
\[v^\omega=\lim_{n\to\omega}v^n.\]

Note that the value $v^\omega\in\T^\omega_p$ does not depend on choice of $\gamma$;
that is, if $\gamma_1$ is an other geodesic starting at $p$ such that $\gamma_1^+(0)=v$,
then 
\[\lim_{n\to\omega}v^n=\lim_{n\to\omega}v_1^n,\]
where $v_1^n=[\gamma_1(\tfrac1n)]^n\in n\cdot \spc{X}$.
The latter follows since
\[\dist{\gamma(t)}{\gamma_1(t)}{\spc{X}}=o(t)\]
and therefore $\dist{v^n}{v_1^n}{n\cdot \spc{X}}\to 0$ s $n\to\infty$.



Set $\iota(v)=v^\omega$.
Since angles between geodesics in $\spc{X}$ are defined, for any $v,w\in \T_p'$ we have
$n\cdot\dist[{{}}]{v_n}{w_n}{}\to\dist{v}{w}{}$.
Thus $\dist{v_\omega}{w_\omega}{}=\dist{v}{w}{}$; that is, $\iota$ is a global isometry of $\T_p'$.

Since $\T_p'$ is dense in $\T_p$,
we can extend $\iota$ to a global isometry $\T_p\to \T^\omega_p$.
\qeds

{\sloppy

\section[Gromov--Hausdorff and ultralimits]{Gromov--Hausdorff convergence and ultralimits}

}

\begin{thm}{Theorem}\label{thm:ultra-GH}
Assume $\spc{X}_n$ is a sequence of complete spaces. 
Let $\spc{X}_n\to \spc{X}_\omega$ as $n\to\omega$,
and $\spc{Y}_n\subset \spc{X}_n$ 
be a sequence of subsets such that $\spc{Y}_n\GHto\spc{Y}_\infty$. 
Then there is a distance preserving map 
$\iota:\spc{Y}_\infty\to \spc{X}_\omega$.

Moreover:

\begin{subthm}{thm:ultra-GH:a}
If $\spc{X}_n\GHto \spc{X}_\infty$ 
and $\spc{X}_\infty$ is compact, then 
$\spc{X}_\infty$ is isometric to $\spc{X}_\omega$.
\end{subthm}

\begin{subthm}{thm:ultra-GH:b}
If $\spc{X}_n\GHto \spc{X}_\infty$ 
and $\spc{X}_\infty$ is proper, then 
$\spc{X}_\infty$ is isometric to a metric component of $\spc{X}_\omega$.
\end{subthm}

\end{thm}

\parit{Proof.} 
For each point $y_\infty\in \spc{Y}_\infty$ 
choose a lifting $y_n\in \spc{Y}_n$.
Pass to the $\omega$-limit $y_\omega\in \spc{X}_\omega$ of $(y_n)$.
Clearly for any $y_\infty,z_\infty\in \spc{Y}_\infty$, 
we have 
\[\dist{y_\infty}{z_\infty}{\spc{Y}_\infty}=\dist{y_\omega}{z_\omega}{\spc{X}_\omega};\] 
that is, the map $y_\infty\mapsto y_\omega$ gives a distance preserving map $\iota:\spc{Y}_\infty\to \spc{X}_\omega$. 


\parit{(\ref{SHORT.thm:ultra-GH:a})$+$(\ref{SHORT.thm:ultra-GH:b}).}
Fix $x_\omega\in \spc{X}_\omega$.
Choose a sequence $x_n\in \spc{X}_n$ 
such that $x_n\to x_\omega$ as $n\to\omega$. 

Denote by $\bm{X}=\spc{X}_\infty\sqcup\spc{X}_1\sqcup\spc{X}_2\sqcup\dots$ the common space for the convergence $\spc{X}_n\GHto \spc{X}_\infty$;
as in the definition of Gromov--Hausdorff convergence.
Consider the sequence $(x_n)$ 
as a sequence of points in~$\bm{X}$.

If the $\omega$-limit $x_\infty$ of $(x_n)$ exists, 
it must lie in $\spc{X}_\infty$. 

The point $x_\infty$, if defined, does not depend on the choice of $(x_n)$.
Indeed, if $y_n\in\spc{X}_n$ is an other sequence such that $y_n\to x_\omega$ as $n\to\omega$, then 
\[
\dist{y_\infty}{x_\infty}{}=\lim_{n\to\omega}\dist{y_n}{x_n}{}=0;
\]
that is, $x_\infty=y_\infty$.


In this way we obtain a map $\nu\:x_\omega\to x_\infty$;
it is defined on a subset of $\Dom\nu \subset\spc{X}_\omega$.
By construction of $\iota$, 
we get  $\iota\circ\nu(x_\omega)=x_\omega$ for any $x_\omega\in \Dom\nu$.

Finally note that if $\spc{X}_\infty$ is compact, then $\nu$ is defined on all of $\spc{X}_\omega$;
this proves (\ref{SHORT.thm:ultra-GH:a}).

If $\spc{X}_\infty$ is proper, choose any point $z_\infty\in \spc{X}_\infty$
and set $z_\omega=\iota(z_\infty)$.
For any point $x_\omega\in \spc{X}_\omega$ at finite distance from $z_\omega$,
for the sequence $x_n$ 
as above we have that $\dist{z_n}{x_n}{}$ is bounded for $\omega$-almost all $n$.
Since $\spc{X}_\infty$ is proper, $\nu(x_\omega)$ is defined;
in other words $\nu$ is defined on the metric component of $z_\omega$.
Hence (\ref{SHORT.thm:ultra-GH:b}) follows.
\qeds

\begin{thm}{Corollary} 
\label{cor:ulara-geod}
The $\omega$-limit of a sequence of complete length spaces is geodesic.
\end{thm}

\parit{Proof.} Given two points $x_\omega,y_\omega\in \spc{X}_\omega$, find two bounded sequences of points $x_n,y_n\in \spc{X}_n$, $x_n\to x_\omega$, $y_n\to y_\omega$ as $n\to\omega$.
Consider a sequence of paths  $\gamma_n\:[0,1]\to \spc{X}_n$ from $_n$ to $y_n$
 such that 
\[\length\gamma_n\le \dist{x_n}{y_n}{}+\tfrac{1}{n}.\]
Apply Theorem~\ref{thm:ultra-GH} 
for the images $\spc{Y}_n=\gamma_n([0,1])\subset \spc{X}_n$.
\qeds

\section{Ultralimits of sets}

Let $\spc{X}_n$ be a sequence of metric spaces and $\spc{X}_n\to \spc{X}_\omega$
as $n\to \omega$.

For a sequence of sets $\Omega_n\subset \spc{X}_n$,
consider the maximal set $\Omega_\omega\subset \spc{X}_\omega$ such that 
for any $x_\omega\in\Omega_\omega$ and any sequence $x_n\in\spc{X}_n$ such that $x_n\to x_\omega$ as $n\to \omega$, we have $x_n\in\Omega_n$ for $\omega$-almost all $n$.

The set $\Omega_\omega$ is called the  \emph{open $\omega$-limit} of $\Omega_n$;
we could also write  $\Omega_n\to \Omega_\omega$ as $n\to\omega$ or $\Omega_\omega=\lim_{n\to\omega}\Omega_n$. 

{\sloppy

Applying Observation~\ref{obs:ultralimit-is-complete} to the sequence of complements $\spc{X}_n\backslash \Omega_n$, we see that $\Omega_\omega$ is open for any sequence $\Omega_n$.
The definition can be applied for arbitrary sequences of sets, but  
open $\omega$-convergence  will be applied here only for sequences of open sets.

}

\section{Ultralimits of functions}

Recall that a family of submaps between metric spaces $\{f_\alpha\: \spc{X}\to\spc{Y}\}_{\alpha\in\mathcal A}$ is called \emph{equicontinuous} if for any $\eps>0$ there is $\delta>0$ such that for any $p,q\in\spc{X}$ with $\dist{p}{q}{}<\delta$ and any $\alpha\in\mathcal A$ it holds that $\dist{f(p)}{f(q)}{}<\eps$.

Let $f_n\:\spc{X}_n\to\RR$ be a sequence of subfunctions.

Set $\Omega_n=\Dom f_n$.
Consider the open $\omega$-limit set $\Omega_\omega\subset \spc{X}_\omega$ of $\Omega_n$.

Assume there is a subfunction $f_\omega\:\spc{X}_\omega\to\RR$
that satisfies the following conditions: 
(1) $\Dom f_\omega=\Omega_\omega$, (2) if $x_n\to x_\omega\in \Omega_\omega$ for a sequence of points $x_n\in\spc{X}_n$, then $f_n(x_n)\to f_\omega(x_\omega)$ as $n\to\omega$.
In this case 
the subfunction $f_\omega\:\spc{X}_\omega\to\RR$ 
is said to be the 
$\omega$-limit of $f_n\:\spc{X}_n\to\RR$.

The following lemma gives a mild condition on a sequence of functions $f_n$
guaranteeing the existence of its $\omega$-limit.

\begin{thm}{Lemma}
Let $\spc{X}_n$ be a sequence of metric spaces
and $f_n\:\spc{X}_n\to\RR$ be a sequence of subfunctions.

Assume for any positive integer $k$, there is an open set $\Omega_n(k)\subset \Dom f_n$
such that the restrictions $f_n|_{\Omega_n(k)}$ are uniformly bounded and continuous
and the open $\omega$-limit of $\Omega_n(n)$ coincides with the open $\omega$-limit of $\Dom f_n$.
Then the $\omega$-limit of $f_n$ is defined.

In particular, if the $f_n$ are uniformly bounded and continuous, then the $\omega$-limit is defined.
\end{thm}

The proof is straightforward.

{\sloppy

\begin{thm}{Exercise}\label{ex:nonconvex-limit}
Construct a sequence of compact length spaces 
$\spc{X}_n$  
with a converging sequence of $\Lip$-Lipschitz concave functions $f_n\:\spc{X}_n\to\RR$ such that
the $\omega$-limit $\spc{X}_\omega$ of $\spc{X}_n$ is compact
and the $\omega$-limit $f_\omega\:\spc{X}_\omega\to\RR$ of $f_n$ is not concave.
\end{thm}

}

If $f\:\spc{X}\to\RR$ is a subfunction, 
the $\omega$-limit of the constant sequence $f_n=f$ is called the $\omega$-power of $f$ and denoted by $f^\omega$.
So
\[f^\omega\:\spc{X}\to\RR,\ \ f^\omega(x_\omega)=\lim_{n\to\omega} f(x_n).\]

Recall that we treat $\spc{X}$ as a subset of its $\omega$-power $\spc{X}^\omega$.
Note that $\Dom f=\spc{X}\cap \Dom f^\omega$.
Moreover, 
if $\oBall(x,\eps)_{\spc{X}}\subset \Dom f$
then $\oBall(x,\eps)_{\spc{X}^\omega}\subset \Dom f^\omega$.


\parbf{Ultradifferential.}
Given a function $f\:\spc{L}\to\RR$, consider sequence of functions $f_n\:n\cdot\spc{L}\to\RR$, defined by 
\[f_n(x^n)=n\cdot(f(x)-f(p)),\]
here $x^n\in n\cdot\spc{L}$ is the point corresopnding to $x\in\spc{L}$.
While $n\cdot(\spc{L},p)\to(\T^\omega,\0)$ as $n\to\omega$, 
functions $f_n$ converge to $\omega$-differential of $f$ at $p$.
It will be denoted by $\dd_p^\omega f$;
\[\dd_p^\omega f\:\T_p^\omega\to\RR,\ \ \dd_p^\omega f=\lim_{n\to\omega} f_n.\] 

Clearly, the $\omega$-differential $\dd_p^\omega f$ of a locally Lipschitz subfunction $f$ is defined at each point $p\in \Dom f$.
















\section{Comments} 

Given two metric spaces $\spc{X}$ and $\spc{Y}$, we will write $\spc{X}\preccurlyeq \spc{Y}$ if there is a noncontracting map $f\:\spc{X}\to \spc{Y}$;
that is, if 
$$ |x-x'|_{\spc{X}}\le|f(x)-f(x')|_{\spc{Y}}$$
for any $x,x'\in \spc{X}$.

Further, given $\eps>0$, we will write $\spc{X}\preccurlyeq \spc{Y}+\eps$
if there is a map $f\:\spc{X}\to \spc{Y}$ such that 
$$|x-x'|_{\spc{X}}\le|f(x)-f(x')|_{\spc{Y}}+\eps$$
for any $x,x'\in \spc{X}$.

Define 
$$\dist[\star]{\spc{X}}{\spc{Y}}{\spc{M}}=\inf\set{\eps}{\spc{X}\preccurlyeq \spc{Y}+\eps
\quad\text{and}\quad
\spc{Y}\preccurlyeq \spc{X}+\eps}$$
It turns out that $\dist[\star]{*}{*}{\spc{M}}$ is a different metric on the set of isometry classes of compact metric spaces; that is, in general $\dist[\star]{\spc{X}}{\spc{Y}}{\spc{M}}\not=|\spc{X}-\spc{Y}|_{\spc{M}}$. 
However, these two metrics define the same topology on $\spc{M}$.
More precicely:

\begin{thm}{Proposition}\label{GH-po}
For any sequence of compact metric spaces $(\spc{X}_n)$ and a compact metric space $\spc{X}_\infty$,
we have
$$|\spc{X}_n-\spc{X}_\infty|_{\spc{M}}\to 0
\quad\iff\quad
\dist[\star]{\spc{X}_n}{\spc{X}_\infty}{\spc{M}}\to 0$$ 
as $n\to\infty$.
\end{thm}

We will not give a proof of this proposition. 
Likely, we will not use it further in the lectures, 
but it might help you to build intuition for Gromov--Hausdorff convergence.
If you want to prove it yourself look in the proof of Theorem~\ref{thm:GH-is-a-metric} 
and try to modify it using ideas from the proof of Problem~\ref{pr:non-contracting=>isometry}.

The Gromov--Hausdorff distance can be defined for arbitrary pair of metric space.
Therefore it is natural to ask why we only consider compact metric spaces.
First note the Gromov--Hausdorff distance from any metric space $\spc{X}$ 
to its completion $\bar {\spc{X}}$ is zero.
Therefore if you want to end up in a metric space, it is better to consider only complete metric spaces.

Further, the distance between one-point-space and a metric spce with infinite diameter is infinite.
Therefore one has to either consider only bounded metric spaces (that is, the spaces with finite diameter)
or relux the definition of metric space which allow metric to take infinite value.

Finally, the class of isometry classes of all bounded complete metric spaces forms a class, but not a set.
Hence again we have two choices: either relux the definition of metric space so its points will form a class, or restrict further the class of spaces for which the isometry classes will form a set.

\begin{thm}{Exercise}
Prove that isometry classes of compact metric spaces form a set. 
\end{thm}

\begin{thm}{Exercise}\label{pr:GH1}
Let $\spc{X}=\{x,y,z\}$ be a three point subset of Euclidean plane with distances
$$|x-y|=|y-z|=|z-x|=1.$$
\begin{enumerate}[(i)]
\item Find the minimal Hausdorff distance from $\spc{X}$ to a one-point subset of the plane.
\item Find the Gromov--Hausdorff distance from $\spc{X}$ to the one-point metric space. 
\end{enumerate}
\end{thm}

\begin{thm}{Exercise}\label{pr:GH2}
Let $\spc{X}$ and $\spc{Y}$ be a compact metric spaces which have isometric $\eps$-nets.
Show that 
$$|\spc{X}-\spc{Y}|_{\spc{M}}\le 2\cdot\eps.$$
Is it always true that 
$$|\spc{X}-\spc{Y}|_{\spc{M}}\le \eps?$$
\end{thm}




\begin{thm}{Exercise}\label{pr:GH3}
Define the \emph{radius of a metric space}\index{radius of a metric space} $\spc{X}$ as 
$$\rad \spc{X}=\inf_x\left\{\sup_y\{|x-y|_{\spc{X}}\}\right\}.$$
Equivalently, 
$$\rad \spc{X}=\inf\set{R>0}{\text{there is}\ x\in \spc{X}\  \text{such that}\ B_R(x)\supset \spc{X}}.$$
 
\begin{enumerate}[(i)]
\item Show that for any compact metric space $\spc{X}$ we have
$$\tfrac12\cdot\diam \spc{X}\le \rad \spc{X}\le \diam \spc{X}.$$
\item Show that for any compact metric spaces $\spc{X},\spc{Y}$ we have
$$|\rad \spc{X}-\rad \spc{Y}|\le 2\cdot |\spc{X}-\spc{Y}|_{\spc{M}}.$$
\end{enumerate}
\end{thm}

\begin{thm}{Exercise}\label{pr:F-X}
Let $\spc{X}$ be a metric space.
If two compact sets $A, B$ in $\spc{X}$ are isometric,
we will write $A\iso B$. 
Set
$$d(A,B)=\inf \set{|A'-B'|_{\mathcal{H}(\spc{X})}}{A'\iso A \ \text{and}\ B'\iso B}.$$
Note that if $\spc{X}=\ell^\infty$, then according to Proposition~\ref{prop:GH-with-fixed-Z}, 
$d$ is a metric on $\mathcal{H}(\spc{X})/\iso$ (that is, on the ``$\iso$''-equivalecne classes of $\mathcal{H}(\spc{X})$).

Show that it does not hold for arbitrary metric space $\spc{X}$.
Understand the reason why it holds for $\spc{X}=\ell^\infty$.
\end{thm}


\begin{thm}{Exercise}\label{pr:GH-variation}
Consider the pairs $(\spc{X},A)$, where $\spc{X}$ is a compact metric space and $A$ is a closed subset in $\spc{X}$.
Two such pairs, say $(\spc{X},A)$ and $(\spc{X}',A')$ will be called equivalent (briefly $(\spc{X},A)\sim(\spc{X}',A')$)
if there is an isometry $\iota\:\spc{X}\to \spc{X}'$ such that $\iota(A)=A'$.

Modify the definition of Gromov--Hausdorff metric to construct a natural metric on the set of $\sim$-equivalence classes of the pairs $(\spc{X},A)$.
\end{thm}

Here we introduce so called Gromov--Hausdorff convergence for metric spaces.
This convergence was introduced by Gromov around 1980, published in \cite{gromov-1981}.
Very soon this notion began to be used in all branches of geometry.
In fact today I have difficulty to understand 
how one could do geometry without this type of convergence.%
(Some types of convergences of metric spaces was considered before Gromov,
but they had lack of generality;
each type of convergence was desined to solve one particular problem.)


\begin{thm}{Exercise}\label{ex:euclid-isom}
\begin{subthm}{}
Let $\spc{X},\spc{Y}$ be two compact sets in the Euclidean plane $\RR^2$.
Show that $\spc{X}$ is isometric to $\spc{Y}$ if and only if there is an motrio $\iota\:\RR^2\to \RR^2$
that sends $\spc{X}$ to $\spc{Y}$.
\end{subthm}

\begin{subthm}{}
Find two isometric subsets $\spc{X},\spc{Y}$ of $\ell^\infty$
such that there is no isometry $\iota\:\ell^\infty\to \ell^\infty$ 
that sends $\spc{X}$ to $\spc{Y}$.
\end{subthm}
\end{thm}

\appendix
\chapter{Semisolutions}
\parbf{\ref{ex:besikovitch=}.}
Let us use the same notation as in the proof of \ref{thm:besikovitch}.

Consider the map $s\:x\mapsto(\distfun_A(x),\distfun_B(x))$.
From the proof of \ref{thm:besikovitch} we get that $\Im s\supset \square$.
Observe that in the case of equality we have that $\Im s= \square$.
Indeed,
the same argument shows that 
\[\vol(s^{-1}(\square),g)\ge \vol\square=1.\]
The set $s^{-1}(\RR^1\backslash \square)$ is an open subset of $\square$.
If it is nonempty, then it has positive volume.
In this case
\[\vol(\square,g)>\vol(s^{-1}(\square),g)\ge 1\]
--- a contradiction.

Summarizing above discussion, there is a geodesic path of $g$-length $1$ connecting a point on one face of cube to the opposite face.

Moreover, for any pair of opposite faces and a point $p\in\square$, there is a geodesic path of $g$-length $1$ from one face to the other that pass thru $p$.
The latter can be shown by cutting $\square$ into two rectangles by a level surface of $\distfun_A$ thru $p$,
applying the above statement to both rectangles and taking the concatination of the obtained geodesic paths with end at $p$.
(The level surface might cut a rectangle with some topology, so have to apply \ref{thm:besikovitch+} instead of \ref{thm:besikovitch}).

Let $\gamma$ be such geodesic path from $A$ to $A'$.
Observe that $\gamma'(t)\z=\nabla_{\gamma(t)}\distfun_A$.
Therefore $\distfun_A$ is differentiable at every point $p\in \square$.
It follows that the map $s$ is differentiable.

Further checking the equality case in each inequality in the proof of \ref{thm:besikovitch}, we get that $s$ is a bijection and the equalities
\[|d_{p}\distfun_A|= 1,\quad|d_{p}\distfun_B|=1,\quad \text{and}\quad \langle d_{p}\distfun_A,d_{p}\distfun_B\rangle= 0\]
hold for almost all $p\in\square$.
Since $d_{p}\distfun_A$ and $d_{p}\distfun_B$ are well defined, we get that the equalities hold everywhere.
That is $s$ is an isometry.

\begin{wrapfigure}{r}{45 mm}
\vskip-4mm
\centering
\includegraphics{mppics/pic-27}
\end{wrapfigure}

\parbf{\ref{ex:hexagon}.}
Consider the hexagon with flat matric and curved sides shown on the diagram.
Observe that its area can be made arbitrary small while keeping the distances from the opposite sides at least 1.

\parbf{\ref{ex:gadograph}.}
Without loss of generality, we may assume that $V$ lies in a unit cube $\square$.
Consider a noncontinuous metric tensor $\bar g$ on $\square$ that coincides with $g$ inside $V$ and with the canonical flat metric tensor outside of $V$.

Observe that the $\bar g$-distances between opposite faces of $\square$ are at least 1.
Indeed this is true for the Euclidean metric and the assumption $\dist{p}{q}{g}\ge\dist{p}{q}{\EE^d}$  guarantees that one cannot make a shortcut in~$V$.
Therefore the $\bar g$-distances between every pair of opposite faces is at least as large as 1 which is the Euclidean distance.

This metric tensor $\bar g$ is not continuous at $\Sigma$, but the same argument as in \ref{thm:besikovitch} can be applied to show that $\vol(\square,\bar g)\ge \vol\square$.
Whence the statement follows.


\parbf{\ref{ex:involution-of-sphere}.}
Let $x\in \mathbb{S}^2$ be a point that minimize the distance $|x-x'|_g$.
Consider a minimizing geodesic $\gamma$ from $x$ to $x'$.
We can assume that 
\[|x-x'|_g=\length \gamma=1.\]

Let $\gamma'$ be the antipodal arc to $\gamma$.
Note that $\gamma'$ intersects $\gamma$ only at the common endpoints $x$ and $x'$.
Indeed, if $p'=q$ for some $p,q\in\gamma$, then $|p-q|\ge 1$.
Since $\length \gamma=1$, the points $p$ and $q$ must be the ends of $\gamma$.

It follows that $\gamma$ together with $\gamma'$ forms a closed simple curve in $\mathbb{S}^2$
that divides the sphere into two disks $D$ and $D'$.

Let us divide $\gamma$ into two equal arcs $\gamma_1$ and $\gamma_2$; each of length $\tfrac12$.
Suppose that $p,q\in\gamma_1$, then 
\begin{align*}
|p-q'|_g&\ge |q-q'|_g-|p-q|_g\ge
\\
&\ge 1-\tfrac12=\tfrac12.
\end{align*}
That is, the minimal distance from $\gamma_1$ to $\gamma_1'$ is at least~$\tfrac12$.
The same way we get that the minimal distance from $\gamma_2$ to $\gamma_2'$ is at least~$\tfrac12$.
By Besicovitch inequality, we get that 
\[\area(D,g)\ge \tfrac14\quad\text{and}\quad \area(D',g)\ge \tfrac14.\]
Therefore 
\[\area(\mathbb{S}^2,g)\ge\tfrac12.\]

\parit{A better estimate.}
Let us indicate how to improve the obtained bound to
\[\area(\mathbb{S}^2,g)\ge1.\]

Suppose $x$, $x'$, $\gamma$ and $\gamma'$ are as above.
Consider the function
\[f(z)=\min_t \{\,|\gamma'(t)-z|_g+t\,\}.\]
Observe that $f$ is 1-Lipschitz.

Show that two points $\gamma'(c)$ and $\gamma(1-c)$ lie on one connected component of the level set $L_c=\set{z\in\mathbb{S}^2}{f(z)=c}$;
in particular 
\[\length L_c\ge 2\cdot|\gamma'(c)-\gamma(1-c)|_g.\]
By the triangle inequality, we have that
\begin{align*}
|\gamma'(c)-\gamma(1-c)|_g&\ge 1-|\gamma(c)-\gamma(1-c)|_g=
\\
&=1-|1-2\cdot c|.
\end{align*}

It remains to apply the coarea formula
\[\area(\mathbb{S}^2,g)\ge \int\limits_0^1\length L_c\cdot dc.\]

\parit{Remarks.}
The bound $\tfrac12$ was proved by Marcel Berger \cite{berger}. 
Christopher Croke conjectured that the optimal bound is $\tfrac4\pi$ and the round sphere is the only space that achieves this \cite[Conjecture 0.3 in][]{croke}.

\begin{wrapfigure}{r}{20 mm}
\vskip-0mm
\centering
\includegraphics{mppics/pic-1305}
\end{wrapfigure}

\parbf{\ref{ex:involution-of-3sphere}.}
Given $\eps>0$, construct a disk $\Delta$ in the plane with 
\[\length\partial \Delta<10\ \ \text{and}\ \ \area \Delta<\eps\]
that admits an continuous involution $\iota$ such that 
\[|\iota(x)-x|\ge 1\]
for any $x\in\partial \Delta$.

An example of $\Delta$ can be guessed from the picture;
the invoultion $\iota$ makes a length preserving half turn of its boundary $\partial \Delta$.


Take the product $\Delta\times \Delta\subset \RR^4$;
it is homeomorphic to the 4-ball.
Note that 
$$\vol_3[\partial(\Delta\times \Delta)]=2\cdot\area \Delta\cdot\length \partial \Delta<20\cdot\eps.$$
The boundary $\partial(\Delta\times \Delta)$ is homeomorphic to $\mathbb{S}^3$
and the restriction of the involution $(x,y)\z\mapsto (\iota(x),\iota(y))$ has the needed property.

All we have to do now is to smooth $\partial(\Delta\times \Delta)$ a little bit.

\parit{Remark.} This example is given by Christopher Croke \cite{croke}.
Note that according to \ref{thm:sys+}, 
the involution $\iota$ cannot be made isometric.

\parbf{\ref{ex:GH-vol}.}
Note that if $\spc{M}_\infty$ is $e^{\pm\eps}$-bilipschitz to a cube, then applying Besicovitch inequality, we get that 
\[\liminf_{n\to\infty} \vol \spc{M}_n\ge e^{-n\cdot \eps}\cdot\vol \spc{M}_\infty.\]

Applying Vitali covering theorem, given $\eps>0$, we can cover whole volume of $\spc{M}_\infty$ by $e^{\pm\eps}$-bilipschitz cubes.
Applying the above observation and summing up the results, we get that 
\[\liminf_{n\to\infty} \vol \spc{M}_n\ge e^{-n\cdot \eps}\cdot\vol \spc{M}_\infty.\]
The statement follows since $\eps$ is arbitrary positive number.

\parit{Remark.} A more general result was obtaind by Sergei Ivanov~\cite{ivanov-1997}.
Note that the statement does not hold without stability of the convergence. In fact any compact metric space can be approximated by Riemannian surface with arbitrary small area.

\parbf{\ref{ex:sysT2}.}
Set $s=\sys(\TT^2,g)$.

Cut $\TT^2$ along a shortest closed noncontractible curve $\gamma_1$.
We get an anulus with a Riemnnian metric on it $(N,g)$.
Denote by $A$ and $A'$ the two components of its boundary.

Assume that $\gamma_2$ is a shortest path that runs from $A$ to $A'$ in $(N,g)$.
The image of $\gamma_2$ in $\TT^2$ connects two points in $\gamma_1$;
further we will use the same notation for $\gamma_2$ and its image in $\TT^2$.
Connect $\gamma_2(0)$ to $\gamma_2(1)$ by a shorter arc in $\gamma_1$.
Note that the obtained closed curve is noncontractible in $\TT^2$.
Therefore its length is at least $s$.
The arc of $\gamma_1$ has length at most half of $\length\gamma_1$.
Whence $\length \gamma_2\ge \tfrac s2$.
In particular the distance from $A$ to $A'$ in $(N,g)$ is at least $\tfrac s2$.

\begin{wrapfigure}{r}{45 mm}
\vskip-4mm
\centering
\includegraphics{mppics/pic-23}
\end{wrapfigure}

Let us cut $(N,g)$ by $\gamma_2$, we obtain a square $(\square,g)$ with Riemnnian metric on it.
Let us keep the notation $A$ and $A'$ for the pair of opposite sides in $(\square,g)$ that correspond to $A$ and $A'$ in $(N,g)$.
From above we have that distance from $A$ to $A'$ is at least $\tfrac s2$.

Denote by $B$ and $B'$ the remaining pair of opposite sides $(\square,g)$.
Suppose that $\gamma_3$ is a path connecting these sides.
Map it the curves $\gamma_i$ back to the torus and let us keep for them the same notation.
The path $\gamma_3$ connects two points on $\gamma_2$.
Since $\gamma_2$ is shortest, the arc of $\gamma_2$ between this pair of points cannot be longer than $\gamma_3$.
This arc together with $\gamma_3$ forms a closed noncontractible curve, so its length has to be at least $s$.
It follows that $\length\gamma_3\ge \tfrac s2$.
That is distance from $B$ to $B'$ in  $(\square,g)$ is at least $\tfrac s2$.

Applying Besikovitch inequality, we get that 
\[\area(\TT^2,g)=\area(\square,g)\ge \tfrac14\cdot s^2.\]

\parit{Remark.}
Alternatively one may notice that any curve in $(N,g)$ that is bordant to $A$ has length at least $\tfrac s2$.
Therefore the level sets defined by $\distfun_A(x)_{(N,g)}=t$ have length at least $\tfrac s2$ if $0\le t\le \tfrac s2$.
Applying coarea fromula we get that
\[\area(\TT^2,g)=\area(N,g)\ge \tfrac12\cdot s^2.\]
This estimate is twice better then the one above, but it is still far from the optimal bound $\tfrac2{\sqrt{3}}\cdot s^2$ in proved by Loewner inequality

\begin{wrapfigure}{r}{44 mm}
\vskip-4mm
\centering
\includegraphics{mppics/pic-25}
\end{wrapfigure}

\parbf{\ref{ex:sysRP2}.}
Set $s\z=\sys (\RP^2,g)$.
Cut $(\RP^2,g)$ along a shortest noncontractible curve $\gamma$.
We obtain $(\DD^2,g)$ --- a disc with metric tensor which we still denote by $g$.
Divide $\gamma$ into two equal arcs $\alpha$ and $\beta$.
Denote by $A$ and $A'$ the two connected components of the inverse image of $\alpha$.
Similarly denote by $B$ and $B'$ the two connected components of the inverse image of $\beta$.

Let $\gamma_1$ be a path from $A$ to $A'$;
map it to $\RP^2$ and keep the same notation for it.
Note that $\gamma_1$ together with a subarc of $\alpha$ forms a closed noncontractible curve in $\RP^2$.
Since $\length\alpha=\tfrac s2$, we have that $\length\gamma_1\ge \tfrac s2$.
It follows that the distance between $A$ and $A'$ in $(\DD^2,g)$ is at least $\tfrac s2$.
The same way we show that the distance between $B$ and $B'$ in $(\DD^2,g)$ is at least $\tfrac s2$.

Note that $(\DD^2,g)$ can be paraneterized by a square with sides $A$, $B$, $A'$ and $B'$ and apply \ref{thm:besikovitch} to show that 
\[\area(\RP^2,g)=\area(\DD^2,g)\ge \tfrac14\cdot s^2.\]

\parit{Remark.}
For the optimal constant was found by Pao Ming Pu see the discussion on page \pageref{page:pu}.
His proof shows that any Riemannian metric on the disc with the boundary globally isometric to a unit circle with angle metric has area at least as large as the unit hemisphere.
It is expected that the same inequality holds for any compact surface bounded by a single curve (not necessary a disc);
this is the so called the {}\emph{filling area conjecture} mentioned in \cite[5.5.B$'$(e$'$)]{gromov-1983}.

\parbf{\ref{ex:sysSg}.} Cut the surface along a shortest noncontractible curve $\gamma$. 
We might get a surface with one or two components of the boundary.
In these two cases repeat the arguments in \ref{ex:sysRP2} or \ref{ex:sysT2} using \ref{thm:besikovitch+} instead of \ref{thm:besikovitch}.


\parbf{\ref{ex:sysS2xS1}.} Consider the product of small 2-sphere with a unit circle.

\parbf{\ref{ex:macrodimension}.}
The following claim resembles Besikovitch inequality;
it is key to the proof:
\begin{itemize}
 \item[$({*})$] Let $a$ be a positive real number.
 Assume that a closed curve $\gamma$ in a metric space $\spc{X}$ can be sudivided into 4 arcs $\alpha$, $\beta$, $\alpha'$, and $\beta'$ in such a way that 
 \begin{itemize}
 \item $|x-x'|>a$ for any $x\in\alpha$ and $x'\in \alpha'$
 and
 \item $|y-y'|>a$ for any $y\in\beta$ and $y'\in \beta'$.
 \end{itemize}
 Then $\gamma$ is not contractable in its $\tfrac a2$-neighborhood.
\end{itemize}

To prove $({*})$, consider two functions defined on $\spc{X}$ as follows:
\begin{align*}
w_1(x)&=\min \{\,a,\distfun_{\alpha}(x)\,\}
\\
w_2(x)&=\min \{\,a,\distfun_{\beta}(x)\,\}
\end{align*}
and the map $\bm{w}\:\spc{X}\to [0,a]\times[0,a]$, defined by
\[\bm{w}\:x\mapsto(w_1(x),w_2(x)).\]

Note that 
\begin{align*}
\bm{w}(\alpha)&=0\times [0,a],
&
\bm{w}(\beta)&=[0,a]\times 0,
\\
\bm{w}(\alpha')&=a\times [0,a],
&
\bm{w}(\beta')&=[0,a]\times a,
\end{align*} 
Therefore, the composition $\bm{w}\circ\gamma$ is a degree 1 map 
\[\mathbb{S}^1\to \partial([0,a]\times[0,a]).\] 
It follows that if $h\:\DD\to \spc{X}$ shrinks $\gamma$, then there is a point $z\in\DD$ such that 
$\bm{w}\circ h(z)=(\tfrac a2,\tfrac a2)$.
Therefore $h(z)$ lies at distance at least $\tfrac a2$ from $\alpha$, $\beta$, $\alpha'$, $\beta'$
and therefore from $\gamma$.
Hence the claim $({*})$ follows.

\medskip

Coming back to the problem, let $\{W_i\}$ be an open covering of the real line with multiplicity $2$ and $\rad W_i<R$ for each $i$;
for example one may take $W_i=((i-\tfrac23)\cdot R,(i+\tfrac23)\cdot R)$.

Choose a point $p\in \spc{X}$.
Denote by $\{V_j\}$ the connected components of $\distfun_p^{-1}(W_i)$ for all $i$.
Note that $\{V_j\}$ is an open finite cover of $\spc{X}$ with multiplicity at most 2.
It remains to show that $\rad V_j<100\cdot R$ for each $j$.

\begin{wrapfigure}{o}{31 mm}
\vskip-2mm
\centering
\includegraphics{mppics/pic-1310}
\end{wrapfigure}

Aarguing by contradiction assume there is a pair of points  $x,y\in V_i$ 
such that $|x\z-y|_{\spc{X}}\ge 100\cdot R$.
Connect $x$ to $y$ with a curve $\tau$ in $V_j$.
Consider the closed curve $\sigma$ formed by $\tau$ and two geodesics $[px]$, $[py]$.


Note that $|p-x|>40$.
Therefore there is a point $m$ on $[px]$ such that $|m-x|=20$.

By the triangle inequality, the subsdivision of $\sigma$ into the arcs $[pm]$, $[mx]$, $\tau$ and $[yp]$ satisfy the conditions of the claim $({*})$ for $a=10\cdot R$.
Hence the statement follows.

\parit{The quasiconverse} does not hold.
As an example take a surface that looks like a long cylinder with two hats,
it is a smooth surface diffeomorphic to a sphere.
\begin{figure}[h!]
\vskip0mm
\centering
\includegraphics{mppics/pic-1315}
\end{figure}
Assuming the cylinder is thin, it has macroscopic dimension 1 at a given scale.
However a circle formed by a section of cylinder around its midpoint by a plane parallel to the base is a circle that cannot be contracted in its small neighborhood.

\parit{Sourse:} \cite[Appendix 1(E$_{2}$)]{gromov-1983}.

\parbf{\ref{ex:width=suprad(inv)},} \textit{``only if'' part.}
Suppose $\width_n\spc{X}<R$.
Consider a covering $\{V_1,\dots,V_k\}$ of $\spc{X}$ guaranteed by the definition of width.
Let $\spc{N}$ be its nerve and $\psi\:\spc{X}\to \spc{N}$ be the map provided by \ref{prop:space->nerve}.

Since the multiplicity of the covering is at most $n+1$, we ahve $\dim \spc{N}\le n$.

Note that if $x\in \spc{N}$ lies in a star of a vertex $v_i$,
then $\psi^{-1}\{x\}\z\subset V_i$;
in particular $\rad[\psi^{-1}\{x\}]<R$.

\parit{``If'' part.}
Choose $x\in \spc{N}$.
Since the inverse image $\psi^{-1}\{x\}$ is compact, $\psi$ is continuous, and $\rad[\psi^{-1}\{x\}]<R$,
there is a neighborhood $U\ni x$ such that the  $\rad[\psi^{-1}(U)]<R$.

Since $\spc{X}$ is compact,  there is a finite cover $\{U_i\}$ of $\spc{N}$ such that $\psi^{-1}(U_i)\subset\spc{X}$ has radius smaller than $R$ for each $i$.
Since $\spc{N}$ has dimension $n$, we can inscribe%
\footnote{Recall that a covering $\{W_i\}$ is inscribed in the covering $\{U_i\}$ if for every $W_i$ is a subset of some $U_j$.} 
in $\{U_i\}$ a finite open cover $\{W_i\}$ with multiplicity at most $n+1$.
It remains to observe that $V_i=\psi^{-1}(W_i)$ defines a finite open cover of $\spc{X}$ with radius less than $R$ and multiplicity at most $n+1$. 


\parbf{\ref{ex:1D-case}.}
Assume that $\spc{P}$ is connected.

Let us show that $\diam\spc{P}<R$.
If this is not the case, then there are points $p,q\in\spc{P}$ on distance $R$ from each other.
Let $\gamma$ be a geodesic from $p$ to $q$.
Clearly $\length\gamma\ge R$ and $\gamma$ lies in $\oBall(p,R)$ except for the endpoint $q$.
Therefore $\length[\oBall(p,R)_{\spc{P}}]\ge R$.
Since $\VolPro_{\spc{P}}(R)\z\ge \length[\oBall(p,R)_{\spc{P}}]$,
the latter contradicts $\VolPro_{\spc{P}}(R)<R$.

In general case, we get that each connected component of $\spc{P}$ has radius smaller that $R$.
Whence the width of $\spc{P}$ is smaller that $R$.

\parit{Second part.} Again, we can assume that $\spc{P}$ is connected.

The examples of line segment or a circle show that the constant $c=\tfrac12$ cannot be improved.
It remains to show that the inequality holds with $c=\tfrac12$.

Choose $p\in\spc{P}$ such that the value
\[\rho(p)=\max\set{\dist{p}{q}{\spc{P}}}{q\in\spc{P}}\]
is minimal.
Suppose $\rho(p)\ge\tfrac 12\cdot R$.
Observe that there is a point $x\in \spc{P}\backslash\{p\}$ that lies on any shortest path starting from $p$ and length $\ge\tfrac 12\cdot R$.
Otherwise for any $r\in(0,\tfrac 12\cdot R)$ there would be at least two points on distance $r$ from $p$;
by coarea inequality we get that the total length of $\spc{P}\cap \oBall(p,\tfrac 12\cdot R)$ is at least $R$ --- a contradiction.

Moving $p$ toward to $x$ reduce $\rho(p)$ which contradicts the choice of~$p$.

\parbf{\ref{ex:connected-sum-essential}.}
Suppose $M$ is an essential manifold and $N$ is arbitrary closed manifold.
Observe that shrinking $N$ to a point produces a map $f\:N\#M\to M$ of degree 1; that is, the fundamental class of $N\#M$ maps to the fundamental class of $M$.

Since $M$ is essential, there is an aspherical space $K$ and a map $\iota\:M\to K$ that sends fundamental class of $M$ to nonzero homology class in $K$.
From above, the composition $\iota\circ f\:N\#M\to K$ sends fundamental class of $N\#M$ to the same homology class in $K$.

\parit{Remark.} Note that we only used that there is a map $N\#M\to K$ of degree 1. If essential manifold is defined using homologies with integer coefficients, then existence of map of nonzero degree is sufficient.


%\chapter{Midterm}\label{chap:midterm}

An oral exam, Th, Feb 27 in class.

\bigskip

\noi 
One theoretical questions from the following list:

\begin{enumerate}
\item 
Semicontinuity of length.
\item
Length spaces and Hopf--Rinow theorem.
\item
Fréchet lemma and Kuratowski embedding.
\item
Hausdorff convergence and Blaschke selection theorem.
\item
Gromov--Hausdorff metric, why it is a metric, almost isometries.
\item
Uniformly totally bonded families and Gromov selection theorem.
\item
Ultralimits and ultrapower of spaces.
\item
Urysohn space.
\item
Injective spaces and injective envelop.
\end{enumerate}

\bigskip

\noi One exercise from the following list:
\\
\ref{ex:almost-min},
\ref{ex:non-contracting-map},
\ref{ex:compact=>complete},
\ref{ex:compact-length},
\\
\ref{ex:Huas-perimeter-area},
\\
\ref{pr:doubling},
\ref{pr:under},
\ref{ex:GH-SC},
\ref{ex:sphere-to-ball},
\\
\ref{ex:ultrapower}, 
\ref{ex:two-geodesics-in-ultrapower},
\ref{ex:lim(tree)},
\\
\ref{ex:geodesics-urysohn},
\ref{ex:sphere-in-urysohn},
\ref{ex:compact-extension},
\\
\ref{ex:+-c},
\ref{ex:ultrametric},
\ref{ex:injective-spaces},
\ref{ex:tripod+square},
\ref{ex:4-on-a-line}.

\bigskip

\noi One more problem for a perfect score.

%%%%%%%%%%%%%%%%%%%%%%%%%%%%
{\small\sloppy
\RequirePackage{snapshot}
\documentclass[twoside]{book}

\usepackage{lectures}
\usepackage[colorlinks=true,
citecolor=black,
linkcolor=black,
anchorcolor=black,
filecolor=black,
menucolor=black,
urlcolor=black,
pdftitle={Metric geometry on manifolds: two lectures},
pdfsubject={Geometry},
pdfauthor={Anton Petrunin}
]{hyperref}
\makeindex

\begin{document}
 
\title{Metric geometry on manifolds:
\\ two lectures}
\author{Anton Petrunin}
\date{}
\maketitle

We discuss Besikovitch inequality, width, and systole of manifolds.

We assume that students familiar with the smooth manifolds, degree of map, CW-complexes and related notions.

These are two final lectures of a graduate course given at Penn State, Spring 2020.
The complete lectures can be found on the authors website;
it includes an introduction to metric geometry \cite{petrunin2020pure}
and elements of Alexandrov geometry based on \cite{alexander-kapovitch-petrunin-2019}.

\thispagestyle{empty}
\tableofcontents
\thispagestyle{empty}

%%%%%%%%%%%%%%%%%%%%%%%%%%%%
%\include{hwa}
%%%%%%%%%%%%%%%%%%%%%%%%%%%%

\include{curvature-free}
\include{systole}
%\include{filling-rad}
%\include{examples}
%\include{AA}
\appendix
\chapter{Semisolutions}
\input{metrics-on-manifolds-sol}

%\include{tickets}
%%%%%%%%%%%%%%%%%%%%%%%%%%%%
{\small\sloppy
\input{metrics-on-manifolds.ind}

\printbibliography[heading=bibintoc]
\fussy
}


\end{document}


\printbibliography[heading=bibintoc]
\fussy
}


\end{document}


\printbibliography[heading=bibintoc]
\fussy
}


\end{document}


\def\emph{\textit}

\printbibliography[heading=bibintoc]
\fussy
}


\end{document}
