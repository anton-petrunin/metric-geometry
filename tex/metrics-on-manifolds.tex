\documentclass[twoside]{book}

\usepackage{lectures}
\usepackage[colorlinks=true,
citecolor=black,
linkcolor=black,
anchorcolor=black,
filecolor=black,
menucolor=black,
urlcolor=black,
pdftitle={Metric geometry on manifolds: two lectures},
pdfsubject={Geometry},
pdfauthor={Anton Petrunin}
]{hyperref}
\makeindex

\begin{document}
%\pagestyle{empty}
 
\title{Metric geometry on manifolds:
\\ two lectures}
\author{Anton Petrunin}
\date{}
\maketitle

We discuss Besicovitch inequality, width, and systole of manifolds.
Prerequisite includes  
basic measure theory, and
definitions of smooth manifolds,
degree of map, 
CW-complexes,
and related notions.

This is a polished version of two final lectures of a graduate course given at Penn State, Spring 2020.
The complete lectures can be found on the author's website;
it includes an introduction to metric geometry \cite{petrunin2020pure}
and elements of Alexandrov geometry based on \cite{alexander-kapovitch-petrunin-2019}.

\parbf{Acknowledgments.}
I want to thank Alexander Lytchak and Alexander Nabutovsky for help.

\thispagestyle{empty}
\tableofcontents
\thispagestyle{empty}

%%%%%%%%%%%%%%%%%%%%%%%%%%%%
%\addtocounter{chapter}{-1}
\chapter{Homework assignments}


It is better to think about all the problems, but you do not have to solve \emph{all} of them.
If a problem is solved, you do not have to write its solutions, but try sketch it.

\section{Due Tue Jan 21}

Exercises: \sout{\ref{ex:almost-min},} \ref{ex:non-contracting-map}, \ref{ex:no-geod}, \sout{\ref{ex:compact=>complete},} \ref{exercise from BH}, \ref{ex:Hausdorff-bry}.

\section{Due Tue Jan 28}

Exercises: \ref{ex:almost-min},  \ref{ex:compact=>complete}, \ref{ex:Huas-perimeter-area}, \ref{ex:GH-po}, \ref{pr:doubling}, \ref{pr:under:if}.

\section{Due Tue Feb 4}
Exercises: 
\ref{ex:compact-length}, 
\ref{pr:under:only-if}, 
\sout{\ref{ex:GH-SC},}
\sout{\ref{ex:sphere-to-ball},}
\ref{ex:ultrapower}, 
\ref{ex:two-geodesics-in-ultrapower}.

\section{Due Tue Feb 11}

Finish exercises \ref{ex:compact-length} , \ref{pr:under:only-if}, \ref{ex:GH-SC}, \ref{ex:sphere-to-ball}.

\noindent
Exercises: \ref{ex:lim(tree)}, \ref{ex:Asym(Lob)}, \ref{ex:geodesics-urysohn}, \ref{ex:sphere-in-urysohn}.

\section{Due Tue Feb 18}

Exercises: \ref{ex:compact-extension}, \ref{ex:+-c}, \ref{ex:ultrametric}, \ref{ex:injective-spaces}, \ref{ex:tripod+square}, \ref{ex:4-on-a-line}.

\noindent Write down a solution of at least one of the exercises.

\section{Due Tue Feb 25}

Finish Exercise \ref{ex:tripod+square:square}.
Prepare questions for review on Tuesday.

\section{Due Tue Mar 3}

Exercises: \ref{ex:sba-2+2-short}, \ref{ex:(3+1)-expanding}, \ref{ex:CAT+CBB}, \ref{ex:product-CBB}, \sout{\ref{ex:CBB-geodesic},} \ref{ex:fat-triangle}.

\noindent Write down a solution of at least one of the exercises.

\section{Due Tue Mar 17}

Exercises: \ref{ex:tringle-inq-angles},
\ref{ex:CBB-geodesic},
\ref{ex:convex-dist},
\ref{ex:reshetnyak-doubling},
\ref{ex:supporting-planes},
\ref{ex:centrally-simmetric-walls}.

\noindent Write down a solution of at least one of the exercises.

\section{Due Tue Mar 24}

Exercises: 
\ref{ex:contractible},
\ref{ex:convex-nbhd},
\ref{ex:closest-point},
\ref{cor:balls:dim=1},
\ref{ex:null-homotopic},
\ref{ex:branching-cover}.

 Write down as many solutions as you can; email then to Zetian Yan (zxy5156) + cc to me (aqp6).

Each working day I will check email before 15:00 and will appear online if you asked me (it is easy for me --- do not hesitate to ask).
We will meet regular hours online (as we did before).

%%%%%%%%%%%%%%%%%%%%%%%%%%%%

\chapter{Volume bounds} 


\section{Riemannian metrics}

We are going to consider mostly Riemannian spaces;
that is smooth manifolds with metric defined by a metric tensor.
These are specially nice length metrics on manifolds.
However most of the statements we are going to discuss have counterpart for general length metrics on manifolds.

Let $M$ be a smooth manifold.
A \emph{metric tensor} on $M$ is a choice of positive definite quadratic forms $g_p$ on each tangent space $\T_pM$ that depends smoothly on the point $p$.
That is, if we fix a local coordinates on $M$ and write $g$ in this coordinates, then each component of $g$ is a smooth function. 

A Riemannian manifold is a smooth manifold $M$ with a choice \emph{metric tensor} $g$ on it.

The metric tensor $g$ can be used to define length of curves and volume of regions in $M$.

\parbf{Lengths and distances.}
If $\gamma\:[a,b]\to M$ is a piecewise smooth curve then 
\[\length_g\gamma=\int_a^b\sqrt{g(\gamma'(t),\gamma'(t))}\cdot dt.\]
Further we can define a metric on $M$ as least lower bound to lengths of piecewise smooth curves connecting two given points;
the described distance between points $x$ and $y$ will be denoted by $\dist{x}{y}{g}$ or $\distfun_x(y)_g$.
The distance function from a point $x$ will be denoted by $(\distfun_x)_g$ or $\distfun_x$ if the choice of $g$ is evident.

The following claim requires a proof, but we will assume that it is obvious.

\begin{thm}{Claim}
Let $(M,g)$ be a Riemannian manifold.
Then the metric $(x,y)\mapsto \dist{x}{y}{g}$ defines a length metric. Moreover this metric completely determines the metric tensor $g$.
\end{thm}

\parbf{Volume.}
If a region $R$ is covered by one chart $\iota\:U\to M$,
then its volume can be defined as an integral 
\[\vol R
\df
\int_{\iota^{-1}(R)}\sqrt{\det{g}}.\]
In the general case we subdivide $R$ into (a countable collection of) regions $R_1,R_2\dots$ and define
\[\vol R\df \vol R_1+\vol R_2+\dots\]

\section{Besikovitch inequality}

\begin{thm}{Theorem}\label{thm:besikovitch}
Let $g$ be a metric tensor on a unit $n$-dimensional cube $\square^n$.
Suppose that the $g$-distances between the opposite faces of $\square^n$ are at leat $1$; that is, any piecewise smooth curve that connects opposite faces has $g$-length at least $1$.
Then $\vol(\square^n, g)\ge 1$.
\end{thm}

\parit{Proof.}
We will consider the case $n=2$; the other cases are proved the same way.

Denote by $A$, $A'$, and $B$, $B'$ the opposite faces of the square~$\square$.
Consider two function 
\begin{align*}
f_A(x)&\df\min\{\,\distfun_A(x)_g,1\,\},
\\
f_B(x)&\df\min\{\,\distfun_B(x)_g,1\,\}.
\end{align*}
Define $f\:\square\to\square$ as a map with coordinate funcions $f_A$ and $f_B$;
that is, $f(x)\df(f_A(x), f_B(x))$.

Observe that $f$ maps each face to itself.
Indeed, 
\[x\in A \quad\Longrightarrow\quad \distfun_A(x)_g=0 \quad\Longrightarrow\quad f_A(x)=0 \quad\Longrightarrow\quad f(x)\in A.\]
Similarly if $x\in B$, then $f(x)\in B$.
Further, 
\[x\in A'
\quad\Longrightarrow\quad 
\distfun_A(x)_g\ge 1 
\quad\Longrightarrow\quad 
f_A(x)=1 
\quad\Longrightarrow\quad 
f(x)\in A'.\]
Similarly if $x\in B'$, then $f(x)\in B'$.

Therefore 
\[f_t(x)= t\cdot x + (1-t)\cdot f(x)\]
defines a homotopy of maps of pair of spaces $(\square,\partial \square)$ from $f$ to the identity map.
It follows that degree of $f$ is $1$; that is, $f$ sends the fundamental class of $(\square,\partial \square)$ to itself.
In particular $f$ is onto.

Suppose that Jacobian  matrix $\Jac_pf$ of $f$ is defined at $p\in \square$.
Choose an orthonormal basis in $\T_p$ with respect to $g$ and the standard basis in the target $\square$.
Observe that the differentials $d_pf_A$ and $d_pf_B$ written in these basises are the rows of $\Jac_pf$.
Evidently $|d_pf_A|\le 1$ and $|d_pf_B|\le 1$.
Since the determinant of a matrix is the volume of the parallelepiped spanned on its rows, we get 
\[|\det(\Jac_pf)|\le |d_pf_A|\cdot|d_pf_B|\le 1.\]
Since $f\:\square\to\square$ is a Lipschitz onto map, the \emph{area formula} implies that 
\[\vol(\square,g)\ge \vol\square=1.\]
\qedsf

The following generalization can be proved along the same lines.

\begin{thm}{Theorem}\label{thm:besikovitch+}
Let $(M,g)$ be Riemannian manifold and its boundary admits a homeomorphism $\partial\square^n\to\partial M$. 
Suppose $d_1,\dots, d_n$ the distances between the the images of pairs of opposite faces of $\square^n$ in $\partial M$.
Then 
\[\vol(M,g)\ge d_1\cdots d_n.\]
\end{thm}

\begin{thm}{Exercise}\label{ex:besikovitch=}
Suppose that we have equality in \ref{thm:besikovitch}.
Show that $(\square^n,g)$ is isometric to $\square^n$.
\end{thm}

\begin{thm}{Exercise}\label{ex:hexagon}
Suppose $g$ is a metric tensor on a regular hexagon $\varhexagon
   $ such that $g$-distances between the opposite sides are at least $1$.
Is there a positive lower bound on $\area(\varhexagon,g)$?
\end{thm}

\begin{thm}{Exercise}\label{ex:gadograph}
Let $V$ be a compact set in $\EE^d$ bounded by a hypersurface $\Sigma$.
Suppose $g$ is a Riemannian metric on $V$ such that 
\[\dist{p}{q}{g}\ge\dist{p}{q}{\EE^d}\]
for any two points $p,q\in \Sigma$.
Show that
\[\vol(V,g)\ge \vol(V)_{\EE^d}.\]
 
\end{thm}

\begin{thm}{Exercise}\label{ex:involution-of-sphere}
Suppose that sphere with Riemannian matric $(\mathbb{S}^2,g)$ admits an involution $\iota$ such that $\dist{x}{\iota(x)}{g}\ge 1$.

Show that $\area(\mathbb{S}^2,g)\ge \tfrac1{1000}$;
try to show that $\area(\mathbb{S}^2,g)\ge \tfrac12$ or $\area(\mathbb{S}^2,g)\ge 1$.
\end{thm}

Christopher Croke conjectured that the optimal bound for this exercise is $\tfrac4\pi$ and the round sphere is the only space that achieves this \cite[see Conjecture 0.3 in][]{croke}.

\begin{thm}{Advanced exercise}\label{ex:involution-of-3sphere}
Construct a metric $g$ on $\mathbb{S}^3$ with arbitrary small $\vol(\mathbb{S}^3,g)$ and such that it admits an involution $\iota$ such that $\dist{x}{\iota(x)}{g}\ge 1$.
\end{thm}

\section{Systolic inequlaity}

Let $\spc{M}$ be a compact Riemannian manifold.
The \emph{systole} of $\spc{M}$ (brifly $\sys\spc{M}$) is defined to be the least length of a noncontractible closed curve in $\spc{M}$.

Let $\Lambda$ be a set of smooth closed $n$-dimensional manifolds.
We say that a systolic inequality holds for $\Lambda$ if there is a constant $c$ such that for any $M\in \Lambda$ and any metric tenor $g$ on $M$ we have
\[[\sys(M,g)]^n\le c\cdot \vol(M,g).\]

\begin{thm}{Exercise}\label{ex:sysT2}
Use \ref{thm:besikovitch} to show that systolic inequality holds for the 2-torus $\TT^2$.
\end{thm}

\begin{thm}{Exercise}\label{ex:sysRP2}
Use \ref{thm:besikovitch} to show that systolic inequality holds for the real projective palane $\RP^2$.
\end{thm}

\begin{thm}{Exercise}\label{ex:sysSg}
Use \ref{thm:besikovitch+} to show that systolic inequality holds for the set of all closed surfaces of positive genus.
\end{thm}

\parbf{Remarks.}
The optimal constants in the systolic inequality are known in the following three cases:
\begin{itemize}
\item For real projective plane $\RP^2$ the constant is $\tfrac\pi2$ --- the equality holds for a quotient of a round sphere by isometric involution. The statement was prove by Pao Ming Pu \cite{pu}.\label{page:pu}
\item For torus $\TT^2$ the constant is $\tfrac2{\sqrt{3}}$ --- the equality holds for a flat torus obtained from a regular hexagon by identifying opposite sides; this is the so called \emph{Loewner's torus inequality}.
\item For the Klein bottle $\RP^2\#\RP^2$  the constant is $\tfrac\pi{2\cdot\sqrt2}$ --- the equality holds for certain nonsmooth metrics \cite{bavard}.
\end{itemize}
The proofs of these results use the so called \emph{uniformization theorem}   available in the 2-dimensional case only.
These proofs are beautiful, but they too far from metric geometry.
A good survey on the subject is written by Christopher Croke and Mikhail Katz \cite{croke-katz}.

\begin{thm}{Exercise}\label{ex:sysS2xS1}
Show that systolic inequality does \emph{not} hold for $\mathbb{S}^2\times\mathbb{S}^1$.
\end{thm}


\begin{thm}{Therorem}\label{thm:sys(torus)}
Systolic intequality holds for the $n$-dimensional torus $\TT^n$. 
\end{thm}

The proof of this theorem and its generalization will take most of the remaining lectures.
In the following section we introduce a key notion in the proof.

\section{Filling radius}

The following definition was introduced by Mikhael Gromov \cite{gromov-1983}.

Let $\spc{M}$ be a closed $n$-dimensional Reimannian manifold.
Applying Kuratowski embedding (\ref{lem:kuratowski}) $x\mapsto \distfun_x$, we may think that $\spc{M}$ as a subset of $\ell^\infty(\spc{M})$ --- the space of functions on $\spc{M}$ equipped with the metric induced by the sup-norm.

Define the \emph{filling radius} of $\spc{M}$ (briefly $\FillRad\spc{M}$) as the least upper bound on values $r>0$ such that $\spc{M}$ bounds in its $r$-neighborhood in $\ell^\infty(\spc{M})$.
In other words, if $\iota_r$ denotes inclusion of $\spc{M}$ in its $r$-neighborhood $B_r(\spc{M})\subset \ell^\infty(\spc{M})$,
then 
\[\FillRad\spc{M}\df\inf\set{r>0}{(\iota_r)_*[\spc{M}]=0\in H_n(B_r(\spc{M}))},\]
where $[\spc{M}]$ denotes the fundamental class of $\spc{M}$.

We assume that the homologies are taken with coefficients in $\ZZ_2$.
In this case $[\spc{M}]\ne0\in H_n(\spc{M})$.
If we choose coefficients $\ZZ$, then it does not hold for nonorientable manifolds.


\begin{thm}{Exercise}\label{ex:fillrad<diam/2}
Show that the inequality
\[\FillRad \spc{M}\le \tfrac12\cdot\diam \spc{M}\]
holds for any compact Riemannian manifold $\spc{M}$.
\end{thm}

\parbf{Remark.}
The optimal bound for the above exercise was found by Mikhail Katz \cite{katz}.
Namely he proved that
\[\FillRad \spc{M}\le \tfrac13\cdot\diam \spc{M}\]
and equality holds if $\spc{M}$ is real projective space with canonical metric.
The proof is beautiful, elementary, and very readable.

\medskip

The following theorem is the main ingredient in the proof of \ref{thm:sys(torus)}.
This theorem will be the main subject of the following lecture.

\begin{thm}{Theorem}\label{thm:FillRad<vol}
Given an integer $n>0$, there is a constant $c(n)$ such that inequality
\[(\FillRad \spc{M})^n\le c(n)\cdot \vol \spc{M}\]
holds for any compact $n$-dimensional Riemannian manifold $\spc{M}$.
\end{thm}

In the following section we show why this theorem is related to \ref{thm:FillRad<vol}.

\section{Systole and filling radius}

\begin{thm}{Theorem}\label{thm:sys<FillRad}
Suppose $\spc{T}= (\TT^n,g)$ is a Riemnnian manifold on $n$-dimensional torus $\TT^n$.
Then 
\[\sys\spc{T}\le 6 \cdot \FillRad \spc{T}.\]
\end{thm}

Note that \ref{thm:sys<FillRad} and \ref{thm:FillRad<vol}  imply \ref{thm:sys(torus)}.

\parit{Proof.}
As usual we consider $\spc{T}$ as a subspace in $\ell^\infty(\spc{T})$.

Set $s=\sys\spc{T}$ and $\FillRad\spc{T}=r$.
Arguing by contradiction, assume $6\cdot r< s$;
so $\eps=\tfrac1{100}\cdot(s-6\cdot r)>0$.

Choose a simplicial complex $\Sigma$ and a map $\sigma\:\Sigma\to \ell^\infty(\spc{T})$ such that the restriction $\sigma|_{\partial\Sigma}$
represents the fundamental class $[\spc{T}]$ of $\spc{T}$
and $\sigma(\Sigma)\subset B_{r+\eps}(\spc{T})$.


Passing to barycentric subdivision few times, we may assume that the $\sigma$-image of any simplex in $\Sigma$ has diameter less than $\eps$.
We may perturb the map slightly to ensure that each edge $e$ of $\Sigma$ is mapped to a geodesic and still $\sigma|_{\partial\Sigma}$
represents the fundamental class $[\spc{T}]$ of $\spc{T}$.

Let us construct a continuous map
$f\:\Sigma\to  \spc{T}$ which agrees with $\sigma$ on $\partial \Sigma$.
Once it is done we get that $[\spc{T}]=0\in H_n(\spc{T})$ --- a contradiction.

Set $f(x)=\sigma(x)$ for every $x\in \partial \Sigma$;
on the remaining part of $\Sigma$ we will construct $f$ recurcevely on the skeletons $\Sigma^0$, $\Sigma^1$, $\Sigma^2$ and so on.

For every vertex $v$, set $f(v)$ to be the closest point in $\spc{T}$ to $\sigma(v)$.
Note that if $v\in\partial\Sigma$, then $f(v)=\sigma(v)$.
This way we defined $f$ on $\Sigma^0$.

Let $e$ be an edge in $\Sigma$ between vertexes $v$ and $w$.
Note that 
\begin{align*}
\dist{f(v)}{f(w)}{}
&\le\dist{f(v)}{\sigma(v)}{}
+\dist{\sigma(v)}{\sigma(w)}{}
+\dist{\sigma(w)}{f(w)}{}\le
\\
&\le (r+\eps)+\eps +(r+\eps)<
\\
&<\tfrac s3.
\end{align*}
Map $e$ to a shortest path $[f(v)\,f(w)]$ in $\spc{T}$;
if $e$ is an edge in $\partial \Sigma$ then no need to change $f$ on it.
This extends $f$ to $\Sigma^1$ such that each edge is mapped to a geodesic of length less that $\tfrac s3$.

Now for each triangle $uvw$ in $\Sigma$, the closed curve formed by $f$-images of its sides has length less than $s$.
That is, it is shorter than any noncontractible closed curve,
and therefore it is null-homotopic in $\spc{T}$.
Hence we can extend $f$ to the $\Sigma^2$.

Finally, since $\spc{T}$ is aspherical, there is no obstruction to extending $f$ to the rest of $\Sigma$.
\qeds

Observe that we use only that $\spc{T}$ is aspherical closed manifold;
this statement will be generalized yet further.

\begin{thm}{Exercise}\label{ex:fillrad-inj}
Modify the proof of \ref{thm:sys<FillRad} to prove the following:

Suppose that $\spc{M}$ is a closed $n$-dimensional Reimannian manifold with \emph{injectivity radius} at least $r$; that is, if $\dist{p}{q}{\spc{M}}<r$, then there is geodesic $[pq]_{\spc{M}}$ is uniquely defined.
Show that
\[\FillRad\spc{M}\ge \tfrac{r}{n+1}.\]
 
\end{thm}

Note that this exercise together with bound on filling radius in \ref{thm:FillRad<vol} imply that lower bound on injectivity radius implies a lower bound on volume.


\chapter{Width}

This lecture is based on a paper of Alexander Nabutovsky \cite{nabutovsky}.

\section{Nerves and partition of unity}

Let $\{V_1,\dots,V_k\}$ be a finite open cover of a compact metric space $\spc{X}$.
Consider an abstract simplicial complex $\spc{N}$, with one vertex $v_i$ for each set $V_i$ such that a simplex with vertexes $v_{i_1},\dots, v_{i_m}$ is included in $\spc{N}$ if 
the intersection $V_{i_1}\cap\dots\cap V_{i_m}$ is nonempty.
The obtained simplicial complex $\spc{N}$ called the \index{nerve}\emph{nerve of the covering $\{V_i\}$}.

Note that $\spc{N}$ is a finite simplicial complex;
it is a subcomplex of a simplex with the vertixes $\{v_1,\dots,v_k\}$.
The nerve $\spc{N}$ has dimension at most $n$ if and only if the covering $\{V_1,\dots,V_k\}$ has multiplicity is at most $n+1$;
that is, any point $x\in\spc{X}$ belongs to
at most $n+1$ sets of the covering.

\begin{thm}{Proposition}\label{thm:part-unit}
 Let $\{V_1,\dots,V_k\}$ be a finite open covering of a compact metric space ${\spc{X}}$.
Then there are Lipschitz functions $\psi_i\:{\spc{X}}\z\to[0,1]$ such that
if $\psi_i(x)>0$ then $x\in V_i$ and
$$\sum_i\psi_i(x)=1$$
for any $x\in {\spc{X}}$.
\end{thm}

\parit{Proof.}
Consider functions $\phi_i\:{\spc{X}}\to\RR$ defined as
$$\phi_i(x)=\distfun_{({\spc{X}}\backslash V_i)} x.$$
Note $\phi_i$ is $1$-Lipschitz
for any $i$
and $\phi_i(x)>0$ if and only if $x\in V_i$.
Since $\{V_i\}$ is a covering, we have that
$$\sum_i\phi_i(x)>0\ \ \text{for any}\ \ x\in {\spc{X}}.$$

Set 
$$\psi_k(x)=\frac{\phi_k(x)}{\sum_i\phi_i(x)}.$$
Observe that by construction the functions $\psi_i$ meet the conditions in the proposition.
\qedsf

A collection of functions $\{\psi_i\}$ that meets the conditions in \ref{thm:part-unit} is called 
a \index{partition of unity}\emph{partition of unity subordinate to the open covering} $\{V_1,\dots,V_k\}$.

Suppose $\{\psi_i\}$ is  
a partition of unity subordinate to the open covering $\{V_1,\dots,V_k\}$.
Note that for any point $x\in {\spc{X}}$, the set
$$\set{v_i}{\psi_i(x)>0}$$
describe vertexes of a simplex in the nerve.
Therefore 
$$\psi\:x\mapsto \psi_1(x)\cdot v_1+\psi_2(x)\cdot v_2+\dots+\psi_k(x)\cdot v_n.$$
describes a Lipschitz map from ${\spc{X}}$ to the nerve $\spc{N}$ of $\{V_i\}$;
here the point $x$ is mapped to the point with barycentric coordinates $\psi_i(x)$.
In other words we proved the following:

\begin{thm}{Proposition}\label{prop:space->nerve}
Let $\spc{N}$ be a nerve of an open covering $\{V_1,\z\dots,V_k\}$ of a compact metric space $\spc{X}$.
Denote by $v_i$ the vertex of $\spc{N}$ that corresponds to $V_i$.

Then there is a Lipschitz map $\psi\:\spc{X}\to\spc{N}$ such that $\psi(V_i)\z\subset\Star_{v_i}$ for every $i$;
that is, for any $x\in V_i$ the point $\psi(x)$ lies the interior of some simplex with vertex $v_i$.
\end{thm}


\section{Width}

Suppose $A$ is a subset of a metric space $\spc{X}$.
The radius of $A$ (briefly $\rad A$) is defined as the least upper bound on the values $R>0$ such that $\oBall(x,R)\supset A$ for some $x\in \spc{X}$.

\begin{thm}{Definition}\label{def:width}
Let $\spc{X}$ be a metric space.
The $n$-th width of $\spc{X}$ (briefly $\width_n\spc{X}$) is defined as the least upper bound on values $R>0$ such that $\spc{X}$ admits a finite open covering $\{V_i\}$ with multiplicity at most $n+1$ and $\rad V_i< R$ for each $i$.
\end{thm}

\parit{Remarks.}
\begin{itemize}
\item Observe that 
\[\width_0\spc{X}\ge\width_1\spc{X}\ge\dots\]
for any compact matric space $\spc{X}$.
Moreover, if $\spc{X}$ is connected, then 
\[\width_0\spc{X}=\rad\spc{X}.\]
\item 
Usually width is defined using diameter instead of radius, but the result differ at most twice.
Namely if $r$ is the radius-width and $d$ --- diameter-width for the same $n$, then 
$r\le d\le 2\cdot r$.

\item Note that \index{Lebesgue covering dimension}\emph{Lebesgue covering dimension} of $\spc{X}$ can be defined as the least number $n$ such that $\width_n\spc{X}=0$.
Another closely related notion is the so called \index{macroscopic dimesion on scale $R$}\emph{macroscopic dimesion on scale $R$};
it is defined as the  least number $n$ such that $\width_n\spc{X}<R$.
\end{itemize}

\begin{thm}{Exercise}\label{ex:macrodimension}
Suppose $\spc{X}$ is a compact metric space such that any closed curve $\gamma$ in $\spc{X}$ can be contracted in its $R$-neighborhood.
Show that $\spc{X}$ has macroscopic dimension at most 1 on scale $100\cdot R$.

What about quasiconverse? That is, suppose a simply connected compact metric space $\spc{X}$ has macroscopic dimension at most 1 on scale $R$, is it true that any closed curve $\gamma$ in $\spc{X}$ can be contracted in its $100\cdot R$-neighborhood?
\end{thm}


The following proposition provides an equivalent definition;
we will not use it, but it provides a good reason for the name \index{width}\emph{width}.

\begin{thm}{Proposition}\label{prop:width=suprad(inv)}
Suppose $\spc{X}$ is a compact metric space.
Then $\width_n\spc{X}<R$ if and only if there is a finite $n$-dimensional simplicial complex $\spc{N}$ and a continuous map $\psi\:\spc{X}\to \spc{N}$
such that $\rad[\psi^{-1}(s)]\z<R$
for any $s\in \spc{N}$.
\end{thm}

\parit{Proof; ``only if'' part.}
Suppose $\width_n\spc{X}<R$.
Consider a covering $\{V_1,\dots,V_k\}$ of $\spc{X}$ guaranteed by the definition of width.
Let $\spc{N}$ be its nerve and $\psi\:\spc{X}\to \spc{N}$ be the map provided by \ref{prop:space->nerve}.

Since the multiplicity of the covering is at most $n+1$, we ahve $\dim \spc{N}\le n$.

Note that if $x\in \spc{N}$ lies in a star of a vertex $v_i$,
then $\psi^{-1}\{x\}\z\subset V_i$;
in particular $\rad[\psi^{-1}\{x\}]<R$.

\parit{``If'' part.}
Choose $x\in \spc{N}$.
Since the inverse image $\psi^{-1}\{x\}$ is compact, $\psi$ is continuous, and $\rad[\psi^{-1}\{x\}]<R$,
there is a neighborhood $U\ni x$ such that the  $\rad[\psi^{-1}(U)]<R$.

Since $\spc{X}$ is compact,  there is a finite cover $\{U_i\}$ of $\spc{N}$ such that $\psi^{-1}(U_i)\subset\spc{X}$ has radius smaller than $R$ for each $i$.
Since $\spc{N}$ has dimension $n$, we can inscribe%
\footnote{Recall that a covering $\{W_i\}$ is inscribed in the covering $\{U_i\}$ if for every $W_i$ is a subset of some $U_j$.} 
in $\{U_i\}$ a finite open cover $\{W_i\}$ with multiplicity at most $n+1$.
It remains to observe that $V_i=\psi^{-1}(W_i)$ defines a finite open cover of $\spc{X}$ with radius less than $R$ and multiplicity at most $n+1$. 
\qeds

\section{Riemannian polyhedrons}

A \index{Riemannian polyhedron}\emph{Riemannian polyhedron} is defined as a finite simplicial complex with a metric tensor on each simplex such that the restriction of the metric on each simplex to a subsymplex coinsides with the metric on the subsmplex.
The dimension of Riemannian polyhedron is defined as the largest dimension it its triangulation.
For Riemannian polhedron one can define length of curves and volume the same way as for Riemannian manifolds.


Further we will apply the notion of width to compact Riemannian polyhedrons.
If $\spc{P}$ is an $n$-dimensional compact Riemannian polyhedron, then 
we suppose that
\[\width\spc{P}\df\width_{n-1}\spc{P}.\]

Let $\spc{P}$ be an $n$-dimensional Riemnnian polyhedron.
Let us define \index{volume profile}\emph{volume profile} of $\spc{P}$ as a function 
returning volume of largest $r$-ball in $\spc{P}$;
that is, $\VolPro_{\spc{P}}\:\RR_+\to\RR_+$ is defined by 
\[\VolPro_{\spc{P}}(r)\df \sup\set{\vol_n \oBall(p,r)}{p\in\spc{P}}.\]
Note that $\VolPro_{\spc{P}}$ is a nondecreasing function and $\VolPro_{\spc{P}}(r)\z\to\vol_n\spc{P}$ as $r\to\infty$.

Note that if $\spc{P}$ is a 1-dimensional connected Riemannian polyhedron, then 
\[\width\spc{P}=\width_0\spc{P}=\rad\spc{P}.\]

\begin{thm}{Exercise}\label{ex:1D-case}
Suppose $\spc{P}$ be a 1-dimensional Riemannian polyhedron.
Suppose $\VolPro_{\spc{P}}(R)<R$ for some $R>0$.
Show that 
\[\width \spc{P}<R.\]
Try to show that $c=\tfrac 12$ is the optimal constant such that 
\[\width \spc{P}<c\cdot R.\]
\end{thm}

\section{Volume profile bounds width}

The following theorem and its corollary is the main goal of this lecture.

\begin{thm}{Theorem}\label{thm:width<volpro}
Let $\spc{P}$ be an $n$-dimensional Reimannian polyhedron. 
If the inequality 
\[R> n\cdot \sqrt[n]{\VolPro_{\spc{P}}(R)}\]
holds for {}\emph{some} $R>0$, then 
\[\width\spc{P}\le  R.\]
\end{thm}

Since $\VolPro_{\spc{P}}(r)\le \vol\spc{P}$ for any $r$,
we get the following:

\begin{thm}{Corollary}\label{thm:width<vol}
For any $n$-dimensional Reimannian polyhedron $\spc{P}$, we have
\[\width\spc{P}\le n\cdot \sqrt[n]{\vol\spc{P}}.\]

\end{thm}

In the proof of \ref{thm:width<volpro}, we will use the following three technical statements,
the proofs are omitted, but they are not hard. 

\begin{thm}{Smoothing procedure}
Let $\spc{P}$ be a Reimannian polyhedron and $f\:\spc{P}\to \RR$ be a 1-Lipschitz function.
Then for any $\delta>0$ there is a  1-Lipschitz function $\tilde f\:\spc{P}\to \RR$ that is smooth on each simplex of the triangulation and $\delta$-close to $f$.
\end{thm}

\begin{thm}{Sard's theorem}
Let $\spc{P}$ be an $n$-dimensional Reimannian polyhedron and $f\:\spc{P}\to \RR$ be a function that is smooth on each simplex.
Then for almost all values $a$, each component of the inverse image $f^{-1}\{a\}$ equipped with the induced metric is a Reimannian polyhedron of dimension at most $n-1$.
\end{thm}


\begin{thm}{Coarea inequality}
Let $\spc{P}$ be an $n$-dimensional Reimannian polyhedron and $f\:\spc{P}\to \RR$ be a 1-Lipschitz function that is smooth on each simplex.
Set $V=\vol_n (f^{-1}[a,b])$.
Then 
\[\int_a^b\vol_{n-1}(f^{-1}\{x\})\cdot dx\ge V .\]
In particular there is a subset of positive measure $A\subset [a,b]$ such that the inequality 
\[\vol_{n-1}(f^{-1}\{x\})\ge \frac V{b-a}\]
holds for any $x\in A$.
\end{thm}

\begin{thm}{Definition}
Let $\spc{P}$ be an $n$-dimensional Riemannian polyhedron.
An $(n-1)$-dimensional subpolyhedron $\spc{Q}\subset\spc{P}$ is called \index{separating subpolyhedron}\emph{$R$-separating} if $\rad U<R$ for each connected component $U$ of the complement $\spc{P}\backslash \spc{Q}$.
\end{thm}



\begin{thm}{Lemma}\label{lem:separating}
Let $\spc{P}$ be an $n$-dimensional Riemannian polyhedron.
Then given $R>0$ and $\eps>0$ there is a $R$-separating subpolyhedron $\spc{Q}\subset\spc{P}$ such that for any $r_0<r_1\le R$ we have
\[\VolPro_{\spc{Q}}(r_0)< \tfrac1{r_1-r_0}\cdot \VolPro_{\spc{P}}(r_1)+\eps.\]

\end{thm}

\parit{Proof.}
Choose a small $\delta>0$.
Applying the smoothing procedure, we can exchange each distance function $\distfun_p$ on $\spc{P}$ by $\delta$-close smooth 1-Lipschitz function, which will be denoted by $\widetilde \distfun_p$.

By Sard's theorem, almost all level sets 
\[\tilde S_c(p)=\set{x\in \spc{P}}{\widetilde \distfun_p(x)=c}\]
are smooth Riemannian polyhedrons of dimension at most $n-1$.
Since $\delta$ is small, the coarea inequality implies that 
for we can choose $c\z\in(r_0+\delta, r_1-\delta)$ such that $\tilde S_c(p)$ is a subpolyhedron and 
\begin{align*}
\vol_{n-1}\tilde S_c(p)&\le \tfrac1{r_1-r_0-2\cdot\delta}\cdot\vol_n[\oBall(p,r_1)]<
\\
&<\tfrac1{r_1-r_0}\cdot \VolPro_{\spc{P}}(r_1)+\tfrac\eps2.
\end{align*}

Suppose $\spc{Q}$ is an $R$-separating subpolyhedron in $\spc{P}$ with almost minimal volume, say its volume is at most $\tfrac\eps2$-far from the greatest lower bound.
Note that cutting from $\spc{Q}$ everything inside $\tilde S_c(p)$ and adding $\tilde S_c(p)$ keeps it to be $R$-separating subpolyhedron.
Since $\spc{Q}$ has almost minimal volume, we have
\[\vol_{n-1}[\spc{Q}\cap \oBall(p,r_0)_{\spc{P}}]-\tfrac\eps2\le \vol_{n-1}S_c(p).\]
Therefore 
\[\vol_{n-1}[\spc{Q}\cap \oBall(p,r_0)_{\spc{P}}]\le\tfrac1{r_1-r_0}\cdot \VolPro_{\spc{P}}(r_1)+\eps\eqlbl{eq:volQ<ProP}\]
Recall that $\spc{Q}$ is equipped with the induced length metric;
therefore $\dist{p}{q}{\spc{Q}}\ge \dist{p}{q}{\spc{P}}$ for any $p,q\in \spc{Q}$;
in particular, 
\[\oBall(p,r_0)_{\spc{Q}}\subset \spc{Q}\cap \oBall(p,r_0)_{\spc{P}}\]
for any $p\in \spc{Q}$ and $r\ge 0$.
Hence \ref{eq:volQ<ProP} implies the lemma.
\qeds

\begin{thm}{Lemma}\label{lem:separating-width}
Let $\spc{Q}$ be an $R$-separating subpolyhedron in an $n$-dimensional Riemannian polyhedron $\spc{P}$.
Suppose $\width\spc{Q}\le R$.
Then $\width\spc{P}\le R$
\end{thm}

\parit{Proof.}
Start with an open covering $\{V_1,\dots,V_k\}$ of $\spc{Q}$ of multiplicity $\le n$ with radiuses of the sets in the intrinsic metric $\le R$.

Note that $\{V_1,\dots,V_k\}$ can be converted into an an open covering of
a small neighbourhood of $\spc{Q}$ in $\spc{P}$ without increasing the multiplicity.
This is can be done by setting 
\[V_i'=\bigcup_{x\in V_i}\oBall(x,r_x),\]
where $r_x=\tfrac1{10}\cdot\inf\set{\dist{x}{y}{}}{y\in \spc{Q}\backslash V_i}$.

Adding to  $\{V_i'\}$ all the components of $\spc{P}\backslash \spc{Q}$,
we increase the multiplicity by at most 1 and obtain a covering of $\spc{P}$.
The statement follows since $\dim \spc{P}= \dim \spc{Q}\z+1$.
\qeds

\parit{Proof of \ref{thm:width<volpro}.}
We apply induction on the dimension $n=\dim\spc{P}$.
The base case $n=1$ is given in \ref{ex:1D-case}.

Suppose that the  $(n-1)$-dimensional case is proved.
Consider an $n$-dimensional Riemannian polyhedron $\spc{P}$ and suppose
\[n\cdot \sqrt[n]{\VolPro\spc{P}(R)}< R\]
for some $R>0$.
Let $\spc{Q}$ be an $R$-separating subpolyhedron in $\spc{P}$ provided by \ref{lem:separating} for a small $\eps>0$.
Applying  \ref{lem:separating} for $r=\tfrac{n-1}n\cdot R$ and $R$, we have that 
\begin{align*}
\VolPro_\spc{Q}(r) &< \frac 1{R-r}\cdot \VolPro_\spc{P}(R)+\eps<
\\
&<\frac {n}{R}\cdot\left(\frac{R}{n}\right)^n=
\\
&=\left(\frac{r}{n-1}\right)^{n-1};
\end{align*}
that is, $(n-1)\cdot \sqrt[n-1]{\VolPro\spc{Q}(r)}< r$.
Since $\dim\spc{Q}\le n-1$, by the induction hypothesis, we get that
\[\width\spc{Q}\le r<R.\]
It remains to apply \ref{lem:separating-width}.
\qeds





\section{Width bounds systole}

\begin{thm}{Theorem}\label{thm:sys<width}
Suppose $\spc{M}$ is a aspherical $n$-dimensional Riemnnian manifold.
Then 
\[\sys\spc{M}\le 6 \cdot \width \spc{M}.\]
\end{thm}

\begin{thm}{Lemma}\label{lem:aspherical-homotopy}
Let $M$ be an aspherical space and $L$ be a connected CW-complex.
Denote by $L^k$ the k-skeleton of $L$.
Then any continuous map $f\:L^2\to M$ can be extended to a continuous map $\bar f\:L\to M$

Moreover, if $p\in L$ is a 0-cell and $q\in M$.
Then a continuous maps of pairs $\phi_0,\phi_1\:(L,p)\to(M,q)$ are homotopic if and only if $\phi_0$ and $\phi_1$ induce the same homomorphism on fundamental groups $\pi_1(L,p)\to\pi_1(M,q)$.
\end{thm}

\parit{Proof.}
Since $M$ is aspherical, any continuous map $\partial\mathbb{D}^n\:\to M$ for $n\ge 3$
is hull-homotopic;
that is, it can be extended to a map $\mathbb{D}^n\:\to M$.

It makes possible to extend $f$ to $L^3$, $L^4$, and so on.
Therefore $f$ can be extended to whole $L$.

The only-if part on the second part of lemma is trivilal; let us show the if part.

Sine $L$ is connected, we can assume that $p$ forms the only 0-cell in $L$;
otherwise we can collapse a maximal sub-tree of the 1-skeleton in $L$ to $p$.
Therefore $L^1$ is formed by loops that generates $\pi_1(L,p)$.

By assumption, the restrictions of $\phi_0$ and $\phi_1$ to $L^1$ are homotopic.
In other words the homotopy $\Phi\:[0,1]\times L$ is defined on the 2-skeleton of $[0,1]\times L$.
It remains to apply the first part of the lemma.
\qeds



\begin{thm}{Lemma}\label{lem:sys-homotopy}
Suppose $\gamma_0,\gamma_1$ are two paths between points in a Riemannian space $\spc{M}$ such that $\dist{\gamma_0(t)}{\gamma_1(t)}{\spc{M}}<r$ for any $t\in[0,1]$.
Let $\alpha$ be a geodesic path from $\gamma_0(0)$ to $\gamma_1(0)$ and $\beta$ be a geodesic path from $\gamma_0(1)$ to $\gamma_1(1)$. 
If $2\cdot r<\sys\spc{M}$, then there is a homotopy $\gamma_t$ from
$\gamma_0$ to $\gamma_1$ such that $\alpha(t)= \gamma_t(0)$ and $\beta(t)\mapsto \gamma_t(1)$.
\end{thm}

\parit{Proof.}
Set $s=\sys\spc{M}$; 
since $2\cdot r<s$, we have that $\eps=\tfrac1{10}(s-2\cdot r)>0$.

Note that we can assume that $\gamma_0$ and $\gamma_1$ are rectifiable;
if not we can homotopy each into a broken geodesic line kipping the assumptions true. 

\begin{wrapfigure}{r}{34mm}
\vskip-0mm
\centering
\includegraphics{mppics/pic-1405}
\end{wrapfigure}

Choose a fine partition $0\z=t_0\z<t_1\z<\z\dots\z<t_n=1$.
Consider a sequence of geodesic paths $\alpha_i$ from $\gamma_0(t_i)$ to $\gamma_1(t_i)$;
we can assume that $\alpha_0=\alpha$ and $\alpha_n=\beta$.
We can assume that each arc $\gamma_j|_{[t_{i-1},t_i]}$ has length smaller than $\eps$.
Therefore every quadrilateral formed by concatenation  of $\alpha_{i-1}$, $\gamma_1|_{[t_{i-1},t_i]}$, reversed $\alpha_i$, and reversed arc $\gamma_0|_{[t_{i-1},t_i]}$ has length smaller than $s$.
It follows that this curve is contractible.
Applying this observation for each quadrilateral, we get the statement.
\qeds


\parit{Proof of \ref{thm:sys<width}.}
Let $\spc{N}$ be the nerve of a covering $\{V_i\}$ of $\spc{M}$ and $\psi\:\spc{M}\to\spc{N}$ be the map provided by \ref{prop:space->nerve}.
As usual, we denote by $v_i$ the vertex of $\spc{N}$ that corresponds to $V_i$.

Set $R=\width \spc{M}$ and $s=\sys\spc{M}$.
Assume we chose $\{V_i\}$ as in the definition of width (\ref{def:width}).
For each $i$ choose a point $p_i\in \spc{M}$ such that $V_i\subset \oBall(p_i,R)$.
Observe that in this case $\dim\spc{N}<n$;
therefore $\psi$ kills the fundamental class of $\spc{M}$.

Let us construct a continuous map  $f\:\spc{N}\to  \spc{M}$ such that
$f\circ\psi$ is homotopic to the identity map on $\spc{M}$.

Note that once $f$ is constructed, the theorem is proved, .
Indeed, since $\psi$ kills the fundamental class of $\spc{M}$, so does $f\circ\psi$.
Therefore $[\spc{M}]=0$ --- a contradiction.

First set $f(v_i)=p$.
It defines the map $f$ on the 0-skeleton $\spc{N}^0$ of the nerve $\spc{N}$.
Further we will be define $f$ step by step on the skeletons of higher dimensions $\spc{N}^1,\spc{N}^2, \dots$

Let us map each edge $[v_iv_j]$ in $\spc{N}$ to a geodesic $[p_ip_j]$.
It defines the map on the 1-skeleton $\spc{N}^1$ of the nerve $\spc{N}$.
Note that image of each edge is shorter that $2\cdot R$.

Suppose $[v_iv_jv_k]$ is a triangle in $\spc{N}$.
Note that perimeter of the triangle $[p_ip_jp_k]$ can not exceed $6\cdot R$.
Since $6\cdot R<s$, the contour of $[p_ip_jp_k]$ is contractible.
Therefore we can extend $f$ to each triangle of~$\spc{N}$.
It defines the map $f$ on $\spc{N}^2$.

Finally, since $\spc{M}$ is aspherical, by \ref{lem:aspherical-homotopy}, the map $f$ can be extended to $\spc{N}^3$, $\spc{N}^4$ and so on.

It remains to show that $f\circ\psi$ is homotopic to the identity map.
Choose a CW structure on $\spc{M}$ with sufficiently small cell, so that each cell lies in one of $V_i$.
Note that $\psi$ is homotopic to a map $\psi_1$ that sends $\spc{M}^k$ to $\spc{N}^k$ for any $k$.
Moreover we may assume that (1) if a 0-cell $x$ of $\spc{M}$ maps to a $v_i$, then $x\in V_i$ and (2) each 1-cell  of $\spc{M}$ maps to an edge of $\spc{N}$.
Choose a 1-cell $e$ in $\spc{M}$; by the construction, $f\circ\psi_1$ maps $e$ to a geodesic $[p_ip_j]$ and $e$ lies $\oBall(p_i,R)$.
Observe that $[p_ip_j]$ is shorter than $2\cdot R$.
It follows that the distance between points on $[p_ip_j]$ and $e$ can not exceed $3\cdot R$.
Choose a geodesic path $\alpha_i$ from every 0 cell $x_i$  of $\spc{M}$ to $p_j=f\circ\psi_1(x_i)$.
It defines a homotopy on $\spc{M}^0$.
Since $6\cdot R<s$, \ref{lem:sys-homotopy} implies that this homotopy can be extended to $\spc{M}^1$.
By \ref{lem:aspherical-homotopy}, it can be extended to whole $\spc{M}$.
\qeds

\begin{thm}{Exercise}\label{ex:sys<width}
Analyze the proof of \ref{thm:sys<width} and improve its inequality to 
 \[\sys\spc{M}\le 4 \cdot \width \spc{M}.\]
\end{thm}

\begin{thm}{Exercise}\label{ex:fillrad-inj}
Modify the proof of \ref{thm:sys<width} to prove the following:

Suppose that $\spc{M}$ is a closed $n$-dimensional Reimannian manifold with \emph{injectivity radius} at least $r$; that is, if $\dist{p}{q}{\spc{M}}<r$, then there is geodesic $[pq]_{\spc{M}}$ is uniquely defined.
Show that
\[\width\spc{M}\ge \tfrac{r}{2\cdot(n+1)}.\]

Use \ref{thm:width<vol} to conclude that  
\[\vol\spc{M}\ge \eps_n \cdot r^n \]
for some $\eps_n>0$ that depends only on $n$.
\end{thm} 

\section{Essential manifolds}

To generalize \ref{thm:sys<width} bit further, we need the following definition.

\begin{thm}{Definition}
A closed manifold $\spc{M}$ is called \index{essential manifold}\emph{essential} if it admits a continuous map $\iota\:\spc{M}\to \spc{K}$ to an aspherical topological space $\spc{K}$ such that $\iota$ sends the fundamental class of $\spc{M}$ to a nonzero homology class in $\spc{K}$.\footnote{We assume that the coefficients are $\ZZ_2$, but one can play with them if necessary.}
\end{thm}

Assume that the manifold $\spc{M}$ is essential and $\iota \:\spc{M}\to \spc{K}$ as in the definition.
Following the proof of \ref{thm:sys<width}, we can homotope the map 
$f\circ\psi\:\spc{M}\to \spc{M}$ to the identity on the 2-skeleton of $\spc{M}$;
further since $\spc{K}$ is aspherical we can homotopy the composition
$\iota\z\circ f\circ\psi$ to  $\iota$. 
Existence of this extension implies that that $\iota$ kills the fundamental class of $\spc{M}$ --- a contradiction.
So, taking \ref{ex:sys<width} into account, we proved the following yet more general theorem.

\begin{thm}{Theorem}\label{thm:sys<width++}
Suppose $\spc{M}$ is an essential Riemnnian space.
Then 
\[\sys\spc{M}\le 4 \cdot \width \spc{M}.\]
\end{thm}

As a corollary form \ref{thm:sys<width++} and \ref{thm:width<vol} we get the so called \index{Gromov's systolic inequality}\emph{Gromov's systolic inequality}:

\begin{thm}{Theorem}\label{thm:sys+}
Suppose $\spc{M}$ is an essential $n$-dimensional Riemannian space.
Then 
\[\sys\spc{M}\le 4 \cdot n\cdot \sqrt[n]{\vol\spc{M}}.\]
\end{thm}


Note that any closed aspherical manifold is essential --- in this case one can take $\iota$ to be the identity map on $\spc{M}$.
The real projective space $\RP^n$ provides an interesting example of an essential manifold which is not aspherical.
Indeed, the infinite dimensional projective space $\RP^\infty$ is aspherical and for the natural embedding $\RP^n\hookrightarrow\RP^\infty$ the image $\RP^n$ does not bound in $\RP^\infty$.
The following exercise provides more examples of that type.

\begin{thm}{Exercise}\label{ex:connected-sum-essential}
Show that connected sum of an essential manifold with any closed manifold is essential.
\end{thm}

\begin{thm}{Exercise}\label{ex:product-essential}
Show that product of two essential manifolds is essential.

Show that product of nonessential closed manifold of dimension at least 1 with any closed manifold is not essential.
\end{thm}

\section{Remarks}

Theorem \ref{thm:sys+} was proved originally by Mikhael Gromov \cite{gromov-1983} with much worse constant.
The given proof is a result of a sequence of simplifications given by Larry Guth \cite{guth},  Panos Papasoglu \cite{papasoglu}, Alexander Nabutovsky and Roman Karasev \cite{nabutovsky}.

In \cite{nabutovsky} the calculations were optimized better which gave the constants 
$c_n=\sqrt[n]{n!}= \tfrac ne+o(n)$ in \ref{thm:width<vol} instead of $n$.
As a result, we have a stronger statement in \ref{thm:sys+}:
\[\sys\spc{M}\le 4 \cdot c_n\cdot \sqrt[n]{\vol\spc{M}}.\]

A wide open conjecture says that the optimal constant is $\pi/\sqrt[n]{\omega_n/2}$ where $\omega_n$ denotes the volume of $n$-dimensional unit sphere.
This is the systole ratio for the $n$-dimensional real projective space with canonical metric; it  grows as $O(\sqrt n)$.



%\chapter{Volume bounds filling radius}

This chapter 
is devoted to a proof of \ref{thm:FillRad<vol};
that is, we will show that \emph{Riemannian manifolds with small volume have small filling radius}.
Note that once it is proved, \ref{thm:sys<FillRad} implies \ref{thm:sys(torus)}.
Moreover \ref{thm:sys<FillRad++} implies the following:

\begin{thm}{Therorem}\label{thm:sys(torus)}
Systolic intequality holds for any essential manifold. 
\end{thm}

This theorem was proved originally by Mikhael Gromov \cite{gromov-1983}.
We follow closely a simplified proof given by Alexander Nabutovsky, which is based on a sequence of other simplifications and improvements; see \cite{nabutovsky} and the references therein.

\section{Nerves and partition of unity}

Let $\{V_1,\dots,V_k\}$ be a finite open cover of a compact metric space $\spc{X}$.
Consider an abstract simplicial complex $\spc{N}$, with one vertex $v_i$ for each set $V_i$ such that a simplex with vertexes $v_{i_1},\dots, v_{i_m}$ is included in $\spc{N}$ if 
the intersection $V_{i_1}\cap\dots\cap V_{i_m}$ is nonempty.
We obtain a simplicial complex $\spc{N}$ called the \index{nerve}\emph{nerve of the covering $\{V_i\}$}.

Note that $\spc{N}$ is a finite simplicial complex and it has dimension at most $n$ if and only if the covering $\{V_1,\dots,V_k\}$ has multiplicity is at most $n+1$;
that is, at most $n+1$ different sets $V_{i_1},\dots, V_{i_{n+1}}$ have a nonempty intersection.
The nerve $\spc{N}$ is a subcomplex of a simplex with the vertixes $\{v_1,\dots,v_k\}$.

\begin{thm}{Proposition}\label{thm:part-unit}
 Let $\{V_1,\dots,V_k\}$ is a finite open covering of a compact metric space ${\spc{X}}$.
Then there are Lipschitz functions $\psi_i\:{\spc{X}}\to[0,1]$ such that
if $\psi_i(x)>0$ then $x\in V_i$ and
$$\sum_i\psi_i(x)=1$$
for any $x\in {\spc{X}}$.
\end{thm}

A collection of functions $\psi_i$ with above properies is called 
a \emph{partition of unity subordinate to the open cover}\index{partition of unity} $\{V_1,\dots,V_k\}$.

\parit{Proof.}
Consider the functions $\phi_i\:{\spc{X}}\to\RR$ defined as
$$\phi_i(x)=\distfun_{({\spc{X}}\backslash V_i)} x.$$
Note $\phi_i$ is $1$-Lipschitz
for any $i$
and $\phi_i(x)>0$ if and only if $x\in V_i$.
In particular, 
$$\sum_i\phi_i(x)>0\ \ \text{for any}\ \ x\in {\spc{X}}.$$

Set 
$$\psi_k(x)=\frac{\phi_k(x)}{\sum_i\phi_i(x)}.$$
It remains to note that by construction the functions $\psi_i$ meet the conditions in the proposition.
\qedsf


Note that in the above proof for any point $x\in {\spc{X}}$,
the set
$$\set{v_i}{\psi_i(x)>0}$$
describe vertexes of a simplices in the nerve.
Therefore 
$$\psi\:x\mapsto \psi_1(x)\cdot v_1+\psi_2(x)\cdot v_2+\dots+\psi_k(x)\cdot v_n.$$
can be thought of as a Lipschitz map from ${\spc{X}}$ to the nerve $\spc{N}$ of $\{V_i\}$;
where the point $x$ is mapped to the point with barycentric coordinates $\psi_i(x)$.
In other words we proved the following:

\begin{thm}{Proposition}\label{prop:space->nerve}
Let $\spc{N}$ be a nerve of an open covering $\{V_1,\z\dots,V_k\}$ of a compact metric space $\spc{X}$.
Denote by $v_i$ the vertex of $\spc{N}$ that corresponds to $V_i$.

Then there is a Lipschitz map from $\psi\:\spc{X}\to\spc{N}$ such that $\psi(V_i)\z\subset\Star_{v_i}$ for every $i$.
\end{thm}


\section{Width}

Suppose $A$ is a subset of a metric space $\spc{X}$.
The radius of $A$ (briefly $\rad A$) is defined as the least upper bound on the values $R>0$ such that $\oBall(x,R)\supset A$ for some $x\in \spc{X}$.

\begin{thm}{Definition}\label{def:width}
Let $\spc{X}$ be a metric space.
The $n$-th width of $\spc{X}$ (briefly $\width_n\spc{X}$) is defined as least upper bound on values $R>0$ such that $\spc{X}$ admits a finite open covering $\{V_i\}$ with multiplicity at most $n+1$ and $\rad V_i< R$ for each $i$.
\end{thm}

\parit{Remarks.}
\begin{itemize}
\item Observe that if $\spc{X}$ is connected, then 
\[\width_0\spc{X}=\rad\spc{X}.\]
\item 
Usually width is defined using diameter instead of radius, but the result differ at most twice.
Namely if $r$ is an $n$-th radius-width and $d$ --- $n$-th diameter-width of the same dimension, then 
$r\le d\le 2\cdot r$.

\item The definition of width reminds the definition of Lebesgue covering dimension.
In fact one says that a space has \emph{macroscopic dimesion} $\le n$ on the space $R$ if it admits an open cover as in the definiton.
\end{itemize}

\begin{thm}{Exercise}\label{ex:macrodimension}
Suppose $\spc{X}$ be a metric space such that any closed curve $\gamma$ in $\spc{X}$ can be contracted in its $R$-neighborhood.
Show that $\spc{X}$ is has macroscopic dimension at most 1 on scale $100\cdot R$.

What about quasiconverse? That is, suppose a simply connected metric space $\spc{X}$ has macroscopic dimension at most 1 on scale $R$, is it true that any closed curve $\gamma$ in $\spc{X}$ can be contracted in its $100\cdot R$-neighborhood?
\end{thm}


The following proposition provides an equivalent definition;
we will not use it, but it provides a good reason for the name width.

\begin{thm}{Proposition}\label{prop:width=suprad(inv)}
Suppose $\spc{X}$ is a compact metric space.
Then $\width_n\spc{X}<R$ if and only if there is a finite $n$-dimensional somplicial complex $\spc{S}$ and a continuous map $\psi\:\spc{X}\to \spc{N}$
such that $\rad[\psi^{-1}(s)]\z<R$
for any $s\in \spc{N}$.
\end{thm}

\parit{Proof; ``only if'' part.}
Suppose $\width_n\spc{X}<R$.
Consider a covering $\{V_1,\dots,V_k\}$ of $\spc{X}$ guaranteed by the definition of width.
Let $\spc{N}$ be its nerve and $\psi\:\spc{X}\to \spc{N}$ be the map provided by \ref{prop:space->nerve}.

Note that if $x\in \spc{N}$ lies in a symplex with a vertex $v_i$,
then $\psi^{-1}(x)\subset V_i$;
in particulr $\psi^{-1}(x)$ can be covered by a ball of radius $R$ in $\spc{X}$.

\parit{``If'' part.}
Choose $x\in \spc{N}$.
Since the inverse image $\psi^{-1}(x)$ is compact, $\psi$ is continuous, and $\rad[\psi^{-1}(x)]<R$,
here is a neighborhood $U\ni x$ such that the  $\rad[\psi^{-1}(U)]<R$.

It follows that there is a finite cover $\{U_i\}$ of $\spc{N}$ such that $\psi^{-1}(U_i)\subset\spc{X}$ has radius smaller than $R$ for each $i$.
Since $\spc{N}$ has dimension $n$, we can inscribe%
\footnote{Recall that a covering $\{W_i\}$ is inscribed in the covering $\{U_i\}$ if for every $W_i$ is a subset of some $U_j$.} 
in $\{U_i\}$ an finite open cover $\{W_i\}$ with multiplicity at most $n+1$.
It remains to observe that $V_i=\psi(W_i)$ defines a finite open cover of $\spc{X}$ with radius less than $R$ and multiplicity at most $n+1$. 
\qeds

Further we will apply the notion of width to compact Riemannian polyhedrons;
If $n$ is the dimension of a compact Riemannian polyhedron $\spc{P}$, then 
we suppose that
\[\width\spc{P}\df\width_{n-1}\spc{P}.\]

\begin{thm}{Exercise}\label{ex:FillRad<width}
Show that for any closed Riemannian manifold $\spc{M}$ we have
\[\FillRad \spc{M}\le 100\cdot \width\spc{M};\]
try to show that in fact
\[\FillRad \spc{M}\le \width\spc{M}.\]

\end{thm}




\section{Volume profile}

A \emph{Riemannian polyhedron} is defined as a finite connected simplicial complex with a metric tensor on each simplex such that the restriction of the metric on each simplex to a subsymplex coinsides with the metric on the subsmplex.
The dimension of Riemannian polyhedron is defined as the largest dimension it its triangulation.
For Riemannian polhedron one can define length of curves and volume the same way as for Riemannian manifolds.

Let $\spc{P}$ be a Riemnnian polyhedron of dimension $n$.
Let us define volume profile of $\spc{P}$ as a function $\VolPro_{\spc{P}}\:\RR_+\to\RR_+$ defined by 
\[\VolPro_{\spc{P}}(r)\df \sup\set{\vol \oBall(p,r)}{p\in\spc{P}}.\]
Note that $\VolPro_{\spc{P}}$ is a nondecreasing function and $\VolPro_{\spc{P}}(r)\z\to\vol\spc{P}$ as $r\to\infty$.

\begin{thm}{Theorem}\label{thm:width<volpro}
There is a constant $c_n>0$ such that the following holds true:

If $\spc{P}$ is an $n$-dimensional Reimannian polyhedron such that 
\[r> c_n\cdot \sqrt[n]{\VolPro_{\spc{P}}(r)}\] 
for some $r>0$, then 
\[\width\spc{P}\le  r.\]
\end{thm}

Since $\VolPro_{\spc{P}}(r)\le \vol\spc{P}$ for any $r$,
Theorem \ref{thm:width<volpro} implies the following:

\begin{thm}{Theorem}\label{thm:width<vol}
There is a constant $c_n>0$ such that 
\[\width\spc{P}\le c_n\cdot \sqrt[n]{\vol\spc{P}}\] 
for any  $n$-dimensional Reimannian polyhedron $\spc{P}$.
\end{thm}

Together with \ref{ex:FillRad<width}, the last theorem implies \ref{thm:FillRad<vol} which is the goal of this lecture.

\section{Proof}

In the proof of \ref{thm:width<volpro}, we will use the following three technical statements,
the proofs are omitted, but they are not hard. 

\begin{thm}{Smoothing procedure}
Let $\spc{P}$ be a Reimannian polyhedron and $f\:\spc{P}\to \RR$ be a 1-Lipschitz function.
Then for any $\delta>0$ there is a  1-Lipschitz function $\tilde f\:\spc{P}\to \RR$ that is smooth on each simplex of the triangulation and $\delta$-close to $f$.
\end{thm}

\begin{thm}{Sard's theorem}
Let $\spc{P}$ be an $n$-dimensional Reimannian polyhedron and $f\:\spc{P}\to \RR$ be a function that is smooth on each simplex.
Then for almost all values $a$ each component of the inverse image $f^{-1}(a)$ is a equipped with the induced metric is a Reimannian polyhedron.
\end{thm}


\begin{thm}{Coarea inequality}
Let $\spc{P}$ be an $n$-dimensional Reimannian polyhedron and $f\:\spc{P}\to \RR$ be a 1-Lipschitz function that is smooth on each simplex.
Then 
\[\vol_n (f^{-1}[a,b]) \le \int_a^b\vol_{n-1}(f^{-1}\{x\})\cdot dx.\]
\end{thm}

Theorem \ref{thm:width<volpro} will be proved by induction on the dimension of $\spc{P}$;
the following exercise provides a base for the induction.
Note that $\spc{P}$ is connected by definition and if it is 1-dimensional, then 
\[\width\spc{P}=\width_0\spc{P}=\rad\spc{P}.\]

\begin{thm}{Exercise}\label{ex:1D-case}
Suppose $\spc{P}$ be a 1-dimensional Riemannian polhedron.
Suppose $\VolPro_{\spc{P}}(r)<r$ for some $r>0$.
Show that 
\[\width \spc{P}<r.\]

\end{thm}


An $(n-1)$-dimensional subpolyhedron $\spc{Q}\subset\spc{P}$ is called $R$-separating if each
connected component of the complement $\spc{P}\backslash \spc{Q}$ has radius smaller than $R$.

\begin{thm}{Lemma}\label{lem:separating}
Let $\spc{P}$ be an $n$-dimensional Riemannian polyhedron.
Then given $R>0$ and $\eps>0$ there is a $R$-separating subpolyhedron $\spc{Q}\subset\spc{P}$ such that for any $r_0<r_1\le R$ we have
\[\VolPro_{\spc{Q}}(r_0)< \tfrac1{r_1-r_0}\cdot \VolPro_{\spc{P}}(r_1)+\eps.\]

\end{thm}

\parit{Proof.}
Choose small $\delta>0$.
Applying the smoothing procedure, we can exchange each distance function $\distfun_p$ on $\spc{P}$ by $\delta$-close smooth 1-Lipschitz function, which will be denoted by $\widetilde \distfun_p$.

By Sard's theorem, almost all level sets $\tilde S_c(p)$ defined by $\widetilde \distfun_p=c$ are smooth Riemannian polyhedrons of dimension $n-1$.

Since $\delta$ is small, the coarea inequality implies that 
for some  $c\z\in(r_0+\delta, r_1-\delta)$ we have
\begin{align*}
\vol_{n-1}\tilde S_c(p)&\le \tfrac1{r_1-r_0-2\cdot\delta}\cdot\vol_n[\oBall(p,r_1)]<
\\
&<\tfrac1{r_1-r_0}\cdot \VolPro_{\spc{P}}(r_1)+\tfrac\eps2.
\end{align*}

Now suppose $\spc{Q}$ is an $R$-separating subpolyhedron in $\spc{P}$ with almost minimal volume, say its volume is at most $\tfrac\eps2$-far from the greatest lower bound.
Note that cutting from $\spc{Q}$ everything inside $\tilde S_c$ and adding $\tilde S_c$ keeps it to be $R$-separating subpolyhedron.
It follows that
\[\vol_{n-1}[\spc{Q}\cap \oBall(p,r_0)_{\spc{P}}]-\tfrac\eps2\le \vol_{n-1}S_c.\]
Therefore 
\[\vol_{n-1}[\spc{Q}\cap \oBall(p,r_0)_{\spc{P}}]\le\tfrac1{r_1-r_0}\cdot \VolPro_{\spc{P}}(r_1)+\eps\eqlbl{eq:volQ<ProP}\]
Recall that $\spc{Q}$ is equipped with the induced length metric;
therefore $\dist{p}{q}{\spc{Q}}\ge \dist{p}{q}{\spc{P}}$ for any $p,q\in \spc{Q}$;
in particular, 
\[\oBall(p,r_0)_{\spc{Q}}\subset \spc{Q}\cap \oBall(p,r_0)_{\spc{P}}.\]
Hence \ref{eq:volQ<ProP} implies the lemma.
\qeds

\begin{thm}{Lemma}\label{lem:separating-width}
Let $\spc{Q}$ be a $R$-separating subpolyhedron in an $n$-dimensional Riemannian polyhedron $\spc{P}$.
Suppose $\width\spc{Q}\le R$.
Then $\width\spc{P}\le R$
\end{thm}

\parit{Proof.}
Start with an open covering $\{V_1,\dots,V_k\}$ of $\spc{Q}$ of multiplicity $\le n$ with radiuses of the sets in the intrinsic metric $\le R$.

Note that $\{V_1,\dots,V_k\}$ can be converted into an an open covering of
a small neighbourhood of $\spc{Q}$ in $\spc{P}$ without increasing the multiplicity.
This is can be done by setting 
\[V_i'=\bigcup_{x\in V_i}\oBall(x,r_x),\]
where $r_x=\tfrac1{10}\cdot\inf\set{\dist{x}{y}{}}{y\in \spc{Q}\backslash V_i}$.

Finally, add all the components of $\spc{P}\backslash \spc{Q}$ to the covering;
it increases the multiplicity by 1.
The statement follows since $\dim \spc{P}= \dim \spc{Q}\z+1$.
\qeds

\parit{Proof of \ref{thm:width<volpro}.}
We apply induction on the dimension $n=\dim\spc{P}$;
the base case $n=1$ is provided by \ref{ex:1D-case},  for $c_1=1$.

Suppose that the constant $c_{n-1}$ is known, choose sufficiently small $c_n$
\[c_n>2\cdot c_{n-1}.\]

Assume $c_n\cdot \sqrt[n]{\VolPro\spc{P}(r)}< r$.
Fix small $\eps>0$.
By taking $r_0=\tfrac r2$ and $r_1=r$ in \ref{lem:separating}, we have an $r$-separating subpolhedron $\spc{Q}$ in $\spc{P}$ such that 
\begin{align*}
\VolPro_\spc{Q}(r_0) &< \tfrac 1 {r_0}\cdot \VolPro_\spc{P}(r)+\eps<
\\
&<\tfrac 1 {r_0}\cdot \left(\frac{2\cdot r_0}{c_n}\right)^n+\eps=
\\
&=\left(\frac2{c_n}\right)^n\cdot r_0^{n-1}+\eps<
\\
&<\left(\frac1{c_{n-1}}\right)^{n-1}\cdot r_0^{n-1};
\end{align*}
that is, $c_{n-1}\cdot \sqrt[n-1]{\VolPro\spc{Q}(r_0)}< r_0$.
By the induction hypothesis 
\[\width\spc{Q}\le r_0<r.\]

Applying \ref{lem:separating-width}, we get $\width\spc{P}<r$
\qeds


%\chapter{Examples}



\section{On semicontinuity}

Recall that according to \ref{ex:GH-vol}, volume is semicontinuos on the space of Riemannian manifolds with respect to stable Gromov--Hausdorff convergence.
Analogous statement for $n$-dimensional Hausdorff measure on a $n$-dimensional manifolds does not hold.

\begin{thm}{Claim}
 
\end{thm}

First let us show that for any $\alpha>0$, the $\alpha$-dimensional Hausdorff measure is not semicontinuous in the space of all compact metric spaces.

Choose a decreasing sequence $\eps_n\to 0$.
Consider the space $\spc{C}$ of infinite binary sequences with distance between two sequences $\bm{a}=(a_0,a_1,\dots)$ and $\bm{b}=(b_0,b_1,\dots)$ defined by 
\[\dist{\bm{a}}{\bm{b}}{\spc{C}}=\eps_n,\]
where $n$ is the minimal index such that $a_n\ne b_n$.
Note that $\spc{C}$ is homeomorphic to the Cantor set and 
given $\alpha>0$,
the sequence $\eps_n$ can be chosen so that its $\alpha$-dimensional Hausdorff measure is infinite.

Note that $\spc{C}$ is a Hausdorff limit of its subsets $\spc{C}_n$ formed by sequences that constantly zero starting from $n$-th element.
The sets $\spc{C}_n$ is finite in particular its $\alpha$-dimensional Hausdorff measure vanish for $\alpha>0$.
This example shows that for any $\alpha>0$, the $\alpha$-dimensional Hausdorff measure is not semicontinuous in the space of all compact metric spaces.

An analogous example can be produced comapct length spaces.
To do this consider a metric binary rooted tree $\spc{T}$ in which edges connecting level $n-1$ to the level $n$ of length $\eps_{n-1}-\eps_n$.
Note that the completion $\bar{\spc{T}}$ of $\spc{T}$ has a subset (its crown) isometric to $\spc{C}$.
Note further that $\bar{\spc{T}}$ is a Hausdorff limit of its subsets $\spc{T}_n$ --- the subtrees up to level $n$.
Note that $\spc{T}_n$ is can be covered by a finite line segments, in particular it has finite $1$-dimensional Hausdorff measure and therefore vanishing $\alpha$-dimensional Hausdorff for any $\alpha>1$.
Since the limit $\bar{\spc{T}}$ contains $\spc{C}$, we can choose a sequence $\eps_n$ so that $\mu_\alpha\spc{C}$ is arbitrary large (or even infinite).
It shows that for any $\alpha>1$, the $\alpha$-dimensional Hausdorff measure is not semicontinuous in the space of all compact length spaces.

This construction can be modified further to obtain an increasing sequence of metric tensors $g_n$ on a disc $\DD$ such that (1) $\vol(\DD,g_n)<1$ for each $n$, (2) the induced metrics $\dist{*}{*}{g_n}$ converge to a metric $\rho$ on $\DD$, and given any Cantor space $\spc{C}$ as described above (3) there is a bilipschitz map $\spc{C}\to(\DD,\rho)$.
Note that the last condition implies that $\mu_2(\DD,rho)$ can be made arbitrary large, or infinite.
Therefore for any $\alpha\ge 2$, the $\alpha$-dimensional Hausdorff measure is not semicontinuous in the space of all compact length spaces homeomorphic to a manifold and equipped with stable convergence.

Now we want to extend nonsemicontinuity even further.
Note that the tree $\bar{\spc{T}}$ admits a length-preserving embedding to the Euclidean space; we may assume that all 



\section{Sub-Riemannian metrics}

Choose a metric space $\spc{X}$.
Note that the function $\alpha\mapsto \mu_\alpha(A)_\spc{X}$ is nondecreasing;
moreover there is a critical value $\alpha_0\in[0,\infty]$ such that $\mu_\alpha(A)_\spc{X}=0$ if $\alpha<\alpha_0$ and $\mu_\alpha(A)_\spc{X}=\infty$ if $\alpha>\alpha_0$.
This value is called \index{Hausdorff dimension}\emph{Hausdorff dimension} of $\spc{X}$, or briefly $\alpha_0=\dim_H\spc{X}$.

The following statement is classical, a proof can be found in .

\begin{thm}{Theorem}
The Hausdorff dimension of any metric space can not be smaller than its Lebesgue covering dimension.
In particular, if a metric space $\spc{X}$ is homeomorphic to an $n$-dimensional manifold, then $\dim_H\spc{X}\ge n$.
 
\end{thm}

Note that the construction described in the previous section can be used to produce a metric on manifold of dimension $n\ge 2$ with arbitrary Hausdorff dimension $\alpha\ge n$.

In this section we will discuss another interesting source of such examples.



%











\begin{thm}{Lemma}
$\spc{M}$ is complete.
\end{thm}

\parit{Proof.}
Let $(\spc{X}_n)$ be a Cauchy sequence in $\spc{M}$.
Passing to a subsequence if necessary, 
we can assume that $|\spc{X}_n-\spc{X}_{n+1}|_{\spc{M}}<\tfrac1{2^n}$ for each $n$.
In particular, for each $n$ one can equip $\spc{W}_n=\spc{X}_n \sqcup \spc{X}_{n+1}$ with a metric such that
inclusions $\spc{X}_n\hookrightarrow \spc{W}_n$ and $\spc{X}_{n+1}\hookrightarrow \spc{W}_n$ are distance preserving
and $$|\spc{X}_n-\spc{X}_{n+1}|_{\mathcal{H}(\spc{W}_n)}\z<\tfrac1{2^n}$$
for each $n$.

Set $\spc{W}$ to be the disjoint union of all $\spc{X}_n$.
Let us equip $\spc{W}$ with a metric defined the following way:
\begin{itemize}
\item for any fixed $n$ and any two points $x_n,x_n'\in \spc{X}_n$ set
$$|x_n-x_n'|_{\spc{W}}=|x_n-x_n'|_{\spc{X}_n}$$
\item for any positive integers $m>n$ and any two points $x_n\in \spc{X}_n$ and $x_m\in \spc{X}_m$ set
$$|x_n-x_m|_{\spc{W}}=\inf\left\{\sum_{i=n}^{m-1}|x_i-x_{i+1}|_{\spc{W}_i}\right\},$$
where the infimum is taken for all sequences $x_i\in \spc{X}_i$.
\end{itemize}

\begin{thm}{Exercise}
Check that this indeed defines a metric on $\spc{W}$.
\end{thm}

Let $\bar{\spc{W}}$ be the completion of $\spc{W}$.
Note that $|\spc{X}_m-\spc{X}_n|<\tfrac1{2^{n-1}}$ if $m>n$.
Therefore the union of $\spc{X}_1\cup \spc{X}_2\cup\dots\cup \spc{X}_n$ forms a $\tfrac1{2^{n-1}}$-net in $\bar{\spc{W}}$.
Since each $\spc{X}_i$ is compact, we get that $\bar{\spc{W}}$ admits a compact $\eps$-net for any $\eps>0$.
According to Problem~\ref{pr:compact-net}, $\bar{\spc{W}}$ is compact.

According to Blaschke's compactness theorem (\ref{thm:compact+Hausdorff}),
we can pass to a subsequence of $(\spc{X}_n)$ which converge in $\mathcal{H}(\bar{\spc{W}})$ and therefore in $\spc{M}$.
\qeds

\parit{Proof of \ref{thm:gromov-compactness}; ``only if'' part.}
If there is no sequence $\eps_n\to0$ as described in the problem, then for a fixed fixed $\delta>0$
there is a sequence of spaces $\spc{X}_n\in\spc{Q}$ such that $$\pack_\delta \spc{X}_n\to\infty
\quad\text{as}\quad
n\to\infty.$$
Since $\spc{Q}$ is compact, 
this sequence has a partial limit say $\spc{X}_\infty\in\spc{Q}$.
It is easy to see that $\pack_{\delta/10} \spc{X}_\infty=\infty$;
the later contradicts Theorem~\ref{thm:finite_pack=compact}.

\parit{``If'' part.}
Let us fix the sequence $\eps_n\to 0$ as in the problem and consider the set $\hat{\spc{Q}}$ of all (isometry classes of all) metric spaces $\spc{X}$ such that
$\pack_{\eps_n} \spc{X}\le n$ for any $n$. 
According to Exercise~\ref{ex:pack-GH}, $\hat{\spc{Q}}$ is closed in $\spc{M}$.
Clearly $\spc{Q}\subset\hat{\spc{Q}}$.
Therefore it is sufficient to prove that $\hat{\spc{Q}}$ is compact.

Note that $\diam \spc{X}\le \eps_1$ for any $\spc{X}\in \hat{\spc{Q}}$.
Given positive integer $n$ consider set of all metric spaces $\spc{W}_n$
with number of points at most $n$ and diameter $\le \eps_1$.
Note that $\spc{W}_n$ is compact for each $n$.
Further a maximal $\eps_n$-packing of any $\spc{X}\in\hat{\spc{Q}}$ forms a subspace from $\spc{W}_n$.
Therefore $\spc{W}_n\cap\hat{\spc{Q}}$ is a comapct $\eps_n$-net in  $\hat{\spc{Q}}$.
Problem~\ref{pr:compact-net} implies that $\hat{\spc{Q}}$ is compact.
\qeds



\section{Comments} 

Given two metric spaces $\spc{X}$ and $\spc{Y}$, we will write $\spc{X}\preccurlyeq \spc{Y}$ if there is a noncontracting map $f\:\spc{X}\to \spc{Y}$;
that is, if 
$$ |x-x'|_{\spc{X}}\le|f(x)-f(x')|_{\spc{Y}}$$
for any $x,x'\in \spc{X}$.

Further, given $\eps>0$, we will write $\spc{X}\preccurlyeq \spc{Y}+\eps$
if there is a map $f\:\spc{X}\to \spc{Y}$ such that 
$$|x-x'|_{\spc{X}}\le|f(x)-f(x')|_{\spc{Y}}+\eps$$
for any $x,x'\in \spc{X}$.

Define 
$$\dist[\star]{\spc{X}}{\spc{Y}}{\spc{M}}=\inf\set{\eps}{\spc{X}\preccurlyeq \spc{Y}+\eps
\quad\text{and}\quad
\spc{Y}\preccurlyeq \spc{X}+\eps}$$
It turns out that $\dist[\star]{*}{*}{\spc{M}}$ is a different metric on the set of isometry classes of compact metric spaces; that is, in general $\dist[\star]{\spc{X}}{\spc{Y}}{\spc{M}}\not=|\spc{X}-\spc{Y}|_{\spc{M}}$. 
However, these two metrics define the same topology on $\spc{M}$.
More precicely:

\begin{thm}{Proposition}\label{GH-po}
For any sequence of compact metric spaces $(\spc{X}_n)$ and a compact metric space $\spc{X}_\infty$,
we have
$$|\spc{X}_n-\spc{X}_\infty|_{\spc{M}}\to 0
\quad\iff\quad
\dist[\star]{\spc{X}_n}{\spc{X}_\infty}{\spc{M}}\to 0$$ 
as $n\to\infty$.
\end{thm}

We will not give a proof of this proposition. 
Likely, we will not use it further in the lectures, 
but it might help you to build intuition for Gromov--Hausdorff convergence.
If you want to prove it yourself look in the proof of Theorem~\ref{thm:GH-is-a-metric} 
and try to modify it using ideas from the proof of Problem~\ref{pr:non-contracting=>isometry}.

The Gromov--Hausdorff distance can be defined for arbitrary pair of metric space.
Therefore it is natural to ask why we only consider compact metric spaces.
First note the Gromov--Hausdorff distance from any metric space $\spc{X}$ 
to its completion $\bar {\spc{X}}$ is zero.
Therefore if you want to end up in a metric space, it is better to consider only complete metric spaces.

Further, the distance between one-point-space and a metric spce with infinite diameter is infinite.
Therefore one has to either consider only bounded metric spaces (that is, the spaces with finite diameter)
or relux the definition of metric space which allow metric to take infinite value.

Finally, the class of isometry classes of all bounded complete metric spaces forms a class, but not a set.
Hence again we have two choices: either relux the definition of metric space so its points will form a class, or restrict further the class of spaces for which the isometry classes will form a set.

\begin{thm}{Exercise}
Prove that isometry classes of compact metric spaces form a set. 
\end{thm}

\begin{thm}{Exercise}\label{pr:GH1}
Let $\spc{X}=\{x,y,z\}$ be a three point subset of Euclidean plane with distances
$$|x-y|=|y-z|=|z-x|=1.$$
\begin{enumerate}[(i)]
\item Find the minimal Hausdorff distance from $\spc{X}$ to a one-point subset of the plane.
\item Find the Gromov--Hausdorff distance from $\spc{X}$ to the one-point metric space. 
\end{enumerate}
\end{thm}

\begin{thm}{Exercise}\label{pr:GH2}
Let $\spc{X}$ and $\spc{Y}$ be a compact metric spaces which have isometric $\eps$-nets.
Show that 
$$|\spc{X}-\spc{Y}|_{\spc{M}}\le 2\cdot\eps.$$
Is it always true that 
$$|\spc{X}-\spc{Y}|_{\spc{M}}\le \eps?$$
\end{thm}




\begin{thm}{Exercise}\label{pr:GH3}
Define the \emph{radius of a metric space}\index{radius of a metric space} $\spc{X}$ as 
$$\rad \spc{X}=\inf_x\left\{\sup_y\{|x-y|_{\spc{X}}\}\right\}.$$
Equivalently, 
$$\rad \spc{X}=\inf\set{R>0}{\text{there is}\ x\in \spc{X}\  \text{such that}\ B_R(x)\supset \spc{X}}.$$
 
\begin{enumerate}[(i)]
\item Show that for any compact metric space $\spc{X}$ we have
$$\tfrac12\cdot\diam \spc{X}\le \rad \spc{X}\le \diam \spc{X}.$$
\item Show that for any compact metric spaces $\spc{X},\spc{Y}$ we have
$$|\rad \spc{X}-\rad \spc{Y}|\le 2\cdot |\spc{X}-\spc{Y}|_{\spc{M}}.$$
\end{enumerate}
\end{thm}

\begin{thm}{Exercise}\label{pr:F-X}
Let $\spc{X}$ be a metric space.
If two compact sets $A, B$ in $\spc{X}$ are isometric,
we will write $A\iso B$. 
Set
$$d(A,B)=\inf \set{|A'-B'|_{\mathcal{H}(\spc{X})}}{A'\iso A \ \text{and}\ B'\iso B}.$$
Note that if $\spc{X}=\ell^\infty$ then according to Proposition~\ref{prop:GH-with-fixed-Z}, 
$d$ is a metric on $\mathcal{H}(\spc{X})/\iso$ (that is, on the ``$\iso$''-equivalecne classes of $\mathcal{H}(\spc{X})$).

Show that it does not hold for arbitrary metric space $\spc{X}$.
Understand the reason why it holds for $\spc{X}=\ell^\infty$.
\end{thm}


\begin{thm}{Exercise}\label{pr:GH-variation}
Consider the pairs $(\spc{X},A)$, where $\spc{X}$ is a compact metric space and $A$ is a closed subset in $\spc{X}$.
Two such pairs, say $(\spc{X},A)$ and $(\spc{X}',A')$ will be called equivalent (briefly $(\spc{X},A)\sim(\spc{X}',A')$)
if there is an isometry $\iota\:\spc{X}\to \spc{X}'$ such that $\iota(A)=A'$.

Modify the definition of Gromov--Hausdorff metric to construct a natural metric on the set of $\sim$-equivalence classes of the pairs $(\spc{X},A)$.
\end{thm}

Here we introduce so called Gromov--Hausdorff convergence for metric spaces.
This convergence was introduced by Gromov around 1980, published in \cite{gromov-1981}.
Very soon this notion began to be used in all branches of geometry.
In fact today I have difficulty to understand 
how one could do geometry without this type of convergence.%
(Some types of convergences of metric spaces was considered before Gromov,
but they had lack of generality;
each type of convergence was desined to solve one particular problem.)


\begin{thm}{Exercise}\label{ex:euclid-isom}
\begin{subthm}{}
Let $\spc{X},\spc{Y}$ be two compact sets in the Euclidean plane $\RR^2$.
Show that $\spc{X}$ is isometric to $\spc{Y}$ if and only if there is an motrio $\iota\:\RR^2\to \RR^2$
that sends $\spc{X}$ to $\spc{Y}$.
\end{subthm}

\begin{subthm}{}
Find two isometric subsets $\spc{X},\spc{Y}$ of $\ell^\infty$
such that there is no isometry $\iota\:\ell^\infty\to \ell^\infty$ 
that sends $\spc{X}$ to $\spc{Y}$.
\end{subthm}
\end{thm}

\backmatter

\newgeometry{top=0.9in, bottom=0.9in,inner=0.5in, outer=0.5in}
\chapter{Semisolutions}
{

\footnotesize
\begin{multicols}{2}
\refstepcounter{chapter}
\setcounter{eqtn}{0}

\parbf{\ref{ex:non-differentiable}.}
Choose a function $r\mapsto \alpha(r)$ such that $\alpha'(r)\cdot r\to 0$ and $\alpha(r)\to\infty$ as $r\to 0$.
Consider the reparametrization of the Euclidean plane given by $\iota\:(r,\theta)\mapsto (r,\theta+\alpha(r))$ in the polar coordinates.
Observe that $\iota$ is not differentiable at the origin, but the metric tensor $g$ induced by $\iota$  is continuous.

\medskip

For more on the subject read the paper of Eugenio Calabi and Philip Hartman \cite{calabi-hartman}. 

\parbf{\ref{ex:volume-preserving+short}};
\ref{SHORT.ex:volume-preserving+short:injective}.
Suppose $p=f(x)=f(y)$ and the points $x,y\in \spc{M}$ are distinct.
Since $f$ is short, we get for any $r>0$ the ball $\oBall(p,r)_{\spc{N}}$ contains the images of $\oBall(x,r)_{\spc{M}}$ and $\oBall(y,r)_{\spc{M}}$.
Since $f$ is volume-preserving, we get
\[
\vol\oBall(x,r)_{\spc{M}}
+
\vol\oBall(y,r)_{\spc{M}}
\le
\vol\oBall(p,r)_{\spc{N}}.
\eqlbl{vol+vol<vol}\]

By \ref{obs:lip-chart}, for any $\eps>0$ and all sufficiently small $r>0$ the volumes of the balls  $\oBall(x,r)_{\spc{M}}$, $\oBall(y,r)_{\spc{M}}$ and $\oBall(p,r)_{\spc{N}}$, lie in the range $\omega_n\cdot e^{\mp2\cdot n\cdot\eps}\cdot r^n$, where $\omega_n$ denotes the volume of the unit ball in the $n$-dimensional Euclidean space.
The latter contradicts \ref{vol+vol<vol} for appropriate choice of $\eps$ and $r$.

\parit{\ref{SHORT.ex:volume-preserving+short:bi}.}
Denote by $\sigma(r,a)$ the volume of union of two $r$-balls in the $n$-dimensional Euclidean space such that the distance between their centers is $a$.
Observe that the function $(a,r)\mapsto \sigma(r,a)$ is continuous and increasing in $a$ and $r$ for $a\le r$.
Further, note that
\[\sigma(\lambda\cdot r,\lambda\cdot a)=\lambda^n\cdot \sigma(r,a)\]
for any $\lambda>0$.

Choose a point $z\in \spc{M}$ and small $\eps>0$.
By \ref{obs:lip-chart} there is $R>0$ such that $\oBall(z,10\cdot R)$ admits a $e^{\mp\eps}$-bilipschitz map to the $n$-dimensional Euclidean space.

Choose $x,y\in \oBall(z, R)$.
The argument used in part \ref{SHORT.ex:volume-preserving+short:injective} implies that 
\[e^{-n\cdot\eps}\cdot \sigma(e^{-\eps}\cdot r, e^{-\eps}\cdot \dist{x}{y}{\spc{M}})
\le 
e^{n\cdot\eps}\cdot \sigma(e^{\eps}\cdot r, e^{\eps}\cdot \dist{f(x)}{f(y)}{\spc{N}}).
\eqlbl{eq:v(r,a)}\]
This inequality implies a lower bound on $\dist{f(x)}{f(y)}{\spc{N}}$ in terms of $\dist{x}{y}{\spc{M}}$.

Use the listed properties of the function $(a,r)\mapsto \sigma(r,a)$ to show that for any $c<1$ there is $\eps>0$ such that \ref{eq:v(r,a)} implies that $b>c\cdot a$ for all sufficiently small $a$.

Finally, since $\spc{M}$ and  $\spc{N}$ are length-metric spaces, part~\ref{SHORT.ex:volume-preserving+short:bi} implies that $f$ is locally distance preserving.
(An inclusion map from a nonconvex open subset to the plane gives an example of volume preserving short map that is not distance preserving.)


\medskip

A more general result is discussed by Paul Creutz and Elefterios Soultanis \cite{creutz-soultanis}.


\parbf{\ref{ex:compact-interior}.} Denote by $\spc{M}$ and $\spc{M}^\circ$ the space of $(M,g)$ and $(M^\circ,g)$;
further denote by $\bar{\spc{M}}^\circ$ the completion of $\spc{M}^\circ$.
Observe that the inclusion $M^\circ\hookrightarrow M$ induces a short onto map $\iota\:\bar{\spc{M}}^\circ\z\to\spc{M}$.

Recall that $M$ is bounded by hypersurface that is locally a graph.
Use it to show that any sufficiently short curve $\gamma$ in $(M,g)$ can be approximated by a curve in $\spc{M}^\circ$ with $g$-length arbitrary close to $\length_g\gamma$.
Conclude that $\iota$ is an isometry.


\parbf{\ref{ex:besikovitch=}.}
From the proof of Besicovitch inequality, one can see that the restriction of $\bm{f}$ to the interior of $\spc{M}$ is
(1) volume-preserving, and 
(2) its differential $d_p\bm{f}\:\T_p\to \T_{\bm{f}(p)}$ is an isometry for almost all $p$.

Since $\bm{f}$ is Lipschitz, (2) can be used to show that $\bm{f}$ is short.
It remains to apply \ref{ex:volume-preserving+short} and \ref{ex:compact-interior}.

\parbf{\ref{ex:hexagon}.}
Consider the hexagon with flat metric and curved sides shown on the diagram.
Observe that its area can be made arbitrarily small while keeping the distances from the opposite sides at least 1.

\begin{Figure}
\begin{minipage}{.48\textwidth}
\centering
\includegraphics{mppics/pic-27}
\end{minipage}\hfill
\begin{minipage}{.48\textwidth}
\centering
\includegraphics{mppics/pic-23}
\end{minipage}
\vskip-4mm
\end{Figure}

\parbf{\ref{ex:cylinder};} \ref{SHORT.ex:cylinder:besicovitch}.
Let $\alpha$ be a shortest curve that runs between the boundary components of the cylinder.
Cut the cylinder along $\alpha$.
We get a square with Riemannian metric on it $(\square,g)$.

Two opposite sides of $\square$ correspond to the boundary components of the cylinder.
The other pair corresponds to the sides of the cut.
By assumption, the $g$-distance between the first pair of sides is at least 1.

Consider a shortest curve $\beta$ that connects this pair of sides;
let us keep the same notation for the projection of $\beta$ in the cylinder.

Note that a cyclic concatenation $\gamma$ of $\beta$ with an arc of $\alpha$ is homotopic to a boundary circle.
Therefore $\length_g\gamma\ge1$.
Since $\alpha$ is a shortest path, its arc cannot be longer than any curve connecting its ends; therefore 
\[\length_g\beta\ge \tfrac 12\cdot\length_g \gamma\ge \tfrac 12.\]
That is, the other pair of sides of $\square$ lies on $g$-distance at least $\tfrac12$ from each other.
By \ref{thm:besikovitch+}, $\area(\square,g)\ge \tfrac12$, hence the result.

\parit{\ref{SHORT.ex:cylinder:coarea}.}
Note that any curve in the cylinder that is bordant to a boundary component has length at least $1$.
Therefore if $0\le t\le  1$, then the level sets 
\[L_t=\set{x\in \mathbb{S}^1\times[0,1]}{\distfun_{\mathbb{S}^1\times\{0\}}(x)_{g}=t}\] have length at least $1$.
Applying the coarea inequality, we get that
\[\area(\mathbb{S}^1\times[0,1],g)\ge 1.\]

\parbf{\ref{ex:gadograph}}; \ref{SHORT.ex:gadograph-besikovitch}.
Argue the same way as in \ref{thm:besikovitch}, but observe in addition that $\vol \Sigma=\vol \bm{f}(\Sigma)=0$ and use it time to time.

\parit{\ref{SHORT.ex:gadograph-gadograph}.}
Without loss of generality, we may assume that $V$ lies in a unit cube~$\square$.
Consider a noncontinuous metric tensor $\bar g$ on $\square$ that coincides with $g$ inside $V$ and with the canonical flat metric tensor outside of~$V$.

Observe that the $\bar g$-distances between opposite faces of $\square$ are at least 1.
Indeed this is true for the Euclidean metric and the assumption $\dist{p}{q}{g}\ge\dist{p}{q}{\EE^d}$  guarantees that one cannot make a shortcut in~$V$.
Therefore, the $\bar g$-distances between every pair of opposite faces is at least as large as 1 which is the Euclidean distance.

Applying part \ref{SHORT.ex:gadograph-besikovitch}, we get that $\vol(\square,\bar g)\ge \vol\square$.
Whence the statement follows.


\parbf{\ref{ex:involution-of-sphere}.}
Let $x\in \mathbb{S}^2$ be a point that minimize the distance $|x-x'|_g$.
Consider a shortest path $\gamma$ from $x$ to $x'$.
We can assume that 
\[|x-x'|_g=\length \gamma=1.\]

Let $\gamma'$ be the antipodal arc to $\gamma$.
Note that $\gamma'$ intersects $\gamma$ only at the common endpoints $x$ and $x'$.
Indeed, if $p'=q$ for some $p,q\in\gamma$, then $|p-q|\ge 1$.
Since $\length \gamma=1$, the points $p$ and $q$ must be the ends of $\gamma$.

It follows that $\gamma$ together with $\gamma'$ forms a closed simple curve in $\mathbb{S}^2$;
it divides the sphere into two disks $D$ and $D'$.

Let us divide $\gamma$ into two equal arcs $\gamma_1$ and $\gamma_2$; each of length $\tfrac12$.
Suppose that $p,q\in\gamma_1$, then 
\begin{align*}
|p-q'|_g&\ge |q-q'|_g-|p-q|_g\ge
\\
&\ge 1-\tfrac12=\tfrac12.
\end{align*}
That is, the minimal distance from $\gamma_1$ to $\gamma_1'$ is at least~$\tfrac12$.
The same way we get that the minimal distance from $\gamma_2$ to $\gamma_2'$ is at least~$\tfrac12$.
By Besicovitch inequality, we get that 
\[\area(D,g)\ge \tfrac14\quad\text{and}\quad \area(D',g)\ge \tfrac14.\]
Therefore 
\[\area(\mathbb{S}^2,g)\ge\tfrac12.\]

\parit{A better estimate.}
Let us indicate how to improve the obtained bound to
\[\area(\mathbb{S}^2,g)\ge1.\]

Suppose $x$, $x'$, $\gamma$ and $\gamma'$ are as above.
Consider the function
\[f(z)=\min_t \{\,|\gamma'(t)-z|_g+t\,\}.\]
Observe that $f$ is 1-Lipschitz.

Show that two points $\gamma'(c)$ and $\gamma(1-c)$ lie on one connected component of the level set $L_c=\set{z\in\mathbb{S}^2}{f(z)=c}$;
in particular 
\[\length L_c\ge 2\cdot|\gamma'(c)-\gamma(1-c)|_g.\]
By the triangle inequality, we have that
\begin{align*}
|\gamma'(c)-\gamma(1-c)|_g&\ge 1-|\gamma(c)-\gamma(1-c)|_g=
\\
&=1-|1-2\cdot c|.
\end{align*}

The coarea inequality (\ref{cor:coarea})
\[\area(\mathbb{S}^2,g)\ge \int\limits_0^1\length L_c\cdot dc\]
finishes the proof.


The bound $\tfrac12$ was proved by Marcel Berger \cite{berger}. 
Christopher Croke conjectured that the optimal bound is $\tfrac4\pi$ and the round sphere is the only space that achieves this \cite[Conjecture 0.3 in][]{croke} --- if you solved the last part of the problem, then publish the result.

\begin{wrapfigure}{r}{20 mm}
\vskip-0mm
\centering
\includegraphics{mppics/pic-1305}
\end{wrapfigure}

\parbf{\ref{ex:involution-of-3sphere}.}
Given $\eps>0$, construct a disk $\Delta$ in the plane with 
\begin{align*}
\length\partial \Delta&<10
\intertext{and}
\area \Delta&<\eps
\end{align*}
that admits an continuous involution $\iota$ such that 
\[|\iota(x)-x|\ge 1\]
for any $x\in\partial \Delta$.

An example of $\Delta$ can be guessed from the picture;
the involution $\iota$ makes a length preserving half turn of its boundary $\partial \Delta$.


Take the product $\Delta\times \Delta\subset \EE^4$;
it is homeomorphic to the 4-ball.
Note that 
$$\vol_3[\partial(\Delta\times \Delta)]=2\cdot\area \Delta\cdot\length \partial \Delta<20\cdot\eps.$$
The boundary $\partial(\Delta\times \Delta)$ is homeomorphic to $\mathbb{S}^3$
and the restriction of the involution $(x,y)\z\mapsto (\iota(x),\iota(y))$ has the needed property.

It remains to smooth $\partial(\Delta\times \Delta)$ a  bit.

\parit{Remark.} This example is given by Christopher Croke \cite{croke}.
Note that according to \ref{thm:sys+}, 
the involution $\iota$ cannot be made isometric.

\parbf{\ref{ex:GH-vol}.}
Note that if $(M,g_\infty)$ is $e^{\mp\eps}$-bilipschitz to a cube, then applying Besicovitch inequality, we get that 
\[\liminf_{n\to\infty} \vol (M,g_n)\ge e^{-n\cdot \eps}\cdot\vol (M,g_\infty).\]

By the Vitali covering theorem, given $\eps>0$, we can cover the whole volume of $(M,g_\infty)$ by $e^{\pm\eps}$-bilipschitz cubes.
Applying the above observation and summing up the results, we get that 
\[\liminf_{n\to\infty} \vol (M,g_n)\ge e^{-n\cdot \eps}\cdot\vol (M,g_\infty).\]
The statement follows since $\eps$ is an arbitrary positive number.

To solve the second part of the exercise, start with $g_\infty$ and construct $g_n$ by  adding many tiny bubbles.
The volume can be increased arbitrarily with an arbitrarily small change of metric.

\parit{Remark.}
A more general result was obtained by Sergei Ivanov~\cite{ivanov-1997}.
Note that the statement does not hold true for Gromov--Hausdorff convergence.
In fact any compact metric space $\spc{X}$ can be GH-approximated by a Riemannian surface with an arbitrarily small area.
To show the latter statement, approximate $\spc{X}$ by a finite graph $\Gamma$, embed $\Gamma$ isometrically to the Euclidean space, and pass to the surface of its neighborhood.

\parbf{\ref{ex:sysT2}.}
Set $s=\sys(\TT^2,g)$.

Cut $\TT^2$ along a shortest closed noncontractible curve $\gamma$.
We get a cylinder $(\mathbb{S}^1,g)$ with a Riemannian metric on it.

Applying the argument in \ref{ex:cylinder:besicovitch}, we get that the $g$-distance between the boundary components is at least $\tfrac s2$.
Then \ref{ex:cylinder:besicovitch} implies that the area of torus is at least $\tfrac{s^2}2$.

\parit{Remark.}
The optimal bound is $\tfrac{\sqrt{3}}{2}\cdot s^2$; see  \ref{sec:besicovitch-remarks}.



\parbf{\ref{ex:sysRP2}.}
Set $s\z=\sys (\RP^2,g)$.
Cut $(\RP^2,g)$ along a shortest noncontractible curve $\gamma$.
We obtain $(\DD^2,g)$ --- a disc with metric tensor which we still denote by $g$.

Divide $\gamma$ into two equal arcs $\alpha$ and $\beta$.
Denote by $A$ and $A'$ the two connected components of the inverse image of $\alpha$.
Similarly denote by $B$ and $B'$ the two connected components of the inverse image of $\beta$.

\begin{Figure}
\vskip-0mm
\centering
\includegraphics{mppics/pic-25}
\end{Figure}

Let $\gamma_1$ be a path from $A$ to $A'$;
map it to $\RP^2$ and keep the same notation for it.
Note that $\gamma_1$ together with a subarc of $\alpha$ forms a closed noncontractible curve in $\RP^2$.
Since $\length\alpha=\tfrac s2$, we have that $\length\gamma_1\ge \tfrac s2$.
It follows that the distance between $A$ and $A'$ in $(\DD^2,g)$ is at least $\tfrac s2$.
The same way we show that the distance between $B$ and $B'$ in $(\DD^2,g)$ is at least $\tfrac s2$.

Note that $(\DD^2,g)$ can be parameterized by a square with sides $A$, $B$, $A'$ and $B'$ and apply \ref{thm:besikovitch} to show that 
\[\area(\RP^2,g)=\area(\DD^2,g)\ge \tfrac14\cdot s^2.\]

\parit{Remark.}
The optimal bound is $\tfrac2 \pi\cdot s^2$; see  \ref{sec:besicovitch-remarks}.
In fact any Riemannian metric on the disc with the boundary globally isometric to a unit circle with angle metric has the area at least as large as the unit hemisphere.
It is expected that the same inequality holds for any compact surface with connected boundary (not necessarily a disc);
this is the so-called \index{filling area conjecture}\emph{filling area conjecture} \cite[it is mentioned Mikhael Gromov in 5.5.$\mathrm{B}'(\mathrm{e}')$ of][]{gromov-1983}.

\parbf{\ref{ex:sysSg}.} Cut the surface along a shortest noncontractible curve $\gamma$. 
We might get a surface with one or two components of the boundary.
In these two cases repeat the arguments in \ref{ex:sysRP2} or \ref{ex:sysT2} using \ref{thm:besikovitch+} instead of \ref{thm:besikovitch}.


\parbf{\ref{ex:sysS2xS1}.} Consider the product of a small 2-sphere with the unit circle.

\parbf{\ref{ex:besikovitch++}.}
Apply the same construction as in the original Besicovitch inequality, assuming that the target rectangle
$[0,d_1]\times\dots\times [0,d_n]$ equipped with the metric induced by the $\ell^\infty$ norm;
apply \ref{prop:bilip-measure} where it is appropriate.

\parbf{\ref{ex:2top-discs}.} Suppose that $\Delta_1\ne\Delta_2$.
Consider the map $f\:\mathbb{S}^n\to \spc{X}$ such that the restriction to north and south hemispheres describe $\Delta_1$ and $\Delta_2$ respectively.
Show that if $\Delta_1\ne\Delta_2$, then $\mathbb{S}^n$ can be parameterized by the boundary of the unit cube $\square$ in such a way that for any pair $A$, $A'$ of opposite faces their images $f(A)$, $f(A')$ do not overlap.

Since $\spc{X}$ is contractible, the map $f$ can be extended to a map of the whole cube.
By \ref{ex:besikovitch++} 
\[\haus_{n+1}[f(\square)]>0,\]
a contradiction.

\refstepcounter{chapter}
\setcounter{eqtn}{0}

\parbf{\ref{ex:macrodimension}.}
The following claim resembles Besicovitch inequality;
it is key to the proof:
\begin{itemize}
 \item[$({*})$] Let $a$ be a positive real number.
 Assume that a closed curve $\gamma$ in a metric space $\spc{X}$ can be subdivided into 4 arcs $\alpha$, $\beta$, $\alpha'$, and $\beta'$ in such a way that 
 \begin{itemize}
 \item $|x-x'|>a$ for any $x\in\alpha$ and $x'\in \alpha'$
 and
 \item $|y-y'|>a$ for any $y\in\beta$ and $y'\in \beta'$.
 \end{itemize}
 Then $\gamma$ is not contractible in its $\tfrac a2$-neighborhood.
\end{itemize}

To prove $({*})$, consider two functions defined on $\spc{X}$ as follows:
\begin{align*}
w_1(x)&=\min \{\,a,\distfun_{\alpha}(x)\,\}
\\
w_2(x)&=\min \{\,a,\distfun_{\beta}(x)\,\}
\end{align*}
and the map $\bm{w}\:\spc{X}\to [0,a]\times[0,a]$, defined by
\[\bm{w}\:x\mapsto(w_1(x),w_2(x)).\]

Note that 
\begin{align*}
\bm{w}(\alpha)&=0\times [0,a],
&
\bm{w}(\beta)&=[0,a]\times 0,
\\
\bm{w}(\alpha')&=a\times [0,a],
&
\bm{w}(\beta')&=[0,a]\times a.
\end{align*} 
Therefore, the composition $\bm{w}\circ\gamma$ is a degree 1 map 
\[\mathbb{S}^1\to \partial([0,a]\times[0,a]).\] 
It follows that if $h\:\DD\to \spc{X}$ shrinks $\gamma$, then there is a point $z\in\DD$ such that 
$\bm{w}\circ h(z)=(\tfrac a2,\tfrac a2)$.
Therefore, $h(z)$ lies at distance at least $\tfrac a2$ from $\alpha$, $\beta$, $\alpha'$, $\beta'$
and therefore from $\gamma$.
It proves the claim.

\medskip

Coming back to the problem, let $\{W_i\}$ be an open covering of the real line with multiplicity $2$ and $\rad W_i<R$ for each $i$;
for example take the covering by the intervals $((i-\tfrac23)\cdot R,(i+\tfrac23)\cdot R)$.

Choose a point $p\in \spc{X}$.
Denote by $\{V_j\}$ the connected components of $\distfun_p^{-1}(W_i)$ for all $i$.
Note that $\{V_j\}$ is an open finite cover of $\spc{X}$ with multiplicity at most 2.
It remains to show that $\rad V_j<100\cdot R$ for each $j$.

\begin{Figure}
\vskip-0mm
\centering
\includegraphics{mppics/pic-1310}
\end{Figure}

Arguing by contradiction assume there is a pair of points  $x,y\in V_i$ 
such that $|x\z-y|_{\spc{X}}\z\ge 100\cdot R$.
Connect $x$ to $y$ with a curve $\tau$ in $V_j$.
Consider the closed curve $\sigma$ formed by $\tau$ and two shortest paths $[px]$, $[py]$.


Note that $|p-x|>40$.
Therefore, there is a point $m$ on $[px]$ such that $|m-x|=20$.

By the triangle inequality, the subdivision of $\sigma$ into the arcs $[pm]$, $[mx]$, $\tau$ and $[yp]$ satisfy the conditions of the claim $({*})$ for $a=10\cdot R$,
hence the statement.

\begin{Figure}
\vskip0mm
\centering
\includegraphics{mppics/pic-1315}
\end{Figure}

\parit{The quasiconverse} does not hold.
As an example take a surface that looks like a long cylinder with closed ends;
it is a smooth surface diffeomorphic to a sphere.
Assuming the cylinder is thin, it has macroscopic dimension 1 at a given scale.
However, a circle formed by a section of cylinder around its midpoint by a plane parallel to the base is a circle that cannot be contracted in its small neighborhood.

\parit{Source:} \cite[Appendix $1(\text{E}_{2})$]{gromov-1983}.

\parbf{\ref{ex:width=suprad(inv)}}; \textit{only-if part.}
Suppose $\width_n\spc{X}<R$.
Consider a covering $\{V_1,\dots,V_k\}$ of $\spc{X}$ guaranteed by the definition of width.
Let $\spc{N}$ be its nerve and $\bm{\psi}\:\spc{X}\to \spc{N}$ be the map provided by \ref{prop:space->nerve}.

Since the multiplicity of the covering is at most $n+1$, we have $\dim \spc{N}\le n$.

Note that if $x\in \spc{N}$ lies in a star of a vertex $v_i$,
then $\bm{\psi}^{-1}\{x\}\z\subset V_i$;
in particular, we have $\rad[\bm{\psi}^{-1}\{x\}]<R$.

\parit{If part.}
Choose $x\in \spc{N}$.
Since the inverse image $\bm{\psi}^{-1}\{x\}$ is compact, $\bm{\psi}$ is continuous, and $\rad[\bm{\psi}^{-1}\{x\}]<R$,
there is a neighborhood $U\ni x$ such that the  $\rad[\bm{\psi}^{-1}(U)]<R$.

Since $\spc{X}$ is compact,  there is a finite cover $\{U_i\}$ of $\spc{N}$ such that $\bm{\psi}^{-1}(U_i)\subset\spc{X}$ has a radius smaller than $R$ for each $i$.
Since $\spc{N}$ has dimension $n$, we can inscribe%
\footnote{Recall that a covering $\{W_i\}$ is inscribed in the covering $\{U_i\}$ if for every $W_i$ is a subset of some $U_j$.} 
in $\{U_i\}$ a finite open cover $\{W_i\}$ with multiplicity at most $n+1$.
It remains to observe that $V_i=\bm{\psi}^{-1}(W_i)$ defines a finite open cover of $\spc{X}$ with  multiplicity at most $n+1$ and $\rad V_i<R$ for any $i$. 

\parbf{\ref{ex:1D-case}.}
Assume that $\spc{P}$ is connected.

Let us show that $\diam\spc{P}<R$.
If this is not the case, then there are points $p,q\in\spc{P}$ on distance $R$ from each other.
Let $\gamma$ be a shortest path from $p$ to $q$.
Clearly $\length\gamma\ge R$ and $\gamma$ lies in $\oBall(p,R)$ except for the endpoint $q$.
Therefore, $\length[\oBall(p,R)_{\spc{P}}]\ge R$.
Since $\VolPro_{\spc{P}}(R)\z\ge \length[\oBall(p,R)_{\spc{P}}]$,
the latter contradicts $\VolPro_{\spc{P}}(R)<R$.

In general case, we get that each connected component of $\spc{P}$ has a radius smaller than $R$.
Whence the width of $\spc{P}$ is smaller than $R$.

\parit{Second part.} Again, we can assume that $\spc{P}$ is connected.

The examples of line segment or a circle show that the constant $c=\tfrac12$ cannot be improved.
It remains to show that the inequality holds with $c=\tfrac12$.

Choose $p\in\spc{P}$ such that the value
\[\rho(p)=\max\set{\dist{p}{q}{\spc{P}}}{q\in\spc{P}}\]
is minimal.
Suppose $\rho(p)\ge\tfrac 12\cdot R$.
Observe that there is a point $x\z\in \spc{P}\backslash\{p\}$ that lies on any shortest path starting from $p$ and length $\ge\tfrac 12\cdot R$.
Otherwise for any $r\in(0,\tfrac 12\cdot R)$ there would be at least two points on distance $r$ from $p$;
by coarea inequality we get that the total length of $\spc{P}\cap \oBall(p,\tfrac 12\cdot R)$ is at least $R$ --- a contradiction.

Moving $p$ toward $x$ reduces $\rho(p)$ which contradicts the choice of~$p$.

\parbf{\ref{ex:sys<width}.}
The inequality $6\cdot R<s$ used twice:
\begin{itemize}
\item to shrink the triangle $[p_ip_jp_k]$ to a point;
\item to extend the constructed homotopy on $\spc{M}^0$ to $\spc{M}^1$.
\end{itemize}

The first issue can be resolved by passing to a barycentric subdivision of $\spc{N}^2$;
denote by $v_{ij}$ and $v_{ijk}$ the new vertices in the subdivision that correspond to edge $[v_iv_j]$ and triangle $[v_iv_jv_k]$ respectively.

Further for each vertex $v_{ij}$ choose a point $p_{ij}\in V_i\cap V_j$ and set $f(v_{ij})=p_{ij}$.
Similarly for each vertex $v_{ijk}$ choose a point $p_{ijk}\z\in V_i\cap V_j\cap V_k$ and set $f(v_{ijk})=p_{ijk}$.

Note that 
\begin{align*}
|p_i-p_{ij}|&<R,
\\
|p_i-p_{ijk}|&<R,
\\
|p_{ij}-p_{ijk}|&<2\cdot R.
\end{align*}
Therefore, perimeter of the triangle $[p_ip_{ij}p_{ijk}]$ in the subdivision is less that $4\cdot R$.
It resolves the first issue.

The second issue disappears if one estimates the distances a bit more carefully.
 
\parbf{\ref{ex:fillrad-inj}.}
Choose a fine covering of $\spc{M}$ with multiplicity at most $n$.
Choose $\bm{\psi}$ from $\spc{M}$ to the nerve $\spc{N}$ of the covering the same way as in the proof of \ref{thm:sys<width}.

It remains to construct $f\:\spc{N}\to\spc{M}$ and show that $f\circ\bm{\psi}$ is homotopic to the identity map.
To do this, apply the same strategy as in the proof of \ref{thm:sys<width} together with the so-called \index{geodesic cone construction}\emph{geodesic cone construction}
described below.

Let $\triangle$ be a simplex in a barycentric subdivision of $\spc{N}$.
Suppose that a map $f$ is defined on one facet $\triangle'$ of $\triangle$ to $\spc{M}$ and $\oBall(p,r)\supset f(\triangle')$.
Then one can extend $f$ to whole $\triangle$ such that the remaining vertex $v$ maps to $p$.
Namely connect each point $f(x)$ to $p$ by minimizing geodesic path $\gamma_x$ (by assumption it is uniquely defined) and set
\[f
\:
t\cdot x\z+(1-t)\cdot v
\mapsto
\gamma_x(t).\]

\parbf{\ref{ex:connected-sum-essential}.}
Suppose $M$ is an essential manifold and $N$ is an arbitrary closed manifold.
Observe that shrinking $N$ to a point produces a map $N\#M\to M$ of degree 1.
In particular, there is a map $f\:N\#M\to M$ that sends the fundamental class of $N\#M$ to the fundamental class of $M$.

Since $M$ is essential, there is an aspherical space $K$ and a map $\iota\:M\to K$ that sends the fundamental class of $M$ to a nonzero homology class in $K$.
From above, the composition $\iota\circ f\:N\#M\to K$ sends the fundamental class of $N\#M$ to the same homology class in~$K$.


\parbf{\ref{ex:product-essential}.}
Suppose $M_1$ and $M_2$ are essential.
Let $\iota_1\:M_1\to K_1$ and $\iota_2\:M_2\to K_2$ are the maps to aspherical spaces as in the definition (\ref{def:essential}).
Show that the map
$(\iota_1,\iota_2)\:M_1\times M_2\to K_1\times K_2$
meets the definition.

\parit{Remarks.}
Choose a group $G$.
Note that there is an aspherical connected space CW-complex $K$ with fundamental group $G$.
The space $K$ is called an \index{K(G,1) space@$K(G,1)$ space}\emph{Eilenberg--MacLane space of type $K(G,1)$}, or briefly a $K(G,1)$ space.
Moreover it is not hard to check that
\begin{itemize}
\item $K$ is uniquely defined up to a weak homotopy equivalence;
\item if $\spc{W}$ is a connected finite CW-complex.
Then any homomorphism $\pi_1(\spc{W},w)\to\pi_1(K,k)$ is induced by a continuous map $\phi\:(\spc{W},w)\to(K,k)$.
Moreover, $\phi$ is uniquely defined up to homotopy equivalence.
\end{itemize}

\begin{itemize}
 \item Suppose that $M$ is a closed manifold, 
$K$ is a $K(\pi_1(M),1)$ space and a map $\iota\:M\to K$ induces an isomorphism of fundamental groups.
Then $M$ is essential if and only if $\iota$ sends the fundamental class of $M$ to a nonzero homology class of $K$.
\end{itemize}

The property described in the last statement is the original definition of essential manifold.
It can be used to prove a converse to the exercise;
namely \textit{the product of a nonessential closed manifold with any closed manifold is \emph{not} essential}.





\end{multicols}
}

\newgeometry{top=0.9in, bottom=0.9in,left=0.9in, right=0.9in, paperwidth=6in, paperheight=9in}

{\small\sloppy
%\RequirePackage{snapshot}
\RequirePackage{snapshot}
\makeatletter
\def\snap@providesfile#1[#2]{%
  \wlog{File: #1 #2}%
  \if\expandafter\snap@graphic@test\expanded{#2}@@\@nil
    \snap@record@graphic#1\relax #2 (type ??)\@nil
  \else
    \expandafter\xdef\csname ver@#1\endcsname{#2}%
  \fi
  \endgroup
}
\makeatother

\documentclass[twoside]{book}

\usepackage{lectures}
\usepackage[colorlinks=true,
citecolor=black,
linkcolor=black,
anchorcolor=black,
filecolor=black,
menucolor=black,
urlcolor=black,
pdftitle={Metric geometry on manifolds: two lectures},
pdfsubject={Geometry},
pdfauthor={Anton Petrunin}
]{hyperref}
\makeindex

\begin{document}
%\pagestyle{empty}
 
\title{Metric geometry on manifolds:
\\ two lectures}
\author{Anton Petrunin}
\date{}
\maketitle

We discuss Besicovitch inequality, width, and systole of manifolds.
I assume that students are familiar with 
measure theory,
smooth manifolds,
degree of map, 
CW-complexes and related notions.

These are two final lectures of a graduate course given at Penn State, Spring 2020.
The complete lectures can be found on the author's website;
it includes an introduction to metric geometry \cite{petrunin2020pure}
and elements of Alexandrov geometry based on \cite{alexander-kapovitch-petrunin-2019}.

\thispagestyle{empty}
\tableofcontents
\thispagestyle{empty}

%%%%%%%%%%%%%%%%%%%%%%%%%%%%
%\addtocounter{chapter}{-1}
\chapter{Homework assignments}


It is better to think about all the problems, but you do not have to solve \emph{all} of them.
If a problem is solved, you do not have to write its solutions, but try sketch it.

\section{Due Tue Jan 21}

Exercises: \sout{\ref{ex:almost-min},} \ref{ex:non-contracting-map}, \ref{ex:no-geod}, \sout{\ref{ex:compact=>complete},} \ref{exercise from BH}, \ref{ex:Hausdorff-bry}.

\section{Due Tue Jan 28}

Exercises: \ref{ex:almost-min},  \ref{ex:compact=>complete}, \ref{ex:Huas-perimeter-area}, \ref{ex:GH-po}, \ref{pr:doubling}, \ref{pr:under:if}.

\section{Due Tue Feb 4}
Exercises: 
\ref{ex:compact-length}, 
\ref{pr:under:only-if}, 
\sout{\ref{ex:GH-SC},}
\sout{\ref{ex:sphere-to-ball},}
\ref{ex:ultrapower}, 
\ref{ex:two-geodesics-in-ultrapower}.

\section{Due Tue Feb 11}

Finish exercises \ref{ex:compact-length} , \ref{pr:under:only-if}, \ref{ex:GH-SC}, \ref{ex:sphere-to-ball}.

\noindent
Exercises: \ref{ex:lim(tree)}, \ref{ex:Asym(Lob)}, \ref{ex:geodesics-urysohn}, \ref{ex:sphere-in-urysohn}.

\section{Due Tue Feb 18}

Exercises: \ref{ex:compact-extension}, \ref{ex:+-c}, \ref{ex:ultrametric}, \ref{ex:injective-spaces}, \ref{ex:tripod+square}, \ref{ex:4-on-a-line}.

\noindent Write down a solution of at least one of the exercises.

\section{Due Tue Feb 25}

Finish Exercise \ref{ex:tripod+square:square}.
Prepare questions for review on Tuesday.

\section{Due Tue Mar 3}

Exercises: \ref{ex:sba-2+2-short}, \ref{ex:(3+1)-expanding}, \ref{ex:CAT+CBB}, \ref{ex:product-CBB}, \sout{\ref{ex:CBB-geodesic},} \ref{ex:fat-triangle}.

\noindent Write down a solution of at least one of the exercises.

\section{Due Tue Mar 17}

Exercises: \ref{ex:tringle-inq-angles},
\ref{ex:CBB-geodesic},
\ref{ex:convex-dist},
\ref{ex:reshetnyak-doubling},
\ref{ex:supporting-planes},
\ref{ex:centrally-simmetric-walls}.

\noindent Write down a solution of at least one of the exercises.

\section{Due Tue Mar 24}

Exercises: 
\ref{ex:contractible},
\ref{ex:convex-nbhd},
\ref{ex:closest-point},
\ref{cor:balls:dim=1},
\ref{ex:null-homotopic},
\ref{ex:branching-cover}.

 Write down as many solutions as you can; email then to Zetian Yan (zxy5156) + cc to me (aqp6).

Each working day I will check email before 15:00 and will appear online if you asked me (it is easy for me --- do not hesitate to ask).
We will meet regular hours online (as we did before).

%%%%%%%%%%%%%%%%%%%%%%%%%%%%

\chapter{Volume bounds} 


\section{Riemannian metrics}

We are going to consider mostly Riemannian spaces;
that is smooth manifolds with metric defined by a metric tensor.
These are specially nice length metrics on manifolds.
However most of the statements we are going to discuss have counterpart for general length metrics on manifolds.

Let $M$ be a smooth manifold.
A \emph{metric tensor} on $M$ is a choice of positive definite quadratic forms $g_p$ on each tangent space $\T_pM$ that depends smoothly on the point $p$.
That is, if we fix a local coordinates on $M$ and write $g$ in this coordinates, then each component of $g$ is a smooth function. 

A Riemannian manifold is a smooth manifold $M$ with a choice \emph{metric tensor} $g$ on it.

The metric tensor $g$ can be used to define length of curves and volume of regions in $M$.

\parbf{Lengths and distances.}
If $\gamma\:[a,b]\to M$ is a piecewise smooth curve then 
\[\length_g\gamma=\int_a^b\sqrt{g(\gamma'(t),\gamma'(t))}\cdot dt.\]
Further we can define a metric on $M$ as least lower bound to lengths of piecewise smooth curves connecting two given points;
the described distance between points $x$ and $y$ will be denoted by $\dist{x}{y}{g}$ or $\distfun_x(y)_g$.
The distance function from a point $x$ will be denoted by $(\distfun_x)_g$ or $\distfun_x$ if the choice of $g$ is evident.

The following claim requires a proof, but we will assume that it is obvious.

\begin{thm}{Claim}
Let $(M,g)$ be a Riemannian manifold.
Then the metric $(x,y)\mapsto \dist{x}{y}{g}$ defines a length metric. Moreover this metric completely determines the metric tensor $g$.
\end{thm}

\parbf{Volume.}
If a region $R$ is covered by one chart $\iota\:U\to M$,
then its volume can be defined as an integral 
\[\vol R
\df
\int_{\iota^{-1}(R)}\sqrt{\det{g}}.\]
In the general case we subdivide $R$ into (a countable collection of) regions $R_1,R_2\dots$ and define
\[\vol R\df \vol R_1+\vol R_2+\dots\]

\section{Besikovitch inequality}

\begin{thm}{Theorem}\label{thm:besikovitch}
Let $g$ be a metric tensor on a unit $n$-dimensional cube $\square^n$.
Suppose that the $g$-distances between the opposite faces of $\square^n$ are at leat $1$; that is, any piecewise smooth curve that connects opposite faces has $g$-length at least $1$.
Then $\vol(\square^n, g)\ge 1$.
\end{thm}

\parit{Proof.}
We will consider the case $n=2$; the other cases are proved the same way.

Denote by $A$, $A'$, and $B$, $B'$ the opposite faces of the square~$\square$.
Consider two function 
\begin{align*}
f_A(x)&\df\min\{\,\distfun_A(x)_g,1\,\},
\\
f_B(x)&\df\min\{\,\distfun_B(x)_g,1\,\}.
\end{align*}
Define $f\:\square\to\square$ as a map with coordinate funcions $f_A$ and $f_B$;
that is, $f(x)\df(f_A(x), f_B(x))$.

Observe that $f$ maps each face to itself.
Indeed, 
\[x\in A \quad\Longrightarrow\quad \distfun_A(x)_g=0 \quad\Longrightarrow\quad f_A(x)=0 \quad\Longrightarrow\quad f(x)\in A.\]
Similarly if $x\in B$, then $f(x)\in B$.
Further, 
\[x\in A'
\quad\Longrightarrow\quad 
\distfun_A(x)_g\ge 1 
\quad\Longrightarrow\quad 
f_A(x)=1 
\quad\Longrightarrow\quad 
f(x)\in A'.\]
Similarly if $x\in B'$, then $f(x)\in B'$.

Therefore 
\[f_t(x)= t\cdot x + (1-t)\cdot f(x)\]
defines a homotopy of maps of pair of spaces $(\square,\partial \square)$ from $f$ to the identity map.
It follows that degree of $f$ is $1$; that is, $f$ sends the fundamental class of $(\square,\partial \square)$ to itself.
In particular $f$ is onto.

Suppose that Jacobian  matrix $\Jac_pf$ of $f$ is defined at $p\in \square$.
Choose an orthonormal basis in $\T_p$ with respect to $g$ and the standard basis in the target $\square$.
Observe that the differentials $d_pf_A$ and $d_pf_B$ written in these basises are the rows of $\Jac_pf$.
Evidently $|d_pf_A|\le 1$ and $|d_pf_B|\le 1$.
Since the determinant of a matrix is the volume of the parallelepiped spanned on its rows, we get 
\[|\det(\Jac_pf)|\le |d_pf_A|\cdot|d_pf_B|\le 1.\]
Since $f\:\square\to\square$ is a Lipschitz onto map, the \emph{area formula} implies that 
\[\vol(\square,g)\ge \vol\square=1.\]
\qedsf

The following generalization can be proved along the same lines.

\begin{thm}{Theorem}\label{thm:besikovitch+}
Let $(M,g)$ be Riemannian manifold and its boundary admits a homeomorphism $\partial\square^n\to\partial M$. 
Suppose $d_1,\dots, d_n$ the distances between the the images of pairs of opposite faces of $\square^n$ in $\partial M$.
Then 
\[\vol(M,g)\ge d_1\cdots d_n.\]
\end{thm}

\begin{thm}{Exercise}\label{ex:besikovitch=}
Suppose that we have equality in \ref{thm:besikovitch}.
Show that $(\square^n,g)$ is isometric to $\square^n$.
\end{thm}

\begin{thm}{Exercise}\label{ex:hexagon}
Suppose $g$ is a metric tensor on a regular hexagon $\varhexagon
   $ such that $g$-distances between the opposite sides are at least $1$.
Is there a positive lower bound on $\area(\varhexagon,g)$?
\end{thm}

\begin{thm}{Exercise}\label{ex:gadograph}
Let $V$ be a compact set in $\EE^d$ bounded by a hypersurface $\Sigma$.
Suppose $g$ is a Riemannian metric on $V$ such that 
\[\dist{p}{q}{g}\ge\dist{p}{q}{\EE^d}\]
for any two points $p,q\in \Sigma$.
Show that
\[\vol(V,g)\ge \vol(V)_{\EE^d}.\]
 
\end{thm}

\begin{thm}{Exercise}\label{ex:involution-of-sphere}
Suppose that sphere with Riemannian matric $(\mathbb{S}^2,g)$ admits an involution $\iota$ such that $\dist{x}{\iota(x)}{g}\ge 1$.

Show that $\area(\mathbb{S}^2,g)\ge \tfrac1{1000}$;
try to show that $\area(\mathbb{S}^2,g)\ge \tfrac12$ or $\area(\mathbb{S}^2,g)\ge 1$.
\end{thm}

Christopher Croke conjectured that the optimal bound for this exercise is $\tfrac4\pi$ and the round sphere is the only space that achieves this \cite[see Conjecture 0.3 in][]{croke}.

\begin{thm}{Advanced exercise}\label{ex:involution-of-3sphere}
Construct a metric $g$ on $\mathbb{S}^3$ with arbitrary small $\vol(\mathbb{S}^3,g)$ and such that it admits an involution $\iota$ such that $\dist{x}{\iota(x)}{g}\ge 1$.
\end{thm}

\section{Systolic inequlaity}

Let $\spc{M}$ be a compact Riemannian manifold.
The \emph{systole} of $\spc{M}$ (brifly $\sys\spc{M}$) is defined to be the least length of a noncontractible closed curve in $\spc{M}$.

Let $\Lambda$ be a set of smooth closed $n$-dimensional manifolds.
We say that a systolic inequality holds for $\Lambda$ if there is a constant $c$ such that for any $M\in \Lambda$ and any metric tenor $g$ on $M$ we have
\[[\sys(M,g)]^n\le c\cdot \vol(M,g).\]

\begin{thm}{Exercise}\label{ex:sysT2}
Use \ref{thm:besikovitch} to show that systolic inequality holds for the 2-torus $\TT^2$.
\end{thm}

\begin{thm}{Exercise}\label{ex:sysRP2}
Use \ref{thm:besikovitch} to show that systolic inequality holds for the real projective palane $\RP^2$.
\end{thm}

\begin{thm}{Exercise}\label{ex:sysSg}
Use \ref{thm:besikovitch+} to show that systolic inequality holds for the set of all closed surfaces of positive genus.
\end{thm}

\parbf{Remarks.}
The optimal constants in the systolic inequality are known in the following three cases:
\begin{itemize}
\item For real projective plane $\RP^2$ the constant is $\tfrac\pi2$ --- the equality holds for a quotient of a round sphere by isometric involution. The statement was prove by Pao Ming Pu \cite{pu}.\label{page:pu}
\item For torus $\TT^2$ the constant is $\tfrac2{\sqrt{3}}$ --- the equality holds for a flat torus obtained from a regular hexagon by identifying opposite sides; this is the so called \emph{Loewner's torus inequality}.
\item For the Klein bottle $\RP^2\#\RP^2$  the constant is $\tfrac\pi{2\cdot\sqrt2}$ --- the equality holds for certain nonsmooth metrics \cite{bavard}.
\end{itemize}
The proofs of these results use the so called \emph{uniformization theorem}   available in the 2-dimensional case only.
These proofs are beautiful, but they too far from metric geometry.
A good survey on the subject is written by Christopher Croke and Mikhail Katz \cite{croke-katz}.

\begin{thm}{Exercise}\label{ex:sysS2xS1}
Show that systolic inequality does \emph{not} hold for $\mathbb{S}^2\times\mathbb{S}^1$.
\end{thm}


\begin{thm}{Therorem}\label{thm:sys(torus)}
Systolic intequality holds for the $n$-dimensional torus $\TT^n$. 
\end{thm}

The proof of this theorem and its generalization will take most of the remaining lectures.
In the following section we introduce a key notion in the proof.

\section{Filling radius}

The following definition was introduced by Mikhael Gromov \cite{gromov-1983}.

Let $\spc{M}$ be a closed $n$-dimensional Reimannian manifold.
Applying Kuratowski embedding (\ref{lem:kuratowski}) $x\mapsto \distfun_x$, we may think that $\spc{M}$ as a subset of $\ell^\infty(\spc{M})$ --- the space of functions on $\spc{M}$ equipped with the metric induced by the sup-norm.

Define the \emph{filling radius} of $\spc{M}$ (briefly $\FillRad\spc{M}$) as the least upper bound on values $r>0$ such that $\spc{M}$ bounds in its $r$-neighborhood in $\ell^\infty(\spc{M})$.
In other words, if $\iota_r$ denotes inclusion of $\spc{M}$ in its $r$-neighborhood $B_r(\spc{M})\subset \ell^\infty(\spc{M})$,
then 
\[\FillRad\spc{M}\df\inf\set{r>0}{(\iota_r)_*[\spc{M}]=0\in H_n(B_r(\spc{M}))},\]
where $[\spc{M}]$ denotes the fundamental class of $\spc{M}$.

We assume that the homologies are taken with coefficients in $\ZZ_2$.
In this case $[\spc{M}]\ne0\in H_n(\spc{M})$.
If we choose coefficients $\ZZ$, then it does not hold for nonorientable manifolds.


\begin{thm}{Exercise}\label{ex:fillrad<diam/2}
Show that the inequality
\[\FillRad \spc{M}\le \tfrac12\cdot\diam \spc{M}\]
holds for any compact Riemannian manifold $\spc{M}$.
\end{thm}

\parbf{Remark.}
The optimal bound for the above exercise was found by Mikhail Katz \cite{katz}.
Namely he proved that
\[\FillRad \spc{M}\le \tfrac13\cdot\diam \spc{M}\]
and equality holds if $\spc{M}$ is real projective space with canonical metric.
The proof is beautiful, elementary, and very readable.

\medskip

The following theorem is the main ingredient in the proof of \ref{thm:sys(torus)}.
This theorem will be the main subject of the following lecture.

\begin{thm}{Theorem}\label{thm:FillRad<vol}
Given an integer $n>0$, there is a constant $c(n)$ such that inequality
\[(\FillRad \spc{M})^n\le c(n)\cdot \vol \spc{M}\]
holds for any compact $n$-dimensional Riemannian manifold $\spc{M}$.
\end{thm}

In the following section we show why this theorem is related to \ref{thm:FillRad<vol}.

\section{Systole and filling radius}

\begin{thm}{Theorem}\label{thm:sys<FillRad}
Suppose $\spc{T}= (\TT^n,g)$ is a Riemnnian manifold on $n$-dimensional torus $\TT^n$.
Then 
\[\sys\spc{T}\le 6 \cdot \FillRad \spc{T}.\]
\end{thm}

Note that \ref{thm:sys<FillRad} and \ref{thm:FillRad<vol}  imply \ref{thm:sys(torus)}.

\parit{Proof.}
As usual we consider $\spc{T}$ as a subspace in $\ell^\infty(\spc{T})$.

Set $s=\sys\spc{T}$ and $\FillRad\spc{T}=r$.
Arguing by contradiction, assume $6\cdot r< s$;
so $\eps=\tfrac1{100}\cdot(s-6\cdot r)>0$.

Choose a simplicial complex $\Sigma$ and a map $\sigma\:\Sigma\to \ell^\infty(\spc{T})$ such that the restriction $\sigma|_{\partial\Sigma}$
represents the fundamental class $[\spc{T}]$ of $\spc{T}$
and $\sigma(\Sigma)\subset B_{r+\eps}(\spc{T})$.


Passing to barycentric subdivision few times, we may assume that the $\sigma$-image of any simplex in $\Sigma$ has diameter less than $\eps$.
We may perturb the map slightly to ensure that each edge $e$ of $\Sigma$ is mapped to a geodesic and still $\sigma|_{\partial\Sigma}$
represents the fundamental class $[\spc{T}]$ of $\spc{T}$.

Let us construct a continuous map
$f\:\Sigma\to  \spc{T}$ which agrees with $\sigma$ on $\partial \Sigma$.
Once it is done we get that $[\spc{T}]=0\in H_n(\spc{T})$ --- a contradiction.

Set $f(x)=\sigma(x)$ for every $x\in \partial \Sigma$;
on the remaining part of $\Sigma$ we will construct $f$ recurcevely on the skeletons $\Sigma^0$, $\Sigma^1$, $\Sigma^2$ and so on.

For every vertex $v$, set $f(v)$ to be the closest point in $\spc{T}$ to $\sigma(v)$.
Note that if $v\in\partial\Sigma$, then $f(v)=\sigma(v)$.
This way we defined $f$ on $\Sigma^0$.

Let $e$ be an edge in $\Sigma$ between vertexes $v$ and $w$.
Note that 
\begin{align*}
\dist{f(v)}{f(w)}{}
&\le\dist{f(v)}{\sigma(v)}{}
+\dist{\sigma(v)}{\sigma(w)}{}
+\dist{\sigma(w)}{f(w)}{}\le
\\
&\le (r+\eps)+\eps +(r+\eps)<
\\
&<\tfrac s3.
\end{align*}
Map $e$ to a shortest path $[f(v)\,f(w)]$ in $\spc{T}$;
if $e$ is an edge in $\partial \Sigma$ then no need to change $f$ on it.
This extends $f$ to $\Sigma^1$ such that each edge is mapped to a geodesic of length less that $\tfrac s3$.

Now for each triangle $uvw$ in $\Sigma$, the closed curve formed by $f$-images of its sides has length less than $s$.
That is, it is shorter than any noncontractible closed curve,
and therefore it is null-homotopic in $\spc{T}$.
Hence we can extend $f$ to the $\Sigma^2$.

Finally, since $\spc{T}$ is aspherical, there is no obstruction to extending $f$ to the rest of $\Sigma$.
\qeds

Observe that we use only that $\spc{T}$ is aspherical closed manifold;
this statement will be generalized yet further.

\begin{thm}{Exercise}\label{ex:fillrad-inj}
Modify the proof of \ref{thm:sys<FillRad} to prove the following:

Suppose that $\spc{M}$ is a closed $n$-dimensional Reimannian manifold with \emph{injectivity radius} at least $r$; that is, if $\dist{p}{q}{\spc{M}}<r$, then there is geodesic $[pq]_{\spc{M}}$ is uniquely defined.
Show that
\[\FillRad\spc{M}\ge \tfrac{r}{n+1}.\]
 
\end{thm}

Note that this exercise together with bound on filling radius in \ref{thm:FillRad<vol} imply that lower bound on injectivity radius implies a lower bound on volume.


\chapter{Width}

This lecture is based on a paper of Alexander Nabutovsky \cite{nabutovsky}.

\section{Nerves and partition of unity}

Let $\{V_1,\dots,V_k\}$ be a finite open cover of a compact metric space $\spc{X}$.
Consider an abstract simplicial complex $\spc{N}$, with one vertex $v_i$ for each set $V_i$ such that a simplex with vertexes $v_{i_1},\dots, v_{i_m}$ is included in $\spc{N}$ if 
the intersection $V_{i_1}\cap\dots\cap V_{i_m}$ is nonempty.
The obtained simplicial complex $\spc{N}$ called the \index{nerve}\emph{nerve of the covering $\{V_i\}$}.

Note that $\spc{N}$ is a finite simplicial complex;
it is a subcomplex of a simplex with the vertixes $\{v_1,\dots,v_k\}$.
The nerve $\spc{N}$ has dimension at most $n$ if and only if the covering $\{V_1,\dots,V_k\}$ has multiplicity is at most $n+1$;
that is, any point $x\in\spc{X}$ belongs to
at most $n+1$ sets of the covering.

\begin{thm}{Proposition}\label{thm:part-unit}
 Let $\{V_1,\dots,V_k\}$ be a finite open covering of a compact metric space ${\spc{X}}$.
Then there are Lipschitz functions $\psi_i\:{\spc{X}}\z\to[0,1]$ such that
if $\psi_i(x)>0$ then $x\in V_i$ and
$$\sum_i\psi_i(x)=1$$
for any $x\in {\spc{X}}$.
\end{thm}

\parit{Proof.}
Consider functions $\phi_i\:{\spc{X}}\to\RR$ defined as
$$\phi_i(x)=\distfun_{({\spc{X}}\backslash V_i)} x.$$
Note $\phi_i$ is $1$-Lipschitz
for any $i$
and $\phi_i(x)>0$ if and only if $x\in V_i$.
Since $\{V_i\}$ is a covering, we have that
$$\sum_i\phi_i(x)>0\ \ \text{for any}\ \ x\in {\spc{X}}.$$

Set 
$$\psi_k(x)=\frac{\phi_k(x)}{\sum_i\phi_i(x)}.$$
Observe that by construction the functions $\psi_i$ meet the conditions in the proposition.
\qedsf

A collection of functions $\{\psi_i\}$ that meets the conditions in \ref{thm:part-unit} is called 
a \index{partition of unity}\emph{partition of unity subordinate to the open covering} $\{V_1,\dots,V_k\}$.

Suppose $\{\psi_i\}$ is  
a partition of unity subordinate to the open covering $\{V_1,\dots,V_k\}$.
Note that for any point $x\in {\spc{X}}$, the set
$$\set{v_i}{\psi_i(x)>0}$$
describe vertexes of a simplex in the nerve.
Therefore 
$$\psi\:x\mapsto \psi_1(x)\cdot v_1+\psi_2(x)\cdot v_2+\dots+\psi_k(x)\cdot v_n.$$
describes a Lipschitz map from ${\spc{X}}$ to the nerve $\spc{N}$ of $\{V_i\}$;
here the point $x$ is mapped to the point with barycentric coordinates $\psi_i(x)$.
In other words we proved the following:

\begin{thm}{Proposition}\label{prop:space->nerve}
Let $\spc{N}$ be a nerve of an open covering $\{V_1,\z\dots,V_k\}$ of a compact metric space $\spc{X}$.
Denote by $v_i$ the vertex of $\spc{N}$ that corresponds to $V_i$.

Then there is a Lipschitz map $\psi\:\spc{X}\to\spc{N}$ such that $\psi(V_i)\z\subset\Star_{v_i}$ for every $i$;
that is, for any $x\in V_i$ the point $\psi(x)$ lies the interior of some simplex with vertex $v_i$.
\end{thm}


\section{Width}

Suppose $A$ is a subset of a metric space $\spc{X}$.
The radius of $A$ (briefly $\rad A$) is defined as the least upper bound on the values $R>0$ such that $\oBall(x,R)\supset A$ for some $x\in \spc{X}$.

\begin{thm}{Definition}\label{def:width}
Let $\spc{X}$ be a metric space.
The $n$-th width of $\spc{X}$ (briefly $\width_n\spc{X}$) is defined as the least upper bound on values $R>0$ such that $\spc{X}$ admits a finite open covering $\{V_i\}$ with multiplicity at most $n+1$ and $\rad V_i< R$ for each $i$.
\end{thm}

\parit{Remarks.}
\begin{itemize}
\item Observe that 
\[\width_0\spc{X}\ge\width_1\spc{X}\ge\dots\]
for any compact matric space $\spc{X}$.
Moreover, if $\spc{X}$ is connected, then 
\[\width_0\spc{X}=\rad\spc{X}.\]
\item 
Usually width is defined using diameter instead of radius, but the result differ at most twice.
Namely if $r$ is the radius-width and $d$ --- diameter-width for the same $n$, then 
$r\le d\le 2\cdot r$.

\item Note that \index{Lebesgue covering dimension}\emph{Lebesgue covering dimension} of $\spc{X}$ can be defined as the least number $n$ such that $\width_n\spc{X}=0$.
Another closely related notion is the so called \index{macroscopic dimesion on scale $R$}\emph{macroscopic dimesion on scale $R$};
it is defined as the  least number $n$ such that $\width_n\spc{X}<R$.
\end{itemize}

\begin{thm}{Exercise}\label{ex:macrodimension}
Suppose $\spc{X}$ is a compact metric space such that any closed curve $\gamma$ in $\spc{X}$ can be contracted in its $R$-neighborhood.
Show that $\spc{X}$ has macroscopic dimension at most 1 on scale $100\cdot R$.

What about quasiconverse? That is, suppose a simply connected compact metric space $\spc{X}$ has macroscopic dimension at most 1 on scale $R$, is it true that any closed curve $\gamma$ in $\spc{X}$ can be contracted in its $100\cdot R$-neighborhood?
\end{thm}


The following proposition provides an equivalent definition;
we will not use it, but it provides a good reason for the name \index{width}\emph{width}.

\begin{thm}{Proposition}\label{prop:width=suprad(inv)}
Suppose $\spc{X}$ is a compact metric space.
Then $\width_n\spc{X}<R$ if and only if there is a finite $n$-dimensional simplicial complex $\spc{N}$ and a continuous map $\psi\:\spc{X}\to \spc{N}$
such that $\rad[\psi^{-1}(s)]\z<R$
for any $s\in \spc{N}$.
\end{thm}

\parit{Proof; ``only if'' part.}
Suppose $\width_n\spc{X}<R$.
Consider a covering $\{V_1,\dots,V_k\}$ of $\spc{X}$ guaranteed by the definition of width.
Let $\spc{N}$ be its nerve and $\psi\:\spc{X}\to \spc{N}$ be the map provided by \ref{prop:space->nerve}.

Since the multiplicity of the covering is at most $n+1$, we ahve $\dim \spc{N}\le n$.

Note that if $x\in \spc{N}$ lies in a star of a vertex $v_i$,
then $\psi^{-1}\{x\}\z\subset V_i$;
in particular $\rad[\psi^{-1}\{x\}]<R$.

\parit{``If'' part.}
Choose $x\in \spc{N}$.
Since the inverse image $\psi^{-1}\{x\}$ is compact, $\psi$ is continuous, and $\rad[\psi^{-1}\{x\}]<R$,
there is a neighborhood $U\ni x$ such that the  $\rad[\psi^{-1}(U)]<R$.

Since $\spc{X}$ is compact,  there is a finite cover $\{U_i\}$ of $\spc{N}$ such that $\psi^{-1}(U_i)\subset\spc{X}$ has radius smaller than $R$ for each $i$.
Since $\spc{N}$ has dimension $n$, we can inscribe%
\footnote{Recall that a covering $\{W_i\}$ is inscribed in the covering $\{U_i\}$ if for every $W_i$ is a subset of some $U_j$.} 
in $\{U_i\}$ a finite open cover $\{W_i\}$ with multiplicity at most $n+1$.
It remains to observe that $V_i=\psi^{-1}(W_i)$ defines a finite open cover of $\spc{X}$ with radius less than $R$ and multiplicity at most $n+1$. 
\qeds

\section{Riemannian polyhedrons}

A \index{Riemannian polyhedron}\emph{Riemannian polyhedron} is defined as a finite simplicial complex with a metric tensor on each simplex such that the restriction of the metric on each simplex to a subsymplex coinsides with the metric on the subsmplex.
The dimension of Riemannian polyhedron is defined as the largest dimension it its triangulation.
For Riemannian polhedron one can define length of curves and volume the same way as for Riemannian manifolds.


Further we will apply the notion of width to compact Riemannian polyhedrons.
If $\spc{P}$ is an $n$-dimensional compact Riemannian polyhedron, then 
we suppose that
\[\width\spc{P}\df\width_{n-1}\spc{P}.\]

Let $\spc{P}$ be an $n$-dimensional Riemnnian polyhedron.
Let us define \index{volume profile}\emph{volume profile} of $\spc{P}$ as a function 
returning volume of largest $r$-ball in $\spc{P}$;
that is, $\VolPro_{\spc{P}}\:\RR_+\to\RR_+$ is defined by 
\[\VolPro_{\spc{P}}(r)\df \sup\set{\vol_n \oBall(p,r)}{p\in\spc{P}}.\]
Note that $\VolPro_{\spc{P}}$ is a nondecreasing function and $\VolPro_{\spc{P}}(r)\z\to\vol_n\spc{P}$ as $r\to\infty$.

Note that if $\spc{P}$ is a 1-dimensional connected Riemannian polyhedron, then 
\[\width\spc{P}=\width_0\spc{P}=\rad\spc{P}.\]

\begin{thm}{Exercise}\label{ex:1D-case}
Suppose $\spc{P}$ be a 1-dimensional Riemannian polyhedron.
Suppose $\VolPro_{\spc{P}}(R)<R$ for some $R>0$.
Show that 
\[\width \spc{P}<R.\]
Try to show that $c=\tfrac 12$ is the optimal constant such that 
\[\width \spc{P}<c\cdot R.\]
\end{thm}

\section{Volume profile bounds width}

The following theorem and its corollary is the main goal of this lecture.

\begin{thm}{Theorem}\label{thm:width<volpro}
Let $\spc{P}$ be an $n$-dimensional Reimannian polyhedron. 
If the inequality 
\[R> n\cdot \sqrt[n]{\VolPro_{\spc{P}}(R)}\]
holds for {}\emph{some} $R>0$, then 
\[\width\spc{P}\le  R.\]
\end{thm}

Since $\VolPro_{\spc{P}}(r)\le \vol\spc{P}$ for any $r$,
we get the following:

\begin{thm}{Corollary}\label{thm:width<vol}
For any $n$-dimensional Reimannian polyhedron $\spc{P}$, we have
\[\width\spc{P}\le n\cdot \sqrt[n]{\vol\spc{P}}.\]

\end{thm}

In the proof of \ref{thm:width<volpro}, we will use the following three technical statements,
the proofs are omitted, but they are not hard. 

\begin{thm}{Smoothing procedure}
Let $\spc{P}$ be a Reimannian polyhedron and $f\:\spc{P}\to \RR$ be a 1-Lipschitz function.
Then for any $\delta>0$ there is a  1-Lipschitz function $\tilde f\:\spc{P}\to \RR$ that is smooth on each simplex of the triangulation and $\delta$-close to $f$.
\end{thm}

\begin{thm}{Sard's theorem}
Let $\spc{P}$ be an $n$-dimensional Reimannian polyhedron and $f\:\spc{P}\to \RR$ be a function that is smooth on each simplex.
Then for almost all values $a$, each component of the inverse image $f^{-1}\{a\}$ equipped with the induced metric is a Reimannian polyhedron of dimension at most $n-1$.
\end{thm}


\begin{thm}{Coarea inequality}
Let $\spc{P}$ be an $n$-dimensional Reimannian polyhedron and $f\:\spc{P}\to \RR$ be a 1-Lipschitz function that is smooth on each simplex.
Set $V=\vol_n (f^{-1}[a,b])$.
Then 
\[\int_a^b\vol_{n-1}(f^{-1}\{x\})\cdot dx\ge V .\]
In particular there is a subset of positive measure $A\subset [a,b]$ such that the inequality 
\[\vol_{n-1}(f^{-1}\{x\})\ge \frac V{b-a}\]
holds for any $x\in A$.
\end{thm}

\begin{thm}{Definition}
Let $\spc{P}$ be an $n$-dimensional Riemannian polyhedron.
An $(n-1)$-dimensional subpolyhedron $\spc{Q}\subset\spc{P}$ is called \index{separating subpolyhedron}\emph{$R$-separating} if $\rad U<R$ for each connected component $U$ of the complement $\spc{P}\backslash \spc{Q}$.
\end{thm}



\begin{thm}{Lemma}\label{lem:separating}
Let $\spc{P}$ be an $n$-dimensional Riemannian polyhedron.
Then given $R>0$ and $\eps>0$ there is a $R$-separating subpolyhedron $\spc{Q}\subset\spc{P}$ such that for any $r_0<r_1\le R$ we have
\[\VolPro_{\spc{Q}}(r_0)< \tfrac1{r_1-r_0}\cdot \VolPro_{\spc{P}}(r_1)+\eps.\]

\end{thm}

\parit{Proof.}
Choose a small $\delta>0$.
Applying the smoothing procedure, we can exchange each distance function $\distfun_p$ on $\spc{P}$ by $\delta$-close smooth 1-Lipschitz function, which will be denoted by $\widetilde \distfun_p$.

By Sard's theorem, almost all level sets 
\[\tilde S_c(p)=\set{x\in \spc{P}}{\widetilde \distfun_p(x)=c}\]
are smooth Riemannian polyhedrons of dimension at most $n-1$.
Since $\delta$ is small, the coarea inequality implies that 
for we can choose $c\z\in(r_0+\delta, r_1-\delta)$ such that $\tilde S_c(p)$ is a subpolyhedron and 
\begin{align*}
\vol_{n-1}\tilde S_c(p)&\le \tfrac1{r_1-r_0-2\cdot\delta}\cdot\vol_n[\oBall(p,r_1)]<
\\
&<\tfrac1{r_1-r_0}\cdot \VolPro_{\spc{P}}(r_1)+\tfrac\eps2.
\end{align*}

Suppose $\spc{Q}$ is an $R$-separating subpolyhedron in $\spc{P}$ with almost minimal volume, say its volume is at most $\tfrac\eps2$-far from the greatest lower bound.
Note that cutting from $\spc{Q}$ everything inside $\tilde S_c(p)$ and adding $\tilde S_c(p)$ keeps it to be $R$-separating subpolyhedron.
Since $\spc{Q}$ has almost minimal volume, we have
\[\vol_{n-1}[\spc{Q}\cap \oBall(p,r_0)_{\spc{P}}]-\tfrac\eps2\le \vol_{n-1}S_c(p).\]
Therefore 
\[\vol_{n-1}[\spc{Q}\cap \oBall(p,r_0)_{\spc{P}}]\le\tfrac1{r_1-r_0}\cdot \VolPro_{\spc{P}}(r_1)+\eps\eqlbl{eq:volQ<ProP}\]
Recall that $\spc{Q}$ is equipped with the induced length metric;
therefore $\dist{p}{q}{\spc{Q}}\ge \dist{p}{q}{\spc{P}}$ for any $p,q\in \spc{Q}$;
in particular, 
\[\oBall(p,r_0)_{\spc{Q}}\subset \spc{Q}\cap \oBall(p,r_0)_{\spc{P}}\]
for any $p\in \spc{Q}$ and $r\ge 0$.
Hence \ref{eq:volQ<ProP} implies the lemma.
\qeds

\begin{thm}{Lemma}\label{lem:separating-width}
Let $\spc{Q}$ be an $R$-separating subpolyhedron in an $n$-dimensional Riemannian polyhedron $\spc{P}$.
Suppose $\width\spc{Q}\le R$.
Then $\width\spc{P}\le R$
\end{thm}

\parit{Proof.}
Start with an open covering $\{V_1,\dots,V_k\}$ of $\spc{Q}$ of multiplicity $\le n$ with radiuses of the sets in the intrinsic metric $\le R$.

Note that $\{V_1,\dots,V_k\}$ can be converted into an an open covering of
a small neighbourhood of $\spc{Q}$ in $\spc{P}$ without increasing the multiplicity.
This is can be done by setting 
\[V_i'=\bigcup_{x\in V_i}\oBall(x,r_x),\]
where $r_x=\tfrac1{10}\cdot\inf\set{\dist{x}{y}{}}{y\in \spc{Q}\backslash V_i}$.

Adding to  $\{V_i'\}$ all the components of $\spc{P}\backslash \spc{Q}$,
we increase the multiplicity by at most 1 and obtain a covering of $\spc{P}$.
The statement follows since $\dim \spc{P}= \dim \spc{Q}\z+1$.
\qeds

\parit{Proof of \ref{thm:width<volpro}.}
We apply induction on the dimension $n=\dim\spc{P}$.
The base case $n=1$ is given in \ref{ex:1D-case}.

Suppose that the  $(n-1)$-dimensional case is proved.
Consider an $n$-dimensional Riemannian polyhedron $\spc{P}$ and suppose
\[n\cdot \sqrt[n]{\VolPro\spc{P}(R)}< R\]
for some $R>0$.
Let $\spc{Q}$ be an $R$-separating subpolyhedron in $\spc{P}$ provided by \ref{lem:separating} for a small $\eps>0$.
Applying  \ref{lem:separating} for $r=\tfrac{n-1}n\cdot R$ and $R$, we have that 
\begin{align*}
\VolPro_\spc{Q}(r) &< \frac 1{R-r}\cdot \VolPro_\spc{P}(R)+\eps<
\\
&<\frac {n}{R}\cdot\left(\frac{R}{n}\right)^n=
\\
&=\left(\frac{r}{n-1}\right)^{n-1};
\end{align*}
that is, $(n-1)\cdot \sqrt[n-1]{\VolPro\spc{Q}(r)}< r$.
Since $\dim\spc{Q}\le n-1$, by the induction hypothesis, we get that
\[\width\spc{Q}\le r<R.\]
It remains to apply \ref{lem:separating-width}.
\qeds





\section{Width bounds systole}

\begin{thm}{Theorem}\label{thm:sys<width}
Suppose $\spc{M}$ is a aspherical $n$-dimensional Riemnnian manifold.
Then 
\[\sys\spc{M}\le 6 \cdot \width \spc{M}.\]
\end{thm}

\begin{thm}{Lemma}\label{lem:aspherical-homotopy}
Let $M$ be an aspherical space and $L$ be a connected CW-complex.
Denote by $L^k$ the k-skeleton of $L$.
Then any continuous map $f\:L^2\to M$ can be extended to a continuous map $\bar f\:L\to M$

Moreover, if $p\in L$ is a 0-cell and $q\in M$.
Then a continuous maps of pairs $\phi_0,\phi_1\:(L,p)\to(M,q)$ are homotopic if and only if $\phi_0$ and $\phi_1$ induce the same homomorphism on fundamental groups $\pi_1(L,p)\to\pi_1(M,q)$.
\end{thm}

\parit{Proof.}
Since $M$ is aspherical, any continuous map $\partial\mathbb{D}^n\:\to M$ for $n\ge 3$
is hull-homotopic;
that is, it can be extended to a map $\mathbb{D}^n\:\to M$.

It makes possible to extend $f$ to $L^3$, $L^4$, and so on.
Therefore $f$ can be extended to whole $L$.

The only-if part on the second part of lemma is trivilal; let us show the if part.

Sine $L$ is connected, we can assume that $p$ forms the only 0-cell in $L$;
otherwise we can collapse a maximal sub-tree of the 1-skeleton in $L$ to $p$.
Therefore $L^1$ is formed by loops that generates $\pi_1(L,p)$.

By assumption, the restrictions of $\phi_0$ and $\phi_1$ to $L^1$ are homotopic.
In other words the homotopy $\Phi\:[0,1]\times L$ is defined on the 2-skeleton of $[0,1]\times L$.
It remains to apply the first part of the lemma.
\qeds



\begin{thm}{Lemma}\label{lem:sys-homotopy}
Suppose $\gamma_0,\gamma_1$ are two paths between points in a Riemannian space $\spc{M}$ such that $\dist{\gamma_0(t)}{\gamma_1(t)}{\spc{M}}<r$ for any $t\in[0,1]$.
Let $\alpha$ be a geodesic path from $\gamma_0(0)$ to $\gamma_1(0)$ and $\beta$ be a geodesic path from $\gamma_0(1)$ to $\gamma_1(1)$. 
If $2\cdot r<\sys\spc{M}$, then there is a homotopy $\gamma_t$ from
$\gamma_0$ to $\gamma_1$ such that $\alpha(t)= \gamma_t(0)$ and $\beta(t)\mapsto \gamma_t(1)$.
\end{thm}

\parit{Proof.}
Set $s=\sys\spc{M}$; 
since $2\cdot r<s$, we have that $\eps=\tfrac1{10}(s-2\cdot r)>0$.

Note that we can assume that $\gamma_0$ and $\gamma_1$ are rectifiable;
if not we can homotopy each into a broken geodesic line kipping the assumptions true. 

\begin{wrapfigure}{r}{34mm}
\vskip-0mm
\centering
\includegraphics{mppics/pic-1405}
\end{wrapfigure}

Choose a fine partition $0\z=t_0\z<t_1\z<\z\dots\z<t_n=1$.
Consider a sequence of geodesic paths $\alpha_i$ from $\gamma_0(t_i)$ to $\gamma_1(t_i)$;
we can assume that $\alpha_0=\alpha$ and $\alpha_n=\beta$.
We can assume that each arc $\gamma_j|_{[t_{i-1},t_i]}$ has length smaller than $\eps$.
Therefore every quadrilateral formed by concatenation  of $\alpha_{i-1}$, $\gamma_1|_{[t_{i-1},t_i]}$, reversed $\alpha_i$, and reversed arc $\gamma_0|_{[t_{i-1},t_i]}$ has length smaller than $s$.
It follows that this curve is contractible.
Applying this observation for each quadrilateral, we get the statement.
\qeds


\parit{Proof of \ref{thm:sys<width}.}
Let $\spc{N}$ be the nerve of a covering $\{V_i\}$ of $\spc{M}$ and $\psi\:\spc{M}\to\spc{N}$ be the map provided by \ref{prop:space->nerve}.
As usual, we denote by $v_i$ the vertex of $\spc{N}$ that corresponds to $V_i$.

Set $R=\width \spc{M}$ and $s=\sys\spc{M}$.
Assume we chose $\{V_i\}$ as in the definition of width (\ref{def:width}).
For each $i$ choose a point $p_i\in \spc{M}$ such that $V_i\subset \oBall(p_i,R)$.
Observe that in this case $\dim\spc{N}<n$;
therefore $\psi$ kills the fundamental class of $\spc{M}$.

Let us construct a continuous map  $f\:\spc{N}\to  \spc{M}$ such that
$f\circ\psi$ is homotopic to the identity map on $\spc{M}$.

Note that once $f$ is constructed, the theorem is proved, .
Indeed, since $\psi$ kills the fundamental class of $\spc{M}$, so does $f\circ\psi$.
Therefore $[\spc{M}]=0$ --- a contradiction.

First set $f(v_i)=p$.
It defines the map $f$ on the 0-skeleton $\spc{N}^0$ of the nerve $\spc{N}$.
Further we will be define $f$ step by step on the skeletons of higher dimensions $\spc{N}^1,\spc{N}^2, \dots$

Let us map each edge $[v_iv_j]$ in $\spc{N}$ to a geodesic $[p_ip_j]$.
It defines the map on the 1-skeleton $\spc{N}^1$ of the nerve $\spc{N}$.
Note that image of each edge is shorter that $2\cdot R$.

Suppose $[v_iv_jv_k]$ is a triangle in $\spc{N}$.
Note that perimeter of the triangle $[p_ip_jp_k]$ can not exceed $6\cdot R$.
Since $6\cdot R<s$, the contour of $[p_ip_jp_k]$ is contractible.
Therefore we can extend $f$ to each triangle of~$\spc{N}$.
It defines the map $f$ on $\spc{N}^2$.

Finally, since $\spc{M}$ is aspherical, by \ref{lem:aspherical-homotopy}, the map $f$ can be extended to $\spc{N}^3$, $\spc{N}^4$ and so on.

It remains to show that $f\circ\psi$ is homotopic to the identity map.
Choose a CW structure on $\spc{M}$ with sufficiently small cell, so that each cell lies in one of $V_i$.
Note that $\psi$ is homotopic to a map $\psi_1$ that sends $\spc{M}^k$ to $\spc{N}^k$ for any $k$.
Moreover we may assume that (1) if a 0-cell $x$ of $\spc{M}$ maps to a $v_i$, then $x\in V_i$ and (2) each 1-cell  of $\spc{M}$ maps to an edge of $\spc{N}$.
Choose a 1-cell $e$ in $\spc{M}$; by the construction, $f\circ\psi_1$ maps $e$ to a geodesic $[p_ip_j]$ and $e$ lies $\oBall(p_i,R)$.
Observe that $[p_ip_j]$ is shorter than $2\cdot R$.
It follows that the distance between points on $[p_ip_j]$ and $e$ can not exceed $3\cdot R$.
Choose a geodesic path $\alpha_i$ from every 0 cell $x_i$  of $\spc{M}$ to $p_j=f\circ\psi_1(x_i)$.
It defines a homotopy on $\spc{M}^0$.
Since $6\cdot R<s$, \ref{lem:sys-homotopy} implies that this homotopy can be extended to $\spc{M}^1$.
By \ref{lem:aspherical-homotopy}, it can be extended to whole $\spc{M}$.
\qeds

\begin{thm}{Exercise}\label{ex:sys<width}
Analyze the proof of \ref{thm:sys<width} and improve its inequality to 
 \[\sys\spc{M}\le 4 \cdot \width \spc{M}.\]
\end{thm}

\begin{thm}{Exercise}\label{ex:fillrad-inj}
Modify the proof of \ref{thm:sys<width} to prove the following:

Suppose that $\spc{M}$ is a closed $n$-dimensional Reimannian manifold with \emph{injectivity radius} at least $r$; that is, if $\dist{p}{q}{\spc{M}}<r$, then there is geodesic $[pq]_{\spc{M}}$ is uniquely defined.
Show that
\[\width\spc{M}\ge \tfrac{r}{2\cdot(n+1)}.\]

Use \ref{thm:width<vol} to conclude that  
\[\vol\spc{M}\ge \eps_n \cdot r^n \]
for some $\eps_n>0$ that depends only on $n$.
\end{thm} 

\section{Essential manifolds}

To generalize \ref{thm:sys<width} bit further, we need the following definition.

\begin{thm}{Definition}
A closed manifold $\spc{M}$ is called \index{essential manifold}\emph{essential} if it admits a continuous map $\iota\:\spc{M}\to \spc{K}$ to an aspherical topological space $\spc{K}$ such that $\iota$ sends the fundamental class of $\spc{M}$ to a nonzero homology class in $\spc{K}$.\footnote{We assume that the coefficients are $\ZZ_2$, but one can play with them if necessary.}
\end{thm}

Assume that the manifold $\spc{M}$ is essential and $\iota \:\spc{M}\to \spc{K}$ as in the definition.
Following the proof of \ref{thm:sys<width}, we can homotope the map 
$f\circ\psi\:\spc{M}\to \spc{M}$ to the identity on the 2-skeleton of $\spc{M}$;
further since $\spc{K}$ is aspherical we can homotopy the composition
$\iota\z\circ f\circ\psi$ to  $\iota$. 
Existence of this extension implies that that $\iota$ kills the fundamental class of $\spc{M}$ --- a contradiction.
So, taking \ref{ex:sys<width} into account, we proved the following yet more general theorem.

\begin{thm}{Theorem}\label{thm:sys<width++}
Suppose $\spc{M}$ is an essential Riemnnian space.
Then 
\[\sys\spc{M}\le 4 \cdot \width \spc{M}.\]
\end{thm}

As a corollary form \ref{thm:sys<width++} and \ref{thm:width<vol} we get the so called \index{Gromov's systolic inequality}\emph{Gromov's systolic inequality}:

\begin{thm}{Theorem}\label{thm:sys+}
Suppose $\spc{M}$ is an essential $n$-dimensional Riemannian space.
Then 
\[\sys\spc{M}\le 4 \cdot n\cdot \sqrt[n]{\vol\spc{M}}.\]
\end{thm}


Note that any closed aspherical manifold is essential --- in this case one can take $\iota$ to be the identity map on $\spc{M}$.
The real projective space $\RP^n$ provides an interesting example of an essential manifold which is not aspherical.
Indeed, the infinite dimensional projective space $\RP^\infty$ is aspherical and for the natural embedding $\RP^n\hookrightarrow\RP^\infty$ the image $\RP^n$ does not bound in $\RP^\infty$.
The following exercise provides more examples of that type.

\begin{thm}{Exercise}\label{ex:connected-sum-essential}
Show that connected sum of an essential manifold with any closed manifold is essential.
\end{thm}

\begin{thm}{Exercise}\label{ex:product-essential}
Show that product of two essential manifolds is essential.

Show that product of nonessential closed manifold of dimension at least 1 with any closed manifold is not essential.
\end{thm}

\section{Remarks}

Theorem \ref{thm:sys+} was proved originally by Mikhael Gromov \cite{gromov-1983} with much worse constant.
The given proof is a result of a sequence of simplifications given by Larry Guth \cite{guth},  Panos Papasoglu \cite{papasoglu}, Alexander Nabutovsky and Roman Karasev \cite{nabutovsky}.

In \cite{nabutovsky} the calculations were optimized better which gave the constants 
$c_n=\sqrt[n]{n!}= \tfrac ne+o(n)$ in \ref{thm:width<vol} instead of $n$.
As a result, we have a stronger statement in \ref{thm:sys+}:
\[\sys\spc{M}\le 4 \cdot c_n\cdot \sqrt[n]{\vol\spc{M}}.\]

A wide open conjecture says that the optimal constant is $\pi/\sqrt[n]{\omega_n/2}$ where $\omega_n$ denotes the volume of $n$-dimensional unit sphere.
This is the systole ratio for the $n$-dimensional real projective space with canonical metric; it  grows as $O(\sqrt n)$.



%\chapter{Volume bounds filling radius}

This chapter 
is devoted to a proof of \ref{thm:FillRad<vol};
that is, we will show that \emph{Riemannian manifolds with small volume have small filling radius}.
Note that once it is proved, \ref{thm:sys<FillRad} implies \ref{thm:sys(torus)}.
Moreover \ref{thm:sys<FillRad++} implies the following:

\begin{thm}{Therorem}\label{thm:sys(torus)}
Systolic intequality holds for any essential manifold. 
\end{thm}

This theorem was proved originally by Mikhael Gromov \cite{gromov-1983}.
We follow closely a simplified proof given by Alexander Nabutovsky, which is based on a sequence of other simplifications and improvements; see \cite{nabutovsky} and the references therein.

\section{Nerves and partition of unity}

Let $\{V_1,\dots,V_k\}$ be a finite open cover of a compact metric space $\spc{X}$.
Consider an abstract simplicial complex $\spc{N}$, with one vertex $v_i$ for each set $V_i$ such that a simplex with vertexes $v_{i_1},\dots, v_{i_m}$ is included in $\spc{N}$ if 
the intersection $V_{i_1}\cap\dots\cap V_{i_m}$ is nonempty.
We obtain a simplicial complex $\spc{N}$ called the \index{nerve}\emph{nerve of the covering $\{V_i\}$}.

Note that $\spc{N}$ is a finite simplicial complex and it has dimension at most $n$ if and only if the covering $\{V_1,\dots,V_k\}$ has multiplicity is at most $n+1$;
that is, at most $n+1$ different sets $V_{i_1},\dots, V_{i_{n+1}}$ have a nonempty intersection.
The nerve $\spc{N}$ is a subcomplex of a simplex with the vertixes $\{v_1,\dots,v_k\}$.

\begin{thm}{Proposition}\label{thm:part-unit}
 Let $\{V_1,\dots,V_k\}$ is a finite open covering of a compact metric space ${\spc{X}}$.
Then there are Lipschitz functions $\psi_i\:{\spc{X}}\to[0,1]$ such that
if $\psi_i(x)>0$ then $x\in V_i$ and
$$\sum_i\psi_i(x)=1$$
for any $x\in {\spc{X}}$.
\end{thm}

A collection of functions $\psi_i$ with above properies is called 
a \emph{partition of unity subordinate to the open cover}\index{partition of unity} $\{V_1,\dots,V_k\}$.

\parit{Proof.}
Consider the functions $\phi_i\:{\spc{X}}\to\RR$ defined as
$$\phi_i(x)=\distfun_{({\spc{X}}\backslash V_i)} x.$$
Note $\phi_i$ is $1$-Lipschitz
for any $i$
and $\phi_i(x)>0$ if and only if $x\in V_i$.
In particular, 
$$\sum_i\phi_i(x)>0\ \ \text{for any}\ \ x\in {\spc{X}}.$$

Set 
$$\psi_k(x)=\frac{\phi_k(x)}{\sum_i\phi_i(x)}.$$
It remains to note that by construction the functions $\psi_i$ meet the conditions in the proposition.
\qedsf


Note that in the above proof for any point $x\in {\spc{X}}$,
the set
$$\set{v_i}{\psi_i(x)>0}$$
describe vertexes of a simplices in the nerve.
Therefore 
$$\psi\:x\mapsto \psi_1(x)\cdot v_1+\psi_2(x)\cdot v_2+\dots+\psi_k(x)\cdot v_n.$$
can be thought of as a Lipschitz map from ${\spc{X}}$ to the nerve $\spc{N}$ of $\{V_i\}$;
where the point $x$ is mapped to the point with barycentric coordinates $\psi_i(x)$.
In other words we proved the following:

\begin{thm}{Proposition}\label{prop:space->nerve}
Let $\spc{N}$ be a nerve of an open covering $\{V_1,\z\dots,V_k\}$ of a compact metric space $\spc{X}$.
Denote by $v_i$ the vertex of $\spc{N}$ that corresponds to $V_i$.

Then there is a Lipschitz map from $\psi\:\spc{X}\to\spc{N}$ such that $\psi(V_i)\z\subset\Star_{v_i}$ for every $i$.
\end{thm}


\section{Width}

Suppose $A$ is a subset of a metric space $\spc{X}$.
The radius of $A$ (briefly $\rad A$) is defined as the least upper bound on the values $R>0$ such that $\oBall(x,R)\supset A$ for some $x\in \spc{X}$.

\begin{thm}{Definition}\label{def:width}
Let $\spc{X}$ be a metric space.
The $n$-th width of $\spc{X}$ (briefly $\width_n\spc{X}$) is defined as least upper bound on values $R>0$ such that $\spc{X}$ admits a finite open covering $\{V_i\}$ with multiplicity at most $n+1$ and $\rad V_i< R$ for each $i$.
\end{thm}

\parit{Remarks.}
\begin{itemize}
\item Observe that if $\spc{X}$ is connected, then 
\[\width_0\spc{X}=\rad\spc{X}.\]
\item 
Usually width is defined using diameter instead of radius, but the result differ at most twice.
Namely if $r$ is an $n$-th radius-width and $d$ --- $n$-th diameter-width of the same dimension, then 
$r\le d\le 2\cdot r$.

\item The definition of width reminds the definition of Lebesgue covering dimension.
In fact one says that a space has \emph{macroscopic dimesion} $\le n$ on the space $R$ if it admits an open cover as in the definiton.
\end{itemize}

\begin{thm}{Exercise}\label{ex:macrodimension}
Suppose $\spc{X}$ be a metric space such that any closed curve $\gamma$ in $\spc{X}$ can be contracted in its $R$-neighborhood.
Show that $\spc{X}$ is has macroscopic dimension at most 1 on scale $100\cdot R$.

What about quasiconverse? That is, suppose a simply connected metric space $\spc{X}$ has macroscopic dimension at most 1 on scale $R$, is it true that any closed curve $\gamma$ in $\spc{X}$ can be contracted in its $100\cdot R$-neighborhood?
\end{thm}


The following proposition provides an equivalent definition;
we will not use it, but it provides a good reason for the name width.

\begin{thm}{Proposition}\label{prop:width=suprad(inv)}
Suppose $\spc{X}$ is a compact metric space.
Then $\width_n\spc{X}<R$ if and only if there is a finite $n$-dimensional somplicial complex $\spc{S}$ and a continuous map $\psi\:\spc{X}\to \spc{N}$
such that $\rad[\psi^{-1}(s)]\z<R$
for any $s\in \spc{N}$.
\end{thm}

\parit{Proof; ``only if'' part.}
Suppose $\width_n\spc{X}<R$.
Consider a covering $\{V_1,\dots,V_k\}$ of $\spc{X}$ guaranteed by the definition of width.
Let $\spc{N}$ be its nerve and $\psi\:\spc{X}\to \spc{N}$ be the map provided by \ref{prop:space->nerve}.

Note that if $x\in \spc{N}$ lies in a symplex with a vertex $v_i$,
then $\psi^{-1}(x)\subset V_i$;
in particulr $\psi^{-1}(x)$ can be covered by a ball of radius $R$ in $\spc{X}$.

\parit{``If'' part.}
Choose $x\in \spc{N}$.
Since the inverse image $\psi^{-1}(x)$ is compact, $\psi$ is continuous, and $\rad[\psi^{-1}(x)]<R$,
here is a neighborhood $U\ni x$ such that the  $\rad[\psi^{-1}(U)]<R$.

It follows that there is a finite cover $\{U_i\}$ of $\spc{N}$ such that $\psi^{-1}(U_i)\subset\spc{X}$ has radius smaller than $R$ for each $i$.
Since $\spc{N}$ has dimension $n$, we can inscribe%
\footnote{Recall that a covering $\{W_i\}$ is inscribed in the covering $\{U_i\}$ if for every $W_i$ is a subset of some $U_j$.} 
in $\{U_i\}$ an finite open cover $\{W_i\}$ with multiplicity at most $n+1$.
It remains to observe that $V_i=\psi(W_i)$ defines a finite open cover of $\spc{X}$ with radius less than $R$ and multiplicity at most $n+1$. 
\qeds

Further we will apply the notion of width to compact Riemannian polyhedrons;
If $n$ is the dimension of a compact Riemannian polyhedron $\spc{P}$, then 
we suppose that
\[\width\spc{P}\df\width_{n-1}\spc{P}.\]

\begin{thm}{Exercise}\label{ex:FillRad<width}
Show that for any closed Riemannian manifold $\spc{M}$ we have
\[\FillRad \spc{M}\le 100\cdot \width\spc{M};\]
try to show that in fact
\[\FillRad \spc{M}\le \width\spc{M}.\]

\end{thm}




\section{Volume profile}

A \emph{Riemannian polyhedron} is defined as a finite connected simplicial complex with a metric tensor on each simplex such that the restriction of the metric on each simplex to a subsymplex coinsides with the metric on the subsmplex.
The dimension of Riemannian polyhedron is defined as the largest dimension it its triangulation.
For Riemannian polhedron one can define length of curves and volume the same way as for Riemannian manifolds.

Let $\spc{P}$ be a Riemnnian polyhedron of dimension $n$.
Let us define volume profile of $\spc{P}$ as a function $\VolPro_{\spc{P}}\:\RR_+\to\RR_+$ defined by 
\[\VolPro_{\spc{P}}(r)\df \sup\set{\vol \oBall(p,r)}{p\in\spc{P}}.\]
Note that $\VolPro_{\spc{P}}$ is a nondecreasing function and $\VolPro_{\spc{P}}(r)\z\to\vol\spc{P}$ as $r\to\infty$.

\begin{thm}{Theorem}\label{thm:width<volpro}
There is a constant $c_n>0$ such that the following holds true:

If $\spc{P}$ is an $n$-dimensional Reimannian polyhedron such that 
\[r> c_n\cdot \sqrt[n]{\VolPro_{\spc{P}}(r)}\] 
for some $r>0$, then 
\[\width\spc{P}\le  r.\]
\end{thm}

Since $\VolPro_{\spc{P}}(r)\le \vol\spc{P}$ for any $r$,
Theorem \ref{thm:width<volpro} implies the following:

\begin{thm}{Theorem}\label{thm:width<vol}
There is a constant $c_n>0$ such that 
\[\width\spc{P}\le c_n\cdot \sqrt[n]{\vol\spc{P}}\] 
for any  $n$-dimensional Reimannian polyhedron $\spc{P}$.
\end{thm}

Together with \ref{ex:FillRad<width}, the last theorem implies \ref{thm:FillRad<vol} which is the goal of this lecture.

\section{Proof}

In the proof of \ref{thm:width<volpro}, we will use the following three technical statements,
the proofs are omitted, but they are not hard. 

\begin{thm}{Smoothing procedure}
Let $\spc{P}$ be a Reimannian polyhedron and $f\:\spc{P}\to \RR$ be a 1-Lipschitz function.
Then for any $\delta>0$ there is a  1-Lipschitz function $\tilde f\:\spc{P}\to \RR$ that is smooth on each simplex of the triangulation and $\delta$-close to $f$.
\end{thm}

\begin{thm}{Sard's theorem}
Let $\spc{P}$ be an $n$-dimensional Reimannian polyhedron and $f\:\spc{P}\to \RR$ be a function that is smooth on each simplex.
Then for almost all values $a$ each component of the inverse image $f^{-1}(a)$ is a equipped with the induced metric is a Reimannian polyhedron.
\end{thm}


\begin{thm}{Coarea inequality}
Let $\spc{P}$ be an $n$-dimensional Reimannian polyhedron and $f\:\spc{P}\to \RR$ be a 1-Lipschitz function that is smooth on each simplex.
Then 
\[\vol_n (f^{-1}[a,b]) \le \int_a^b\vol_{n-1}(f^{-1}\{x\})\cdot dx.\]
\end{thm}

Theorem \ref{thm:width<volpro} will be proved by induction on the dimension of $\spc{P}$;
the following exercise provides a base for the induction.
Note that $\spc{P}$ is connected by definition and if it is 1-dimensional, then 
\[\width\spc{P}=\width_0\spc{P}=\rad\spc{P}.\]

\begin{thm}{Exercise}\label{ex:1D-case}
Suppose $\spc{P}$ be a 1-dimensional Riemannian polhedron.
Suppose $\VolPro_{\spc{P}}(r)<r$ for some $r>0$.
Show that 
\[\width \spc{P}<r.\]

\end{thm}


An $(n-1)$-dimensional subpolyhedron $\spc{Q}\subset\spc{P}$ is called $R$-separating if each
connected component of the complement $\spc{P}\backslash \spc{Q}$ has radius smaller than $R$.

\begin{thm}{Lemma}\label{lem:separating}
Let $\spc{P}$ be an $n$-dimensional Riemannian polyhedron.
Then given $R>0$ and $\eps>0$ there is a $R$-separating subpolyhedron $\spc{Q}\subset\spc{P}$ such that for any $r_0<r_1\le R$ we have
\[\VolPro_{\spc{Q}}(r_0)< \tfrac1{r_1-r_0}\cdot \VolPro_{\spc{P}}(r_1)+\eps.\]

\end{thm}

\parit{Proof.}
Choose small $\delta>0$.
Applying the smoothing procedure, we can exchange each distance function $\distfun_p$ on $\spc{P}$ by $\delta$-close smooth 1-Lipschitz function, which will be denoted by $\widetilde \distfun_p$.

By Sard's theorem, almost all level sets $\tilde S_c(p)$ defined by $\widetilde \distfun_p=c$ are smooth Riemannian polyhedrons of dimension $n-1$.

Since $\delta$ is small, the coarea inequality implies that 
for some  $c\z\in(r_0+\delta, r_1-\delta)$ we have
\begin{align*}
\vol_{n-1}\tilde S_c(p)&\le \tfrac1{r_1-r_0-2\cdot\delta}\cdot\vol_n[\oBall(p,r_1)]<
\\
&<\tfrac1{r_1-r_0}\cdot \VolPro_{\spc{P}}(r_1)+\tfrac\eps2.
\end{align*}

Now suppose $\spc{Q}$ is an $R$-separating subpolyhedron in $\spc{P}$ with almost minimal volume, say its volume is at most $\tfrac\eps2$-far from the greatest lower bound.
Note that cutting from $\spc{Q}$ everything inside $\tilde S_c$ and adding $\tilde S_c$ keeps it to be $R$-separating subpolyhedron.
It follows that
\[\vol_{n-1}[\spc{Q}\cap \oBall(p,r_0)_{\spc{P}}]-\tfrac\eps2\le \vol_{n-1}S_c.\]
Therefore 
\[\vol_{n-1}[\spc{Q}\cap \oBall(p,r_0)_{\spc{P}}]\le\tfrac1{r_1-r_0}\cdot \VolPro_{\spc{P}}(r_1)+\eps\eqlbl{eq:volQ<ProP}\]
Recall that $\spc{Q}$ is equipped with the induced length metric;
therefore $\dist{p}{q}{\spc{Q}}\ge \dist{p}{q}{\spc{P}}$ for any $p,q\in \spc{Q}$;
in particular, 
\[\oBall(p,r_0)_{\spc{Q}}\subset \spc{Q}\cap \oBall(p,r_0)_{\spc{P}}.\]
Hence \ref{eq:volQ<ProP} implies the lemma.
\qeds

\begin{thm}{Lemma}\label{lem:separating-width}
Let $\spc{Q}$ be a $R$-separating subpolyhedron in an $n$-dimensional Riemannian polyhedron $\spc{P}$.
Suppose $\width\spc{Q}\le R$.
Then $\width\spc{P}\le R$
\end{thm}

\parit{Proof.}
Start with an open covering $\{V_1,\dots,V_k\}$ of $\spc{Q}$ of multiplicity $\le n$ with radiuses of the sets in the intrinsic metric $\le R$.

Note that $\{V_1,\dots,V_k\}$ can be converted into an an open covering of
a small neighbourhood of $\spc{Q}$ in $\spc{P}$ without increasing the multiplicity.
This is can be done by setting 
\[V_i'=\bigcup_{x\in V_i}\oBall(x,r_x),\]
where $r_x=\tfrac1{10}\cdot\inf\set{\dist{x}{y}{}}{y\in \spc{Q}\backslash V_i}$.

Finally, add all the components of $\spc{P}\backslash \spc{Q}$ to the covering;
it increases the multiplicity by 1.
The statement follows since $\dim \spc{P}= \dim \spc{Q}\z+1$.
\qeds

\parit{Proof of \ref{thm:width<volpro}.}
We apply induction on the dimension $n=\dim\spc{P}$;
the base case $n=1$ is provided by \ref{ex:1D-case},  for $c_1=1$.

Suppose that the constant $c_{n-1}$ is known, choose sufficiently small $c_n$
\[c_n>2\cdot c_{n-1}.\]

Assume $c_n\cdot \sqrt[n]{\VolPro\spc{P}(r)}< r$.
Fix small $\eps>0$.
By taking $r_0=\tfrac r2$ and $r_1=r$ in \ref{lem:separating}, we have an $r$-separating subpolhedron $\spc{Q}$ in $\spc{P}$ such that 
\begin{align*}
\VolPro_\spc{Q}(r_0) &< \tfrac 1 {r_0}\cdot \VolPro_\spc{P}(r)+\eps<
\\
&<\tfrac 1 {r_0}\cdot \left(\frac{2\cdot r_0}{c_n}\right)^n+\eps=
\\
&=\left(\frac2{c_n}\right)^n\cdot r_0^{n-1}+\eps<
\\
&<\left(\frac1{c_{n-1}}\right)^{n-1}\cdot r_0^{n-1};
\end{align*}
that is, $c_{n-1}\cdot \sqrt[n-1]{\VolPro\spc{Q}(r_0)}< r_0$.
By the induction hypothesis 
\[\width\spc{Q}\le r_0<r.\]

Applying \ref{lem:separating-width}, we get $\width\spc{P}<r$
\qeds


%\chapter{Examples}



\section{On semicontinuity}

Recall that according to \ref{ex:GH-vol}, volume is semicontinuos on the space of Riemannian manifolds with respect to stable Gromov--Hausdorff convergence.
Analogous statement for $n$-dimensional Hausdorff measure on a $n$-dimensional manifolds does not hold.

\begin{thm}{Claim}
 
\end{thm}

First let us show that for any $\alpha>0$, the $\alpha$-dimensional Hausdorff measure is not semicontinuous in the space of all compact metric spaces.

Choose a decreasing sequence $\eps_n\to 0$.
Consider the space $\spc{C}$ of infinite binary sequences with distance between two sequences $\bm{a}=(a_0,a_1,\dots)$ and $\bm{b}=(b_0,b_1,\dots)$ defined by 
\[\dist{\bm{a}}{\bm{b}}{\spc{C}}=\eps_n,\]
where $n$ is the minimal index such that $a_n\ne b_n$.
Note that $\spc{C}$ is homeomorphic to the Cantor set and 
given $\alpha>0$,
the sequence $\eps_n$ can be chosen so that its $\alpha$-dimensional Hausdorff measure is infinite.

Note that $\spc{C}$ is a Hausdorff limit of its subsets $\spc{C}_n$ formed by sequences that constantly zero starting from $n$-th element.
The sets $\spc{C}_n$ is finite in particular its $\alpha$-dimensional Hausdorff measure vanish for $\alpha>0$.
This example shows that for any $\alpha>0$, the $\alpha$-dimensional Hausdorff measure is not semicontinuous in the space of all compact metric spaces.

An analogous example can be produced comapct length spaces.
To do this consider a metric binary rooted tree $\spc{T}$ in which edges connecting level $n-1$ to the level $n$ of length $\eps_{n-1}-\eps_n$.
Note that the completion $\bar{\spc{T}}$ of $\spc{T}$ has a subset (its crown) isometric to $\spc{C}$.
Note further that $\bar{\spc{T}}$ is a Hausdorff limit of its subsets $\spc{T}_n$ --- the subtrees up to level $n$.
Note that $\spc{T}_n$ is can be covered by a finite line segments, in particular it has finite $1$-dimensional Hausdorff measure and therefore vanishing $\alpha$-dimensional Hausdorff for any $\alpha>1$.
Since the limit $\bar{\spc{T}}$ contains $\spc{C}$, we can choose a sequence $\eps_n$ so that $\mu_\alpha\spc{C}$ is arbitrary large (or even infinite).
It shows that for any $\alpha>1$, the $\alpha$-dimensional Hausdorff measure is not semicontinuous in the space of all compact length spaces.

This construction can be modified further to obtain an increasing sequence of metric tensors $g_n$ on a disc $\DD$ such that (1) $\vol(\DD,g_n)<1$ for each $n$, (2) the induced metrics $\dist{*}{*}{g_n}$ converge to a metric $\rho$ on $\DD$, and given any Cantor space $\spc{C}$ as described above (3) there is a bilipschitz map $\spc{C}\to(\DD,\rho)$.
Note that the last condition implies that $\mu_2(\DD,rho)$ can be made arbitrary large, or infinite.
Therefore for any $\alpha\ge 2$, the $\alpha$-dimensional Hausdorff measure is not semicontinuous in the space of all compact length spaces homeomorphic to a manifold and equipped with stable convergence.

Now we want to extend nonsemicontinuity even further.
Note that the tree $\bar{\spc{T}}$ admits a length-preserving embedding to the Euclidean space; we may assume that all 



\section{Sub-Riemannian metrics}

Choose a metric space $\spc{X}$.
Note that the function $\alpha\mapsto \mu_\alpha(A)_\spc{X}$ is nondecreasing;
moreover there is a critical value $\alpha_0\in[0,\infty]$ such that $\mu_\alpha(A)_\spc{X}=0$ if $\alpha<\alpha_0$ and $\mu_\alpha(A)_\spc{X}=\infty$ if $\alpha>\alpha_0$.
This value is called \index{Hausdorff dimension}\emph{Hausdorff dimension} of $\spc{X}$, or briefly $\alpha_0=\dim_H\spc{X}$.

The following statement is classical, a proof can be found in .

\begin{thm}{Theorem}
The Hausdorff dimension of any metric space can not be smaller than its Lebesgue covering dimension.
In particular, if a metric space $\spc{X}$ is homeomorphic to an $n$-dimensional manifold, then $\dim_H\spc{X}\ge n$.
 
\end{thm}

Note that the construction described in the previous section can be used to produce a metric on manifold of dimension $n\ge 2$ with arbitrary Hausdorff dimension $\alpha\ge n$.

In this section we will discuss another interesting source of such examples.



%











\begin{thm}{Lemma}
$\spc{M}$ is complete.
\end{thm}

\parit{Proof.}
Let $(\spc{X}_n)$ be a Cauchy sequence in $\spc{M}$.
Passing to a subsequence if necessary, 
we can assume that $|\spc{X}_n-\spc{X}_{n+1}|_{\spc{M}}<\tfrac1{2^n}$ for each $n$.
In particular, for each $n$ one can equip $\spc{W}_n=\spc{X}_n \sqcup \spc{X}_{n+1}$ with a metric such that
inclusions $\spc{X}_n\hookrightarrow \spc{W}_n$ and $\spc{X}_{n+1}\hookrightarrow \spc{W}_n$ are distance preserving
and $$|\spc{X}_n-\spc{X}_{n+1}|_{\mathcal{H}(\spc{W}_n)}\z<\tfrac1{2^n}$$
for each $n$.

Set $\spc{W}$ to be the disjoint union of all $\spc{X}_n$.
Let us equip $\spc{W}$ with a metric defined the following way:
\begin{itemize}
\item for any fixed $n$ and any two points $x_n,x_n'\in \spc{X}_n$ set
$$|x_n-x_n'|_{\spc{W}}=|x_n-x_n'|_{\spc{X}_n}$$
\item for any positive integers $m>n$ and any two points $x_n\in \spc{X}_n$ and $x_m\in \spc{X}_m$ set
$$|x_n-x_m|_{\spc{W}}=\inf\left\{\sum_{i=n}^{m-1}|x_i-x_{i+1}|_{\spc{W}_i}\right\},$$
where the infimum is taken for all sequences $x_i\in \spc{X}_i$.
\end{itemize}

\begin{thm}{Exercise}
Check that this indeed defines a metric on $\spc{W}$.
\end{thm}

Let $\bar{\spc{W}}$ be the completion of $\spc{W}$.
Note that $|\spc{X}_m-\spc{X}_n|<\tfrac1{2^{n-1}}$ if $m>n$.
Therefore the union of $\spc{X}_1\cup \spc{X}_2\cup\dots\cup \spc{X}_n$ forms a $\tfrac1{2^{n-1}}$-net in $\bar{\spc{W}}$.
Since each $\spc{X}_i$ is compact, we get that $\bar{\spc{W}}$ admits a compact $\eps$-net for any $\eps>0$.
According to Problem~\ref{pr:compact-net}, $\bar{\spc{W}}$ is compact.

According to Blaschke's compactness theorem (\ref{thm:compact+Hausdorff}),
we can pass to a subsequence of $(\spc{X}_n)$ which converge in $\mathcal{H}(\bar{\spc{W}})$ and therefore in $\spc{M}$.
\qeds

\parit{Proof of \ref{thm:gromov-compactness}; ``only if'' part.}
If there is no sequence $\eps_n\to0$ as described in the problem, then for a fixed fixed $\delta>0$
there is a sequence of spaces $\spc{X}_n\in\spc{Q}$ such that $$\pack_\delta \spc{X}_n\to\infty
\quad\text{as}\quad
n\to\infty.$$
Since $\spc{Q}$ is compact, 
this sequence has a partial limit say $\spc{X}_\infty\in\spc{Q}$.
It is easy to see that $\pack_{\delta/10} \spc{X}_\infty=\infty$;
the later contradicts Theorem~\ref{thm:finite_pack=compact}.

\parit{``If'' part.}
Let us fix the sequence $\eps_n\to 0$ as in the problem and consider the set $\hat{\spc{Q}}$ of all (isometry classes of all) metric spaces $\spc{X}$ such that
$\pack_{\eps_n} \spc{X}\le n$ for any $n$. 
According to Exercise~\ref{ex:pack-GH}, $\hat{\spc{Q}}$ is closed in $\spc{M}$.
Clearly $\spc{Q}\subset\hat{\spc{Q}}$.
Therefore it is sufficient to prove that $\hat{\spc{Q}}$ is compact.

Note that $\diam \spc{X}\le \eps_1$ for any $\spc{X}\in \hat{\spc{Q}}$.
Given positive integer $n$ consider set of all metric spaces $\spc{W}_n$
with number of points at most $n$ and diameter $\le \eps_1$.
Note that $\spc{W}_n$ is compact for each $n$.
Further a maximal $\eps_n$-packing of any $\spc{X}\in\hat{\spc{Q}}$ forms a subspace from $\spc{W}_n$.
Therefore $\spc{W}_n\cap\hat{\spc{Q}}$ is a comapct $\eps_n$-net in  $\hat{\spc{Q}}$.
Problem~\ref{pr:compact-net} implies that $\hat{\spc{Q}}$ is compact.
\qeds



\section{Comments} 

Given two metric spaces $\spc{X}$ and $\spc{Y}$, we will write $\spc{X}\preccurlyeq \spc{Y}$ if there is a noncontracting map $f\:\spc{X}\to \spc{Y}$;
that is, if 
$$ |x-x'|_{\spc{X}}\le|f(x)-f(x')|_{\spc{Y}}$$
for any $x,x'\in \spc{X}$.

Further, given $\eps>0$, we will write $\spc{X}\preccurlyeq \spc{Y}+\eps$
if there is a map $f\:\spc{X}\to \spc{Y}$ such that 
$$|x-x'|_{\spc{X}}\le|f(x)-f(x')|_{\spc{Y}}+\eps$$
for any $x,x'\in \spc{X}$.

Define 
$$\dist[\star]{\spc{X}}{\spc{Y}}{\spc{M}}=\inf\set{\eps}{\spc{X}\preccurlyeq \spc{Y}+\eps
\quad\text{and}\quad
\spc{Y}\preccurlyeq \spc{X}+\eps}$$
It turns out that $\dist[\star]{*}{*}{\spc{M}}$ is a different metric on the set of isometry classes of compact metric spaces; that is, in general $\dist[\star]{\spc{X}}{\spc{Y}}{\spc{M}}\not=|\spc{X}-\spc{Y}|_{\spc{M}}$. 
However, these two metrics define the same topology on $\spc{M}$.
More precicely:

\begin{thm}{Proposition}\label{GH-po}
For any sequence of compact metric spaces $(\spc{X}_n)$ and a compact metric space $\spc{X}_\infty$,
we have
$$|\spc{X}_n-\spc{X}_\infty|_{\spc{M}}\to 0
\quad\iff\quad
\dist[\star]{\spc{X}_n}{\spc{X}_\infty}{\spc{M}}\to 0$$ 
as $n\to\infty$.
\end{thm}

We will not give a proof of this proposition. 
Likely, we will not use it further in the lectures, 
but it might help you to build intuition for Gromov--Hausdorff convergence.
If you want to prove it yourself look in the proof of Theorem~\ref{thm:GH-is-a-metric} 
and try to modify it using ideas from the proof of Problem~\ref{pr:non-contracting=>isometry}.

The Gromov--Hausdorff distance can be defined for arbitrary pair of metric space.
Therefore it is natural to ask why we only consider compact metric spaces.
First note the Gromov--Hausdorff distance from any metric space $\spc{X}$ 
to its completion $\bar {\spc{X}}$ is zero.
Therefore if you want to end up in a metric space, it is better to consider only complete metric spaces.

Further, the distance between one-point-space and a metric spce with infinite diameter is infinite.
Therefore one has to either consider only bounded metric spaces (that is, the spaces with finite diameter)
or relux the definition of metric space which allow metric to take infinite value.

Finally, the class of isometry classes of all bounded complete metric spaces forms a class, but not a set.
Hence again we have two choices: either relux the definition of metric space so its points will form a class, or restrict further the class of spaces for which the isometry classes will form a set.

\begin{thm}{Exercise}
Prove that isometry classes of compact metric spaces form a set. 
\end{thm}

\begin{thm}{Exercise}\label{pr:GH1}
Let $\spc{X}=\{x,y,z\}$ be a three point subset of Euclidean plane with distances
$$|x-y|=|y-z|=|z-x|=1.$$
\begin{enumerate}[(i)]
\item Find the minimal Hausdorff distance from $\spc{X}$ to a one-point subset of the plane.
\item Find the Gromov--Hausdorff distance from $\spc{X}$ to the one-point metric space. 
\end{enumerate}
\end{thm}

\begin{thm}{Exercise}\label{pr:GH2}
Let $\spc{X}$ and $\spc{Y}$ be a compact metric spaces which have isometric $\eps$-nets.
Show that 
$$|\spc{X}-\spc{Y}|_{\spc{M}}\le 2\cdot\eps.$$
Is it always true that 
$$|\spc{X}-\spc{Y}|_{\spc{M}}\le \eps?$$
\end{thm}




\begin{thm}{Exercise}\label{pr:GH3}
Define the \emph{radius of a metric space}\index{radius of a metric space} $\spc{X}$ as 
$$\rad \spc{X}=\inf_x\left\{\sup_y\{|x-y|_{\spc{X}}\}\right\}.$$
Equivalently, 
$$\rad \spc{X}=\inf\set{R>0}{\text{there is}\ x\in \spc{X}\  \text{such that}\ B_R(x)\supset \spc{X}}.$$
 
\begin{enumerate}[(i)]
\item Show that for any compact metric space $\spc{X}$ we have
$$\tfrac12\cdot\diam \spc{X}\le \rad \spc{X}\le \diam \spc{X}.$$
\item Show that for any compact metric spaces $\spc{X},\spc{Y}$ we have
$$|\rad \spc{X}-\rad \spc{Y}|\le 2\cdot |\spc{X}-\spc{Y}|_{\spc{M}}.$$
\end{enumerate}
\end{thm}

\begin{thm}{Exercise}\label{pr:F-X}
Let $\spc{X}$ be a metric space.
If two compact sets $A, B$ in $\spc{X}$ are isometric,
we will write $A\iso B$. 
Set
$$d(A,B)=\inf \set{|A'-B'|_{\mathcal{H}(\spc{X})}}{A'\iso A \ \text{and}\ B'\iso B}.$$
Note that if $\spc{X}=\ell^\infty$ then according to Proposition~\ref{prop:GH-with-fixed-Z}, 
$d$ is a metric on $\mathcal{H}(\spc{X})/\iso$ (that is, on the ``$\iso$''-equivalecne classes of $\mathcal{H}(\spc{X})$).

Show that it does not hold for arbitrary metric space $\spc{X}$.
Understand the reason why it holds for $\spc{X}=\ell^\infty$.
\end{thm}


\begin{thm}{Exercise}\label{pr:GH-variation}
Consider the pairs $(\spc{X},A)$, where $\spc{X}$ is a compact metric space and $A$ is a closed subset in $\spc{X}$.
Two such pairs, say $(\spc{X},A)$ and $(\spc{X}',A')$ will be called equivalent (briefly $(\spc{X},A)\sim(\spc{X}',A')$)
if there is an isometry $\iota\:\spc{X}\to \spc{X}'$ such that $\iota(A)=A'$.

Modify the definition of Gromov--Hausdorff metric to construct a natural metric on the set of $\sim$-equivalence classes of the pairs $(\spc{X},A)$.
\end{thm}

Here we introduce so called Gromov--Hausdorff convergence for metric spaces.
This convergence was introduced by Gromov around 1980, published in \cite{gromov-1981}.
Very soon this notion began to be used in all branches of geometry.
In fact today I have difficulty to understand 
how one could do geometry without this type of convergence.%
(Some types of convergences of metric spaces was considered before Gromov,
but they had lack of generality;
each type of convergence was desined to solve one particular problem.)


\begin{thm}{Exercise}\label{ex:euclid-isom}
\begin{subthm}{}
Let $\spc{X},\spc{Y}$ be two compact sets in the Euclidean plane $\RR^2$.
Show that $\spc{X}$ is isometric to $\spc{Y}$ if and only if there is an motrio $\iota\:\RR^2\to \RR^2$
that sends $\spc{X}$ to $\spc{Y}$.
\end{subthm}

\begin{subthm}{}
Find two isometric subsets $\spc{X},\spc{Y}$ of $\ell^\infty$
such that there is no isometry $\iota\:\ell^\infty\to \ell^\infty$ 
that sends $\spc{X}$ to $\spc{Y}$.
\end{subthm}
\end{thm}

\appendix
\chapter{Semisolutions}
\refstepcounter{chapter}
\setcounter{eqtn}{0}

\parbf{\ref{ex:non-differentiable}.}
Choose a function $r\mapsto \alpha(r)$ such that $\alpha'(r)\cdot r\to 0$ and $\alpha(r)\to\infty$ as $r\to 0$.
Consider the reparametrization of the Euclidean plane given by $\iota\:(r,\theta)\mapsto (r,\theta+\alpha(r))$ in the polar coordinates.
Observe that $\iota$ is not differentiable at the origin, but the metric tensor $g$ induced by $\iota$  is continuous.

\medskip

For more on the subject read the paper of Eugenio Calabi and Philip Hartman \cite{calabi-hartman}. 

\parbf{\ref{ex:volume-preserving+short}};
\ref{SHORT.ex:volume-preserving+short:injective}.
Suppose $p=f(x)=f(y)$ and the points $x,y\in \spc{M}$ are distinct.
Since $f$ is short, we get for any $r>0$ the ball $\oBall(p,r)_{\spc{N}}$ contains the images of $\oBall(x,r)_{\spc{M}}$ and $\oBall(y,r)_{\spc{M}}$.
Since $f$ is volume-preserving, we get
\[
\vol\oBall(x,r)_{\spc{M}}
+
\vol\oBall(y,r)_{\spc{M}}
\le
\vol\oBall(p,r)_{\spc{N}}.
\eqlbl{vol+vol<vol}\]

By \ref{obs:lip-chart}, for any $\eps>0$ and all sufficiently small $r>0$ the volumes of the balls  $\oBall(x,r)_{\spc{M}}$, $\oBall(y,r)_{\spc{M}}$ and $\oBall(p,r)_{\spc{N}}$, lie in the range $\omega_n\cdot e^{\mp2\cdot n\cdot\eps}\cdot r^n$, where $\omega_n$ denotes the volume of the unit ball in the $n$-dimensional Euclidean space.
The latter contradicts \ref{vol+vol<vol} for appropriate choice of $\eps$ and $r$.

\parit{\ref{SHORT.ex:volume-preserving+short:bi}.}
Denote by $\sigma(r,a)$ the volume of union of two $r$-balls in the $n$-dimensional Euclidean space such that the distance between their centers is $a$.
Observe that the function $(a,r)\mapsto \sigma(r,a)$ is continuous and increasing in $a$ and $r$ for $a\le r$.
Further, note that
\[\sigma(\lambda\cdot r,\lambda\cdot a)=\lambda^n\cdot \sigma(r,a)\]
for any $\lambda>0$.

Choose a point $z\in \spc{M}$ and small $\eps>0$.
By \ref{obs:lip-chart} there is $R>0$ such that $\oBall(z,10\cdot R)$ admits a $e^{\mp\eps}$-bilipschitz map to the $n$-dimensional Euclidean space.

Choose $x,y\in \oBall(z, R)$.
The argument used in part \ref{SHORT.ex:volume-preserving+short:injective} implies that 
\[e^{-n\cdot\eps}\cdot \sigma(e^{-\eps}\cdot r, e^{-\eps}\cdot \dist{x}{y}{\spc{M}})
\le 
e^{n\cdot\eps}\cdot \sigma(e^{\eps}\cdot r, e^{\eps}\cdot \dist{f(x)}{f(y)}{\spc{N}}).
\eqlbl{eq:v(r,a)}\]
This inequality implies a lower bound on $\dist{f(x)}{f(y)}{\spc{N}}$ in terms of $\dist{x}{y}{\spc{M}}$.

Use the listed properties of the function $(a,r)\mapsto \sigma(r,a)$ to show that for any $c<1$ there is $\eps>0$ such that \ref{eq:v(r,a)} implies that $b>c\cdot a$ for all sufficiently small $a$.

Finally, since $\spc{M}$ and  $\spc{N}$ are length-metric spaces, part~\ref{SHORT.ex:volume-preserving+short:bi} implies that $f$ is locally distance preserving.
(An inclusion map from a nonconvex open subset to the plane gives an example of volume preserving short map that is not distance preserving.)


\medskip

A more general result is discussed by Paul Creutz and Elefterios Soultanis \cite{creutz-soultanis}.


\parbf{\ref{ex:compact-interior}.} Denote by $\spc{M}$ and $\spc{M}^\circ$ the space of $(M,g)$ and $(M^\circ,g)$;
further denote by $\bar{\spc{M}}^\circ$ the completion of $\spc{M}^\circ$.
Observe that the inclusion $M^\circ\hookrightarrow M$ induces a short onto map $\iota\:\bar{\spc{M}}^\circ\z\to\spc{M}$.

Recall that $M$ is bounded by hypersurface that is locally a graph.
Use it to show that any sufficiently short curve $\gamma$ in $(M,g)$ can be approximated by a curve in $\spc{M}^\circ$ with $g$-length arbitrary close to $\length_g\gamma$.
Conclude that $\iota$ is an isometry.


\parbf{\ref{ex:besikovitch=}.}
From the proof of Besicovitch inequality, one can see that the restriction of $\bm{f}$ to the interior of $\spc{M}$ is
(1) volume-preserving, and 
(2) its differential $d_p\bm{f}\:\T_p\to \T_{\bm{f}(p)}$ is an isometry for almost all $p$.

Since $\bm{f}$ is Lipschitz, (2) can be used to show that $\bm{f}$ is short.
It remains to apply \ref{ex:volume-preserving+short} and \ref{ex:compact-interior}.

\parbf{\ref{ex:hexagon}.}
Consider the hexagon with flat metric and curved sides shown on the diagram.
Observe that its area can be made arbitrarily small while keeping the distances from the opposite sides at least 1.

\begin{Figure}
\begin{minipage}{.48\textwidth}
\centering
\includegraphics{mppics/pic-27}
\end{minipage}\hfill
\begin{minipage}{.48\textwidth}
\centering
\includegraphics{mppics/pic-23}
\end{minipage}
\vskip-4mm
\end{Figure}

\parbf{\ref{ex:cylinder};} \ref{SHORT.ex:cylinder:besicovitch}.
Let $\alpha$ be a shortest curve that runs between the boundary components of the cylinder.
Cut the cylinder along $\alpha$.
We get a square with Riemannian metric on it $(\square,g)$.

Two opposite sides of $\square$ correspond to the boundary components of the cylinder.
The other pair corresponds to the sides of the cut.
By assumption, the $g$-distance between the first pair of sides is at least 1.

Consider a shortest curve $\beta$ that connects this pair of sides;
let us keep the same notation for the projection of $\beta$ in the cylinder.

Note that a cyclic concatenation $\gamma$ of $\beta$ with an arc of $\alpha$ is homotopic to a boundary circle.
Therefore $\length_g\gamma\ge1$.
Since $\alpha$ is a shortest path, its arc cannot be longer than any curve connecting its ends; therefore 
\[\length_g\beta\ge \tfrac 12\cdot\length_g \gamma\ge \tfrac 12.\]
That is, the other pair of sides of $\square$ lies on $g$-distance at least $\tfrac12$ from each other.
By \ref{thm:besikovitch+}, $\area(\square,g)\ge \tfrac12$, hence the result.

\parit{\ref{SHORT.ex:cylinder:coarea}.}
Note that any curve in the cylinder that is bordant to a boundary component has length at least $1$.
Therefore if $0\le t\le  1$, then the level sets 
\[L_t=\set{x\in \mathbb{S}^1\times[0,1]}{\distfun_{\mathbb{S}^1\times\{0\}}(x)_{g}=t}\] have length at least $1$.
Applying the coarea inequality, we get that
\[\area(\mathbb{S}^1\times[0,1],g)\ge 1.\]

\parbf{\ref{ex:gadograph}}; \ref{SHORT.ex:gadograph-besikovitch}.
Argue the same way as in \ref{thm:besikovitch}, but observe in addition that $\vol \Sigma=\vol \bm{f}(\Sigma)=0$ and use it time to time.

\parit{\ref{SHORT.ex:gadograph-gadograph}.}
Without loss of generality, we may assume that $V$ lies in a unit cube~$\square$.
Consider a noncontinuous metric tensor $\bar g$ on $\square$ that coincides with $g$ inside $V$ and with the canonical flat metric tensor outside of~$V$.

Observe that the $\bar g$-distances between opposite faces of $\square$ are at least 1.
Indeed this is true for the Euclidean metric and the assumption $\dist{p}{q}{g}\ge\dist{p}{q}{\EE^d}$  guarantees that one cannot make a shortcut in~$V$.
Therefore, the $\bar g$-distances between every pair of opposite faces is at least as large as 1 which is the Euclidean distance.

Applying part \ref{SHORT.ex:gadograph-besikovitch}, we get that $\vol(\square,\bar g)\ge \vol\square$.
Whence the statement follows.


\parbf{\ref{ex:involution-of-sphere}.}
Let $x\in \mathbb{S}^2$ be a point that minimize the distance $|x-x'|_g$.
Consider a shortest path $\gamma$ from $x$ to $x'$.
We can assume that 
\[|x-x'|_g=\length \gamma=1.\]

Let $\gamma'$ be the antipodal arc to $\gamma$.
Note that $\gamma'$ intersects $\gamma$ only at the common endpoints $x$ and $x'$.
Indeed, if $p'=q$ for some $p,q\in\gamma$, then $|p-q|\ge 1$.
Since $\length \gamma=1$, the points $p$ and $q$ must be the ends of $\gamma$.

It follows that $\gamma$ together with $\gamma'$ forms a closed simple curve in $\mathbb{S}^2$;
it divides the sphere into two disks $D$ and $D'$.

Let us divide $\gamma$ into two equal arcs $\gamma_1$ and $\gamma_2$; each of length $\tfrac12$.
Suppose that $p,q\in\gamma_1$, then 
\begin{align*}
|p-q'|_g&\ge |q-q'|_g-|p-q|_g\ge
\\
&\ge 1-\tfrac12=\tfrac12.
\end{align*}
That is, the minimal distance from $\gamma_1$ to $\gamma_1'$ is at least~$\tfrac12$.
The same way we get that the minimal distance from $\gamma_2$ to $\gamma_2'$ is at least~$\tfrac12$.
By Besicovitch inequality, we get that 
\[\area(D,g)\ge \tfrac14\quad\text{and}\quad \area(D',g)\ge \tfrac14.\]
Therefore 
\[\area(\mathbb{S}^2,g)\ge\tfrac12.\]

\parit{A better estimate.}
Let us indicate how to improve the obtained bound to
\[\area(\mathbb{S}^2,g)\ge1.\]

Suppose $x$, $x'$, $\gamma$ and $\gamma'$ are as above.
Consider the function
\[f(z)=\min_t \{\,|\gamma'(t)-z|_g+t\,\}.\]
Observe that $f$ is 1-Lipschitz.

Show that two points $\gamma'(c)$ and $\gamma(1-c)$ lie on one connected component of the level set $L_c=\set{z\in\mathbb{S}^2}{f(z)=c}$;
in particular 
\[\length L_c\ge 2\cdot|\gamma'(c)-\gamma(1-c)|_g.\]
By the triangle inequality, we have that
\begin{align*}
|\gamma'(c)-\gamma(1-c)|_g&\ge 1-|\gamma(c)-\gamma(1-c)|_g=
\\
&=1-|1-2\cdot c|.
\end{align*}

The coarea inequality (\ref{cor:coarea})
\[\area(\mathbb{S}^2,g)\ge \int\limits_0^1\length L_c\cdot dc\]
finishes the proof.


The bound $\tfrac12$ was proved by Marcel Berger \cite{berger}. 
Christopher Croke conjectured that the optimal bound is $\tfrac4\pi$ and the round sphere is the only space that achieves this \cite[Conjecture 0.3 in][]{croke} --- if you solved the last part of the problem, then publish the result.

\begin{wrapfigure}{r}{20 mm}
\vskip-0mm
\centering
\includegraphics{mppics/pic-1305}
\end{wrapfigure}

\parbf{\ref{ex:involution-of-3sphere}.}
Given $\eps>0$, construct a disk $\Delta$ in the plane with 
\begin{align*}
\length\partial \Delta&<10
\intertext{and}
\area \Delta&<\eps
\end{align*}
that admits an continuous involution $\iota$ such that 
\[|\iota(x)-x|\ge 1\]
for any $x\in\partial \Delta$.

An example of $\Delta$ can be guessed from the picture;
the involution $\iota$ makes a length preserving half turn of its boundary $\partial \Delta$.


Take the product $\Delta\times \Delta\subset \EE^4$;
it is homeomorphic to the 4-ball.
Note that 
$$\vol_3[\partial(\Delta\times \Delta)]=2\cdot\area \Delta\cdot\length \partial \Delta<20\cdot\eps.$$
The boundary $\partial(\Delta\times \Delta)$ is homeomorphic to $\mathbb{S}^3$
and the restriction of the involution $(x,y)\z\mapsto (\iota(x),\iota(y))$ has the needed property.

It remains to smooth $\partial(\Delta\times \Delta)$ a  bit.

\parit{Remark.} This example is given by Christopher Croke \cite{croke}.
Note that according to \ref{thm:sys+}, 
the involution $\iota$ cannot be made isometric.

\parbf{\ref{ex:GH-vol}.}
Note that if $(M,g_\infty)$ is $e^{\mp\eps}$-bilipschitz to a cube, then applying Besicovitch inequality, we get that 
\[\liminf_{n\to\infty} \vol (M,g_n)\ge e^{-n\cdot \eps}\cdot\vol (M,g_\infty).\]

By the Vitali covering theorem, given $\eps>0$, we can cover the whole volume of $(M,g_\infty)$ by $e^{\pm\eps}$-bilipschitz cubes.
Applying the above observation and summing up the results, we get that 
\[\liminf_{n\to\infty} \vol (M,g_n)\ge e^{-n\cdot \eps}\cdot\vol (M,g_\infty).\]
The statement follows since $\eps$ is an arbitrary positive number.

To solve the second part of the exercise, start with $g_\infty$ and construct $g_n$ by  adding many tiny bubbles.
The volume can be increased arbitrarily with an arbitrarily small change of metric.

\parit{Remark.}
A more general result was obtained by Sergei Ivanov~\cite{ivanov-1997}.
Note that the statement does not hold true for Gromov--Hausdorff convergence.
In fact any compact metric space $\spc{X}$ can be GH-approximated by a Riemannian surface with an arbitrarily small area.
To show the latter statement, approximate $\spc{X}$ by a finite graph $\Gamma$, embed $\Gamma$ isometrically to the Euclidean space, and pass to the surface of its neighborhood.

\parbf{\ref{ex:sysT2}.}
Set $s=\sys(\TT^2,g)$.

Cut $\TT^2$ along a shortest closed noncontractible curve $\gamma$.
We get a cylinder $(\mathbb{S}^1,g)$ with a Riemannian metric on it.

Applying the argument in \ref{ex:cylinder:besicovitch}, we get that the $g$-distance between the boundary components is at least $\tfrac s2$.
Then \ref{ex:cylinder:besicovitch} implies that the area of torus is at least $\tfrac{s^2}2$.

\parit{Remark.}
The optimal bound is $\tfrac{\sqrt{3}}{2}\cdot s^2$; see  \ref{sec:besicovitch-remarks}.



\parbf{\ref{ex:sysRP2}.}
Set $s\z=\sys (\RP^2,g)$.
Cut $(\RP^2,g)$ along a shortest noncontractible curve $\gamma$.
We obtain $(\DD^2,g)$ --- a disc with metric tensor which we still denote by $g$.

Divide $\gamma$ into two equal arcs $\alpha$ and $\beta$.
Denote by $A$ and $A'$ the two connected components of the inverse image of $\alpha$.
Similarly denote by $B$ and $B'$ the two connected components of the inverse image of $\beta$.

\begin{Figure}
\vskip-0mm
\centering
\includegraphics{mppics/pic-25}
\end{Figure}

Let $\gamma_1$ be a path from $A$ to $A'$;
map it to $\RP^2$ and keep the same notation for it.
Note that $\gamma_1$ together with a subarc of $\alpha$ forms a closed noncontractible curve in $\RP^2$.
Since $\length\alpha=\tfrac s2$, we have that $\length\gamma_1\ge \tfrac s2$.
It follows that the distance between $A$ and $A'$ in $(\DD^2,g)$ is at least $\tfrac s2$.
The same way we show that the distance between $B$ and $B'$ in $(\DD^2,g)$ is at least $\tfrac s2$.

Note that $(\DD^2,g)$ can be parameterized by a square with sides $A$, $B$, $A'$ and $B'$ and apply \ref{thm:besikovitch} to show that 
\[\area(\RP^2,g)=\area(\DD^2,g)\ge \tfrac14\cdot s^2.\]

\parit{Remark.}
The optimal bound is $\tfrac2 \pi\cdot s^2$; see  \ref{sec:besicovitch-remarks}.
In fact any Riemannian metric on the disc with the boundary globally isometric to a unit circle with angle metric has the area at least as large as the unit hemisphere.
It is expected that the same inequality holds for any compact surface with connected boundary (not necessarily a disc);
this is the so-called \index{filling area conjecture}\emph{filling area conjecture} \cite[it is mentioned Mikhael Gromov in 5.5.$\mathrm{B}'(\mathrm{e}')$ of][]{gromov-1983}.

\parbf{\ref{ex:sysSg}.} Cut the surface along a shortest noncontractible curve $\gamma$. 
We might get a surface with one or two components of the boundary.
In these two cases repeat the arguments in \ref{ex:sysRP2} or \ref{ex:sysT2} using \ref{thm:besikovitch+} instead of \ref{thm:besikovitch}.


\parbf{\ref{ex:sysS2xS1}.} Consider the product of a small 2-sphere with the unit circle.

\parbf{\ref{ex:besikovitch++}.}
Apply the same construction as in the original Besicovitch inequality, assuming that the target rectangle
$[0,d_1]\times\dots\times [0,d_n]$ equipped with the metric induced by the $\ell^\infty$ norm;
apply \ref{prop:bilip-measure} where it is appropriate.

\parbf{\ref{ex:2top-discs}.} Suppose that $\Delta_1\ne\Delta_2$.
Consider the map $f\:\mathbb{S}^n\to \spc{X}$ such that the restriction to north and south hemispheres describe $\Delta_1$ and $\Delta_2$ respectively.
Show that if $\Delta_1\ne\Delta_2$, then $\mathbb{S}^n$ can be parameterized by the boundary of the unit cube $\square$ in such a way that for any pair $A$, $A'$ of opposite faces their images $f(A)$, $f(A')$ do not overlap.

Since $\spc{X}$ is contractible, the map $f$ can be extended to a map of the whole cube.
By \ref{ex:besikovitch++} 
\[\haus_{n+1}[f(\square)]>0,\]
a contradiction.

\refstepcounter{chapter}
\setcounter{eqtn}{0}

\parbf{\ref{ex:macrodimension}.}
The following claim resembles Besicovitch inequality;
it is key to the proof:
\begin{itemize}
 \item[$({*})$] Let $a$ be a positive real number.
 Assume that a closed curve $\gamma$ in a metric space $\spc{X}$ can be subdivided into 4 arcs $\alpha$, $\beta$, $\alpha'$, and $\beta'$ in such a way that 
 \begin{itemize}
 \item $|x-x'|>a$ for any $x\in\alpha$ and $x'\in \alpha'$
 and
 \item $|y-y'|>a$ for any $y\in\beta$ and $y'\in \beta'$.
 \end{itemize}
 Then $\gamma$ is not contractible in its $\tfrac a2$-neighborhood.
\end{itemize}

To prove $({*})$, consider two functions defined on $\spc{X}$ as follows:
\begin{align*}
w_1(x)&=\min \{\,a,\distfun_{\alpha}(x)\,\}
\\
w_2(x)&=\min \{\,a,\distfun_{\beta}(x)\,\}
\end{align*}
and the map $\bm{w}\:\spc{X}\to [0,a]\times[0,a]$, defined by
\[\bm{w}\:x\mapsto(w_1(x),w_2(x)).\]

Note that 
\begin{align*}
\bm{w}(\alpha)&=0\times [0,a],
&
\bm{w}(\beta)&=[0,a]\times 0,
\\
\bm{w}(\alpha')&=a\times [0,a],
&
\bm{w}(\beta')&=[0,a]\times a.
\end{align*} 
Therefore, the composition $\bm{w}\circ\gamma$ is a degree 1 map 
\[\mathbb{S}^1\to \partial([0,a]\times[0,a]).\] 
It follows that if $h\:\DD\to \spc{X}$ shrinks $\gamma$, then there is a point $z\in\DD$ such that 
$\bm{w}\circ h(z)=(\tfrac a2,\tfrac a2)$.
Therefore, $h(z)$ lies at distance at least $\tfrac a2$ from $\alpha$, $\beta$, $\alpha'$, $\beta'$
and therefore from $\gamma$.
It proves the claim.

\medskip

Coming back to the problem, let $\{W_i\}$ be an open covering of the real line with multiplicity $2$ and $\rad W_i<R$ for each $i$;
for example take the covering by the intervals $((i-\tfrac23)\cdot R,(i+\tfrac23)\cdot R)$.

Choose a point $p\in \spc{X}$.
Denote by $\{V_j\}$ the connected components of $\distfun_p^{-1}(W_i)$ for all $i$.
Note that $\{V_j\}$ is an open finite cover of $\spc{X}$ with multiplicity at most 2.
It remains to show that $\rad V_j<100\cdot R$ for each $j$.

\begin{Figure}
\vskip-0mm
\centering
\includegraphics{mppics/pic-1310}
\end{Figure}

Arguing by contradiction assume there is a pair of points  $x,y\in V_i$ 
such that $|x\z-y|_{\spc{X}}\z\ge 100\cdot R$.
Connect $x$ to $y$ with a curve $\tau$ in $V_j$.
Consider the closed curve $\sigma$ formed by $\tau$ and two shortest paths $[px]$, $[py]$.


Note that $|p-x|>40$.
Therefore, there is a point $m$ on $[px]$ such that $|m-x|=20$.

By the triangle inequality, the subdivision of $\sigma$ into the arcs $[pm]$, $[mx]$, $\tau$ and $[yp]$ satisfy the conditions of the claim $({*})$ for $a=10\cdot R$,
hence the statement.

\begin{Figure}
\vskip0mm
\centering
\includegraphics{mppics/pic-1315}
\end{Figure}

\parit{The quasiconverse} does not hold.
As an example take a surface that looks like a long cylinder with closed ends;
it is a smooth surface diffeomorphic to a sphere.
Assuming the cylinder is thin, it has macroscopic dimension 1 at a given scale.
However, a circle formed by a section of cylinder around its midpoint by a plane parallel to the base is a circle that cannot be contracted in its small neighborhood.

\parit{Source:} \cite[Appendix $1(\text{E}_{2})$]{gromov-1983}.

\parbf{\ref{ex:width=suprad(inv)}}; \textit{only-if part.}
Suppose $\width_n\spc{X}<R$.
Consider a covering $\{V_1,\dots,V_k\}$ of $\spc{X}$ guaranteed by the definition of width.
Let $\spc{N}$ be its nerve and $\bm{\psi}\:\spc{X}\to \spc{N}$ be the map provided by \ref{prop:space->nerve}.

Since the multiplicity of the covering is at most $n+1$, we have $\dim \spc{N}\le n$.

Note that if $x\in \spc{N}$ lies in a star of a vertex $v_i$,
then $\bm{\psi}^{-1}\{x\}\z\subset V_i$;
in particular, we have $\rad[\bm{\psi}^{-1}\{x\}]<R$.

\parit{If part.}
Choose $x\in \spc{N}$.
Since the inverse image $\bm{\psi}^{-1}\{x\}$ is compact, $\bm{\psi}$ is continuous, and $\rad[\bm{\psi}^{-1}\{x\}]<R$,
there is a neighborhood $U\ni x$ such that the  $\rad[\bm{\psi}^{-1}(U)]<R$.

Since $\spc{X}$ is compact,  there is a finite cover $\{U_i\}$ of $\spc{N}$ such that $\bm{\psi}^{-1}(U_i)\subset\spc{X}$ has a radius smaller than $R$ for each $i$.
Since $\spc{N}$ has dimension $n$, we can inscribe%
\footnote{Recall that a covering $\{W_i\}$ is inscribed in the covering $\{U_i\}$ if for every $W_i$ is a subset of some $U_j$.} 
in $\{U_i\}$ a finite open cover $\{W_i\}$ with multiplicity at most $n+1$.
It remains to observe that $V_i=\bm{\psi}^{-1}(W_i)$ defines a finite open cover of $\spc{X}$ with  multiplicity at most $n+1$ and $\rad V_i<R$ for any $i$. 

\parbf{\ref{ex:1D-case}.}
Assume that $\spc{P}$ is connected.

Let us show that $\diam\spc{P}<R$.
If this is not the case, then there are points $p,q\in\spc{P}$ on distance $R$ from each other.
Let $\gamma$ be a shortest path from $p$ to $q$.
Clearly $\length\gamma\ge R$ and $\gamma$ lies in $\oBall(p,R)$ except for the endpoint $q$.
Therefore, $\length[\oBall(p,R)_{\spc{P}}]\ge R$.
Since $\VolPro_{\spc{P}}(R)\z\ge \length[\oBall(p,R)_{\spc{P}}]$,
the latter contradicts $\VolPro_{\spc{P}}(R)<R$.

In general case, we get that each connected component of $\spc{P}$ has a radius smaller than $R$.
Whence the width of $\spc{P}$ is smaller than $R$.

\parit{Second part.} Again, we can assume that $\spc{P}$ is connected.

The examples of line segment or a circle show that the constant $c=\tfrac12$ cannot be improved.
It remains to show that the inequality holds with $c=\tfrac12$.

Choose $p\in\spc{P}$ such that the value
\[\rho(p)=\max\set{\dist{p}{q}{\spc{P}}}{q\in\spc{P}}\]
is minimal.
Suppose $\rho(p)\ge\tfrac 12\cdot R$.
Observe that there is a point $x\z\in \spc{P}\backslash\{p\}$ that lies on any shortest path starting from $p$ and length $\ge\tfrac 12\cdot R$.
Otherwise for any $r\in(0,\tfrac 12\cdot R)$ there would be at least two points on distance $r$ from $p$;
by coarea inequality we get that the total length of $\spc{P}\cap \oBall(p,\tfrac 12\cdot R)$ is at least $R$ --- a contradiction.

Moving $p$ toward $x$ reduces $\rho(p)$ which contradicts the choice of~$p$.

\parbf{\ref{ex:sys<width}.}
The inequality $6\cdot R<s$ used twice:
\begin{itemize}
\item to shrink the triangle $[p_ip_jp_k]$ to a point;
\item to extend the constructed homotopy on $\spc{M}^0$ to $\spc{M}^1$.
\end{itemize}

The first issue can be resolved by passing to a barycentric subdivision of $\spc{N}^2$;
denote by $v_{ij}$ and $v_{ijk}$ the new vertices in the subdivision that correspond to edge $[v_iv_j]$ and triangle $[v_iv_jv_k]$ respectively.

Further for each vertex $v_{ij}$ choose a point $p_{ij}\in V_i\cap V_j$ and set $f(v_{ij})=p_{ij}$.
Similarly for each vertex $v_{ijk}$ choose a point $p_{ijk}\z\in V_i\cap V_j\cap V_k$ and set $f(v_{ijk})=p_{ijk}$.

Note that 
\begin{align*}
|p_i-p_{ij}|&<R,
\\
|p_i-p_{ijk}|&<R,
\\
|p_{ij}-p_{ijk}|&<2\cdot R.
\end{align*}
Therefore, perimeter of the triangle $[p_ip_{ij}p_{ijk}]$ in the subdivision is less that $4\cdot R$.
It resolves the first issue.

The second issue disappears if one estimates the distances a bit more carefully.
 
\parbf{\ref{ex:fillrad-inj}.}
Choose a fine covering of $\spc{M}$ with multiplicity at most $n$.
Choose $\bm{\psi}$ from $\spc{M}$ to the nerve $\spc{N}$ of the covering the same way as in the proof of \ref{thm:sys<width}.

It remains to construct $f\:\spc{N}\to\spc{M}$ and show that $f\circ\bm{\psi}$ is homotopic to the identity map.
To do this, apply the same strategy as in the proof of \ref{thm:sys<width} together with the so-called \index{geodesic cone construction}\emph{geodesic cone construction}
described below.

Let $\triangle$ be a simplex in a barycentric subdivision of $\spc{N}$.
Suppose that a map $f$ is defined on one facet $\triangle'$ of $\triangle$ to $\spc{M}$ and $\oBall(p,r)\supset f(\triangle')$.
Then one can extend $f$ to whole $\triangle$ such that the remaining vertex $v$ maps to $p$.
Namely connect each point $f(x)$ to $p$ by minimizing geodesic path $\gamma_x$ (by assumption it is uniquely defined) and set
\[f
\:
t\cdot x\z+(1-t)\cdot v
\mapsto
\gamma_x(t).\]

\parbf{\ref{ex:connected-sum-essential}.}
Suppose $M$ is an essential manifold and $N$ is an arbitrary closed manifold.
Observe that shrinking $N$ to a point produces a map $N\#M\to M$ of degree 1.
In particular, there is a map $f\:N\#M\to M$ that sends the fundamental class of $N\#M$ to the fundamental class of $M$.

Since $M$ is essential, there is an aspherical space $K$ and a map $\iota\:M\to K$ that sends the fundamental class of $M$ to a nonzero homology class in $K$.
From above, the composition $\iota\circ f\:N\#M\to K$ sends the fundamental class of $N\#M$ to the same homology class in~$K$.


\parbf{\ref{ex:product-essential}.}
Suppose $M_1$ and $M_2$ are essential.
Let $\iota_1\:M_1\to K_1$ and $\iota_2\:M_2\to K_2$ are the maps to aspherical spaces as in the definition (\ref{def:essential}).
Show that the map
$(\iota_1,\iota_2)\:M_1\times M_2\to K_1\times K_2$
meets the definition.

\parit{Remarks.}
Choose a group $G$.
Note that there is an aspherical connected space CW-complex $K$ with fundamental group $G$.
The space $K$ is called an \index{K(G,1) space@$K(G,1)$ space}\emph{Eilenberg--MacLane space of type $K(G,1)$}, or briefly a $K(G,1)$ space.
Moreover it is not hard to check that
\begin{itemize}
\item $K$ is uniquely defined up to a weak homotopy equivalence;
\item if $\spc{W}$ is a connected finite CW-complex.
Then any homomorphism $\pi_1(\spc{W},w)\to\pi_1(K,k)$ is induced by a continuous map $\phi\:(\spc{W},w)\to(K,k)$.
Moreover, $\phi$ is uniquely defined up to homotopy equivalence.
\end{itemize}

\begin{itemize}
 \item Suppose that $M$ is a closed manifold, 
$K$ is a $K(\pi_1(M),1)$ space and a map $\iota\:M\to K$ induces an isomorphism of fundamental groups.
Then $M$ is essential if and only if $\iota$ sends the fundamental class of $M$ to a nonzero homology class of $K$.
\end{itemize}

The property described in the last statement is the original definition of essential manifold.
It can be used to prove a converse to the exercise;
namely \textit{the product of a nonessential closed manifold with any closed manifold is \emph{not} essential}.





%\chapter{Midterm}\label{chap:midterm}

An oral exam, Th, Feb 27 in class.

\bigskip

\noi 
One theoretical questions from the following list:

\begin{enumerate}
\item 
Semicontinuity of length.
\item
Length spaces and Hopf--Rinow theorem.
\item
Fréchet lemma and Kuratowski embedding.
\item
Hausdorff convergence and Blaschke selection theorem.
\item
Gromov--Hausdorff metric, why it is a metric, almost isometries.
\item
Uniformly totally bonded families and Gromov selection theorem.
\item
Ultralimits and ultrapower of spaces.
\item
Urysohn space.
\item
Injective spaces and injective envelop.
\end{enumerate}

\bigskip

\noi One exercise from the following list:
\\
\ref{ex:almost-min},
\ref{ex:non-contracting-map},
\ref{ex:compact=>complete},
\ref{ex:compact-length},
\\
\ref{ex:Huas-perimeter-area},
\\
\ref{pr:doubling},
\ref{pr:under},
\ref{ex:GH-SC},
\ref{ex:sphere-to-ball},
\\
\ref{ex:ultrapower}, 
\ref{ex:two-geodesics-in-ultrapower},
\ref{ex:lim(tree)},
\\
\ref{ex:geodesics-urysohn},
\ref{ex:sphere-in-urysohn},
\ref{ex:compact-extension},
\\
\ref{ex:+-c},
\ref{ex:ultrametric},
\ref{ex:injective-spaces},
\ref{ex:tripod+square},
\ref{ex:4-on-a-line}.

\bigskip

\noi One more problem for a perfect score.

%%%%%%%%%%%%%%%%%%%%%%%%%%%%
{\small\sloppy
%\RequirePackage{snapshot}
\RequirePackage{snapshot}
\makeatletter
\def\snap@providesfile#1[#2]{%
  \wlog{File: #1 #2}%
  \if\expandafter\snap@graphic@test\expanded{#2}@@\@nil
    \snap@record@graphic#1\relax #2 (type ??)\@nil
  \else
    \expandafter\xdef\csname ver@#1\endcsname{#2}%
  \fi
  \endgroup
}
\makeatother

\documentclass[twoside]{book}

\usepackage{lectures}
\usepackage[colorlinks=true,
citecolor=black,
linkcolor=black,
anchorcolor=black,
filecolor=black,
menucolor=black,
urlcolor=black,
pdftitle={Metric geometry on manifolds: two lectures},
pdfsubject={Geometry},
pdfauthor={Anton Petrunin}
]{hyperref}
\makeindex

\begin{document}
%\pagestyle{empty}
 
\title{Metric geometry on manifolds:
\\ two lectures}
\author{Anton Petrunin}
\date{}
\maketitle

We discuss Besicovitch inequality, width, and systole of manifolds.
I assume that students are familiar with 
measure theory,
smooth manifolds,
degree of map, 
CW-complexes and related notions.

These are two final lectures of a graduate course given at Penn State, Spring 2020.
The complete lectures can be found on the author's website;
it includes an introduction to metric geometry \cite{petrunin2020pure}
and elements of Alexandrov geometry based on \cite{alexander-kapovitch-petrunin-2019}.

\thispagestyle{empty}
\tableofcontents
\thispagestyle{empty}

%%%%%%%%%%%%%%%%%%%%%%%%%%%%
%\addtocounter{chapter}{-1}
\chapter{Homework assignments}


It is better to think about all the problems, but you do not have to solve \emph{all} of them.
If a problem is solved, you do not have to write its solutions, but try sketch it.

\section{Due Tue Jan 21}

Exercises: \sout{\ref{ex:almost-min},} \ref{ex:non-contracting-map}, \ref{ex:no-geod}, \sout{\ref{ex:compact=>complete},} \ref{exercise from BH}, \ref{ex:Hausdorff-bry}.

\section{Due Tue Jan 28}

Exercises: \ref{ex:almost-min},  \ref{ex:compact=>complete}, \ref{ex:Huas-perimeter-area}, \ref{ex:GH-po}, \ref{pr:doubling}, \ref{pr:under:if}.

\section{Due Tue Feb 4}
Exercises: 
\ref{ex:compact-length}, 
\ref{pr:under:only-if}, 
\sout{\ref{ex:GH-SC},}
\sout{\ref{ex:sphere-to-ball},}
\ref{ex:ultrapower}, 
\ref{ex:two-geodesics-in-ultrapower}.

\section{Due Tue Feb 11}

Finish exercises \ref{ex:compact-length} , \ref{pr:under:only-if}, \ref{ex:GH-SC}, \ref{ex:sphere-to-ball}.

\noindent
Exercises: \ref{ex:lim(tree)}, \ref{ex:Asym(Lob)}, \ref{ex:geodesics-urysohn}, \ref{ex:sphere-in-urysohn}.

\section{Due Tue Feb 18}

Exercises: \ref{ex:compact-extension}, \ref{ex:+-c}, \ref{ex:ultrametric}, \ref{ex:injective-spaces}, \ref{ex:tripod+square}, \ref{ex:4-on-a-line}.

\noindent Write down a solution of at least one of the exercises.

\section{Due Tue Feb 25}

Finish Exercise \ref{ex:tripod+square:square}.
Prepare questions for review on Tuesday.

\section{Due Tue Mar 3}

Exercises: \ref{ex:sba-2+2-short}, \ref{ex:(3+1)-expanding}, \ref{ex:CAT+CBB}, \ref{ex:product-CBB}, \sout{\ref{ex:CBB-geodesic},} \ref{ex:fat-triangle}.

\noindent Write down a solution of at least one of the exercises.

\section{Due Tue Mar 17}

Exercises: \ref{ex:tringle-inq-angles},
\ref{ex:CBB-geodesic},
\ref{ex:convex-dist},
\ref{ex:reshetnyak-doubling},
\ref{ex:supporting-planes},
\ref{ex:centrally-simmetric-walls}.

\noindent Write down a solution of at least one of the exercises.

\section{Due Tue Mar 24}

Exercises: 
\ref{ex:contractible},
\ref{ex:convex-nbhd},
\ref{ex:closest-point},
\ref{cor:balls:dim=1},
\ref{ex:null-homotopic},
\ref{ex:branching-cover}.

 Write down as many solutions as you can; email then to Zetian Yan (zxy5156) + cc to me (aqp6).

Each working day I will check email before 15:00 and will appear online if you asked me (it is easy for me --- do not hesitate to ask).
We will meet regular hours online (as we did before).

%%%%%%%%%%%%%%%%%%%%%%%%%%%%

\chapter{Volume bounds} 


\section{Riemannian metrics}

We are going to consider mostly Riemannian spaces;
that is smooth manifolds with metric defined by a metric tensor.
These are specially nice length metrics on manifolds.
However most of the statements we are going to discuss have counterpart for general length metrics on manifolds.

Let $M$ be a smooth manifold.
A \emph{metric tensor} on $M$ is a choice of positive definite quadratic forms $g_p$ on each tangent space $\T_pM$ that depends smoothly on the point $p$.
That is, if we fix a local coordinates on $M$ and write $g$ in this coordinates, then each component of $g$ is a smooth function. 

A Riemannian manifold is a smooth manifold $M$ with a choice \emph{metric tensor} $g$ on it.

The metric tensor $g$ can be used to define length of curves and volume of regions in $M$.

\parbf{Lengths and distances.}
If $\gamma\:[a,b]\to M$ is a piecewise smooth curve then 
\[\length_g\gamma=\int_a^b\sqrt{g(\gamma'(t),\gamma'(t))}\cdot dt.\]
Further we can define a metric on $M$ as least lower bound to lengths of piecewise smooth curves connecting two given points;
the described distance between points $x$ and $y$ will be denoted by $\dist{x}{y}{g}$ or $\distfun_x(y)_g$.
The distance function from a point $x$ will be denoted by $(\distfun_x)_g$ or $\distfun_x$ if the choice of $g$ is evident.

The following claim requires a proof, but we will assume that it is obvious.

\begin{thm}{Claim}
Let $(M,g)$ be a Riemannian manifold.
Then the metric $(x,y)\mapsto \dist{x}{y}{g}$ defines a length metric. Moreover this metric completely determines the metric tensor $g$.
\end{thm}

\parbf{Volume.}
If a region $R$ is covered by one chart $\iota\:U\to M$,
then its volume can be defined as an integral 
\[\vol R
\df
\int_{\iota^{-1}(R)}\sqrt{\det{g}}.\]
In the general case we subdivide $R$ into (a countable collection of) regions $R_1,R_2\dots$ and define
\[\vol R\df \vol R_1+\vol R_2+\dots\]

\section{Besikovitch inequality}

\begin{thm}{Theorem}\label{thm:besikovitch}
Let $g$ be a metric tensor on a unit $n$-dimensional cube $\square^n$.
Suppose that the $g$-distances between the opposite faces of $\square^n$ are at leat $1$; that is, any piecewise smooth curve that connects opposite faces has $g$-length at least $1$.
Then $\vol(\square^n, g)\ge 1$.
\end{thm}

\parit{Proof.}
We will consider the case $n=2$; the other cases are proved the same way.

Denote by $A$, $A'$, and $B$, $B'$ the opposite faces of the square~$\square$.
Consider two function 
\begin{align*}
f_A(x)&\df\min\{\,\distfun_A(x)_g,1\,\},
\\
f_B(x)&\df\min\{\,\distfun_B(x)_g,1\,\}.
\end{align*}
Define $f\:\square\to\square$ as a map with coordinate funcions $f_A$ and $f_B$;
that is, $f(x)\df(f_A(x), f_B(x))$.

Observe that $f$ maps each face to itself.
Indeed, 
\[x\in A \quad\Longrightarrow\quad \distfun_A(x)_g=0 \quad\Longrightarrow\quad f_A(x)=0 \quad\Longrightarrow\quad f(x)\in A.\]
Similarly if $x\in B$, then $f(x)\in B$.
Further, 
\[x\in A'
\quad\Longrightarrow\quad 
\distfun_A(x)_g\ge 1 
\quad\Longrightarrow\quad 
f_A(x)=1 
\quad\Longrightarrow\quad 
f(x)\in A'.\]
Similarly if $x\in B'$, then $f(x)\in B'$.

Therefore 
\[f_t(x)= t\cdot x + (1-t)\cdot f(x)\]
defines a homotopy of maps of pair of spaces $(\square,\partial \square)$ from $f$ to the identity map.
It follows that degree of $f$ is $1$; that is, $f$ sends the fundamental class of $(\square,\partial \square)$ to itself.
In particular $f$ is onto.

Suppose that Jacobian  matrix $\Jac_pf$ of $f$ is defined at $p\in \square$.
Choose an orthonormal basis in $\T_p$ with respect to $g$ and the standard basis in the target $\square$.
Observe that the differentials $d_pf_A$ and $d_pf_B$ written in these basises are the rows of $\Jac_pf$.
Evidently $|d_pf_A|\le 1$ and $|d_pf_B|\le 1$.
Since the determinant of a matrix is the volume of the parallelepiped spanned on its rows, we get 
\[|\det(\Jac_pf)|\le |d_pf_A|\cdot|d_pf_B|\le 1.\]
Since $f\:\square\to\square$ is a Lipschitz onto map, the \emph{area formula} implies that 
\[\vol(\square,g)\ge \vol\square=1.\]
\qedsf

The following generalization can be proved along the same lines.

\begin{thm}{Theorem}\label{thm:besikovitch+}
Let $(M,g)$ be Riemannian manifold and its boundary admits a homeomorphism $\partial\square^n\to\partial M$. 
Suppose $d_1,\dots, d_n$ the distances between the the images of pairs of opposite faces of $\square^n$ in $\partial M$.
Then 
\[\vol(M,g)\ge d_1\cdots d_n.\]
\end{thm}

\begin{thm}{Exercise}\label{ex:besikovitch=}
Suppose that we have equality in \ref{thm:besikovitch}.
Show that $(\square^n,g)$ is isometric to $\square^n$.
\end{thm}

\begin{thm}{Exercise}\label{ex:hexagon}
Suppose $g$ is a metric tensor on a regular hexagon $\varhexagon
   $ such that $g$-distances between the opposite sides are at least $1$.
Is there a positive lower bound on $\area(\varhexagon,g)$?
\end{thm}

\begin{thm}{Exercise}\label{ex:gadograph}
Let $V$ be a compact set in $\EE^d$ bounded by a hypersurface $\Sigma$.
Suppose $g$ is a Riemannian metric on $V$ such that 
\[\dist{p}{q}{g}\ge\dist{p}{q}{\EE^d}\]
for any two points $p,q\in \Sigma$.
Show that
\[\vol(V,g)\ge \vol(V)_{\EE^d}.\]
 
\end{thm}

\begin{thm}{Exercise}\label{ex:involution-of-sphere}
Suppose that sphere with Riemannian matric $(\mathbb{S}^2,g)$ admits an involution $\iota$ such that $\dist{x}{\iota(x)}{g}\ge 1$.

Show that $\area(\mathbb{S}^2,g)\ge \tfrac1{1000}$;
try to show that $\area(\mathbb{S}^2,g)\ge \tfrac12$ or $\area(\mathbb{S}^2,g)\ge 1$.
\end{thm}

Christopher Croke conjectured that the optimal bound for this exercise is $\tfrac4\pi$ and the round sphere is the only space that achieves this \cite[see Conjecture 0.3 in][]{croke}.

\begin{thm}{Advanced exercise}\label{ex:involution-of-3sphere}
Construct a metric $g$ on $\mathbb{S}^3$ with arbitrary small $\vol(\mathbb{S}^3,g)$ and such that it admits an involution $\iota$ such that $\dist{x}{\iota(x)}{g}\ge 1$.
\end{thm}

\section{Systolic inequlaity}

Let $\spc{M}$ be a compact Riemannian manifold.
The \emph{systole} of $\spc{M}$ (brifly $\sys\spc{M}$) is defined to be the least length of a noncontractible closed curve in $\spc{M}$.

Let $\Lambda$ be a set of smooth closed $n$-dimensional manifolds.
We say that a systolic inequality holds for $\Lambda$ if there is a constant $c$ such that for any $M\in \Lambda$ and any metric tenor $g$ on $M$ we have
\[[\sys(M,g)]^n\le c\cdot \vol(M,g).\]

\begin{thm}{Exercise}\label{ex:sysT2}
Use \ref{thm:besikovitch} to show that systolic inequality holds for the 2-torus $\TT^2$.
\end{thm}

\begin{thm}{Exercise}\label{ex:sysRP2}
Use \ref{thm:besikovitch} to show that systolic inequality holds for the real projective palane $\RP^2$.
\end{thm}

\begin{thm}{Exercise}\label{ex:sysSg}
Use \ref{thm:besikovitch+} to show that systolic inequality holds for the set of all closed surfaces of positive genus.
\end{thm}

\parbf{Remarks.}
The optimal constants in the systolic inequality are known in the following three cases:
\begin{itemize}
\item For real projective plane $\RP^2$ the constant is $\tfrac\pi2$ --- the equality holds for a quotient of a round sphere by isometric involution. The statement was prove by Pao Ming Pu \cite{pu}.\label{page:pu}
\item For torus $\TT^2$ the constant is $\tfrac2{\sqrt{3}}$ --- the equality holds for a flat torus obtained from a regular hexagon by identifying opposite sides; this is the so called \emph{Loewner's torus inequality}.
\item For the Klein bottle $\RP^2\#\RP^2$  the constant is $\tfrac\pi{2\cdot\sqrt2}$ --- the equality holds for certain nonsmooth metrics \cite{bavard}.
\end{itemize}
The proofs of these results use the so called \emph{uniformization theorem}   available in the 2-dimensional case only.
These proofs are beautiful, but they too far from metric geometry.
A good survey on the subject is written by Christopher Croke and Mikhail Katz \cite{croke-katz}.

\begin{thm}{Exercise}\label{ex:sysS2xS1}
Show that systolic inequality does \emph{not} hold for $\mathbb{S}^2\times\mathbb{S}^1$.
\end{thm}


\begin{thm}{Therorem}\label{thm:sys(torus)}
Systolic intequality holds for the $n$-dimensional torus $\TT^n$. 
\end{thm}

The proof of this theorem and its generalization will take most of the remaining lectures.
In the following section we introduce a key notion in the proof.

\section{Filling radius}

The following definition was introduced by Mikhael Gromov \cite{gromov-1983}.

Let $\spc{M}$ be a closed $n$-dimensional Reimannian manifold.
Applying Kuratowski embedding (\ref{lem:kuratowski}) $x\mapsto \distfun_x$, we may think that $\spc{M}$ as a subset of $\ell^\infty(\spc{M})$ --- the space of functions on $\spc{M}$ equipped with the metric induced by the sup-norm.

Define the \emph{filling radius} of $\spc{M}$ (briefly $\FillRad\spc{M}$) as the least upper bound on values $r>0$ such that $\spc{M}$ bounds in its $r$-neighborhood in $\ell^\infty(\spc{M})$.
In other words, if $\iota_r$ denotes inclusion of $\spc{M}$ in its $r$-neighborhood $B_r(\spc{M})\subset \ell^\infty(\spc{M})$,
then 
\[\FillRad\spc{M}\df\inf\set{r>0}{(\iota_r)_*[\spc{M}]=0\in H_n(B_r(\spc{M}))},\]
where $[\spc{M}]$ denotes the fundamental class of $\spc{M}$.

We assume that the homologies are taken with coefficients in $\ZZ_2$.
In this case $[\spc{M}]\ne0\in H_n(\spc{M})$.
If we choose coefficients $\ZZ$, then it does not hold for nonorientable manifolds.


\begin{thm}{Exercise}\label{ex:fillrad<diam/2}
Show that the inequality
\[\FillRad \spc{M}\le \tfrac12\cdot\diam \spc{M}\]
holds for any compact Riemannian manifold $\spc{M}$.
\end{thm}

\parbf{Remark.}
The optimal bound for the above exercise was found by Mikhail Katz \cite{katz}.
Namely he proved that
\[\FillRad \spc{M}\le \tfrac13\cdot\diam \spc{M}\]
and equality holds if $\spc{M}$ is real projective space with canonical metric.
The proof is beautiful, elementary, and very readable.

\medskip

The following theorem is the main ingredient in the proof of \ref{thm:sys(torus)}.
This theorem will be the main subject of the following lecture.

\begin{thm}{Theorem}\label{thm:FillRad<vol}
Given an integer $n>0$, there is a constant $c(n)$ such that inequality
\[(\FillRad \spc{M})^n\le c(n)\cdot \vol \spc{M}\]
holds for any compact $n$-dimensional Riemannian manifold $\spc{M}$.
\end{thm}

In the following section we show why this theorem is related to \ref{thm:FillRad<vol}.

\section{Systole and filling radius}

\begin{thm}{Theorem}\label{thm:sys<FillRad}
Suppose $\spc{T}= (\TT^n,g)$ is a Riemnnian manifold on $n$-dimensional torus $\TT^n$.
Then 
\[\sys\spc{T}\le 6 \cdot \FillRad \spc{T}.\]
\end{thm}

Note that \ref{thm:sys<FillRad} and \ref{thm:FillRad<vol}  imply \ref{thm:sys(torus)}.

\parit{Proof.}
As usual we consider $\spc{T}$ as a subspace in $\ell^\infty(\spc{T})$.

Set $s=\sys\spc{T}$ and $\FillRad\spc{T}=r$.
Arguing by contradiction, assume $6\cdot r< s$;
so $\eps=\tfrac1{100}\cdot(s-6\cdot r)>0$.

Choose a simplicial complex $\Sigma$ and a map $\sigma\:\Sigma\to \ell^\infty(\spc{T})$ such that the restriction $\sigma|_{\partial\Sigma}$
represents the fundamental class $[\spc{T}]$ of $\spc{T}$
and $\sigma(\Sigma)\subset B_{r+\eps}(\spc{T})$.


Passing to barycentric subdivision few times, we may assume that the $\sigma$-image of any simplex in $\Sigma$ has diameter less than $\eps$.
We may perturb the map slightly to ensure that each edge $e$ of $\Sigma$ is mapped to a geodesic and still $\sigma|_{\partial\Sigma}$
represents the fundamental class $[\spc{T}]$ of $\spc{T}$.

Let us construct a continuous map
$f\:\Sigma\to  \spc{T}$ which agrees with $\sigma$ on $\partial \Sigma$.
Once it is done we get that $[\spc{T}]=0\in H_n(\spc{T})$ --- a contradiction.

Set $f(x)=\sigma(x)$ for every $x\in \partial \Sigma$;
on the remaining part of $\Sigma$ we will construct $f$ recurcevely on the skeletons $\Sigma^0$, $\Sigma^1$, $\Sigma^2$ and so on.

For every vertex $v$, set $f(v)$ to be the closest point in $\spc{T}$ to $\sigma(v)$.
Note that if $v\in\partial\Sigma$, then $f(v)=\sigma(v)$.
This way we defined $f$ on $\Sigma^0$.

Let $e$ be an edge in $\Sigma$ between vertexes $v$ and $w$.
Note that 
\begin{align*}
\dist{f(v)}{f(w)}{}
&\le\dist{f(v)}{\sigma(v)}{}
+\dist{\sigma(v)}{\sigma(w)}{}
+\dist{\sigma(w)}{f(w)}{}\le
\\
&\le (r+\eps)+\eps +(r+\eps)<
\\
&<\tfrac s3.
\end{align*}
Map $e$ to a shortest path $[f(v)\,f(w)]$ in $\spc{T}$;
if $e$ is an edge in $\partial \Sigma$ then no need to change $f$ on it.
This extends $f$ to $\Sigma^1$ such that each edge is mapped to a geodesic of length less that $\tfrac s3$.

Now for each triangle $uvw$ in $\Sigma$, the closed curve formed by $f$-images of its sides has length less than $s$.
That is, it is shorter than any noncontractible closed curve,
and therefore it is null-homotopic in $\spc{T}$.
Hence we can extend $f$ to the $\Sigma^2$.

Finally, since $\spc{T}$ is aspherical, there is no obstruction to extending $f$ to the rest of $\Sigma$.
\qeds

Observe that we use only that $\spc{T}$ is aspherical closed manifold;
this statement will be generalized yet further.

\begin{thm}{Exercise}\label{ex:fillrad-inj}
Modify the proof of \ref{thm:sys<FillRad} to prove the following:

Suppose that $\spc{M}$ is a closed $n$-dimensional Reimannian manifold with \emph{injectivity radius} at least $r$; that is, if $\dist{p}{q}{\spc{M}}<r$, then there is geodesic $[pq]_{\spc{M}}$ is uniquely defined.
Show that
\[\FillRad\spc{M}\ge \tfrac{r}{n+1}.\]
 
\end{thm}

Note that this exercise together with bound on filling radius in \ref{thm:FillRad<vol} imply that lower bound on injectivity radius implies a lower bound on volume.


\chapter{Width}

This lecture is based on a paper of Alexander Nabutovsky \cite{nabutovsky}.

\section{Nerves and partition of unity}

Let $\{V_1,\dots,V_k\}$ be a finite open cover of a compact metric space $\spc{X}$.
Consider an abstract simplicial complex $\spc{N}$, with one vertex $v_i$ for each set $V_i$ such that a simplex with vertexes $v_{i_1},\dots, v_{i_m}$ is included in $\spc{N}$ if 
the intersection $V_{i_1}\cap\dots\cap V_{i_m}$ is nonempty.
The obtained simplicial complex $\spc{N}$ called the \index{nerve}\emph{nerve of the covering $\{V_i\}$}.

Note that $\spc{N}$ is a finite simplicial complex;
it is a subcomplex of a simplex with the vertixes $\{v_1,\dots,v_k\}$.
The nerve $\spc{N}$ has dimension at most $n$ if and only if the covering $\{V_1,\dots,V_k\}$ has multiplicity is at most $n+1$;
that is, any point $x\in\spc{X}$ belongs to
at most $n+1$ sets of the covering.

\begin{thm}{Proposition}\label{thm:part-unit}
 Let $\{V_1,\dots,V_k\}$ be a finite open covering of a compact metric space ${\spc{X}}$.
Then there are Lipschitz functions $\psi_i\:{\spc{X}}\z\to[0,1]$ such that
if $\psi_i(x)>0$ then $x\in V_i$ and
$$\sum_i\psi_i(x)=1$$
for any $x\in {\spc{X}}$.
\end{thm}

\parit{Proof.}
Consider functions $\phi_i\:{\spc{X}}\to\RR$ defined as
$$\phi_i(x)=\distfun_{({\spc{X}}\backslash V_i)} x.$$
Note $\phi_i$ is $1$-Lipschitz
for any $i$
and $\phi_i(x)>0$ if and only if $x\in V_i$.
Since $\{V_i\}$ is a covering, we have that
$$\sum_i\phi_i(x)>0\ \ \text{for any}\ \ x\in {\spc{X}}.$$

Set 
$$\psi_k(x)=\frac{\phi_k(x)}{\sum_i\phi_i(x)}.$$
Observe that by construction the functions $\psi_i$ meet the conditions in the proposition.
\qedsf

A collection of functions $\{\psi_i\}$ that meets the conditions in \ref{thm:part-unit} is called 
a \index{partition of unity}\emph{partition of unity subordinate to the open covering} $\{V_1,\dots,V_k\}$.

Suppose $\{\psi_i\}$ is  
a partition of unity subordinate to the open covering $\{V_1,\dots,V_k\}$.
Note that for any point $x\in {\spc{X}}$, the set
$$\set{v_i}{\psi_i(x)>0}$$
describe vertexes of a simplex in the nerve.
Therefore 
$$\psi\:x\mapsto \psi_1(x)\cdot v_1+\psi_2(x)\cdot v_2+\dots+\psi_k(x)\cdot v_n.$$
describes a Lipschitz map from ${\spc{X}}$ to the nerve $\spc{N}$ of $\{V_i\}$;
here the point $x$ is mapped to the point with barycentric coordinates $\psi_i(x)$.
In other words we proved the following:

\begin{thm}{Proposition}\label{prop:space->nerve}
Let $\spc{N}$ be a nerve of an open covering $\{V_1,\z\dots,V_k\}$ of a compact metric space $\spc{X}$.
Denote by $v_i$ the vertex of $\spc{N}$ that corresponds to $V_i$.

Then there is a Lipschitz map $\psi\:\spc{X}\to\spc{N}$ such that $\psi(V_i)\z\subset\Star_{v_i}$ for every $i$;
that is, for any $x\in V_i$ the point $\psi(x)$ lies the interior of some simplex with vertex $v_i$.
\end{thm}


\section{Width}

Suppose $A$ is a subset of a metric space $\spc{X}$.
The radius of $A$ (briefly $\rad A$) is defined as the least upper bound on the values $R>0$ such that $\oBall(x,R)\supset A$ for some $x\in \spc{X}$.

\begin{thm}{Definition}\label{def:width}
Let $\spc{X}$ be a metric space.
The $n$-th width of $\spc{X}$ (briefly $\width_n\spc{X}$) is defined as the least upper bound on values $R>0$ such that $\spc{X}$ admits a finite open covering $\{V_i\}$ with multiplicity at most $n+1$ and $\rad V_i< R$ for each $i$.
\end{thm}

\parit{Remarks.}
\begin{itemize}
\item Observe that 
\[\width_0\spc{X}\ge\width_1\spc{X}\ge\dots\]
for any compact matric space $\spc{X}$.
Moreover, if $\spc{X}$ is connected, then 
\[\width_0\spc{X}=\rad\spc{X}.\]
\item 
Usually width is defined using diameter instead of radius, but the result differ at most twice.
Namely if $r$ is the radius-width and $d$ --- diameter-width for the same $n$, then 
$r\le d\le 2\cdot r$.

\item Note that \index{Lebesgue covering dimension}\emph{Lebesgue covering dimension} of $\spc{X}$ can be defined as the least number $n$ such that $\width_n\spc{X}=0$.
Another closely related notion is the so called \index{macroscopic dimesion on scale $R$}\emph{macroscopic dimesion on scale $R$};
it is defined as the  least number $n$ such that $\width_n\spc{X}<R$.
\end{itemize}

\begin{thm}{Exercise}\label{ex:macrodimension}
Suppose $\spc{X}$ is a compact metric space such that any closed curve $\gamma$ in $\spc{X}$ can be contracted in its $R$-neighborhood.
Show that $\spc{X}$ has macroscopic dimension at most 1 on scale $100\cdot R$.

What about quasiconverse? That is, suppose a simply connected compact metric space $\spc{X}$ has macroscopic dimension at most 1 on scale $R$, is it true that any closed curve $\gamma$ in $\spc{X}$ can be contracted in its $100\cdot R$-neighborhood?
\end{thm}


The following proposition provides an equivalent definition;
we will not use it, but it provides a good reason for the name \index{width}\emph{width}.

\begin{thm}{Proposition}\label{prop:width=suprad(inv)}
Suppose $\spc{X}$ is a compact metric space.
Then $\width_n\spc{X}<R$ if and only if there is a finite $n$-dimensional simplicial complex $\spc{N}$ and a continuous map $\psi\:\spc{X}\to \spc{N}$
such that $\rad[\psi^{-1}(s)]\z<R$
for any $s\in \spc{N}$.
\end{thm}

\parit{Proof; ``only if'' part.}
Suppose $\width_n\spc{X}<R$.
Consider a covering $\{V_1,\dots,V_k\}$ of $\spc{X}$ guaranteed by the definition of width.
Let $\spc{N}$ be its nerve and $\psi\:\spc{X}\to \spc{N}$ be the map provided by \ref{prop:space->nerve}.

Since the multiplicity of the covering is at most $n+1$, we ahve $\dim \spc{N}\le n$.

Note that if $x\in \spc{N}$ lies in a star of a vertex $v_i$,
then $\psi^{-1}\{x\}\z\subset V_i$;
in particular $\rad[\psi^{-1}\{x\}]<R$.

\parit{``If'' part.}
Choose $x\in \spc{N}$.
Since the inverse image $\psi^{-1}\{x\}$ is compact, $\psi$ is continuous, and $\rad[\psi^{-1}\{x\}]<R$,
there is a neighborhood $U\ni x$ such that the  $\rad[\psi^{-1}(U)]<R$.

Since $\spc{X}$ is compact,  there is a finite cover $\{U_i\}$ of $\spc{N}$ such that $\psi^{-1}(U_i)\subset\spc{X}$ has radius smaller than $R$ for each $i$.
Since $\spc{N}$ has dimension $n$, we can inscribe%
\footnote{Recall that a covering $\{W_i\}$ is inscribed in the covering $\{U_i\}$ if for every $W_i$ is a subset of some $U_j$.} 
in $\{U_i\}$ a finite open cover $\{W_i\}$ with multiplicity at most $n+1$.
It remains to observe that $V_i=\psi^{-1}(W_i)$ defines a finite open cover of $\spc{X}$ with radius less than $R$ and multiplicity at most $n+1$. 
\qeds

\section{Riemannian polyhedrons}

A \index{Riemannian polyhedron}\emph{Riemannian polyhedron} is defined as a finite simplicial complex with a metric tensor on each simplex such that the restriction of the metric on each simplex to a subsymplex coinsides with the metric on the subsmplex.
The dimension of Riemannian polyhedron is defined as the largest dimension it its triangulation.
For Riemannian polhedron one can define length of curves and volume the same way as for Riemannian manifolds.


Further we will apply the notion of width to compact Riemannian polyhedrons.
If $\spc{P}$ is an $n$-dimensional compact Riemannian polyhedron, then 
we suppose that
\[\width\spc{P}\df\width_{n-1}\spc{P}.\]

Let $\spc{P}$ be an $n$-dimensional Riemnnian polyhedron.
Let us define \index{volume profile}\emph{volume profile} of $\spc{P}$ as a function 
returning volume of largest $r$-ball in $\spc{P}$;
that is, $\VolPro_{\spc{P}}\:\RR_+\to\RR_+$ is defined by 
\[\VolPro_{\spc{P}}(r)\df \sup\set{\vol_n \oBall(p,r)}{p\in\spc{P}}.\]
Note that $\VolPro_{\spc{P}}$ is a nondecreasing function and $\VolPro_{\spc{P}}(r)\z\to\vol_n\spc{P}$ as $r\to\infty$.

Note that if $\spc{P}$ is a 1-dimensional connected Riemannian polyhedron, then 
\[\width\spc{P}=\width_0\spc{P}=\rad\spc{P}.\]

\begin{thm}{Exercise}\label{ex:1D-case}
Suppose $\spc{P}$ be a 1-dimensional Riemannian polyhedron.
Suppose $\VolPro_{\spc{P}}(R)<R$ for some $R>0$.
Show that 
\[\width \spc{P}<R.\]
Try to show that $c=\tfrac 12$ is the optimal constant such that 
\[\width \spc{P}<c\cdot R.\]
\end{thm}

\section{Volume profile bounds width}

The following theorem and its corollary is the main goal of this lecture.

\begin{thm}{Theorem}\label{thm:width<volpro}
Let $\spc{P}$ be an $n$-dimensional Reimannian polyhedron. 
If the inequality 
\[R> n\cdot \sqrt[n]{\VolPro_{\spc{P}}(R)}\]
holds for {}\emph{some} $R>0$, then 
\[\width\spc{P}\le  R.\]
\end{thm}

Since $\VolPro_{\spc{P}}(r)\le \vol\spc{P}$ for any $r$,
we get the following:

\begin{thm}{Corollary}\label{thm:width<vol}
For any $n$-dimensional Reimannian polyhedron $\spc{P}$, we have
\[\width\spc{P}\le n\cdot \sqrt[n]{\vol\spc{P}}.\]

\end{thm}

In the proof of \ref{thm:width<volpro}, we will use the following three technical statements,
the proofs are omitted, but they are not hard. 

\begin{thm}{Smoothing procedure}
Let $\spc{P}$ be a Reimannian polyhedron and $f\:\spc{P}\to \RR$ be a 1-Lipschitz function.
Then for any $\delta>0$ there is a  1-Lipschitz function $\tilde f\:\spc{P}\to \RR$ that is smooth on each simplex of the triangulation and $\delta$-close to $f$.
\end{thm}

\begin{thm}{Sard's theorem}
Let $\spc{P}$ be an $n$-dimensional Reimannian polyhedron and $f\:\spc{P}\to \RR$ be a function that is smooth on each simplex.
Then for almost all values $a$, each component of the inverse image $f^{-1}\{a\}$ equipped with the induced metric is a Reimannian polyhedron of dimension at most $n-1$.
\end{thm}


\begin{thm}{Coarea inequality}
Let $\spc{P}$ be an $n$-dimensional Reimannian polyhedron and $f\:\spc{P}\to \RR$ be a 1-Lipschitz function that is smooth on each simplex.
Set $V=\vol_n (f^{-1}[a,b])$.
Then 
\[\int_a^b\vol_{n-1}(f^{-1}\{x\})\cdot dx\ge V .\]
In particular there is a subset of positive measure $A\subset [a,b]$ such that the inequality 
\[\vol_{n-1}(f^{-1}\{x\})\ge \frac V{b-a}\]
holds for any $x\in A$.
\end{thm}

\begin{thm}{Definition}
Let $\spc{P}$ be an $n$-dimensional Riemannian polyhedron.
An $(n-1)$-dimensional subpolyhedron $\spc{Q}\subset\spc{P}$ is called \index{separating subpolyhedron}\emph{$R$-separating} if $\rad U<R$ for each connected component $U$ of the complement $\spc{P}\backslash \spc{Q}$.
\end{thm}



\begin{thm}{Lemma}\label{lem:separating}
Let $\spc{P}$ be an $n$-dimensional Riemannian polyhedron.
Then given $R>0$ and $\eps>0$ there is a $R$-separating subpolyhedron $\spc{Q}\subset\spc{P}$ such that for any $r_0<r_1\le R$ we have
\[\VolPro_{\spc{Q}}(r_0)< \tfrac1{r_1-r_0}\cdot \VolPro_{\spc{P}}(r_1)+\eps.\]

\end{thm}

\parit{Proof.}
Choose a small $\delta>0$.
Applying the smoothing procedure, we can exchange each distance function $\distfun_p$ on $\spc{P}$ by $\delta$-close smooth 1-Lipschitz function, which will be denoted by $\widetilde \distfun_p$.

By Sard's theorem, almost all level sets 
\[\tilde S_c(p)=\set{x\in \spc{P}}{\widetilde \distfun_p(x)=c}\]
are smooth Riemannian polyhedrons of dimension at most $n-1$.
Since $\delta$ is small, the coarea inequality implies that 
for we can choose $c\z\in(r_0+\delta, r_1-\delta)$ such that $\tilde S_c(p)$ is a subpolyhedron and 
\begin{align*}
\vol_{n-1}\tilde S_c(p)&\le \tfrac1{r_1-r_0-2\cdot\delta}\cdot\vol_n[\oBall(p,r_1)]<
\\
&<\tfrac1{r_1-r_0}\cdot \VolPro_{\spc{P}}(r_1)+\tfrac\eps2.
\end{align*}

Suppose $\spc{Q}$ is an $R$-separating subpolyhedron in $\spc{P}$ with almost minimal volume, say its volume is at most $\tfrac\eps2$-far from the greatest lower bound.
Note that cutting from $\spc{Q}$ everything inside $\tilde S_c(p)$ and adding $\tilde S_c(p)$ keeps it to be $R$-separating subpolyhedron.
Since $\spc{Q}$ has almost minimal volume, we have
\[\vol_{n-1}[\spc{Q}\cap \oBall(p,r_0)_{\spc{P}}]-\tfrac\eps2\le \vol_{n-1}S_c(p).\]
Therefore 
\[\vol_{n-1}[\spc{Q}\cap \oBall(p,r_0)_{\spc{P}}]\le\tfrac1{r_1-r_0}\cdot \VolPro_{\spc{P}}(r_1)+\eps\eqlbl{eq:volQ<ProP}\]
Recall that $\spc{Q}$ is equipped with the induced length metric;
therefore $\dist{p}{q}{\spc{Q}}\ge \dist{p}{q}{\spc{P}}$ for any $p,q\in \spc{Q}$;
in particular, 
\[\oBall(p,r_0)_{\spc{Q}}\subset \spc{Q}\cap \oBall(p,r_0)_{\spc{P}}\]
for any $p\in \spc{Q}$ and $r\ge 0$.
Hence \ref{eq:volQ<ProP} implies the lemma.
\qeds

\begin{thm}{Lemma}\label{lem:separating-width}
Let $\spc{Q}$ be an $R$-separating subpolyhedron in an $n$-dimensional Riemannian polyhedron $\spc{P}$.
Suppose $\width\spc{Q}\le R$.
Then $\width\spc{P}\le R$
\end{thm}

\parit{Proof.}
Start with an open covering $\{V_1,\dots,V_k\}$ of $\spc{Q}$ of multiplicity $\le n$ with radiuses of the sets in the intrinsic metric $\le R$.

Note that $\{V_1,\dots,V_k\}$ can be converted into an an open covering of
a small neighbourhood of $\spc{Q}$ in $\spc{P}$ without increasing the multiplicity.
This is can be done by setting 
\[V_i'=\bigcup_{x\in V_i}\oBall(x,r_x),\]
where $r_x=\tfrac1{10}\cdot\inf\set{\dist{x}{y}{}}{y\in \spc{Q}\backslash V_i}$.

Adding to  $\{V_i'\}$ all the components of $\spc{P}\backslash \spc{Q}$,
we increase the multiplicity by at most 1 and obtain a covering of $\spc{P}$.
The statement follows since $\dim \spc{P}= \dim \spc{Q}\z+1$.
\qeds

\parit{Proof of \ref{thm:width<volpro}.}
We apply induction on the dimension $n=\dim\spc{P}$.
The base case $n=1$ is given in \ref{ex:1D-case}.

Suppose that the  $(n-1)$-dimensional case is proved.
Consider an $n$-dimensional Riemannian polyhedron $\spc{P}$ and suppose
\[n\cdot \sqrt[n]{\VolPro\spc{P}(R)}< R\]
for some $R>0$.
Let $\spc{Q}$ be an $R$-separating subpolyhedron in $\spc{P}$ provided by \ref{lem:separating} for a small $\eps>0$.
Applying  \ref{lem:separating} for $r=\tfrac{n-1}n\cdot R$ and $R$, we have that 
\begin{align*}
\VolPro_\spc{Q}(r) &< \frac 1{R-r}\cdot \VolPro_\spc{P}(R)+\eps<
\\
&<\frac {n}{R}\cdot\left(\frac{R}{n}\right)^n=
\\
&=\left(\frac{r}{n-1}\right)^{n-1};
\end{align*}
that is, $(n-1)\cdot \sqrt[n-1]{\VolPro\spc{Q}(r)}< r$.
Since $\dim\spc{Q}\le n-1$, by the induction hypothesis, we get that
\[\width\spc{Q}\le r<R.\]
It remains to apply \ref{lem:separating-width}.
\qeds





\section{Width bounds systole}

\begin{thm}{Theorem}\label{thm:sys<width}
Suppose $\spc{M}$ is a aspherical $n$-dimensional Riemnnian manifold.
Then 
\[\sys\spc{M}\le 6 \cdot \width \spc{M}.\]
\end{thm}

\begin{thm}{Lemma}\label{lem:aspherical-homotopy}
Let $M$ be an aspherical space and $L$ be a connected CW-complex.
Denote by $L^k$ the k-skeleton of $L$.
Then any continuous map $f\:L^2\to M$ can be extended to a continuous map $\bar f\:L\to M$

Moreover, if $p\in L$ is a 0-cell and $q\in M$.
Then a continuous maps of pairs $\phi_0,\phi_1\:(L,p)\to(M,q)$ are homotopic if and only if $\phi_0$ and $\phi_1$ induce the same homomorphism on fundamental groups $\pi_1(L,p)\to\pi_1(M,q)$.
\end{thm}

\parit{Proof.}
Since $M$ is aspherical, any continuous map $\partial\mathbb{D}^n\:\to M$ for $n\ge 3$
is hull-homotopic;
that is, it can be extended to a map $\mathbb{D}^n\:\to M$.

It makes possible to extend $f$ to $L^3$, $L^4$, and so on.
Therefore $f$ can be extended to whole $L$.

The only-if part on the second part of lemma is trivilal; let us show the if part.

Sine $L$ is connected, we can assume that $p$ forms the only 0-cell in $L$;
otherwise we can collapse a maximal sub-tree of the 1-skeleton in $L$ to $p$.
Therefore $L^1$ is formed by loops that generates $\pi_1(L,p)$.

By assumption, the restrictions of $\phi_0$ and $\phi_1$ to $L^1$ are homotopic.
In other words the homotopy $\Phi\:[0,1]\times L$ is defined on the 2-skeleton of $[0,1]\times L$.
It remains to apply the first part of the lemma.
\qeds



\begin{thm}{Lemma}\label{lem:sys-homotopy}
Suppose $\gamma_0,\gamma_1$ are two paths between points in a Riemannian space $\spc{M}$ such that $\dist{\gamma_0(t)}{\gamma_1(t)}{\spc{M}}<r$ for any $t\in[0,1]$.
Let $\alpha$ be a geodesic path from $\gamma_0(0)$ to $\gamma_1(0)$ and $\beta$ be a geodesic path from $\gamma_0(1)$ to $\gamma_1(1)$. 
If $2\cdot r<\sys\spc{M}$, then there is a homotopy $\gamma_t$ from
$\gamma_0$ to $\gamma_1$ such that $\alpha(t)= \gamma_t(0)$ and $\beta(t)\mapsto \gamma_t(1)$.
\end{thm}

\parit{Proof.}
Set $s=\sys\spc{M}$; 
since $2\cdot r<s$, we have that $\eps=\tfrac1{10}(s-2\cdot r)>0$.

Note that we can assume that $\gamma_0$ and $\gamma_1$ are rectifiable;
if not we can homotopy each into a broken geodesic line kipping the assumptions true. 

\begin{wrapfigure}{r}{34mm}
\vskip-0mm
\centering
\includegraphics{mppics/pic-1405}
\end{wrapfigure}

Choose a fine partition $0\z=t_0\z<t_1\z<\z\dots\z<t_n=1$.
Consider a sequence of geodesic paths $\alpha_i$ from $\gamma_0(t_i)$ to $\gamma_1(t_i)$;
we can assume that $\alpha_0=\alpha$ and $\alpha_n=\beta$.
We can assume that each arc $\gamma_j|_{[t_{i-1},t_i]}$ has length smaller than $\eps$.
Therefore every quadrilateral formed by concatenation  of $\alpha_{i-1}$, $\gamma_1|_{[t_{i-1},t_i]}$, reversed $\alpha_i$, and reversed arc $\gamma_0|_{[t_{i-1},t_i]}$ has length smaller than $s$.
It follows that this curve is contractible.
Applying this observation for each quadrilateral, we get the statement.
\qeds


\parit{Proof of \ref{thm:sys<width}.}
Let $\spc{N}$ be the nerve of a covering $\{V_i\}$ of $\spc{M}$ and $\psi\:\spc{M}\to\spc{N}$ be the map provided by \ref{prop:space->nerve}.
As usual, we denote by $v_i$ the vertex of $\spc{N}$ that corresponds to $V_i$.

Set $R=\width \spc{M}$ and $s=\sys\spc{M}$.
Assume we chose $\{V_i\}$ as in the definition of width (\ref{def:width}).
For each $i$ choose a point $p_i\in \spc{M}$ such that $V_i\subset \oBall(p_i,R)$.
Observe that in this case $\dim\spc{N}<n$;
therefore $\psi$ kills the fundamental class of $\spc{M}$.

Let us construct a continuous map  $f\:\spc{N}\to  \spc{M}$ such that
$f\circ\psi$ is homotopic to the identity map on $\spc{M}$.

Note that once $f$ is constructed, the theorem is proved, .
Indeed, since $\psi$ kills the fundamental class of $\spc{M}$, so does $f\circ\psi$.
Therefore $[\spc{M}]=0$ --- a contradiction.

First set $f(v_i)=p$.
It defines the map $f$ on the 0-skeleton $\spc{N}^0$ of the nerve $\spc{N}$.
Further we will be define $f$ step by step on the skeletons of higher dimensions $\spc{N}^1,\spc{N}^2, \dots$

Let us map each edge $[v_iv_j]$ in $\spc{N}$ to a geodesic $[p_ip_j]$.
It defines the map on the 1-skeleton $\spc{N}^1$ of the nerve $\spc{N}$.
Note that image of each edge is shorter that $2\cdot R$.

Suppose $[v_iv_jv_k]$ is a triangle in $\spc{N}$.
Note that perimeter of the triangle $[p_ip_jp_k]$ can not exceed $6\cdot R$.
Since $6\cdot R<s$, the contour of $[p_ip_jp_k]$ is contractible.
Therefore we can extend $f$ to each triangle of~$\spc{N}$.
It defines the map $f$ on $\spc{N}^2$.

Finally, since $\spc{M}$ is aspherical, by \ref{lem:aspherical-homotopy}, the map $f$ can be extended to $\spc{N}^3$, $\spc{N}^4$ and so on.

It remains to show that $f\circ\psi$ is homotopic to the identity map.
Choose a CW structure on $\spc{M}$ with sufficiently small cell, so that each cell lies in one of $V_i$.
Note that $\psi$ is homotopic to a map $\psi_1$ that sends $\spc{M}^k$ to $\spc{N}^k$ for any $k$.
Moreover we may assume that (1) if a 0-cell $x$ of $\spc{M}$ maps to a $v_i$, then $x\in V_i$ and (2) each 1-cell  of $\spc{M}$ maps to an edge of $\spc{N}$.
Choose a 1-cell $e$ in $\spc{M}$; by the construction, $f\circ\psi_1$ maps $e$ to a geodesic $[p_ip_j]$ and $e$ lies $\oBall(p_i,R)$.
Observe that $[p_ip_j]$ is shorter than $2\cdot R$.
It follows that the distance between points on $[p_ip_j]$ and $e$ can not exceed $3\cdot R$.
Choose a geodesic path $\alpha_i$ from every 0 cell $x_i$  of $\spc{M}$ to $p_j=f\circ\psi_1(x_i)$.
It defines a homotopy on $\spc{M}^0$.
Since $6\cdot R<s$, \ref{lem:sys-homotopy} implies that this homotopy can be extended to $\spc{M}^1$.
By \ref{lem:aspherical-homotopy}, it can be extended to whole $\spc{M}$.
\qeds

\begin{thm}{Exercise}\label{ex:sys<width}
Analyze the proof of \ref{thm:sys<width} and improve its inequality to 
 \[\sys\spc{M}\le 4 \cdot \width \spc{M}.\]
\end{thm}

\begin{thm}{Exercise}\label{ex:fillrad-inj}
Modify the proof of \ref{thm:sys<width} to prove the following:

Suppose that $\spc{M}$ is a closed $n$-dimensional Reimannian manifold with \emph{injectivity radius} at least $r$; that is, if $\dist{p}{q}{\spc{M}}<r$, then there is geodesic $[pq]_{\spc{M}}$ is uniquely defined.
Show that
\[\width\spc{M}\ge \tfrac{r}{2\cdot(n+1)}.\]

Use \ref{thm:width<vol} to conclude that  
\[\vol\spc{M}\ge \eps_n \cdot r^n \]
for some $\eps_n>0$ that depends only on $n$.
\end{thm} 

\section{Essential manifolds}

To generalize \ref{thm:sys<width} bit further, we need the following definition.

\begin{thm}{Definition}
A closed manifold $\spc{M}$ is called \index{essential manifold}\emph{essential} if it admits a continuous map $\iota\:\spc{M}\to \spc{K}$ to an aspherical topological space $\spc{K}$ such that $\iota$ sends the fundamental class of $\spc{M}$ to a nonzero homology class in $\spc{K}$.\footnote{We assume that the coefficients are $\ZZ_2$, but one can play with them if necessary.}
\end{thm}

Assume that the manifold $\spc{M}$ is essential and $\iota \:\spc{M}\to \spc{K}$ as in the definition.
Following the proof of \ref{thm:sys<width}, we can homotope the map 
$f\circ\psi\:\spc{M}\to \spc{M}$ to the identity on the 2-skeleton of $\spc{M}$;
further since $\spc{K}$ is aspherical we can homotopy the composition
$\iota\z\circ f\circ\psi$ to  $\iota$. 
Existence of this extension implies that that $\iota$ kills the fundamental class of $\spc{M}$ --- a contradiction.
So, taking \ref{ex:sys<width} into account, we proved the following yet more general theorem.

\begin{thm}{Theorem}\label{thm:sys<width++}
Suppose $\spc{M}$ is an essential Riemnnian space.
Then 
\[\sys\spc{M}\le 4 \cdot \width \spc{M}.\]
\end{thm}

As a corollary form \ref{thm:sys<width++} and \ref{thm:width<vol} we get the so called \index{Gromov's systolic inequality}\emph{Gromov's systolic inequality}:

\begin{thm}{Theorem}\label{thm:sys+}
Suppose $\spc{M}$ is an essential $n$-dimensional Riemannian space.
Then 
\[\sys\spc{M}\le 4 \cdot n\cdot \sqrt[n]{\vol\spc{M}}.\]
\end{thm}


Note that any closed aspherical manifold is essential --- in this case one can take $\iota$ to be the identity map on $\spc{M}$.
The real projective space $\RP^n$ provides an interesting example of an essential manifold which is not aspherical.
Indeed, the infinite dimensional projective space $\RP^\infty$ is aspherical and for the natural embedding $\RP^n\hookrightarrow\RP^\infty$ the image $\RP^n$ does not bound in $\RP^\infty$.
The following exercise provides more examples of that type.

\begin{thm}{Exercise}\label{ex:connected-sum-essential}
Show that connected sum of an essential manifold with any closed manifold is essential.
\end{thm}

\begin{thm}{Exercise}\label{ex:product-essential}
Show that product of two essential manifolds is essential.

Show that product of nonessential closed manifold of dimension at least 1 with any closed manifold is not essential.
\end{thm}

\section{Remarks}

Theorem \ref{thm:sys+} was proved originally by Mikhael Gromov \cite{gromov-1983} with much worse constant.
The given proof is a result of a sequence of simplifications given by Larry Guth \cite{guth},  Panos Papasoglu \cite{papasoglu}, Alexander Nabutovsky and Roman Karasev \cite{nabutovsky}.

In \cite{nabutovsky} the calculations were optimized better which gave the constants 
$c_n=\sqrt[n]{n!}= \tfrac ne+o(n)$ in \ref{thm:width<vol} instead of $n$.
As a result, we have a stronger statement in \ref{thm:sys+}:
\[\sys\spc{M}\le 4 \cdot c_n\cdot \sqrt[n]{\vol\spc{M}}.\]

A wide open conjecture says that the optimal constant is $\pi/\sqrt[n]{\omega_n/2}$ where $\omega_n$ denotes the volume of $n$-dimensional unit sphere.
This is the systole ratio for the $n$-dimensional real projective space with canonical metric; it  grows as $O(\sqrt n)$.



%\chapter{Volume bounds filling radius}

This chapter 
is devoted to a proof of \ref{thm:FillRad<vol};
that is, we will show that \emph{Riemannian manifolds with small volume have small filling radius}.
Note that once it is proved, \ref{thm:sys<FillRad} implies \ref{thm:sys(torus)}.
Moreover \ref{thm:sys<FillRad++} implies the following:

\begin{thm}{Therorem}\label{thm:sys(torus)}
Systolic intequality holds for any essential manifold. 
\end{thm}

This theorem was proved originally by Mikhael Gromov \cite{gromov-1983}.
We follow closely a simplified proof given by Alexander Nabutovsky, which is based on a sequence of other simplifications and improvements; see \cite{nabutovsky} and the references therein.

\section{Nerves and partition of unity}

Let $\{V_1,\dots,V_k\}$ be a finite open cover of a compact metric space $\spc{X}$.
Consider an abstract simplicial complex $\spc{N}$, with one vertex $v_i$ for each set $V_i$ such that a simplex with vertexes $v_{i_1},\dots, v_{i_m}$ is included in $\spc{N}$ if 
the intersection $V_{i_1}\cap\dots\cap V_{i_m}$ is nonempty.
We obtain a simplicial complex $\spc{N}$ called the \index{nerve}\emph{nerve of the covering $\{V_i\}$}.

Note that $\spc{N}$ is a finite simplicial complex and it has dimension at most $n$ if and only if the covering $\{V_1,\dots,V_k\}$ has multiplicity is at most $n+1$;
that is, at most $n+1$ different sets $V_{i_1},\dots, V_{i_{n+1}}$ have a nonempty intersection.
The nerve $\spc{N}$ is a subcomplex of a simplex with the vertixes $\{v_1,\dots,v_k\}$.

\begin{thm}{Proposition}\label{thm:part-unit}
 Let $\{V_1,\dots,V_k\}$ is a finite open covering of a compact metric space ${\spc{X}}$.
Then there are Lipschitz functions $\psi_i\:{\spc{X}}\to[0,1]$ such that
if $\psi_i(x)>0$ then $x\in V_i$ and
$$\sum_i\psi_i(x)=1$$
for any $x\in {\spc{X}}$.
\end{thm}

A collection of functions $\psi_i$ with above properies is called 
a \emph{partition of unity subordinate to the open cover}\index{partition of unity} $\{V_1,\dots,V_k\}$.

\parit{Proof.}
Consider the functions $\phi_i\:{\spc{X}}\to\RR$ defined as
$$\phi_i(x)=\distfun_{({\spc{X}}\backslash V_i)} x.$$
Note $\phi_i$ is $1$-Lipschitz
for any $i$
and $\phi_i(x)>0$ if and only if $x\in V_i$.
In particular, 
$$\sum_i\phi_i(x)>0\ \ \text{for any}\ \ x\in {\spc{X}}.$$

Set 
$$\psi_k(x)=\frac{\phi_k(x)}{\sum_i\phi_i(x)}.$$
It remains to note that by construction the functions $\psi_i$ meet the conditions in the proposition.
\qedsf


Note that in the above proof for any point $x\in {\spc{X}}$,
the set
$$\set{v_i}{\psi_i(x)>0}$$
describe vertexes of a simplices in the nerve.
Therefore 
$$\psi\:x\mapsto \psi_1(x)\cdot v_1+\psi_2(x)\cdot v_2+\dots+\psi_k(x)\cdot v_n.$$
can be thought of as a Lipschitz map from ${\spc{X}}$ to the nerve $\spc{N}$ of $\{V_i\}$;
where the point $x$ is mapped to the point with barycentric coordinates $\psi_i(x)$.
In other words we proved the following:

\begin{thm}{Proposition}\label{prop:space->nerve}
Let $\spc{N}$ be a nerve of an open covering $\{V_1,\z\dots,V_k\}$ of a compact metric space $\spc{X}$.
Denote by $v_i$ the vertex of $\spc{N}$ that corresponds to $V_i$.

Then there is a Lipschitz map from $\psi\:\spc{X}\to\spc{N}$ such that $\psi(V_i)\z\subset\Star_{v_i}$ for every $i$.
\end{thm}


\section{Width}

Suppose $A$ is a subset of a metric space $\spc{X}$.
The radius of $A$ (briefly $\rad A$) is defined as the least upper bound on the values $R>0$ such that $\oBall(x,R)\supset A$ for some $x\in \spc{X}$.

\begin{thm}{Definition}\label{def:width}
Let $\spc{X}$ be a metric space.
The $n$-th width of $\spc{X}$ (briefly $\width_n\spc{X}$) is defined as least upper bound on values $R>0$ such that $\spc{X}$ admits a finite open covering $\{V_i\}$ with multiplicity at most $n+1$ and $\rad V_i< R$ for each $i$.
\end{thm}

\parit{Remarks.}
\begin{itemize}
\item Observe that if $\spc{X}$ is connected, then 
\[\width_0\spc{X}=\rad\spc{X}.\]
\item 
Usually width is defined using diameter instead of radius, but the result differ at most twice.
Namely if $r$ is an $n$-th radius-width and $d$ --- $n$-th diameter-width of the same dimension, then 
$r\le d\le 2\cdot r$.

\item The definition of width reminds the definition of Lebesgue covering dimension.
In fact one says that a space has \emph{macroscopic dimesion} $\le n$ on the space $R$ if it admits an open cover as in the definiton.
\end{itemize}

\begin{thm}{Exercise}\label{ex:macrodimension}
Suppose $\spc{X}$ be a metric space such that any closed curve $\gamma$ in $\spc{X}$ can be contracted in its $R$-neighborhood.
Show that $\spc{X}$ is has macroscopic dimension at most 1 on scale $100\cdot R$.

What about quasiconverse? That is, suppose a simply connected metric space $\spc{X}$ has macroscopic dimension at most 1 on scale $R$, is it true that any closed curve $\gamma$ in $\spc{X}$ can be contracted in its $100\cdot R$-neighborhood?
\end{thm}


The following proposition provides an equivalent definition;
we will not use it, but it provides a good reason for the name width.

\begin{thm}{Proposition}\label{prop:width=suprad(inv)}
Suppose $\spc{X}$ is a compact metric space.
Then $\width_n\spc{X}<R$ if and only if there is a finite $n$-dimensional somplicial complex $\spc{S}$ and a continuous map $\psi\:\spc{X}\to \spc{N}$
such that $\rad[\psi^{-1}(s)]\z<R$
for any $s\in \spc{N}$.
\end{thm}

\parit{Proof; ``only if'' part.}
Suppose $\width_n\spc{X}<R$.
Consider a covering $\{V_1,\dots,V_k\}$ of $\spc{X}$ guaranteed by the definition of width.
Let $\spc{N}$ be its nerve and $\psi\:\spc{X}\to \spc{N}$ be the map provided by \ref{prop:space->nerve}.

Note that if $x\in \spc{N}$ lies in a symplex with a vertex $v_i$,
then $\psi^{-1}(x)\subset V_i$;
in particulr $\psi^{-1}(x)$ can be covered by a ball of radius $R$ in $\spc{X}$.

\parit{``If'' part.}
Choose $x\in \spc{N}$.
Since the inverse image $\psi^{-1}(x)$ is compact, $\psi$ is continuous, and $\rad[\psi^{-1}(x)]<R$,
here is a neighborhood $U\ni x$ such that the  $\rad[\psi^{-1}(U)]<R$.

It follows that there is a finite cover $\{U_i\}$ of $\spc{N}$ such that $\psi^{-1}(U_i)\subset\spc{X}$ has radius smaller than $R$ for each $i$.
Since $\spc{N}$ has dimension $n$, we can inscribe%
\footnote{Recall that a covering $\{W_i\}$ is inscribed in the covering $\{U_i\}$ if for every $W_i$ is a subset of some $U_j$.} 
in $\{U_i\}$ an finite open cover $\{W_i\}$ with multiplicity at most $n+1$.
It remains to observe that $V_i=\psi(W_i)$ defines a finite open cover of $\spc{X}$ with radius less than $R$ and multiplicity at most $n+1$. 
\qeds

Further we will apply the notion of width to compact Riemannian polyhedrons;
If $n$ is the dimension of a compact Riemannian polyhedron $\spc{P}$, then 
we suppose that
\[\width\spc{P}\df\width_{n-1}\spc{P}.\]

\begin{thm}{Exercise}\label{ex:FillRad<width}
Show that for any closed Riemannian manifold $\spc{M}$ we have
\[\FillRad \spc{M}\le 100\cdot \width\spc{M};\]
try to show that in fact
\[\FillRad \spc{M}\le \width\spc{M}.\]

\end{thm}




\section{Volume profile}

A \emph{Riemannian polyhedron} is defined as a finite connected simplicial complex with a metric tensor on each simplex such that the restriction of the metric on each simplex to a subsymplex coinsides with the metric on the subsmplex.
The dimension of Riemannian polyhedron is defined as the largest dimension it its triangulation.
For Riemannian polhedron one can define length of curves and volume the same way as for Riemannian manifolds.

Let $\spc{P}$ be a Riemnnian polyhedron of dimension $n$.
Let us define volume profile of $\spc{P}$ as a function $\VolPro_{\spc{P}}\:\RR_+\to\RR_+$ defined by 
\[\VolPro_{\spc{P}}(r)\df \sup\set{\vol \oBall(p,r)}{p\in\spc{P}}.\]
Note that $\VolPro_{\spc{P}}$ is a nondecreasing function and $\VolPro_{\spc{P}}(r)\z\to\vol\spc{P}$ as $r\to\infty$.

\begin{thm}{Theorem}\label{thm:width<volpro}
There is a constant $c_n>0$ such that the following holds true:

If $\spc{P}$ is an $n$-dimensional Reimannian polyhedron such that 
\[r> c_n\cdot \sqrt[n]{\VolPro_{\spc{P}}(r)}\] 
for some $r>0$, then 
\[\width\spc{P}\le  r.\]
\end{thm}

Since $\VolPro_{\spc{P}}(r)\le \vol\spc{P}$ for any $r$,
Theorem \ref{thm:width<volpro} implies the following:

\begin{thm}{Theorem}\label{thm:width<vol}
There is a constant $c_n>0$ such that 
\[\width\spc{P}\le c_n\cdot \sqrt[n]{\vol\spc{P}}\] 
for any  $n$-dimensional Reimannian polyhedron $\spc{P}$.
\end{thm}

Together with \ref{ex:FillRad<width}, the last theorem implies \ref{thm:FillRad<vol} which is the goal of this lecture.

\section{Proof}

In the proof of \ref{thm:width<volpro}, we will use the following three technical statements,
the proofs are omitted, but they are not hard. 

\begin{thm}{Smoothing procedure}
Let $\spc{P}$ be a Reimannian polyhedron and $f\:\spc{P}\to \RR$ be a 1-Lipschitz function.
Then for any $\delta>0$ there is a  1-Lipschitz function $\tilde f\:\spc{P}\to \RR$ that is smooth on each simplex of the triangulation and $\delta$-close to $f$.
\end{thm}

\begin{thm}{Sard's theorem}
Let $\spc{P}$ be an $n$-dimensional Reimannian polyhedron and $f\:\spc{P}\to \RR$ be a function that is smooth on each simplex.
Then for almost all values $a$ each component of the inverse image $f^{-1}(a)$ is a equipped with the induced metric is a Reimannian polyhedron.
\end{thm}


\begin{thm}{Coarea inequality}
Let $\spc{P}$ be an $n$-dimensional Reimannian polyhedron and $f\:\spc{P}\to \RR$ be a 1-Lipschitz function that is smooth on each simplex.
Then 
\[\vol_n (f^{-1}[a,b]) \le \int_a^b\vol_{n-1}(f^{-1}\{x\})\cdot dx.\]
\end{thm}

Theorem \ref{thm:width<volpro} will be proved by induction on the dimension of $\spc{P}$;
the following exercise provides a base for the induction.
Note that $\spc{P}$ is connected by definition and if it is 1-dimensional, then 
\[\width\spc{P}=\width_0\spc{P}=\rad\spc{P}.\]

\begin{thm}{Exercise}\label{ex:1D-case}
Suppose $\spc{P}$ be a 1-dimensional Riemannian polhedron.
Suppose $\VolPro_{\spc{P}}(r)<r$ for some $r>0$.
Show that 
\[\width \spc{P}<r.\]

\end{thm}


An $(n-1)$-dimensional subpolyhedron $\spc{Q}\subset\spc{P}$ is called $R$-separating if each
connected component of the complement $\spc{P}\backslash \spc{Q}$ has radius smaller than $R$.

\begin{thm}{Lemma}\label{lem:separating}
Let $\spc{P}$ be an $n$-dimensional Riemannian polyhedron.
Then given $R>0$ and $\eps>0$ there is a $R$-separating subpolyhedron $\spc{Q}\subset\spc{P}$ such that for any $r_0<r_1\le R$ we have
\[\VolPro_{\spc{Q}}(r_0)< \tfrac1{r_1-r_0}\cdot \VolPro_{\spc{P}}(r_1)+\eps.\]

\end{thm}

\parit{Proof.}
Choose small $\delta>0$.
Applying the smoothing procedure, we can exchange each distance function $\distfun_p$ on $\spc{P}$ by $\delta$-close smooth 1-Lipschitz function, which will be denoted by $\widetilde \distfun_p$.

By Sard's theorem, almost all level sets $\tilde S_c(p)$ defined by $\widetilde \distfun_p=c$ are smooth Riemannian polyhedrons of dimension $n-1$.

Since $\delta$ is small, the coarea inequality implies that 
for some  $c\z\in(r_0+\delta, r_1-\delta)$ we have
\begin{align*}
\vol_{n-1}\tilde S_c(p)&\le \tfrac1{r_1-r_0-2\cdot\delta}\cdot\vol_n[\oBall(p,r_1)]<
\\
&<\tfrac1{r_1-r_0}\cdot \VolPro_{\spc{P}}(r_1)+\tfrac\eps2.
\end{align*}

Now suppose $\spc{Q}$ is an $R$-separating subpolyhedron in $\spc{P}$ with almost minimal volume, say its volume is at most $\tfrac\eps2$-far from the greatest lower bound.
Note that cutting from $\spc{Q}$ everything inside $\tilde S_c$ and adding $\tilde S_c$ keeps it to be $R$-separating subpolyhedron.
It follows that
\[\vol_{n-1}[\spc{Q}\cap \oBall(p,r_0)_{\spc{P}}]-\tfrac\eps2\le \vol_{n-1}S_c.\]
Therefore 
\[\vol_{n-1}[\spc{Q}\cap \oBall(p,r_0)_{\spc{P}}]\le\tfrac1{r_1-r_0}\cdot \VolPro_{\spc{P}}(r_1)+\eps\eqlbl{eq:volQ<ProP}\]
Recall that $\spc{Q}$ is equipped with the induced length metric;
therefore $\dist{p}{q}{\spc{Q}}\ge \dist{p}{q}{\spc{P}}$ for any $p,q\in \spc{Q}$;
in particular, 
\[\oBall(p,r_0)_{\spc{Q}}\subset \spc{Q}\cap \oBall(p,r_0)_{\spc{P}}.\]
Hence \ref{eq:volQ<ProP} implies the lemma.
\qeds

\begin{thm}{Lemma}\label{lem:separating-width}
Let $\spc{Q}$ be a $R$-separating subpolyhedron in an $n$-dimensional Riemannian polyhedron $\spc{P}$.
Suppose $\width\spc{Q}\le R$.
Then $\width\spc{P}\le R$
\end{thm}

\parit{Proof.}
Start with an open covering $\{V_1,\dots,V_k\}$ of $\spc{Q}$ of multiplicity $\le n$ with radiuses of the sets in the intrinsic metric $\le R$.

Note that $\{V_1,\dots,V_k\}$ can be converted into an an open covering of
a small neighbourhood of $\spc{Q}$ in $\spc{P}$ without increasing the multiplicity.
This is can be done by setting 
\[V_i'=\bigcup_{x\in V_i}\oBall(x,r_x),\]
where $r_x=\tfrac1{10}\cdot\inf\set{\dist{x}{y}{}}{y\in \spc{Q}\backslash V_i}$.

Finally, add all the components of $\spc{P}\backslash \spc{Q}$ to the covering;
it increases the multiplicity by 1.
The statement follows since $\dim \spc{P}= \dim \spc{Q}\z+1$.
\qeds

\parit{Proof of \ref{thm:width<volpro}.}
We apply induction on the dimension $n=\dim\spc{P}$;
the base case $n=1$ is provided by \ref{ex:1D-case},  for $c_1=1$.

Suppose that the constant $c_{n-1}$ is known, choose sufficiently small $c_n$
\[c_n>2\cdot c_{n-1}.\]

Assume $c_n\cdot \sqrt[n]{\VolPro\spc{P}(r)}< r$.
Fix small $\eps>0$.
By taking $r_0=\tfrac r2$ and $r_1=r$ in \ref{lem:separating}, we have an $r$-separating subpolhedron $\spc{Q}$ in $\spc{P}$ such that 
\begin{align*}
\VolPro_\spc{Q}(r_0) &< \tfrac 1 {r_0}\cdot \VolPro_\spc{P}(r)+\eps<
\\
&<\tfrac 1 {r_0}\cdot \left(\frac{2\cdot r_0}{c_n}\right)^n+\eps=
\\
&=\left(\frac2{c_n}\right)^n\cdot r_0^{n-1}+\eps<
\\
&<\left(\frac1{c_{n-1}}\right)^{n-1}\cdot r_0^{n-1};
\end{align*}
that is, $c_{n-1}\cdot \sqrt[n-1]{\VolPro\spc{Q}(r_0)}< r_0$.
By the induction hypothesis 
\[\width\spc{Q}\le r_0<r.\]

Applying \ref{lem:separating-width}, we get $\width\spc{P}<r$
\qeds


%\chapter{Examples}



\section{On semicontinuity}

Recall that according to \ref{ex:GH-vol}, volume is semicontinuos on the space of Riemannian manifolds with respect to stable Gromov--Hausdorff convergence.
Analogous statement for $n$-dimensional Hausdorff measure on a $n$-dimensional manifolds does not hold.

\begin{thm}{Claim}
 
\end{thm}

First let us show that for any $\alpha>0$, the $\alpha$-dimensional Hausdorff measure is not semicontinuous in the space of all compact metric spaces.

Choose a decreasing sequence $\eps_n\to 0$.
Consider the space $\spc{C}$ of infinite binary sequences with distance between two sequences $\bm{a}=(a_0,a_1,\dots)$ and $\bm{b}=(b_0,b_1,\dots)$ defined by 
\[\dist{\bm{a}}{\bm{b}}{\spc{C}}=\eps_n,\]
where $n$ is the minimal index such that $a_n\ne b_n$.
Note that $\spc{C}$ is homeomorphic to the Cantor set and 
given $\alpha>0$,
the sequence $\eps_n$ can be chosen so that its $\alpha$-dimensional Hausdorff measure is infinite.

Note that $\spc{C}$ is a Hausdorff limit of its subsets $\spc{C}_n$ formed by sequences that constantly zero starting from $n$-th element.
The sets $\spc{C}_n$ is finite in particular its $\alpha$-dimensional Hausdorff measure vanish for $\alpha>0$.
This example shows that for any $\alpha>0$, the $\alpha$-dimensional Hausdorff measure is not semicontinuous in the space of all compact metric spaces.

An analogous example can be produced comapct length spaces.
To do this consider a metric binary rooted tree $\spc{T}$ in which edges connecting level $n-1$ to the level $n$ of length $\eps_{n-1}-\eps_n$.
Note that the completion $\bar{\spc{T}}$ of $\spc{T}$ has a subset (its crown) isometric to $\spc{C}$.
Note further that $\bar{\spc{T}}$ is a Hausdorff limit of its subsets $\spc{T}_n$ --- the subtrees up to level $n$.
Note that $\spc{T}_n$ is can be covered by a finite line segments, in particular it has finite $1$-dimensional Hausdorff measure and therefore vanishing $\alpha$-dimensional Hausdorff for any $\alpha>1$.
Since the limit $\bar{\spc{T}}$ contains $\spc{C}$, we can choose a sequence $\eps_n$ so that $\mu_\alpha\spc{C}$ is arbitrary large (or even infinite).
It shows that for any $\alpha>1$, the $\alpha$-dimensional Hausdorff measure is not semicontinuous in the space of all compact length spaces.

This construction can be modified further to obtain an increasing sequence of metric tensors $g_n$ on a disc $\DD$ such that (1) $\vol(\DD,g_n)<1$ for each $n$, (2) the induced metrics $\dist{*}{*}{g_n}$ converge to a metric $\rho$ on $\DD$, and given any Cantor space $\spc{C}$ as described above (3) there is a bilipschitz map $\spc{C}\to(\DD,\rho)$.
Note that the last condition implies that $\mu_2(\DD,rho)$ can be made arbitrary large, or infinite.
Therefore for any $\alpha\ge 2$, the $\alpha$-dimensional Hausdorff measure is not semicontinuous in the space of all compact length spaces homeomorphic to a manifold and equipped with stable convergence.

Now we want to extend nonsemicontinuity even further.
Note that the tree $\bar{\spc{T}}$ admits a length-preserving embedding to the Euclidean space; we may assume that all 



\section{Sub-Riemannian metrics}

Choose a metric space $\spc{X}$.
Note that the function $\alpha\mapsto \mu_\alpha(A)_\spc{X}$ is nondecreasing;
moreover there is a critical value $\alpha_0\in[0,\infty]$ such that $\mu_\alpha(A)_\spc{X}=0$ if $\alpha<\alpha_0$ and $\mu_\alpha(A)_\spc{X}=\infty$ if $\alpha>\alpha_0$.
This value is called \index{Hausdorff dimension}\emph{Hausdorff dimension} of $\spc{X}$, or briefly $\alpha_0=\dim_H\spc{X}$.

The following statement is classical, a proof can be found in .

\begin{thm}{Theorem}
The Hausdorff dimension of any metric space can not be smaller than its Lebesgue covering dimension.
In particular, if a metric space $\spc{X}$ is homeomorphic to an $n$-dimensional manifold, then $\dim_H\spc{X}\ge n$.
 
\end{thm}

Note that the construction described in the previous section can be used to produce a metric on manifold of dimension $n\ge 2$ with arbitrary Hausdorff dimension $\alpha\ge n$.

In this section we will discuss another interesting source of such examples.



%











\begin{thm}{Lemma}
$\spc{M}$ is complete.
\end{thm}

\parit{Proof.}
Let $(\spc{X}_n)$ be a Cauchy sequence in $\spc{M}$.
Passing to a subsequence if necessary, 
we can assume that $|\spc{X}_n-\spc{X}_{n+1}|_{\spc{M}}<\tfrac1{2^n}$ for each $n$.
In particular, for each $n$ one can equip $\spc{W}_n=\spc{X}_n \sqcup \spc{X}_{n+1}$ with a metric such that
inclusions $\spc{X}_n\hookrightarrow \spc{W}_n$ and $\spc{X}_{n+1}\hookrightarrow \spc{W}_n$ are distance preserving
and $$|\spc{X}_n-\spc{X}_{n+1}|_{\mathcal{H}(\spc{W}_n)}\z<\tfrac1{2^n}$$
for each $n$.

Set $\spc{W}$ to be the disjoint union of all $\spc{X}_n$.
Let us equip $\spc{W}$ with a metric defined the following way:
\begin{itemize}
\item for any fixed $n$ and any two points $x_n,x_n'\in \spc{X}_n$ set
$$|x_n-x_n'|_{\spc{W}}=|x_n-x_n'|_{\spc{X}_n}$$
\item for any positive integers $m>n$ and any two points $x_n\in \spc{X}_n$ and $x_m\in \spc{X}_m$ set
$$|x_n-x_m|_{\spc{W}}=\inf\left\{\sum_{i=n}^{m-1}|x_i-x_{i+1}|_{\spc{W}_i}\right\},$$
where the infimum is taken for all sequences $x_i\in \spc{X}_i$.
\end{itemize}

\begin{thm}{Exercise}
Check that this indeed defines a metric on $\spc{W}$.
\end{thm}

Let $\bar{\spc{W}}$ be the completion of $\spc{W}$.
Note that $|\spc{X}_m-\spc{X}_n|<\tfrac1{2^{n-1}}$ if $m>n$.
Therefore the union of $\spc{X}_1\cup \spc{X}_2\cup\dots\cup \spc{X}_n$ forms a $\tfrac1{2^{n-1}}$-net in $\bar{\spc{W}}$.
Since each $\spc{X}_i$ is compact, we get that $\bar{\spc{W}}$ admits a compact $\eps$-net for any $\eps>0$.
According to Problem~\ref{pr:compact-net}, $\bar{\spc{W}}$ is compact.

According to Blaschke's compactness theorem (\ref{thm:compact+Hausdorff}),
we can pass to a subsequence of $(\spc{X}_n)$ which converge in $\mathcal{H}(\bar{\spc{W}})$ and therefore in $\spc{M}$.
\qeds

\parit{Proof of \ref{thm:gromov-compactness}; ``only if'' part.}
If there is no sequence $\eps_n\to0$ as described in the problem, then for a fixed fixed $\delta>0$
there is a sequence of spaces $\spc{X}_n\in\spc{Q}$ such that $$\pack_\delta \spc{X}_n\to\infty
\quad\text{as}\quad
n\to\infty.$$
Since $\spc{Q}$ is compact, 
this sequence has a partial limit say $\spc{X}_\infty\in\spc{Q}$.
It is easy to see that $\pack_{\delta/10} \spc{X}_\infty=\infty$;
the later contradicts Theorem~\ref{thm:finite_pack=compact}.

\parit{``If'' part.}
Let us fix the sequence $\eps_n\to 0$ as in the problem and consider the set $\hat{\spc{Q}}$ of all (isometry classes of all) metric spaces $\spc{X}$ such that
$\pack_{\eps_n} \spc{X}\le n$ for any $n$. 
According to Exercise~\ref{ex:pack-GH}, $\hat{\spc{Q}}$ is closed in $\spc{M}$.
Clearly $\spc{Q}\subset\hat{\spc{Q}}$.
Therefore it is sufficient to prove that $\hat{\spc{Q}}$ is compact.

Note that $\diam \spc{X}\le \eps_1$ for any $\spc{X}\in \hat{\spc{Q}}$.
Given positive integer $n$ consider set of all metric spaces $\spc{W}_n$
with number of points at most $n$ and diameter $\le \eps_1$.
Note that $\spc{W}_n$ is compact for each $n$.
Further a maximal $\eps_n$-packing of any $\spc{X}\in\hat{\spc{Q}}$ forms a subspace from $\spc{W}_n$.
Therefore $\spc{W}_n\cap\hat{\spc{Q}}$ is a comapct $\eps_n$-net in  $\hat{\spc{Q}}$.
Problem~\ref{pr:compact-net} implies that $\hat{\spc{Q}}$ is compact.
\qeds



\section{Comments} 

Given two metric spaces $\spc{X}$ and $\spc{Y}$, we will write $\spc{X}\preccurlyeq \spc{Y}$ if there is a noncontracting map $f\:\spc{X}\to \spc{Y}$;
that is, if 
$$ |x-x'|_{\spc{X}}\le|f(x)-f(x')|_{\spc{Y}}$$
for any $x,x'\in \spc{X}$.

Further, given $\eps>0$, we will write $\spc{X}\preccurlyeq \spc{Y}+\eps$
if there is a map $f\:\spc{X}\to \spc{Y}$ such that 
$$|x-x'|_{\spc{X}}\le|f(x)-f(x')|_{\spc{Y}}+\eps$$
for any $x,x'\in \spc{X}$.

Define 
$$\dist[\star]{\spc{X}}{\spc{Y}}{\spc{M}}=\inf\set{\eps}{\spc{X}\preccurlyeq \spc{Y}+\eps
\quad\text{and}\quad
\spc{Y}\preccurlyeq \spc{X}+\eps}$$
It turns out that $\dist[\star]{*}{*}{\spc{M}}$ is a different metric on the set of isometry classes of compact metric spaces; that is, in general $\dist[\star]{\spc{X}}{\spc{Y}}{\spc{M}}\not=|\spc{X}-\spc{Y}|_{\spc{M}}$. 
However, these two metrics define the same topology on $\spc{M}$.
More precicely:

\begin{thm}{Proposition}\label{GH-po}
For any sequence of compact metric spaces $(\spc{X}_n)$ and a compact metric space $\spc{X}_\infty$,
we have
$$|\spc{X}_n-\spc{X}_\infty|_{\spc{M}}\to 0
\quad\iff\quad
\dist[\star]{\spc{X}_n}{\spc{X}_\infty}{\spc{M}}\to 0$$ 
as $n\to\infty$.
\end{thm}

We will not give a proof of this proposition. 
Likely, we will not use it further in the lectures, 
but it might help you to build intuition for Gromov--Hausdorff convergence.
If you want to prove it yourself look in the proof of Theorem~\ref{thm:GH-is-a-metric} 
and try to modify it using ideas from the proof of Problem~\ref{pr:non-contracting=>isometry}.

The Gromov--Hausdorff distance can be defined for arbitrary pair of metric space.
Therefore it is natural to ask why we only consider compact metric spaces.
First note the Gromov--Hausdorff distance from any metric space $\spc{X}$ 
to its completion $\bar {\spc{X}}$ is zero.
Therefore if you want to end up in a metric space, it is better to consider only complete metric spaces.

Further, the distance between one-point-space and a metric spce with infinite diameter is infinite.
Therefore one has to either consider only bounded metric spaces (that is, the spaces with finite diameter)
or relux the definition of metric space which allow metric to take infinite value.

Finally, the class of isometry classes of all bounded complete metric spaces forms a class, but not a set.
Hence again we have two choices: either relux the definition of metric space so its points will form a class, or restrict further the class of spaces for which the isometry classes will form a set.

\begin{thm}{Exercise}
Prove that isometry classes of compact metric spaces form a set. 
\end{thm}

\begin{thm}{Exercise}\label{pr:GH1}
Let $\spc{X}=\{x,y,z\}$ be a three point subset of Euclidean plane with distances
$$|x-y|=|y-z|=|z-x|=1.$$
\begin{enumerate}[(i)]
\item Find the minimal Hausdorff distance from $\spc{X}$ to a one-point subset of the plane.
\item Find the Gromov--Hausdorff distance from $\spc{X}$ to the one-point metric space. 
\end{enumerate}
\end{thm}

\begin{thm}{Exercise}\label{pr:GH2}
Let $\spc{X}$ and $\spc{Y}$ be a compact metric spaces which have isometric $\eps$-nets.
Show that 
$$|\spc{X}-\spc{Y}|_{\spc{M}}\le 2\cdot\eps.$$
Is it always true that 
$$|\spc{X}-\spc{Y}|_{\spc{M}}\le \eps?$$
\end{thm}




\begin{thm}{Exercise}\label{pr:GH3}
Define the \emph{radius of a metric space}\index{radius of a metric space} $\spc{X}$ as 
$$\rad \spc{X}=\inf_x\left\{\sup_y\{|x-y|_{\spc{X}}\}\right\}.$$
Equivalently, 
$$\rad \spc{X}=\inf\set{R>0}{\text{there is}\ x\in \spc{X}\  \text{such that}\ B_R(x)\supset \spc{X}}.$$
 
\begin{enumerate}[(i)]
\item Show that for any compact metric space $\spc{X}$ we have
$$\tfrac12\cdot\diam \spc{X}\le \rad \spc{X}\le \diam \spc{X}.$$
\item Show that for any compact metric spaces $\spc{X},\spc{Y}$ we have
$$|\rad \spc{X}-\rad \spc{Y}|\le 2\cdot |\spc{X}-\spc{Y}|_{\spc{M}}.$$
\end{enumerate}
\end{thm}

\begin{thm}{Exercise}\label{pr:F-X}
Let $\spc{X}$ be a metric space.
If two compact sets $A, B$ in $\spc{X}$ are isometric,
we will write $A\iso B$. 
Set
$$d(A,B)=\inf \set{|A'-B'|_{\mathcal{H}(\spc{X})}}{A'\iso A \ \text{and}\ B'\iso B}.$$
Note that if $\spc{X}=\ell^\infty$ then according to Proposition~\ref{prop:GH-with-fixed-Z}, 
$d$ is a metric on $\mathcal{H}(\spc{X})/\iso$ (that is, on the ``$\iso$''-equivalecne classes of $\mathcal{H}(\spc{X})$).

Show that it does not hold for arbitrary metric space $\spc{X}$.
Understand the reason why it holds for $\spc{X}=\ell^\infty$.
\end{thm}


\begin{thm}{Exercise}\label{pr:GH-variation}
Consider the pairs $(\spc{X},A)$, where $\spc{X}$ is a compact metric space and $A$ is a closed subset in $\spc{X}$.
Two such pairs, say $(\spc{X},A)$ and $(\spc{X}',A')$ will be called equivalent (briefly $(\spc{X},A)\sim(\spc{X}',A')$)
if there is an isometry $\iota\:\spc{X}\to \spc{X}'$ such that $\iota(A)=A'$.

Modify the definition of Gromov--Hausdorff metric to construct a natural metric on the set of $\sim$-equivalence classes of the pairs $(\spc{X},A)$.
\end{thm}

Here we introduce so called Gromov--Hausdorff convergence for metric spaces.
This convergence was introduced by Gromov around 1980, published in \cite{gromov-1981}.
Very soon this notion began to be used in all branches of geometry.
In fact today I have difficulty to understand 
how one could do geometry without this type of convergence.%
(Some types of convergences of metric spaces was considered before Gromov,
but they had lack of generality;
each type of convergence was desined to solve one particular problem.)


\begin{thm}{Exercise}\label{ex:euclid-isom}
\begin{subthm}{}
Let $\spc{X},\spc{Y}$ be two compact sets in the Euclidean plane $\RR^2$.
Show that $\spc{X}$ is isometric to $\spc{Y}$ if and only if there is an motrio $\iota\:\RR^2\to \RR^2$
that sends $\spc{X}$ to $\spc{Y}$.
\end{subthm}

\begin{subthm}{}
Find two isometric subsets $\spc{X},\spc{Y}$ of $\ell^\infty$
such that there is no isometry $\iota\:\ell^\infty\to \ell^\infty$ 
that sends $\spc{X}$ to $\spc{Y}$.
\end{subthm}
\end{thm}

\appendix
\chapter{Semisolutions}
\refstepcounter{chapter}
\setcounter{eqtn}{0}

\parbf{\ref{ex:non-differentiable}.}
Choose a function $r\mapsto \alpha(r)$ such that $\alpha'(r)\cdot r\to 0$ and $\alpha(r)\to\infty$ as $r\to 0$.
Consider the reparametrization of the Euclidean plane given by $\iota\:(r,\theta)\mapsto (r,\theta+\alpha(r))$ in the polar coordinates.
Observe that $\iota$ is not differentiable at the origin, but the metric tensor $g$ induced by $\iota$  is continuous.

\medskip

For more on the subject read the paper of Eugenio Calabi and Philip Hartman \cite{calabi-hartman}. 

\parbf{\ref{ex:volume-preserving+short}};
\ref{SHORT.ex:volume-preserving+short:injective}.
Suppose $p=f(x)=f(y)$ and the points $x,y\in \spc{M}$ are distinct.
Since $f$ is short, we get for any $r>0$ the ball $\oBall(p,r)_{\spc{N}}$ contains the images of $\oBall(x,r)_{\spc{M}}$ and $\oBall(y,r)_{\spc{M}}$.
Since $f$ is volume-preserving, we get
\[
\vol\oBall(x,r)_{\spc{M}}
+
\vol\oBall(y,r)_{\spc{M}}
\le
\vol\oBall(p,r)_{\spc{N}}.
\eqlbl{vol+vol<vol}\]

By \ref{obs:lip-chart}, for any $\eps>0$ and all sufficiently small $r>0$ the volumes of the balls  $\oBall(x,r)_{\spc{M}}$, $\oBall(y,r)_{\spc{M}}$ and $\oBall(p,r)_{\spc{N}}$, lie in the range $\omega_n\cdot e^{\mp2\cdot n\cdot\eps}\cdot r^n$, where $\omega_n$ denotes the volume of the unit ball in the $n$-dimensional Euclidean space.
The latter contradicts \ref{vol+vol<vol} for appropriate choice of $\eps$ and $r$.

\parit{\ref{SHORT.ex:volume-preserving+short:bi}.}
Denote by $\sigma(r,a)$ the volume of union of two $r$-balls in the $n$-dimensional Euclidean space such that the distance between their centers is $a$.
Observe that the function $(a,r)\mapsto \sigma(r,a)$ is continuous and increasing in $a$ and $r$ for $a\le r$.
Further, note that
\[\sigma(\lambda\cdot r,\lambda\cdot a)=\lambda^n\cdot \sigma(r,a)\]
for any $\lambda>0$.

Choose a point $z\in \spc{M}$ and small $\eps>0$.
By \ref{obs:lip-chart} there is $R>0$ such that $\oBall(z,10\cdot R)$ admits a $e^{\mp\eps}$-bilipschitz map to the $n$-dimensional Euclidean space.

Choose $x,y\in \oBall(z, R)$.
The argument used in part \ref{SHORT.ex:volume-preserving+short:injective} implies that 
\[e^{-n\cdot\eps}\cdot \sigma(e^{-\eps}\cdot r, e^{-\eps}\cdot \dist{x}{y}{\spc{M}})
\le 
e^{n\cdot\eps}\cdot \sigma(e^{\eps}\cdot r, e^{\eps}\cdot \dist{f(x)}{f(y)}{\spc{N}}).
\eqlbl{eq:v(r,a)}\]
This inequality implies a lower bound on $\dist{f(x)}{f(y)}{\spc{N}}$ in terms of $\dist{x}{y}{\spc{M}}$.

Use the listed properties of the function $(a,r)\mapsto \sigma(r,a)$ to show that for any $c<1$ there is $\eps>0$ such that \ref{eq:v(r,a)} implies that $b>c\cdot a$ for all sufficiently small $a$.

Finally, since $\spc{M}$ and  $\spc{N}$ are length-metric spaces, part~\ref{SHORT.ex:volume-preserving+short:bi} implies that $f$ is locally distance preserving.
(An inclusion map from a nonconvex open subset to the plane gives an example of volume preserving short map that is not distance preserving.)


\medskip

A more general result is discussed by Paul Creutz and Elefterios Soultanis \cite{creutz-soultanis}.


\parbf{\ref{ex:compact-interior}.} Denote by $\spc{M}$ and $\spc{M}^\circ$ the space of $(M,g)$ and $(M^\circ,g)$;
further denote by $\bar{\spc{M}}^\circ$ the completion of $\spc{M}^\circ$.
Observe that the inclusion $M^\circ\hookrightarrow M$ induces a short onto map $\iota\:\bar{\spc{M}}^\circ\z\to\spc{M}$.

Recall that $M$ is bounded by hypersurface that is locally a graph.
Use it to show that any sufficiently short curve $\gamma$ in $(M,g)$ can be approximated by a curve in $\spc{M}^\circ$ with $g$-length arbitrary close to $\length_g\gamma$.
Conclude that $\iota$ is an isometry.


\parbf{\ref{ex:besikovitch=}.}
From the proof of Besicovitch inequality, one can see that the restriction of $\bm{f}$ to the interior of $\spc{M}$ is
(1) volume-preserving, and 
(2) its differential $d_p\bm{f}\:\T_p\to \T_{\bm{f}(p)}$ is an isometry for almost all $p$.

Since $\bm{f}$ is Lipschitz, (2) can be used to show that $\bm{f}$ is short.
It remains to apply \ref{ex:volume-preserving+short} and \ref{ex:compact-interior}.

\parbf{\ref{ex:hexagon}.}
Consider the hexagon with flat metric and curved sides shown on the diagram.
Observe that its area can be made arbitrarily small while keeping the distances from the opposite sides at least 1.

\begin{Figure}
\begin{minipage}{.48\textwidth}
\centering
\includegraphics{mppics/pic-27}
\end{minipage}\hfill
\begin{minipage}{.48\textwidth}
\centering
\includegraphics{mppics/pic-23}
\end{minipage}
\vskip-4mm
\end{Figure}

\parbf{\ref{ex:cylinder};} \ref{SHORT.ex:cylinder:besicovitch}.
Let $\alpha$ be a shortest curve that runs between the boundary components of the cylinder.
Cut the cylinder along $\alpha$.
We get a square with Riemannian metric on it $(\square,g)$.

Two opposite sides of $\square$ correspond to the boundary components of the cylinder.
The other pair corresponds to the sides of the cut.
By assumption, the $g$-distance between the first pair of sides is at least 1.

Consider a shortest curve $\beta$ that connects this pair of sides;
let us keep the same notation for the projection of $\beta$ in the cylinder.

Note that a cyclic concatenation $\gamma$ of $\beta$ with an arc of $\alpha$ is homotopic to a boundary circle.
Therefore $\length_g\gamma\ge1$.
Since $\alpha$ is a shortest path, its arc cannot be longer than any curve connecting its ends; therefore 
\[\length_g\beta\ge \tfrac 12\cdot\length_g \gamma\ge \tfrac 12.\]
That is, the other pair of sides of $\square$ lies on $g$-distance at least $\tfrac12$ from each other.
By \ref{thm:besikovitch+}, $\area(\square,g)\ge \tfrac12$, hence the result.

\parit{\ref{SHORT.ex:cylinder:coarea}.}
Note that any curve in the cylinder that is bordant to a boundary component has length at least $1$.
Therefore if $0\le t\le  1$, then the level sets 
\[L_t=\set{x\in \mathbb{S}^1\times[0,1]}{\distfun_{\mathbb{S}^1\times\{0\}}(x)_{g}=t}\] have length at least $1$.
Applying the coarea inequality, we get that
\[\area(\mathbb{S}^1\times[0,1],g)\ge 1.\]

\parbf{\ref{ex:gadograph}}; \ref{SHORT.ex:gadograph-besikovitch}.
Argue the same way as in \ref{thm:besikovitch}, but observe in addition that $\vol \Sigma=\vol \bm{f}(\Sigma)=0$ and use it time to time.

\parit{\ref{SHORT.ex:gadograph-gadograph}.}
Without loss of generality, we may assume that $V$ lies in a unit cube~$\square$.
Consider a noncontinuous metric tensor $\bar g$ on $\square$ that coincides with $g$ inside $V$ and with the canonical flat metric tensor outside of~$V$.

Observe that the $\bar g$-distances between opposite faces of $\square$ are at least 1.
Indeed this is true for the Euclidean metric and the assumption $\dist{p}{q}{g}\ge\dist{p}{q}{\EE^d}$  guarantees that one cannot make a shortcut in~$V$.
Therefore, the $\bar g$-distances between every pair of opposite faces is at least as large as 1 which is the Euclidean distance.

Applying part \ref{SHORT.ex:gadograph-besikovitch}, we get that $\vol(\square,\bar g)\ge \vol\square$.
Whence the statement follows.


\parbf{\ref{ex:involution-of-sphere}.}
Let $x\in \mathbb{S}^2$ be a point that minimize the distance $|x-x'|_g$.
Consider a shortest path $\gamma$ from $x$ to $x'$.
We can assume that 
\[|x-x'|_g=\length \gamma=1.\]

Let $\gamma'$ be the antipodal arc to $\gamma$.
Note that $\gamma'$ intersects $\gamma$ only at the common endpoints $x$ and $x'$.
Indeed, if $p'=q$ for some $p,q\in\gamma$, then $|p-q|\ge 1$.
Since $\length \gamma=1$, the points $p$ and $q$ must be the ends of $\gamma$.

It follows that $\gamma$ together with $\gamma'$ forms a closed simple curve in $\mathbb{S}^2$;
it divides the sphere into two disks $D$ and $D'$.

Let us divide $\gamma$ into two equal arcs $\gamma_1$ and $\gamma_2$; each of length $\tfrac12$.
Suppose that $p,q\in\gamma_1$, then 
\begin{align*}
|p-q'|_g&\ge |q-q'|_g-|p-q|_g\ge
\\
&\ge 1-\tfrac12=\tfrac12.
\end{align*}
That is, the minimal distance from $\gamma_1$ to $\gamma_1'$ is at least~$\tfrac12$.
The same way we get that the minimal distance from $\gamma_2$ to $\gamma_2'$ is at least~$\tfrac12$.
By Besicovitch inequality, we get that 
\[\area(D,g)\ge \tfrac14\quad\text{and}\quad \area(D',g)\ge \tfrac14.\]
Therefore 
\[\area(\mathbb{S}^2,g)\ge\tfrac12.\]

\parit{A better estimate.}
Let us indicate how to improve the obtained bound to
\[\area(\mathbb{S}^2,g)\ge1.\]

Suppose $x$, $x'$, $\gamma$ and $\gamma'$ are as above.
Consider the function
\[f(z)=\min_t \{\,|\gamma'(t)-z|_g+t\,\}.\]
Observe that $f$ is 1-Lipschitz.

Show that two points $\gamma'(c)$ and $\gamma(1-c)$ lie on one connected component of the level set $L_c=\set{z\in\mathbb{S}^2}{f(z)=c}$;
in particular 
\[\length L_c\ge 2\cdot|\gamma'(c)-\gamma(1-c)|_g.\]
By the triangle inequality, we have that
\begin{align*}
|\gamma'(c)-\gamma(1-c)|_g&\ge 1-|\gamma(c)-\gamma(1-c)|_g=
\\
&=1-|1-2\cdot c|.
\end{align*}

The coarea inequality (\ref{cor:coarea})
\[\area(\mathbb{S}^2,g)\ge \int\limits_0^1\length L_c\cdot dc\]
finishes the proof.


The bound $\tfrac12$ was proved by Marcel Berger \cite{berger}. 
Christopher Croke conjectured that the optimal bound is $\tfrac4\pi$ and the round sphere is the only space that achieves this \cite[Conjecture 0.3 in][]{croke} --- if you solved the last part of the problem, then publish the result.

\begin{wrapfigure}{r}{20 mm}
\vskip-0mm
\centering
\includegraphics{mppics/pic-1305}
\end{wrapfigure}

\parbf{\ref{ex:involution-of-3sphere}.}
Given $\eps>0$, construct a disk $\Delta$ in the plane with 
\begin{align*}
\length\partial \Delta&<10
\intertext{and}
\area \Delta&<\eps
\end{align*}
that admits an continuous involution $\iota$ such that 
\[|\iota(x)-x|\ge 1\]
for any $x\in\partial \Delta$.

An example of $\Delta$ can be guessed from the picture;
the involution $\iota$ makes a length preserving half turn of its boundary $\partial \Delta$.


Take the product $\Delta\times \Delta\subset \EE^4$;
it is homeomorphic to the 4-ball.
Note that 
$$\vol_3[\partial(\Delta\times \Delta)]=2\cdot\area \Delta\cdot\length \partial \Delta<20\cdot\eps.$$
The boundary $\partial(\Delta\times \Delta)$ is homeomorphic to $\mathbb{S}^3$
and the restriction of the involution $(x,y)\z\mapsto (\iota(x),\iota(y))$ has the needed property.

It remains to smooth $\partial(\Delta\times \Delta)$ a  bit.

\parit{Remark.} This example is given by Christopher Croke \cite{croke}.
Note that according to \ref{thm:sys+}, 
the involution $\iota$ cannot be made isometric.

\parbf{\ref{ex:GH-vol}.}
Note that if $(M,g_\infty)$ is $e^{\mp\eps}$-bilipschitz to a cube, then applying Besicovitch inequality, we get that 
\[\liminf_{n\to\infty} \vol (M,g_n)\ge e^{-n\cdot \eps}\cdot\vol (M,g_\infty).\]

By the Vitali covering theorem, given $\eps>0$, we can cover the whole volume of $(M,g_\infty)$ by $e^{\pm\eps}$-bilipschitz cubes.
Applying the above observation and summing up the results, we get that 
\[\liminf_{n\to\infty} \vol (M,g_n)\ge e^{-n\cdot \eps}\cdot\vol (M,g_\infty).\]
The statement follows since $\eps$ is an arbitrary positive number.

To solve the second part of the exercise, start with $g_\infty$ and construct $g_n$ by  adding many tiny bubbles.
The volume can be increased arbitrarily with an arbitrarily small change of metric.

\parit{Remark.}
A more general result was obtained by Sergei Ivanov~\cite{ivanov-1997}.
Note that the statement does not hold true for Gromov--Hausdorff convergence.
In fact any compact metric space $\spc{X}$ can be GH-approximated by a Riemannian surface with an arbitrarily small area.
To show the latter statement, approximate $\spc{X}$ by a finite graph $\Gamma$, embed $\Gamma$ isometrically to the Euclidean space, and pass to the surface of its neighborhood.

\parbf{\ref{ex:sysT2}.}
Set $s=\sys(\TT^2,g)$.

Cut $\TT^2$ along a shortest closed noncontractible curve $\gamma$.
We get a cylinder $(\mathbb{S}^1,g)$ with a Riemannian metric on it.

Applying the argument in \ref{ex:cylinder:besicovitch}, we get that the $g$-distance between the boundary components is at least $\tfrac s2$.
Then \ref{ex:cylinder:besicovitch} implies that the area of torus is at least $\tfrac{s^2}2$.

\parit{Remark.}
The optimal bound is $\tfrac{\sqrt{3}}{2}\cdot s^2$; see  \ref{sec:besicovitch-remarks}.



\parbf{\ref{ex:sysRP2}.}
Set $s\z=\sys (\RP^2,g)$.
Cut $(\RP^2,g)$ along a shortest noncontractible curve $\gamma$.
We obtain $(\DD^2,g)$ --- a disc with metric tensor which we still denote by $g$.

Divide $\gamma$ into two equal arcs $\alpha$ and $\beta$.
Denote by $A$ and $A'$ the two connected components of the inverse image of $\alpha$.
Similarly denote by $B$ and $B'$ the two connected components of the inverse image of $\beta$.

\begin{Figure}
\vskip-0mm
\centering
\includegraphics{mppics/pic-25}
\end{Figure}

Let $\gamma_1$ be a path from $A$ to $A'$;
map it to $\RP^2$ and keep the same notation for it.
Note that $\gamma_1$ together with a subarc of $\alpha$ forms a closed noncontractible curve in $\RP^2$.
Since $\length\alpha=\tfrac s2$, we have that $\length\gamma_1\ge \tfrac s2$.
It follows that the distance between $A$ and $A'$ in $(\DD^2,g)$ is at least $\tfrac s2$.
The same way we show that the distance between $B$ and $B'$ in $(\DD^2,g)$ is at least $\tfrac s2$.

Note that $(\DD^2,g)$ can be parameterized by a square with sides $A$, $B$, $A'$ and $B'$ and apply \ref{thm:besikovitch} to show that 
\[\area(\RP^2,g)=\area(\DD^2,g)\ge \tfrac14\cdot s^2.\]

\parit{Remark.}
The optimal bound is $\tfrac2 \pi\cdot s^2$; see  \ref{sec:besicovitch-remarks}.
In fact any Riemannian metric on the disc with the boundary globally isometric to a unit circle with angle metric has the area at least as large as the unit hemisphere.
It is expected that the same inequality holds for any compact surface with connected boundary (not necessarily a disc);
this is the so-called \index{filling area conjecture}\emph{filling area conjecture} \cite[it is mentioned Mikhael Gromov in 5.5.$\mathrm{B}'(\mathrm{e}')$ of][]{gromov-1983}.

\parbf{\ref{ex:sysSg}.} Cut the surface along a shortest noncontractible curve $\gamma$. 
We might get a surface with one or two components of the boundary.
In these two cases repeat the arguments in \ref{ex:sysRP2} or \ref{ex:sysT2} using \ref{thm:besikovitch+} instead of \ref{thm:besikovitch}.


\parbf{\ref{ex:sysS2xS1}.} Consider the product of a small 2-sphere with the unit circle.

\parbf{\ref{ex:besikovitch++}.}
Apply the same construction as in the original Besicovitch inequality, assuming that the target rectangle
$[0,d_1]\times\dots\times [0,d_n]$ equipped with the metric induced by the $\ell^\infty$ norm;
apply \ref{prop:bilip-measure} where it is appropriate.

\parbf{\ref{ex:2top-discs}.} Suppose that $\Delta_1\ne\Delta_2$.
Consider the map $f\:\mathbb{S}^n\to \spc{X}$ such that the restriction to north and south hemispheres describe $\Delta_1$ and $\Delta_2$ respectively.
Show that if $\Delta_1\ne\Delta_2$, then $\mathbb{S}^n$ can be parameterized by the boundary of the unit cube $\square$ in such a way that for any pair $A$, $A'$ of opposite faces their images $f(A)$, $f(A')$ do not overlap.

Since $\spc{X}$ is contractible, the map $f$ can be extended to a map of the whole cube.
By \ref{ex:besikovitch++} 
\[\haus_{n+1}[f(\square)]>0,\]
a contradiction.

\refstepcounter{chapter}
\setcounter{eqtn}{0}

\parbf{\ref{ex:macrodimension}.}
The following claim resembles Besicovitch inequality;
it is key to the proof:
\begin{itemize}
 \item[$({*})$] Let $a$ be a positive real number.
 Assume that a closed curve $\gamma$ in a metric space $\spc{X}$ can be subdivided into 4 arcs $\alpha$, $\beta$, $\alpha'$, and $\beta'$ in such a way that 
 \begin{itemize}
 \item $|x-x'|>a$ for any $x\in\alpha$ and $x'\in \alpha'$
 and
 \item $|y-y'|>a$ for any $y\in\beta$ and $y'\in \beta'$.
 \end{itemize}
 Then $\gamma$ is not contractible in its $\tfrac a2$-neighborhood.
\end{itemize}

To prove $({*})$, consider two functions defined on $\spc{X}$ as follows:
\begin{align*}
w_1(x)&=\min \{\,a,\distfun_{\alpha}(x)\,\}
\\
w_2(x)&=\min \{\,a,\distfun_{\beta}(x)\,\}
\end{align*}
and the map $\bm{w}\:\spc{X}\to [0,a]\times[0,a]$, defined by
\[\bm{w}\:x\mapsto(w_1(x),w_2(x)).\]

Note that 
\begin{align*}
\bm{w}(\alpha)&=0\times [0,a],
&
\bm{w}(\beta)&=[0,a]\times 0,
\\
\bm{w}(\alpha')&=a\times [0,a],
&
\bm{w}(\beta')&=[0,a]\times a.
\end{align*} 
Therefore, the composition $\bm{w}\circ\gamma$ is a degree 1 map 
\[\mathbb{S}^1\to \partial([0,a]\times[0,a]).\] 
It follows that if $h\:\DD\to \spc{X}$ shrinks $\gamma$, then there is a point $z\in\DD$ such that 
$\bm{w}\circ h(z)=(\tfrac a2,\tfrac a2)$.
Therefore, $h(z)$ lies at distance at least $\tfrac a2$ from $\alpha$, $\beta$, $\alpha'$, $\beta'$
and therefore from $\gamma$.
It proves the claim.

\medskip

Coming back to the problem, let $\{W_i\}$ be an open covering of the real line with multiplicity $2$ and $\rad W_i<R$ for each $i$;
for example take the covering by the intervals $((i-\tfrac23)\cdot R,(i+\tfrac23)\cdot R)$.

Choose a point $p\in \spc{X}$.
Denote by $\{V_j\}$ the connected components of $\distfun_p^{-1}(W_i)$ for all $i$.
Note that $\{V_j\}$ is an open finite cover of $\spc{X}$ with multiplicity at most 2.
It remains to show that $\rad V_j<100\cdot R$ for each $j$.

\begin{Figure}
\vskip-0mm
\centering
\includegraphics{mppics/pic-1310}
\end{Figure}

Arguing by contradiction assume there is a pair of points  $x,y\in V_i$ 
such that $|x\z-y|_{\spc{X}}\z\ge 100\cdot R$.
Connect $x$ to $y$ with a curve $\tau$ in $V_j$.
Consider the closed curve $\sigma$ formed by $\tau$ and two shortest paths $[px]$, $[py]$.


Note that $|p-x|>40$.
Therefore, there is a point $m$ on $[px]$ such that $|m-x|=20$.

By the triangle inequality, the subdivision of $\sigma$ into the arcs $[pm]$, $[mx]$, $\tau$ and $[yp]$ satisfy the conditions of the claim $({*})$ for $a=10\cdot R$,
hence the statement.

\begin{Figure}
\vskip0mm
\centering
\includegraphics{mppics/pic-1315}
\end{Figure}

\parit{The quasiconverse} does not hold.
As an example take a surface that looks like a long cylinder with closed ends;
it is a smooth surface diffeomorphic to a sphere.
Assuming the cylinder is thin, it has macroscopic dimension 1 at a given scale.
However, a circle formed by a section of cylinder around its midpoint by a plane parallel to the base is a circle that cannot be contracted in its small neighborhood.

\parit{Source:} \cite[Appendix $1(\text{E}_{2})$]{gromov-1983}.

\parbf{\ref{ex:width=suprad(inv)}}; \textit{only-if part.}
Suppose $\width_n\spc{X}<R$.
Consider a covering $\{V_1,\dots,V_k\}$ of $\spc{X}$ guaranteed by the definition of width.
Let $\spc{N}$ be its nerve and $\bm{\psi}\:\spc{X}\to \spc{N}$ be the map provided by \ref{prop:space->nerve}.

Since the multiplicity of the covering is at most $n+1$, we have $\dim \spc{N}\le n$.

Note that if $x\in \spc{N}$ lies in a star of a vertex $v_i$,
then $\bm{\psi}^{-1}\{x\}\z\subset V_i$;
in particular, we have $\rad[\bm{\psi}^{-1}\{x\}]<R$.

\parit{If part.}
Choose $x\in \spc{N}$.
Since the inverse image $\bm{\psi}^{-1}\{x\}$ is compact, $\bm{\psi}$ is continuous, and $\rad[\bm{\psi}^{-1}\{x\}]<R$,
there is a neighborhood $U\ni x$ such that the  $\rad[\bm{\psi}^{-1}(U)]<R$.

Since $\spc{X}$ is compact,  there is a finite cover $\{U_i\}$ of $\spc{N}$ such that $\bm{\psi}^{-1}(U_i)\subset\spc{X}$ has a radius smaller than $R$ for each $i$.
Since $\spc{N}$ has dimension $n$, we can inscribe%
\footnote{Recall that a covering $\{W_i\}$ is inscribed in the covering $\{U_i\}$ if for every $W_i$ is a subset of some $U_j$.} 
in $\{U_i\}$ a finite open cover $\{W_i\}$ with multiplicity at most $n+1$.
It remains to observe that $V_i=\bm{\psi}^{-1}(W_i)$ defines a finite open cover of $\spc{X}$ with  multiplicity at most $n+1$ and $\rad V_i<R$ for any $i$. 

\parbf{\ref{ex:1D-case}.}
Assume that $\spc{P}$ is connected.

Let us show that $\diam\spc{P}<R$.
If this is not the case, then there are points $p,q\in\spc{P}$ on distance $R$ from each other.
Let $\gamma$ be a shortest path from $p$ to $q$.
Clearly $\length\gamma\ge R$ and $\gamma$ lies in $\oBall(p,R)$ except for the endpoint $q$.
Therefore, $\length[\oBall(p,R)_{\spc{P}}]\ge R$.
Since $\VolPro_{\spc{P}}(R)\z\ge \length[\oBall(p,R)_{\spc{P}}]$,
the latter contradicts $\VolPro_{\spc{P}}(R)<R$.

In general case, we get that each connected component of $\spc{P}$ has a radius smaller than $R$.
Whence the width of $\spc{P}$ is smaller than $R$.

\parit{Second part.} Again, we can assume that $\spc{P}$ is connected.

The examples of line segment or a circle show that the constant $c=\tfrac12$ cannot be improved.
It remains to show that the inequality holds with $c=\tfrac12$.

Choose $p\in\spc{P}$ such that the value
\[\rho(p)=\max\set{\dist{p}{q}{\spc{P}}}{q\in\spc{P}}\]
is minimal.
Suppose $\rho(p)\ge\tfrac 12\cdot R$.
Observe that there is a point $x\z\in \spc{P}\backslash\{p\}$ that lies on any shortest path starting from $p$ and length $\ge\tfrac 12\cdot R$.
Otherwise for any $r\in(0,\tfrac 12\cdot R)$ there would be at least two points on distance $r$ from $p$;
by coarea inequality we get that the total length of $\spc{P}\cap \oBall(p,\tfrac 12\cdot R)$ is at least $R$ --- a contradiction.

Moving $p$ toward $x$ reduces $\rho(p)$ which contradicts the choice of~$p$.

\parbf{\ref{ex:sys<width}.}
The inequality $6\cdot R<s$ used twice:
\begin{itemize}
\item to shrink the triangle $[p_ip_jp_k]$ to a point;
\item to extend the constructed homotopy on $\spc{M}^0$ to $\spc{M}^1$.
\end{itemize}

The first issue can be resolved by passing to a barycentric subdivision of $\spc{N}^2$;
denote by $v_{ij}$ and $v_{ijk}$ the new vertices in the subdivision that correspond to edge $[v_iv_j]$ and triangle $[v_iv_jv_k]$ respectively.

Further for each vertex $v_{ij}$ choose a point $p_{ij}\in V_i\cap V_j$ and set $f(v_{ij})=p_{ij}$.
Similarly for each vertex $v_{ijk}$ choose a point $p_{ijk}\z\in V_i\cap V_j\cap V_k$ and set $f(v_{ijk})=p_{ijk}$.

Note that 
\begin{align*}
|p_i-p_{ij}|&<R,
\\
|p_i-p_{ijk}|&<R,
\\
|p_{ij}-p_{ijk}|&<2\cdot R.
\end{align*}
Therefore, perimeter of the triangle $[p_ip_{ij}p_{ijk}]$ in the subdivision is less that $4\cdot R$.
It resolves the first issue.

The second issue disappears if one estimates the distances a bit more carefully.
 
\parbf{\ref{ex:fillrad-inj}.}
Choose a fine covering of $\spc{M}$ with multiplicity at most $n$.
Choose $\bm{\psi}$ from $\spc{M}$ to the nerve $\spc{N}$ of the covering the same way as in the proof of \ref{thm:sys<width}.

It remains to construct $f\:\spc{N}\to\spc{M}$ and show that $f\circ\bm{\psi}$ is homotopic to the identity map.
To do this, apply the same strategy as in the proof of \ref{thm:sys<width} together with the so-called \index{geodesic cone construction}\emph{geodesic cone construction}
described below.

Let $\triangle$ be a simplex in a barycentric subdivision of $\spc{N}$.
Suppose that a map $f$ is defined on one facet $\triangle'$ of $\triangle$ to $\spc{M}$ and $\oBall(p,r)\supset f(\triangle')$.
Then one can extend $f$ to whole $\triangle$ such that the remaining vertex $v$ maps to $p$.
Namely connect each point $f(x)$ to $p$ by minimizing geodesic path $\gamma_x$ (by assumption it is uniquely defined) and set
\[f
\:
t\cdot x\z+(1-t)\cdot v
\mapsto
\gamma_x(t).\]

\parbf{\ref{ex:connected-sum-essential}.}
Suppose $M$ is an essential manifold and $N$ is an arbitrary closed manifold.
Observe that shrinking $N$ to a point produces a map $N\#M\to M$ of degree 1.
In particular, there is a map $f\:N\#M\to M$ that sends the fundamental class of $N\#M$ to the fundamental class of $M$.

Since $M$ is essential, there is an aspherical space $K$ and a map $\iota\:M\to K$ that sends the fundamental class of $M$ to a nonzero homology class in $K$.
From above, the composition $\iota\circ f\:N\#M\to K$ sends the fundamental class of $N\#M$ to the same homology class in~$K$.


\parbf{\ref{ex:product-essential}.}
Suppose $M_1$ and $M_2$ are essential.
Let $\iota_1\:M_1\to K_1$ and $\iota_2\:M_2\to K_2$ are the maps to aspherical spaces as in the definition (\ref{def:essential}).
Show that the map
$(\iota_1,\iota_2)\:M_1\times M_2\to K_1\times K_2$
meets the definition.

\parit{Remarks.}
Choose a group $G$.
Note that there is an aspherical connected space CW-complex $K$ with fundamental group $G$.
The space $K$ is called an \index{K(G,1) space@$K(G,1)$ space}\emph{Eilenberg--MacLane space of type $K(G,1)$}, or briefly a $K(G,1)$ space.
Moreover it is not hard to check that
\begin{itemize}
\item $K$ is uniquely defined up to a weak homotopy equivalence;
\item if $\spc{W}$ is a connected finite CW-complex.
Then any homomorphism $\pi_1(\spc{W},w)\to\pi_1(K,k)$ is induced by a continuous map $\phi\:(\spc{W},w)\to(K,k)$.
Moreover, $\phi$ is uniquely defined up to homotopy equivalence.
\end{itemize}

\begin{itemize}
 \item Suppose that $M$ is a closed manifold, 
$K$ is a $K(\pi_1(M),1)$ space and a map $\iota\:M\to K$ induces an isomorphism of fundamental groups.
Then $M$ is essential if and only if $\iota$ sends the fundamental class of $M$ to a nonzero homology class of $K$.
\end{itemize}

The property described in the last statement is the original definition of essential manifold.
It can be used to prove a converse to the exercise;
namely \textit{the product of a nonessential closed manifold with any closed manifold is \emph{not} essential}.





%\chapter{Midterm}\label{chap:midterm}

An oral exam, Th, Feb 27 in class.

\bigskip

\noi 
One theoretical questions from the following list:

\begin{enumerate}
\item 
Semicontinuity of length.
\item
Length spaces and Hopf--Rinow theorem.
\item
Fréchet lemma and Kuratowski embedding.
\item
Hausdorff convergence and Blaschke selection theorem.
\item
Gromov--Hausdorff metric, why it is a metric, almost isometries.
\item
Uniformly totally bonded families and Gromov selection theorem.
\item
Ultralimits and ultrapower of spaces.
\item
Urysohn space.
\item
Injective spaces and injective envelop.
\end{enumerate}

\bigskip

\noi One exercise from the following list:
\\
\ref{ex:almost-min},
\ref{ex:non-contracting-map},
\ref{ex:compact=>complete},
\ref{ex:compact-length},
\\
\ref{ex:Huas-perimeter-area},
\\
\ref{pr:doubling},
\ref{pr:under},
\ref{ex:GH-SC},
\ref{ex:sphere-to-ball},
\\
\ref{ex:ultrapower}, 
\ref{ex:two-geodesics-in-ultrapower},
\ref{ex:lim(tree)},
\\
\ref{ex:geodesics-urysohn},
\ref{ex:sphere-in-urysohn},
\ref{ex:compact-extension},
\\
\ref{ex:+-c},
\ref{ex:ultrametric},
\ref{ex:injective-spaces},
\ref{ex:tripod+square},
\ref{ex:4-on-a-line}.

\bigskip

\noi One more problem for a perfect score.

%%%%%%%%%%%%%%%%%%%%%%%%%%%%
{\small\sloppy
%\RequirePackage{snapshot}
\RequirePackage{snapshot}
\makeatletter
\def\snap@providesfile#1[#2]{%
  \wlog{File: #1 #2}%
  \if\expandafter\snap@graphic@test\expanded{#2}@@\@nil
    \snap@record@graphic#1\relax #2 (type ??)\@nil
  \else
    \expandafter\xdef\csname ver@#1\endcsname{#2}%
  \fi
  \endgroup
}
\makeatother

\documentclass[twoside]{book}

\usepackage{lectures}
\usepackage[colorlinks=true,
citecolor=black,
linkcolor=black,
anchorcolor=black,
filecolor=black,
menucolor=black,
urlcolor=black,
pdftitle={Metric geometry on manifolds: two lectures},
pdfsubject={Geometry},
pdfauthor={Anton Petrunin}
]{hyperref}
\makeindex

\begin{document}
%\pagestyle{empty}
 
\title{Metric geometry on manifolds:
\\ two lectures}
\author{Anton Petrunin}
\date{}
\maketitle

We discuss Besicovitch inequality, width, and systole of manifolds.
I assume that students are familiar with 
measure theory,
smooth manifolds,
degree of map, 
CW-complexes and related notions.

These are two final lectures of a graduate course given at Penn State, Spring 2020.
The complete lectures can be found on the author's website;
it includes an introduction to metric geometry \cite{petrunin2020pure}
and elements of Alexandrov geometry based on \cite{alexander-kapovitch-petrunin-2019}.

\thispagestyle{empty}
\tableofcontents
\thispagestyle{empty}

%%%%%%%%%%%%%%%%%%%%%%%%%%%%
%\include{hwa}
%%%%%%%%%%%%%%%%%%%%%%%%%%%%

\include{curvature-free}
\include{systole}
%\include{filling-rad}
%\include{examples}
%\include{AA}
\appendix
\chapter{Semisolutions}
\input{metrics-on-manifolds-sol}

%\include{tickets}
%%%%%%%%%%%%%%%%%%%%%%%%%%%%
{\small\sloppy
\input{metrics-on-manifolds.ind}

\printbibliography[heading=bibintoc]
\fussy
}


\end{document}


\printbibliography[heading=bibintoc]
\fussy
}


\end{document}


\printbibliography[heading=bibintoc]
\fussy
}


\end{document}


\def\emph{\textit}

\printbibliography[heading=bibintoc]
\fussy
}


\end{document}
