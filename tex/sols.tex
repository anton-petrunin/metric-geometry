\chapter{Semisolutions}

\parbf{Exercise~\ref{ex:almost-min}.}
Assume the statement is wrong. 
Then for any point $x\in \spc{X}$, there is a point $x'\in \spc{X}$ such that 
\[\dist{x}{x'}{}< \rho(x)
\quad\text{and}\quad
\rho(x')\le\frac{\rho(x)}{1+\eps}.\]
Consider a sequence of points $(x_n)$ such that $x_{n+1}\z=x_n'$.
Clearly 
\[\dist{x_{n+1}}{x_n}{}
\le
\frac{\rho(x_0)}{\eps\cdot(1+\eps)^n}
\quad\hbox{and}\quad
\rho(x_n)\le \frac{\rho(x_0)}{(1+\eps)^n}.\] 
Therefore $(x_n)$ is Cauchy.
Since $\spc{X}$, the sequence $(x_n)$;
denote its limit by $x_\infty$.
Since $\rho$ is a continuous function we get
\begin{align*}\rho(x_\infty)&=\lim_{n\to\infty}\rho(x_n)=
\\&=0.
\end{align*}

The latter contradicts that $\rho>0$.
\qeds

\parbf{Exercise~\ref{ex:non-contracting-map}.}
Given any pair of point $x_0,y_0\in \spc{K}$, 
consider two sequences $x_0,x_1,\dots$ and $y_0,y_1,\dots$
such that $x_{n+1}=f(x_n)$ and $y_{n+1}\z=f(y_n)$ for each $n$.

Since $\spc{K}$ is compact, 
we can choose an increasing sequence of integers $n_k$
such that both sequences $(x_{n_i})_{i=1}^\infty$ and $(y_{n_i})_{i=1}^\infty$
converge.
In particular, both are Cauchy sequences;
that is,
\[
|x_{n_i}-x_{n_j}|_{\spc{K}}, |y_{n_i}-y_{n_j}|_{\spc{K}}\to 0
\quad
\text{as}
\quad\min\{i,j\}\to\infty.
\]


Since $f$ is non-contracting, we get
\[
|x_0-x_{|n_i-n_j|}|
\le 
|x_{n_i}-x_{n_j}|.
\]

It follows that  
there is a sequence $m_i\to\infty$ such that
\[
x_{m_i}\to x_0\quad\text{and}\quad y_{m_i}\to y_0\quad\text{as}\quad i\to\infty.
\leqno({*})\]

Set \[\ell_n=|x_n-y_n|_{\spc{K}}.\]
Since $f$ is non-contracting, the sequence $(\ell_n)$ is non-decreasing.

By $({*})$,  $\ell_{m_i}\to\ell_0$ as $m_i\to\infty$.
It follows that $(\ell_n)$ is a constant sequence.

In particular 
\[|x_0-y_0|_{\spc{K}}=\ell_0=\ell_1=|f(x_0)-f(y_0)|_{\spc{K}}\]
for any pair of points $(x_0,y_0)$ in $\spc{K}$.
That is, $f$ is distance preserving, in particular injective.

From $({*})$, we also get that $f(\spc{K})$ is everywhere dense.
Since $\spc{K}$ is compact $f\:\spc{K}\to \spc{K}$ is surjective. Hence the result follows.\qeds


This is a basic lemma in the introduction to Gromov--Hausdorff distance \cite[see 7.3.30 in][]{burago-burago-ivanov}.
I learned this proof from Travis Morrison, 
a student in my MASS class at Penn State, Fall 2011.

Note that as an easy corollary one can see that any surjective non-expanding map from a compact metric space to itself is an isometry.




\parbf{Exercise~\ref{ex:no-geod}.}
We assume that the space is not trivial, otherwise a one-point space is an example.

Consider the unit ball $(B,\rho_0)$
in the space $c_0$ of all sequences converging to zero equipped with the sup-norm.

Consider another metric $\rho_1$ which is different from $\rho_0$ by the conformal factor
\[\phi(\bm{x})=2+\tfrac{1}2\cdot x_1+\tfrac{1}4\cdot x_2+\tfrac{1}8\cdot x_3+\dots,\]
where $\bm{x}=(x_1,x_2\,\dots)\in B$.
That is, if $\bm{x}(t)$, $t\in[0,\ell]$, is a curve parametrized by $\rho_0$-length 
then its $\rho_1$-length is 
\[\length_{\rho_1}\bm{x}=\int\limits_0^\ell\phi\circ\bm{x}.\]
Note that the metric $\rho_1$ is bi-Lipschitz to~$\rho_0$.

Assume $\bm{x}(t)$ and $\bm{x}'(t)$ are two curves parametrized by $\rho_0$-length that differ only in the $m$-th coordinate, denoted by $x_m(t)$ and $x_m'(t)$ correspondingly.
Note that if $x'_m(t)\le x_m(t)$ for any $t$ and 
the function $x'_m(t)$ is locally $1$-Lipschitz at all $t$ such that $x'_m(t)< x_m(t)$, then 
\[\length_{\rho_1}\bm{x}'\le \length_{\rho_1}\bm{x}.\]
Moreover this inequality is strict if $x'_m(t)< x_m(t)$ for some~$t$.

Fix a curve $\bm{x}(t)$, $t\in[0,\ell]$, parametrized by  $\rho_0$-length.
We can choose $m$ large, so that $x_m(t)$ is sufficiently close to $0$ for any~$t$.
In particular, for some values $t$, we have $y_m(t)<x_m(t)$, where
\[y_m(t)=(1-\tfrac t\ell)\cdot x_m(0)
+\tfrac t\ell\cdot x_m(\ell)
-\tfrac 1{100}\cdot \min\{t,\ell-t\}.\]
Consider the curve $\bm{x}'(t)$ as above with
\[x'_m(t)=\min\{x_m(t),y_m(t)\}.\]
Note that $\bm{x}'(t)$ and $\bm{x}(t)$ have the same end points, and by the above
\[\length_{\rho_1}\bm{x}'<\length_{\rho_1}\bm{x}.\]
That is, for any curve $\bm{x}(t)$ in $(B,\rho_1)$, we can find a shorter curve $\bm{x}'(t)$ with the same end points.
In particular, $(B,\rho_1)$ has no geodesics.\qeds

This example was suggested by Fedor Nazarov~\cite{nazarov}.

\parbf{Exercise~\ref{ex:compact=>complete}.}
Choose a Cauchy sequence $(x_n)$ in $(\spc{X},\|*-*\|)$; it sufficient to show that a subsequence of $(x_n)$ converges.

Note that the sequence $(x_n)$ is Cauchy in $(\spc{X},|*-*|)$;
denote its limit by $x_\infty$.

After passing to a subsequence, we can assume that $\|x_n-x_{n+1}\|\z<\tfrac1{2^n}$.
It follows that there is a 1-Lipschitz path $\gamma$ in $(\spc{X},\|*-*\|)$ such that $x_n=\gamma(\tfrac1{2^n})$ for each $n$ and $x_\infty=\gamma(0)$.

It follows that
\begin{align*}
\|x_\infty-x_n\|&\le \length\gamma|_{[0,\frac1{2^n}]}\le
\\
&\le \tfrac1{2^n}.
\end{align*}
In particular $x_n$ converges.\qeds

Source: \cite[Lemma 2.3]{petrunin-stadler}.


\begin{wrapfigure}{o}{20 mm}
\vskip-0mm
\centering
\includegraphics{mppics/pic-1}
\end{wrapfigure}

\parbf{Exercise~\ref{exercise from BH}.}
Consider the following subset of $\RR^2$ equipped with the induced length metric
\[
\spc{X}
=
\bigl((0,1]\times\{0,1\}\bigr)
\cup
\bigl(\{1,\tfrac12,\tfrac13,\dots\}\times[0,1]\bigr)
\]
Note that $\spc{X}$ is locally compact and geodesic.

Its completion $\bar{\spc{X}}$ is isometric to the closure of $\spc{X}$ equipped with the induced length metric;
$\bar{\spc{X}}$ is obtained from $\spc{X}$ by adding two points $p=(0,0)$ and $q=(0,1)$.

The point $p$ admits no compact neighborhood in $\bar{\spc{X}}$ 
and there is no geodesic connecting $p$ to $q$ in~$\bar{\spc{X}}$. \qeds

This exercise and solution is taken from \cite{bridson-haefliger}.


\parbf{Exercise~\ref{ex:Hausdorff-bry}.}
The answer is ``no'' in both parts.

For the first part let $X$ be a unit disc and $Y$ a finite $\eps$-net in $X$.
Evidently $|X-Y|_{\mathcal{H}(\RR^2)}<\eps$, 
but
$|\partial X-\partial Y|_{\mathcal{H}(\RR^2)}\approx 1$.

For the second part take $X$ to be a unit disc and $Y=\partial X$ to be its boundary circle.
Note that $\partial X=\partial Y$ in particular $\dist{\partial X}{\partial Y}{\mathcal{H}(\RR^2)}=0$ while $\dist{ X}{ Y}{\mathcal{H}(\RR^2)}=1$.
\qeds

A more interesting example for the second part can be build on \emph{lakes of Wada} --- and example of three open bounded topological disks in the plane that have identical boundary.

\parbf{Exercise~\ref{ex:Huas-perimeter-area}.}
Let $A$ be a compact convex set in the plane.
Denote by $A^r$ the closed $r$-neighborhood of $A$.
Recall that by Steiner's formula we have
\[\area A^r=\area A+r\cdot\perim A+\pi\cdot r^2.\]
Taking derivative and applying coarea formula, we get 
\[\perim A^r=\perim A+2\cdot\pi\cdot r.\]

Observe that if $A$ lies in a compact set $B$ bounded by a colsed curve then 
\[\perim A\le \perim B.\]
Indeed the closest-point projection $\RR^2\to A$ is short and it maps $\partial B$ onto $\partial A$.

It remains to observe that if $A_n\to A_\infty$, then for any $r>0$ we have that
\[A_\infty^r\supset A_n
\quad\text{and}\quad
A_\infty\subset A_n^r\]
for all large $n$.

\parbf{Exercise~\ref{ex:GH-po}.}
In order to check that $\dist{*}{*}{\spc{M}'}$ is a metric, it is sufficient to show that
\[\dist{\spc{X}}{\spc{Y}}{\spc{M}'}=0 
\quad\Longrightarrow\quad
\spc{X}\iso\spc{Y};\]
the remaining conditions are trivial.

If $\dist{\spc{X}}{\spc{Y}}{\spc{M}'}=0$, then there is a sequence of maps $f_n\:\spc{X}\to \spc{Y}$ such that 
\[\dist{f_n(x)}{f_n(x')}{\spc{Y}}\ge \dist{x}{x'}{\spc{X}}-\tfrac1n.\]

Choose a countable dense set $S$ in $\spc{X}$.
Passing to a subsequence of $f_n$ we can assume that $f_n(x)$ converges for any $x\in S$ as $n\to\infty$;
denote its limit by $f_\infty(x)$.

For each point $x\in\spc{X}$ choose a sequence $x_m\in S$ converging to $x$.
Since $\spc{Y}$ is compact, we can assume in addition that $y_m=f_\infty(x_m)$ converges in $\spc{Y}$.
Set $f_\infty(x)=y$.
Note that the map $f_\infty\:\spc{X}\to \spc{Y}$ is  distance non-decreasing.

The same way we can construct a distance non-decreasing map 
$g_\infty\:\spc{Y}\to \spc{X}$.

By Exercise~\ref{ex:non-contracting-map}, the compositions $f_\infty\circ g_\infty\:\spc{Y}\to \spc{Y}$ and $g_\infty\z\circ f_\infty\:\spc{X}\to \spc{X}$ are isometrises.
Therefore $f_\infty$ and $g_\infty$ are isometries as well.

(The proof of the second part is coming.) %???

\parbf{Exercise~\ref{pr:doubling}.}
Choose a space $\spc{X}$ in $\spc{Q}(C,D)$, denote a $C$-doubling measure by $\mu$.
Without loss of generality we may assume that $\mu(\spc{X})\z=1$.

The doubling condition implies that 
\[\mu[\oBall(p,\tfrac D{2^n})]\ge\tfrac 1{C^n}\]
for any point $x\in \spc{X}$.
It follows that 
\[\pack_{\frac D{2^n}}\spc{X}\le C^n.\]

By Exercise~\ref{ex:pack-net}, for any $\eps\ge\frac D{2^{n-1}}$, the space $\spc{X}$ admits an $\eps$-net with at most $C^n$ points.
Whence $\spc{Q}(C,D)$ is uniformly totally bounded.

\parbf{Exercise~\ref{pr:under}.} 
Since $\spc{Y}$ is compact, it has a finite $\eps$-net for any $\eps>0$.
For each $\eps>0$ choose a finite $\eps$-net $\{y_1,\dots,y_{n_\eps}\}$ in $\spc{Y}$.

Suppose $f\:\spc{X}\to \spc{Y}$ be a distance non-decreasing map.
Choose one point $x_i$ in each nonempty subset $B_i=f^{-1}[\oBall(y_i,\eps)]$.
Note that the subset $B_i$ has diameter at most $2\cdot \eps$ and 
\[\spc{X}=\bigcup_iB_i.\]
Therefore the set of points $\{x_i\}$ forms a $2\cdot\eps$ net in $\spc{X}$.
Whence (\ref{SHORT.pr:under:if}) follows.

\parit{(\ref{SHORT.pr:under:only-if}).} Let $\spc{Q}$ be a uniformly totally bounded family of spaces. 
Suppose that each space in $\spc{Q}$ has an $\tfrac1{2^n}$-net with at most $M_n$ points; we may assume that $M_0=1$.

Consider the space $\spc{Y}$ of all infinite integer sequences $m_0,m_1,\dots$ such that $1\le m_n\le M_n$ for any $n$.
Given two sequences $\bm{\ell},\bm{m}\in\spc{Y}$, set 
\[\dist{\bm{\ell}}{\bm{m}}{\spc{Y}}=\tfrac1{2^{n-1}},\]
where $n$ is minimal index such that $\ell_n\ne m_n$.

It remains to observe that any space $\spc{X}$ in $\spc{Q}$ admits a distance non-decreasing map $\spc{X}\to \spc{Y}$.
