\chapter{Semisolutions}

\parbf{Exercise~\ref{ex:almost-min}.}
Assume the statement is wrong. 
Then for any point $x\in \spc{X}$, there is a point $x'\in \spc{X}$ such that 
\[\dist{x}{x'}{}< \rho(x)
\quad\text{and}\quad
\rho(x')\le\frac{\rho(x)}{1+\eps}.\]
Consider a sequence of points $(x_n)$ such that $x_{n+1}\z=x_n'$.
Clearly 
\[\dist{x_{n+1}}{x_n}{}
\le
\frac{\rho(x_0)}{\eps\cdot(1+\eps)^n}
\quad\hbox{and}\quad
\rho(x_n)\le \frac{\rho(x_0)}{(1+\eps)^n}.\] 
Therefore $(x_n)$ is Cauchy.
Since $\spc{X}$, the sequence $(x_n)$;
denote its limit by $x_\infty$.
Since $\rho$ is a continuous function we get
\begin{align*}\rho(x_\infty)&=\lim_{n\to\infty}\rho(x_n)=
\\&=0.
\end{align*}

The latter contradicts that $\rho>0$.
\qeds

\parbf{Exercise~\ref{ex:non-contracting-map}.}
Given any pair of point $x_0,y_0\in \spc{K}$, 
consider two sequences $x_0,x_1,\dots$ and $y_0,y_1,\dots$
such that $x_{n+1}=f(x_n)$ and $y_{n+1}\z=f(y_n)$ for each $n$.

Since $\spc{K}$ is compact, 
we can choose an increasing sequence of integers $n_k$
such that both sequences $(x_{n_i})_{i=1}^\infty$ and $(y_{n_i})_{i=1}^\infty$
converge.
In particular, both are Cauchy sequences;
that is,
\[
|x_{n_i}-x_{n_j}|_{\spc{K}}, |y_{n_i}-y_{n_j}|_{\spc{K}}\to 0
\quad
\text{as}
\quad\min\{i,j\}\to\infty.
\]


Since $f$ is non-contracting, we get
\[
|x_0-x_{|n_i-n_j|}|
\le 
|x_{n_i}-x_{n_j}|.
\]

It follows that  
there is a sequence $m_i\to\infty$ such that
\[
x_{m_i}\to x_0\quad\text{and}\quad y_{m_i}\to y_0\quad\text{as}\quad i\to\infty.
\leqno({*})\]

Set \[\ell_n=|x_n-y_n|_{\spc{K}}.\]
Since $f$ is non-contracting, the sequence $(\ell_n)$ is non-decreasing.

By $({*})$,  $\ell_{m_i}\to\ell_0$ as $m_i\to\infty$.
It follows that $(\ell_n)$ is a constant sequence.

In particular 
\[|x_0-y_0|_{\spc{K}}=\ell_0=\ell_1=|f(x_0)-f(y_0)|_{\spc{K}}\]
for any pair of points $(x_0,y_0)$ in $\spc{K}$.
That is, $f$ is distance preserving, in particular injective.

From $({*})$, we also get that $f(\spc{K})$ is everywhere dense.
Since $\spc{K}$ is compact $f\:\spc{K}\to \spc{K}$ is surjective. Hence the result follows.\qeds


This is a basic lemma in the introduction to Gromov--Hausdorff distance \cite[see 7.3.30 in][]{burago-burago-ivanov}.
I learned this proof from Travis Morrison, 
a student in my MASS class at Penn State, Fall 2011.

Note that as an easy corollary one can see that any surjective non-expanding map from a compact metric space to itself is an isometry.




\parbf{Exercise~\ref{ex:no-geod}.}
We assume that the space is not trivial, otherwise a one-point space is an example.

Consider the unit ball $(B,\rho_0)$
in the space $c_0$ of all sequences converging to zero equipped with the sup-norm.

Consider another metric $\rho_1$ which is different from $\rho_0$ by the conformal factor
\[\phi(\bm{x})=2+\tfrac{1}2\cdot x_1+\tfrac{1}4\cdot x_2+\tfrac{1}8\cdot x_3+\dots,\]
where $\bm{x}=(x_1,x_2\,\dots)\in B$.
That is, if $\bm{x}(t)$, $t\in[0,\ell]$, is a curve parametrized by $\rho_0$-length 
then its $\rho_1$-length is 
\[\length_{\rho_1}\bm{x}=\int\limits_0^\ell\phi\circ\bm{x}.\]
Note that the metric $\rho_1$ is bi-Lipschitz to~$\rho_0$.

Assume $\bm{x}(t)$ and $\bm{x}'(t)$ are two curves parametrized by $\rho_0$-length that differ only in the $m$-th coordinate, denoted by $x_m(t)$ and $x_m'(t)$ correspondingly.
Note that if $x'_m(t)\le x_m(t)$ for any $t$ and 
the function $x'_m(t)$ is locally $1$-Lipschitz at all $t$ such that $x'_m(t)< x_m(t)$, then 
\[\length_{\rho_1}\bm{x}'\le \length_{\rho_1}\bm{x}.\]
Moreover this inequality is strict if $x'_m(t)< x_m(t)$ for some~$t$.

Fix a curve $\bm{x}(t)$, $t\in[0,\ell]$, parametrized by  $\rho_0$-length.
We can choose $m$ large, so that $x_m(t)$ is sufficiently close to $0$ for any~$t$.
In particular, for some values $t$, we have $y_m(t)<x_m(t)$, where
\[y_m(t)=(1-\tfrac t\ell)\cdot x_m(0)
+\tfrac t\ell\cdot x_m(\ell)
-\tfrac 1{100}\cdot \min\{t,\ell-t\}.\]
Consider the curve $\bm{x}'(t)$ as above with
\[x'_m(t)=\min\{x_m(t),y_m(t)\}.\]
Note that $\bm{x}'(t)$ and $\bm{x}(t)$ have the same end points, and by the above
\[\length_{\rho_1}\bm{x}'<\length_{\rho_1}\bm{x}.\]
That is, for any curve $\bm{x}(t)$ in $(B,\rho_1)$, we can find a shorter curve $\bm{x}'(t)$ with the same end points.
In particular, $(B,\rho_1)$ has no geodesics.\qeds

This example was suggested by Fedor Nazarov~\cite{nazarov}.

\parbf{Exercise~\ref{ex:compact=>complete}.}
Choose a Cauchy sequence $(x_n)$ in $(\spc{X},\|*-*\|)$; it sufficient to show that a subsequence of $(x_n)$ converges.

Note that the sequence $(x_n)$ is Cauchy in $(\spc{X},|*-*|)$;
denote its limit by $x_\infty$.

After passing to a subsequence, we can assume that $\|x_n-x_{n+1}\|\z<\tfrac1{2^n}$.
It follows that there is a 1-Lipschitz path $\gamma$ in $(\spc{X},\|*-*\|)$ such that $x_n=\gamma(\tfrac1{2^n})$ for each $n$ and $x_\infty=\gamma(0)$.

It follows that
\begin{align*}
\|x_\infty-x_n\|&\le \length\gamma|_{[0,\frac1{2^n}]}\le
\\
&\le \tfrac1{2^n}.
\end{align*}
In particular $x_n$ converges.\qeds

Source: \cite[Lemma 2.3]{petrunin-stadler}.


\begin{wrapfigure}{o}{20 mm}
\vskip-0mm
\centering
\includegraphics{mppics/pic-1}
\end{wrapfigure}

\parbf{Exercise~\ref{exercise from BH}.}
Consider the following subset of $\RR^2$ equipped with the induced length metric
\[
\spc{X}
=
\bigl((0,1]\times\{0,1\}\bigr)
\cup
\bigl(\{1,\tfrac12,\tfrac13,\dots\}\times[0,1]\bigr)
\]
Note that $\spc{X}$ is locally compact and geodesic.

Its completion $\bar{\spc{X}}$ is isometric to the closure of $\spc{X}$ equipped with the induced length metric;
$\bar{\spc{X}}$ is obtained from $\spc{X}$ by adding two points $p=(0,0)$ and $q=(0,1)$.

The point $p$ admits no compact neighborhood in $\bar{\spc{X}}$ 
and there is no geodesic connecting $p$ to $q$ in~$\bar{\spc{X}}$. \qeds

This exercise and its solution is taken from \cite{bridson-haefliger}.


\parbf{Exercise~\ref{ex:compact-length}.} By Frechet lemma (\ref{lem:frechet}) we can identify $\spc{K}$ with a compact subset of $\ell^\infty$.

Denote by $\spc{L}=\Conv\spc{K}$ --- it is defined as the minimal convex closed set in $\ell^\infty$ that contains $\spc{K}$.
(In other words, $\spc{L}$ is the intersection of all convex closed sets that contain $\spc{K}$.)
Observe that $\spc{L}$ is a length space.

Let us show that since $\spc{K}$ is compact, so is $\spc{L}$.
By construction $\spc{L}$ is closed subset of $\ell^\infty$, in particular it is a complete space.
By \ref{totally-bounded}, it remains to show that $\spc{L}$ is totally bounded.

Recall that Minkowski sum $A + B$ of two sets $A$ and $B$ in a vector space is defined by
\[A + B = \set{a+b}{a\in A,\ b\in B}.\]
Observe that Minkowski sum of two convex sets is convex.

Denote by $\bar B_\eps$ the closed $\eps$-ball in $\ell^\infty$ centered at the origin.
Choose a finite $\eps$-net $N$ in $\spc{K}$ for some $\eps>0$.
Note that $P=\Conv N$ is a convex polyhedron, in particular $\Conv N$ is compact.

Observe that $N+\bar B_\eps$ is closed $\eps$-neighborhood of $N$;
therefore $N+\bar B_\eps\supset K$.
Therefore $P+\bar B_\eps\supset \spc{L}$;
in particular $P$ is a $2\cdot\eps$-net in $\spc{L}$.
That is, $\spc{L}$ admits a compact $\eps$-net for any $\eps>0$.
Therefore $\spc{L}$ is totally bounded (see \ref{ex:compact-net}).



\parbf{Exercise~\ref{ex:Hausdorff-bry}.}
The answer is ``no'' in both parts.

For the first part let $X$ be a unit disc and $Y$ a finite $\eps$-net in $X$.
Evidently $|X-Y|_{\mathcal{H}(\RR^2)}<\eps$, 
but
$|\partial X-\partial Y|_{\mathcal{H}(\RR^2)}\approx 1$.

For the second part take $X$ to be a unit disc and $Y=\partial X$ to be its boundary circle.
Note that $\partial X=\partial Y$ in particular $\dist{\partial X}{\partial Y}{\mathcal{H}(\RR^2)}=0$ while $\dist{ X}{ Y}{\mathcal{H}(\RR^2)}=1$.
\qeds

A more interesting example for the second part can be build on \emph{lakes of Wada} --- and example of three open bounded topological disks in the plane that have identical boundary.

\parbf{Exercise~\ref{ex:Huas-perimeter-area}.}
Let $A$ be a compact convex set in the plane.
Denote by $A^r$ the closed $r$-neighborhood of $A$.
Recall that by Steiner's formula we have
\[\area A^r=\area A+r\cdot\perim A+\pi\cdot r^2.\]
Taking derivative and applying coarea formula, we get 
\[\perim A^r=\perim A+2\cdot\pi\cdot r.\]

Observe that if $A$ lies in a compact set $B$ bounded by a colsed curve, then 
\[\perim A\le \perim B.\]
Indeed the closest-point projection $\RR^2\to A$ is short and it maps $\partial B$ onto $\partial A$.

It remains to observe that if $A_n\to A_\infty$, then for any $r>0$ we have that
\[A_\infty^r\supset A_n
\quad\text{and}\quad
A_\infty\subset A_n^r\]
for all large $n$.

\parbf{Exercise~\ref{ex:GH-po}.}
In order to check that $\dist{*}{*}{\spc{M}'}$ is a metric, it is sufficient to show that
\[\dist{\spc{X}}{\spc{Y}}{\spc{M}'}=0 
\quad\Longrightarrow\quad
\spc{X}\iso\spc{Y};\]
the remaining conditions are trivial.

If $\dist{\spc{X}}{\spc{Y}}{\spc{M}'}=0$, then there is a sequence of maps $f_n\:\spc{X}\to \spc{Y}$ such that 
\[\dist{f_n(x)}{f_n(x')}{\spc{Y}}\ge \dist{x}{x'}{\spc{X}}-\tfrac1n.\]

Choose a countable dense set $S$ in $\spc{X}$.
Passing to a subsequence of $f_n$ we can assume that $f_n(x)$ converges for any $x\in S$ as $n\to\infty$;
denote its limit by $f_\infty(x)$.

For each point $x\in\spc{X}$ choose a sequence $x_m\in S$ converging to $x$.
Since $\spc{Y}$ is compact, we can assume in addition that $y_m=f_\infty(x_m)$ converges in $\spc{Y}$.
Set $f_\infty(x)=y$.
Note that the map $f_\infty\:\spc{X}\to \spc{Y}$ is  distance non-decreasing.

The same way we can construct a distance non-decreasing map 
$g_\infty\:\spc{Y}\to \spc{X}$.

By Exercise~\ref{ex:non-contracting-map}, the compositions $f_\infty\circ g_\infty\:\spc{Y}\to \spc{Y}$ and $g_\infty\z\circ f_\infty\:\spc{X}\to \spc{X}$ are isometrises.
Therefore $f_\infty$ and $g_\infty$ are isometries as well.

%Observe that 
%$$|\spc{X}_n-\spc{X}_\infty|_{\mathcal{M}}\to 0 
%\quad\Longrightarrow\quad 
%\dist{\spc{X}_n}{\spc{X}_\infty}{\spc{M}'}\to 0$$
%follows from Proposition~\ref{prop:GH-e-isom} and Exercise~\ref{ex:alm-isom:inverse}.
%To prove that 
%$$|\spc{X}_n-\spc{X}_\infty|_{\mathcal{M}}\to 0 
%\quad\Longleftarrow\quad 
%\dist{\spc{X}_n}{\spc{X}_\infty}{\spc{M}'}\to 0,$$
%Suppose that $f_n\:\spc{X}_n\to\spc{X}_\infty$ and $g_n\:\spc{X}_\infty\to\spc{X}_n$ are $\eps_n$-almost distance non-decreasing maps for $\eps_n\to 0$.
%Arguing as above, pass to a partial limit $h$ of the sequence $f_n\circ g_n\:\spc{X}_\infty\to\spc{X}_\infty$.
%Note that $h$ is a distance non-deceasing map from a compact space to an itself.
%By Exercise~\ref{ex:non-contracting-map}, $h$ is an isometry.

(The proof of the second part is coming.) %???

\parbf{Exercise~\ref{pr:doubling}.}
Choose a space $\spc{X}$ in $\spc{Q}(C,D)$, denote a $C$-doubling measure by $\mu$.
Without loss of generality we may assume that $\mu(\spc{X})\z=1$.

The doubling condition implies that 
\[\mu[\oBall(p,\tfrac D{2^n})]\ge\tfrac 1{C^n}\]
for any point $x\in \spc{X}$.
It follows that 
\[\pack_{\frac D{2^n}}\spc{X}\le C^n.\]

By Exercise~\ref{ex:pack-net}, for any $\eps\ge\frac D{2^{n-1}}$, the space $\spc{X}$ admits an $\eps$-net with at most $C^n$ points.
Whence $\spc{Q}(C,D)$ is uniformly totally bounded.

\parbf{Exercise~\ref{pr:under}.} 
Since $\spc{Y}$ is compact, it has a finite $\eps$-net for any $\eps>0$.
For each $\eps>0$ choose a finite $\eps$-net $\{y_1,\dots,y_{n_\eps}\}$ in $\spc{Y}$.

Suppose $f\:\spc{X}\to \spc{Y}$ be a distance non-decreasing map.
Choose one point $x_i$ in each nonempty subset $B_i=f^{-1}[\oBall(y_i,\eps)]$.
Note that the subset $B_i$ has diameter at most $2\cdot \eps$ and 
\[\spc{X}=\bigcup_iB_i.\]
Therefore the set of points $\{x_i\}$ forms a $2\cdot\eps$ net in $\spc{X}$.
Whence (\ref{SHORT.pr:under:if}) follows.

\parit{(\ref{SHORT.pr:under:only-if}).} Let $\spc{Q}$ be a uniformly totally bounded family of spaces. 
Suppose that each space in $\spc{Q}$ has an $\tfrac1{2^n}$-net with at most $M_n$ points; we may assume that $M_0=1$.

Consider the space $\spc{Y}$ of all infinite integer sequences $m_0,m_1,\dots$ such that $1\le m_n\le M_n$ for any $n$.
Given two sequences $(\ell_n)$, and $(m_n)$ of points in $\spc{Y}$, set 
\[\dist{(\ell_n)}{(m_n)}{\spc{Y}}=\tfrac C{2^{n}},\]
where $n$ is minimal index such that $\ell_n\ne m_n$ and $C$ is a positive constant.

Observe that $\spc{Y}$ is compact.
Indeed it is complete and the sequences constant starting from index $n$ form a finite $\tfrac C{2^{n}}$-net in $\spc{Y}$.

Given a space $\spc{X}$ in $\spc{Q}$,
choose a sequnce of $\tfrac1{2^n}$ nets 
$N_n\subset\spc{X}$ for each natural $n$.
We can assume that $|N_n|\le M_n$; let us enumerate the points in $N_n$ by $\{1,\dots,M_n\}$.
Consider the map $f:\spc{X}\to\spc{Y}$ defined by $f:x\to (m_1(x),m_2(x),\dots)$ where $m_n(x)$ is a number of the point in $N_n$ that lies on the distance $<\tfrac1{2^n}$ from $x$.

If $\tfrac1{2^{n-2}}\ge \dist{x}{x'}{\spc{X}}>\tfrac1{2^{n-1}}$, then $m_n(x)\ne m_n(x')$.
It follows that $\dist{f(x)}{f(x')}{\spc{Y}}\ge \tfrac C{2^{n}}$.
In particular, if $C>10$, then 
\[\dist{f(x)}{f(x')}{\spc{Y}}\ge \dist{x}{x'}{\spc{X}}\]
for any $x,x'\in \spc{X}$.
That is, $f$ is a distance non-decreasing map $\spc{X}\to \spc{Y}$.

\parbf{Exercise~\ref{ex:GH-SC},} \textit{(\ref{SHORT.ex:GH-SC:nonsc-limit})}.
Suppose $\spc{X}_n\GHto \spc{X}$ and $\spc{X}_n$ are simply connected length metric space.
It is sufficient to show that any nontrivial covering map $f\:\tilde{\spc{X}}\to \spc{X}$ corresponds to a nontrivial covering map $f_n\:\tilde{\spc{X}}_n\to \spc{X}_n$ for large $n$.

The latter can be constructed by covering $\spc{X}_n$ by small balls that lie close to sets in $\spc{X}$ evenly covered by $f$, prepare few copies of these sets and glue them the same way as the inverse images of the evenly covered sets in $\spc{X}$ glued to obtain $\tilde{\spc{X}}$. %???

\begin{wrapfigure}{r}{40 mm}
\vskip-0mm
\centering
\includegraphics{mppics/pic-2}
\end{wrapfigure}

\parit{(\ref{SHORT.ex:GH-SC:nonsc-limit}).}
Let $\spc{V}$ be a cone over Hawaiian earring.
Consider the \emph{doubled cone} $\spc{W}$ --- two copies of $\spc{V}$ with glued base points earrings (see the diagrm).

The space $\spc{W}$ can be equipped with length metric for example the induced length metric from the shown embedding.

Note that $\spc{V}$ is simply connected, but $\spc{W}$ is not --- it is a good exercise in topology.

If we delete from the earrings all small circles, then the obtained double cone becomes simply connected and it remains to be close to $\spc{W}$ in the sense of Gromov--Hausdorff.

\parit{Comment.}
Note that from part (\ref{SHORT.ex:GH-SC:nonsc-limit}), the limit does not admit a nontrivial covering.
So if we define fundamental group right --- as the inverse image of groups of deck transformations for all its coverings, then one may say that Gromov--Haudorff limit of simply connected length spaces is simply connected.

\parbf{Exercise~\ref{ex:sphere-to-ball},}
\textit{(\ref{SHORT.ex:sphere-to-ball:2}).}
Suppose that a metric on $\mathbb{S}^2$ is close to the disc $\DD^2$.
Note that $\mathbb{S}^2$ contains a circle $\gamma$ that is close to the boundary curve of $\DD^2$.
By Jordan curve theorem, $\gamma$ divides $\mathbb{S}^2$ into two discs, say $D_1$ and $D_2$.

By \ref{ex:GH-SC:nonsc-limit}, the Gromov--Hausdorff limit of $D_1$ and $D_2$ have to contain whole $\DD^2$, otherwise the limit would admit a nontrivial covering.
Consider points $p_1\in D_1$ and $p_2\in D_2$ that a close to the center of $\DD^2$.
On one hand the distance $\dist{p_1}{p_2}{n}$ have to be very small.
On the other hand, any curve from $p_1$ to $p_2$ must cross $\gamma$, so it has legnth about 2 at least --- a contradiction.



\parit{(\ref{SHORT.ex:sphere-to-ball:3}).}
Make fine burrows in the standard 3-ball without changing its topology,
but at the same time come sufficiently close to any point in the ball.

Consider the \emph{doubling} of the obtained ball along its boundary;
that is, two copies of the ball with identified corresponding points on their boundaries.
The obtained space is homeomorphic to $\mathbb{S}^3$.
Note that the burrows can be made 
so that the obtained space is sufficiently close to the original ball 
in the Gromov--Hausdorff metric.\qeds

Source: \cite[Exercises 7.5.13 and 7.5.17]{burago-burago-ivanov}. 


\parbf{Exercise~\ref{ex:ultrapower}.}
Part (\ref{SHORT.ex:ultrapower:a}) follows directly from the definitions.
Further we consider $\spc{X}$ as a subset of $\spc{X}^\omega$.

\parit{(\ref{SHORT.ex:ultrapower:compact}).}
Suppose $\spc{X}$ compact.
Given a sequence $(x_n)$ in $\spc{X}$, denote its $\omega$-limit in $\spc{X}^\omega$ by $x^\omega$ and its $\omega$-limit in $\spc{X}$ by $x_\omega$.

Observe that $x^\omega=\iota(x_\omega)$.
Therefore $\iota$ is onto.

If $\spc{X}$ is not compact, we can choose a sequence $(x_n)$ such that $\dist{x_m}{x_n}{}>\eps$ for fixed $\eps>0$ and $m\ne n$.
Observe that
\[\lim_{n\to\omega}\dist{x_n}{y}{\spc{X}}\ge \tfrac\eps2\]
for any $y\in\spc{X}$.
It follows that $x_\omega$ lies on the distance at least $\tfrac\eps2$ from $\spc{X}$.

\parit{(\ref{SHORT.ex:ultrapower:proper}).}
A sequence of points $(x_n)$ in $\spc{X}$ will be called $\omega$-bounded if there is a real constant $C$ such that
\[\dist{p}{x_n}{\spc{X}}\le C\] 
for $\omega$-almost all $n$.

The same argument as in (\ref{SHORT.ex:ultrapower:compact}) shows that any $\omega$-bounded sequence has its $\omega$-limit in $\spc{X}$.
Further if $(x_n)$ is not  $\omega$-bounded, then 
\[\lim_{n\to\omega}\dist{p}{x_n}{\spc{X}}=\infty;\]
that is $x_\omega$ does not lie in the metric component of $p$ in $\spc{X}^\omega$.

\parbf{Exercise~\ref{ex:lim(tree)}.}
Observe that if a path $\gamma$ in a metric tree from $p$ to $q$ pass thru a point $x$ on distance $\ell$ from $[pq]$, then 
\[\length\gamma\ge \dist{p}{q}{}+2\cdot \ell.\eqlbl{eq:+ell}\]

Suppose that $\spc{T}_n$ is a sequence of metric trees that $\omega$-converges to $\spc{T}_\omega$.
By \ref{obs:ultralimit-is-geodesic}, the space $\spc{T}_\omega$.

The uniqueness will follow from \ref{eq:+ell}.
Indeed, if for a geodesic $[p_\omega q_\omega]$ there is another geodesic $\gamma_\omega$ connecting its ends,
then it have to pass thru a point $x_\omega\notin [p_\omega q_\omega]$.
Choose a sequences $p_n,q_n,x_n\in\spc{T}_n$ such that $p_n\to p_\omega$, $q_n\to q_\omega$, $x_n\to x_\omega$ and $n\to\omega$.
Then 
\begin{align*}
\dist{p_\omega}{q_\omega}{}&=\length\gamma\ge \lim_{n\to\omega}(\dist{p_n}{x_n}{}+\dist{q_n}{x_n}{})\ge
\\
&\ge \lim_{n\to\omega}(\dist{p_n}{q_n}{}+2\ell_n)=
\\
\dist{p_\omega}{q_\omega}{}+2\cdot\ell_\omega.
\end{align*}
Since $x_\omega\notin [p_\omega q_\omega]$, we have that $\ell_\omega>0$ --- a contradiction.

To prove the last property consider sequence of centers of tripods $m_n$ for points $x_n,y_n,z_n\in \spc{T}_n$ and observe that its ultralimit $m_\omega$ is a the ceter of tripod with ends at $x_\omega,y_\omega,z_\omega\in \spc{T}_\omega$.

\parbf{Exercise~\ref{ex:Asym(Lob)}.} Coming soon. %???

\parbf{Exercise~\ref{ex:geodesics-urysohn}.}
Construct a separable space that has infinite number of geodesics between a pair of points, say a square will $\ell^\infty$-metric in $\RR^2$ and apply universality of Urysohn space (\ref{prop:sep-in-urys}).

\parbf{Exercise~\ref{ex:sc-urysohn}.}
It is sufficient to show that any compact subspace $\spc{K}$ of Urysohn space can be contracted to a point.

Note that any compact space $\spc{K}$ can be extended to a contractible compact space $\spc{K}'$; for example we may embed $\spc{K}$ into $\ell^\infty$ and pass to its convex hull, as it was done in \ref{ex:compact-length}.

By \ref{thm:compact-homogeneous}, there is an isometric embedding of $\spc{K}'$ that agrees with inclusion of $\spc{K}$.
Since $\spc{K}$ is contractible in $\spc{K}'$, it is contractible in $\spc{U}$.

\medskip

In fact one can contract whole Urysohn space using the following construction.

Note that points in the space $\spc{X}_\infty$ constructed in the proof of \ref{prop:univeral-separable} can be multiplied number $t\in [0,1]$ --- simply multiply each function by factor $t$.
That defines a map 
\[\lambda_t\:\spc{X}_\infty\to \spc{X}_\infty\]
that scales all distances by factor $t$.
The map $\lambda_t$ can be extended to the completion of $\spc{X}_\infty$, which is isometic to $\spc{U}_d$ (or $\spc{U}$).

Observe that 
the map $\lambda_1$ is the identity  and $\lambda_0$ maps whole space to a single point, say $x_0$ --- that is the only point of $\spc{X}_0$.
Further note that the map $(t,p)\mapsto \lambda_t(p)$ is continuous ---  in particular $\spc{U}_d$ and $\spc{U}$ are contractible.

As a bonus, observe that for any point $p\in \spc{U}_d$ the curve $t\mapsto \lambda_t(p)$ is a geodesic path from $p$ to $x_0$.

Source: \cite[(d) on page 82]{gromov-2007}.


\parbf{Exercise~\ref{ex:sphere-in-urysohn}.}
Observe that $S$ is an $d$-Urysohn space and apply uniqueness (\ref{thm:urysohn-unique}).

\parbf{Exercise~\ref{ex:compact-extension}.} 
The following claim is a key to the proof.

\begin{thm}{Claim}
Suppose $f\: K\to\RR$ is an extension function defined on a compact subset $K$ of the Urysohn space $\spc{U}$.
Then there is a point $p\in \spc{U}$ such that 
$\dist{p}{x}{}=f(x)$ for any $x\in K$.
\end{thm}

\parit{Proof.}
Without loss of generality we may assume that $f(x)>0$ for any $x\in K$.
Since $K$ is compact, we may fix $\eps>0$ such that $f(x)>\eps$.

Consider the sequenc $\eps_n=\tfrac\eps{100\cdot 2^n}$.
Choose a sequence of $\eps_n$-nets $N_n\subset K$.
Applying universality of $\spc{U}$ recursively, we may choose a point $p_n$ such that $\dist{p_n}{x}{}=f(x)$ for any $x\in N_n$ and $\dist{p_n}{p_{n-1}}{}\z=10\cdot\eps_{n-1}$.
Observe that the sequence $(p_n)$ is Cauchy and its limit $p$ meets 
$\dist{p}{x}{}=f(x)$ for any $x\in K$.
\qeds

Choose a sequence of points $(x_n)$ in $\spc{S}$.
Applying the claim, we may extend the map from $K$ to $K\cup\{x_1\}$, and further to $K\cup\{x_1,x_2\}$, and so on.
As a result we extend the distance-preserving map $f$ to whole sequence $(x_n)$.
It remains to extend it continuisly to whole space $\spc{S}$.

\parbf{Exercise~\ref{ex:+-c}.}
If $c<0$ then $r>s$.
The latter is impossible since $r$ is extremal and $s$ is admissible.

Observe that the function $\bar r=\min\{\,r,s+c\}$ is admissible.
Indeed if $\bar r(x)=r(x)$ and $\bar r(y)=r(y)$ then 
\[\bar r(x)+\bar r(y)=r(x)+ r(y)\ge \dist{x}{y}{}.\]
Further if $\bar r(x)=s(x)+c$ then 
\begin{align*}
\bar r(x)+\bar r(y)&\ge [s(x)+c]+ [s(y)-c]= 
\\
&=s(x)+s(y) \ge 
\\
&\ge\dist{x}{y}{}.
\end{align*}

Since $r$ is extremal, we have $r=\bar r$;
that is $r\le s+c$.

\parbf{Exercise~\ref{ex:inj=complete-geodesic-contractible}.}
Choose an injective space $\spc{Y}$.

\textit{(\ref{SHORT.ex:inj=complete-geodesic-contractible:complete}).}
Fix a Cauchy sequence $(x_n)$ in $\spc{Y}$;
we need to show that it has a limit $x_\infty\in \spc{Y}$.
Consider metric on $\spc{X}=\NN\cup\{\infty\}$ defined by 
\begin{align*}
\dist{m}{n}{\spc{X}}&=\dist{x_m}{x_n}{\spc{Y}},
\\
\dist{m}{\infty}{\spc{X}}&=\lim_{n\to\infty}\dist{x_m}{x_n}{\spc{Y}}.
\end{align*}
Since the sequence is Cauchy, so is the sequence $\ell_n=\dist{p}{x_n}{\spc{Y}}$.
Therefore the last limit is defined.

By construction the map $n\mapsto x_n$ is distance preserving on $\NN\subset \spc{X}$.
Since $\spc{Y}$ is injective, this map can be exteneded to $\infty$ as a short map; set $\infty\mapsto x_\infty$.
Since $\dist{x_n}{x_\infty}{\spc{Y}}\le \dist{n}{\infty}{\spc{X}}$ 
and $\dist{n}{\infty}{\spc{X}}\to 0$, we get that
$x_n\to x_\infty$ as $n\to\infty$.

\textit{(\ref{SHORT.ex:inj=complete-geodesic-contractible:geodesic}).}
Applying the definition of injective space, we get a midpoint for any pair of points in $\spc{Y}$.
By (\ref{SHORT.ex:inj=complete-geodesic-contractible:complete}),
$\spc{Y}$ is a complete space.
It remains to apply \ref{lem:mid>geod:geod}.

\textit{(\ref{SHORT.ex:inj=complete-geodesic-contractible:contractible}).}
Let $k\:\spc{Y}\hookrightarrow \ell^\infty(\spc{Y})$ be the Kuratowski embedding (\ref{lem:kuratowski}).
Observe that $\ell^\infty(\spc{Y})$ is contractible;
in particular, there is a homotopy $k_t\:\spc{Y}\hookrightarrow \ell^\infty(\spc{Y})$ such that $k_0=k$ and $k_1$ is a constant map.
(In fact one can take $k_t=(1-t)\cdot k$.)

Since $k$ is distance preserving and $\spc{Y}$ is injective,
there is a short map $f\:\ell^\infty(\spc{Y})\to \spc{Y}$ such that the composition $f\circ k$ is the identity map on $\spc{Y}$.
The composition $f\circ k_t\:\spc{Y}\hookrightarrow \spc{Y}$ is a needed homotopy. 

\parbf{Exercise~\ref{ex:ultrametric}.}
Choose three points $x,y,z\in\spc{X}$ and set $\spc{A}=\{x,z\}$.
Let $f\:\spc{A}\to \spc{Y}$ be an isometry.
Then $F(y)=f(x)$ or $F(y)=f(z)$.
If  $f(y)=f(x)$, then
\begin{align*}
\dist{y}{z}{\spc{X}}&\ge  \dist{F(y)}{f(z)}{\spc{Y}}=
\\
 &=\dist{x}{z}{\spc{X}}.
\end{align*}
Analogously if $f(y)=f(z)$, then $\dist{x}{y}{\spc{X}}\ge\dist{x}{z}{\spc{X}}$.

It remains to observe that the strong triangle inequality holds in both cases.

\parbf{Exercise~\ref{ex:injective-spaces}.}
Suppose a short map $f\:A\to\spc{Y}$ is defined on a subset of a metric space $\spc{X}$.
We need to construct a short extension $F$ of $f$.

\parit{(\ref{SHORT.ex:injective-spaces:R}).}
In this case $\spc{Y}=\RR$.
Without loss of generality, we may assume that $A\ne\emptyset$, otherwise map whole $\spc{X}$ to a single point.
Set 
\[F(x)=\inf\set{f(a)-\dist{a}{x}{}}{a\in A}.\] 
Observe that $F$ is short and $F(a)=f(a)$ for any $a\in A$.

\parit{(\ref{SHORT.ex:injective-spaces:tree}).}
In this case $\spc{Y}$ is a complete metric tree.
Fix a point $p\in \spc{X}$ and $q\in\spc{Y}$.
Given a point $a\in A$,
let $x_a\in\cBall[f(a),\dist{a}{p}{}]$ be the point closest to $f(x)$.
Note that $x_a\in[q\,f(a)]$ and either $x_a=q$ or $x_a$ lies on distance $\dist{a}{p}{}$ from $f(a)$.

Note that the geodesics $[q\,x_a]$ are nested;
that is, for any $a,b\in A$ we have either $[q\,x_a]\subset [q\,x_b]$ or $[q\,x_b]\subset [q\,x_a]$.
Moreover, in the first case we have $\dist{x_b}{f(a)}{}\le \dist{p}{a}{}$ and in the second $\dist{x_a}{f(b)}{}\le \dist{p}{b}{}$.

It follows that the closure of the union of all geodesics $[q\,x_a]$ for $a\in\spc{A}$ is a geodesic.
Denote by $x$ its end (it exists since $\spc{Y}$ is complete).
It remains to observe that $\dist{x}{f(a)}{}\le \dist{p}{a}{}$ for any $a\in\spc{A}$;
that is, one one can take $f(p)=x$.

\parit{(\ref{SHORT.ex:injective-spaces:ell-infty}).}
In this case $\spc{Y}=(\RR^2,\ell^\infty)$.
Note that the map $\spc{X}\to (\RR^2,\ell^\infty)$ is short if and only if both of its coordinate projections are short.
It remains to apply (\ref{SHORT.ex:injective-spaces:R}).

\parbf{Exercise~\ref{ex:tripod+square};} \textit{(\ref{SHORT.ex:tripod+square:tripod}).}
Let $f$ be an extremal function.
Observe that at least two of the numbers $f(a)+f(b)$, $f(b)+f(c)$, and $f(c)+f(a)$ are $1$.
It follows that for some $x\in[0,\tfrac12]$, we have 
\begin{align*}
f(a)&=1\pm x,&
f(b)&=1\pm x,&
f(c)&=1\pm x,
\end{align*}
where we have one ``$-$'' and two ``$+$'' in these three formulas.

Suppose that
\begin{align*}
g(a)&=1\pm y,& g(b)&=1\pm y,& g(c)&=1\pm y
\end{align*}
is another extramal function.
Then $|f-g|\z=|x-y|$ if $g$ has ``$-$'' at the same place as $f$ and $|f-g|=|x+y|$ otherwise.

\begin{wrapfigure}{o}{30 mm}
\vskip-4mm
\centering
\includegraphics{mppics/pic-3}
\bigskip
\includegraphics{mppics/pic-4}
\end{wrapfigure}

It follows that $\Inj\spc{X}$ is isometric to a tripod --- that is, $\Inj\spc{X}$ can be made from three segments of length $\tfrac12$ and by gluing then at one end.

\parit{(\ref{SHORT.ex:tripod+square:square}).}
Assume $f$ is an extramal function.
Observe that 
$f(x)+f(y)=f(p)+f(q)=2$;
in particular, two values $a=f(x)-1$ and $b=f(p)-1$ completely describe the function $f$.
Since $f$ is extremal, we also have that 
\[(1\pm a)+(1\pm b)\ge 1\]
for all 4 choices of signs;
that is, $|a|+|b|\le 1$.

It follows that $\Inj\spc{X}$ is isometric to the rhombus $|a|+|b|\le 1$ in the $(a,b)$-plane with the metric induced by the $\ell^\infty$-norm.





\parbf{Exercise~\ref{ex:4-on-a-line}.}
Recall that 
\[\dist{f}{g}{\Inj\spc{X}}=\sup\set{|f(x)-g(x)|}{x\in\spc{X}}\]
and 
\[\dist{f}{p}{\Inj\spc{X}}=f(p)\]
for any $f,g\in \Inj\spc{X}$ and $p\in \spc{X}$.

Since $\spc{X}$ is compact we can find a point $p\in\spc{X}$ such that 
\[\dist{f}{g}{\Inj\spc{X}}=|f(p)-g(p)|=\left|\dist{f}{p}{\Inj\spc{X}}-\dist{g}{p}{\Inj\spc{X}}\right|.\]
Without loss of generality we may assume that 
\[\dist{f}{p}{\Inj\spc{X}}
=
\dist{g}{p}{\Inj\spc{X}}
+
\dist{f}{g}{\Inj\spc{X}}.\]
Applying \ref{lem:opposite}, we can find a point $q\in\spc{X}$ such that 
\[\dist{q}{p}{\Inj\spc{X}}
=
\dist{f}{p}{\Inj\spc{X}}
+
\dist{f}{q}{\Inj\spc{X}},\]
whence the result.

\parbf{Exercise~\ref{ex:sba-2+2-short};} \textit{``only if'' part.}
Let us start with two model triangles $[\tilde x\tilde y\tilde p]=\modtrig(xyp)$ and $[\tilde x\tilde y\tilde q]=\modtrig(xyq)$ such that $\tilde p$ and $\tilde q$ lie on the opposite sides of the line $\tilde x\tilde y$.

Suppose $[\tilde x \tilde y]$ intersects $[\tilde p\tilde q]$ at a point $\tilde z$.
In this case by $\CAT(0)$ comparison we have that
\[\dist{\tilde p}{\tilde q}{\EE^2}=\dist{\tilde p}{\tilde z}{\EE^2}-\dist{\tilde z}{\tilde q}{\EE^2}\le \dist{p}{q}{\spc{X}}.\]

\begin{wrapfigure}{o}{30 mm}
\vskip-0mm
\centering
\includegraphics{mppics/pic-5}
\end{wrapfigure}

Let us fix points $\tilde x$ and $\tilde y$, and the distances from $\tilde x$ to the remaining three points and reduce the angles $\alpha=\mangle\hinge{\tilde x}{\tilde p}{\tilde y}$ and $\beta=\mangle\hinge{\tilde x}{\tilde q}{\tilde y}$.
It results in decreasing distances $\dist{\tilde p}{\tilde q}{}$, $\dist{\tilde p}{\tilde y}{}$, and $\dist{\tilde q}{\tilde y}{}$.
If $\alpha=\beta=0$, then 
\begin{align*}
\dist{\tilde p}{\tilde q}{\EE^2}&=
\biggl|\dist{\tilde x}{\tilde p}{\EE^2}-\dist{\tilde x}{\tilde q}{\EE^2}\biggr|=
\\
&=\biggl|\dist{ x}{ p}{\spc{X}}-\dist{ x}{ q}{\spc{X}}\biggr|\ge
\\
&\ge\dist{p}{q}{\spc{X}}.
\end{align*}
By the intermediate value theorem, there are intermediate values of $\alpha$ and $\beta$ so that $\dist{\tilde p}{\tilde q}{\EE^2}\z=\dist{p}{q}{\spc{X}}$.
By construction, $\dist{\tilde x}{\tilde p}{\EE^2}=\dist{x}{p}{\spc{X}}$, $\dist{\tilde x}{\tilde q}{\EE^2}=\dist{x}{q}{\spc{X}}$, $\dist{\tilde y}{\tilde p}{\EE^2}\le\dist{y}{p}{\spc{X}}$, $\dist{\tilde y}{\tilde q}{\EE^2}\le\dist{y}{q}{\spc{X}}$.

\begin{wrapfigure}{o}{30 mm}
\vskip-0mm
\centering
\includegraphics{mppics/pic-6}
\end{wrapfigure}

Now suppose $[\tilde p \tilde q]$ does not intersect $[\tilde x\tilde y]$.
Without loss of generality, we may assume that $[\tilde p \tilde q]$ crosses the line $\tilde x\tilde y$ behind $\tilde x$.

Let us rotate $\tilde p$ around $\tilde x$ so that $\tilde x$ will lie between $\tilde p$ and $\tilde q$.
It will result in decreasing the distance $\dist{\tilde p}{\tilde y}{}$,
by triangle inequality we have that 
\begin{align*}
\dist{\tilde p}{\tilde q}{\EE^2}&=\dist{\tilde p}{\tilde x}{\EE^2}+\dist{\tilde x}{\tilde q}{\EE^2}=
\\
&=\dist{p}{x}{\spc{X}}+\dist{x}{q}{\spc{X}}\ge
\\
&\ge \dist{p}{q}{\spc{X}}.
\end{align*}
Repeating the argument above produces the needed configuration.

\parit{``If'' part.}
Suppose $\tilde p,\tilde q,\tilde x,\tilde y\in\EE^2$ satisfies the conditions 
\begin{align*}
\dist{\tilde p}{\tilde q}{}&=\dist{p}{q}{},
&
\dist{\tilde x}{\tilde y}{}&=\dist{x}{y}{},
\\
\dist{\tilde p}{\tilde x}{}&\le \dist{p}{x}{},
&
\dist{\tilde p}{\tilde y}{}&\le \dist{p}{y}{},
\\
\dist{\tilde q}{\tilde x}{}&\le \dist{q}{x}{},
&
\dist{\tilde q}{\tilde y}{}&\le \dist{q}{y}{}.
\end{align*}

Fix $\tilde z\in [\tilde x\tilde y]$.
By triangle inequality 
\[\dist{\tilde p}{\tilde z}{}+\dist{\tilde z}{\tilde q}{}\ge \dist{\tilde p}{\tilde q}{}=\dist{p}{q}{}.\]

Note that if 
$\dist{\tilde p'}{\tilde x}{}\ge \dist{\tilde p}{\tilde x}{}$
and
$\dist{\tilde p'}{\tilde y}{}\ge \dist{\tilde p}{\tilde y}{}$,
then $\dist{\tilde p'}{\tilde z}{}\ge \dist{\tilde p}{\tilde z}{}$.
In particular if $[\tilde x\tilde y\tilde p']=\modtrig(xyp)$ and $[\tilde x\tilde y\tilde q']=\modtrig(xyq)$, then
\begin{align*}
 \dist{\tilde p'}{\tilde z}{}+\dist{\tilde q'}{\tilde z}{}\ge \dist{\tilde p}{\tilde z}{}+\dist{\tilde z}{\tilde q}{}.
\end{align*}
Whence the ``if'' part follows.

\parbf{Exercise~\ref{ex:(3+1)-expanding}.}
Set $\tilde\alpha=\angk pxy$, $\tilde\beta=\angk pyz$ and $\tilde\gamma=\angk pzx$.

If $\spc{X}$ is $\Alex0$, then
\[\tilde\alpha+\tilde\beta+\tilde\gamma\le 2\cdot \pi.\]
Note that we can find $\alpha,\beta,\gamma$ such that 
\[
\tilde\alpha\le\alpha\le\pi,
\quad \tilde\beta\le\beta\le\pi,
\quad\tilde\gamma\le\gamma\le\pi,\]
and
\[\alpha+\beta+\gamma=2\cdot \pi.\]
Consider a model configuration $\tilde p$, $\tilde x$, $\tilde y$, $\tilde z\in\EE^2$ such that 
\begin{align*}
\dist{\tilde p}{\tilde x}{\EE^2}&=\dist{p}{x}{\spc{X}},
&
\dist{\tilde p}{\tilde y}{\EE^2}&=\dist{p}{y}{\spc{X}},
&
\dist{\tilde p}{\tilde z}{\EE^2}&=\dist{p}{z}{\spc{X}},
\\
\mangle\hinge {\tilde p}{\tilde x}{\tilde y}&=\alpha,
&
\mangle\hinge {\tilde p}{\tilde y}{\tilde z}&=\beta,
&
\mangle\hinge {\tilde p}{\tilde z}{\tilde x}&=\gamma.
\end{align*}

Since increasing angle in a triangle increase the opposite side, we have 
\begin{align*}
\dist{x}{y}{\spc{X}}&\le\dist{\tilde x}{\tilde y}{\EE^2},
&
\dist{y}{z}{\spc{X}}&\le\dist{\tilde y}{\tilde z}{\EE^2},
&
\dist{z}{x}{\spc{X}}&\le\dist{\tilde z}{\tilde x}{\EE^2}.
\end{align*}
Whence the ``only-if'' part follows.

Now suppse that we have a model configuration $\tilde p,\tilde x,\tilde y,\tilde z\in\EE^2$
such that 
\begin{align*}
\dist{p}{x}{\spc{X}}&=\dist{\tilde p}{\tilde x}{\EE^2},
&
\dist{p}{y}{\spc{X}}&=\dist{\tilde p}{\tilde y}{\EE^2},
&
\dist{p}{z}{\spc{X}}&=\dist{\tilde p}{\tilde z}{\EE^2},
\\
\dist{x}{y}{\spc{X}}&\le\dist{\tilde x}{\tilde y}{\EE^2},
&
\dist{y}{z}{\spc{X}}&\le\dist{\tilde y}{\tilde z}{\EE^2},
&
\dist{z}{x}{\spc{X}}&\le\dist{\tilde z}{\tilde x}{\EE^2}.
\end{align*}

Set
\begin{align*} 
\alpha&=\mangle\hinge {\tilde p}{\tilde x}{\tilde y},
&
\beta&=\mangle\hinge {\tilde p}{\tilde y}{\tilde z},
&
\gamma&=\mangle\hinge {\tilde p}{\tilde z}{\tilde x}.
\end{align*}
Observe that 
\[\alpha+\beta+\gamma\le 2\cdot \pi.\]
Since increasing a side in a triangle increase the opposite angle, we have that
\[
\tilde\alpha\le\alpha,
\quad \tilde\beta\le\beta,
\quad\tilde\gamma\le\gamma.\]
Whence the ``if'' part follows.

\parbf{Exercise~\ref{ex:CAT+CBB}.}
Set $\tilde \alpha=\angk pxq$, $\tilde\beta=\angk pyq$ and $\tilde\gamma=\angk pxy$.

Note that the quadruple $p,x,y,z$ is euclidean if 
\[\tilde\alpha+\tilde\beta+\tilde\gamma\le 2\cdot\pi
\eqlbl{eq:a+b+c=<2pi}\]
and the triple of numbers $\tilde\alpha,\tilde\beta,\tilde\gamma$ satisfies all triangle inequalities.
Without loss of generality we may assume that
$\tilde\alpha\le\tilde\beta\le\tilde\gamma$;
in this case the triangle inequities hold if 
\[\tilde\gamma\le \tilde\alpha+\tilde\beta.\eqlbl{eq:a+b>=c}\]

Note that the inequality \ref{eq:a+b+c=<2pi} follow from $\CBB(0)$ comparison.

Consider two model triangles $[\tilde x\tilde y\tilde p]=\modtrig(xyp)$ and $[\tilde x\tilde y\tilde q]=\modtrig(xyq)$ such that $\tilde p$ and $\tilde q$ lie on the opposite sides of the line $\tilde x\tilde y$.

Suppose $[\tilde x \tilde y]$ intersects $[\tilde p\tilde q]$ at a point $\tilde z$.
In this case by $\CAT(0)$ comparison we have that
\[\dist{\tilde x}{\tilde y}{\EE^2}=\dist{\tilde x}{\tilde z}{\EE^2}-\dist{\tilde z}{\tilde y}{\EE^2}\le \dist{x}{y}{\spc{X}}.\]
Which is equivalent to \ref{eq:a+b>=c}.

If $[\tilde x \tilde y]$ crosses the line $[\tilde p\tilde q]$ behind $\tilde p$,
then $\tilde\alpha+\tilde\beta>\pi$ and therefore \ref{eq:a+b>=c} follows from \ref{eq:a+b+c=<2pi}.

Finally if $[\tilde x \tilde y]$ crosses the line $[\tilde p\tilde q]$ behind $\tilde q$,
then by $\CBB(0)$ comparison with center at $q$, we have that 
\[\angk qxp+\angk qyp+\angk qxy\le 2\cdot\pi\]
It follows that 
\[\dist{\tilde x}{\tilde y}{\EE^2}\ge\dist{ x}{ y}{\spc{X}}\]
and therefore 
\[\tilde \gamma\le \mangle\hinge{\tilde p}{\tilde x}{\tilde y}.\]
Since $\mangle\hinge{\tilde p}{\tilde x}{\tilde y}=\tilde\alpha+\tilde\beta$ we get \ref{eq:a+b>=c}.

\parbf{Exercise~\ref{ex:product-CBB}.}
We will use the charcterization of $\CBB(0)$ space provided by \ref{ex:(3+1)-expanding}; the rest is nearly identical to the proof of \ref{ex:product-CAT}.

Fix a quadruple in $\spc{U}\times \spc{V}$:
\begin{align*}
p&=(p_1,p_2),
&
x&=(x_1,x_2),
&
y&=(y_1,y_2),
&
z&=(z_1,z_2).
\end{align*}
For the quadruple $p_1,x_1,y_1,z_1$ in $\spc{U}$,
construct model configurations  $\tilde p_1,\tilde x_1,\tilde y_1,\tilde z_1$ in $\EE^2$ provided by \ref{ex:(3+1)-expanding}.  
Similarly, for the quadruple $p_2,q_2,x_2,y_2$ in $\spc{V}$
construct model configurations  $\tilde p_2,\tilde x_2,\tilde y_2,\tilde z_2$ in $\EE^2$

Consider four points in $\EE^4=\EE^2\times\EE^2$ 
\begin{align*}
\tilde p&=(\tilde p_1,\tilde p_2),
&
\tilde x&=(\tilde x_1,\tilde x_2),
&
\tilde y&=(\tilde y_1,\tilde y_2),
&
\tilde z&=(\tilde z_1,\tilde z_2).
\end{align*}
The inequalities in  \ref{ex:(3+1)-expanding} imply that
\begin{align*}
\dist{p}{x}{\spc{X}}&=\dist{\tilde p}{\tilde x}{\EE^4},
&
\dist{p}{y}{\spc{X}}&=\dist{\tilde p}{\tilde y}{\EE^4},
&
\dist{p}{z}{\spc{X}}&=\dist{\tilde p}{\tilde z}{\EE^4},
\\
\dist{x}{y}{\spc{X}}&\le\dist{\tilde x}{\tilde y}{\EE^4},
&
\dist{y}{z}{\spc{X}}&\le\dist{\tilde y}{\tilde z}{\EE^4},
&
\dist{z}{x}{\spc{X}}&\le\dist{\tilde z}{\tilde x}{\EE^4}
\end{align*}
It remains to observe that one can move $\tilde z$ into the plane of $\tilde p$, $\tilde x$, and $\tilde y$ keeping the distance $\dist{\tilde p}{\tilde z}{\EE^4}$ and nondecreasing the rest of diatances. 

\parbf{Exercise~\ref{ex:fat-triangle}.}
Apply \ref{ex:noncreasing-CBB} twice.

More precisely, consider a triangle $[xyz]$ in the space; let $[\tilde x\tilde y\tilde z]\z=\modtrig(xyz)$.
Choose points $p\in[xy]$ and $q\in[xz]$;
consider the corresponding points $\tilde p\in[\tilde x\tilde y]$ and $\tilde q\in[\tilde x\tilde z]$.
We need to show that 
\[\dist{\tilde p}{\tilde q}{\EE^2}\le \dist{ p}{q}{\spc{X}}.
\eqlbl{eq:pq=<pq}\]

By \ref{ex:noncreasing-CBB}, we have
\[\angk{x}{p}{q}\ge\angk{x}{y}{z}.\]
Whence \ref{eq:pq=<pq} follows.
  
