\chapter{Exotic aspherical manifolds}\label{chapter:aspherical}

This lecture is nearly a copy of \cite[Sections 3.6--3.8]{alexander-kapovitch-petrunin-2019};
here we describe a set of rules for gluing Euclidean cubes that produce a 
locally $\CAT(0)$ space and use these rules to construct exotic examples of aspherical manifolds.

\section{Cubical complexes}

The definition of a cubical complex
mostly repeats the definition of a simplicial complex, 
with simplices replaced by cubes.

Formally, a \index{cubical complex}\emph{cubical complex} is defined as a subcomplex 
of the unit cube in the Euclidean space $\RR^N$ of large dimension;
that is, a collection of faces of the cube
such that together with each face it contains all its sub-faces.
Each cube face in this collection 
will be called a \index{cube}\emph{cube} of the cubical complex.

Note that according to this definition, 
any cubical complex is finite.

The union of all the cubes in a cubical complex $\spc{Q}$ will be called its \index{underlying space}\emph{underlying space}.
A homeomorphism from the underlying space of $\spc{Q}$ to a topological space $\spc{X}$ is called a \index{cubulation}\emph{cubulation of}~$\spc{X}$.

The underlying space of a cubical complex $\spc{Q}$ will be always considered with the length metric
induced from~$\RR^N$.
In particular, with this metric, 
each cube of $\spc{Q}$ is isometric to the unit cube of the corresponding dimension.

It is straightforward to construct a triangulation 
of the underlying space of $\spc{Q}$ 
such that each simplex is isometric to a Euclidean simplex.
In particular the underlying space of $\spc{Q}$ is a Euclidean polyhedral space.

The link of a cube in a cubical complex is defined similarly to the link of a simplex in a simplicial complex.
It is a simplicial complex that admits a natural all-right triangulation --- each simplex corresponds to an adjusted cube.

\parbf{Cubical analog of a simplicial complex.}
Let $\spc{S}$ be a finite simplicial complex and $\{v_1,\dots,v_N\}$ be the set of its vertices.

Consider $\RR^N$ with the standard basis $\{e_1,\dots,e_N\}$.
Denote by $\square^N$ the standard unit cube in $\RR^N$;
that is, 
\[\square^N=\set{(x_1,\dots,x_N)\in \RR^N}{0\le x_i\le 1\ \text{for each}\ i}.\]

Given a $k$-dimensional simplex $\<v_{i_0},\dots,v_{i_k}\>$ in $\spc{S}$, 
mark the $(k\z+1)$-dimensional faces in $\square^N$ (there are  $2^{N-k}$ of them)
which are parallel to the coordinate $(k+1)$-plane 
spanned by $e_{i_0},\dots,e_{i_k}$.


Note that the set of all marked faces of $\square^{N}$
forms a cubical complex;
it will be called 
the \index{cubical analog}\emph{cubical analog} of $\spc{S}$
and will be denoted as $\square_\spc{S}$.

\begin{thm}{Proposition}\label{prob:cubical-analog}
Let $\spc{S}$ be a finite connected simplicial complex
and $\spc{Q}=\square_{\spc{S}}$ be its cubical analog.
Then the underlying space of $\spc{Q}$ is connected 
and the link of any vertex of $\spc{Q}$
is isometric to  ${\spc{S}}$
equipped with the spherical right-angled metric.

In particular, if $\spc{S}$ is a flag complex,
then $\spc{Q}$ is a locally $\CAT(0)$
and therefore its universal cover $\tilde{\spc{Q}}$ is $\CAT(0)$.
\end{thm}

\parit{Proof.}
The first part of the proposition follows 
from the construction of $\square_{\spc{S}}$.

If ${\spc{S}}$ is flag, 
then by the flag condition (\ref{thm:flag}) 
the link of any cube in $\spc{Q}$ is $\CAT(1)$.
Therefore, by the cone construction (\ref{ex:cone+susp})
$\spc{Q}$
is locally $\CAT(0)$.
It remains to apply the globalization theorem 
(\ref{thm:hadamard-cartan}).
\qeds

From Proposition \ref{prob:cubical-analog}, 
it follows that the cubical analog
of any flag complex is aspherical.
The following exercise states that the  converse also holds; see \cite[5.4]{davis-2001}.

\begin{thm}{Exercise}\label{ex:flag-aspherical}
Show that a finite simplicial complex is flag 
if and only if its cubical analog is aspherical.
\end{thm}

\section{Construction}


By the globalization theorem (\ref{thm:hadamard-cartan}),
any proper length $\CAT(0)$ space is contractible.
Therefore all proper length, locally $\CAT(0)$ spaces 
are \index{aspherical}\emph{aspherical};
that is, they have contractible universal covers.
This observation can be used to construct examples of  aspherical spaces. 

Let $\spc X$ be a proper topological space.
Recall that $\spc X$ is called 
\index{simply connected space at infinity}\emph{simply connected at infinity} 
if for any compact set $K\subset\spc X$
there is a bigger compact set $K'\supset K$
such that  $\spc X\backslash K'$ is path connected 
and any loop which lies in $\spc X\backslash K'$
is null-homotopic in  $\spc X\backslash K$.

Recall that path connected spaces are not empty by definition.
Therefore compact spaces are not simply connected at infinity.

The following example was constructed by Michael Davis in \cite{davis-1983}.

\begin{thm}{Proposition}\label{prop:aspherical}
For any  $m\ge 4$ there is a closed aspherical 
$m$-dimensional manifold whose universal cover is not simply connected at infinity.

In particular, the universal cover of this manifold 
is not homeomorphic to the $m$-dimensional Euclidean space.
\end{thm}

The proof requires the following lemma.

\begin{thm}{Lemma}\label{lem:example-pi_infty}
Let $\spc{S}$ be a finite flag complex,
$\spc{Q}=\square_{\spc{S}}$ be its cubical analog
and $\tilde{\spc{Q}}$ be the universal cover of~$\spc{Q}$.

Assume  $\tilde{\spc{Q}}$ is simply connected at infinity.
Then $\spc{S}$ is simply connected.
\end{thm}

\parit{Proof.}
Assume $\spc{S}$ is not simply connected. Equip $\spc{S}$ with an all-right spherical metric.
Choose a shortest noncontractible circle $\gamma\:\mathbb{S}^1\to\spc{S}$ formed by the edges of~$\spc{S}$.

Note that $\gamma$ forms a one-dimensional subcomplex of $\spc{S}$ which is a closed local geodesic.
Denote by $G$ the subcomplex of $\spc{Q}$ which corresponds to~$\gamma$.

Fix a vertex $v\in G$;
let $G_v$ be the connected component of $v$ in~$G$.
Let $\tilde G$ be a connected component of the inverse image of $G_v$ in $\tilde{\spc{Q}}$
for the universal cover $\tilde{\spc{Q}}\to \spc{Q}$.
Fix a point $\tilde v\in\tilde G$ in the inverse image of~$v$.

\begin{wrapfigure}{r}{25mm}
\vskip0mm
\centering
\includegraphics{mppics/pic-1100}
\end{wrapfigure}
 
Note that 
\begin{clm}{}\label{tilde-G-convex}
$\tilde G$ is a convex set in~$\tilde{\spc{Q}}$.
\end{clm}


Indeed, according to Proposition \ref{prob:cubical-analog},
$\tilde{\spc{Q}}$ is $\CAT(0)$.
By Exercise \ref{ex:locally-convex},
it is sufficient to show that $\tilde G$ is locally convex in $\tilde{\spc{Q}}$,
or equivalently, $G$ is locally convex in~$\spc{Q}$.

Note that the latter can only fail if $\gamma$ contains two vertices, say $\xi$ 
and 
$\zeta$ in $\spc{S}$,
which are joined by an edge not in $\gamma$; 
denote this edge by~$e$.

Each edge of $\spc{S}$ has length~$\tfrac\pi2$.
Therefore each of two circles formed by $e$ and an arc of $\gamma$
from $\xi$ to $\zeta$ is shorter that~$\gamma$.
Moreover,
at least one of them is noncontractible 
since $\gamma$ is 
noncontractible.
That is, 
$\gamma$ is not a shortest noncontractible circle, a contradiction.
\claimqeds

Further, note that 
$\tilde G$ is homeomorphic to the plane, 
since $\tilde G$ is 
a two-dimensional manifold without boundary which 
by the above is $\CAT(0)$ and hence is contractible.

Denote by $C_R$ the circle of radius $R$ in $\tilde G$ centered at~$\tilde v$.
All $C_R$ are homotopic to each other in $\tilde G\backslash\{\tilde v\}$ and therefore in $\tilde{\spc{Q}}\backslash \{\tilde v\}$.

Note that the map $\tilde{\spc{Q}}\backslash \{\tilde v\}\to \spc{S}$
which returns the direction of $[{\tilde v}{x}]$  for any $x\ne \tilde v$, maps $C_R$ to a circle homotopic to~$\gamma$.
Therefore $C_R$ is not contractible in $\tilde{\spc{Q}}\backslash \{\tilde v\}$.

If $R$ is large, 
the circle $C_R$  
lies outside of any fixed compact set $K'$ in~$\tilde{\spc{Q}}$.
From above $C_R$ is not contractible in $\tilde{\spc{Q}}\backslash K$
if $K\supset \tilde v$.
It follows that $\tilde{\spc{Q}}$ is not simply connected at infinity, a contradiction.
\qeds

The proof of the following exercise is analogous.
It will be used later in the proof of Proposition~\ref{prop:loc-CAT-mnfld} --- a more geometric version of Proposition~\ref{prop:aspherical}.

\begin{thm}{Exercise}\label{ex:example-pi_infty-new}
Under the assumptions of Lemma~\ref{lem:example-pi_infty}, 
for any vertex $v$ in $\spc{S}$
the complement $\spc{S}\backslash\{v\}$ is simply connected.
\end{thm}

\parit{Proof of \ref{prop:aspherical}.}
Let $\Sigma^{m-1}$ be an $(m-1)$-dimensional smooth homology sphere that  is not simply connected, and bounds a contractible smooth compact $m$-dimensional manifold~$\spc{W}$. 

For $m\ge 5$ the existence of such $(\spc{W}, \Sigma)$ follows from \cite{kervaire}. 
For $m=4$ it follows from the construction in \cite{mazur}.

Pick any triangulation $\tau$ of $W$ and let $\spc{S}$ be the resulting subcomplex that triangulates~$\Sigma$.


We can assume that $\spc{S}$ is flag; 
otherwise pass to the barycentric subdivision 
of $\tau$ and apply Exercise~\ref{ex:baricenric-flag}.


Let $\spc{Q}=\square_{\spc{S}}$ be the cubical analog of~$\spc{S}$.

By Proposition~\ref{prob:cubical-analog},
$\spc{Q}$ is a homology manifold.
It follows that $\spc{Q}$ is a piecewise linear manifold 
with a finite number of singularities at its vertices.


Removing a small contractible neighborhood $V_v$ of each vertex $v$ in $\spc{Q}$,
we can obtain a piecewise linear manifold $\spc{N}$
whose boundary is formed by several copies of~$\Sigma$.

Let us glue a copy of  $\spc{W}$ along its boundary to each copy of $\Sigma$ in the boundary of~$\spc{N}$.
This results in a  closed piecewise linear manifold 
$\spc{M}$ which is homotopically equivalent to~$\spc{Q}$.

Indeed, since both $V_v$ and $\spc{W}$ are contractible, the identity map of  their common boundary $\Sigma$ can be extended to a homotopy equivalence $V_v\to\spc{W}$ relative to the boundary.
Therefore the identity map on $\spc{N}$ extends to homotopy equivalences 
$f\: \spc Q\to \spc M$ and $g\:\spc M\to \spc Q$.

Finally, by Lemma~\ref{lem:example-pi_infty},  
the universal cover $\tilde{\spc{Q}}$ of $\spc{Q}$
is not simply connected at infinity.

The same holds for 
the universal cover $\tilde{\spc{M}}$ of $\spc{M}$.
The latter follows since the constructed homotopy equivalences 
$f\: \spc Q\to \spc M$ and $g\:\spc M\z\to \spc Q$ 
lift to {}\emph{proper maps} 
$\tilde f \: \tilde{\spc{Q}}\to \tilde{\spc{M}}$
and $\tilde g \: \tilde{\spc{M}}\to \tilde{\spc{Q}}$;
that is, for any compact sets $A\subset \tilde{\spc{Q}}$ and $B\subset\tilde{\spc{M}}$, the inverse images $\tilde g^{-1}(A)$ and $\tilde f^{-1}(B)$ are compact.
\qeds


The following proposition was proved by
Fredric Ancel, 
Michael Davis,
and Craig Guilbault \cite{ancel-davis-guilbault};
it could be considered as a more geometric version of Proposition~\ref{prop:aspherical}.


\begin{thm}{Proposition}\label{prop:loc-CAT-mnfld}
Given $m\ge 5$, there is a Euclidean polyhedral space $\spc{P}$ such that:
\begin{subthm}{}
$\spc{P}$ is homeomorphic to a closed $m$-dimensional manifold.
\end{subthm}

\begin{subthm}{}
$\spc{P}$ is locally $\CAT(0)$.
\end{subthm}

\begin{subthm}{}
The universal cover of $\spc{P}$ is not simply connected at infinity.
\end{subthm}
\end{thm}

There are no three-dimensional examples of that type;
see \cite{rolfsen} by Dale Rolfsen.
In \cite{thurston}, Paul Thurston conjectured that the same holds in the four-dimensional case.

\parit{Proof.}
Apply Exercise~\ref{ex:example-pi_infty-new} to the barycentric subdivision of the simplicial complex $\spc{S}$ provided by Exercise~\ref{ex:funny-S}.
\qeds

\begin{thm}{Exercise}\label{ex:funny-S}
Given an integer $m\ge 5$,
construct a finite $(m-1)$-dimensional simplicial complex $\spc{S}$ such that $\Cone\spc{S}$ is homeomorphic to $\EE^m$
and $\pi_1(\spc{S}\backslash\{v\})\ne0$ for some vertex $v$ in~$\spc{S}$.
\end{thm} 

\section{Remarks}

As was mentioned earlier, the motivation for the notion of $\CAT(\kappa)$ spaces comes from the fact that a Riemannian manifold is locally  $\CAT(\kappa)$ if and only if it has $\sec\le\kappa$.
This easily follows from Rauch comparison for Jacobi fields and Proposition~\ref{prop:thin=cat}.

In the globalization theorem (\ref{thm:hadamard-cartan}), properness can be weakened to completeness; see our book \cite{alexander-kapovitch-petrunin-2025} and the references therein.

The condition on polyhedral $\CAT(\kappa)$ spaces given in Theorem~\ref{thm:PL-CAT} might look easy to use, 
but in fact, it is hard to check even in very simple cases.
For example the description of those coverings of $\mathbb{S}^3$ branching at three 
great circles which are $\CAT(1)$ requires quite a bit of work;
see \cite{charney-davis-1993} --- try to guess the answer before reading.

Another example is the space $\spc{B}_4$ that  is the universal cover of $\CC^4$ infinitely branching in six complex planes $z_i=z_j$ with the induced length metric.
So far it is not known if $\spc{B}_4$ is $\CAT(0)$ \cite{panov-petrunin-2016}.
Understanding this space could be helpful for studying the braid group on 4 strings.
This circle of questions is closely related to the generalization of the flag condition (\ref{thm:flag}) to  spherical simplices with few acute dihedral angles.


The construction used in the proof of  Proposition~\ref{prop:aspherical} admits a number of interesting modifications,  
several of which are discussed in the survey \cite{davis-2001} by Michael Davis.

A similar argument was used by Michael Davis, 
Tadeusz Ja\-nu\-szkie\-wicz,
and  Jean-Fran\c{c}ois Lafont in \cite{davis-januszkiewicz-lafont}.
They constructed a closed smooth four-dimensional manifold $M$ with  universal cover $\tilde M$ diffeomorphic to $\RR^4$, such that $M$ admits a polyhedral metric which is locally $\CAT(0)$, but does not admit a Riemannian metric with nonpositive sectional curvature.
Another example of that type was constructed by Stephan Stadler; see~\cite{stadler}.
There are no lower dimensional examples of this type ---
the two-dimensional case follows from the  classification of surfaces,
and 
the three-dimensional case follows from the geometrization conjecture.

It is noteworthy that any complete, simply connected Riemannian manifold with nonpositive curvature is homeomorphic to the Euclidean space of the same dimension.
In fact, by the globalization theorem
(\ref{thm:hadamard-cartan}), 
the exponential map at a point of such a manifold is a homeomorphism.
In particular, there is no Riemannian analog of Proposition~\ref{prop:loc-CAT-mnfld}.

Recall that a triangulation of an $m$-dimensional manifold defines a piecewise linear structure if the link  of every simplex $\Delta$ is homeomorphic to the sphere of dimension $m-1-\dim\Delta$.
According to Stone's theorem, see \cite{stone, davis-januszkiewicz}, the triangulation of $\spc{P}$ in Proposition~\ref{prop:loc-CAT-mnfld} 
cannot be made piecewise linear --- despite  the fact that $\spc{P}$ is a manifold, its triangulation does not induce a piecewise linear structure.

The flag condition also leads to the so-called {}\emph{hyperbolization} procedure, a flexible tool for constructing  aspherical spaces;
a good survey on the subject is given by Ruth Charney and Michael Davis in \cite{charney-davis-1995}.

The $\CAT(0)$ property of a cube complex admits interesting (and useful) geometric descriptions if one exchanged the $\ell^2$-metric to a natural $\ell^1$ or $\ell^\infty$ on each cube.

\begin{thm}{Theorem}
The following three conditions are equivalent.

\begin{subthm}{cube-2} A cube complex $Q$ equiped with $\ell^2$-metric is $\CAT(0)$.
\end{subthm}

\begin{subthm}{cube-infty} A cube complex $Q$ equiped with $\ell^\infty$-metric is injective.
\end{subthm}

\begin{subthm}{cube-1} A cube complex $Q$ equiped with $\ell^1$-metric is \index{median space}\emph{median}.
 The later means that for any three points $x,y,z$ there is a {}\emph{unique} point   $m$ (it is called \index{median}\emph{median} of $x$, $y$, and $z$) that lies on {}\emph{some} geodesics $[xy]$, $[xz]$ and $[yz]$.
\end{subthm}
\end{thm}

A very readable paper on the subject was written by Brian Bowditch \cite{bowditch-2020}; two easy parts of the theorem are included in the following exercise.

\begin{thm}{Exercise}\label{ex:cube-infty=>cube-2} Prove the implication \ref{SHORT.cube-infty}$\Rightarrow$\ref{SHORT.cube-2} and/or \ref{SHORT.cube-1}$\Rightarrow$\ref{SHORT.cube-2} in the theorem.
\end{thm}


All the topics discussed in this lecture link Alexandrov geometry with the fundamental group.
The theory of {}\emph{hyperbolic groups}, 
a branch of {}\emph{geometric group theory}, 
introduced by 
Mikhael Gromov \cite{gromov-1987},
could be considered as a further step in this direction.



