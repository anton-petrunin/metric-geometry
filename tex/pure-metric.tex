\RequirePackage{snapshot}
\documentclass[twoside]{book}

\usepackage{lectures}
\usepackage[colorlinks=true,
citecolor=black,
linkcolor=black,
anchorcolor=black,
filecolor=black,
menucolor=black,
urlcolor=black,
pdftitle={Lectures on pure metric geometry},
pdfsubject={Geometry},
pdfauthor={Anton Petrunin}
]{hyperref}
\makeindex

\begin{document}
 
\title{Lectures on pure metric geometry}
\author{Anton Petrunin}
\date{}
\maketitle


After reviewing basic definitions in metric geometry we discuss 
Uryson's \emph{universal space},
\emph{injective spaces} that are metric analogs of convex sets,
an \emph{ultralimits} of metric spaces that provides a very general way to pass to a limit without any assumptions on the sequence of spaces.
These topics are simple, usefull, and conceptual,
but they are not included in the (today the only) textbook in metric geometry written by Dmitry Burago, Yuri Burago, and Sergey Ivanov \cite{burago-burago-ivanov}.

These a lectures are taken from a graduate course given at Penn State, Spring 2020.
The complete lecture notes (14 lectures) can be found on the authors website.
Considerable part of the text is a compilation from \cite{alexander-kapovitch-petrunin-2019, alexander-kapovitch-petrunin-2025, petrunin-yashinski, petrunin-2009, petrunin-zamorabarrera} and its drafts.

\thispagestyle{empty}
\tableofcontents
\thispagestyle{empty}

\chapter{Definitions}

In this lecture we give some conventions used further
and remind some the definitions related to metric spaces.


\section{Metric spaces}
\label{sec:metric spaces}

The distance between two points $x$ and $y$ in a metric space $\spc{X}$ will be denoted by $\dist{x}{y}{}$ or $\dist{x}{y}{\spc{X}}$.
The latter notation is used if we need to emphasize 
that the distance is taken in the space~${\spc{X}}$.

Let us recall the definition of metric. 

\begin{thm}{Definition}\label{def:metric}
A \index{metric}\emph{metric} on a set $\spc{X}$ is a real-valued function $(x,y)\mapsto\dist{x}{y}{\spc{X}}$ that satisfies the following conditions for any three points $x,y,z\in \spc{X}$:
\begin{enumerate}[(i)]
\item $\dist{x}{y}{\spc{X}}\ge 0$,
\item\label{metric=0} $\dist{x}{y}{\spc{X}}= 0$ $\iff$ $x=y$,
\item $\dist{x}{y}{\spc{X}}=\dist{y}{x}{\spc{X}}$,
\item $\dist{x}{y}{\spc{X}}+\dist{y}{z}{\spc{X}}\ge\dist{x}{z}{\spc{X}}$,
\end{enumerate}
\end{thm}

A set $\spc{X}$ with a metric on it is called \index{metric space}\emph{metric space};
most of the time we keep the same notation for the metric space and its underlying set.

The function 
\[\distfun_x\:y\mapsto \dist{x}{y}{}\]
is called the \index{distance function}\emph{distance function} from~$x$. 

Given $R\in[0,\infty]$ and $x\in \spc{X}$, the sets
\begin{align*}
\oBall(x,R)&=\{y\in \spc{X}\mid \dist{x}{y}{}<R\},
\\
\cBall[x,R]&=\{y\in \spc{X}\mid \dist{x}{y}{}\le R\}
\end{align*}
are called, respectively, the  \index{open ball}\emph{open} and  the \index{closed ball}\emph{closed  balls}   of radius $R$ with center~$x$.
Again, if we need to emphasize that these balls are taken in the metric space $\spc{X}$,
we write 
\[\oBall(x,R)_{\spc{X}}\quad\text{and}\quad\cBall[x,R]_{\spc{X}}.\]

\begin{thm}{Exercise}
Show that
\[\dist{p}{q}{\spc{X}}+\dist{x}{y}{\spc{X}}\le\dist{p}{x}{\spc{X}}+\dist{p}{y}{\spc{X}}+\dist{q}{x}{\spc{X}}+\dist{q}{y}{\spc{X}}\]
for any points $p$, $q$, $x$, and $y$ in a metric space $\spc{X}$.
\end{thm}

\section{Variations of definition}

\parbf{Pseudometrics.}
A metric for which the distance between two distinct points can be zero is called a \index{pseudometric}\emph{pseudometric}.
In other words, to define pseudometric, we need to remove condition (\ref{metric=0}) from \ref{def:metric}.

The following observation show that
nearly any question about pseudometric spaces can be reduced to a question about genuine metric spaces.

Assume $\spc{X}$ is a pseudometric space.
Consider an equivalence relation $\sim$ on $\spc{X}$ defined by
$x\sim y$ if and only if $\dist{x}{y}{}=0$. 
Note that if $x\sim x'$, then $\dist{y}{x}{}=\dist{y}{x'}{}$ for any $y\in\spc{X}$.
Thus, $\dist{*}{*}{}$ defines a metric on the
quotient set $\spc{X}/{\sim}$.
This way we obtain a metric space $\spc{X}'$.
The space $\spc{X}'$ is called the 
\emph{corresponding metric space} for the pseudometric space $\spc{X}$.
Often we do not distinguish between $\spc{X}'$ and~$\spc{X}$. 

\parbf{$\bm{\infty}$-metrics.}
One may also consider metrics with values in $\RR\cup\{\infty\}$;
we might call them \index{metric!$\infty$-metric}\emph{$\infty$-metrics}, but most of the time we use the term {}\emph{metric}.

Again nearly any question about $\infty$-metric spaces can be reduced to a question about genuine metric spaces. 

Indeed, let us write $x\approx y$ if  $\dist{x}{y}{}<\infty$;
this is another equivalence relation on $\spc{X}$.
The equivalence class of a point $x\in\spc{X}$ will be called the \index{metric component}\emph{metric component} 
 of $x$; it will be denoted by $\spc{X}_x$.
One could think of $\spc{X}_x$ as  $\oBall(x,\infty)_{\spc{X}}$ --- the open ball centered at $x$ and radius $\infty$ in $\spc{X}$.

It follows that any $\infty$-metric space is a {}\emph{disjoint union} of genuine metric spaces --- the metric components of the original $\infty$-metric space.

\begin{thm}{Exercise}
Given two sets $A$ and $B$ on the plane, set 
\[\dist{A}{B}{}=\mu(A\backslash B)+\mu(B\backslash A),\]
where $\mu$ denotes the Lebesgue measure.
\begin{subthm}{}
Show that $\dist{*}{*}{}$ is a pseudometric on the set of bounded measurable sets of the plane.
\end{subthm}

\begin{subthm}{}
Show that $\dist{*}{*}{}$ is an $\infty$-metric on the set of all open sets of the plane.
\end{subthm}
\end{thm}

\section{Completeness}

A metric space $\spc{X}$ is called \index{complete space}\emph{complete} if every Cauchy sequence of points in $\spc{X}$ converges in $\spc{X}$.

\begin{thm}{Exercise}\label{ex:almost-min}
Suppose that $\rho$ is a positive continuous function on a complete metric space $\spc{X}$.
Show that for any $\eps>0$ there is a point $x\in \spc{X}$ such that 
\[\rho(x)<(1+\eps)\cdot\rho(y)\]
for any point $y\in \oBall(x,\rho(x))$.
\end{thm}

Most of the time we will assume that a metric space is complete.
The following construction produces a complete metric space $\bar{\spc{X}}$ for any given metric space $\spc{X}$.


\parbf{Completion.}
Given a metric space $\spc{X}$, 
consider the set $\spc{C}$ of all Cauchy sequences in $\spc{X}$.
Note that for any two Cauchy sequences $(x_n)$ and $(y_n)$ the right hand side in \ref{eq:cauchy-dist} is defined; moreover it defines a pseudometric on~$\spc{C}$
\[\dist{(x_n)}{(y_n)}{\spc{C}}\df\lim_{n\to\infty}\dist{x_n}{y_n}{\spc{X}}.\eqlbl{eq:cauchy-dist}\]
The corresponding metric space $\bar{\spc{X}}$ is called a \index{completion}\emph{completion} of $\spc{X}$.

Note that the original space $\spc{X}$ forms a dense subset in its completion $\bar{\spc{X}}$.
More precisely,  for each point $x\in\spc{X}$ one can consider a constant sequence $x_n=x$ which is Cauchy.
It defines a natural map $\spc{X}\to \bar{\spc{X}}$.
It is easy to check that this map is distance-preserving.
In partucular we can (and will) consider $\spc{X}$ as a subset of $\bar{\spc{X}}$.

\begin{thm}{Exercise}
Show that completion of a metric space is complete.
\end{thm}


\section{Compact spaces}

Let us recall few equivalent definitions of compact metric spaces.

\begin{thm}{Definition}\label{def:compact}
A metric space $\spc{K}$ is compact if and only if one of the following equivalent condition holds:

\begin{subthm}{}
 Every open cover of $\spc{K}$ has a finite subcover.
\end{subthm}

\begin{subthm}{}
 For any open cover of $\spc{K}$ there is $\eps>0$ such that any $\eps$-ball in $\spc{K}$ lie in one element of the cover. (The value $\eps$ is called a \index{Lebesgue number}\emph{Lebesgue number} of the covering.)
\end{subthm}

\begin{subthm}{}
 Every sequence of points in $\spc{K}$ has a subsequence that converges in $\spc{K}$.
\end{subthm}

\begin{subthm}{totally-bounded}
The space $\spc{K}$ is complete and \index{totally bounded space}\emph{totally bounded}; that is, for any $\eps>0$, the space $\spc{K}$ admits a finite cover by open $\eps$-balls.
\end{subthm}

\end{thm}

A subset $N$ of a metric space $\spc{K}$ is called \index{net}\emph{$\eps$-net} if any other point $x$ lies on the distance less than $\eps$ from a point in $N$.
Note that totally bounded spaces can be defined as spaces that admit a finite $\eps$-net for any $\eps>0$.

\begin{thm}{Exercise}\label{ex:compact-net}
Show that a space $\spc{K}$ is totally bounded if and only if it contains a compact $\eps$-net for any $\eps>0$. 
\end{thm}


Let $\pack_\eps\spc{X}$ be exact upper bound on the number of points $x_1,\z\dots,x_n\in \spc{X}$ such that $\dist{x_i}{x_j}{}\ge\eps$ if $i\ne j$.

If $n=\pack_\eps\spc{X}<\infty$, then
the collection of points $x_1,\dots,x_n$ is called a \index{maximal packing}\emph{maximal $\eps$-packing}.
Note that $n$ is the maximal number of open disjoint $\tfrac\eps2$-balls in $\spc{X}$.

\begin{thm}{Exercise}\label{ex:pack-net}
Show that a complete space $\spc{X}$ is compact if and only of $\pack_\eps\spc{X}\z<\infty$ for any $\eps>0$.

Show that any maximal $\eps$-packing is an $\eps$-net.
\end{thm}


\begin{thm}{Exercise}\label{ex:non-contracting-map}
Let $\spc{K}$  be a compact metric space and
\[f\:\spc{K}\z\to \spc{K}\] 
be a distance-nondecreasing map.
Prove that $f$ is an \index{isometry}\emph{isometry};
that is, $f$ is a distance-preserving bijection.
\end{thm}

A metric space $\spc{X}$ is called \index{locally compact space}\emph{locally compact} if any point in $\spc{X}$ admits a compact neighborhood;
in other words, for any point $x\in\spc{X}$ a closed ball $\cBall[x,r]$ is compact for some $r>0$.

\section{Proper spaces}

A metric space $\spc{X}$ is called \index{proper space}\emph{proper} if all closed bounded sets in $\spc{X}$ are compact. 
This condition is equivalent to each of the following statements:
\begin{itemize}
\item For some (and therefore any) point $p\in \spc{X}$ and any $R<\infty$, 
the closed ball $\cBall[p,R]_{\spc{X}}$ is compact. 
\item The function $\distfun_p\:\spc{X}\to\RR$ is \index{proper function}\emph{proper} for some (and therefore any) point $p\in \spc{X}$;
that is, for any compact set $K\subset \RR$, its inverse image 
\[\distfun_p^{-1}(K)=\set{x\in \spc{X}}{\dist{p}{x}{\spc{X}}\in K}\]
is compact.
\end{itemize}

\begin{thm}{Exercise}\label{ex:loc-compact-not-proper}
Give an example of space which is locally compact but not proper.
\end{thm}

\section{Geodesics}
\label{sec:geods}

Let $\spc{X}$ be a metric space 
and $\II$\index{$\II$} a real interval. 
A~globally isometric map $\gamma\:\II\to \spc{X}$ is called a \index{geodesic}\emph{geodesic}%
\footnote{Various authors call it differently: {}\emph{shortest path}, {}\emph{minimizing geodesic}.
Also note that the meaning of the term \emph{geodesic} is different from what is used in Riemannian geometry, altho they are closely related.}; 
in other words, $\gamma\:\II\to \spc{X}$ is a geodesic if 
\[\dist{\gamma(s)}{\gamma(t)}{\spc{X}}=|s-t|\]
for any pair $s,t\in \II$.

If $\gamma\:[a,b]\to \spc{X}$ is a geodesic and $p=\gamma(a)$, $q=\gamma(b)$, then we say that $\gamma$ is a geodesic from point $p$ to point $q$.
In this case the image of $\gamma$ is denoted by $[p q]$\index{$[{*}{*}]$} and with an abuse of notations  we also call it a \index{geodesic}\emph{geodesic}.


We may write $[p q]_{\spc{X}}$ 
to emphasize that the geodesic $[p q]$ is in the space  ${\spc{X}}$.
We also use the following shortcut notation:
\begin{align*}
\left] p q \right[&=[pq]\backslash\{p,q\},
&
\left] p q \right]&=[pq]\backslash\{p\},
&
\left[ p q \right[&=[pq]\backslash\{q\}.
\end{align*}

In general, a geodesic from $p$ to $q$ need not exist and if it exists, it need not  be unique.  
However, once we write $[p q]$ we assume mean that we have made a choice of geodesic.

A \index{geodesic path}\emph{geodesic path} is a geodesic with constant-speed parametrization by $[0,1]$.

A curve $\gamma\:\II\to \spc{X}$  is called a \index{geodesic!local geodesic}\emph{local geodesic} if for any $t\in\II$ there is a neighborhood $U$ of $t$ in $\II$ such that the restriction $\gamma|_U$ is a  geodesic.
A constant-speed parametrization of a local geodesic by the unit interval $[0,1]$ is called a \index{geodesic!local geodesic}\emph{local geodesic path}. 

\section{Geodesic spaces and metric trees}

A metric space is called \index{geodesic}\emph{geodesic} if any pair of its points can be joined by a geodesic.

A geodesic space $\spc{T}$ is called a \index{metric tree}\emph{metric tree} if any pair of points in $\spc{T}$ are connected by a unique geodesic,
and the union of any two geodesics $[xy]$, and $[yz]$ contain the geodesic $[xz]_{\spc{T}}$.
In other words any triangle in $\spc{T}$ is a tripod;
that is, for any three geodesics $[xy]$, $[yz]$, and $[zx]$ have a common point.

\begin{thm}{Exercise}
Show that spheres in metric trees are ultrametric spaces;
that is, if $\Sigma$ is a sphere in a metric tree $\spc{T}$, then
\[\dist{x}{z}{\spc{T}}
\le
\max\{\,\dist{x}{y}{\spc{T}},\dist{y}{z}{\spc{T}}\,\}\]
for any $x,y,z\in\Sigma$.
\end{thm}

\section{Length}

A \index{curve}\emph{curve} is defined as a continuous map from a real interval to a metric space.
If the real interval is $[0,1]$, then the curve is called a \index{path}\emph{path}.

\begin{thm}{Definition}
Let $\spc{X}$ be a metric space and
$\alpha\: \II\to \spc{X}$ be a curve.
We define the \index{length}\emph{length} of $\alpha$ as 
\[
\length \alpha \df \sup_{t_0\le t_1\le\ldots\le t_n}\sum_i \dist{\alpha(t_i)}{\alpha(t_{i-1})}{}.
\]

A curve $\alpha$ is called \index{rectifiable curve}\emph{rectifiable} if $\length \alpha<\infty$.
\end{thm}



\begin{thm}{Theorem}\label{thm:length-semicont}
Length is a lower semi-continuous with respect to pointwise convergence of curves. 

More precisely, assume that a sequence
of curves $\gamma_n\:\II\to \spc{X}$ in a metric space $\spc{X}$ converges pointwise 
to a curve $\gamma_\infty\:\II\to \spc{X}$;
that is, for any fixed $t \in \II$, $\gamma_n(t)\z\to\gamma_\infty(t)$ as $n\to\infty$. 
Then 
$$\liminf_{n\to\infty} \length\gamma_n \ge \length\gamma_\infty.\eqlbl{eq:semicont-length}$$
\end{thm}


\begin{wrapfigure}{o}{20 mm}
\vskip-0mm
\centering
\includegraphics{mppics/pic-100}
\end{wrapfigure}


Note that the inequality \ref{eq:semicont-length} might be strict.
For example the diagonal $\gamma_\infty$ of the unit square 
can be  approximated by a stairs-like
polygonal curves $\gamma_n$
with sides parallel to the sides of the square ($\gamma_6$ is on the picture).
In this case
\[\length\gamma_\infty=\sqrt{2}\quad
\text{and}\quad \length\gamma_n=2\]
for any $n$.

\parit{Proof.}
Fix a sequence $t_0<t_1<\dots<t_k$ in $\II$.
Set 
\begin{align*}\Sigma_n
&\df
|\gamma_n(t_0)-\gamma_n(t_1)|+\dots+|\gamma_n(t_{k-1})-\gamma_n(t_k)|.
\\
\Sigma_\infty
&\df
|\gamma_\infty(t_0)-\gamma_\infty(t_1)|+\dots+|\gamma_\infty(t_{k-1})-\gamma_\infty(t_k)|.
\end{align*}

Note that for each $i$ we have 
\[|\gamma_n(t_{i-1})-\gamma_n(t_i)|\to|\gamma_\infty(t_{i-1})-\gamma_\infty(t_i)|\]
and therefore
\[\Sigma_n\to \Sigma_\infty\] 
as $n\to\infty$.
Note that 
\[\Sigma_n\le\length\gamma_n\]
for each $n$.
Hence
$$\liminf_{n\to\infty} \length\gamma_n \ge \Sigma_\infty.\eqlbl{>=Sigma-infty}$$

If $\gamma_\infty$ is rectifiable, we can assume that 
\begin{align*}
\length\gamma_\infty<\Sigma_\infty+\eps.
\end{align*}
for any given $\eps>0$.
By \ref{>=Sigma-infty} it follows that 
$$\liminf_{n\to\infty} \length\gamma_n > \length\gamma_\infty-\eps$$
for any $\eps>0$; whence \ref{eq:semicont-length} follows.

It remains to consider the case when $\gamma_\infty$ is not rectifiable; 
that is, $\length\gamma_\infty=\infty$.
In this case we can choose a partition so that $\Sigma_\infty>L$ for any real number $L$.
By \ref{>=Sigma-infty} it follows that 
$$\liminf_{n\to\infty} \length\gamma_n > L$$
for any given $L$; whence 
\[\liminf_{n\to\infty}\length\gamma_n=\infty\]
and \ref{eq:semicont-length} follows.
\qeds

\section{Length spaces}\label{sec:intrinsic}

If for any $\eps>0$ and any pair of points $x$ and $y$ in a metric space $\spc{X}$, there is a path $\alpha$ connecting $x$ to $y$ such that
\[\length\alpha< \dist{x}{y}{}+\eps,\]
then $\spc{X}$ is called a \index{length space}\emph{length space} and the metric on $\spc{X}$ is called a \index{length metric}\emph{length metric}.\label{page:length metric}

If $\spc{X}$ is an $\infty$-metric space, then in the above definition we assume in addition that $x$ and $y$ lie in one metric component; that is, $\dist{x}{y}{\spc{X}}<\infty$.
In other words an $\infty$-metric space $\spc{X}$ is a length space if each metric component of $\spc{X}$ is a length space.

Note that any geodesic space is a length space.
The following example shows that the converse does not hold.


\begin{thm}{Example}
Suppose a space $\spc{X}$ is obtained by gluing a countable collection of disjoint intervals $\{\II_n\}$ of length $1+\tfrac1n$, where for each $\II_n$ the left end is glued to $p$ and the right end to~$q$.

Observe that the space $\spc{X}$ carries a natural complete length metric with respect to which $\dist{p}{q}{}=1$ but there is no geodesic connecting $p$ to~$q$.
\end{thm}



\begin{thm}{Exercise}\label{ex:no-geod}
Give an example of a complete length space $\spc{X}$ such that no pair of distinct points in $\spc{X}$ can be joined by a geodesic.
\end{thm}

Directly from the definition, it follows that if a path $\alpha\:[0,1]\to\spc{X}$ connects two points $x$ and $y$ 
(that is, if $\alpha(0)=x$ and $\alpha(1)=y$), then 
\[\length\alpha\ge \dist{x}{y}{}.\]
Set 
\[\yetdist{x}{y}{}=\inf\{\length\alpha\}\]
where the greatest lower bound is taken for all paths connecing $x$ and~$y$.
It is straightforward to check that $(x,y)\mapsto \yetdist{x}{y}{}$ is an $\infty$-metric; moreover $(\spc{X},\yetdist{*}{*}{})$ is a length space.
The metric $\yetdist{*}{*}{}$ is called \index{induced length metric}\emph{induced length metric}.

\begin{thm}{Exercise}\label{ex:compact+connceted}
Let $\spc{X}$ be a complete length space.
Show that for any compact subset $K$ in $\spc{X}$
there is a compact path connected subset $K'$ that contains $K$.  
\end{thm}

\begin{thm}{Exercise}\label{ex:compact=>complete}
Suppose $(\spc{X},\dist{*}{*}{})$ is a complete metric space.
Show that $(\spc{X},\yetdist{*}{*}{})$ is complete.
\end{thm}

Let $A$ be a subset of a metric space $\spc{X}$.
Given two points $x,y\in A$,
consider the value
\[\dist{x}{y}{A}=\inf_{\alpha}\{\length\alpha\},\]
where the greatest lower bound is taken for all paths $\alpha$ from $x$ to $y$ in~$A$.
In other words $\dist{*}{*}{A}$ denotes the induced length metric on the subspace $A$.%
\footnote{The notation $\dist{*}{*}{A}$ conflicts with the previously defined notation for distance $\dist{x}{y}{\spc{X}}$ in a metric space $\spc{X}$. However, most of the time we will work with ambient length spaces where the meaning will be unambiguous.}

Let $\spc{X}$ be a metric space and $x,y\in\spc{X}$.

\begin{enumerate}[(i)]
\item A point $z\in \spc{X}$ is called a \index{midpoint}\emph{midpoint} between $x$ and $y$
if 
\[\dist{x}{z}{}=\dist{y}{z}{}=\tfrac12\cdot\dist[{{}}]{x}{y}{}.\]
\item Assume $\eps\ge 0$.
A point $z\in \spc{X}$ is called an \index{$\eps$-midpoint}\emph{$\eps$-midpoint} between $x$ and $y$
if 
\[\dist{x}{z}{},\quad\dist{y}{z}{}\le\tfrac12\cdot\dist[{{}}]{x}{y}{}+\eps.\]
\end{enumerate}


Note that a $0$-midpoint is the same as a midpoint.


\begin{thm}{Lemma}\label{lem:mid>geod}
Let $\spc{X}$ be a complete metric space.
\begin{subthm}{lem:mid>length}
Assume that for any pair of points $x,y\in \spc{X}$  
 and any $\eps>0$
there is an $\eps$-midpoint~$z$.
Then $\spc{X}$ is a length space.
\end{subthm}

\begin{subthm}{lem:mid>geod:geod}
Assume that for any pair of points $x,y\in \spc{X}$, 
there is a midpoint~$z$.
Then $\spc{X}$ is a geodesic space.
\end{subthm}
\end{thm}

\parit{Proof.}
We first prove \ref{SHORT.lem:mid>length}.
Let $x,y\in \spc{X}$ be a pair of points.

Set $\eps_n=\frac\eps{4^n}$, $\alpha(0)=x$ and $\alpha(1)=y$.

Let $\alpha(\tfrac12)$ be an $\eps_1$-midpoint between $\alpha(0)$ and $\alpha(1)$.
Further, let $\alpha(\frac14)$ 
and $\alpha(\frac34)$ be $\eps_2$-midpoints between the pairs $(\alpha(0),\alpha(\tfrac12))$ 
and $(\alpha(\tfrac12),\alpha(1))$ respectively.
Applying the above procedure recursively,
on the $n$-th step we define $\alpha(\tfrac{k}{2^n})$,
for every odd integer $k$ such that $0<\tfrac k{2^n}<1$, 
as an $\eps_{n}$-midpoint between the already defined
$\alpha(\tfrac{k-1}{2^n})$ and $\alpha(\tfrac{k+1}{2^n})$.


In this way we define $\alpha(t)$ for $t\in W$,
where $W$ denotes the set of dyadic rationals in $[0,1]$.
Since $\spc{X}$ is complete, the map $\alpha$ can be extended continuously to $[0,1]$.
Moreover,
\[\begin{aligned}
\length\alpha&\le \dist{x}{y}{}+\sum_{n=1}^\infty 2^{n-1}\cdot\eps_n\le
\\
&\le \dist{x}{y}{}+\tfrac\eps2.
\end{aligned}
\eqlbl{eq:eps-midpoint}
\]
Since $\eps>0$ is arbitrary, we get \ref{SHORT.lem:mid>length}.

To prove \ref{SHORT.lem:mid>geod:geod}, 
one should repeat the same argument 
taking midpoints instead of $\eps_n$-midpoints.
In this case \ref{eq:eps-midpoint} holds for $\eps_n=\eps=0$.
\qeds

Since in a compact space a sequence of $\tfrac1n$-midpoints $z_n$ contains a convergent subsequence, Lemma~\ref{lem:mid>geod} immediately implies

\begin{thm}{Proposition}\label{prop:length+proper=>geodesic}
Any proper length space is geodesic.
\end{thm}

\begin{thm}{Hopf--Rinow theorem}\label{thm:Hopf-Rinow}
Any complete, locally compact length space is proper.
\end{thm}

Before reading the proof, it is instructive to solve \ref{ex:loc-compact-not-proper}.

\parit{Proof.}
Let $\spc{X}$ be a locally compact length space.
Given $x\in \spc{X}$, denote by $\rho(x)$ the supremum of all $R>0$ such that
the closed ball $\cBall[x,R]$ is compact.
Since $\spc{X}$ is locally compact, 
$$\rho(x)>0
\quad\text{for any}\quad
x\in \spc{X}.\eqlbl{eq:rho>0}$$
It is sufficient to show that $\rho(x)=\infty$ for some (and therefore any) point $x\in \spc{X}$.

\begin{clm}{} If $\rho(x)<\infty$, then $B=\cBall[x,\rho(x)]$ is compact.
\end{clm}

Indeed, $\spc{X}$ is a length space;
therefore for any $\eps>0$, 
the set $\cBall[x,\rho(x)-\eps]$ is a compact $\eps$-net in~$B$.
Since $B$ is closed and hence complete, it must be compact.
\claimqeds
Next we claim that
\begin{clm}{} $|\rho(x)-\rho(y)|\le \dist{x}{y}{\spc{X}}$ for any $x,y\in \spc{X}$;
in particular $\rho\:\spc{X}\to\RR$ is a continuous function.
\end{clm}

Indeed, 
assume the contrary; that is, $\rho(x)+|x-y|<\rho(y)$ for some $x,y\in \spc{X}$. 
Then 
$\cBall[x,\rho(x)+\eps]$ is a closed subset of $\cBall[y,\rho(y)]$ for some $\eps>0$.
Then  compactness of $\cBall[y,\rho(y)]$ implies compactness of $\cBall[x,\rho(x)+\eps]$, a contradiction.\claimqeds

Set $\eps=\min\set{\rho(y)}{y\in B}$; the minimum is defined since $B$ is compact and $\rho$ is continuous.
From \ref{eq:rho>0}, we have $\eps>0$.

Choose a finite $\tfrac\eps{10}$-net $\{a_1,a_2,\dots,a_n\}$ in $B=\cBall[x,\rho(x)]$.
The union $W$ of the closed balls $\cBall[a_i,\eps]$ is compact.
Clearly 
$\cBall[x,\rho(x)+\frac\eps{10}]\subset W$.
Therefore $\cBall[x,\rho(x)+\frac\eps{10}]$ is compact,
a contradiction.
\qeds

\begin{thm}{Exercise}\label{exercise from BH}
Construct a geodesic space $\spc{X}$ that is locally compact,
but whose completion $\bar{\spc{X}}$ is neither geodesic nor locally compact.
\end{thm}

\begin{thm}{Advanced exercise}\label{ex:gross}
Show that for any compact length-metric space $X$ there is number $\ell=\ell(X)$ such that for any finite collection of points there is a point $z$ that lies of average distance $\ell$ from the collection;
that is, for any $x_1,\dots,x_n\in X$ there is $z\in X$ such that
\[\tfrac1n\cdot\sum_i|x_i-z|_X=\ell.\]
\end{thm}







\chapter{Universal spaces}\label{chap:urysohn}

The Urysohn space is the main hero of this lecture.
It shares some fundamental properties with classical spaces (spheres, Euclidean, and Lobachevsky spaces),
but also has many counterintuitive properties.

This space often serves as a counterexample to plausible conjectures,
so it is worth to know it.
In addition, this space is beautiful.



\section{Embedding in a normed space}

Recall that a function $v\mapsto |v|$ on a vector space $\spc{V}$ is called \index{norm}\emph{norm} if it satisfies the following condition for any two vectors $v,w\in \spc{V}$ and a scalar $\alpha$:
\begin{itemize}
\item $|v|\ge 0$;
\item $|\alpha\cdot v|=|\alpha|\cdot |v|$;
\item $|v|+|w|\ge|v+w|$.
\end{itemize}

As an example, consider \index{$\ell^\infty$}$\ell^\infty$ --- the space of real sequences equipped with \index{sup-norm}\emph{sup-norm}; that is, the norm of $\bm{a}=(a_1,a_2,\dots)$ is defined by
\[|\bm{a}|_{\ell^\infty}
\df
\sup_n\{\,|a_n|\,\}.\]


It is straightforward to check that for any normed space the function $(v,w)\mapsto |v-w|$ defines a metric on it.
Therefore, any normed space is an example of metric space;
moreover, it is a geodesic space.
Often we do not distinguish normed space from the corresponding metric space.
(By the Mazur--Ulam theorem, the metric remembers the affine structure of the space; so, to recover the original normed space we only need to specify the origin.
A slick proof of this theorem was given by Jussi V\"{a}is\"{a}l\"{a} \cite{vaisala}.)

Recall that \index{diameter}\emph{diameter} of a metric space $\spc{X}$ (briefly $\diam \spc{X}$) is defined as the least upper bound on the distances between pairs of its points;
that is,
\[\diam \spc{X}
\df
\sup\set{\dist{x}{y}{\spc{X}}}{x,y\in \spc{X}}.\]
If $\diam\spc{X}<\infty$, then the space $\spc{X}$ is called \index{bounded space}\emph{bounded}.



\begin{thm}{Lemma}\label{lem:frechet}
Suppose $\spc{X}$ is a bounded \index{separable space}\emph{separable} metric space;
that is, $\spc{X}$ contains a countable, dense set, say $\{w_n\}$.
Given $x\in \spc{X}$, set $a_n(x)=\dist{w_n}{x}{\spc{X}}$.
Then 
\[\iota\:x\mapsto (a_1(x), a_2(x),\dots)\]
defines a distance-preserving embedding $\iota\:\spc{X}\hookrightarrow \ell^\infty$.
\end{thm}

\parit{Proof.} 
By the triangle inequality 
\[|a_n(x)-a_n(y)|\le \dist{x}{y}{\spc{X}}.\eqlbl{eq:a-a=<dist}\]
Therefore, $\iota$ is \index{short map}\emph{short} (in other words, $\iota$ is distance-expanding).

Again by triangle inequality we have 
\[|a_n(x)-a_n(y)|\ge \dist{x}{y}{\spc{X}}-2\cdot\dist{w_n}{x}{\spc{X}}.\]
Since the set $\{w_n\}$ is dense, we can choose $w_n$ arbitrarily close to $x$.
Whence 
\[\sup_n\{\,|a_n(x)-a_n(y)|\,\}\ge \dist{x}{y}{\spc{X}};\eqlbl{eq:a-a>=dist}\]
that is, $\iota$ is distance-noncontracting.

Finally, observe that \ref{eq:a-a=<dist} and \ref{eq:a-a>=dist} imply the lemma.
\qeds

\begin{thm}{Exercise}\label{ex:compact-length}
Show that any compact metric space $\spc{K}$ is isometric to a subspace of a compact geodesic space. 
\end{thm}

The following exercise generalizes the lemma to arbitrary separable spaces.

\begin{thm}{Exercise}\label{ex:frechet}
Suppose $\{w_n\}$ is a countable, dense set in a metric space $\spc{X}$.
Choose $x_0\in \spc{X}$;
given $x\in \spc{X}$, set 
\[a_n(x)=\dist{w_n}{x}{\spc{X}}-\dist{w_n}{x_0}{\spc{X}}.\]
Show that $\iota\:x\mapsto (a_1(x), a_2(x),\dots)$ defines a distance-preserving embedding $\iota\:\spc{X}\hookrightarrow \ell^\infty$.

Conclude that any separable metric space $\spc{X}$ admits a distance-preserving embedding $\iota\:\spc{X}\hookrightarrow \ell^\infty$.
\end{thm}

The following lemma implies that {}\textit{any metric space is isometric to a subset of a normed vector space};
its proof is nearly identical to the proof of \ref{ex:frechet}.
Given a set $\spc{X}$, denote by \index{$\ell^\infty(\spc{X})$}$\ell^\infty(\spc{X})$ the space of all bounded functions on $\spc{X}$ equipped with sup-norm; 
that is,
\[|f-g|_{\ell^\infty}=\sup\set{|f(x)-f(x)}{x\in \spc{X}}.\]

\begin{thm}{Lemma}\label{lem:kuratowski}
Let $x_0$ be a point in a metric space $\spc{X}$.
Then the map $\iota\:\spc{X}\to \ell^\infty(\spc{X})$ defined by 
\[\iota\:x\mapsto (\distfun_x-\distfun_{x_0})\]
is distance-preserving.

In particular, any metric space $\spc{X}$ admits a distance-preserving into $\ell^\infty(\spc{X})$.
\end{thm}

\section{Extension property}
\label{sec:Extension property}

If a metric space $\spc{X}$ is a subspace of a semimetric space $\spc{X}'$, then we say that $\spc{X}'$ is an \index{extension}\emph{extension} of $\spc{X}$.
If in addition, $\diam\spc{X}'\le d$, then we say that $\spc{X}'$ is a {}\emph{$d$-extension}.

If the complement $\spc{X}'\setminus \spc{X}$ contains a single point, say $p$, then $\spc{X}'$ is called a \index{one-point extension}\emph{one-point extension} of $\spc{X}$.
In this case, to define a metric on $\spc{X}'$, it is sufficient to specify the distance function from $p$; that is, a function $f\:\spc{X}\to\RR$ defined by 
\[f(x)\df\dist{p}{x}{\spc{X}'}.\]
Any function $f$ of that type will be called an \index{extension function}\emph{extension function}\label{page:extension function} or {}\emph{$d$-extension function} respectively.

The extension function $f$ cannot be taken arbitrarily --- the triangle inequality implies that 
\[f(x)+f(y)\ge \dist{x}{y}{\spc{X}}\ge |f(x)-f(y)|\]
for any $x,y\in \spc{X}$.
In particular, $f$ is a non-negative 1-Lipschitz function on $\spc{X}$.
For a $d$-extension, we need to assume in addition that $\diam\spc{X}\z\le d$ and $f(x)\le d$ for any $x\in \spc{X}$.
A straightforward check shows that these conditions are necessary and sufficient.

\begin{thm}{Exercise}\label{ex:extension-of-extension}
Let $\spc{X}$ be a subspace of metric space $\spc{Y}$.
Assume $f$ is an extension function on $\spc{X}$.

\begin{subthm}{ex:extension-of-extension:a}
Show that 
\[\bar f(y)
\df
\inf_{x\in \spc{X}} \{\,f(x)+\dist{x}{y}{\spc{Y}}\,\}\]
defines an extension function on $\spc{Y}$.
\end{subthm}

\begin{subthm}{}
Assume that $\diam \spc{Y}\le d$ and $f(x)\le d$ for any $x\in  \spc{X}$.
Show that 
\[\bar f_d
\df
\min \{\, \bar f,d\,\}\]
is a $d$-extension function on $\spc{Y}$.
\end{subthm}

\end{thm}

The functions $\bar f$ and $\bar f_d$ in the above exercise are called \index{Katětov extensions}\emph{Katětov extensions} of $f$ and the minimal possible $\spc{X}$ is called its \index{support of extension function}\emph{support}, briefly \index{$\supp$}$\supp \bar f=\spc{X}$.

\begin{thm}{Definition}\label{def:finite+1}
A metric space $\spc{U}$ meets the \index{extension property}\emph{extension property}  if for any finite subspace $\spc{F}\subset\spc{U}$ and any extension function $f\:\spc{F}\to\RR$ there is a point $p\in \spc{U}$ such that $\dist{p}{x}{}=f(x)$ for any $x\in \spc{F}$.

If we assume in addition that $\diam \spc{U}\le d$ and instead of extension functions we consider only $d$-extension functions, then we arrive at a definition of {}\emph{$d$-extension property}.

If in addition, $\spc{U}$ is separable and complete, then it is called \index{Urysohn space}\emph{Urysohn space} or {}\emph{$d$-Urysohn space} respectively.
\end{thm}


\begin{thm}{Proposition}\label{prop:univeral-separable}
There is a separable metric space with the ($d$-) extension property (for any $d\ge 0$).
\end{thm}

\parit{Proof.}
Choose $d\ge 0$.
Let us construct a separable metric space with  the $d$-extension property.

Let $\spc{X}$ be a metric space such that $\diam\spc{X}\le d$.
Denote by $\spc{X}^d$ the space of all $d$-extension functions on $\spc{X}$ equipped with the metric defined by the sup-norm.
Note that the map $\spc{X} \to \spc{X}^d$ defined by $x\mapsto\distfun_x$ is a distance-preserving embedding,
so we can (and will) treat $\spc{X}$ as a subspace of $\spc{X}^d$; equivalently, $\spc{X}^d$ is an extension of $\spc{X}$.

Let us iterate this construction.
Start with a one-point space $\spc{X}_0$ and consider a sequence of spaces $(\spc{X}_n)$ defined by $\spc{X}_{n+1}\z\df\spc{X}_n^d$.
Note that the sequence is nested;
that is, $\spc{X}_0\subset \spc{X}_1\subset\dots$
and the union
\[\spc{X}_\infty=\bigcup_n\spc{X}_n;\]
comes with metric such that
$\dist{x}{y}{\spc{X}_\infty} = \dist{x}{y}{\spc{X}_n}$
if $x,y\in\spc{X}_n$.

Note that if $\spc{X}$ is compact, then so is $\spc{X}^d$.
It follows that each space $\spc{X}_n$ is compact.
In particular, $\spc{X}_\infty$ is a countable union of compact spaces;
therefore $\spc{X}_\infty$ is separable.

Any finite subspace $\spc{F}$ of $\spc{X}_\infty$ lies in some $\spc{X}_n$ for $n<\infty$.
By construction, given an extension function $f\:\spc{F}\to\RR$,
there is a point $p\in \spc{X}_{n+1}$ that meets the condition in \ref{def:finite+1}.
That is, $\spc{X}_\infty$ has the $d$-extension property.

The construction of a separable metric space with the extension property requires only two changes.
First, the sequence should be defined by $\spc{X}_{n+1}\z\df\spc{X}_n^{d_n}$, where $d_n$ is an increasing sequence such that $d_n\to\infty$.
Second, the point $p$ should be taken in $\spc{X}_{n+k}$ for sufficiently large $k$, so that $d_{n+k}>\max\{f(x)\}$
(here one has to apply \ref{ex:extension-of-extension:a}).%

(Alternatively, one can start with any separable space $\spc{X}_0$ and consider a nested sequence $\spc{X}_0\subset \spc{X}_1\subset{}\dots$ where $\spc{X}_{n+1}$ is the space of all extension functions on $\spc{X}_{n}$ with at most $n+1$ points in its support.
The last condition is needed to keep $\spc{X}_{n}$ separable.)
\qeds

Given a metric space $\spc{X}$, denote by $\spc{X}^\infty$ the space of all extension functions on $\spc{X}$ equipped with the metric defined by the sup-norm.

\begin{thm}{Exercise}\label{ex:inf-extension}
Construct a proper length space $\spc{X}$ such that $\spc{X}^\infty$ is not separable.
\end{thm}


\begin{thm}{Proposition}\label{prop:completion-univeral}
If a metric space $\spc{V}$ meets the ($d$-) extension property, then so does its completion.
\end{thm}

\parit{Proof.} 
Let us assume $\spc{V}$ meets the extension property.
We will show that its completion $\spc{U}=\bar{\spc{V}}$ meets the extension property as well.
The $d$-extension case can be proved along the same lines.

Note that $\spc{V}$ is a dense subset in a complete space $\spc{U}$.
Observe that $\spc{U}$ has the {}\emph{approximate extension property};
that is, if $\spc{F}\z\subset\spc{U}$ is a finite set, $\eps>0$, and $f\:\spc{F}\to \RR$ is an extension function, then
there exists $p\in \spc{U}$ such that
\[\dist{p}{x}{}\lg f(x)\pm\eps\eqlbl{eq:|p-x|><f(x)}\]
for any $x\in\spc{F}$.
Indeed, consider the Katětov extension $\bar f\:\spc{U}\to\RR$ of~$f$.
Since $\spc{V}$ is dense in $\spc{U}$, we can choose a finite set $\spc{F}'\in \spc{V}$ such that for any $x\in \spc{F}$ there is $x'\in \spc{F}'$ with $\dist{x}{x'}{}<\tfrac\eps2$.
Let $p$ be the point provided by the extension property for the restriction $\bar f|_{\spc{F}'}$.
It remains to observe $p$ meets \ref{eq:|p-x|><f(x)}.

It follows that there is a sequence of points $p_n\in \spc{U}$ such that for any $x\in \spc{F}$, 
\[\dist{p_n}{x}{}\lg f(x)\pm\tfrac1{2^n}.\]

Moreover, we can assume that 
\[\dist{p_n}{p_{n+1}}{} < \tfrac1{2^n}\eqlbl{eq:|pn-pn|}\]
for all large $n$.
Indeed, consider the sets $\spc{F}_n=\spc{F}\cup\{p_n\}$ and the functions $f_n\:\spc{F}_n\to\RR$ defined by $f_n(x)\df f(x)$ and
\[f_n(p_n)
\df
\max\set{\bigl|\dist{p_n}{x}{}- f(x)\bigr|}{x\in \spc{F}}\]
 if $x\ne p_n$.
Observe that $f_n$ is an extension function for large $n$ and
$f_n(p_n)\z<\tfrac1{2^n}$.
Therefore, applying the approximate extension property recursively we get~\ref{eq:|pn-pn|}.

Therefore, the sequence $p_n$ is Cauchy.
Note that its limit meets the condition in the definition of extension property (\ref{def:finite+1}).
\qeds

Note that \ref{prop:univeral-separable} and \ref{prop:completion-univeral} imply the following:

\begin{thm}{Theorem}\label{thm:urysohn-exists}
Urysohn space and $d$-Urysohn space exist for any $d>0$.
\end{thm}

Here is a slightly stronger statement:

\begin{thm}{Theorem}\label{thm:urysohn-exists+}
Any separable metric space $\spc{X}$ admits a distance-preserving embedding into an Urysohn space $\spc{U}$ such that any isometry of $\spc{X}$ can be extended to an isometry of $\spc{U}$.
\end{thm}

\parit{Sketch of proof.}
Start with $\spc{X}_0=\spc{X}$ and construct a nested sequence of spaces $\spc{X}_0\subset\spc{X}_1 \subset{}\dots$ as at the alternative end of the proof of~\ref{prop:univeral-separable}.
Note that 
any isometry $\spc{X}_n\to \spc{X}_n$ can be extended to a unique isometry $\spc{X}_{n+1}\to \spc{X}_{n+1}$.
It follows that any isometry of $\spc{X}$ can be extended to an isometry of $\spc{X}'=\bigcup_n\spc{X}_n$.

Now, consider new nested sequence $\spc{X}\subset \spc{X}'\subset \spc{X}''\subset \dots$;
denote its union by $\spc{Y}$.
Arguing as in \ref{prop:univeral-separable} and \ref{prop:completion-univeral} we get that the completion of $\spc{Y}$ is an Urysohn space, say $\spc{U}$, that comes with a distance-preserving inclusion $\spc{X}\hookrightarrow \spc{U}$.

From above, for any isometry of $\spc{X}$ can be extended to isometries of $\spc{X}'$, $\spc{X}''$ and so on.
They all define an isometry of $\spc{Y}$;
passing to its continuous extension, we get an isometry of $\spc{U}$.
\qeds


\section{Universality}

A metric space will be called \index{universal space}\emph{universal} if it has a subspace isometric to any given separable metric space.
In \ref{ex:frechet}, we proved that $\ell^\infty$ is a universal space. 
The following proposition shows that an Urysohn space is universal as well.
Unlike $\ell^\infty$, Urysohn spaces are separable;
so it might be considered as a \textit{better} universal space.
Theorem \ref{thm:compact-homogeneous} will give another reason why Urysohn spaces are better.

\begin{thm}{Proposition}\label{prop:sep-in-urys}
An Urysohn space is universal.
That is, if $\spc{U}$ is an Urysohn space, then any separable metric space $\spc{S}$ admits a distance-preserving embedding $\spc{S}\hookrightarrow\spc{U}$.

Moreover, for any finite subspace $\spc{F}\subset \spc{S}$,
any distance-preserving embedding $\spc{F}\hookrightarrow \spc{U}$ can be extended to a distance-preserving embedding $\spc{S}\hookrightarrow\spc{U}$.

A $d$-Urysohn space is $d$-universal;
that is, the above statements hold provided that $\diam\spc{S}\le d$.  
\end{thm}

\parit{Proof.}
We will prove the second statement;
the first statement is its partial case for $\spc{F}=\emptyset$.

The required isometry will be denoted by $x\mapsto x'$.

Choose a dense sequence of points $s_1,s_2,\dotsc\in\spc{S}$.
We may assume that $\spc{F}=\{s_1,\dots,s_n\}$, so $s_i'\in \spc{U}$ are defined for $i\le n$.

The sequence $s_i'$ for $i>n$ can be defined recursively using the extension property in $\spc{U}$.
Namely, suppose that $s_1',\dots,s_{i-1}'$ are already defined.
Since $\spc{U}$ meets the extension property, there is a point $s_i'\in \spc{U}$ such that
\[\dist{s_i'}{s_j'}{\spc{U}}=\dist{s_i}{s_j}{\spc{S}}\]
for any $j<i$.

The constructed map $s_i\mapsto s_i'$ is distance-preserving.
Therefore it can be continuously extended to the whole $\spc{S}$.
It remains to observe that the constructed map $\spc{S}\hookrightarrow\spc{U}$ is distance-preserving.
\qeds

\begin{thm}{Exercise}\label{ex:geodesics-urysohn}
Show that any two distinct points in an Urysohn space can be joined by an infinite number of distinct geodesics.
\end{thm}

\begin{thm}{Exercise}\label{ex:compact-extension}
Modify the proofs of \ref{prop:completion-univeral} and \ref{prop:sep-in-urys} to prove the following theorem.
\end{thm}

\begin{thm}{Theorem}\label{thm:compact-extension}
Let $K$ be a compact set in a separable space $\spc{S}$.
Then any distance-preserving map from $K$ to an Urysohn space can be extended to 
a distance-preserving map of the whole $\spc{S}$.
\end{thm}

\begin{thm}{Exercise}\label{ex:sc-urysohn}
Show that ($d$-) Urysohn space is simply-connected.
\end{thm}



\section{Uniqueness and homogeneity}

\begin{thm}{Theorem}\label{thm:urysohn-unique}
Suppose $\spc{F}\subset \spc{U}$ and $\spc{F}'\subset \spc{U}'$ be finite isometric subspaces in a pair of ($d$-)Urysohn spaces $\spc{U}$ and $\spc{U}'$.
Then any isometry $\iota\:\spc{F}\leftrightarrow \spc{F}'$ can be extended to an isometry $\spc{U}\leftrightarrow \spc{U}'$.

In particular, ($d$-)Urysohn space is unique up to isometry.
\end{thm}

Note that \ref{prop:sep-in-urys} implies that there are distance-preserving maps $\spc{U}\z\to \spc{U}'$ and $\spc{U}'\to \spc{U}$.
The next exercise shows that it does not solely imply the existence of an isometry $\spc{U}\leftrightarrow \spc{U}'$.

\begin{thm}{Exercise}\label{ex:no-isom}
Construct two metric spaces $\spc{X}$ and $\spc{Y}$ such that 
there are distance-preserving maps $\spc{X}\to \spc{Y}$ and $\spc{Y}\to \spc{X}$, but no isometry $\spc{X}\leftrightarrow \spc{Y}$.
\end{thm}


The following construction uses the idea of \ref{prop:sep-in-urys}, but it is applied \index{back-and-forth}\emph{back-and-forth} to ensure that the obtained distance-preserving map is onto.

\parit{Proof.}
Choose dense sequences $a_1,a_2,\dots{}\in \spc{U}$ and $b'_1,b'_2,\dots{}\in \spc{U}'$.
We can assume that $\spc{F}=\{a_1,\dots,a_n\}$, $\spc{F}'=\{b_1',\dots,b_n'\}$ and $\iota(a_i)=b_i'$ for $i\le n$.

The required isometry $\spc{U}\leftrightarrow \spc{U}'$ will be denoted by $u \leftrightarrow u'$.
Set $a_i=b_i$ and $a'_i=b'_i$ if $i\le n$.

Let us define recursively $a_{n+1}',b_{n+1}, a_{n+2}', b_{n+2},\dots$ --- on the odd step we define the images of $a_{n+1},a_{n+2},\dots$ and on the even steps we define inverse images of $b'_{n+1},b'_{n+2},\dots$
The same argument as in the proof of \ref{prop:sep-in-urys} shows that we can construct two sequences $a_1',a_2',\dots{}\in \spc{U}'$ and $b_1,b_2,\dots\in \spc{U}$ such that
\begin{align*}
\dist{a_i}{a_j}{\spc{U}}&=\dist{a_i'}{a_j'}{\spc{U}'},
\\
\dist{a_i}{b_j}{\spc{U}}&=\dist{a_i'}{b_j'}{\spc{U}'},
\\
\dist{b_i}{b_j}{\spc{U}}&=\dist{b_i'}{b_j'}{\spc{U}'}
\end{align*}
for all $i$ and $j$.

It remains to observe that the constructed distance-preserving bijection defined by $a_i\leftrightarrow a_i'$ and $b_i\leftrightarrow b_i'$ extends
continuously to an isometry $\spc{U}\leftrightarrow \spc{U}'$. 
\qeds

Observe that \ref{thm:urysohn-unique} implies that the Urysohn space (as well as the $d$-Urysohn space) is \index{homogeneous}\emph{finite-set-homogeneous}; that is,
\begin{itemize}
 \item any distance-preserving map from a finite subset to the whole space can be extended to an isometry.
\end{itemize}

Recall that $S(p,r)_{\spc{X}}$ denotes the sphere of radius $r$ centered at $p$ in a metric space $\spc{X}$;
that is, 
$$S(p,r)_{\spc{X}}=\set{x\in \spc{X}}{\dist{p}{x}{\spc{X}}=r}.$$

\begin{thm}{Exercise}\label{ex:sphere-in-urysohn}
Choose $d\in [0,\infty]$.
Denote by $\spc{U}_d$ the $d$-Urysohn space,
so $\spc{U}_\infty$ is the Urysohn space.

\begin{subthm}{ex:sphere-in-urysohn:sphere}
Assume that $L=S(p,r)_{\spc{U}_d}\ne \emptyset$.
Show that $L$ is isometric to $\spc{U}_{\ell}$; find $\ell$ in terms of $r$ and $d$.
\end{subthm}

\begin{subthm}{ex:sphere-in-urysohn:midpoint}
Let $\ell=\dist{p}{q}{\spc{U}_d}$.
Show that the subset $M\subset\spc{U}_d$ of midpoints between $p$ and $q$ is isometric to $\spc{U}_\ell$.
\end{subthm}

\begin{subthm}{ex:sphere-in-urysohn:homogeneous}
Show that $\spc{U}_d$ is \emph{not} countable-set-homogeneous;
that is, there is a distance-preserving map from a countable subset of $\spc{U}_d$ to $\spc{U}_d$ that cannot be extended to an isometry of $\spc{U}_d$.
\end{subthm}

\end{thm}

In fact, the Urysohn space is compact-set-homogeneous; more precisely the following theorem holds.

\begin{thm}{Theorem}\label{thm:compact-homogeneous}
Let $K$ be a compact set in a ($d$-)Urysohn space~$\spc{U}$.
Then any distance-preserving map $K\to \spc{U}$ can be extended to an isometry of $\spc{U}$.
\end{thm}

A proof can be obtained by modifying the proofs of \ref{prop:completion-univeral} and \ref{thm:urysohn-unique}
the same way as it is done in \ref{ex:compact-extension}.

\begin{thm}{Exercise}\label{ex:shere}
Let $S$ be a unit sphere in the Urysohn space $\spc{U}$.
Show that for any two distinct points $x,y\in \spc{U}$ there is a point $z\in S$ such that 
$\dist{x}{z}{}\ne \dist{y}{z}{}$.

Conclude that two isometries of $\spc{U}$ coincide if they coincide on $S$.
\end{thm}

\begin{thm}{Exercise}\label{ex:ext(shere)}
Let $B$ be an open unit ball in the Urysohn space $\spc{U}$.
Show that $\spc{U}\setminus B$ is isometric to $\spc{U}$.

Use it to construct an isometry of a unit sphere $S$ in $\spc{U}$ that cannot be extended to an isometry of $\spc{U}$.
\end{thm}

\begin{thm}{Exercise}\label{ex:katetov}

\begin{subthm}{ex:katetov:inclusion}
Show that there is a distance-preserving inclusion of the Urysohn space $\iota\:\spc{U}\hookrightarrow \spc{U}$ 
such that $\spc{U}'=\iota(\spc{U})$ is nowhere dense in $\spc{U}$ and any isometry of $\spc{U}'$ 
can be extended to an isometry of the whole~$\spc{U}$.
\end{subthm}

\begin{subthm}{ex:katetov:sol}
Consider a nested sequence $\spc{U}_0\subset \spc{U}_1\subset\dots$ of Urysohn spaces 
with each inclusion $\spc{U}_n\hookrightarrow \spc{U}_{n+1}$ as in \ref{SHORT.ex:katetov:inclusion}.
Show that the union $\bigcup_n\spc{U}_n$ is a noncomplete finite-set-homogeneous metric space that meets the extension property.
\end{subthm}

\end{thm}

{\sloppy

\begin{thm}{Exercise}\label{ex:homogeneous}
Which of the following metric spaces are 
one-point-homogeneous, finite-set-homogeneous, compact-set-homogeneous, countable-set-homogeneous?

\begin{subthm}{ex:homogeneous:euclidean}
Euclidean plane,
\end{subthm}

\begin{subthm}{ex:homogeneous:hilbert}
 Hilbert space $\ell^2$,
\end{subthm}

\begin{subthm}{ex:homogeneous:ell-infty}
 $\ell^\infty$,
\end{subthm}

\begin{subthm}{ex:homogeneous:ell-1}
 \index{$\ell^1$}$\ell^1$ --- the space of all real absolutely converging series $\bm{a}\z=(a_1,a_2,\dots)$ with the norm $|\bm{a}|_{\ell^1}=\sum_i|a_i|$.
 
\end{subthm}
\end{thm}

}

\begin{thm}{Exercise}\label{ex:homogeneous-tree}
Show that any separable one-point-homogeneous metric tree is isometric to the real line $\RR$ or the one-point space.
\end{thm}


\section{Remarks}

The statement in \ref{ex:frechet} was proved by Maurice René Fréchet in the paper where he first defined metric spaces \cite{frechet};
its extension \ref{lem:kuratowski} was given by Kazimierz Kuratowski~\cite{kuratowski}.

The following two exercises show that in this respect $\ell^\infty$ is very different from $\ell^1$.
For more on the subject, see \cite{deza-laurent}.

Let $S$ be a subset of $X$.
The \index{cut metric}\emph{cut metric} $\delta_S$ on $X$ is a semimetric such that $\delta_S(x, y) = 1$ if $x$ and $y$ are separated
by $S$ and otherwise $\delta_S(x, y) = 0$.


\begin{thm}{Exercise}\label{ex:cut}
Show that a finite metric space $\spc{F}$ admits a distance-preserving embedding into $\ell^1$ if and only if the metric of $\spc{F}$ can be written as a nonnegative linear combination%
\footnote{that is, linear combination with nonnegative coefficients.} of cut metrics on $\spc{F}$.
\end{thm}

Recall that the vertex set of any graph $\Gamma$ comes with the shortest-path distance ---
the distance between two vertices is the minimal number of edges in a path connecting them.

\begin{wrapfigure}{r}{15mm}
\vskip-8mm
\centering
\includegraphics{mppics/pic-210}
\end{wrapfigure}

\begin{thm}{Exercise}\label{ex:K23}
Use \ref{ex:cut} to show that the metric for complete biparted graph $K_{2,3}$ (see the diagram) does not admit a distance-preserving embedding into $\ell^1$.
\end{thm}

The question about existence of a separable universal space was posed by Maurice René Fréchet and answered by
Pavel Urysohn~\cite{urysohn}.
Exercise \ref{ex:katetov} answers a question posed by Pavel Urysohn \cite[§$2(6)$]{urysohn}.
It was solved by Miroslav Katětov \cite{katetov},
but long after that, it was again mentioned as an open problem \cite[p. 83]{gromov-2007}.

The idea of Urysohn's construction was reused in graph theory; it produces the so-called \index{Rado graph}\emph{Rado graph},
also known as {}\emph{Erd\H{o}s--Rényi graph} or \emph{random graph}; see \cite{cameron}.
In fact, the Urysohn space is the random metric space in \textit{certain sense} \cite{vershik}.

\textit{The ($d$-) Urysohn space is homeomorphic to the Hilbert space};
the latter was proved by Vladimir Uspenskij \cite{uspenskij} using the so-called Toruńczyk criterion.

The finite-set-homogeneous spaces include Euclidean spaces, hyperbolic spaces, and spheres all with standard length metrics and arbitrary finite dimensions.
In fact, these are the only examples of locally compact three-point-homogeneous length spaces.
The latter was proved by Herbert Busemann \cite{busemann-1942}; it also follows from the more general result of Jacques Tits about two-point-homogeneous spaces \cite{tits}.
The same conclusion holds for complete all-set-homogeneous geodesic spaces with local uniqueness of geodesics;
it was proved by Garrett Birkhoff \cite{birkhoff}.
The answer might be the same for complete separable all-set-homogeneous length spaces.
Without the separability condition, we also get the so-called \emph{universal metric trees} with finite valence \cite{dyubina-polterovich}; no other examples seem to be known \cite{lebedeva-petrunin2211.09671}.  

{\sloppy

\begin{thm}{Exercise}\label{ex:RP-not}
Show that the real projective plane $\RP^2$ with the standard metric is two-point-homogeneous, but not three-point-homogeneous.
\end{thm}

}

\begin{thm}{Exercise}\label{ex:hom-cube}
Let $Q$ be the set of vertices on the $n$-dimensional cube;
assume $n$ is large.
Show that $Q$ is three-point-homogeneous, but not four-point-homogeneous.
\end{thm}

I do not know examples of metric spaces that are $n$-point-homogeneous, but not $(n+1)$-point-homogeneous for large $n$ \cite{petrunin-431426}.

\chapter{Injective spaces}\label{chap:injective}


Injective hull is a useful construction that provides a canonical choice of a specially nice (injective) space that includes a given metric space. 
This construction is similar to the convex hull in Euclidean space.
The following exercise gives a bridge from the latter to the former.

\begin{thm}{Advanced exercise}\label{ex:conv-short}
Show that $A\subset \RR^n$ is a closed convex set if and only if for any  $B\subset \RR^n$ any short map $B\to A$ can be extended to a short map $\RR^n\to A$.
\end{thm}

\section{Definition}

\begin{thm}{Definition}\label{def:injective}
A metric space $\spc{Y}$ is called \index{injective space}\emph{injective} if for any metric space $\spc{X}$ and any of its subspace $\spc{A}$,
any short map $f\:\spc{A}\to \spc{Y}$ can be extended to a short map $F\:\spc{X}\to \spc{Y}$;
that is, $f=F|_{\spc{A}}$.
\end{thm}

\begin{thm}{Exercise}\label{ex:inj=complete-geodesic-contractible}
Show that any injective space is 
\begin{multicols}{3}

\begin{subthm}{ex:inj=complete-geodesic-contractible:complete}
complete,
\end{subthm}

\begin{subthm}{ex:inj=complete-geodesic-contractible:geodesic}
geodesic, and
\end{subthm}

\begin{subthm}{ex:inj=complete-geodesic-contractible:contractible}
contractible.
\end{subthm}

\end{multicols}

\end{thm}

\begin{thm}{Exercise}\label{ex:bicombing}
Show that for any injective space $\spc{Y}$ there is a map $m\:\spc{Y}\times\spc{Y}\to\spc{Y}$ (the \index{midpoint map}\emph{midpoint map}) such that the inequality
\[2\cdot \dist{p}{m(x,y)}{\spc{Y}}\le\dist{p}{x}{\spc{Y}}+\dist{p}{y}{\spc{Y}}\]
holds for any $p,x,y\in \spc{Y}$.
\end{thm}

\begin{thm}{Exercise}\label{ex:injective-spaces}
Show that the following spaces are injective:
\begin{subthm}{ex:injective-spaces:R}
the real line;
\end{subthm}

\begin{subthm}{ex:injective-spaces:tree}
complete metric tree;
\end{subthm}

\begin{subthm}{ex:injective-spaces:ell-infty}
The space $\ell^\infty(\spc{S})$ for any set $\spc{S}$ (defined in \ref{lem:kuratowski}).
In particular, the coordinate plane with the metric induced by the $\ell^\infty$-norm.
\end{subthm}

\end{thm}

\begin{thm}{Exercise}\label{ex:extr-ball}
Let $\spc{Y}$ be an injective space.

\begin{subthm}{ex:extr-ball:one}
Show that any closed ball in $\spc{Y}$ is injective.
\end{subthm}

\begin{subthm}{ex:extr-ball:many}
Show that the intersection of an arbitrary collection of closed balls in $\spc{Y}$ is injective.
\end{subthm}

\end{thm}

\begin{thm}{Advanced exercise}\label{ex:extr-fixed}
Let $\spc{Y}$ be a bounded injective space.
Show that any short map $s\:\spc{Y}\to\spc{Y}$ has a fixed point. 
\end{thm}


\section{Admissible and extremal functions}

Let $\spc{X}$ be a metric space.
A function $r\:\spc{X}\to(-\infty,\infty]$ is called \label{page:admissible function}\index{admissible function}\emph{admissible} if the following inequality
\[r(x)+r(y)\ge \dist{x}{y}{\spc{X}}\eqlbl{eq:admissible}\]
holds for any $x,y\in \spc{X}$.

\begin{thm}{Observation}\label{obs:admissible}

\begin{subthm}{obs:admissible:nonnegative}
Any admissible function is nonnegative.
\end{subthm}

\begin{subthm}{obs:admissible:balls}
If $\spc{X}$ is a geodesic space, then a function $r\:\spc{X}\to\RR$ is admissible if and only if 
\[\cBall[x,r(x)]\cap\cBall[y,r(y)]\ne \emptyset\]
for any $x,y\in \spc{X}$.
\end{subthm}
 
\end{thm}

\parit{Proof; \ref{SHORT.obs:admissible:nonnegative}.} Apply \ref{eq:admissible} for $x=y$.

\parit{\ref{SHORT.obs:admissible:balls}.} Apply the triangle inequality and the existence of a geodesic $[xy]$.
\qeds

A minimal admissible function will be called \label{page:extremal function}\index{extremal function}\emph{extremal}.
More precisely, an admissible function $r\:\spc{X}\to\RR$ is extremal 
if for any admissible function $s\:\spc{X}\to\RR$ we have
\[s\le r\quad\Longrightarrow\quad s=r.\]

Applying Zorn's lemma, we get the following.

\begin{thm}{Observation}\label{obs:extremal:below}
For any admissible function $s$ there is an extremal function $r$ such that $r\le s$.
\end{thm}

\begin{thm}{Lemma}\label{lem:+-c}
Let $r$ be an extremal function and $s$ an admissible function on a metric space $\spc{X}$.
Suppose that $r\ge s-c$ for some constant~$c$.
Then $r\le s+c$; in particular, $c\ge 0$.
\end{thm}

\parit{Proof.}
Note that if $c<0$, then $r>s$.
The latter is impossible since $r$ is extremal and $s$ is admissible.

Observe that the function $\bar r=\min\{\,r,s+c\,\}$ is admissible.
Indeed, choose $x,y\in \spc{X}$.
If $\bar r(x)=r(x)$ and $\bar r(y)=r(y)$, then 
\[\bar r(x)+\bar r(y)=r(x)+ r(y)\ge \dist{x}{y}{}.\]
Further, if $\bar r(x)=s(x)+c$, then 
\begin{align*}
\bar r(x)+\bar r(y)&\ge [s(x)+c]+ [s(y)-c]= 
\\
&=s(x)+s(y) \ge 
\\
&\ge\dist{x}{y}{}.
\end{align*}

Since $r$ is extremal, we have $r=\bar r$;
that is, $r\le s+c$.
\qeds

\begin{thm}{Observations}\label{obs:extremal}
Let $\spc{X}$ be a metric space.

\begin{subthm}{obs:extremal:distfun}
For any point $p\in\spc{X}$ the distance function $r\z=\distfun_p$ is extremal.
\end{subthm}

\begin{subthm}{lem:extremal-lipschitz}
Any extremal function $r$ on $\spc{X}$ is \index{1-Lipschitz function}\emph{1-Lipschitz};
that is,
\[|r(p)-r(q)|\le \dist{p}{q}{}\]
for any $p,q\in\spc{X}$.
In other words, any extremal function is an extension function [see \ref{sec:Extension property}].
\end{subthm}

\begin{subthm}{lem:opposite}
An admissible function $r$ on $\spc{X}$ is extremal if and only if
for any point $p\in\spc{X}$ and any $\delta>0$, there is a point $q\in \spc{X}$
such that 
\[r(p)+r(q)<\dist{p}{q}{\spc{X}}+\delta.\]
\end{subthm}

\begin{subthm}{lem:opposite-compact}
Suppose $\spc{X}$ is compact.
Then an admissible function $r$ on $\spc{X}$ is extremal if and only if
for any point $p\in\spc{X}$ there is a point $q\in \spc{X}$
such that 
\[r(p)+r(q)=\dist{p}{q}{\spc{X}}.\]
\end{subthm}

\end{thm}

\parit{Proof; \ref{SHORT.obs:extremal:distfun}.}
By the triangle inequality, \ref{eq:admissible} holds;
that is, $r=\distfun_p$ is an admissible function.

Further, if $s\le r$ is another admissible function, then $s(p)=0$ and \ref{eq:admissible} implies that $s(x)\z\ge\dist{p}{x}{}$.
Whence $s=r$.

\parit{\ref{SHORT.lem:extremal-lipschitz}.}
By \ref{SHORT.obs:extremal:distfun}, $\distfun_p$ is admissible.
Since $r$ is admissible, we have that
\[r\ge \distfun_p-r(p).\]
Since $r$ is extremal, \ref{lem:+-c} implies that
\[r\le \distfun_p+r(p),\]
or, equivalently,
\[r(q)-r(p)\le \dist{p}{q}{}\]
for any $p,q\in\spc{X}$.
Whence the statement follows.

\parit{\ref{SHORT.lem:opposite}.}
Assume $r$ is extremal.
Arguing by contradiction, assume there is $\delta>0$ such that
\[r(q)\ge \distfun_p(q)-r(p)+\delta\]
for any $q$.
By \ref{SHORT.obs:extremal:distfun}, $\distfun_p$ is extremal; in particular, admissible.
Therefore \ref{lem:+-c} implies that
\[r(q)\le \distfun_p(q)+r(p)-\delta\]
for any $q$.
Taking $q=p$, we get $r(p)\le r(p)-\delta$, a contradiction.

Now suppose $r$ is not extremal; that is, there is an admissible function $s\le r$ such that $r(p)-s(p)=\delta>0$ for some $p$.
Then, for any $q$, we have
\[r(p)+r(q)\ge s(p)+s(q)+\delta\ge \dist{p}{q}{\spc{X}}+\delta\]
--- a contradiction.

\parit{\ref{SHORT.lem:opposite-compact}.}
The if part follows from \ref{SHORT.lem:opposite}.

Denote by $q_n$ the point provided by \ref{SHORT.lem:opposite} for $\delta=\tfrac1n$.
Let $q$ be a partial limit of $q_n$. 
Then 
\[r(p)+r(q)\le\dist{p}{q}{\spc{X}}.\]
Since $r$ is admissible, the opposite inequality holds;
whence the only-if part follows.
\qeds

\begin{thm}{Exercise}\label{ex:circle}
Consider the unit circle 
\[\mathbb{S}^1=\set{(x,y)}{x^2+y^2=1}\]
in the plane with induced length metric.
Show that $r\:\mathbb{S}^1\to\RR$ is extremal if and only if it is 1-Lipschitz and 
\[r(p)+r(-p)=\pi\] for any $p\in\mathbb{S}^1$.
\end{thm}

\begin{thm}{Exercise}\label{ex:retraction}
Given a real-valued function $s$ on a metric space $\spc{X}$,
consider the function
\[s^*(x)=\sup\set{\dist{x}{y}{\spc{X}}-s(y)}{y\in \spc{X}}\]
Show that the function $\tfrac12\cdot(s+s^*)$ is admissible for any $s$.
\end{thm}

\section{Equivalent conditions}

\begin{thm}{Theorem}\label{thm:injective=hyperconvex}
For any metric space $\spc{Y}$ the following conditions are equivalent:

\begin{subthm}{thm:injective=hyperconvex:injective}
$\spc{Y}$ is injective
\end{subthm}


\begin{subthm}{thm:injective=hyperconvex:extremal}
If $r\:\spc{Y}\to\RR$ is an extremal function, then there is a point $p\in \spc{Y}$ such that 
\[\dist{p}{x}{}= r(x)\]
for any $x\in \spc{Y}$.
\end{subthm}

\begin{subthm}{thm:injective=hyperconvex:balls}
$\spc{Y}$ is \index{hyperconvex space}\emph{hyperconvex};
that is, if $\set{\cBall[x_\alpha,r_\alpha]}{\alpha\in\IndexSet}$ is a family of closed balls in $\spc{Y}$ such that 
 \[r_\alpha+r_\beta\ge \dist{x_\alpha}{x_\beta}{}\]
 for any $\alpha,\beta\in \IndexSet$, then all the balls in the family $\{\cBall[x_\alpha,r_\alpha]\}_{\alpha\in\IndexSet}$ have a common point.
\end{subthm}

\end{thm}

\parit{Proof.} We will prove implications 
\ref{SHORT.thm:injective=hyperconvex:injective}$\Rightarrow$\ref{SHORT.thm:injective=hyperconvex:extremal}$\Rightarrow$\ref{SHORT.thm:injective=hyperconvex:balls}$\Rightarrow$\ref{SHORT.thm:injective=hyperconvex:injective}.

\parit{\ref{SHORT.thm:injective=hyperconvex:injective}$\Rightarrow$\ref{SHORT.thm:injective=hyperconvex:extremal}.}
By \ref{lem:extremal-lipschitz}, $r$ is an extension function.
Applying the definition of injective space to a one-point extension of $\spc{Y}$, we get a point $p\in \spc{Y}$ such that 
\[\dist{p}{x}{}=\distfun_p(x)\le r(x)\]
for any $x\in \spc{Y}$.
By \ref{obs:extremal:distfun}, the distance function $\distfun_p$ is extremal.
Since  $r$ is extremal, we get $\distfun_p= r$.


\parit{\ref{SHORT.thm:injective=hyperconvex:extremal}$\Rightarrow$\ref{SHORT.thm:injective=hyperconvex:balls}.}
By \ref{obs:admissible:balls}, part \ref{SHORT.thm:injective=hyperconvex:balls} is equivalent to the following statement:
\begin{itemize}
 \item If $r\:\spc{Y}\to\RR$ is an admissible function, then there is a point $p\in \spc{Y}$ such that 
\[\dist{p}{x}{}\le r(x)\eqlbl{eq:|p-x|=<r(x)}\]
for any $x\in \spc{Y}$.
\end{itemize}
Indeed, set $r(x)\df\inf\set{r_\alpha}{x_\alpha=x}$.
(If $x_\alpha\ne x$ for any $\alpha$, then $r(x)=\infty$.)
The condition in \ref{SHORT.thm:injective=hyperconvex:balls} implies that $r$ is admissible.
It remains to observe that $p\in \cBall[x_\alpha,r_\alpha]$ for every $\alpha$ if and only if \ref{eq:|p-x|=<r(x)} holds.

By \ref{obs:extremal:below}, for any admissible function $r$ there is an extremal function $\bar r\le r$;
hence \ref{SHORT.thm:injective=hyperconvex:extremal}$\Rightarrow$\ref{SHORT.thm:injective=hyperconvex:balls}.

\parit{\ref{SHORT.thm:injective=hyperconvex:balls}$\Rightarrow$\ref{SHORT.thm:injective=hyperconvex:injective}.}
Arguing by contradiction, suppose $\spc{Y}$ is not injective;
that is, there is a metric space $\spc{X}$ with a subset $\spc{A}$
such that a short map $f\:\spc{A}\to \spc{Y}$ cannot be extended to a short map $F\:\spc{X}\to \spc{Y}$.
By Zorn's lemma, we may assume that $\spc{A}$ is a maximal subset; that is, the domain of $f$ cannot be enlarged by a single point.%
\footnote{In this case, $\spc{A}$ must be closed, but we will not use it.}

Fix a point $p$ in the complement $\spc{X}\setminus \spc{A}$.
To extend $f$ to $p$, we need to choose $f(p)$ in the intersection of the balls 
$\cBall[f(x),r(x)]$, where $r(x)=\dist{p}{x}{}$.
Therefore, this intersection for all $x\in \spc{A}$ has to be empty.

Since $f$ is short, we have that 
\begin{align*}
r(x)+r(y)&\ge \dist{x}{y}{\spc{X}}\ge
\\
&\ge \dist{f(x)}{f(y)}{\spc{Y}}.
\end{align*}
By \ref{SHORT.thm:injective=hyperconvex:balls} the balls 
$\cBall[f(x),r(x)]$ have a common point --- a contradiction. 
\qeds

\begin{thm}{Exercise}\label{ex:one-point-gluing}
Suppose a length space $\spc{W}$ has two subspaces $\spc{X}$ and $\spc{Y}$ such that $\spc{X}\cup\spc{Y}=\spc{W}$ and $\spc{X}\cap\spc{Y}$ is a one-point set.
Assume $\spc{X}$ and $\spc{Y}$ are injective.
Show that  $\spc{W}$ is injective
\end{thm}

\begin{thm}{Exercise}\label{ex:Rm-ell-infty}
Show that an $m$-dimensional normed space is injective if and only if it is isometric to $\RR^m$ with $\ell^\infty$-norm; that is,
\[|(x_1,\dots,x_m)|=\max_i\{\,|x_i|\,\}.\]
\end{thm}

A metric space $\spc{Y}$ is called \index{finitely hyperconvex}\emph{finitely hyperconvex} or \index{countably hyperconvex}\emph{countably hyperconvex} if the condition in \ref{thm:injective=hyperconvex:balls} holds only for any finite or respectively countable family of balls.

\begin{thm}{Exercise}\label{ex:compact-hyperconvex}
Show that any proper finitely hyperconvex metric space is hyperconvex.
\end{thm}


\begin{thm}{Exercise}\label{ex:urysohn-hyperconvex}
Show that the $d$-Urysohn space is finitely hyperconvex, but not countably hyperconvex.
Conclude that the $d$-Urysohn space is not injective.

Try to do the same for the Urysohn space.
\end{thm}

\begin{thm}{Exercise}\label{ex:almost-hyperconvex}
Let $\spc{Y}$ be a complete metric space.
Suppose $\spc{Y}$ is \index{almost hyperconvex}\emph{almost hyperconvex},
that is, for any $\eps>0$ any family of closed balls $\set{\cBall[x_\alpha,r_\alpha+\eps]}{\alpha\in\IndexSet}$ has a common point if 
$r_\alpha+r_\beta\ge \dist{x_\alpha}{x_\beta}{}$ for all $\alpha,\beta\in \IndexSet$.
Show that $\spc{Y}$ is hyperconvex.
\end{thm}


\section{Space of extremal functions}
\label{sec:extremal-functions}

Let $\spc{X}$ be a metric space.
Consider the space $\Inj \spc{X}$ of extremal functions on $\spc{X}$ equipped with sup-norm; \label{page:InjX}
that is,
\[\dist{f}{g}{\Inj \spc{X}}\df\sup\set{|f(x)-g(x)|}{x\in \spc{X}}.\]

Recall that by \ref{obs:extremal:distfun}, any distance function is extremal.
It follows that the map $x\mapsto \distfun_x$ produces a distance-preserving embedding $\spc{X}\hookrightarrow\Inj \spc{X}$.
So we can (and will) treat $\spc{X}$ as a subspace of $\Inj \spc{X}$,
or, equivalently, $\Inj \spc{X}$ as an extension of $\spc{X}$.
In particular, from now on, a point $x\in\spc{X}$ can refer to the function $\distfun_x\:\spc{X}\to\RR$ and the other way around.

Since any extremal function is 1-Lipschitz, for any $f\in \Inj \spc{X}$ and $p\in \spc{X}$, we have that
$f(x)\le f(p)+\distfun_p(x)$.
By \ref{lem:+-c}, we also get $f(x)\ge -f(p)+\distfun_p(x)$.
Therefore
\[
\begin{aligned}
\dist{f}{p}{\Inj \spc{X}}&=\sup\set{|f(x)-\distfun_p(x)|}{x\in \spc{X}}=
\\
&=f(p).
\end{aligned}
\eqlbl{eq:f(p)=|f-p|}
\]
In particular, the statement in \ref{lem:opposite} can be written as 
\[\dist{f}{p}{\Inj\spc{X}}+\dist{f}{q}{\Inj\spc{X}}<\dist{p}{q}{\Inj\spc{X}}+\delta.\]

\begin{thm}{Exercise}\label{ex:Inj(compact)}
Show that $\Inj\spc{X}$ is compact if and only if so is $\spc{X}$.
\end{thm}

\begin{thm}{Exercise}\label{ex:tripod+square}
Describe the set of all extremal functions on a metric space $\spc{X}$ and the metric space $\Inj \spc{X}$ in each of the following cases:

\begin{subthm}{ex:tripod+square:2}
$\spc{X}$ is a metric space with exactly two points $v,w$ on distance 1 from each other.
\end{subthm}


\begin{subthm}{ex:tripod+square:tripod} 
$\spc{X}$ is a metric space with exactly three points $a,b,c$ such that 
\[\dist{a}{b}{\spc{X}}=\dist{b}{c}{\spc{X}}=\dist{c}{a}{\spc{X}}=1.\]
\end{subthm}

\begin{subthm}{ex:tripod+square:square}
$\spc{X}$ is  a metric space with exactly four points $p,q,x,y$ such that 
\[\dist{p}{x}{\spc{X}}=\dist{p}{y}{\spc{X}}=\dist{q}{x}{\spc{X}}=\dist{q}{y}{\spc{X}}=1\]
and
\[\dist{p}{q}{\spc{X}}=\dist{x}{y}{\spc{X}}=2.\]
\end{subthm}

\end{thm}

\begin{thm}{Exercise}\label{ex:kur-inj}
Assume $\spc{X}$ is a compact metric space.
Recall that the map $x\mapsto \distfun_x$ gives an isometric embedding $\spc{X}\hookrightarrow\ell^\infty(\spc{X})$; so we can think that $\spc{X}$ is a subset of $\ell^\infty(\spc{X})$.

Given two points $x,y\in \spc{X}$, denote by $G_{x,y}$ the union of all geodesics from $x$ to $y$ in $\ell^\infty(\spc{X})$.
Show that $\Inj\spc{X}$ is isometric to
\[G=\bigcap_{x\in \spc{X}}\left(\bigcup_{y\in \spc{X}}G_{x,y}\right).\]

\end{thm}


\begin{thm}{Proposition}\label{prop:InjX-is-injective}
$\Inj\spc{X}$ is injective for any metric space $\spc{X}$. 
\end{thm}

\begin{thm}{Lemma}\label{lem:r|X-extremal}
Let $\spc{X}$ be a metric space.
Then 
\[\sigma\in \Inj(\Inj \spc{X})
\quad\Longrightarrow\quad
\sigma|_\spc{X}\in \Inj \spc{X}.\]
\end{thm}

In other words, if $\sigma$ is an extremal function on $\Inj \spc{X}$,
then the restriction of $\sigma$ to $\spc{X}$ is an extremal function on $\spc{X}$.

\parit{Proof.}
Arguing by contradiction, suppose that there is an admissible function $s\:\spc{X}\to \RR$ such that $s(x)\le \sigma(x)$ for any $x\in\spc{X}$ and $s(p)\z< \sigma(p)$ for some point $p\in\spc{X}$.
Consider another function $\bar \sigma\:\Inj \spc{X}\to\RR$ such that $\bar \sigma(f)\df \sigma(f)$ if $f\ne p$ and $\bar \sigma(p)\df s(p)$.

Let us show that $\bar \sigma$ is admissible; that is, 
\[\dist{f}{g}{\Inj \spc{X}}\le\bar \sigma(f)+\bar \sigma(g)
\eqlbl{r-admissible}\]
for any $f,g\in \Inj \spc{X}$.

Since $\sigma$ is admissible and $\bar \sigma= \sigma$ on $(\Inj \spc{X})\setminus \{p\}$, it is sufficient to prove \ref{r-admissible} assuming $f\ne g=p$.
By \ref{eq:f(p)=|f-p|}, we have $\dist{f}{p}{\Inj \spc{X}}=f(p)$.
Therefore, \ref{r-admissible} boils down to the following inequality
\[\sigma(f)+s(p)\ge f(p).\eqlbl{eq:r(f)+s(p)>=f(p)}\]
for any $f\in\Inj \spc{X}$.

Fix small $\delta>0$. 
Let $q\in\spc{X}$ be the point provided by \ref{lem:opposite}.
Then
\begin{align*}
\sigma(f)+s(p)&\ge [\sigma(f)-\sigma(q)]+[\sigma(q)+s(p)]\ge
\intertext{since $\sigma$ is 1-Lipschitz, and $\sigma(q)\ge s(q)$, we can continue}
&\ge -\dist{q}{f}{\Inj \spc{X}}+[s(q)+s(p)]\ge
\intertext{by \ref{eq:f(p)=|f-p|} and since $s$ is admissible}
&\ge -f(q)+\dist{p}{q}{}>
\intertext{and by \ref{lem:opposite}}
&> f(p)-\delta.
\end{align*}
Since $\delta>0$ is arbitrary, \ref{eq:r(f)+s(p)>=f(p)} and \ref{r-admissible} follow.

Summarizing: the function $\bar \sigma$ is admissible, $\bar \sigma\le \sigma$ and $\bar \sigma(p)<\sigma(p)$;
that is, $\sigma$ is not extremal --- a contradiction.
\qeds

\parit{Proof of \ref{prop:InjX-is-injective}.}
Choose a function $\sigma\in\Inj(\Inj\spc{X})$.
By \ref{lem:r|X-extremal}, $s\z\df \sigma|_{\spc{X}}\in \Inj\spc{X}$;
that is, $s$ is extremal.
By \ref{thm:injective=hyperconvex:extremal},
it is sufficient to show that  
\[\sigma(f)\ge\dist{s}{f}{\Inj\spc{X}}
\eqlbl{eq:r(f)>=| r-f|}\]
for any $f\in\Inj\spc{X}$.

Since $\sigma$ is $1$-Lipschitz (\ref{lem:extremal-lipschitz}) we have that
\[
s(x)-f(x)=\sigma(x)-\dist{f}{x}{\Inj \spc{X}}\le \sigma(f).
\]
for any $x\in\spc{X}$.
By \ref{lem:+-c},
$
s(x)-f(x)\ge -\sigma(f)
$
for any $x\in\spc{X}$.
Whence \ref{eq:r(f)>=| r-f|} follows.
\qeds

\begin{thm}{Exercise}\label{ex:4-on-a-line}
Let $\spc{X}$ be a compact metric space.
Show that for any two points $f,g\in\Inj \spc{X}$ lie on a geodesic $[pq]$ with $p,q\in \spc{X}$.
\end{thm}

A metric space $\spc{X}$ is called \index{$\delta$-hyperbolic}\emph{$\delta$-hyperbolic} if 
\[\dist{p}{q}{}+\dist{x}{y}{}\le
\max\{\,\dist{p}{x}{}+\dist{q}{y}{},
\,
\dist{p}{y}{}+\dist{q}{x}{}\,\}+2\cdot\delta\]
for any $p,q,x,y\in \spc{X}$.

\begin{thm}{Advanced exercise}\label{ex:delta-hyp}
Show that $\Inj \spc{X}$ is $\delta$-hyperbolic if and only if so is $\spc{X}$.
\end{thm}


\section{Injective envelope}

An extension $\spc{E}$ of a metric space $\spc{X}$ will be called its \index{injective envelope}\emph{injective envelope} if $\spc{E}$ is an injective space, and there is no proper injective subspace of $\spc{E}$ that contains $\spc{X}$.

Two injective envelopes $e\:\spc{X}\hookrightarrow \spc{E}$ and $f\:\spc{X}\hookrightarrow \spc{F}$ are called  equivalent if there is an isometry $\iota\: \spc{E}\to\spc{F}$ such that $f=\iota\circ e$.

\begin{thm}{Theorem}\label{thm:inj-envelope}
For any metric space $\spc{X}$, its extension $\Inj\spc{X}$ is an injective envelope.

Moreover, any other injective envelope of $\spc{X}$ is equivalent to $\Inj\spc{X}$.
\end{thm}

\parit{Proof.} 
Suppose $S\subset \Inj\spc{X}$ is an injective subspace containing $\spc{X}$.
Since $S$ is injective, there is a short map $w\:\Inj\spc{X}\to S$ that fixes all points in $\spc{X}$.

Suppose that $w\:f\mapsto f'$; observe that $f(x)\ge f'(x)$ for any $x\in \spc{X}$.
Since $f$ is extremal, $f=f'$;
that is, $w$ is the identity map, and therefore $S=\Inj\spc{X}$.

Assume we have another injective envelope $e\:\spc{X}\hookrightarrow \spc{E}$.
Then there are short maps $v\:\spc{E}\to \Inj\spc{X}$ and $w\:\Inj\spc{X}\to \spc{E}$ such that $x=v\circ e(x)$ and $e(x)=w(x)$ for any $x\in\spc{X}$.
From above, the composition $v\circ w$ is the identity on $\Inj\spc{X}$.
In particular, $w$ is distance-preserving.

The composition $w\circ v\:\spc{E}\to \spc{E}$ is a short map that fixes points in $e(\spc{X})$.
Since $e\:\spc{X}\hookrightarrow \spc{E}$ is an injective envelope, the composition $w\circ v$ and therefore $w$ are onto.
Whence $w$ is an isometry.
\qeds

\begin{thm}{Exercise}\label{ex:inj-envelope}
Suppose $e\:\spc{X}\hookrightarrow \spc{E}$ and $f\:\spc{X}\hookrightarrow \spc{F}$ are two injective envelopes of $\spc{X}$.
Show that there is a unique isometry $\iota\:\spc{E}\to \spc{F}$ such that $\iota\circ e=f$.
\end{thm}

\begin{thm}{Exercise}\label{ex:d-p-inclusion}
Suppose $\spc{X}$ is a subspace of a metric space $\spc{U}$.
Show that the inclusion $\spc{X}\hookrightarrow\spc{U}$ can be extended to a distance-preserving inclusion $\Inj\spc{X}\hookrightarrow\Inj\spc{U}$.
\end{thm}

\begin{thm}{Exercise}\label{ex:hemisphere-inj}
Consider the hemisphere 
\begin{align*}
\mathbb{S}^2_+&=\set{(x,y,z)\in\RR^3}{x^2+y^2+z^2=1,\quad z\ge0}
\intertext{and its boundary}
\mathbb{S}^1&=\set{(x,y,z)\in\RR^3}{x^2+y^2+z^2=1,\quad z=0};
\end{align*}
 both with induced length metrics.
 
Show that there is unique isometric embedding $\iota\:\mathbb{S}^2_+\hookrightarrow\Inj\mathbb{S}^1$ such that $\iota(u)=u$ for any $u\in \mathbb{S}^1$.
\end{thm}


\section{Remarks}

Injective spaces were introduced by Nachman Aronszajn and Prom Panitchpakdi \cite{aronszajn-panitchpakdi}.
The injective envelope was introduced by John Isbell \cite{isbell}; it is also known as \index{tight span}\emph{tight span} and \index{hyperconvex hull}\emph{hyperconvex hull}.

It was observed by John Isbell \cite{isbell2} that \textit{if $\spc{X}$ is a Banach space, then its injective hull $\Inj\spc{X}$ has a natural structure of Banach space} (which is unique by the Mazur--Ulam theorem).
Moreover, $\spc{X}$ is a linear subspace of $\Inj\spc{X}$.
 
Let us mention that a metric space $\spc{X}$ is called \index{convex space}\emph{convex} if for any pair of points $x_1,x_2\in \spc{X}$ and any $r_1,r_2\ge 0$ we have 
\[r_1+r_2\ge \dist{x_1}{x_2}{\spc{X}}\qquad\Longrightarrow\qquad\cBall[x,r_1]_\spc{X}\cap \cBall[y,r_2]_\spc{X}\ne\emptyset;\]
in other words, a pair of balls intersect if the triangle inequality does not forbid it.
Clearly, hyperconvexity (\ref{thm:injective=hyperconvex:balls}) is stronger than convexity.
Note that \textit{any geodesic space is convex}.
The converse does not hold in general, but by \ref{lem:mid>geod:geod} \textit{any complete convex space is geodesic}.

More generally, a metric space $\spc{X}$ is called \index{$n$-heperconvex space}\emph{$n$-heperconvex} if the condition in \ref{thm:injective=hyperconvex:balls} holds only for families with at most $n$ balls; so \textit{convex means $2$-hyperconvex}.

The following striking result was proved by Benjamin Miesch and Maël Pavón \cite{miesch-pavon2016}.

\begin{thm}{Theorem}
Any complete $4$-hyperconvex space is finitely hyperconvex.
\end{thm}

So, by \ref{ex:compact-hyperconvex}, it follows that \textit{any proper $4$-hyperconvex space is hyperconvex}.

\begin{thm}{Exercise}\label{ex:3-4-hypreconvex}
Show that $\ell^1$ is $3$- but not $4$-hyperconvex.
\end{thm}
 

Recall that if the following inequality
\[\dist{x}{z}{\spc{X}}
\le
\max\{\,\dist{x}{y}{\spc{X}},\dist{y}{z}{\spc{X}}\,\}\]
holds for any three points $x,y,z$ in a metric space $\spc{X}$,
then $\spc{X}$ is called an \index{ultrametric space}\emph{ultrametric space}.
In some sense, ultrametric spaces are dual to injective spaces.

\begin{thm}{Exercise}\label{ex:ultrametric}
Suppose that a metric space $\spc{X}$ satisfies the following property:
For any subspace $\spc{A}$ in $\spc{X}$ and any other metric space $\spc{Y}$, any short map $f\:\spc{A}\to \spc{Y}$ can be extended to a short map $F\:\spc{X}\to \spc{Y}$.

Show that $\spc{X}$ is an ultrametric space.
\end{thm}

A subspace $\spc{S}$ of a metric space $\spc{X}$ is called its \index{short retract}\emph{short retract} if there is a short map $\spc{X}\to \spc{S}$ that is the identity on $\spc{S}$.

\begin{thm}{Exercise}\label{ex:ultrametric-converse}
Show that any compact subspace $\spc{K}$ of an ultrametric space $\spc{X}$ is its short retract.

Construct an example of a complete ultrametric space $\spc{X}$ with a closed subspace $\spc{Q}$ that is not its short retract.
\end{thm}

The following exercise gives a sufficient condition for the existence of a short extension.

\begin{thm}{Exercise}\label{ex:petrunin-stadler}
Let $\spc{X}$ and $\spc{Y}$ be metric spaces, $A\subset \spc{X}$, and $f\:A\z\to \spc{Y}$ be a short map.
Assume $\spc{Y}$ is compact and for any finite set $F\subset \spc{X}$ there is a short map $F\to \spc{Y}$ that agrees with $f$ on $F\cap A$.
Show that there is a short map $\spc{X}\to \spc{Y}$ that agrees with $f$ on $A$.
\end{thm}

%\chapter{Space of sets}

\section{Hausdorff distance}

Let $\spc{X}$ be a metric space.
Given a subset $A\subset \spc{X}$,
consider the distance function to $A$
$$\distfun_A: \spc{X} \to [0,\infty)$$
defined as 
$$\distfun_A(x)
\df
\inf_{a\in A}\{\,\dist ax{\spc{X}}\,\}.$$

\begin{thm}{Definition}\label{def:hausdorff-convergence}
Let $A$ and $B$ be two compact subsets of a metric space $\spc{X}$.
Then the \index{Hausdorff distance}\emph{Hausdorff distance} between $A$ and $B$ is defined as 
$$|A-B|_{\Haus\spc{X}}
\df
\sup_{x\in \spc{X}}\{\,|\distfun_A(x)-\distfun_B(x)|\,\}.
$$

\end{thm}

The following observation gives a useful reformulation of the definition:

\begin{thm}{Observation}\label{obs:Haus-nbhds}
Suppose $A$ and $B$ be two compact subsets of a metric space $\spc{X}$.
Then $|A-B|_{\Haus\spc{X}}< R$ if and only if and only if 
$B$ lies in an $R$-neighborhood of $A$, 
and 
$A$ lies in an $R$-neighborhood of~$B$.
\end{thm}



Note that the set of all nonempty compact subsets of a metric space $\spc{X}$ equipped with the Hausdorff metric forms a metric space.
This new metric space will be denoted as $\Haus\spc{X}$.


\begin{thm}{Exercise}\label{ex:diam}
Let $\spc{X}$ be a metric space.
Given a subset $A\subset \spc{X}$ define its \index{diameter}\emph{diameter} as 
$$\diam A\df\sup_{a,b\in A} |a-b|.$$

Show that 
$$\diam\:\Haus\spc{X}\to \RR$$ 
is a \index{Lipschitz function}\emph{$2$-Lipschitz function};
that is,
\[|\diam A-\diam B|\le 2\cdot\dist{A}{B}{\Haus\spc{X}}\]
for any two compact nonempty sets $A,B\subset\spc{X}$.
\end{thm}


\begin{thm}{Exercise}\label{ex:Hausdorff-bry}
Let $A$ and $B$ be two compact subsets in the Euclidean plane $\RR^2$.
Assume $|A-B|_{\Haus\RR^2}<\eps$.

\begin{subthm}{ex:Hausdorff-bry:conv}
Show that $|\Conv A-\Conv B|_{\Haus\RR^2}<\eps$, where $\Conv A$ denoted the convex hull of $A$.
\end{subthm}
\begin{subthm}{ex:Hausdorff-bry:bry}
Is it true that
$|\partial A-\partial B|_{\Haus\RR^2}<\eps$,
where $\partial A$ denotes the boundary of $A$.

Does the converse hold? That is, assume $A$ and $B$ be two compact subsets in $\RR^2$
and $|\partial A-\partial B|_{\Haus\RR^2}<\eps$; 
is it true that $|A-B|_{\Haus\RR^2}\z<\eps$?
\end{subthm}

\end{thm}

Note that part \ref{SHORT.ex:Hausdorff-bry:conv} implies that $A\mapsto \Conv A$ defines a short map $\Haus\RR^2\to \Haus\RR^2$. 

\begin{thm}{Exercise}\label{ex:Haus-func}
Let $A$ and $B$ be two compact subsets in metric space $\spc{X}$.
Show that 
\[\dist{A}{B}{\Haus\spc{X}}=\sup_f\, \{\,\max_{a\in A}\{f(a)\}-\max_{b\in B}\{f(b)\,\},\]
where the least upper bound is taken for all $1$-Lipschitz functions $f$.

\end{thm}


\section{Hausdorff convergence}

\begin{thm}{Blaschke selection theorem}\label{thm:compact+Hausdorff}
A metric space $\spc{X}$ is compact if and only if
so is $\Haus\spc{X}$.
\end{thm}

The Hausdorff metric can be used to define convergence.
Namely, suppose $K_1,K_2,\dots$, and $K_\infty$ are compact sets in a metric space $\spc{X}$.
If $|K_\infty-K_n|_{\Haus\spc{X}}\to0$ as $n\to\infty$, then we say that 
the sequence $K_n$ {}\emph{converges} to $K_\infty$ \index{convergence in the sense of Hausdorff}\emph{in the sense of Hausdorff};
or we can say that $K_\infty$ is \emph{Hausdorff limit} of the sequence $K_n$.

Note that the theorem implies that from any sequence of compact sets in $\spc{X}$ one can select a subsequence that converges in the sense of Hausdorff; 
for that reason, it is called a \emph{selection} theorem. 

\parit{Proof; ``only if'' part.}
Consider the map $\iota$ that sends point $x\in \spc{X}$ to the one-point subset $\{x\}$ of $\spc{X}$.
Note that $\iota\:\spc{X}\to \Haus\spc{X}$ is distance-preserving.

Suppose that $A\subset \spc{X}$.
Note that $\diam A=0$ if and only if $A$ is a one-point set.
Therefore, from Exercise~\ref{ex:diam}, it follows 
that $\iota(\spc{X})$ is a closed subset of the compact space $\Haus\spc{X}$.
Whence $\iota(\spc{X})$, and therefore $\spc{X}$, are compact.
\qeds

To prove the ``if'' part we will need the following two lemmas.

\begin{thm}{Monotone convergence}\label{lem:decreasing-converges}
Let $K_1\supset K_2\supset\dots$ be a nested sequence of nonempty compact sets in a metric space $\spc{X}$.
Then $K_\infty\z=\bigcap_n K_n$ is the Hausdorff limit of $K_n$;
that is, $|K_\infty-K_n|_{\Haus\spc{X}}\to0$ as $n\to\infty$.
\end{thm}

\parit{Proof.}
By finite intersection property, $K_\infty$ is a nonempty compact set.

If the assertion were false, then there is $\eps>0$ such that for each $n$ 
one can choose $x_n\in K_n$
such that $\distfun_{K_\infty}(x_n)\ge\eps$.
Note that $x_n\in K_1$ for each $n$.
Since $K_1$ is compact, 
there is 
a \index{partial limit}\emph{partial limit}%
\footnote{Partial limit is a limit of a subsequence.}
 $x_\infty$ of $x_n$.
Clearly, $\distfun_{K_\infty}(x_\infty)\ge \eps$.

On the other hand, since $K_n$ is closed and $x_m\in K_n$ for $m\ge n$,
we get $x_\infty\in K_n$ for each $n$.
It follows that $x_\infty\in K_\infty$ and therefore $\distfun_{K_\infty}(x_\infty)=0$ ---
a contradiction.\qeds


\begin{thm}{Lemma}\label{lem:complete+Hausdorff}
If $\spc{X}$ is a compact metric space, then $\Haus\spc{X}$
is complete.
\end{thm}

\parit{Proof.}
Let $(Q_n)$ be a Cauchy sequence in $\Haus\spc{X}$.
Passing to a subsequence of $Q_n$ we may assume that 
$$|Q_n-Q_{n+1}|_{\Haus\spc{X}}\le \tfrac1{10^n}\eqlbl{eq:eps=1/10}$$
for each $n$.

Denote by $K_n$ the closed $\tfrac1{10^n}$-neighborhood of $Q_n$;
that is,
\begin{align*}
K_n&= \set{x\in \spc{X}}{\distfun_{Q_n}(x)\le \tfrac1{10^n}}
\end{align*}
Since $\spc{X}$ is compact so is each $K_n$.

By \ref{obs:Haus-nbhds}, $|Q_n-K_n|_{\Haus\spc{X}}\le \tfrac1{10^n}$.
From \ref{eq:eps=1/10}, we get
$K_n\supset K_{n+1}$ 
for each $n$.
Set 
$$K_\infty=\bigcap_{n=1}^\infty K_n.$$
By the monotone convergence (\ref{lem:decreasing-converges}),
 $|K_n-K_\infty|_{\Haus\spc{X}}\to 0$ as $n\to\infty$.
Since $|Q_n-K_n|_{\Haus\spc{X}}\le \tfrac1{10^n}$, we get $|Q_n-K_\infty|_{\Haus\spc{X}}\to 0$ as $n\to\infty$ --- hence the lemma.
\qeds

\begin{thm}{Exercise}\label{ex:closure-union}
Let $\spc{X}$ be a complete metric space and $K_1,K_2,\dots$ be a sequence of compact sets 
that converges in the sense of Hausdorff.
Show that the union $K_1\cup K_2\cup\dots$ is a compact closure.

Use this statement to show that in Lemma~\ref{lem:complete+Hausdorff} compactness of $\spc{X}$ can be exchanged to completeness.
\end{thm}

\parit{Proof of ``if'' part in \ref{thm:compact+Hausdorff}.}
According to Lemma~\ref{lem:complete+Hausdorff},
$\Haus\spc{X}$ is complete.
It remains to show that $\Haus\spc{X}$ is totally bounded (\ref{totally-bounded});
that is, given $\eps>0$ there is a finite $\eps$-net in $\Haus\spc{X}$.

Choose a finite $\eps$-net $A$ in $\spc{X}$.
Denote by $B$ the set of all subsets of $A$.
Note that  $B$ is a finite set in $\Haus\spc{X}$.
For each compact set $K\subset \spc{X}$, consider the subset $K'$ of all points $a\in A$
such that $\distfun_K(a)\le \eps$.
Observe that $K' \in B$ and $|K-K'|_{\Haus\spc{X}}\le\eps$.
In other words, $B$ is a finite $\eps$-net in $\Haus\spc{X}$.
\qeds

\begin{thm}{Exercise}\label{ex:Haus-length}
Let $\spc{X}$ be a complete metric space.
Show that $\spc{X}$ is a length space if and only if so is $\Haus\spc{X}$.
\end{thm}

\section{An application}

The following statement is called \index{isoperimetric inequality}\emph{isoperimetric inequality in the plane}.

\begin{thm}{Theorem}\label{thm:isoperimetric}
Among the plane figures bounded by closed curves of length at most $\ell$ the round disk has the maximal area.
\end{thm}

In this section, we will sketch a proof of the isoperimetric inequality that uses the Hausdorff convergence.
It is based on the following exercise.

\begin{thm}{Exercise}\label{ex:Huas-perimeter-area}
Let $\spc{C}$ be a subspace of $\Haus\RR^2$ formed by all compact convex subsets in $\RR^2$.
Show that perimeter\footnote{If the set degenerates to a line segment of length $\ell$, then its perimeter is defined as $2\cdot \ell$.} and area are continuous on~$\spc{C}$.
That is, if a sequence of convex compact plane sets $X_n$ converges to $X_\infty$ in the sense of Hausdorff, then 
\[\perim X_n\to \perim X_\infty\quad\text{and}\quad\area X_n\to\area X_\infty\]
as $n\to\infty$.
\end{thm}

\parit{Semiproof of \ref{thm:isoperimetric}.}
It is sufficient to consider only convex figures of the given perimeter; if a figure is not convex, pass to its convex hull and observe that it has a larger area and smaller perimeter.


Note that the selection theorem (\ref{thm:compact+Hausdorff}) together with the exercise imply the existence of figure $D$ with perimeter $\ell$ and maximal area.

It remains to show that $D$ is a round disk.
This is a problem in elementary geometry.

Let us cut $D$ along a chord $[ab]$ into two lenses, $L_1$ and $L_2$.
Denote by $L_1'$ the reflection of $L_1$ across the perpendicular bisector of $[ab]$.
Note that $D$ and $D'=L_1'\cup L_2$ have the same perimeter and area.
That is, $D'$ has perimeter $\ell$ and maximal possible area;
in particular, $D'$ is convex.

The following exercise will finish the proof.
\qeds

{

\begin{wrapfigure}{o}{57 mm}
\vskip-5mm
\centering
\includegraphics{mppics/pic-405}
\end{wrapfigure}

\begin{thm}{Exercise}\label{ex:round-disc}
Suppose $D$ is a convex figure such that for any chord $[ab]$ of $D$ the above construction produces a convex figure $D'$.
Show that $D$ is a round disk.
\end{thm}


}

Another popular way to prove that $D$ is a round disk is given by the so-called {}\emph{Steiner's 4-joint method} \cite{blaschke}.

\section{Remarks}\label{sec:H-variation}

It seems that Hausdorff convergence was first introduced by Felix Hausdorff~\cite{hausdorff}.
A couple of years later an equivalent definition was given by Wilhelm Blaschke~\cite{blaschke}.

The following refinement of the definition was introduced by  Zdeněk Frolík \cite{frolik},
later it was rediscovered by Robert Wijsman~\cite{wijsman}.  
This refinement is also called \index{Hausdorff convergence}\emph{Hausdorff convergence};
in fact, it takes an intermediate place between the original Hausdorff convergence and {}\emph{closed convergence}, also introduced by Hausdorff in \cite{hausdorff}.

\begin{thm}{Definition}\label{def:gen-Haus-conv}
Let $A_1,A_2,\dots$ be a sequence of closed sets in a metric space $\spc{X}$.
We say that the sequence $A_n$ converges to a closed set $A_\infty$ in the sense of Hausdorff if for any $x\in\spc{X}$, we have
$\distfun_{A_n}(x)\z\to \distfun_{A_\infty}(x)$ as $n\to\infty$.
\end{thm}

For example, suppose $\spc{X}$ is the Euclidean plane and $A_n$ is the circle with radius $n$ and center at the point $(n,0)$.
If we use the standard definition (\ref{def:hausdorff-convergence}), then the sequence $(A_n)$ diverges, but it converges to the $y$-axis in the sense of Definition~\ref{def:gen-Haus-conv}.

The following exercise is analogous to the Blaschke selection theorem (\ref{thm:compact+Hausdorff}) for the modified Hausdorff convergence.

\begin{thm}{Exercise}\label{ex:generalized-selection}
Let $\spc{X}$ be a proper metric space
and $A_1,A_2,\dots$ be a sequence of closed sets in~$\spc{X}$.
Assume that for some (and therefore any) point  $x\in\spc{X}$, 
the sequence $a_n=\distfun_{A_n}(x)$ is bounded.
Show that the sequence  $A_1,A_2,\dots$ has a convergent subsequence in the sense of Definition~\ref{def:gen-Haus-conv}.
\end{thm}

\chapter{Space of spaces}

\section{Gromov--Hausdorff metric}

The goal of this section is to cook up a metric space out of metric spaces.
More precisely, we want to define the so-called  Gromov--Hausdorff metric on the set of {}\emph{isometry classes} of compact metric spaces.
(Being isometric is an equivalence relation, 
and an isometry class is an equivalence class with respect to this equivalence relation.)

The obtained metric space will be denoted by $\GH$.
Given two metric spaces $\spc{X}$ and $\spc{Y}$,
denote by $[\spc{X}]$ and $[\spc{Y}]$ their isometry classes;
that is, $\spc{X}'\in [\spc{X}]$ if and only if $\spc{X}'\iso \spc{X}$.
Pedantically, the Gromov--Hausdorff distance from $[\spc{X}]$ 
to $[\spc{Y}]$ should be denoted as $|[\spc{X}]-[\spc{Y}]|_{\GH}$;
but we will write it as $|\spc{X}\z-\spc{Y}|_{\GH}$ and say (not quite correctly) 
``$|\spc{X}\z-\spc{Y}|_{\GH}$ is the Gromov--Hausdorff distance from  $\spc{X}$ 
to  $\spc{Y}$''.
In other words, from now on the term {}\emph{metric space} might also stand for its {}\emph{isometry class}.

The metric on $\GH$ is defined as the maximal metric such that \textit{the distance between subspaces in a metric space is not greater than the Hausdorff distance between them}.
Here is a formal definition:

\begin{thm}{Definition}\label{def:GH}
Let $\spc{X}$ and $\spc{Y}$ be compact metric spaces. 
The Gromov--Hausdorff distance $|\spc{X}-\spc{Y}|_{\GH}$ is defined by the following
relation.
 
Given  $r > 0$, we have that $|\spc{X}-\spc{Y}|_{\GH} < r$ if and only if there exist a metric
space $\spc{Z}$ and subspaces $\spc{X}'$ and $\spc{Y}'$ in $\spc{Z}$ that are isometric to $\spc{X}$ and $\spc{Y}$
respectively and such that $|\spc{X}'-\spc{Y}'|_{\Haus\spc{Z}} < r$. 
(Here $|\spc{X}'-\spc{Y}'|_{\Haus\spc{Z}}$ denotes the Hausdorff distance between sets $\spc{X}'$ and $\spc{Y}'$ in $\spc{Z}$.)
\end{thm}

Note that passing to the subspace $\spc{X}'\cup\spc{Y}'$ of $\spc{Z}$ does not affect the definition.
Therefore we can always assume that $\spc{Z}$ is compact.

\begin{thm}{Theorem}\label{thm:GH-is-a-metric}
The set of isometry classes of compact metric spaces equipped with Gromov--Hausdorff metric forms a metric space (which is denoted by $\GH$).

In other words, for arbitrary  compact metric spaces $\spc{X}$, $\spc{Y}$ and $\spc{Z}$ the following conditions hold:

\begin{subthm}{GH-1} $|\spc{X}-\spc{Y}|_{\GH}\ge 0$;
\end{subthm}

\begin{subthm}{GH-2} $|\spc{X}-\spc{Y}|_{\GH}=0$ if and only if $\spc{X}$ is isometric to $\spc{Y}$;
\end{subthm}

\begin{subthm}{GH-3} $|\spc{X}-\spc{Y}|_{\GH}=|\spc{Y}-\spc{X}|_{\GH}$;
\end{subthm}

\begin{subthm}{GH-4} $|\spc{X}-\spc{Y}|_{\GH}+|\spc{Y}-\spc{Z}|_{\GH}\ge |\spc{X}-\spc{Z}|_{\GH}$.
\end{subthm}
\end{thm}


Note that \ref{SHORT.GH-1}, \ref{SHORT.GH-3},
and the ``if''-part of \ref{SHORT.GH-2} follow directly from Definition \ref{def:GH}.
Part \ref{SHORT.GH-4} will be proved in Section~\ref{sec:GH-approx}.
The ``only-if''-part of \ref{SHORT.GH-2} will be proved in Section~\ref{sec:alm-isom}.

Recall that $a\cdot\spc{X}$ denotes $\spc{X}$ \index{scaled space}\emph{scaled} by factor $a>0$;
that is, $a\cdot\spc{X}$ is a metric space with the underlying set of $\spc{X}$ and the metric defined by
\[\dist{x}{y}{a\cdot\spc{X}}\df a\cdot\dist{x}{y}{\spc{X}}.\]

\begin{thm}{Exercise}\label{ex:d_GH-and-diam}
Let $\spc{X}$ be a compact metric space,
$\spc{P}$ be the one-point metric space.

Prove that 
\begin{subthm}{ex:d_GH-and-diam:point}
\[|\spc{X}-\spc{P}|_{\GH}=\tfrac12\cdot \diam \spc{X}.\]

\end{subthm}

\begin{subthm}{ex:d_GH-and-diam:scale}
\[|a\cdot\spc{X}-b\cdot \spc{X}|_{\GH}=\tfrac12\cdot|a-b|\cdot\diam\spc{X}.\]
\end{subthm}


\end{thm}

\begin{thm}{Exercise}\label{ex:rectangle}
Let $\spc{A}_r$ be a rectangle $1$ by $r$ in the Euclidean plane 
and $\spc{B}_r$ be a closed line interval of length $r$.
Show that 
\[|\spc{A}_r-\spc{B}_r|_{\GH}>\tfrac1{10}\]
for all large $r$.
\end{thm}

\begin{thm}{Advanced exercise}\label{ex:GH-inj}
Let $\spc{X}$ and $\spc{Y}$ be compact metric spaces;
denote by $\hat{\spc{X}}$ and $\hat{\spc{Y}}$ their injective envelopes (see \ref{sec:extremal-functions}).
Show that 
\[|\hat{\spc{X}}-\hat{\spc{Y}}|_{\GH}\le 2\cdot|\spc{X}- \spc{Y}|_{\GH}.\] 

\end{thm}

\section{Approximations}\label{sec:GH-approx}

\begin{thm}{Definition}\label{ex:defGHR}
Let $\spc{X}$ and $\spc{Y}$ be two metric spaces.
A relation $\approx$ between points in $\spc{X}$ and $\spc{Y}$ is called $\eps$-approximation if the following conditions hold:
\begin{itemize}
\item For any $x\in  \spc{X}$ there is $y\in \spc{Y}$ such that $x\approx y$.
\item For any $y\in  \spc{Y}$ there is $x\in \spc{X}$ such that $x\approx y$.
\item If for some $x, x'\in  \spc{X}$ and $y,y'\in \spc{Y}$ we have $x\approx y$ and $x'\approx y'$, then 
\[\bigl|\dist{x}{x'}{\spc{X}}-\dist{y}{y'}{\spc{Y}}\bigr|<2\cdot\eps.\]
\end{itemize}

\end{thm}

\begin{thm}{Exercise}\label{ex:H-R}
Let $\spc{X}$ and $\spc{Y}$ be two compact metric spaces.
Show that
\[\dist{\spc{X}}{\spc{Y}}{\GH}<\eps\]
if and only if there is an $\eps$-approximation between $\spc{X}$ and $\spc{Y}$.

In other words $\dist{\spc{X}}{\spc{Y}}{\GH}$ is the greatest lower bound of values $\eps>0$ such that  there is an $\eps$-approximation between $\spc{X}$ and $\spc{Y}$.
\end{thm}

\parit{Proof of \ref{GH-4}.}
Suppose that 
\begin{itemize}
\item $\approx_1$ is a relation between points in $\spc{X}$ and $\spc{Y}$,
\item $\approx_2$ is a relation between points in $\spc{Y}$ and $\spc{Z}$.
\end{itemize}
Consider the relation $\approx_3$ between points in $\spc{X}$ and $\spc{Z}$ such that
$x\approx_3 z$ if and only if there is $y\in  \spc{Y}$ such that 
$x\approx_1 y$ and $y\approx_2 z$.

It is straightforward to check that if $\approx_1$ is an $\eps_1$-approximation and $\approx_2$ is an $\eps_2$-approximation, then $\approx_3$ is an $(\eps_1+\eps_2)$-approximation.

Applying \ref{ex:H-R}, we get that if 
\[|\spc{X}-\spc{Y}|_{\GH}<\eps_1
\quad\text{and}\quad
|\spc{Y}-\spc{Z}|_{\GH}<\eps_2,
\]
then 
\[|\spc{X}-\spc{Z}|_{\GH}<\eps_1+\eps_2.\]
Hence \ref{GH-4} follows.
\qeds


\section{Almost isometries}\label{sec:alm-isom}

\begin{thm}{Definition} Let $\spc{X}$ and $\spc{Y}$ be metric spaces and $\eps>0$. 
A  map\footnote{possibly noncontinuous} $f\: \spc{X} \z\to \spc{Y}$ is called an \index{almost isometry}\emph{$\eps$-isometry} 
if $f(\spc{X})$ is an $\eps$-net in $\spc{Y}$ and
\[\bigl|\dist{x}{x'}{\spc{X}}-\dist{f(x)}{f(x')}{\spc{Y}}\bigr|<\eps.\]
for any $x,x'\in \spc{X}$.
\end{thm}

\begin{thm}{Exercise}\label{ex:eps-isom}
Let $\spc{X}$ and $\spc{Y}$ be compact metric spaces.

\begin{subthm}{ex:eps-isom:GH>isom}
If $\dist{\spc{X}}{\spc{Y}}{\GH}<\eps$, then there is a $2\cdot\eps$-isometry $f\:\spc{X}\to\spc{Y}$.
\end{subthm}

\begin{subthm}{ex:eps-isom:isom>GH}
If there is an $\eps$-isometry $f\:\spc{X}\to\spc{Y}$, then $\dist{\spc{X}}{\spc{Y}}{\GH}<\eps$.
\end{subthm}

\end{thm}

\parit{Proof of the ``only if''-part in \ref{GH-2}.}
\label{page:GH-2-proof}
Let $\spc{X}$ and $\spc{Y}$ be compact metric spaces.
Suppose that $\dist{\spc{X}}{\spc{Y}}{\GH}<\eps$ for any $\eps>0$;
we need to show that there is an isometry $\spc{X}\to\spc{Y}$.

By \ref{ex:eps-isom:GH>isom}, for each positive integer $n$, we can choose a $\tfrac1n$-isometry $f_n\:\spc{X}\to\spc{Y}$.

Since $\spc{X}$ is compact, 
we can choose a countable dense set
$S$ in~$\spc{X}$.
Applying the diagonal procedure if necessary, we can assume that for every $x \in S$ the sequence $f_n(x)$ 
converges in $\spc{Y}$. 
Consider the pointwise limit map  $f_\infty \: S \to \spc{Y}$,
 $$f_\infty(x) \df \lim_{n\to\infty} f_n (x)$$ for every $x \in S$. 
Since $$|f_n (x)- f_n (x')|_{\spc{Y}}\lg |x- x'|_\spc{X} \pm\tfrac1n,$$ 
we have 
$$|f_\infty(x)-f_\infty (x')|_{\spc{Y}} 
= \lim_{n\to\infty} |f_n(x)-f_n (x')|_{\spc{Y}} 
= |x -x'|_\spc{X}$$ for all
$x, x' \in S$; 
that is, the map $f_\infty\:S\to \spc{Y}$ is distance-preserving. 
Therefore, $f_\infty$ can be extended to a distance-preserving map from the whole $\spc{X}$ to $\spc{Y}$.

The latter can be done by setting 
$$f_\infty(x)=\lim_{n\to\infty} f_\infty(x_n)$$ 
for some sequence $x_n$ of points  in $S$
that converges to $x$ in $\spc{X}$.
Indeed, if $x_n\to x$, then the sequence $x_n$ is Cauchy.
Since $f_\infty$ is distance-preserving, $y_n=f_\infty(x_n)$ is also a Cauchy sequence in $\spc{Y}$;
therefore it converges.
It remains to observe that this construction does not depend on the choice of the sequence $x_n$.

This way we obtain a distance-preserving map $f_\infty\:\spc{X}\to \spc{Y}$. 
It remains to show that $f_\infty$ is surjective; that is, $f_\infty(\spc{X})=\spc{Y}$.

The same argument produces a distance-preserving map $g_\infty\:\spc{Y}\z\to \spc{X}$.
If $f_\infty$ is not surjective, then neither is the composition $f_\infty\z\circ g_\infty\:\spc{Y}\to \spc{Y}$.
So $f_\infty \z\circ g_\infty$ is a distance-preserving map from a compact space to itself which is not an isometry.
The latter contradicts \ref{ex:non-contracting-map}. 
\qeds

\section{Convergence}

The Gromov--Hausdorff metric is used to define Gromov--Hausdorff convergence.
Namely, a sequence of compact metric spaces $\spc{X}_n$ converges to compact metric spaces $\spc{X}_\infty$ in the sense of Gromov--Hausdorff if 
\[\dist{\spc{X}_n}{\spc{X}_\infty}{\GH}\to 0\quad\text{as}\quad n\to\infty.\]

This convergence is more important than the metric ---
in all applications, we use only the topology on $\GH$
and we do not care about the particular value of Gromov--Hausdorff distance between spaces.
The following observation follows from \ref{ex:eps-isom}:

\begin{thm}{Observation}\label{obs:GH-e-isom}
A sequence of compact metric spaces $(\spc{X}_n)$ converges to  $\spc{X}_\infty$ in the sense of Gromov--Hausdorff if and only if there is a sequence $\eps_n\to0+$
and an $\eps_n$-isometry $f_n\:\spc{X}_n\to \spc{X}_\infty$ for each $n$.
\end{thm}

In the following exercises \textit{converge} means in the sense of Gromov--Hausdorff.

\begin{thm}{Exercise}\label{ex:GH-SC}
\begin{subthm}{ex:GH-SC:circle}
Show that a sequence of compact simply connected length spaces cannot converge to a circle.
\end{subthm}

\begin{subthm}{ex:GH-SC:nonsc-limit}
Construct a sequence of compact simply connected length spaces that converges to a compact non-simply connected space.
\end{subthm}
\end{thm}

\begin{thm}{Exercise}\label{ex:sphere-to-ball}
\begin{subthm}{ex:sphere-to-ball:2}
Show that a sequence of length metrics on the 2-sphere cannot converge to the unit disk.
\end{subthm}

\begin{subthm}{ex:sphere-to-ball:3}
Construct a sequence of length metrics on the 3-sphere that converges to a unit 3-ball.
\end{subthm}

\end{thm}

Given two metric spaces $\spc{X}$ and $\spc{Y}$, we will write $\spc{X}\le \spc{Y}$ if there is a noncontracting map $f\:\spc{X}\to \spc{Y}$;
that is, if 
$$ |x-x'|_{\spc{X}}\le|f(x)-f(x')|_{\spc{Y}}$$
for any $x,x'\in \spc{X}$.

Further, given $\eps>0$, we will write $\spc{X}\le \spc{Y}+\eps$
if there is a map $f\:\spc{X}\to \spc{Y}$ such that 
$$|x-x'|_{\spc{X}}\le|f(x)-f(x')|_{\spc{Y}}+\eps$$
for any $x,x'\in \spc{X}$.

\section{Uniformly totally bonded families}

\begin{thm}{Definition}\label{def:utb}
A family $\spc{Q}$ of (isometry classes) of compact metric spaces is called  \index{uniformly totally bonded family}\emph{uniformly totally bonded} if it meets the following two conditions:

\begin{subthm}{}
spaces in $\spc{Q}$ have uniformly bounded diameters; that is, there is $D\in\RR$ such that
\[\diam\spc{X}\le D\]
for any space $\spc{X}$ in $\spc{Q}$.
\end{subthm}

\begin{subthm}{}
For any $\eps>0$ there is $n\in\NN$ such that any space $\spc{X}$ in $\spc{Q}$ admits an $\eps$-net with at most $n$ points.
\end{subthm}
\end{thm}

\begin{thm}{Exercise}\label{ex:utb+pack}
Let $\spc{Q}$ be a family of compact spaces with uniformly bounded diameters.
Show that $\spc{Q}$ is uniformly totally bonded if for any $\eps>0$ there is $n\in\NN$ such that 
\[\pack_\eps\spc{X}\le n\]
for any space $\spc{X}$ in $\spc{Q}$.
\end{thm}


Fix a real constant $C$.
A Borel measure $\mu$ on a metric space $\spc{X}$ is called \index{doubling space}\emph{$C$-doubling} if
\[\mu[\oBall(p,2\cdot r)]< C\cdot\mu[\oBall(p,r)]\]
for any point $p\in \spc{X}$ and any $r>0$.
A Borel measure is called \index{doubling measure}\emph{doubling} if it is {}\emph{$C$-doubling} for some real constant $C$.

\begin{thm}{Exercise}\label{pr:doubling}
Let $\spc{Q}(C,D)$ be the set of all the compact metric spaces with diameter at most $D$ that admit a $C$-doubling measure.
Show that $\spc{Q}(C,D)$ is totally bounded.
\end{thm}

Recall that we write $\spc{X}\le\spc{Y}$ if there is a distance-nondecreasing map $\spc{X}\to\spc{Y}$.

\begin{thm}{Exercise}\label{pr:under}

\begin{subthm}{pr:under:if}
Let $\spc{Y}$ be a compact metric space.
Show that the set of all spaces $\spc{X}$ such that $\spc{X}\le\spc{Y}$
is uniformly totally bounded.
\end{subthm}

\begin{subthm}{pr:under:only-if}
Show that for any uniformly totally bounded set $\spc{Q}\subset\GH$ there is a compact space $\spc{Y}$
such that $\spc{X}\le\spc{Y}$ for any $\spc{X}$ in $\spc{Q}$.
\end{subthm}

\end{thm}

\section{Gromov's selection theorem}

The following theorem is analogous to Blaschke selection theorems (\ref{thm:compact+Hausdorff}).

\begin{thm}{Gromov selection theorem}\label{thm:gromov-compactness}
Let $\spc{Q}$ be a closed subset of $\GH$.
Then $\spc{Q}$ is compact if and only if the elements of $Q$ are uniformly totally bounded.
\end{thm}

\begin{thm}{Lemma}\label{lem:GH-complete}
The space $\GH$ is complete.
\end{thm}


Let us define gluing of metric spaces that will be used in the proof of the lemma.

Suppose 
$\spc{U}$ and $\spc{V}$ are metric spaces 
with isometric closed sets $A\subset\spc{U}$ and $A'\subset\spc{V}$;
let $\iota\:A\to A'$ be an isometry.
Consider the space $\spc{W}$ of all equivalence classes in $\spc{U}\sqcup\spc{V}$ with the equivalence relation given by $a\sim\iota(a)$ for any $a\in A$.

It is straightforward to check that the following defines a metric on~$\spc{W}$:
\begin{align*}
\dist{u}{u'}{\spc{W}}&\df\dist{u}{u'}{\spc{U}}
\\
\dist{v}{v'}{\spc{W}}&\df\dist{v}{v'}{\spc{V}}
\\
\dist{u}{v}{\spc{W}}&\df\min\set{\dist{u}{a}{\spc{U}}+\dist{v}{\iota(a)}{\spc{V}}}{a\in A}
\end{align*}
where $u,u'\in \spc{U}$ and $v,v'\in \spc{V}$.

The  space $\spc{W}$ is called the \index{gluing}\emph{gluing} of $\spc{U}$ and  $\spc{V}$ along~$\iota$; briefly, we can write
$\spc{W}=\spc{U}\sqcup_\iota\spc{V}$.
If one applies this construction to two copies of one space $\spc{U}$ with a set $A\subset \spc{U}$ and the identity map $\iota\:A\to A$, then the obtained space is called the \index{double}\emph{double} of $\spc{U}$ along~$A$; this space can be denoted by $\sqcup_A^2\spc{U}$.

Note that the inclusions $\spc{U}\hookrightarrow \spc{W}$ and $\spc{V}\hookrightarrow \spc{W}$ are distance preserving.
Therefore we can and will conside $\spc{U}$ and $\spc{V}$ as the subspaces of $\spc{W}$;
this way the subsets $A$ and $A'$ will be identified and denoted further by~$A$.
Note that $A=\spc{U}\cap \spc{V}\subset \spc{W}$.

\parit{Proof.}
Let $\spc{X}_1,\spc{X}_2,\dots$ be a Cauchy sequence in $\GH$.
Passing to a subsequence if necessary, 
we can assume that $|\spc{X}_n-\spc{X}_{n+1}|_{\GH}<\tfrac1{2^n}$ for each~$n$.
In particular, for each $n$ there is a metric space $\spc{V}_n$ with distance preserving inclusions $\spc{X}_n\hookrightarrow \spc{V}_n$ and $\spc{X}_{n+1}\hookrightarrow \spc{V}_n$ such that
\[|\spc{X}_n-\spc{X}_{n+1}|_{\Haus\spc{V}_n}<\tfrac1{2^n}\]
for each $n$.
Moreover, we may assume that $\spc{V}_n=\spc{X}_n\cup\spc{X}_{n+1}$.

Let us glue $\spc{V}_1$ to $\spc{V}_2$ along $\spc{X}_2$;
to the obtained space glue $\spc{V}_3$ along $\spc{X}_3$, and so on.
The obtained metric space $\spc{W}$
has an underlying set formed by the disjoint union of all $\spc{X}_n$ such that each inclusion $\spc{X}_n\z\hookrightarrow\spc{W}$ is distance preserving and
\[|\spc{X}_n-\spc{X}_{n+1}|_{\Haus\spc{W}}<\tfrac1{2^n}\]
for each $n$.
In particular,
\[|\spc{X}_m-\spc{X}_n|_{\Haus\spc{W}}<\tfrac1{2^{n-1}}\eqlbl{eq:|x_m-X_n|}\] 
if $m>n$.

Denote by $\bar{\spc{W}}$ the completion of $\spc{W}$.
Observe that the union $\spc{X}_1\z\cup \spc{X}_2\cup\z\dots\cup \spc{X}_n$ is compact and \ref{eq:|x_m-X_n|} implies that it forms a $\tfrac1{2^{n-1}}$-net in $\bar{\spc{W}}$.
Whence $\bar{\spc{W}}$ is compact; see \ref{totally-bounded} and \ref{ex:compact-net}.

Applying Blaschke selection theorem (\ref{thm:compact+Hausdorff}),
we can pass to a subsequence of $\spc{X}_n$ that converges in $\Haus\bar{\spc{W}}$; denote its limit by $\spc{X}_\infty$.
It remains to observe that $\spc{X}_\infty$ is the Gromov--Hausdorff limit of $(\spc{X}_n)$.
\qeds

\parit{Proof of \ref{thm:gromov-compactness}; ``only if'' part.}
Suppose that there is no sequence $\eps_n\to0$ as described in \ref{def:utb}.
Observe that in this case
there is a sequence of spaces $\spc{X}_n\in\spc{Q}$ such that 
$$\pack_\delta \spc{X}_n\to\infty
\quad\text{as}\quad
n\to\infty$$
for some fixed $\delta>0$.

Since $\spc{Q}$ is compact, 
this sequence has a partial limit, say $\spc{X}_\infty\in\spc{Q}$.
Observe that $\pack_{\delta} \spc{X}_\infty=\infty$.
Therefore, $\spc{X}_\infty$ is not compact --- a contradiction.

\parit{``If'' part.}
Let $\eps_n$ be a sequence as in the definition of uniformly totally bonded families (\ref{def:utb}).

Note that $\diam \spc{X}\le \eps_1$ for any $\spc{X}\in \spc{Q}$.
Given a positive integer $n$ consider the set of all metric spaces $\spc{W}_n$
with the number of points at most $n$ and diameter $\le \eps_1$.
Note that $\spc{W}_n$ is a compact set in $\GH$ for each $n$.

Further, a subspace formed by a maximal $\eps_n$-net of any $\spc{X}\in\spc{Q}$ belongs to $\spc{W}_n$.
Therefore, $\spc{W}_n\cap\spc{Q}$ is a compact $\eps_n$-net in  $\spc{Q}$.
That is, $\spc{Q}$ has a compact $\eps$-net for any $\eps>0$.
Since $\spc{Q}$ is closed in a complete space $\GH$, it implies that $\spc{Q}$ is compact.
\qeds

\begin{thm}{Exercise}\label{ex-GH-length}
Show that the space $\GH$ is 

\begin{subthm}{ex-GH-length:length}
length,
\end{subthm}

\begin{subthm}{ex-GH-length:geodesic}
geodesic.
\end{subthm}

\end{thm}

\begin{thm}{Exercise}\label{ex:GH-po}
For two metric spaces $\spc{X}$ and $\spc{Y}$,
we write $\spc{X}\le \spc{Y}+\eps$ if
there is a map $f\:\spc{X}\to \spc{Y}$ such that 
\[\dist{x}{x'}{\spc{X}}\le \dist{f(x)}{f(x')}{\spc{Y}}+\eps\]
for any $x,x'\in \spc{X}$.

\begin{subthm}{ex:GH-po:a}
Show that 
$$\dist{\spc{X}}{\spc{Y}}{\GH'}=\inf\set{\eps>0}{\spc{X}\le \spc{Y}+\eps
\quad\text{and}\quad
\spc{Y}\le \spc{X}+\eps}$$
defines a metric on the space of (isometry classes) of compact metric spaces.
\end{subthm}

\begin{subthm}{ex:GH-po:b}
Moreover $\dist{*}{*}{\GH'}$ is equivalent to the Gromov--Hausdorff metric;
that is,
$$|\spc{X}_n-\spc{X}_\infty|_{\GH}\to 0 
\quad\iff\quad 
\dist{\spc{X}_n}{\spc{X}_\infty}{\GH'}\to 0$$ 
as $n\to\infty$.
\end{subthm}
\end{thm}

\section{Universal ambient space}

Recall that a metric space is called universal if it contains an isometric copy of any separable metric space (in particular, any compact metric space).
Examples of universal spaces include Urysohn space and $\ell^\infty$ --- the space of bounded infinite sequences with the metric defined by $\sup$-norm; see \ref{prop:sep-in-urys} and \ref{ex:frechet}.

The following proposition says that the space $\spc{W}$ in Definition~\ref{def:GH} can be exchanged to a fixed universal space.

\begin{thm}{Proposition}\label{prop:GH-with-fixed-Z}
Let $\spc{U}$ be a universal space.
Then for any compact metric spaces $\spc{X}$ and $\spc{Y}$ we have
$$|\spc{X}-\spc{Y}|_{\GH} = \inf \{|\spc{X}'-\spc{Y}'|_{\Haus\spc{U}}\}$$ 
where the greatest lower bound is taken over all pairs of sets $\spc{X}'$ and $\spc{Y}'$ in $\spc{U}$
which isometric to  $\spc{X}$ and $\spc{Y}$ respectively.  
\end{thm}




\parit{Proof of \ref{prop:GH-with-fixed-Z}.}
By the definition (\ref{def:GH}), we have that 
\[|\spc{X}-\spc{Y}|_{\GH} \le \inf \{|\spc{X}'-\spc{Y}'|_{\Haus\spc{U}}\};\]
it remains to prove the opposite inequality.

Suppse $|\spc{X}-\spc{Y}|_{\GH}<\eps$;
let $\spc{X}'$, $\spc{Y}'$ and $\spc{Z}$ be as in \ref{def:GH}.
We can assume that $\spc{Z}=\spc{X}'\cup\spc{Y}'$;
otherwise pass to the subspace $\spc{X}'\cup\spc{Y}'$ of~$\spc{Z}$.
In this case, $\spc{Z}$ is compact;
in particular, it is separable.

Since $\spc{U}$ is universal, there is a distance-preserving embedding of $\spc{Z}$ in $\spc{U}$;
let us keep the same notation for $\spc{X}'$, $\spc{Y}'$, and their images.
It follows that 
\[|\spc{X}'-\spc{Y}'|_{\Haus\spc{U}}<\eps,\]
--- hence the result.
\qeds

\begin{thm}{Exercise}\label{ex:GH-urysohn}
Let $\spc{U}_\infty$ be the Urysohn space.
Given two compact set $A$ and $B$ in $\spc{U}_\infty$ define 
\[\|A-B\|=\inf\{|A-\iota(B)|_{\Haus\spc{U}_\infty}\},\]
where the greatest lower bound is taken for all isometrics $\iota$ of $\spc{U}_\infty$.
Show that $\|{*}\z-{*}\|$ defines a pseudometric%
\footnote{The value $\|A-B\|$ is called Hausdorff distance \emph{up to isometry} from $A$ to $B$ in $\spc{U}_\infty$.}
on nonempty compact subsets of $\spc{U}_\infty$ and its corresponding metric space is isometric to $\GH$.
\end{thm}

\section{Remarks}

Suppose $\spc{X}_n\GHto \spc{X}_\infty$, then there is a metric on the disjoint union 
\[\bm{X}=\bigsqcup_{n\in \NN\cup\{\infty\}} \spc{X}_n\] 
that satisfies the following property:

\begin{thm}{Property}\label{propery:GH}
The restriction of metric on each $\spc{X}_n$ and $\spc{X}_\infty$ coincides with its original metric 
and $\spc{X}_n\Hto \spc{X}_\infty$ as subsets in $\bm{X}$.
\end{thm}


Indeed, since $\spc{X}_n\GHto \spc{X}_\infty$, there is a metric on $\spc{V}_n=\spc{X}_n\sqcup \spc{X}_\infty$ such that the restriction of metric on each $\spc{X}_n$ and $\spc{X}_\infty$ coincides with its original metric and $\dist{\spc{X}_n}{\spc{X}_\infty}{\Haus\spc{V}_n}<\eps_n$ for some sequence $\eps_n\to 0$.
Gluing all $\spc{V}_n$ along $\spc{X}_\infty$, we obtain the required space $\bm{X}$.

In other words, the metric on $\bm{X}$ defines the convergence $\spc{X}_n\z\GHto \spc{X}_\infty$.
This metric makes it possible to talk about limits of sequences $x_n\in \spc{X}_n$ as $n\to\infty$, as well as weak limits of a sequence of Borel measures $\mu_n$ on $\spc{X}_n$ and so on.

For that reason, it is useful to define convergence by specifying the metric on $\bm{X}$ that satisfies the property
for the variation of Hausdorff convergence described in Section~\ref{sec:H-variation}.
This approach is very flexible;
in particular, it can be used to define Gromov--Hausdorff convergence of arbitrary metric spaces (net necessarily compact).

In this case, a limit space for this generalized convergence is not uniquely defined.
\begin{figure}[h!]
\vskip-0mm
\centering
\includegraphics{mppics/pic-500}
\end{figure}
For example, if each space $\spc{X}_n$ in the sequence is isometric to the half-line, then its limit might be isometric to the half-line or the whole line.
The first convergence is evident and the second could be guessed from the diagram.



Often the isometry class of the limit can be fixed by marking a point $p_n$ in each space $\spc{X}_n$, it is called \index{pointed convergence}\emph{pointed Gromov--Hausdorff convergence} --- we say that $(\spc{X}_n,p_n)$ converges to $(\spc{X}_\infty,p_\infty)$ if there is a metric on $\bm{X}$ such that $\spc{X}_n\Hto \spc{X}_\infty$ and $p_n\to p_\infty$.
For example, the sequence $(\spc{X}_n,p_n)=(\RR_+,0)$ converges to $(\RR_+,0)$, while $(\spc{X}_n,p_n)=(\RR_+,n)$ converges to $(\RR,0)$.

The pointed convergence works nicely only for proper metric spaces;
the following theorem is an analog of Gromov's selection theorem for this convergence.

\begin{thm}{Theorem}\label{thm:pointed-gromov-compactness}%
Let $\spc{Q}$ be a set of isometry classes of pointed proper metric spaces
$(\spc{X},p)$.
Assume that for any $R>0$, the $R$-balls in the spaces centered at the marked points form a uniformly totally bounded family of spaces.
Then $\spc{Q}$ is precompact with respect to pointed Gromov--Hausdorff convergence. 
\end{thm}







%\chapter{Metric plus measure}

\section{Borel sets}

Let us remind few definitions assuming knowleage of basic measure theory;
comprehensive treatments can be found in \cite{billingsley} and \cite{bogachev}.

Let $\spc{X}$ be a metric space.
\index{Borel set}\emph{Borel set} is any subset of $\spc{X}$ that can be formed from open sets using the countable union, countable intersection, and complement.
In other words, Borel sets form the minimal sigma-algebra that included open sets.

A measure on metric space will be always assumed to be \index{Borel measure}\emph{Borel};
that is, it is defined on the sigma-algebra of Borel sets.
A Borel measure can be uniquely determined by its values on all open (or closed) sets.

A measure $\mu$ on $\spc{X}$ is called \index{probability measure}\emph{probability measure} if $\mu\spc{X}=1$.

Recall that \index{delta-measure}\emph{delta-measure} is a probability measure with support at one point.
Delta-measure with support in $\{x\}$ will be denoted by~\index{$\delta_{x}$}$\delta_{x}$; so
\[\text{if}\quad x\in A,\quad\text{then}\quad \delta_x(A)=1,\quad\text{otherwise}\quad\delta_x(A)=0.\]

Let $\mu_n$ be a sequence of Borel measures on $\spc{X}$.
A measure $\mu_\infty$ is a \index{weak limit}\emph{weak limit} of $\mu_n$ if 
\[\int_{\spc{X}}f\cdot(\mu_n-\mu_\infty)\to0\gamma
\quad\text{as}\quad
n\to\infty
\]
for any continuous function $f\:\spc{X}\to \RR$.

Suppose $\mu$ is a measure on a metric space $\spc{X}$ and $f\:\spc{X}\to \spc{Y}$ is a measurable map;
that is, for any Borel set $B\subset \spc{Y}$, its inverse image $f^{-1}B$ is a Borel set in $\spc{Y}$.

Consider the unit interval with its Lebesgue mesure.
If $\spc{X}$ is a complete separable metric space with probability measure $\mu$, then there is a measurable map $[0,1]\to \spc{X}$

\section{Metric on measures}

Imagine that we need to transport dirt from one pile of a given shape to make another pile of a needed shape.
Suppose that cost of transporting a unit of dirt equals to the traveled distance.%
\footnote{This is the simplest cost function one can imagine.
One may consider other cost functions; for example, if the cost proportional to the square of the distance, then the problem has more applications.}
We are free to choose a destination point for dirt from a given place.
How to minimize the total cost?

To formalize this question,
suppose that the piles of dirt are described by Borel probability measures $\mu$ and $\nu$ on a metric space~$\spc{X}$.

To describe where each piece of dirt goes, we will use the so-called \index{plan}\emph{plan} for $\mu$ and $\nu$.
It is a probability measure $\pi$ on the product $\spc{X}\times\spc{X}$ such that 
for all measurable sets $A \subset \spc{X}$, we have 
\[\mu A= \pi [A \times \spc{X}],
\quad\text{and}\quad
\nu A=\pi [\spc{X}\times A].
\eqlbl{eq:marginals}\]
Equivalently it can be described as a measure that satisfies the following identity
\begin{align*}
\int_{(x,y)\in \spc{X}\times\spc{X}}f(x)\cdot g(y) \cdot \pi
&=
\int_{x\in \spc{X}}f(x)\cdot \mu
\oldcdot \int_{y\in \spc{X}}g(y)\cdot \nu,
\end{align*}
for any continuous functions $f,g\:\spc{X}\to \RR$.

Given a measure $\pi$ on $\spc{X}\times\spc{X}$, the measures $\mu$ and $\nu$ defined by \ref{eq:marginals} are called first and second \index{marginal}\emph{marginals} of $\pi$;
so the statement \textit{$\pi$ is a plan for $\mu$ and $\nu$} is equivalent to \textit{$\mu$ and $\nu$ are the first and second marginals of $\pi$ respectively}.

\begin{thm}{Claim}\label{clm:plan-exists}
There is a plan $\pi$ for any two given Borel probability measures $\mu$ and $\nu$.
\end{thm}

The plan constructed in the proof distributes equally each piece of dirt in the new pile.
As we will see this plan is usually far from optimum.

\parit{Proof.}
Consider the measure $\pi$ that is uniquely defined  defined by the identity
\[\pi(A\times B)=\mu A\cdot \mu B\]
for any Borel subsets $A,B\subset\spc{X}$.
Observe that $\pi$ is a plan for $\mu$ and~$\nu$.
\qeds

Denote by $\Pi(\mu,\nu)$ the set of all plans for $\mu$ and $\nu$;
by \ref{clm:plan-exists}, $\Pi(\mu,\nu)\z\ne\emptyset$.
It is straightforwrd to check that the following formula defines a metric on the space of probability measures on $\spc{X}$.
\[\dist{\mu}{\nu}{\Wass_1\spc{X}}
\df
\inf_{\pi\in\Pi(\mu,\nu)}
\left\{\,\int_{(x,y)\in\spc{X}\times\spc{X}}\dist{x}{y}{\spc{X}}\cdot\pi\,\right\}.\]
This metric is called \index{Wasserstein distance}\emph{Wasserstein distance of order 1} between $\mu$ and $\nu$.

In genereral, the Wasserstein distance $\dist{\mu}{\nu}{}$ might take infinite value, but all measures with compact support lie on finite distance from each other in the obtained $\infty$-metric space.
The metric component of these measures is called \index{Wasserstein space}\emph{Wasserstein space} of order 1 over $\spc{X}$; 
it is denoted by $\Wass_1\spc{X}$.
In other words, $\Wass_1\spc{X}$ is the space of all Borel probability measures $\mu$ such that 
$\int\distfun_p\cdot\mu<\infty$ for some (and therefore any) point $p\in \spc{X}$.

\begin{thm}{Exercise}\label{ex:wasserstein-infty}
Construct two Borel probability measures $\mu$ and $\nu$ on $\RR$ with Wasserstein distance $\dist{\mu}{\nu}{}=\infty$.
\end{thm}


\begin{thm}{Exercise}\label{ex:wasserstein-compact}
Show that $\Wass_1\spc{X}$ is a compact if and only if so is~$\spc{X}$.
\end{thm}

\begin{thm}{Exercise}\label{ex:wasserstein-length}
Show that the Wasserstein space $\Wass_1\spc{X}$ is a geodesic space for any metric space $\spc{X}$.
\end{thm}

\section{Optimal plan}

A plan $\pi$ for given measures $\mu$ and $\nu$ is called \index{optimal plan}\emph{optimal} if 
\[\dist{\mu}{\nu}{\Wass_1\spc{X}}
=\int_{(x,y)\in\spc{X}\times\spc{X}}\dist{x}{y}{\spc{X}}\cdot\pi.\]

\begin{thm}{Theorem} %Vilani:Theorem 1.4
Let $\mu$ and $\nu$ be probability Borel measures on a compact metric space $\spc{X}$.
Then there is an optimal plan $\pi$ for $\mu$ and~$\nu$.
\end{thm}

\parit{Proof.}
By the definition of Wasserstein distance, we can choose a sequence of plans $\pi_n$ for $\mu$ and $\nu$ such that 
\[\int_{(x,y)\in\spc{X}\times\spc{X}}\dist{x}{y}{\spc{X}}\cdot\pi_n\to \dist{\mu}{\nu}{\Wass_1\spc{X}}\]
as $n\to \infty$.

Observe that $\pi_n$ has a weak partial limit, say $\pi$.
Moreover, $\pi$ is an optimal plan for $\mu$ and $\nu$.
\qeds

\begin{thm}{Theorem}
Any optimal plan $\pi$ is \index{cyclic monotonicity}\emph{cyclically monotonic}.
That is, suppose $\pi$ is an optimal plan for probability measures $\mu$ and $\nu$ on a metric space $\spc{X}$.
Then any sequence of pairs $(x_1,y_1),\dots,(x_n,y_n)\in\supp\pi\subset\spc{X}\times\spc{X}$ we have
\[\sum_i\dist{x_i}{y_i}{}
\le
\sum_i\dist{x_{i+1}}{y_i}{},\]
here the index $i$ in the sum is taken modulo $n$; in particular $x_{n+1}\z=x_1$.
\end{thm}

\parit{Proof.}
Assume that the cyclic monotonicity does not hold;
that is,
\[R=\sum_i\dist{x_i}{y_i}{}
-
\sum_i\dist{x_{i+1}}{y_i}{}>0,\]
for some $(x_0,y_0),\dots,(x_n,y_n)\in\supp\pi$.
We need to show that $\pi$ is not optimal;
in other words we need to construct another plan $\pi'$ for $\mu$ and $\nu$ such that 
\[\int_{(x,y)\in\spc{X}\times\spc{X}}\dist{x}{y}{\spc{X}}\cdot(\pi'-\pi)<0.\eqlbl{pi'<pi}\]

Assume $\spc{X}$ is finite.
In this case we can choose $\eps>0$ such that 
$\pi\{(x_i,y_i)\}>\eps$ for each $i$.
Let
\[\pi'=\pi-\eps\cdot\sum_i(\sigma_i-\sigma_i')\eqlbl{eq:pi'}\]
where $\sigma_i=\delta_{(x_i,y_i)}$ and $\sigma_i'=\delta_{(x_{i+1},y_i)}$.
It remains to observe that $\pi'$ is a plan for $\mu$ and $\nu$ that satisfies \ref{pi'<pi}.

The general case is similar, we only need to redefine $\eps$, $\sigma_i$, and~$\sigma_i'$.
Note that given $r>0$, we can choose a probability measures $\sigma_i$ with support in $\oBall((x_i,y_i),r)_{\spc{X}\times\spc{X}}$ such that $\eps\cdot \sigma_i<\pi$ for some fixed $\eps>0$ and every $i$.
Further, denote by $\zeta_i$ and $\eta_i$ the first and second marginals of $\sigma_i$.
Observe that $\supp\zeta_i\subset\oBall(x_i,r)$ and $\supp\eta_i\subset\oBall(y_i,r)$ for all $i$.
Let $\sigma_i'$ be a plan for $\zeta_{i+1}$ and $\eta_i$.
Evidently 
\begin{align*}
\int_{(x,y)\in\spc{X}\times\spc{X}}\dist{x}{y}{}\cdot \sigma_i
&\lessgtr
\dist{x_i}{y_i}{}\pm 2\cdot r,
\\
\int_{(x,y)\in\spc{X}\times\spc{X}}\dist{x}{y}{}\cdot \sigma_i'
&\lessgtr
\dist{x_{i+1}}{y_i}{}\pm 2\cdot r.
\end{align*}
Taking $r<\tfrac R{10\cdot n}$, we get  \ref{pi'<pi}. 
\qeds




\section{Capitalistic approach}

Imagine that measures $\mu$ and $\nu$ describe the production and consumer of beer in the space.
A transportaition company transports beer from $\mu$ to $\nu$ and want to maximize its profit by adjusting price $f(x)$ of beer the point $x$; they buy beer at price $f(x)$ per unit, move it to an other point $y$ and sale it with (presumably higher) price $f(y)$.
However, the function $f$ is 1-Lipschitz condition;
otherwise the profit goes to second-hand dealers, or maybe it is a government regulation.
In other words, we need to maximize the following expression
\[\int_{\spc{X}} f\cdot(\mu-\nu)\]
for all $1$-Lipschitz functions $f$.
The maximal profit defines a metric

\begin{thm}{Theorem}
Let $\mu$ and $\nu$ be probability Borel measures on a compact metric space $\spc{X}$.
Then
\[\dist{\mu}{\nu}{\Wass_1\spc{X}}=\sup\int_{\spc{X}} f\cdot(\mu-\nu),\]
where the least upper bound is taken for all $1$-Lipschitz functions $f\:\spc{X}\z\to\RR$.
\end{thm}

The definition of Wassershtein metric described in the previous section reminds communist's planed economy.
The right-hand side in the above equation reminds capitalistic system.
Indeed, think that measures $\mu$ and $\nu$ describe the production and consumer of beer in the space.
A transportaition company trnasports beer from $\mu$ to $\nu$ and want to maximize its profit by adjusting price $f(x)$ of beer the point $x$.
However, the function $f$ is 1-Lipschitz condition --- this is a government regulation.




\parit{Proof.}
By the definition of Wasserstein metric, we can choose a sequence $\pi_n$ of plans  

Let us choose an optimal plan $\pi$ for $\mu$ and $\nu$; it exists by ???.
We need to find a 1-Lipschitz function $f\:\spc{X}\to\RR$ such that 
\[
\int_{\spc{X}} f\cdot(\mu-\nu)=\int_{(x,y)\in\spc{X}\times\spc{X}}\dist{x}{y}{\spc{X}}\cdot \pi.
\eqlbl{eq:f(mu-nu)}
\]

Choose $x_0\in \supp\mu$.
Note that adding a constant to $f$ does not change the left hand side in \ref{eq:f(mu-nu)}.
Therefore we can assume assume that $f(x_0)=0$ and set
\[f(x)=\sup\{\,|x_0-y_0|+\dots+|x_n-y_n|-(|x_1-y_0|+\dots+|x_n-y_{n-1}|)-|x-y_n|\,\}\]
where the least upper bound is taken for all sequences $(x_0,y_0),\z\dots,(x_n,y_n)\z\in\supp\pi$.

\qeds

\section{Metric-measure space}

A metric measure space is a metric $\spc{X}$ space with a choice of Borel probability measure $\vol$ on it.
In a metric-measure we ignore sets with vanishing volume; in other words, passing from $\spc{X}$ to the support of $\vol$ does not change the metric-measure space.

Alternatively we may start with unit interval $[0,1]$ equipped with Lebesgue measure and equip it with measurable semimetric $[0,1]\times [0,1]\to \RR$.





\section{Space of measures}


It can be equipped with the \index{Wasserstein metric}\emph{Wasserstein metric}
\[\dist{\mu}{\nu}{}\df\sup\left\{\,\int_{\spc{X}} f\cdot(\mu-\nu)\,\right\},\]
where the least upper bound is taken for all $1$-Lipschitz functions $f\:\spc{X}\to\RR$.

The Wasserstein distance $\dist{\mu}{\nu}{}$ might take infinite value, but all measures with compact support lie on finite distance from each other in the obtained $\infty$-metric space.
The metric component of these measures is called \index{Wasserstein space}\emph{Wasserstein space} of order 1 over $\spc{X}$; 
it is denoted by $\Wass_1\spc{X}$.



\section{Misc}

Suppose $\pi_n$ is a sequence of plans for $\mu$ and $\nu$.
Assume that $\pi_n$ weakly converges to a probability measure $\pi$ on $\spc{X}\times\spc{X}$.

is a weak limit of a sequence of plans $\pi_n$, then $\pi$ is a plan for $\mu$ and $\nu$ if for each $n$ $\pi_n$ is a plane for $\mu$ and $\nu$ 

Suppose that $f\:\spc{X}\to \RR$ is a 1-Lipschitz function,
so $f(x)-f(y)\le\dist{x}{y}{\spc{X}}$ for any $x,y\in \spc{X}$.
It follows that 
\begin{align*}
\int_{\spc{X}} f\cdot(\mu-\nu)&=\int_{x\in\spc{X}}f(x)\cdot\mu-\int_{y\in\spc{X}}f(y)\cdot\nu=
\\
&=\int_{(x,y)\in\spc{X}\times\spc{X}}[f(x)-f(y)]\cdot \pi\le
\\
&\le\int_{(x,y)\in\spc{X}\times\spc{X}}\dist{x}{y}{\spc{X}}\cdot \pi,
\end{align*}
where $\pi$ is a plan for $\mu$ and $\nu$.
By the definition of Wasserstein metric, we get  
\[\dist{\mu}{\nu}{\Wass_1\spc{X}}\le \int_{(x,y)\in\spc{X}\times\spc{X}}\dist{x}{y}{\spc{X}}\cdot\pi\eqlbl{wass=<int.plan}\]
for any plan $\pi$.

Next we want to show that equality holds in \ref{wass=<int.plan} for some plan $\pi$; such plans will be called \index{optimal plan}\emph{optimal}.


\parit{Proof.}
Choose a point $x_0\in \supp\mu$.
Given  $p\in \spc{X}$,
let
\[f(p)=\inf\left\{\sum_{i=0}^n\dist{x_i}{y_i}{}-\sum_{i=0}^n\dist{x_{i+1}}{y_i}{}-\dist{y_n}{p}{}\right\},
\eqlbl{eq:f(p)}\]
where the least upper bound is taken for all sequences of pairs 
\[(x_0,y_0),\z\dots,(x_n,y_n)\in \supp\pi.\eqlbl{eq:sequence}\]

Fix a sequence as in \ref{eq:sequence} and  denote by $F_\sigma(p)$ the expression under infimum in \ref{eq:f(p)}.

Let us show that 
\[F_\sigma(x_0)\ge 0.\]
Indeed, suppose $F_\sigma(x_0)<-\eps<0$.
Since $(x_i,y_i)\in \supp\pi$, we have $x_i\in\supp\mu$ and $y_i\in\supp\nu$ for any $i$.
Therefore we can choose sets $X_i\subset \oBall(x_i,\tfrac{\eps}{10\cdot n})$ and $Y_i\subset \oBall(y_i,\tfrac{\eps}{10\cdot n})$ such that 
$\mu(X_0)=\nu(Y_0)=\dots=\mu(X_n)=\nu(Y_n)$



Let us denote by $F(p)$ the expression under infimum in \ref{eq:f(p)}.
By the triangle inequality, 
\[F(q)\le F(p)+\dist{p}{q}{}.\]
Passing to the least upper bound in this inequality, we get
\[f(q)\le f(p)+\dist{p}{q}{}\]
for any $p,q\in\spc{X}$.
Hence $f$ is a 1-Lipschitz function.

Further, let us show that
\[(x,y)\in\supp\pi
\quad\Longrightarrow\quad
f(y)-f(x)=\dist{x}{y}{}\]





Suppose that cyclic monotonicity fails;
that is, there is a sequence of pairs $(x_1,y_1),\dots,(x_n,y_n)\in\spc{X}\times\spc{X}$ such that
\[\dist{x_1}{y_1}{}+\dots+\dist{x_n}{y_n}{}
>
\dist{x_1}{y_2}{}+\dots+\dist{x_{n-1}}{y_n}{}+\dist{x_{n}}{y_1}{}.\]
In this case, it would be more optimal to transport measure from a neighborhood of $x_i$ to a neighborhood of $y_{i+1}$ (
here and further we assume that the indexes are taken modulo $n$, so $n+1=1$).
The latter contradicts optimality of $\pi$.

The following argument makes it precise.
Choose small $\eps>0$.
For each $n$,
choose disjoint sets $X_i$ and $Y_i$ in $\eps$-neighborhood of $x_i$ and $y_i$
such that for some $\delta>0$ we have 
\[\pi [X_i\times Y_i]=\delta\]
for each $i$.

Let us modify the plan $\pi$ in the union $X_1\times Y_1 \cup\dots\cup X_n\times Y_n$ and such that 
$\pi'(X_i\times Y_{i+1})=\delta$ for each $i$;


Observe that
\[\int_{(x,y)\in\spc{X}\times\spc{X}}\dist{x}{y}{\spc{X}}\cdot(\pi'-\pi)>\]
\qeds

\chapter{Ultralimits}

Ultralimits provide a very general way to pass to a limit.
This procedure works for \textit{any} sequence of metric spaces, its result reminds limit in the sense of Gromov--Hausdorff, but has some strange features; for example, the limit of a constant sequence of spaces $\spc{X}_n=\spc{X}$ is \textit{not} $\spc{X}$ (see \ref{ex:ultrapower:compact}).

In geometry, ultralimits are used only as a canonical way to pass to a convergent subsequence.
It is useful in the proofs where one needs to repeat ``pass to convergent subsequence'' too many times.

This lecture is based on the introductory part of the paper by Bruce Kleiner and Bernhard Leeb \cite{kleiner-leeb}.

\section{Faces of ultrafilters}

Recall that $\NN$ denotes the set of natural numbers, $\NN=\{1,2,\dots\}$

\begin{thm}{Definition}
A finitely additive measure $\omega$ 
on $\NN$ 
is called an \index{ultrafilter}\emph{ultrafilter} if it satisfies the following condition:
\begin{subthm}{}
$\omega(\NN)=1$ and 
$\omega(S)=0$ or $1$ for any subset $S\subset \NN$.
\end{subthm}
An ultrafilter $\omega$ is called 
\emph{nonprincipal}\index{ultrafilter!nonprincipal ultrafilter}\index{nonprincipal ultrafilter} if in addition 
\begin{subthm}{}
$\omega(F)=0$ for any finite subset $F\subset \NN$.
\end{subthm}
\end{thm}

If $\omega(S)=0$ for some subset $S\subset \NN$,
we say that $S$ is \index{$\omega$-small}\emph{$\omega$-small}. 
If $\omega(S)=1$, we say that $S$ contains \index{$\omega$-almost all}\emph{$\omega$-almost all} elements of $\NN$.

\begin{thm}{Advanced exercise}\label{ex:ultrakatetov}
Let $\omega$ be an ultrafilter on $\NN$ and $f\:\NN\z\to \NN$.
Suppose that $\omega(S)\le \omega(f^{-1}(S))$ for any set $S\subset \NN$.
Show that $f(n)=n$ for $\omega$-almost all $n\in\NN$.
\end{thm}


\parbf{Classical definition.}
More commonly, a nonprincipal ultrafilter is defined as a collection, say $\mathfrak{F}$, of sets in $\NN$ such that
\begin{enumerate}
\item\label{filter:supset} if $P\in \mathfrak{F}$ and $Q\supset P$, then $Q\in \mathfrak{F}$,
\item\label{filter:cap} if $P, Q\in \mathfrak{F}$, then $P\cap Q\in \mathfrak{F}$,
\item\label{filter:ultra} for any subset $P\subset\NN$, either $P$ or its complement is an element of $\mathfrak{F}$.
\item\label{filter:non-prin} if $F\subset \NN $ is finite, then $F\notin \mathfrak{F}$.
\end{enumerate}
Setting $P\in\mathfrak{F}\Leftrightarrow\omega(P)=1$ makes these two definitions equivalent.

A nonempty collection of sets $\mathfrak{F}$ that does not include the empty set and satisfies only conditions \ref{filter:supset} and \ref{filter:cap} is called a \index{filter}\emph{filter}; 
if in addition $\mathfrak{F}$ satisfies condition \ref{filter:ultra} it is called an \index{ultrafilter}\emph{ultrafilter}.
From Zorn's lemma, it follows that every filter contains an ultrafilter.
Thus there is an ultrafilter $\mathfrak{F}$ contained in the filter of all complements of finite sets; clearly, this ultrafilter $\mathfrak{F}$ is nonprincipal.


\parbf{Stone--\v{C}ech compactification.}
Given a set $S\subset \NN$, consider subset $\Omega_S$ of all ultrafilters $\omega$ such that $\omega(S)=1$.
It is straightforward to check that the sets $\Omega_S$ for all $S\subset \NN$ form a topology on the set of ultrafilters on $\NN$. 
The obtained space was first considered by Andrey Tikhonov and called \index{Stone--\v{C}ech compactification}\emph{Stone--\v{C}ech compactification} of $\NN$;
it is usually denoted as $\beta\NN$.

Let $\omega_n$ denotes the principal ultrafilter such that $\omega_n(\{n\})=1$; that is, $\omega_n(S)=1$ if and only if $n\in S$.
Note that $n\mapsto\omega_n$ defines a natural embedding $\NN\hookrightarrow\beta\NN$. 
Using the described embedding, we can (and will) consider $\NN$ as a subset of $\beta\NN$.

The space $\beta\NN$ is the maximal compact Hausdorff space that contains $\NN$  as an everywhere dense subset.
More precisely, for any compact Hausdorff space $\spc{X}$ 
and a map $f\:\NN\to \spc{X}$ there is a unique continuous map $\bar f\:\beta\NN\to X$ such that the restriction $\bar f|_\NN$ coincides with $f$. 

\section{Ultralimits of points}
\label{ultralimits}\index{ultralimit}

Let us fix a nonprincipal  ultrafilter $\omega$ once and for all.

Assume $x_n$ is a sequence of points in a metric space $\spc{X}$. 
Let us define the \index{$\omega$-limit}\emph{$\omega$-limit} of a sequence $x_1,x_2,\dots$ as the point $x_\omega$ 
such that for any $\eps>0$, point $x_n$ lie in $\oBall(x_\omega,\eps)$ for $\omega$-almost all $n$; 
that is,
\[\omega\set{n\in\NN}{\dist{x_\omega}{x_n}{}<\eps}=1.\]
In this case, we will write 
\[x_\omega=\lim_{n\to\omega} x_n
\ \ \text{or}\ \ 
x_n\to x_\omega\ \text{as}\ n\to\omega.\]

For example, if $\omega_n$ is the \textit{principal} ultrafilter such that $\omega_n\{n\}=1$ for some $n\in\NN$, then
$x_{\omega_n}=x_n$.

The sequence $x_n$ can be regarded as a map $\NN\to\spc{X}$ defined by $n\mapsto x_n$.
If $\spc{X}$ is compact, then the map $\NN\to\spc{X}$ can be extended to a continuous map $\beta\NN\to\spc{X}$ from the Stone--\v{C}ech compactification $\beta\NN$ of $\NN$.
Then the $\omega$-limit $x_\omega$ can be regarded as the image of $\omega$.

Note that the $\omega$-limits of a sequence and its subsequence may differ.
For example, sequence $y_n=-(-1)^n$ is a subsequence of $x_n=(-1)^n$, but for any ultrafilter $\omega$, we have
\[\lim_{n\to\omega}x_n
\ne
\lim_{n\to\omega}y_n.\] 

\begin{thm}{Proposition}\label{prop:ultra/partial}
Let $\omega$ be a nonprincipal ultrafilter.
Assume $x_n$ is a sequence of points in a metric space $\spc{X}$
and $x_n\to  x_\omega$ as $n\to\omega$.
Then $x_\omega$ is a partial limit of the sequence $x_n$;
that is, there is a subsequence $(x_n)_{n\in S}$ that converges to $x_\omega$ in the usual sense.
\end{thm}

\parit{Proof.}
Given $\eps>0$, 
set $S_\eps=\set{n\in\NN}{\dist{x_n}{x_\omega}{}<\eps}$.

Note that $\omega(S_\eps)=1$ for any $\eps>0$.
Since $\omega$ is nonprincipal, the set $S_\eps$ is infinite.
Therefore, we can choose an increasing sequence $n_k$
such that $n_k\in S_{\frac1k}$ for each $k\in \NN$.
Clearly, $x_{n_k}\to x_\omega$ as $k\to\infty$.
\qeds

The following proposition 
is analogous to the statement that any sequence in a compact metric space 
has a convergent subsequence;
it can be proved the same way.

\begin{thm}{Proposition}\label{prop:ultra/compact}
Let $\spc{X}$ be a compact metric space.
Then
any sequence $x_n$ of points in $\spc{X}$ has a unique $\omega$-limit $x_\omega$.

In particular, a bounded sequence of real numbers has a unique $\omega$-limit.
\end{thm}

The following lemma is an ultralimit analog of the Cauchy convergence test.

\begin{thm}{Lemma}\label{lem:X-X^w}
Let $x_n$ be a sequence of points in a complete space~$\spc{X}$. 
Assume for each subsequence $y_n$ of $x_n$, 
the $\omega$-limit 
\[y_\omega=\lim_{n\to\omega}y_{n}\in \spc{X}\]
is defined and does not depend on the choice of subsequence, then the sequence $x_n$ converges in the usual sense.
\end{thm}

\parit{Proof.} If $x_n$ is not a Cauchy sequence, then for some $\eps>0$, there is a subsequence $y_n$ of $x_n$ such that $\dist{x_n}{y_n}{}\ge\eps$ for all $n$.

It follows that $\dist{x_\omega}{y_\omega}{}\ge \eps$ --- a contradiction.\qeds

\begin{thm}{Exercise}\label{ex:linear}
Recall that $\ell^\infty$ denotes the space of bounded sequences of real numbers.
Show that there is a linear functional $L\:\ell^\infty\to\RR$ such that
for any sequence $\bm{s}=(s_1,s_2,\dots)\in \ell^\infty$ the image $L(\bm{s})$ is a partial limit of $s_1,s_2,\dots$
\end{thm}

\begin{thm}{Exercise}\label{ex:ultrakatetov+}
Suppose that $f\:\NN\to\NN$ is a map such that 
\[\lim_{n\to\omega}x_n=\lim_{n\to\omega}x_{f(n)}\]
for any bounded sequence $x_n$ of real numbers.
Show that $f(n)=n$ for $\omega$-almost all $n\in\NN$.
\end{thm}

\section{An illustration}

\begin{thm}{Claim}
Let $\spc{X}$ and $\spc{Y}$ be compact spaces.
Suppose that for every $n\in\NN$ there is a $\tfrac1n$-isometry $f_n\:\spc{X}\to \spc{Y}$.
Then there is an isometry $\spc{X}\to \spc{Y}$.
\end{thm}

We give a proof of this claim only as an illustration for ulralimits.

\parit{Proof.}
Consider the $\omega$-limit $f_\omega$ of~$f_n$;
according to \ref{prop:ultra/compact}, $f_\omega$ is defined.
Since 
\[|f_n(x)-f_n(x')|\lege |x-x'|\pm\tfrac1n\]
we get that 
\[|f_\omega(x)-f_\omega(x')|= |x-x'|\]
for any $x,x'\in \spc{X}$;
that is, $f_\omega$ is distance-preserving.

Further, since $f_n$ is a $\tfrac1n$-isometry,
for any $y\in \spc{Y}$ there is a sequence $x_n\in \spc{X}$ such that $|f_n(x_n)-y|\le \tfrac1n$.
Therefore,
\[f_\omega(x_\omega)=y,\]
where $x_\omega$ is the $\omega$-limit of $x_n$;
that is, $f_\omega$ is onto.

It follows that $f_\omega\:\spc{X}\to\spc{Y}$ is an isometry.
\qeds

\section{Ultralimits of spaces}\label{sec:Ultralimit of spaces}

Recall that $\omega$ denotes a nonprincipal ultrafilter on the set of natural numbers.

Let $\spc{X}_n$ be a sequence of metric spaces.
Consider all sequences of points $x_n\in \spc{X}_n$.
On the set of all such sequences,
define a semimetric by
\[\dist{(x_n)}{(y_n)}{}
=
\lim_{n\to\omega} \dist{x_n}{y_n}{\spc{X}_n}.
\eqlbl{eq:olim-dist}\]
Note that the $\omega$-limit on the right-hand side is always defined 
and takes a value in $[0,\infty]$. 
(The $\omega$-convergence to $\infty$ is defined analogously to the usual convergence to $\infty$).

Set $\spc{X}_\omega$ to be the corresponding metric space; 
that is, the underlying set of $\spc{X}_\omega$ is formed by classes of equivalence of sequences of points $x_n\in\spc{X}_n$ 
defined by 
\[(x_n)\sim(y_n)
\ \Leftrightarrow\ 
\lim_{n\to\omega} \dist{x_n}{y_n}{}=0\]
and the distance is defined by \ref{eq:olim-dist}.

The space $\spc{X}_\omega$ is called the \index{$\omega$-limit space}\emph{$\omega$-limit} of $\spc{X}_n$.
(It is also called \index{$\omega$-product}\emph{$\omega$-product}; this term is motivated by the fact that   
$\spc{X}_\omega$ is a quotient of the product $\prod\spc{X}_n$)
Typically  $\spc{X}_\omega$ will denote the  
$\omega$-limit of sequence $\spc{X}_n$;
we may also write  
\[\spc{X}_n\to\spc{X}_\omega\ \ \text{as}\ \  n\to\omega\ \ \text{or}\ \ \spc{X}_\omega=\lim_{n\to\omega}\spc{X}_n.\]

Given a sequence $x_n\in \spc{X}_n$,
we will denote by $x_\omega$ its equivalence class which is a point in $\spc{X}_\omega$;
it can be written as
\[x_n\to x_\omega \ \ \text{as}\ \  n\to\omega,\ \ \text{or}\ \ x_\omega=\lim_{n\to\omega} x_n.\]

\begin{thm}{Observation}\label{obs:ultralimit-is-complete}
The $\omega$-limit of any sequence of metric spaces is complete. 
\end{thm}

We will repeat the proof of \ref{ex:complete-completion} using a slightly different language.

\parit{Proof.}
Let $\spc{X}_n$ be a sequence of metric spaces and $\spc{X}_n\to\spc{X}_\omega$ as $n\to\omega$.
Choose a Cauchy sequence $x_1,x_2,\dots{}\in\spc{X}_\omega$.
Passing to a subsequence, we can assume that $\dist{x_k}{x_{m}}{\spc{X}_\omega}<\tfrac1{k}$ if $k<m$.

Choose a double sequence $x_{n,m}\in \spc{X}_n$ such that for any fixed $m$ we have $x_{n,m}\to x_m$ as $n\to\omega$.
Note that for any $k<m$ the inequality $\dist{x_{n,k}}{x_{n,m}}{}<\tfrac1{k}$ holds for $\omega$-almost all $n$.

Given $m\in\NN$, consider the subset $S_m\subset\NN$ defined by
\[S_m=\set{n\ge m}{\dist{x_{n,k}}{x_{n,l}}{}<\tfrac1{k} \quad\text{for all}\quad k<l\le m}.\]
Note that 
\begin{itemize}
\item $\NN= S_1\supset S_2\supset\dots$
\item $\omega(S_m)=1$ for each $m$, and
\item $\min S_m\ge m$.
\end{itemize}

Consider the sequence $y_n=x_{n,m(n)}$, where $m(n)$ is the largest value such that $n\in S_{m(n)}$;
from above, $m(n)\le n$.
Denote by $y_\omega\in \spc{X}_\omega$ the $\omega$-limit of $y_n$.

Observe that $|y_m-x_{n,m}|<\tfrac1{m}$ for $\omega$-almost all $n$.
It follows that $|x_m-y_\omega|\le \tfrac1{m}$ for any $m$.
Therefore, $x_n\to y_\omega$ as $n\to \infty$.
That is, any Cauchy sequence in $\spc{X}_\omega$ converges.
\qeds

\begin{thm}{Observation}\label{obs:ultralimit-is-geodesic}
The $\omega$-limit of any sequence of length spaces is geodesic. 
\end{thm}

\parit{Proof.}
If $\spc{X}_n$ is a sequence of length spaces, then for any sequence of pairs $x_n, y_n\in X_n$ there is a sequence of $\tfrac1n$-midpoints $z_n$.

Let $x_n\to x_\omega$, $y_n\to y_\omega$ and $z_n\to z_\omega$ as $n\to \omega$.
Note that $z_\omega$ is a midpoint of $x_\omega$ and $y_\omega$ in $\spc{X}_\omega$.

By Observation~\ref{obs:ultralimit-is-complete}, $\spc{X}_\omega$ is complete.
Applying Lemma~\ref{lem:mid>geod} we get the statement.
\qeds


\begin{thm}{Exercise}\label{ex:lim(tree)}
Show that an ultralimit of metric trees is a metric tree. 
\end{thm}

\begin{thm}{Exercise}\label{ex:ultracompact}
Suppose that $\spc{X}_\infty$ and $\spc{X}_1,\spc{X}_2,\dots$ are compact metric spaces.
Assume $\spc{X}_n\GHto\spc{X}_\infty$.
Show that $\spc{X}_\omega\iso\spc{X}_\infty$.
\end{thm}


\section{Ultrapower}

If all the metric spaces in the sequence are identical $\spc{X}_n=\spc{X}$, 
its $\omega$-limit 
$\lim_{n\to\omega}\spc{X}_n$
is denoted by $\spc{X}^\omega$
and called \index{ultrapower}\index{$\omega$-power}\emph{$\omega$-power} of $\spc{X}$ (also known as \index{ultracompletion}\index{$\omega$-completion}\emph{$\omega$-completion}).



\begin{thm}{Exercise}\label{ex:ultrapower}
For any point $x\in \spc{X}$, consider the constant sequence $x_n=x$
and set $\iota(x)=\lim_{n\to\omega}x_n\in\spc{X}^\omega$.

\begin{subthm}{ex:ultrapower:a}
Show that $\iota\:\spc{X}\to\spc{X}^\omega$ is distance-preserving embedding. (So we can and will consider $\spc{X}$ as a subset of $\spc{X}^\omega$.)
\end{subthm}

\begin{subthm}{ex:ultrapower:compact} 
Show that $\iota$ is onto if and only if $\spc{X}$ is compact.
\end{subthm}

\begin{subthm}{ex:ultrapower:proper} 
Show that if $\spc{X}$ is proper, then $\iota(\spc{X})$ forms a metric component of $\spc{X}^\omega$; that is, a subset of $\spc{X}^\omega$ that lies at a finite distance from a given point.
\end{subthm}

\end{thm}

Note that \ref{SHORT.ex:ultrapower:compact} implies that the inclusion $\spc{X}\hookrightarrow\spc{X}^\omega$ is not onto if the space $\spc{X}$ is not compact.
However, the spaces $\spc{X}$ and $\spc{X}^\omega$ might be isometric; here is an example:

\begin{thm}{Exercise}\label{ex:isom-ultrapower}
Let $\spc{X}$ be an infinite countable set with discrete metric;
that is $\dist{x}{y}{\spc{X}}=1$ if $x\ne y$.
Show that 

\begin{subthm}{ex:isom-ultrapower:no}
$\spc{X}^\omega$ is not isometric to $\spc{X}$.
\end{subthm}

\begin{subthm}{ex:isom-ultrapower:yes}
$\spc{X}^\omega$ is  isometric to $(\spc{X}^\omega)^\omega$.
\end{subthm}

\end{thm}

\begin{thm}{Exercise}\label{ex:ultrapower(ultrapower)}
Given a nonprincipal ultrafilter $\omega$, construct an ultrafilter $\omega_1$ such that 
\[\spc{X}^{\omega_1}\iso(\spc{X}^\omega)^\omega\]
for any metric space~$\spc{X}$.

\end{thm}


\begin{thm}{Observation}\label{obs:ultrapower-is-geodesic}
Let $\spc{X}$ be a complete metric space. 
Then $\spc{X}^\omega$ is geodesic space if and only if $\spc{X}$ is a length space.
\end{thm}

\parit{Proof.}
The if part follows from \ref{obs:ultralimit-is-geodesic}; it remains to prove the only-if part

Assume $\spc{X}^\omega$ is geodesic space.
Then any pair of points $x,y\in \spc{X}$ has a midpoint $z_\omega\in\spc{X}^\omega$.
Fix a sequence of points $z_n\in  \spc{X}$ such that $z_n\to z_\omega$ as $n\to \omega$.

Note that 
$\dist{x}{z_n}{\spc{X}}\to \tfrac12\cdot \dist{x}{y}{\spc{X}}$
and 
$\dist{y}{z_n}{\spc{X}}\to \tfrac12\cdot \dist{x}{y}{\spc{X}}$
as 
$n\to\omega$.
In particular, for any $\eps>0$, the point $z_n$ is an $\eps$-midpoint of $x$ and $y$ for $\omega$-almost all $n$.
It remains to apply \ref{lem:mid>geod}.
\qeds

\begin{thm}{Exercise}\label{ex:two-geodesics-in-ultrapower}
Assume $\spc{X}$ is a complete length space 
and $p,q\in\spc{X}$ cannot be joined by a geodesic in $\spc{X}$.  
Show that there are at least continuum of distinct geodesics between $p$ and $q$ 
in the ultrapower $\spc{X}^\omega$.
\end{thm}

\begin{thm}{Exercise}\label{ex:notproper-limit}
Construct a proper metric space $\spc{X}$ such that $\spc{X}^\omega$ is not proper;
that is, there is a point $p\in \spc{X}^\omega$ and $R<\infty$ such that the closed ball $\cBall[p,R]_{\spc{X}^\omega}$ is not compact.
\end{thm}

\section{Tangent and asymptotic spaces}
\label{sec:tan+asymptotic}

Choose a space $\spc{X}$ and a sequence $\lambda_n$ of positive numbers.
Consider the sequence of \index{rescaled space}\emph{rescalings} $\spc{X}_n=\lambda_n\cdot\spc{X}=(\spc{X},\lambda_n\cdot\dist{*}{*}{\spc{X}})$.

Choose a point $p\in \spc{X}$ and denote by $p_n$ the corresponding point in $\spc{X}_n$.
Consider the $\omega$-limit $\spc{X}_\omega$ of $\spc{X}_n$ (one may denote it by $\lambda_\omega\cdot \spc{X}$);
set $p_\omega$ to be the $\omega$-limit of $p_n$.

If $\lambda_n\to \infty$ as $n\to\omega$, then the metric component of $p_\omega$ in $\spc{X}_\omega$ is called \index{$\lambda_\omega$-tangent space}\emph{$\lambda_\omega$-tangent space} at $p$ and denoted by $\T_p^{\lambda_\omega}\spc{X}$ (or $\T_p^{\omega}\spc{X}$ if $\lambda_n=n$).\label{page:ultratangent space}

If $\lambda_n\to 0$ as $n\to\omega$, then the metric component of $p_\omega$ is called \index{$\lambda_\omega$-asymptotic space}\emph{$\lambda_\omega$-asymptotic space}%
\footnote{Often it is called an {}\emph{asymptotic cone} despite that it is not a cone in general; this name is used since in good cases it has a cone structure.} and denoted by $\Asym\spc{X}$ or $\Asym^{\lambda_\omega}\spc{X}$.
Note that the space $\Asym\spc{X}$ and its point $p_\omega$ do not depend on the choice of $p\in \spc{X}$.

The following exercise states that the constructions above depend on the sequence $\lambda_n$ and a nonprincipal ultrafilter $\omega$.

\begin{thm}{Exercise}\label{ex:ultraT}
Construct a metric space $\spc{X}$ with a point $p$ such that the tangent space
$\T_p^{\lambda_\omega}\spc{X}$ (or its asymptotic cone $\Asym^{\lambda_\omega}\spc{X}$) depends on the sequence $\lambda_n$ and/or ultrafilter~$\omega$.
\end{thm}

For nice spaces, different choices of the sequence of coefficients and ultrafilter may give the same space; 
some examples are given in the following exercise.

\begin{thm}{Exercise}\label{ex:Asym(Lob)}
Let $\spc{T}=\Asym\spc{L}$, where $\spc{L}$ is the Lobachevsky plane, or Lobachevsky space, or 3-regular%
\footnote{that is, the degree of any vertex is 3.}
metric tree with unit edge length (choose your favorite space from these three).

\begin{subthm}{ex:Asym(Lob):metric-tree}
Show that $\spc{T}$ is a complete metric tree.
\end{subthm}

\begin{subthm}{ex:Asym(Lob):homogeneous}
Show that $\spc{T}$ is one-point-homogeneous; that is, given two points $s,t\in \spc{T}$ there is an isometry of $\spc{T}$ that maps $s$ to $t$.
\end{subthm}

\begin{subthm}{ex:Asym(Lob):continuum}
Show that $\spc{T}$ has \index{degree}\emph{continuum degree} at any point;
that is, for any point $t\in \spc{T}$ the set of connected components of the complement $\spc{T}\setminus\{t\}$ has cardinality continuum.
\end{subthm}

\end{thm}


\begin{thm}{Exercise}\label{ex:T(Sx[0,1]/Sx0)}
Consider the quotient space $\spc{X}=\mathbb{S}^1\times[0,1]/\mathbb{S}^1\times\{0\}$;
that is, 
$\dist{(u_1,t_1)}{(u_2,t_2)}{\spc{X}}
=
\min\{\,\dist{(u_1,t_1)}{(u_2,t_2)}{\mathbb{S}^1\times[0,1]},t_1+t_2\,\}$.
Describe the ultratangent space $\T_o^{\omega}\spc{X}$, where $o\in\spc{X}$ is the point that corresponds to $\mathbb{S}^1\times\{0\}$.
\end{thm}


\section{Remarks}

A nonprincipal ultrafilter $\omega$ is called 
\emph{selective}\index{ultrafilter!selective ultrafilter}\index{selective ultrafilter} if for any partition of $\NN$ into sets $\{C_\alpha\}_{\alpha\in\IndexSet}$ such that $\omega(C_\alpha)\z=0$ for each $\alpha$, 
there is a set $S\subset \NN$ such that $\omega(S)=1$ and $S\cap C_\alpha$ is a one-point set for each $\alpha\in\IndexSet$.

The existence of a selective ultrafilter follows from the continuum hypothesis \cite{rudin}.

If needed, we may assume that the chosen ultrafilter $\omega$ is selective.
In this case \textit{the subsequence $(x_n)_{n\in S}$ in \ref{prop:ultra/partial} can be chosen so that $\omega(S)=1$}.


\appendix
\chapter{Semisolutions}
\refstepcounter{chapter}
\setcounter{eqtn}{0}

\parbf{\ref{ex:quad-inq}.}
Add four triangle inequalities (\ref{metric:triangle}).

\parbf{\ref{ex:normal}.}
Consider the function 
\[f(x)=\frac{\distfun_Ax}{\distfun_Ax+\distfun_Bx},\]
where $\distfun_Ax\z\df\inf_{a\in A}\dist{a}{x}{}$.
Show that $f$ is continuous and satisfies the needed property.

\parbf{\ref{ex:tietze}.}
Use \ref{ex:normal} to construct an approximation of the needed function and pass to a limit or find a proof of the \index{Tietze extension theorem}\emph{Tietze extension theorem}.

\parbf{\ref{ex:pseudo-infty-metric}};\ref{SHORT.ex:pseudo-infty-metric:pseudo}.
Note that if $\mu(A)=\mu(B)=0$, then $|A-B|=0$.
Therefore, \ref{metric=0} does not hold for bounded closed subsets.
It is straightforward to check the remaining conditions in~\ref{def:metric} hold true.

\parit{\ref{SHORT.ex:pseudo-infty-metric:infty}.}
Note that the distance from the empty set to the whole plane is infinite; so the value $|A-B|$ might be infinite.
It is straightforward to check the remaining conditions in~\ref{def:metric}.

\parit{Remark.}
Metrics of the form $\dist{A}{B}{}=\mu(A\bigtriangleup B)$ are very special.
In particular, they satisfy the so-called \index{hypermetric inequality}\emph{hypermetric inequalities}; that is, for any sequence of sets $A_1,\dots, A_n$ and any sequence of integers $b_1,\dots,b_n$ such that $\sum_ib_i=1$ we have
\[\sum_{i,j}b_i\cdot b_j\cdot \dist{A_i}{A_j}{}\le 0.\]
Note that for $n=3$ and $b_1=b_2=-b_3=1$ we get the usual triangle inequality.
For more on the subject, see \cite{deza-laurent}.

\parbf{\ref{ex:gluing}.}
Choose $\delta>0$ and an increasing linear bijection $\ell\:[a,b]\to [c,d]$.
Show that $\ell$ has arbitrarily close increasing piecewise-linear bijection $s\:[a,b]\to [c,d]$ such that derivative at any point is either $<\delta$ or $>\tfrac1\delta$.

Start with the identity map $[0,1]\to [0,1]$;
iterate the above construction for smaller and smaller $\delta$ and pass to the limit.
This way we obtain an increasing  bijection $x\leftrightarrow x'$ from $[0,1]$ to itself
such that for any $\eps>0$ there is a partition $0=t_0<t_1<\dots <t_{2\cdot n}=1$ of $[0,1]$ with 
\begin{align*}
\eps&>|t_0-t_1|+|t_1'-t_2'|+|t_2-t_3|+\dots
\\
&\dots+|t_{2\cdot n-2}-t_{2\cdot n-1}|+|t_{2\cdot n-1}'-t_{2\cdot n}'|.
\end{align*}
Make a conclusion.

\parbf{\ref{ex:almost-min}.}
Assume the statement is wrong. 
Then for any point $x\in \spc{X}$, there is a point $x'\in \spc{X}$ such that 
\begin{align*}
\dist{x}{x'}{}&<\rho(x)
\intertext{and}
\rho(x')&\le\frac{\rho(x)}{1+\eps}.
\end{align*}
Consider a sequence $x_1,x_2,\dots\in \spc{X}$ such that $x_{n+1}\z=x_n'$.
Show that this is a Cauchy sequence.
Since $\spc{X}$ is complete, $x_n$ converges;
denote its limit by $x_\infty$.
Since $\rho$ is a continuous function we get
\begin{align*}
\rho(x_\infty)&=\lim_{n\to\infty}\rho(x_n)=0.
\end{align*}

The latter contradicts that $\rho>0$.

\parbf{\ref{ex:complete-completion}.}
Let $\bar {\spc{X}}$ be the completion of $\spc{X}$.
By the definition, for any $y\in \bar {\spc{X}}$ there is a Cauchy sequence $x_n$ in  $\spc{X}$ that converges to $y$.

Choose a Cauchy sequence $y_m$ in $\bar {\spc{X}}$.
From above, we can choose points $x_{n,m}\in \spc{X}$ such that $x_{n,m}\to y_m$ for any $m$.
Choose $z_m=x_{n_m,m}$ such that $|y_m-z_m|<\tfrac1m$.
Observe that $z_m$ is Cauchy.
Therefore, its limit $z_\infty$ lie in $\bar{\spc{X}}$.
Finally, show that $x_m\to z_\infty$.

\parbf{\ref{ex:compact-net}.}
A compact $\eps$-net $N$ in $\spc{K}$ contains a finite $\eps$ net $F$.
Show and use that $F$ is a $2\cdot\eps$-net of $\spc{K}$.

\parbf{\ref{ex:non-contracting-map}.}
Given a pair of points $x_0,y_0\in \spc{K}$, 
consider two sequences $x_0,x_1,\dots$ and $y_0,y_1,\dots$
such that $x_{n+1}=f(x_n)$ and $y_{n+1}\z=f(y_n)$ for each $n$.

Since $\spc{K}$ is compact, 
we can choose an increasing sequence of integers $n_k$
such that both sequences $(x_{n_i})_{i=1}^\infty$ and $(y_{n_i})_{i=1}^\infty$
converge.
In particular, both are Cauchy;
that is,
\[
|x_{n_i}-x_{n_j}|_{\spc{K}}\to 0 
\quad\text{and}\quad
|y_{n_i}-y_{n_j}|_{\spc{K}}\to 0
\]
as $\min\{i,j\}\to\infty$.

Since $f$ is distance-noncontracting, 
\[
|x_0-x_{|n_i-n_j|}|
\le 
|x_{n_i}-x_{n_j}|
\]
for any $i$ and $j$.
Therefore, there is a sequence $m_i\to\infty$ such that
\[
x_{m_i}\to x_0\quad\text{and}\quad y_{m_i}\to y_0
\leqno({*})\]
as $i\to\infty$.

Since $f$ is distance-noncontracting, the sequence $\ell_n=|x_n-y_n|_{\spc{K}}$ is nondecreasing.
By $({*})$,  $\ell_{m_i}\to\ell_0$ as $m_i\to\infty$.
It follows that 
\[\ell_0=\ell_1=\dots\]
In particular, 
\[|x_0-y_0|_{\spc{K}}=\ell_0=\ell_1=|f(x_0)-f(y_0)|_{\spc{K}}\]
for any pair of points $(x_0,y_0)$ in $\spc{K}$.
That is, the map $f$ is distance-preserving; hence $f$ is injective.
From $({*})$, we also get that $f(\spc{K})$ is everywhere dense.
Since $\spc{K}$ is compact $f\:\spc{K}\to \spc{K}$ is surjective --- hence the result.

\parit{Remarks.}
This is a basic lemma in the introduction to Gromov--Hausdorff distance \cite[see 7.3.30 in][]{burago-burago-ivanov}.
The presented proof is not quite standard,
I learned it from Travis Morrison, 
a student in my MASS class at Penn State, Fall 2011.

Note that this exercise implies that \textit{any surjective non-expanding map from a compact metric space to itself is an isometry}. 

\parbf{\ref{ex:loc-compact-not-proper}.}
Check an infinite set with a discrete metric.

\parbf{\ref{ex:pogorelov}.}
Set $B_p=B(x,\tfrac \pi2)_{\mathbb{S}^2}$.
The triangle inequality follows since
\[
(B_x\setminus B_y)
\cup 
(B_y\setminus B_z)
\supseteq
B_x\setminus B_z.
\leqno(*)\]
The remaining conditions in Definition \ref{def:metric} are evident.

Observe that
$B_x\setminus B_y$
does not overlap with
$B_y\z\setminus B_z$ and  we get equality in $(*)$ if and only if $y$ lies on the great circle arc from $x$ to $z$.
Therefore, the second statement follows.


\begin{wrapfigure}{r}{24 mm}
\vskip-0mm
\centering
\includegraphics{mppics/pic-29}
\end{wrapfigure}

\parit{Remarks.}
This construction is due to 
Aleksei Pogorelov \cite{pogorelov}.
It is closely related to the construction given 
by David Hilbert \cite{hilbert}
which was the motivating example for his fourth problem. 
See also the remark after the solution of~\ref{ex:pseudo-infty-metric}.

\parbf{\ref{ex:4-point-trees}.}
We may assume that none of the points $p,x,y,z$ lies on a geodesic between the other two.

Let $K$ be the set in the tree covered by all six geodesics with the endpoints $p,x,y,z$.
Observe that $K$ looks like an H or like an X; make a conclusion.

\parit{Remarks.}The value $\tfrac12\cdot(|p-x|+|p-y|-|x-y|)$ is called \index{Gromov's product}\emph{Gromov's product} of $x$ and $y$ with the origin at $p$;
usually it is denoted by $(x|y)_p$.

Note that a four-point metric space admits an isometric embedding into a metric tree if and only if one of these two equivalent conditions holds.
Moreover, a metric space admits an isometric embedding into a metric tree if every its four-point subspace admits such embedding.


\parbf{\ref{ex:spheres-in-trees}.}
Apply \ref{ex:4-point-trees}.

\parbf{\ref{ex:1-Lip-G-delta}.}
Note that $\spc{P}$ is complete.
Choose $\eps>0$.
Use \ref{thm:length-semicont} to show that the set of paths of length $>1-\eps$ is open in~$\spc{P}$;
show that this set is also dense in~$\spc{P}$.
Apply Baire's theorem (\ref{thm:baire}).

\parit{Remark.}
You might find it surprising that \textit{most of the short maps from the sphere to the plane are \index{length-preserving map}\emph{length-preserving}}; that is, they preserve lengths of all curves.
The latter follows from the result of 
Bernd Kirchheim, 
Emanuele Spadaro,
and 
L{\'a}szl{\'o} Sz{\'e}kelyhidi \cite{KSS}.
(While most of the maps have this property, it is not at all easy to construct a single such example.) 


\parbf{\ref{ex:no-geod}.}
\textit{Formally speaking, a one-point space is a solution,
but we will construct a nontrivial example.}

Recall that $c_0\subset\ell^\infty$ denotes the space of all real sequences converging to zero.
Consider the unit ball $B$ in $c_0$;
denote by $\rho_0$ the metric on $B$.

Let \[\phi(\bm{x})=2+\tfrac{1}2\cdot x_1+\tfrac{1}4\cdot x_2+\tfrac{1}8\cdot x_3+\dots,\]
where $\bm{x}=(x_1,x_2\,\dots)\in B$.
Consider another length metric $\rho_1$ on $B$ that is different from $\rho_0$ by the conformal factor $\phi$;
that is, if $t\mapsto\bm{x}(t)$ for $t\in[0,\ell]$ is a curve parametrized by $\rho_0$-length,
then its $\rho_1$-length is defined by
\[\length_{\rho_1}\bm{x}\df\int\limits_0^\ell\phi\circ\bm{x}(t)\cdot dt.\]
Note that the metric $\rho_1$ is bilipschitz to~$\rho_0$.

Assume $t\mapsto \bm{x}(t)$ and $t\mapsto \bm{x}'(t)$ are two curves parametrized by $\rho_0$-length that differ only in the $m$-th coordinate; denote them by $x_m(t)$ and $x_m'(t)$ respectively.
Show that if $x'_m(t)\le x_m(t)$ for any $t$ and 
the function $x'_m(t)$ is locally $1$-Lipschitz at all $t$ such that $x'_m(t)< x_m(t)$, then 
\[\length_{\rho_1}\bm{x}'\le \length_{\rho_1}\bm{x}.\]
Moreover, this inequality is strict if $x'_m(t)\z< x_m(t)$ for some~$t$.

Fix a curve $\bm{x}(t)$, $t\in[0,\ell]$, parametrized by  $\rho_0$-length.
We can choose large $m$ so that $x_m(t)$ is sufficiently close to $0$ for any~$t$.
In this case, it is easy to construct a function $t\mapsto x'_m$ that meets the above conditions.
It follows that for any curve $\bm{x}(t)$ in $(B,\rho_1)$, we can find a shorter curve $\bm{x}'(t)$ with the same endpoints.
In particular, $(B,\rho_1)$ has no geodesics.

\parit{Remark.}
This solution was suggested by Fedor Nazarov~\cite{nazarov}.

\parbf{\ref{ex:compact+connceted}.}
Choose a sequence of positive numbers $\varepsilon_n\to 0$ and a finite $\varepsilon_n$-net $N_n$ of $K$ for each $n$.
We can assume that $\eps_0>\diam K$, and $N_0$ is a one-point set.
If $\dist{x}{y}{}<\eps_k$ for some $x\in N_{k+1}$ and $y\in N_{k}$, then connect them by a curve of length at most $\eps_k$.

Let $K'$ be the union of all these curves and $K$.
Show that $K'$ is compact and path-connected.

\parit{Source:} This problem is due to Eugene Bilokopytov \cite{bilokopytov}.

\parbf{\ref{ex:compact=>complete}.}
Choose a Cauchy sequence $x_n$ in $(\spc{X},\|*\z-*\|)$; it is sufficient to show that a subsequence of $x_n$ converges.

Observe that the sequence $x_n$ is Cauchy in $(\spc{X},|*-*|)$;
denote its limit by $x_\infty$.

Passing to a subsequence, we can assume that $\|x_n-x_{n+1}\|\z<\tfrac1{2^n}$.
It follows that there is a 1-Lipschitz path $\gamma$ in $(\spc{X},\|*-*\|)$ such that $x_n=\gamma(\tfrac1{2^n})$ for each $n$ and $x_\infty=\gamma(0)$.
Therefore,
\begin{align*}
\|x_\infty-x_n\|&\le \length\gamma|_{[0,\frac1{2^n}]}\le \tfrac1{2^n}.
\end{align*}
In particular, $x_n$ converges to $x_\infty$ in $(\spc{X},\|*\z-*\|)$.

\parit{Source:} \cite[Corollary]{hu-kirk}; see also \cite[Lemma 2.3]{petrunin-stadler}.

\parbf{\ref{ex:menger}.} Choose two points $x,y\in \spc{X}$;
let $\ell\z=\dist{x}{y}{}$.
Suppose $f\:E\to \spc{X}$ is a distance-preserving map such that $0,\ell\in E\subset [0,\ell]$,
$f(0)=x$, and  $f(\ell)=y$.

Show that we can choose $f$ so that $E$ is maximal;
that is, $f$ cannot be extended to a distance-preserving map on a larger subset of $[0,\ell]$.

Show that there is no open interval $(a,b)$ in the complement of $E$ such that $a,b\in E$.

Apply the completeness of $\spc{X}$ to show that $E$ is closed.
Conclude that $E=[0,\ell]$.

\parbf{\ref{ex:eps-nbhd(ball)}.}
Let $U$ be the $\eps$-neighborhood of $\oBall(x,R)_{\spc{X}}$.
By the triangle inequality, $U\z\subset \oBall(x,R+\eps)_{\spc{X}}$;
this inclusion holds in any metric space.

Choose $y\in \oBall(x,R+\eps)_{\spc{X}}$, so $\dist{x}{y}{{\spc{X}}}\z<R+\eps$.
Since ${\spc{X}}$ is a length space, there is a curve $\gamma$ from $x$ to $y$ with length less than $R+\eps$.
Show and use that $\gamma$ contains a point $m$ such that $\dist{x}{m}{{\spc{X}}}<R$ and $\dist{y}{m}{{\spc{X}}}<\eps$.

\parbf{\ref{exercise from BH}.}
Consider the following subset of $\RR^2$ equipped with the induced length metric
\[
\spc{X}
=
\bigl((0,1]\times\{0,1\}\bigr)
\cup
\bigl(\{1,\tfrac12,\tfrac13,\dots\}\times[0,1]\bigr)
\]
Note that $\spc{X}$ is locally compact and geodesic.

Its completion $\bar{\spc{X}}$ is isometric to the closure of $\spc{X}$ equipped with the induced length metric.
Note that $\bar{\spc{X}}$ is obtained from $\spc{X}$ by adding two points $p=(0,0)$ and $q\z=(0,1)$.

{

\begin{wrapfigure}{r}{20 mm}
\vskip-4mm
\centering
\includegraphics{mppics/pic-1}
\end{wrapfigure}

Observe that $p$ admits no compact neighborhood in $\bar{\spc{X}}$ and there is no geodesic connecting $p$ to $q$ in~$\bar{\spc{X}}$. 


\parit{Source:} \cite[I.3.6(4)]{bridson-haefliger}.

}

\parbf{\ref{ex:gross}.}
Suppose this number does not exist.
Show that there are two point-arrays $(x_1,\z\dots,x_n)$ and $(y_1,\dots,y_m)$
such that
\[
\min_{z\in \spc{X}}\{\,f(z)\,\}>\max_{z\in \spc{X}}\{\,h(z)\,\},
\leqno({*})
\]
where
\begin{align*}
f(z)&=\tfrac1n\cdot\sum_i|x_i-z|_{\spc{X}}
\intertext{and}
h(z)&=\tfrac1m\cdot\sum_j|y_j-z|_{\spc{X}}.
\end{align*}


Note that
\begin{align*}\tfrac1m\cdot\sum_j f(y_j)&=\tfrac1{m\cdot n}\cdot\sum_{i,j}|x_i-y_j|_{\spc{X}}=
\\
&=\tfrac1n\cdot\sum_i h(x_i);
\end{align*}
that is, the average value of $f(y_j)$ coincides with the average value of $h(x_i)$.
The latter contradicts~$({*})$.

\parit{Remark.}
The value $\ell$ is uniquely defined;
it is called the \index{rendezvous value}\emph{rendezvous value} of ${\spc{X}}$.
This is a result of Oliver Gross \cite{gross}.

%%%%%%%%%%%%%%%%%%%%%%%%%%%%%%

%%%%%%%%%%%%%%%%%%%%%%%%%
\refstepcounter{chapter}
\setcounter{eqtn}{0}

\parbf{\ref{ex:compact-length}.}
By the Fréchet lemma (\ref{lem:frechet}) we can identify $\spc{K}$ with a compact subset in $\ell^\infty$.

Denote by $\spc{L}$ the \index{closed convex hull}\emph{closed convex hull} of $\spc{K}$;
that is, $\spc{L}$ is the minimal convex closed set in $\ell^\infty$ that contains $\spc{K}$.
(In other words, $\spc{L}$ is the minimal closed set containing $\spc{K}$ such that if $x,y\in \spc{L}$, then 
$t\cdot x+(1-t)\cdot y\in \spc{L}$ for any $t\in[0,1]$.)

Observe that $\spc{L}$ is a length space.
It remains to show that $\spc{L}$ is compact.

By construction, $\spc{L}$ is a closed subset of $\ell^\infty$; in particular, it is complete.
By \ref{totally-bounded}, it remains to show that $\spc{L}$ is totally bounded.

Recall that Minkowski sum $A + B$ of two sets $A$ and $B$ in a vector space is defined by
\[A + B 
\df
\set{a+b}{a\in A,\ b\in B}.\]
Observe that the Minkowski sum of two convex sets is convex.

Denote by $\bar B_\eps$ the closed $\eps$-ball in $\ell^\infty$ centered at the origin.
Choose a finite $\eps$-net $N$ in $\spc{K}$ for some $\eps>0$.
Note that $P=\Conv N$ is a convex polyhedron; in particular, $\Conv N$ is compact.

Observe that $N+\bar B_\eps$ is a closed $\eps$-neighborhood of $N$.
It follows that $N+\bar B_\eps\supset K$ and therefore $P+\bar B_\eps\supset \spc{L}$.
In particular, $P$ is a $2\cdot\eps$-net in $\spc{L}$;
since $P$ is compact and $\eps>0$ is arbitrary, $\spc{L}$ is totally bounded (see \ref{ex:compact-net}).

\parit{Remark.}
Alternatively, one may use that \textit{the injective envelope of a compact space is compact}; see \ref{ex:inj=complete-geodesic-contractible:geodesic}, \ref{ex:Inj(compact)}, and \ref{prop:InjX-is-injective}.

\parbf{\ref{ex:frechet}.}
Modify the proof of \ref{lem:frechet}.

\begin{wrapfigure}{r}{23mm}
\vskip-6mm
\centering
\includegraphics{mppics/pic-200}
\end{wrapfigure}

\parbf{\ref{ex:inf-extension}.}
Consider the metric tree $\spc{T}$ shown on the diagram;
it is a half-line $[0,\infty)$ with attached an interval of length $n+1$ to each integer~$n\ge 0$.
Denote by $o$ the origin of the half-line
and by $x_n$ the endpoint of $n^{\text{th}}$ interval.

Observe that if $m\ne n$, then
\[|x_m-x_n|_{\spc{T}}\ge |o-x_n|_{\spc{T}}+1.\]
Show and use that for any binary sequence $\eps_n$ there is an extension function $f$ such that 
\[f(x_n)=|o-x_n|_{\spc{T}}+\eps_n.\]


\parit{Remark.}
An if-and-only-if condition on $\spc{X}$ that have separable $\spc{X}^\infty$ was found by Julien Melleray \cite[2.8]{melleray}.
A similar condition was used by Herbert Federer to describe metric spaces where Besicovitch covering lemma holds \cite[2.8.9]{federer}.

\parbf{\ref{ex:geodesics-urysohn}.}
Choose a separable space $\spc{X}$ that has an infinite number of geodesics between a pair of points with the given distance between them;
say a square in $\RR^2$ with $\ell^\infty$-metric will do.
Apply to $\spc{X}$ universality of Urysohn space (\ref{prop:sep-in-urys}).

\parbf{\ref{ex:compact-extension}.} 
First let us prove the following claim:

\begin{itemize}
\item 
Suppose $f\: K\to\RR$ is an extension function defined on a compact subset $K$ of the Urysohn space $\spc{U}$.
Then there is a point $p\in \spc{U}$ such that 
$\dist{p}{x}{}=f(x)$ for any $x\in K$.
\end{itemize}

Without loss of generality, we may assume that $f>0$.
Since $K$ is compact, we may fix $\eps>0$ such that $f(x)>\eps$ for any $x\in K$.

Consider the sequence $\eps_n=\tfrac\eps{100\cdot 2^n}$.
Choose a sequence of $\eps_n$-nets $N_n\subset K$.
Applying the universality of $\spc{U}$ recursively, we may choose a point $p_n$ such that $\dist{p_n}{x}{}=f(x)$ for any $x\in N_n$ and $\dist{p_n}{p_{n-1}}{}\z=10\cdot\eps_{n-1}$.
Observe that the sequence $p_n$ is Cauchy and its limit $p$ meets 
$\dist{p}{x}{}=f(x)$ for any $x\in K$.

Now, choose a sequence $x_n$ of points that is dense in $\spc{S}$.
Applying the claim, we may extend the map from $K$ to $K\cup\{x_1\}$, further to $K\cup\{x_1,x_2\}$, and so on.
As a result, we extend the distance-preserving map $f$ to the whole sequence $x_n$.
It remains to extend it continuously to the whole space~$\spc{S}$.

\parbf{\ref{ex:sc-urysohn}.}
It is sufficient to show that any compact subspace $\spc{K}$ of the Urysohn space $\spc{U}$ can be contracted to a point.

Note that any compact space $\spc{K}$ can be extended to a contractible compact space $\spc{K}'$; for example, we may embed $\spc{K}$ into $\ell^\infty$ and pass to its convex hull, as it was done in \ref{ex:compact-length}.

By \ref{thm:compact-homogeneous}, there is an isometric embedding of $\spc{K}'$ that agrees with the inclusion $\spc{K}\hookrightarrow\spc{U}$.
Since $\spc{K}$ is contractible in $\spc{K}'$, it is contractible in $\spc{U}$.

\parit{A better way.}
One can contract the whole Urysohn space using the following construction.

Note that points in $\spc{X}_\infty$ constructed in the proof of \ref{prop:univeral-separable} can be multiplied by $t\in [0,1]$ --- simply multiply each function by $t$.
That defines a map 
\[\lambda_t\:\spc{X}_\infty\to \spc{X}_\infty\]
that rescales all distances by factor $t$.
The map $\lambda_t$ can be extended to the completion of $\spc{X}_\infty$, which is isometric to $\spc{U}_d$ (or $\spc{U}$).

Observe that 
the map $\lambda_1$ is the identity  and $\lambda_0$ maps the whole space to a single point, say $x_0$ --- this is the only point of $\spc{X}_0$.
Further, note that $(t,p)\mapsto \lambda_t(p)$ is a continuous map; in particular, $\spc{U}_d$ and $\spc{U}$ are contractible.

As a bonus, observe that for any point $p\in \spc{U}_d$ the curve $t\mapsto \lambda_t(p)$ is a geodesic path from $p$ to $x_0$.

\parit{Source:} \cite[$\text{(d)}$ on page 82]{gromov-2007}.

\parbf{\ref{ex:no-isom}.}
Consider two infinite metric trees as on the diagram. 

\begin{Figure}
\vskip-0mm
\centering
\includegraphics{mppics/pic-205}
\end{Figure}

\parit{Remark.}
A more sophisticated example: $\spc{X}\z=\ell^\infty$ and $\spc{Y}=L^\infty([0,1])$.
Try to prove that it qualifies; see also \cite{buehler}.

%Given a bounded sequence $\bm{a}=(a_1,a_2,\dots)$, consider the function $f$ such that $f(0)=0$ and $f(x)=a_n$ if $\tfrac1{n+1}<x\le \tfrac1n$.
%Note that $\bm{a}\mapsto f$ is a distance-preserving map $\ell^\infty\to L^\infty([0,1])$.

%Further, enumerate all subintervals of $[0,1]$ with rational ends, $I_1,I_2,\dots$
%Given a function $f\in L^\infty([0,1])$ consider sequence $\bm{a}\z=(a_1,a_2,\dots)$ such that $a_n$ is the mean value of $f$ on $I_n$.
%Observe that $f\mapsto \bm{a}$ is a distance-preserving map $L^\infty([0,1])\to \ell^\infty$.

%It remains to show that $\spc{X}=\ell^\infty$ and $\spc{Y}=L^\infty([0,1])$ are not isometric???



\parbf{\ref{ex:sphere-in-urysohn}}; \ref{SHORT.ex:sphere-in-urysohn:sphere} and \ref{SHORT.ex:sphere-in-urysohn:midpoint}.
Observe that $L$ and $M$ satisfy the definition of $d$-Urysohn space and apply the uniqueness (\ref{thm:urysohn-unique}).
Note that
\[\ell=\diam L=\min\{2\cdot r, d\}.\]

\parit{\ref{SHORT.ex:sphere-in-urysohn:homogeneous}.} 
Use \ref{SHORT.ex:sphere-in-urysohn:sphere}, maybe twice.

\parbf{\ref{ex:shere}.}
Let $p$ be the center of the sphere;
without loss of generality, we can assume that $\dist{p}{x}{}\le \dist{p}{y}{}$.

Consider function $f\:\{p,x,y\}\to\RR$ defined by $f(p)=1$, $f(x)=1+\dist{p}{x}{}$, and $f(y)=1+\dist{p}{y}{}-\eps$.
Suppose $\eps>0$ is sufficiently small;
show that $f$ is an extension function on $\{p,x,y\}$.

By the extension property, there is a point $z\in \spc{U}$ such that $\dist{p}{z}{}=f(p)$, $\dist{x}{z}{}=f(x)$, and $\dist{y}{z}{}=f(y)$.
Whence the statement follows.

\parit{Source:} This problem is taken from a survey of Julien Melleray
 \cite[Prop. 4.3]{melleray}, where it was attributed to Matatyahu Rubin.


\parbf{\ref{ex:ext(shere)}.} 
Observe that the complement $\spc{V}=\spc{U}\setminus B$ is complete.
Show that it $\spc{V}$ satisfies the extension property.
Conclude that $\spc{V}$ is an Urysohn space and apply \ref{thm:urysohn-unique}.

For the second part, observe that there is an isometry $\iota\:\spc{U}\to \spc{V}$.
Moreover, if $p$ is the center of $B$, then we can assume that $\iota$ has a fixed point $x$ such that $\dist{p}{x}{}>2$.

Consider the unit sphere $S$ centered at $x$.
The restriction of $\iota$ to $S$ is an isometry of $S$.
Use \ref{ex:shere} to show that it cannot be extended to an isometry of $\spc{U}$.

\parit{Source:} \cite[Sec. 4.4]{melleray}.

\parbf{\ref{ex:katetov}.}
Apply \ref{thm:urysohn-unique} and the construction in \ref{thm:urysohn-exists+}.

\parbf{\ref{ex:homogeneous}}; \ref{SHORT.ex:homogeneous:euclidean}.
The euclidean plane is homogeneous in every sense.

\parit{\ref{SHORT.ex:homogeneous:hilbert}.}
The Hilbert space $\ell^2$ is finite-set-homogeneous, but not compact-set-homogeneous, nor countable-set-homogeneous.

\parit{\ref{SHORT.ex:homogeneous:ell-infty}.}
$\ell^\infty$ is one-point-homogeneous, but not two-point-homogeneous.
Try to show that there is no isometry of $\ell^\infty$ such that
\begin{align*}
(0,0,0,\dots)&\mapsto (0,0,0,\dots),
\\
(1,1,1,\dots)&\mapsto (1,0,0,\dots).
\end{align*}

\parit{\ref{SHORT.ex:homogeneous:ell-1}.}
$\ell^1$ is one-point-homogeneous, but not two-point-homogeneous.
Try to show that there is no isometry of $\ell^\infty$ such that
\begin{align*}
(0,0,0,\dots)&\mapsto (0,0,0,\dots),
\\
(2,0,0\dots)&\mapsto (1,1,0,\dots).
\end{align*}

\parbf{\ref{ex:homogeneous-tree}.}
Let $\spc{T}$ be a one-point-homogeneous metric tree.
Note that all points in $\spc{T}$ have the same degree $d$;
that is, for any point $t\in \spc{T}$ the set of connected components of the complement $\spc{T}\setminus\{t\}$ has the same cardinality $d$.

Show that if $d=0$, then $\spc{T}$ is a one-point space;
there is no tree with $d=1$,
and if $d=2$, then $\spc{T}\iso\RR$.

Suppose $d\ge 3$.
Choose a geodesic $\gamma$ in $\spc{T}$.
Show that number of connected components of $\spc{T}\setminus\gamma$ has cardinality continuum.
Observe and use that one can choose a point $p_\alpha$ in each connected component such that $\dist{p_\alpha}{p_\beta}{\spc{T}}>1$ if $\alpha\ne\beta$.

\parbf{\ref{ex:horobry}.}
Assume $F_{\spc{X}}$ is not an embedding.
That is, there is a sequence of points $x_1,x_2,\dots$ 
and a point $x_\infty$ such that $f_{x_n}\to f_{x_\infty}$ in $C(\spc{X},\RR)$
as $n\to \infty$, 
while $|x_n-x_\infty|_{\spc{X}}>\eps$ 
for some fixed $\eps>0$ and all~$n$.

By \ref{prop:length+proper=>geodesic}, any pair of points $x,y\in \spc{X}$ can be connected by a minimizing geodesic $[xy]$.
Choose $\bar x_n$ on a geodesic $[x_\infty x_n]$ such that $|x_\infty-\bar x_n|=\eps$.
Note that 
\begin{align*}
f_{x_n}(x_\infty)-f_{x_n}(\bar x_n)&=\eps,
\\
f_{x_\infty}(x_\infty)-f_{x_n}(\bar x_n)&=-\eps
\end{align*}
for all $n$.

Since $\spc{X}$ is proper, we can pass to a subsequence of $x_n$ so that the sequence  $\bar x_n$ converges;
denote its limit by $\bar x_\infty$.
The above identities imply that
\begin{align*}
f_{x_n}(\bar x_\infty)&\not\to f_{x_\infty}(\bar x_\infty)
\quad
\text{or}
\\
f_{x_n}(x_\infty)&\not\to f_{x_\infty}( x_\infty)
\end{align*}
--- a contradiction.

For the second part, take $\spc{Y}$ to be the set of non-negative integers with the metric $\rho$ defined by
\[\rho(m,n)=m+n\] 
for $m\ne n$.

\medskip

\parit{Source:}
I learned this example from Linus Kramer and Alexander Lytchak;
it was also mentioned in the lectures of Anders Karlsson
and attributed to Uri Bader \cite[2.3]{karlsson}.

\parbf{\ref{ex:cut}.}
Suppose that our metric is $\sum a_S\cdot\delta_S$ with $a_S\ge 0$ for any $S\subset F$.
Enumerate all the subsets $S_1,\dots,S_{2^n}$;
set $S_i=F$ for all $i>2^n$. 
Consider the maps $x\mapsto (a_1,a_2,\dots)$ where $a_i=0$ if $x\in S_i$ and otherwise $a_i=1$.
Observe that it defines a distance-preserving map $F\to \ell^1$. 

The if part is proved.
For the only-if part, check the statement for subsets of the real line, and use it.

\parbf{\ref{ex:K23}.}
Show that for any proper subset $S$ in the vertex set there are three vertices $x,y,z$ such that $\dist{x}{y}{} +\dist{y}{z}{}=\dist{x}{z}{}$ and either 
$x,z\in S$ and $y\notin S$, or $x,z\notin S$ and $y\in S$.
Then apply \ref{ex:cut}.

\parbf{\ref{ex:RP-not}.}
For the first part, show and use that the quotient of $\RP^2$ by the isotropy group of one point is isometric to a line segment.

For the second part, choose three points on a closed geodesic at equal distances from each other.
Show and use that there is an isometric three-point set in $\RP^2$ that does not lie on a closed geodesic.

\parit{Source:} \cite[V \S 2]{busemann-1942}.

\parbf{\ref{ex:hom-cube}.}
Denote by $\dim(x_1,\dots,x_m)$ the dimension of the minimal face of the cube that contains all the points $x_1,\dots,x_m\in Q$.
Show and use that 
\[\dim(x_1,\dots,x_m)=\dim(x_1',\dots,x_m')\]
for any isometry $x\mapsto x'$ of $Q$.

\parit{Source:} \cite[prop. 6 and 7]{berestovskii-nikonorov}.

\parbf{\ref{ex:conv-short};} \textit{only-if part}.
To check convexity, assume that $B$ is a two-point subset.
For closeness, assume that $B$ is a countable set of $A$.

\parit{If part.}
Learn about the Kirszbraun theorem and apply it together with the closest-point projection.

\refstepcounter{chapter}
\setcounter{eqtn}{0}


\parbf{\ref{ex:inj=complete-geodesic-contractible}.}
Choose an injective space $\spc{Y}$.

\textit{\ref{SHORT.ex:inj=complete-geodesic-contractible:complete}.}
Fix a Cauchy sequence $x_n$ in $\spc{Y}$;
we need to show that it has a limit $x_\infty\in \spc{Y}$.
Consider metric on $\spc{X}=\NN\cup\{\infty\}$ defined by 
\begin{align*}
\dist{m}{n}{\spc{X}}&\df\dist{x_m}{x_n}{\spc{Y}},
\\
\dist{m}{\infty}{\spc{X}}&\df\lim_{n\to\infty}\dist{x_m}{x_n}{\spc{Y}}.
\end{align*}
Since the sequence is Cauchy, so is the sequence $\ell_n=\dist{x_m}{x_n}{\spc{Y}}$ for any $m$.
Therefore, the last limit is defined.

By construction, the map $n\mapsto x_n$ is distance-preserving on $\NN\subset \spc{X}$.
Since $\spc{Y}$ is injective, this map can be extended to $\infty$ as a short map; set $\infty\mapsto x_\infty$.
Since $\dist{x_n}{x_\infty}{\spc{Y}}\le \dist{n}{\infty}{\spc{X}}$ 
and $\dist{n}{\infty}{\spc{X}}\to 0$, we get that
$x_n\to x_\infty$ as $n\to\infty$.

\textit{\ref{SHORT.ex:inj=complete-geodesic-contractible:geodesic}.}
Applying the definition of injective space, we get a midpoint for any pair of points in $\spc{Y}$.
By \ref{SHORT.ex:inj=complete-geodesic-contractible:complete},
$\spc{Y}$ is a complete space.
It remains to apply \ref{lem:mid>geod:geod}.

\textit{\ref{SHORT.ex:inj=complete-geodesic-contractible:contractible}.}
Let $k\:\spc{Y}\hookrightarrow \ell^\infty(\spc{Y})$ be the Kuratowski embedding (\ref{lem:kuratowski}).
Observe that $\ell^\infty(\spc{Y})$ is contractible;
in particular, there is a homotopy $k_t\:\spc{Y}\hookrightarrow \ell^\infty(\spc{Y})$ such that $k_0=k$ and $k_1$ is a constant map.
(In fact, one can take $k_t=(1-t)\cdot k$.)

Since $k$ is distance-preserving and $\spc{Y}$ is injective,
there is a short map $f\:\ell^\infty(\spc{Y})\to \spc{Y}$ such that the composition $f\circ k$ is the identity map on $\spc{Y}$.
The composition $f\circ k_t\:\spc{Y}\hookrightarrow \spc{Y}$ provides the needed homotopy. 

\parbf{\ref{ex:bicombing}.}
By \ref{lem:kuratowski}, the space $\spc{Y}$ can be considered as a subset in $\ell^\infty(\spc{Y})$.
Given $x,y\in \spc{Y}$, let $\tilde\gamma_{x,y}(t)=(1-t)\cdot x+t\cdot y\in \ell^\infty(\spc{Y})$.
Observe that $\tilde\gamma_{x,y}$ meets all the conditions.
Apply the definition of injective space to $\ell^\infty(\spc{Y})$.

\parit{Remark.} The choice of geodesic paths as in the exercise is called \index{geodesic bicombing}\emph{geodesic bicombing}; it was introduced by Urs Lang \cite[3.6]{lang-2013}.

\parbf{\ref{ex:injective-spaces}.}
Suppose that a short map $f\:A\to\spc{Y}$ is defined on a subset $A$ of a metric space $\spc{X}$.
We need to construct a short extension $F$ of $f$.
Without loss of generality, we may assume that $A\ne\emptyset$;
otherwise, map the whole $\spc{X}$ to a single point.
By Zorn's lemma, it is sufficient to enlarge $A$ by a single point $x\notin A$.

\parit{\ref{SHORT.ex:injective-spaces:R}.}
Suppose $\spc{Y}=\RR$.
Set 
\[F(x)=\inf\set{f(a)-\dist{a}{x}{}}{a\in A}.\] 
Observe that $F$ is short and $F(a)=f(a)$ for any $a\in A$.

\parit{\ref{SHORT.ex:injective-spaces:tree}.}
Suppose  $\spc{Y}$ is a complete metric tree.
Fix points $p\in \spc{X}$ and $q\in\spc{Y}$.
Given a point $a\in A$,
let $x_a\in\cBall[f(a),\dist{a}{p}{}]$ be the point closest to $f(x)$.
Note that $x_a\in[q\,f(a)]$ and either $x_a=q$ or $x_a$ lies on distance $\dist{a}{p}{}$ from $f(a)$.

Note that the geodesics $[q\,x_a]$ are nested;
that is, for any $a,b\in A$ we have either $[q\,x_a]\z\subset [q\,x_b]$ or $[q\,x_b]\z\subset [q\,x_a]$.
Moreover, in the first case, we have $\dist{x_b}{f(a)}{}\le \dist{p}{a}{}$ and in the second $\dist{x_a}{f(b)}{}\le \dist{p}{b}{}$.

It follows that the closure of the union of all geodesics $[q\,x_a]$ for $a\in\spc{A}$ is a geodesic.
Denote by $x$ its endpoint; it exists since $\spc{Y}$ is complete.
It remains to observe that $\dist{x}{f(a)}{}\le \dist{p}{a}{}$ for any $a\in\spc{A}$;
that is, one can take $f(p)=x$.

\parit{\ref{SHORT.ex:injective-spaces:ell-infty}.}
Suppose $\spc{Y}=(\RR^2,\ell^\infty)$.
Note that $\spc{X}\z\to (\RR^2,\ell^\infty)$ is a short map if and only if both of its coordinate projections are short.
It remains to apply \ref{SHORT.ex:injective-spaces:R}.
The general case of $\ell^\infty(\spc{S})$ can be done the same way.

More generally, \textit{any $\ell^\infty$-product of injective spaces is injective};
in particular, if $\spc{Y}$ and $\spc{Z}$ are injective then the product $\spc{Y}\times\spc{Z}$ equipped with the metric 
\[\dist{(y,z)}{(y',z')}{\spc{Y}\times\spc{Z}}=\max\{\,\dist{y}{y'}{\spc{Y}},\dist{z}{z'}{\spc{Z}}\,\}\]
is injective as well.

\parbf{\ref{ex:extr-ball}}; \ref{SHORT.ex:extr-ball:one}.
Let $\spc{B}=\cBall[o,R]_{\spc{Y}}$.
Choose a metric space $\spc{X}$ with a subset $A$.
Given a short map $f\:A\to \spc{B}$ we need to find its short extension $\spc{X}\to \spc{B}$.

Since $\diam\spc{B}\le 2\cdot R$, we may assume that  $\diam \spc{X}\le 2\cdot R$;
if not pass to the metric defined by $\dist{x}{y}{}=\max\{\,\dist{x}{y}{\spc{X}},2\cdot R\,\}$.

Let us add point $w$ to $\spc{X}$ such that $\dist{w}{x}{}=R$ for any $x\in\spc{X}$;
denote the obtained space $\spc{X}'$.
Let $f'\:A\cup\{w\}\to \spc{B}$ be an extension of $f$ by $w\mapsto o$; note that $f'$ is short.

Since $\spc{Y}$ is injective, there is a short extension $F\:\spc{X}'\to \spc{Y}$ of $f'$.
Show and use that $F(\spc{X}')\subset \spc{B}$.

\parit{\ref{SHORT.ex:extr-ball:many}.}
Let $\spc{B}=\cap_\alpha\cBall[o_\alpha,R_\alpha]_{\spc{Y}}$.
Try to modify the argument in \ref{SHORT.ex:extr-ball:one}.

(Note that one may assume that $\diam \spc{X}\z\le 2\cdot \inf_\alpha\{\,R_\alpha\,\}$.
Consider the space $\spc{X}'\z=\spc{X}\cup\{w_\alpha\}$ such that $\dist{w_\alpha}{x}{}=R_\alpha$ for any $x\in \spc{X}$ and $\dist{w_\alpha}{w_\beta}{}=R_\alpha+R_\beta$ if $\alpha\ne\beta$.
Further, consider an extension of $f$ by $w_\alpha\mapsto o_\alpha$.)

\parbf{\ref{ex:extr-fixed}.}
Let $\diam \spc{Y}=2\cdot R$.
We can assume that $R>0$; otherwise there is nothing to prove.
Denote by $\spc{Z}$ a minimal (with respect to inclusion) intersection of closed $R$-balls in $\spc{Y}$ such that $s(\spc{Z})\subset\spc{Z}$.

Consider 
the intersection 
\[\spc{Y}'=\spc{Z}\cap\left(\bigcap_{p\in \spc{Z}} \cBall[p,R]_{\spc{Y}}\right).\]
By \ref{ex:extr-ball:many}, $\spc{Y}'$ is injective.
Use that $\spc{Z}$ is minimal to show that $s(\spc{Y}')\subset \spc{Y}'$.
Show that $\diam \spc{Y}'\le \tfrac12\cdot\diam \spc{Y}$.

Consider a sequence of nested injective spaces $\spc{Y}=\spc{Y}_0\supset \spc{Y}_1\supset\dots$ such that $\spc{Y}_{n+1}\z=\spc{Y}_{n}'$.
Choose a point $y_n\in \spc{Y}_{n}$ for each $n$.
Show that the sequence $y_n$ is Cauchy.
By \ref{ex:inj=complete-geodesic-contractible:complete}, $y_n$ converges, say to $y_\infty$.
Observe that $y_\infty$ is a fixed point of $s$.

\parbf{\ref{ex:circle};} \textit{only-if part}.
Suppose $r$ is extremal.
By \ref{lem:extremal-lipschitz}, $r$ is $1$-Lipschitz.
Since $\mathbb{S}^1$ is compact, \ref{lem:opposite-compact} implies that for any $p\in \mathbb{S}^1$ there is $q\in \mathbb{S}^1$ such that 
\[r(p)+r (q) = \dist{p}{q}{\mathbb{S}^1}.\]
Therefore
\begin{align*}
\pi&=\dist{p}{(-p)}{\mathbb{S}^1}\le 
\\
&\le 
r(p)+r(-p)=
\\
&=
r(p)+r(q) +r(-p) -r(q)\le
\\
&\le
\dist{p}{q}{\mathbb{S}^1}+\dist{q}{(-p)}{\mathbb{S}^1}=
\\
&=\pi.
\end{align*}
So, we have equality in both places, and the only-if part follows.

\parit{If part.}
Assume $r$ is a 1-Lipschitz function such that $r(p)+r(-p)=\pi$.
Then 
\begin{align*}
\dist pq{\mathbb{S}^1}&=
\dist{p}{(-p)}{\mathbb{S}^1}-\dist{q}{(-p)}{\mathbb{S}^1}\ge
\\
&\ge\pi -(r(-p)-r(q))=
\\
&=r(p)+r(q).
\end{align*}
Therefore $r$ is admissible.

Finally, if $r$ is not extremal, then there is an admissible function $s\le r$ such that $s(p)<r(p)$ for some $p$.
The latter contradicts the equality $r(p)+r(-p)=\pi$.

\parit{Source:} \cite[Proposition 2.7]{zuest}.

\parbf{\ref{ex:retraction}.}
Show and use that
$s^*(x)+s(y)\ge \dist{x}{y}{}$
for any $x,y\in \spc{X}$.

\parit{Remarks.}
It is easy to check that $q\:s\z\mapsto \tfrac12\cdot(s+s^*)$ is a short map on the space of admissible functions (with sup-norm).
Moreover, iterating $q$ and passing to the limit, we get a short retraction from the space of admissible functions to the space of extremal functions on $\spc{X}$ \cite[see 3.1 in][]{lang-2013}.
The existence of such a map will also follow from \ref{thm:inj-envelope}.

\parbf{\ref{ex:one-point-gluing}.}
Apply \ref{thm:injective=hyperconvex:balls}.

\parit{Comment.}
Conditions under which gluings of injective spaces is injective were studied by Benjamin Miesch and Maël Pavón \cite{miesch,miesch-pavon}.

\parbf{\ref{ex:Rm-ell-infty}.}
Let $B=\cBall[0,1]$ and $P\supset B$ be a parallelepiped of minimal volume.
Choose the basis $e_1,\dots,e_m$ parallel to the edges of $P$ so that in the corresponding coordinates the parallelepiped is described by inequalities
$|x_i|\le 1$ for all $i$.

Let $B_i=\cBall[(1+R)\cdot e_i,R]$ for some $R>0$.
Show that $ e_i\in B$ for any $i$; in particular $B\cap B_i\ne\emptyset$.
Show $P$ can be chosen so that $B_i\cap B_j\ne \emptyset$ for all $i$ and~$j$ and all large $R>0$.
Apply hyperconvexity to show that $e_1+\dots+ e_m\in B$.
The same way, show that $\pm e_1\pm \dots\pm e_m\in B$ for all choices of signs.
Conclude that $B=P$.

\parbf{\ref{ex:compact-hyperconvex}.}
Observe that closed balls are compact and
apply the finite intersection property.

\parbf{\ref{ex:urysohn-hyperconvex}.}
Denote by $\spc{U}_d$ the $d$-Urysohn space,
so $\spc{U}_\infty$ is the Urysohn space.

The extension property implies finite hyperconvexity.
It remains to show that $\spc{U}_d$ is not countably hyperconvex.

Suppose that $d<\infty$.
Then $\diam\spc{U}_d=d$ and for any point $x\in\spc{U}_d$ there is a point $y\in\spc{U}_d$ such that $\dist{x}{y}{\spc{U}_d}=d$.
It follows that there is no point $z\in\spc{U}_d$ such that $\dist{z}{x}{\spc{U}_d}\le \tfrac d2$ for any $x\in\spc{U}_d$.
Whence $\spc{U}_d$ is not countably hyperconvex.

Use \ref{ex:sphere-in-urysohn:midpoint} to reduce the case $d=\infty$ to the case $d<\infty$.

\parbf{\ref{ex:almost-hyperconvex}.}
Let $p_0$ be a point provided by the definition of almost hyperconvexity;
that is $\dist{x_\alpha}{p_0}{}\le r_\alpha+\eps_0$ for a given $\eps_0>0$.
We may assume that $\delta_0=\sup\{\,\dist{x_\alpha}{p_0}{}- r_\alpha\,\}>0$; otherwise the problem is solved.
Clearly, $\delta_0\le \eps_0$.

Let $p_1$ be a point provided by the definition for $\eps_1<\tfrac1{10}\cdot\delta_0$ we get a point 
$p_1$ such that $\dist{x_\alpha}{p_1}{}\le r_\alpha+\eps_1$ and $\dist{p_0}{p_1}{}\le \delta_0+\eps_1$.
Again, we may assume that $\delta_1=\sup\{\,\dist{x_\alpha}{p_1}{}- r_\alpha,\dist{p_0}{p_1}{}\,\}>0$, and we have $\delta_1\le \eps_1$.

Continuing this way, we get a sequence $p_0,p_1,\dots$ that either terminates and in this case the problem is solved, or it is an infinte Cauchy sequence.
In the latter case, its limit $p_\infty$ satisfies $\dist{x_\alpha}{p_\infty}{}\le r_\alpha$ for any $\alpha$.

\parit{Comment.}
This solution reminds the proof of \ref{prop:completion-univeral};
a more exact statement was proved by Benjamin Miesch and Maël Pavón \cite[2.2]{miesch-pavon2016};
namely they show that almost $n$-hyperconvexity implies $(n-1)$-hyperconvexity.


\parbf{\ref{ex:Inj(compact)}.}
Observe and use that the functions in $\Inj\spc{X}$ are 1-Lipschitz and uniformly bounded.

\parbf{\ref{ex:tripod+square}}; \ref{SHORT.ex:tripod+square:2}.
Use \ref{lem:opposite-compact} to show that if $f$ is extremal if and only if $f(v)=x$ and $f(w)=1-x$ for some $x\in [0,1]$.
Conclude that $\Inj\spc{X}$ is isometric to the unit interval $[0,1]$.

\parit{\ref{SHORT.ex:tripod+square:tripod}.}
Let $f$ be an extremal function.
By \ref{lem:opposite-compact}, at least two of the numbers $f(a)+f(b)$, $f(b)+f(c)$, and $f(c)+f(a)$ are $1$.
It follows that for some $x\in[0,\tfrac12]$, we have 
\begin{align*}
f(a)&=1\pm x,&
f(b)&=1\pm x,&
f(c)&=1\pm x,
\end{align*}
where we have one ``minus'' and two ``pluses'' in these three formulas.

Suppose that
\begin{align*}
g(a)&=1\pm y,& g(b)&=1\pm y,& g(c)&=1\pm y
\end{align*}
is another extremal function.
Then $|f-g|\z=|x-y|$ if $g$ has ``minus'' at the same place as $f$ and $|f-g|=|x+y|$ otherwise.

It follows that $\Inj\spc{X}$ is isometric to a {}\emph{tripod} --- three segments of length $\tfrac12$ glued at one end.

\begin{Figure}
\begin{minipage}{.48\textwidth}
\centering
\includegraphics{mppics/pic-3}
\end{minipage}\hfill
\begin{minipage}{.48\textwidth}
\centering
\includegraphics{mppics/pic-4}
\end{minipage}
\vskip-4mm
\end{Figure}

\parit{\ref{SHORT.ex:tripod+square:square}.}
Assume $f$ is an extremal function.
Use \ref{lem:opposite-compact} to show that
\begin{align*}
2&=f(x)+f(y)=
\\
&=f(p)+f(q);
\end{align*}
in particular, two values $a=f(x)-1$ and $b\z=f(p)-1$ completely describe the function $f$.
Since $f$ is extremal, we also have that 
\[(1\pm a)+(1\pm b)\ge 1\]
for all 4 choices of signs;
equivalently, 
\[|a|+|b|\le 1.\]

It follows that $\Inj\spc{X}$ is isometric to the rhombus $|a|+|b|\le 1$ in the $(a,b)$-plane with the metric induced by the $\ell^\infty$-norm.

\parit{Remarks.}
If $\spc{X}$ has $n$-points, then (evidently) $\Inj\spc{X}$ is a polyhedral complex in $(\RR^n,\ell^\infty) \z=\ell^\infty(\spc{X})$;
each face of the complex is defined by equalities and inequalities of the following type: $x_i+x_j\ge \const$ and  $x_i+x_j= \const$.
It is easy to see (and follows from \ref{ex:Rm-ell-infty}) that each face is isometric to a convex polyhedron in  $(\RR^k,\ell^\infty)$ for some $k\le n$;
in fact $k\le n/2$.
The structure of the complex can be encoded by certain graphs with the vertex set $\spc{X}$ \cite[see Section 4 in][]{lang-2013}.

\parbf{\ref{ex:kur-inj}.}
Recall that $x\mapsto \distfun_x$ gives an isometric embedding $\spc{X}\z\hookrightarrow\ell^\infty(\spc{X})$;
so we can identify $\spc{X}$ with a subset of $\ell^\infty(\spc{X})$.
Further, $\Inj\spc{X}$ is a subset of $\ell^\infty(\spc{X})$.
It is sufficient to show that $\Inj\spc{X}=G$.

Use \ref{lem:opposite-compact} to show that $\Inj\spc{X}\subset G$.

Given $g\in G$, show that $g(x)=\dist{g}{x}{\ell^\infty(\spc{X})}$.
Conclude that $g$ is admissible and apply \ref{lem:opposite-compact}.

\parit{Source:} Private communications with Rostislav Matveyev.

\parbf{\ref{ex:4-on-a-line}.}
Recall that 
\[\dist{f}{g}{\Inj\spc{X}}=\sup\set{|f(x)-g(x)|}{x\in\spc{X}}\]
and 
\[\dist{f}{p}{\Inj\spc{X}}=f(p)\]
for any $f,g\in \Inj\spc{X}$ and $p\in \spc{X}$.

Since $\spc{X}$ is compact we can find a point $p\in\spc{X}$ such that 
\begin{align*}
\dist{f}{g}{\Inj\spc{X}}&=|f(p)-g(p)|=
\\
&=\left|\dist{f}{p}{\Inj\spc{X}}-\dist{g}{p}{\Inj\spc{X}}\right|.
\end{align*}
Without loss of generality, we may assume that 
\[\dist{f}{p}{\Inj\spc{X}}
=
\dist{g}{p}{\Inj\spc{X}}
+
\dist{f}{g}{\Inj\spc{X}}.\]
Applying \ref{lem:opposite-compact}, we can find a point $q\in\spc{X}$ such that 
\[\dist{q}{p}{\Inj\spc{X}}
=
\dist{f}{p}{\Inj\spc{X}}
+
\dist{f}{q}{\Inj\spc{X}},\]
whence the result.

Since $\Inj\spc{X}$ is injective (\ref{prop:InjX-is-injective}), by \ref{ex:inj=complete-geodesic-contractible:geodesic} it has to be geodesic. It remains to note that the concatenation of geodesics $[pq]$, $[gf]$, and $[fq]$ is a required geodesic $[pq]$.

\parbf{\ref{ex:delta-hyp}.} The only-if part follows since $\spc{X}$ is isometric to a subset of $\Inj\spc{X}$.

The if part means that 
\[
\begin{aligned}
\dist{f}{g}{}+\dist{v}{w}{}\le
\max\{\,
&\dist{f}{v}{}+\dist{g}{w}{},\\
\dist{f}{w}{}+&\dist{g}{v}{}
\,\}+2\cdot\delta
\end{aligned}
\eqlbl{eq:fgvw-hyp}\]
for any $f,g,v,w\in \Inj\spc{X}$.

Suppose $\spc{X}$ is compact. 
Applying \ref{ex:4-on-a-line}, we can choose $p,q,x,y\in \spc{X}$  such that 
\[
\begin{aligned}
\dist{p}{f}{}+\dist{f}{g}{}+\dist{g}{q}{}&=\dist{p}{q}{}
\\
\dist{x}{v}{}+\dist{v}{w}{}+\dist{w}{y}{}&=\dist{x}{y}{}
\end{aligned}
\eqlbl{eq:pfgq+xvwy}
\]

Since $\spc{X}$ is $\delta$-hyperbolic, we have
\[\begin{aligned}
\dist{p}{q}{}+\dist{x}{y}{}\le
\max\{\,&\dist{p}{x}{}+\dist{q}{y}{},
\\
\dist{p}{y}{}+&\dist{q}{x}{}\,\}+2\cdot\delta.
\end{aligned}\]
Show that this inequality, together with the triangle inequality and \ref{eq:pfgq+xvwy} imply \ref{eq:fgvw-hyp}.

For the noncompact case, prove an approximate version of \ref{eq:pfgq+xvwy} and apply it the same way.

\parbf{\ref{ex:inj-envelope}.}
Show that there is unique isometry of $\Inj\spc{X}$ that is indentity of $\spc{X}$.
Use it together with \ref{thm:inj-envelope}.


\parbf{\ref{ex:d-p-inclusion}.}
Show that there is a pair of short maps 
$\Inj\spc{X}\to\Inj\spc{U}\to\Inj\spc{X}$ 
such that their composition is the identity on $\spc{X}$.
Make a conclusion.

\parbf{\ref{ex:hemisphere-inj}.}
Apply \ref{lem:opposite-compact} to show that for any $u\in\mathbb{S}^2_+$ the restriction $f_u\z\df\distfun_u|_{\mathbb{S}^1}$ is extremal function on $\mathbb{S}^1$.
Moreover, the function $f_u$ uniquely determines $u$. 
Make a conclusion.

\parbf{\ref{ex:3-4-hypreconvex}.}
Observe that coordinate functions are monotonic on any geodesic in $\ell^1$.
Use it to show that $\ell^1$ is a \emph{median space};
that is, for any three points $x,y,z$ there is a {}\emph{unique} point $m$ (it is called \index{median}\emph{median} of $x$, $y$, and $z$) that lies on {}\emph{some} geodesics $[xy]$, $[xz]$ and $[yz]$.
Apply it to show that $\ell^1$ is 3-hyperconvex.

The 4-hyperconvexity fails for the unit balls centered at four even vertices of the cube $([0,1]^3,\ell^1)$.


\parbf{\ref{ex:ultrametric}.}
Choose three points $x,y,z\in\spc{X}$ and set $\spc{A}=\{x,z\}$.
Let $f\:\spc{A}\z\to \spc{A}$ be the identity map.
Then $F(y)=x$ or $F(y)=z$.
The strong triangle inequality easily follows in both cases.

\parbf{\ref{ex:ultrametric-converse}}; \textit{main part.}
Choose a maximal subset $A\z\supset K$ that admits a short retraction $f\:A\to K$;
it exists by Zorn's lemma.
If $A$ is the whole space, then the problem is solved.
Otherwise, choose $p\notin A$.

Choose a sequence of points $a_n\in A$ such that $\dist{a_n}{p}{}$ converge to the exact lower bound on the distances from points in $A$ to $p$.
Since $K$ is compact, we can pass to a subsequence of $a_n$ such that $f(a_n)$ converges.
Let 
\[f(p)=\lim f(a_n).\]

It remains to check that 
\[\dist{f(a)}{f(p)}{}\le\dist{a}{p}{}\eqlbl{eq:short-retract}\]
for any $a\in A$.
Choose $\eps>0$; note that 
\begin{align*}
\dist{a_n}{p}{}&<\dist{a}{p}{}+\eps
\intertext{and}
\dist{f(a_n)}{f(p)}{}&<\dist{f(a)}{f(a_n)}{}+\eps
\end{align*}
for all large~$n$.
Therefore, 
\begin{align*}
\dist{f(a)}{f(p)}{}&\le \max\{\,\dist{f(a)}{f(a_n)}{},
\\
&\qquad\dist{f(a_n)}{f(p)}{}\,\}\le
\\
&\le \dist{f(a)}{f(a_n)}{}+\eps\le
\\
&\le \dist{a}{a_n}{} +\eps\le 
\\
&\le \max\{\,\dist{a}{p}{},\dist{a_n}{p}{}\,\}+\eps< 
\\
&< \dist{a}{p}{}+2\cdot\eps.
\end{align*}
Since $\eps>0$ is arbitrary, we get \ref{eq:short-retract}.

\parit{Example.}
Consider set of $\{\infty,1,2,\dots\}$ with metric defined by 
\[|m-n|=1+\frac1{\min\{m,n\}}\]
for $m\ne n$.
Observe that the space is complete, the subset $\{1,2,\dots\}$ is closed, but it is not a short retract of the ambient space.

\parbf{\ref{ex:petrunin-stadler}.} Consider the space $\spc{Y}^{\spc{X}}$ of all maps $\spc{X}\z\to \spc{Y}$ equipped with the product topology.

Denote by $\mathfrak{S}_F$ the set of maps $h\in \spc{Y}^\spc{X}$ such that the restriction $h|_F$  is short and agrees with $f$ in $F\cap A$.
Note that the sets $\mathfrak{S}_F\subset \spc{Y}^\spc{X}$ are closed and any finite intersection of these sets is nonempty.

According to Tikhonov's theorem, $\spc{Y}^{\spc{X}}$ is compact.
By the finite intersection property, the intersection $\bigcap_F\mathfrak{S}_F$ for all finite sets $F\subset X$ is nonempty.
Hence the statement follows.

\parit{Source:} \cite{petrunin-stadler}.

%%%%%%%%%%%%%%%%%%%%%%%%%%%%%%
\refstepcounter{chapter}
\setcounter{eqtn}{0}

\parbf{\ref{ex:ultrakatetov}.} 
Let $F=\set{n\in \NN}{f(n)=n}$; we need to show that $\omega(F)=1$.

Consider an oriented graph $\Gamma$ with vertex set $\NN\setminus F$ such that $m$ is connected to $n$ if $f(m)=n$.
Show that each connected component of $\Gamma$ has at most one cycle.
Use it to subdivide vertices of $\Gamma$ into three sets $S_1$, $S_2$, and $S_3$ such that $f(S_i)\cap S_i=\emptyset$ for each $i$.

Conclude that $\omega(S_1)=\omega(S_2)=\omega(S_3)=0$ and hence \[\omega(F)=\omega(\NN\setminus(S_1\cup S_2\cup S_3))=1.\]

\parit{Source:} 
The presented proof was given by Robert Solovay \cite{solovay}, but
the key statement is due to Miroslav Katětov \cite{katetov}.

\parbf{\ref{ex:linear}.}
Choose a nonprincipal ultrafilter $\omega$ and set $L(\bm{s})=s_\omega$.
It remains to observe that $L$ is linear.

\parit{Remark.} 
This construction identifies ultrafilters with vectors in $(\ell^\infty)^*$.
Recall that $\ell^\infty=(\ell^1)^*$ and $\ell^1\subsetneq(\ell^\infty)^*$.
A principle ultrafilter is a basis vector in $\ell^1$; 
nonprincipal ultrafilters lie in $(\ell^\infty)^*\setminus\ell^1$.
The set of ultrafilters is the closure of basis vectors in $\ell^1$ with respect to weak*-topology on $(\ell^\infty)^*$.


\parbf{\ref{ex:ultrakatetov+}.}
Use \ref{ex:ultrakatetov}.

\parbf{\ref{ex:lim(tree)}.}
Let $\gamma$ be a path from $p$ to $q$ in a metric tree $\spc{T}$.
Assume that $\gamma$ passes thru a point $x$ on distance $\ell$ from $[pq]$.
Then 
\[\length\gamma\ge \dist{p}{q}{}+2\cdot \ell.
\eqlbl{eq:+ell}\]

Suppose that $\spc{T}_n$ is a sequence of metric trees that $\omega$-converges to $\spc{T}_\omega$.
By \ref{obs:ultralimit-is-geodesic}, the space $\spc{T}_\omega$ is geodesic.

The uniqueness of geodesics follows from \ref{eq:+ell}.
Indeed, if for a geodesic $[p_\omega q_\omega]$ there is another geodesic $\gamma_\omega$ connecting its ends, then it has to pass thru a point $x_\omega\notin [p_\omega q_\omega]$.
Choose sequences $p_n,q_n,x_n\in\spc{T}_n$ such that $p_n\to p_\omega$, $q_n\to q_\omega$, and $x_n\to x_\omega$ as $n\to\omega$.
Then 
\begin{align*}
\dist{p_\omega}{q_\omega}{}&=\length\gamma\ge 
\\
&\ge\lim_{n\to\omega}(\dist{p_n}{x_n}{}+\dist{q_n}{x_n}{})\ge
\\
&\ge \lim_{n\to\omega}(\dist{p_n}{q_n}{}+2\cdot\ell_n)=
\\
&=\dist{p_\omega}{q_\omega}{}+2\cdot\ell_\omega.
\end{align*}
Since $x_\omega\notin [p_\omega q_\omega]$, we have that $\ell_\omega>0$ --- a contradiction.

It remains to show that any geodesic triangle $\spc{T}_\omega$ is a tripod.
Consider the sequence of centers of tripods $m_n$ for given sequences of points $x_n,y_n,z_n\in \spc{T}_n$.
Observe that its ultralimit $m_\omega$ is the center of the tripod with ends at $x_\omega,y_\omega,z_\omega\in \spc{T}_\omega$.

\parbf{\ref{ex:ultracompact}.}
Construct $\bm{X}$ and distance-preserving embeddings $\spc{X}_n\hookrightarrow\bm{X}$ that satisfy \ref{propery:GH}.
Given $x_\infty\in \spc{X}_\infty$, choose a sequence $x_n\in \spc{X}_n$ such that $x_n\to x_\infty$ in $\bm{X}$.
Let $x_\omega$ be $\omega$-limit of the sequence $x_n$ in $\bm{X}$.
Note that $x_\omega\in \spc{X}_\infty$.
Show that the map $x_\infty\mapsto x_\omega$ is defined; that is, it does not depend on the choice of the sequence $x_n$.
Further, show that the map $x_\infty\mapsto x_\omega$ is an isometry of $\spc{X}_\infty$.
Make a conclusion.

\parbf{\ref{ex:ultrapower}.}
Further, we consider $\spc{X}$ as a subset of $\spc{X}^\omega$.

\parit{\ref{SHORT.ex:ultrapower:a}.} Follows directly from the definitions.

\parit{\ref{SHORT.ex:ultrapower:compact}.}
Suppose $\spc{X}$ compact.
Given a sequence $x_1,x_2,\dots{}\in\spc{X}$, denote its $\omega$-limit in $\spc{X}^\omega$ by $x^\omega$ and its $\omega$-limit in $\spc{X}$ by $x_\omega$.

Observe that $x^\omega=\iota(x_\omega)$.
Therefore, $\iota$ is onto.

If $\spc{X}$ is not compact, we can choose a sequence $x_n$ such that $\dist{x_m}{x_n}{}>\eps$ for fixed $\eps>0$ and all $m\ne n$.
Observe that
\[\lim_{n\to\omega}\dist{x_n}{y}{\spc{X}}\ge \tfrac\eps2\]
for any $y\in\spc{X}$.
It follows that $x_\omega$ lies at the distance $\ge\tfrac\eps2$ from~$\spc{X}$.

\parit{\ref{SHORT.ex:ultrapower:proper}.}
A sequence of points $x_n$ in $\spc{X}$ will be called $\omega$-bounded if there is a real constant $C$ such that
\[\dist{p}{x_n}{\spc{X}}\le C\] 
for $\omega$-almost all $n$.

The same argument as in \ref{SHORT.ex:ultrapower:compact} shows that any $\omega$-bounded sequence has its $\omega$-limit in $\spc{X}$.
Further, if $(x_n)$ is not  $\omega$-bounded, then 
\[\lim_{n\to\omega}\dist{p}{x_n}{\spc{X}}=\infty;\]
that is, $x_\omega$ does not lie in the metric component of $p$ in $\spc{X}^\omega$.

\parbf{\ref{ex:isom-ultrapower}.}
Let us identify points in $\spc{X}$ with nonnegative integers.
Consider the set $\mathcal{A}$ of all sequences $a_n$ such that $a_0=0$ and $a_{n+1}=a_n+\eps_n\cdot 2^n$ where $\eps_n\in\{0,1\}$ for any $n$.
Observe that $\mathcal{A}$ has cardinality continuum and distinct sequences in $\mathcal{A}$ have distinct $\omega$-limits.
Conclude that the cardinality of $\spc{X}^\omega$ is at least continuum.

Show and use that the spaces $\spc{X}^\omega$ and $(\spc{X}^\omega)^\omega$ have discrete metrics and both have cardinality at most continuum.


\parbf{\ref{ex:ultrapower(ultrapower)}.}
Choose a bijection $\iota\:\NN\to \NN\times \NN$.
Given a set $S\subset \NN$, consider the sequence $S_1$, $S_2,\dots$ of subsets in $\NN$ defined by $m\in S_n$ if $(m,n)\z=\iota(k)$ for some $k\in S$.
Set $\omega_1(S)=1$ if and only if $\omega(S_n)=1$ for $\omega$-almost all $n$.
It remains to check that $\omega_1$ meets the conditions of the exercise.

\parit{Comment.}
It turns out that $\omega_1\ne \omega$ for any $\iota$;
see the post of Andreas Blass \cite{blass}.

\parbf{\ref{ex:two-geodesics-in-ultrapower}.}
Arguing as in \ref{obs:ultrapower-is-geodesic}, we get a pair of points $x$ and $y$ in $\spc{X}$ such that
\[\dist{p}{x}{}+\dist{x}{y}{}+\dist{y}{q}{}=\dist{p}{q}{}\]
and there is no midpoint between $x$ and $y$ in $\spc{X}$
(possibly $p=x$ and $q=y$).
Note that it is sufficient to show that there is a continuum of distinct midpoints in $\spc{X}^\omega$ between $x$ and $y$ in $\spc{X}$.

Since $\spc{X}$ is a length space, we can choose a $\tfrac1n$-midpoint $m_n\in\spc{X}$ between $x$ and $y$.
Note that the sequence $m_n$ contains no converging subsequence.
Conclude that we may pass to a subsequence of $m_n$ such that $\dist{m_i}{m_j}{}>\eps$ for a fixed $\eps>0$ and any $i\ne j$.

Argue as in \ref{ex:isom-ultrapower} to show that there is a continuum of distinct $\omega$-limits of subsequences of $m_n$;
each such limit is a midpoint between $x$ and $y$.

\parit{\ref{SHORT.ex:sphere-in-urysohn:homogeneous}.} 
Use \ref{SHORT.ex:sphere-in-urysohn:sphere}, maybe twice.

\parbf{\ref{ex:notproper-limit}.} Consider the infinite metric $\spc{T}$ tree with unit edges shown
on the diagram.
Observe that $\spc{T}$ is proper.

\begin{Figure}
\vskip-0mm
\centering
\includegraphics{mppics/pic-605}
\end{Figure}

Consider the vertex $v_\omega=\lim_{n\to\omega}v_n$ in the ultrapower $\spc{T}^\omega$.
Observe that $\omega$ has an infinite degree.
Conclude that $\spc{T}^\omega$ is not locally compact.

\parbf{\ref{ex:ultraT}.}
Consider a product of an infinite sequence of two-point spaces.

\parit{Remark.}
There are such examples with cocompact isometric action of finitely generated group \cite{thomas-velickovic}.

\parbf{\ref{ex:Asym(Lob)}.} Assume $\spc{L}$ is the Lobachevsky plane.

\parbf{\ref{SHORT.ex:Asym(Lob):metric-tree}.}
Show that there is $\delta>0$ such that sides of any geodesic triangle in $\spc{L}$ intersect a disk of radius $\delta$.
Conclude that any geodesic triangle in $\Asym\spc{L}$ is a tripod.

\parit{\ref{SHORT.ex:Asym(Lob):homogeneous}.} Observe that $\spc{L}$ is one-point-homogeneous and use it.

\parit{\ref{SHORT.ex:Asym(Lob):continuum}.} 
By \ref{SHORT.ex:Asym(Lob):homogeneous}, it is sufficient to show that $p_\omega$ has a continuum degree.

Choose distinct geodesics $\gamma_1,\gamma_2\:[0,\infty)\z\to L$ that start at a point $p$.
Show that the limits of $\gamma_1$ and $\gamma_2$ run in the different connected components of $(\Asym\spc{L})\setminus \{p_\omega\}$.
Since there is a continuum of distinct geodesics starting at $p$,
we get that the degree of $p_\omega$ is at least continuum.

On the other hand, the set of sequences of points in $\spc{L}$  has cardinality continuum.
In particular, the set of points in $\Asym\spc{L}$ has cardinality at most continuum.
It follows that the degree of any vertex is at most continuum.

The proof for the Lobachevsky space goes along the same lines.

For the infinite three-regular tree, part \ref{SHORT.ex:Asym(Lob):metric-tree} follows from \ref{ex:lim(tree)}.
The three-regular tree is only vertex-homogeneous; the latter is sufficient to prove \ref{SHORT.ex:Asym(Lob):homogeneous}.
No changes are needed in~\ref{SHORT.ex:Asym(Lob):continuum}.

\parit{Remark.}
According to the result of Anna Dyubina and Iosif Polterovich \cite{dyubina-polterovich}, the properties \ref{SHORT.ex:Asym(Lob):homogeneous} and \ref{SHORT.ex:Asym(Lob):continuum} describe the tree $\spc{T}$ up to isometry.
In particular, the asymptotic space of the Lobachevsky plane does not depend on the choice of the ultrafilter and the sequence $\lambda_n\to \infty$.


\parbf{\ref{ex:T(Sx[0,1]/Sx0)}.}
Denote by $o_\omega$ the point in $\T^\omega_o\spc{X}$ that corresponds to $o$.
Argue as in \ref{ex:Asym(Lob):continuum} to show that $\T^\omega_o\spc{X}\setminus \{o_\omega\}$ has continuum connected components.
Further, show that each connected component $\spc{W}_\alpha$ is isometric to $\RR\times (0,\infty)$ with the metric described by
\begin{align*}
&\dist{(x_1,t_1)}{(x_2,t_2)}{}=
\\
&\qquad=\min\{\,\dist{(x_1,t_1)}{(x_2,t_2)}{\RR^2},t_1+t_2\,\}.
\end{align*}

Conclude that the space $\T^\omega_o\spc{X}$ can be described as follows.
Consider continuum copies $\spc{W}_\alpha$ as above;
denote by $(x,t)_\alpha$ the point in $\spc{W}_\alpha$ with coordinates $(x,t)$.
The tangent space is the disjoint union of single point $o_\omega$ and all $\spc{W}_\alpha$ 
such that $\dist{(x_1,t_1)_\alpha}{(x_2,t_2)_\alpha}{}$ is the same as in $\spc{W}_\alpha$ and for the remaining pairs, we have $\dist{o_\omega}{(x,t)_\alpha}{}=t$ and $\dist{(x_1,t_1)_\alpha}{(x_2,t_2)_\beta}{}=t_1+t_2$
if $\alpha\ne\beta$.


%%%%%%%%%%%%%%%%%%%%%%%%%%%%
{\small\sloppy
\documentclass[twoside]{book}

\usepackage{lectures}
\usepackage[colorlinks=true,
citecolor=black,
linkcolor=black,
anchorcolor=black,
filecolor=black,
menucolor=black,
urlcolor=black,
pdftitle={Pure metric geometry: introductory lectures},
pdfsubject={Geometry},
pdfauthor={Anton Petrunin}
]{hyperref}
\makeindex

\begin{document}
%\pagestyle{empty}\renewcommand\includegraphics[2][{}]{}\def\emph{\textit}
%\overfullrule=100mm

 
\title{Pure metric geometry:\\
introductory lectures}
\author{Anton Petrunin}
\date{}
\maketitle

\section*{Preface}

This text can serve as an introductory part to a variety of courses in metric geometry.
Here is a graph of essential dependencies of the lectures; some statements (mostly exercises) add more dependencies, but they can be ignored.
\begin{figure}[!ht]
\centering
\begin{tikzpicture}[->,>=stealth',shorten >=1pt,auto,scale=1.4,
  thick,main node/.style={circle,draw,font=\sffamily\bfseries,minimum size=8mm}]

  \node[main node] (1) at (1,0) {\ref{chap:defs}};
  \node[main node] (2) at (.5,-5/6){\ref{chap:urysohn}};
  \node[main node] (3) at (1.5,-5/6) {\ref{chap:injective}};
  \node[main node] (4) at (2,0) {\ref{chap:hausdorff}};
  \node[main node] (5) at (3,0) {\ref{chap:GH}};
  \node[main node] (6) at (4,0) {\ref{chap:ultralimits}};
  

  \path[every node/.style={font=\sffamily\small}]
   (1) edge node{}(2)
   (1) edge node{}(3)
   (1) edge node{}(4)
   (4) edge node{}(5)
   (5) edge node{}(6);
\end{tikzpicture}
\end{figure}
The necessary definitions introduced in (\ref{chap:defs}).
In (\ref{chap:urysohn}) we discuss the Urysohn space.
In (\ref{chap:injective}) we discuss injective spaces.
In (\ref{chap:hausdorff}) we introduce Hausdorff metric.
In (\ref{chap:GH}) and (\ref{chap:ultralimits}) we discuss two types of convergences of metric spaces --- the Gromov--Hausdorff limit and ultralimit.

Applications are given only as illustrations.
We stick to domestic affairs of metric spaces, keeping away from any extra structure. 
(Adding an extra structure brings an extra tool and often opens a huge field for development.
The examples include Alexandrov geometry,
geometric group theory,
metric-measure spaces and optimal transport.)

These notes are based on the minicourse given at SPbSU (Fall 2022) and the introductory part of a course at PSU (Spring 2020).
The latter included additional material from \cite{alexander-kapovitch-petrunin-2019,petrunin2020mnfld,nabutovsky}.
A part of the text is a compilation from \cite{alexander-kapovitch-petrunin-2019, alexander-kapovitch-petrunin-2025, petrunin-yashinski, petrunin-2022-PIGTIKAL, petrunin-zamorabarrera} and its drafts.

I want to thank
Sergei Ivanov,
Urs Lang,
Alexander Lytchak,
Rostislav Matveyev,
Julien Melleray,
and Sergio Zamora Barrera for help.
The present work is partially supported by NSF grant DMS-2005279
and the Simons Foundation grant \#584781.

\thispagestyle{empty}
\tableofcontents
\thispagestyle{empty}

\chapter{Definitions}

In this lecture we give some conventions used further
and remind some the definitions related to metric spaces.


\section{Metric spaces}
\label{sec:metric spaces}

The distance between two points $x$ and $y$ in a metric space $\spc{X}$ will be denoted by $\dist{x}{y}{}$ or $\dist{x}{y}{\spc{X}}$.
The latter notation is used if we need to emphasize 
that the distance is taken in the space~${\spc{X}}$.

Let us recall the definition of metric. 

\begin{thm}{Definition}\label{def:metric}
A \index{metric}\emph{metric} on a set $\spc{X}$ is a real-valued function $(x,y)\mapsto\dist{x}{y}{\spc{X}}$ that satisfies the following conditions for any three points $x,y,z\in \spc{X}$:
\begin{enumerate}[(i)]
\item $\dist{x}{y}{\spc{X}}\ge 0$,
\item\label{metric=0} $\dist{x}{y}{\spc{X}}= 0$ $\iff$ $x=y$,
\item $\dist{x}{y}{\spc{X}}=\dist{y}{x}{\spc{X}}$,
\item $\dist{x}{y}{\spc{X}}+\dist{y}{z}{\spc{X}}\ge\dist{x}{z}{\spc{X}}$,
\end{enumerate}
\end{thm}

A set $\spc{X}$ with a metric on it is called \index{metric space}\emph{metric space};
most of the time we keep the same notation for the metric space and its underlying set.

The function 
\[\distfun_x\:y\mapsto \dist{x}{y}{}\]
is called the \index{distance function}\emph{distance function} from~$x$. 

Given $R\in[0,\infty]$ and $x\in \spc{X}$, the sets
\begin{align*}
\oBall(x,R)&=\{y\in \spc{X}\mid \dist{x}{y}{}<R\},
\\
\cBall[x,R]&=\{y\in \spc{X}\mid \dist{x}{y}{}\le R\}
\end{align*}
are called, respectively, the  \index{open ball}\emph{open} and  the \index{closed ball}\emph{closed  balls}   of radius $R$ with center~$x$.
Again, if we need to emphasize that these balls are taken in the metric space $\spc{X}$,
we write 
\[\oBall(x,R)_{\spc{X}}\quad\text{and}\quad\cBall[x,R]_{\spc{X}}.\]

\begin{thm}{Exercise}
Show that
\[\dist{p}{q}{\spc{X}}+\dist{x}{y}{\spc{X}}\le\dist{p}{x}{\spc{X}}+\dist{p}{y}{\spc{X}}+\dist{q}{x}{\spc{X}}+\dist{q}{y}{\spc{X}}\]
for any points $p$, $q$, $x$, and $y$ in a metric space $\spc{X}$.
\end{thm}

\section{Variations of definition}

\parbf{Pseudometrics.}
A metric for which the distance between two distinct points can be zero is called a \index{pseudometric}\emph{pseudometric}.
In other words, to define pseudometric, we need to remove condition (\ref{metric=0}) from \ref{def:metric}.

The following observation show that
nearly any question about pseudometric spaces can be reduced to a question about genuine metric spaces.

Assume $\spc{X}$ is a pseudometric space.
Consider an equivalence relation $\sim$ on $\spc{X}$ defined by
$x\sim y$ if and only if $\dist{x}{y}{}=0$. 
Note that if $x\sim x'$, then $\dist{y}{x}{}=\dist{y}{x'}{}$ for any $y\in\spc{X}$.
Thus, $\dist{*}{*}{}$ defines a metric on the
quotient set $\spc{X}/{\sim}$.
This way we obtain a metric space $\spc{X}'$.
The space $\spc{X}'$ is called the 
\emph{corresponding metric space} for the pseudometric space $\spc{X}$.
Often we do not distinguish between $\spc{X}'$ and~$\spc{X}$. 

\parbf{$\bm{\infty}$-metrics.}
One may also consider metrics with values in $\RR\cup\{\infty\}$;
we might call them \index{metric!$\infty$-metric}\emph{$\infty$-metrics}, but most of the time we use the term {}\emph{metric}.

Again nearly any question about $\infty$-metric spaces can be reduced to a question about genuine metric spaces. 

Indeed, let us write $x\approx y$ if  $\dist{x}{y}{}<\infty$;
this is another equivalence relation on $\spc{X}$.
The equivalence class of a point $x\in\spc{X}$ will be called the \index{metric component}\emph{metric component} 
 of $x$; it will be denoted by $\spc{X}_x$.
One could think of $\spc{X}_x$ as  $\oBall(x,\infty)_{\spc{X}}$ --- the open ball centered at $x$ and radius $\infty$ in $\spc{X}$.

It follows that any $\infty$-metric space is a {}\emph{disjoint union} of genuine metric spaces --- the metric components of the original $\infty$-metric space.

\begin{thm}{Exercise}
Given two sets $A$ and $B$ on the plane, set 
\[\dist{A}{B}{}=\mu(A\backslash B)+\mu(B\backslash A),\]
where $\mu$ denotes the Lebesgue measure.
\begin{subthm}{}
Show that $\dist{*}{*}{}$ is a pseudometric on the set of bounded measurable sets of the plane.
\end{subthm}

\begin{subthm}{}
Show that $\dist{*}{*}{}$ is an $\infty$-metric on the set of all open sets of the plane.
\end{subthm}
\end{thm}

\section{Completeness}

A metric space $\spc{X}$ is called \index{complete space}\emph{complete} if every Cauchy sequence of points in $\spc{X}$ converges in $\spc{X}$.

\begin{thm}{Exercise}\label{ex:almost-min}
Suppose that $\rho$ is a positive continuous function on a complete metric space $\spc{X}$.
Show that for any $\eps>0$ there is a point $x\in \spc{X}$ such that 
\[\rho(x)<(1+\eps)\cdot\rho(y)\]
for any point $y\in \oBall(x,\rho(x))$.
\end{thm}

Most of the time we will assume that a metric space is complete.
The following construction produces a complete metric space $\bar{\spc{X}}$ for any given metric space $\spc{X}$.


\parbf{Completion.}
Given a metric space $\spc{X}$, 
consider the set $\spc{C}$ of all Cauchy sequences in $\spc{X}$.
Note that for any two Cauchy sequences $(x_n)$ and $(y_n)$ the right hand side in \ref{eq:cauchy-dist} is defined; moreover it defines a pseudometric on~$\spc{C}$
\[\dist{(x_n)}{(y_n)}{\spc{C}}\df\lim_{n\to\infty}\dist{x_n}{y_n}{\spc{X}}.\eqlbl{eq:cauchy-dist}\]
The corresponding metric space $\bar{\spc{X}}$ is called a \index{completion}\emph{completion} of $\spc{X}$.

Note that the original space $\spc{X}$ forms a dense subset in its completion $\bar{\spc{X}}$.
More precisely,  for each point $x\in\spc{X}$ one can consider a constant sequence $x_n=x$ which is Cauchy.
It defines a natural map $\spc{X}\to \bar{\spc{X}}$.
It is easy to check that this map is distance-preserving.
In partucular we can (and will) consider $\spc{X}$ as a subset of $\bar{\spc{X}}$.

\begin{thm}{Exercise}
Show that completion of a metric space is complete.
\end{thm}


\section{Compact spaces}

Let us recall few equivalent definitions of compact metric spaces.

\begin{thm}{Definition}\label{def:compact}
A metric space $\spc{K}$ is compact if and only if one of the following equivalent condition holds:

\begin{subthm}{}
 Every open cover of $\spc{K}$ has a finite subcover.
\end{subthm}

\begin{subthm}{}
 For any open cover of $\spc{K}$ there is $\eps>0$ such that any $\eps$-ball in $\spc{K}$ lie in one element of the cover. (The value $\eps$ is called a \index{Lebesgue number}\emph{Lebesgue number} of the covering.)
\end{subthm}

\begin{subthm}{}
 Every sequence of points in $\spc{K}$ has a subsequence that converges in $\spc{K}$.
\end{subthm}

\begin{subthm}{totally-bounded}
The space $\spc{K}$ is complete and \index{totally bounded space}\emph{totally bounded}; that is, for any $\eps>0$, the space $\spc{K}$ admits a finite cover by open $\eps$-balls.
\end{subthm}

\end{thm}

A subset $N$ of a metric space $\spc{K}$ is called \index{net}\emph{$\eps$-net} if any other point $x$ lies on the distance less than $\eps$ from a point in $N$.
Note that totally bounded spaces can be defined as spaces that admit a finite $\eps$-net for any $\eps>0$.

\begin{thm}{Exercise}\label{ex:compact-net}
Show that a space $\spc{K}$ is totally bounded if and only if it contains a compact $\eps$-net for any $\eps>0$. 
\end{thm}


Let $\pack_\eps\spc{X}$ be exact upper bound on the number of points $x_1,\z\dots,x_n\in \spc{X}$ such that $\dist{x_i}{x_j}{}\ge\eps$ if $i\ne j$.

If $n=\pack_\eps\spc{X}<\infty$, then
the collection of points $x_1,\dots,x_n$ is called a \index{maximal packing}\emph{maximal $\eps$-packing}.
Note that $n$ is the maximal number of open disjoint $\tfrac\eps2$-balls in $\spc{X}$.

\begin{thm}{Exercise}\label{ex:pack-net}
Show that a complete space $\spc{X}$ is compact if and only of $\pack_\eps\spc{X}\z<\infty$ for any $\eps>0$.

Show that any maximal $\eps$-packing is an $\eps$-net.
\end{thm}


\begin{thm}{Exercise}\label{ex:non-contracting-map}
Let $\spc{K}$  be a compact metric space and
\[f\:\spc{K}\z\to \spc{K}\] 
be a distance-nondecreasing map.
Prove that $f$ is an \index{isometry}\emph{isometry};
that is, $f$ is a distance-preserving bijection.
\end{thm}

A metric space $\spc{X}$ is called \index{locally compact space}\emph{locally compact} if any point in $\spc{X}$ admits a compact neighborhood;
in other words, for any point $x\in\spc{X}$ a closed ball $\cBall[x,r]$ is compact for some $r>0$.

\section{Proper spaces}

A metric space $\spc{X}$ is called \index{proper space}\emph{proper} if all closed bounded sets in $\spc{X}$ are compact. 
This condition is equivalent to each of the following statements:
\begin{itemize}
\item For some (and therefore any) point $p\in \spc{X}$ and any $R<\infty$, 
the closed ball $\cBall[p,R]_{\spc{X}}$ is compact. 
\item The function $\distfun_p\:\spc{X}\to\RR$ is \index{proper function}\emph{proper} for some (and therefore any) point $p\in \spc{X}$;
that is, for any compact set $K\subset \RR$, its inverse image 
\[\distfun_p^{-1}(K)=\set{x\in \spc{X}}{\dist{p}{x}{\spc{X}}\in K}\]
is compact.
\end{itemize}

\begin{thm}{Exercise}\label{ex:loc-compact-not-proper}
Give an example of space which is locally compact but not proper.
\end{thm}

\section{Geodesics}
\label{sec:geods}

Let $\spc{X}$ be a metric space 
and $\II$\index{$\II$} a real interval. 
A~globally isometric map $\gamma\:\II\to \spc{X}$ is called a \index{geodesic}\emph{geodesic}%
\footnote{Various authors call it differently: {}\emph{shortest path}, {}\emph{minimizing geodesic}.
Also note that the meaning of the term \emph{geodesic} is different from what is used in Riemannian geometry, altho they are closely related.}; 
in other words, $\gamma\:\II\to \spc{X}$ is a geodesic if 
\[\dist{\gamma(s)}{\gamma(t)}{\spc{X}}=|s-t|\]
for any pair $s,t\in \II$.

If $\gamma\:[a,b]\to \spc{X}$ is a geodesic and $p=\gamma(a)$, $q=\gamma(b)$, then we say that $\gamma$ is a geodesic from point $p$ to point $q$.
In this case the image of $\gamma$ is denoted by $[p q]$\index{$[{*}{*}]$} and with an abuse of notations  we also call it a \index{geodesic}\emph{geodesic}.


We may write $[p q]_{\spc{X}}$ 
to emphasize that the geodesic $[p q]$ is in the space  ${\spc{X}}$.
We also use the following shortcut notation:
\begin{align*}
\left] p q \right[&=[pq]\backslash\{p,q\},
&
\left] p q \right]&=[pq]\backslash\{p\},
&
\left[ p q \right[&=[pq]\backslash\{q\}.
\end{align*}

In general, a geodesic from $p$ to $q$ need not exist and if it exists, it need not  be unique.  
However, once we write $[p q]$ we assume mean that we have made a choice of geodesic.

A \index{geodesic path}\emph{geodesic path} is a geodesic with constant-speed parametrization by $[0,1]$.

A curve $\gamma\:\II\to \spc{X}$  is called a \index{geodesic!local geodesic}\emph{local geodesic} if for any $t\in\II$ there is a neighborhood $U$ of $t$ in $\II$ such that the restriction $\gamma|_U$ is a  geodesic.
A constant-speed parametrization of a local geodesic by the unit interval $[0,1]$ is called a \index{geodesic!local geodesic}\emph{local geodesic path}. 

\section{Geodesic spaces and metric trees}

A metric space is called \index{geodesic}\emph{geodesic} if any pair of its points can be joined by a geodesic.

A geodesic space $\spc{T}$ is called a \index{metric tree}\emph{metric tree} if any pair of points in $\spc{T}$ are connected by a unique geodesic,
and the union of any two geodesics $[xy]$, and $[yz]$ contain the geodesic $[xz]_{\spc{T}}$.
In other words any triangle in $\spc{T}$ is a tripod;
that is, for any three geodesics $[xy]$, $[yz]$, and $[zx]$ have a common point.

\begin{thm}{Exercise}
Show that spheres in metric trees are ultrametric spaces;
that is, if $\Sigma$ is a sphere in a metric tree $\spc{T}$, then
\[\dist{x}{z}{\spc{T}}
\le
\max\{\,\dist{x}{y}{\spc{T}},\dist{y}{z}{\spc{T}}\,\}\]
for any $x,y,z\in\Sigma$.
\end{thm}

\section{Length}

A \index{curve}\emph{curve} is defined as a continuous map from a real interval to a metric space.
If the real interval is $[0,1]$, then the curve is called a \index{path}\emph{path}.

\begin{thm}{Definition}
Let $\spc{X}$ be a metric space and
$\alpha\: \II\to \spc{X}$ be a curve.
We define the \index{length}\emph{length} of $\alpha$ as 
\[
\length \alpha \df \sup_{t_0\le t_1\le\ldots\le t_n}\sum_i \dist{\alpha(t_i)}{\alpha(t_{i-1})}{}.
\]

A curve $\alpha$ is called \index{rectifiable curve}\emph{rectifiable} if $\length \alpha<\infty$.
\end{thm}



\begin{thm}{Theorem}\label{thm:length-semicont}
Length is a lower semi-continuous with respect to pointwise convergence of curves. 

More precisely, assume that a sequence
of curves $\gamma_n\:\II\to \spc{X}$ in a metric space $\spc{X}$ converges pointwise 
to a curve $\gamma_\infty\:\II\to \spc{X}$;
that is, for any fixed $t \in \II$, $\gamma_n(t)\z\to\gamma_\infty(t)$ as $n\to\infty$. 
Then 
$$\liminf_{n\to\infty} \length\gamma_n \ge \length\gamma_\infty.\eqlbl{eq:semicont-length}$$
\end{thm}


\begin{wrapfigure}{o}{20 mm}
\vskip-0mm
\centering
\includegraphics{mppics/pic-100}
\end{wrapfigure}


Note that the inequality \ref{eq:semicont-length} might be strict.
For example the diagonal $\gamma_\infty$ of the unit square 
can be  approximated by a stairs-like
polygonal curves $\gamma_n$
with sides parallel to the sides of the square ($\gamma_6$ is on the picture).
In this case
\[\length\gamma_\infty=\sqrt{2}\quad
\text{and}\quad \length\gamma_n=2\]
for any $n$.

\parit{Proof.}
Fix a sequence $t_0<t_1<\dots<t_k$ in $\II$.
Set 
\begin{align*}\Sigma_n
&\df
|\gamma_n(t_0)-\gamma_n(t_1)|+\dots+|\gamma_n(t_{k-1})-\gamma_n(t_k)|.
\\
\Sigma_\infty
&\df
|\gamma_\infty(t_0)-\gamma_\infty(t_1)|+\dots+|\gamma_\infty(t_{k-1})-\gamma_\infty(t_k)|.
\end{align*}

Note that for each $i$ we have 
\[|\gamma_n(t_{i-1})-\gamma_n(t_i)|\to|\gamma_\infty(t_{i-1})-\gamma_\infty(t_i)|\]
and therefore
\[\Sigma_n\to \Sigma_\infty\] 
as $n\to\infty$.
Note that 
\[\Sigma_n\le\length\gamma_n\]
for each $n$.
Hence
$$\liminf_{n\to\infty} \length\gamma_n \ge \Sigma_\infty.\eqlbl{>=Sigma-infty}$$

If $\gamma_\infty$ is rectifiable, we can assume that 
\begin{align*}
\length\gamma_\infty<\Sigma_\infty+\eps.
\end{align*}
for any given $\eps>0$.
By \ref{>=Sigma-infty} it follows that 
$$\liminf_{n\to\infty} \length\gamma_n > \length\gamma_\infty-\eps$$
for any $\eps>0$; whence \ref{eq:semicont-length} follows.

It remains to consider the case when $\gamma_\infty$ is not rectifiable; 
that is, $\length\gamma_\infty=\infty$.
In this case we can choose a partition so that $\Sigma_\infty>L$ for any real number $L$.
By \ref{>=Sigma-infty} it follows that 
$$\liminf_{n\to\infty} \length\gamma_n > L$$
for any given $L$; whence 
\[\liminf_{n\to\infty}\length\gamma_n=\infty\]
and \ref{eq:semicont-length} follows.
\qeds

\section{Length spaces}\label{sec:intrinsic}

If for any $\eps>0$ and any pair of points $x$ and $y$ in a metric space $\spc{X}$, there is a path $\alpha$ connecting $x$ to $y$ such that
\[\length\alpha< \dist{x}{y}{}+\eps,\]
then $\spc{X}$ is called a \index{length space}\emph{length space} and the metric on $\spc{X}$ is called a \index{length metric}\emph{length metric}.\label{page:length metric}

If $\spc{X}$ is an $\infty$-metric space, then in the above definition we assume in addition that $x$ and $y$ lie in one metric component; that is, $\dist{x}{y}{\spc{X}}<\infty$.
In other words an $\infty$-metric space $\spc{X}$ is a length space if each metric component of $\spc{X}$ is a length space.

Note that any geodesic space is a length space.
The following example shows that the converse does not hold.


\begin{thm}{Example}
Suppose a space $\spc{X}$ is obtained by gluing a countable collection of disjoint intervals $\{\II_n\}$ of length $1+\tfrac1n$, where for each $\II_n$ the left end is glued to $p$ and the right end to~$q$.

Observe that the space $\spc{X}$ carries a natural complete length metric with respect to which $\dist{p}{q}{}=1$ but there is no geodesic connecting $p$ to~$q$.
\end{thm}



\begin{thm}{Exercise}\label{ex:no-geod}
Give an example of a complete length space $\spc{X}$ such that no pair of distinct points in $\spc{X}$ can be joined by a geodesic.
\end{thm}

Directly from the definition, it follows that if a path $\alpha\:[0,1]\to\spc{X}$ connects two points $x$ and $y$ 
(that is, if $\alpha(0)=x$ and $\alpha(1)=y$), then 
\[\length\alpha\ge \dist{x}{y}{}.\]
Set 
\[\yetdist{x}{y}{}=\inf\{\length\alpha\}\]
where the greatest lower bound is taken for all paths connecing $x$ and~$y$.
It is straightforward to check that $(x,y)\mapsto \yetdist{x}{y}{}$ is an $\infty$-metric; moreover $(\spc{X},\yetdist{*}{*}{})$ is a length space.
The metric $\yetdist{*}{*}{}$ is called \index{induced length metric}\emph{induced length metric}.

\begin{thm}{Exercise}\label{ex:compact+connceted}
Let $\spc{X}$ be a complete length space.
Show that for any compact subset $K$ in $\spc{X}$
there is a compact path connected subset $K'$ that contains $K$.  
\end{thm}

\begin{thm}{Exercise}\label{ex:compact=>complete}
Suppose $(\spc{X},\dist{*}{*}{})$ is a complete metric space.
Show that $(\spc{X},\yetdist{*}{*}{})$ is complete.
\end{thm}

Let $A$ be a subset of a metric space $\spc{X}$.
Given two points $x,y\in A$,
consider the value
\[\dist{x}{y}{A}=\inf_{\alpha}\{\length\alpha\},\]
where the greatest lower bound is taken for all paths $\alpha$ from $x$ to $y$ in~$A$.
In other words $\dist{*}{*}{A}$ denotes the induced length metric on the subspace $A$.%
\footnote{The notation $\dist{*}{*}{A}$ conflicts with the previously defined notation for distance $\dist{x}{y}{\spc{X}}$ in a metric space $\spc{X}$. However, most of the time we will work with ambient length spaces where the meaning will be unambiguous.}

Let $\spc{X}$ be a metric space and $x,y\in\spc{X}$.

\begin{enumerate}[(i)]
\item A point $z\in \spc{X}$ is called a \index{midpoint}\emph{midpoint} between $x$ and $y$
if 
\[\dist{x}{z}{}=\dist{y}{z}{}=\tfrac12\cdot\dist[{{}}]{x}{y}{}.\]
\item Assume $\eps\ge 0$.
A point $z\in \spc{X}$ is called an \index{$\eps$-midpoint}\emph{$\eps$-midpoint} between $x$ and $y$
if 
\[\dist{x}{z}{},\quad\dist{y}{z}{}\le\tfrac12\cdot\dist[{{}}]{x}{y}{}+\eps.\]
\end{enumerate}


Note that a $0$-midpoint is the same as a midpoint.


\begin{thm}{Lemma}\label{lem:mid>geod}
Let $\spc{X}$ be a complete metric space.
\begin{subthm}{lem:mid>length}
Assume that for any pair of points $x,y\in \spc{X}$  
 and any $\eps>0$
there is an $\eps$-midpoint~$z$.
Then $\spc{X}$ is a length space.
\end{subthm}

\begin{subthm}{lem:mid>geod:geod}
Assume that for any pair of points $x,y\in \spc{X}$, 
there is a midpoint~$z$.
Then $\spc{X}$ is a geodesic space.
\end{subthm}
\end{thm}

\parit{Proof.}
We first prove \ref{SHORT.lem:mid>length}.
Let $x,y\in \spc{X}$ be a pair of points.

Set $\eps_n=\frac\eps{4^n}$, $\alpha(0)=x$ and $\alpha(1)=y$.

Let $\alpha(\tfrac12)$ be an $\eps_1$-midpoint between $\alpha(0)$ and $\alpha(1)$.
Further, let $\alpha(\frac14)$ 
and $\alpha(\frac34)$ be $\eps_2$-midpoints between the pairs $(\alpha(0),\alpha(\tfrac12))$ 
and $(\alpha(\tfrac12),\alpha(1))$ respectively.
Applying the above procedure recursively,
on the $n$-th step we define $\alpha(\tfrac{k}{2^n})$,
for every odd integer $k$ such that $0<\tfrac k{2^n}<1$, 
as an $\eps_{n}$-midpoint between the already defined
$\alpha(\tfrac{k-1}{2^n})$ and $\alpha(\tfrac{k+1}{2^n})$.


In this way we define $\alpha(t)$ for $t\in W$,
where $W$ denotes the set of dyadic rationals in $[0,1]$.
Since $\spc{X}$ is complete, the map $\alpha$ can be extended continuously to $[0,1]$.
Moreover,
\[\begin{aligned}
\length\alpha&\le \dist{x}{y}{}+\sum_{n=1}^\infty 2^{n-1}\cdot\eps_n\le
\\
&\le \dist{x}{y}{}+\tfrac\eps2.
\end{aligned}
\eqlbl{eq:eps-midpoint}
\]
Since $\eps>0$ is arbitrary, we get \ref{SHORT.lem:mid>length}.

To prove \ref{SHORT.lem:mid>geod:geod}, 
one should repeat the same argument 
taking midpoints instead of $\eps_n$-midpoints.
In this case \ref{eq:eps-midpoint} holds for $\eps_n=\eps=0$.
\qeds

Since in a compact space a sequence of $\tfrac1n$-midpoints $z_n$ contains a convergent subsequence, Lemma~\ref{lem:mid>geod} immediately implies

\begin{thm}{Proposition}\label{prop:length+proper=>geodesic}
Any proper length space is geodesic.
\end{thm}

\begin{thm}{Hopf--Rinow theorem}\label{thm:Hopf-Rinow}
Any complete, locally compact length space is proper.
\end{thm}

Before reading the proof, it is instructive to solve \ref{ex:loc-compact-not-proper}.

\parit{Proof.}
Let $\spc{X}$ be a locally compact length space.
Given $x\in \spc{X}$, denote by $\rho(x)$ the supremum of all $R>0$ such that
the closed ball $\cBall[x,R]$ is compact.
Since $\spc{X}$ is locally compact, 
$$\rho(x)>0
\quad\text{for any}\quad
x\in \spc{X}.\eqlbl{eq:rho>0}$$
It is sufficient to show that $\rho(x)=\infty$ for some (and therefore any) point $x\in \spc{X}$.

\begin{clm}{} If $\rho(x)<\infty$, then $B=\cBall[x,\rho(x)]$ is compact.
\end{clm}

Indeed, $\spc{X}$ is a length space;
therefore for any $\eps>0$, 
the set $\cBall[x,\rho(x)-\eps]$ is a compact $\eps$-net in~$B$.
Since $B$ is closed and hence complete, it must be compact.
\claimqeds
Next we claim that
\begin{clm}{} $|\rho(x)-\rho(y)|\le \dist{x}{y}{\spc{X}}$ for any $x,y\in \spc{X}$;
in particular $\rho\:\spc{X}\to\RR$ is a continuous function.
\end{clm}

Indeed, 
assume the contrary; that is, $\rho(x)+|x-y|<\rho(y)$ for some $x,y\in \spc{X}$. 
Then 
$\cBall[x,\rho(x)+\eps]$ is a closed subset of $\cBall[y,\rho(y)]$ for some $\eps>0$.
Then  compactness of $\cBall[y,\rho(y)]$ implies compactness of $\cBall[x,\rho(x)+\eps]$, a contradiction.\claimqeds

Set $\eps=\min\set{\rho(y)}{y\in B}$; the minimum is defined since $B$ is compact and $\rho$ is continuous.
From \ref{eq:rho>0}, we have $\eps>0$.

Choose a finite $\tfrac\eps{10}$-net $\{a_1,a_2,\dots,a_n\}$ in $B=\cBall[x,\rho(x)]$.
The union $W$ of the closed balls $\cBall[a_i,\eps]$ is compact.
Clearly 
$\cBall[x,\rho(x)+\frac\eps{10}]\subset W$.
Therefore $\cBall[x,\rho(x)+\frac\eps{10}]$ is compact,
a contradiction.
\qeds

\begin{thm}{Exercise}\label{exercise from BH}
Construct a geodesic space $\spc{X}$ that is locally compact,
but whose completion $\bar{\spc{X}}$ is neither geodesic nor locally compact.
\end{thm}

\begin{thm}{Advanced exercise}\label{ex:gross}
Show that for any compact length-metric space $X$ there is number $\ell=\ell(X)$ such that for any finite collection of points there is a point $z$ that lies of average distance $\ell$ from the collection;
that is, for any $x_1,\dots,x_n\in X$ there is $z\in X$ such that
\[\tfrac1n\cdot\sum_i|x_i-z|_X=\ell.\]
\end{thm}







\chapter{Universal spaces}\label{chap:urysohn}

The Urysohn space is the main hero of this lecture.
It shares some fundamental properties with classical spaces (spheres, Euclidean, and Lobachevsky spaces),
but also has many counterintuitive properties.

This space often serves as a counterexample to plausible conjectures,
so it is worth to know it.
In addition, this space is beautiful.



\section{Embedding in a normed space}

Recall that a function $v\mapsto |v|$ on a vector space $\spc{V}$ is called \index{norm}\emph{norm} if it satisfies the following condition for any two vectors $v,w\in \spc{V}$ and a scalar $\alpha$:
\begin{itemize}
\item $|v|\ge 0$;
\item $|\alpha\cdot v|=|\alpha|\cdot |v|$;
\item $|v|+|w|\ge|v+w|$.
\end{itemize}

As an example, consider \index{$\ell^\infty$}$\ell^\infty$ --- the space of real sequences equipped with \index{sup-norm}\emph{sup-norm}; that is, the norm of $\bm{a}=(a_1,a_2,\dots)$ is defined by
\[|\bm{a}|_{\ell^\infty}
\df
\sup_n\{\,|a_n|\,\}.\]


It is straightforward to check that for any normed space the function $(v,w)\mapsto |v-w|$ defines a metric on it.
Therefore, any normed space is an example of metric space;
moreover, it is a geodesic space.
Often we do not distinguish normed space from the corresponding metric space.
(By the Mazur--Ulam theorem, the metric remembers the affine structure of the space; so, to recover the original normed space we only need to specify the origin.
A slick proof of this theorem was given by Jussi V\"{a}is\"{a}l\"{a} \cite{vaisala}.)

Recall that \index{diameter}\emph{diameter} of a metric space $\spc{X}$ (briefly $\diam \spc{X}$) is defined as the least upper bound on the distances between pairs of its points;
that is,
\[\diam \spc{X}
\df
\sup\set{\dist{x}{y}{\spc{X}}}{x,y\in \spc{X}}.\]
If $\diam\spc{X}<\infty$, then the space $\spc{X}$ is called \index{bounded space}\emph{bounded}.



\begin{thm}{Lemma}\label{lem:frechet}
Suppose $\spc{X}$ is a bounded \index{separable space}\emph{separable} metric space;
that is, $\spc{X}$ contains a countable, dense set, say $\{w_n\}$.
Given $x\in \spc{X}$, set $a_n(x)=\dist{w_n}{x}{\spc{X}}$.
Then 
\[\iota\:x\mapsto (a_1(x), a_2(x),\dots)\]
defines a distance-preserving embedding $\iota\:\spc{X}\hookrightarrow \ell^\infty$.
\end{thm}

\parit{Proof.} 
By the triangle inequality 
\[|a_n(x)-a_n(y)|\le \dist{x}{y}{\spc{X}}.\eqlbl{eq:a-a=<dist}\]
Therefore, $\iota$ is \index{short map}\emph{short} (in other words, $\iota$ is distance-expanding).

Again by triangle inequality we have 
\[|a_n(x)-a_n(y)|\ge \dist{x}{y}{\spc{X}}-2\cdot\dist{w_n}{x}{\spc{X}}.\]
Since the set $\{w_n\}$ is dense, we can choose $w_n$ arbitrarily close to $x$.
Whence 
\[\sup_n\{\,|a_n(x)-a_n(y)|\,\}\ge \dist{x}{y}{\spc{X}};\eqlbl{eq:a-a>=dist}\]
that is, $\iota$ is distance-noncontracting.

Finally, observe that \ref{eq:a-a=<dist} and \ref{eq:a-a>=dist} imply the lemma.
\qeds

\begin{thm}{Exercise}\label{ex:compact-length}
Show that any compact metric space $\spc{K}$ is isometric to a subspace of a compact geodesic space. 
\end{thm}

The following exercise generalizes the lemma to arbitrary separable spaces.

\begin{thm}{Exercise}\label{ex:frechet}
Suppose $\{w_n\}$ is a countable, dense set in a metric space $\spc{X}$.
Choose $x_0\in \spc{X}$;
given $x\in \spc{X}$, set 
\[a_n(x)=\dist{w_n}{x}{\spc{X}}-\dist{w_n}{x_0}{\spc{X}}.\]
Show that $\iota\:x\mapsto (a_1(x), a_2(x),\dots)$ defines a distance-preserving embedding $\iota\:\spc{X}\hookrightarrow \ell^\infty$.

Conclude that any separable metric space $\spc{X}$ admits a distance-preserving embedding $\iota\:\spc{X}\hookrightarrow \ell^\infty$.
\end{thm}

The following lemma implies that {}\textit{any metric space is isometric to a subset of a normed vector space};
its proof is nearly identical to the proof of \ref{ex:frechet}.
Given a set $\spc{X}$, denote by \index{$\ell^\infty(\spc{X})$}$\ell^\infty(\spc{X})$ the space of all bounded functions on $\spc{X}$ equipped with sup-norm; 
that is,
\[|f-g|_{\ell^\infty}=\sup\set{|f(x)-f(x)}{x\in \spc{X}}.\]

\begin{thm}{Lemma}\label{lem:kuratowski}
Let $x_0$ be a point in a metric space $\spc{X}$.
Then the map $\iota\:\spc{X}\to \ell^\infty(\spc{X})$ defined by 
\[\iota\:x\mapsto (\distfun_x-\distfun_{x_0})\]
is distance-preserving.

In particular, any metric space $\spc{X}$ admits a distance-preserving into $\ell^\infty(\spc{X})$.
\end{thm}

\section{Extension property}
\label{sec:Extension property}

If a metric space $\spc{X}$ is a subspace of a semimetric space $\spc{X}'$, then we say that $\spc{X}'$ is an \index{extension}\emph{extension} of $\spc{X}$.
If in addition, $\diam\spc{X}'\le d$, then we say that $\spc{X}'$ is a {}\emph{$d$-extension}.

If the complement $\spc{X}'\setminus \spc{X}$ contains a single point, say $p$, then $\spc{X}'$ is called a \index{one-point extension}\emph{one-point extension} of $\spc{X}$.
In this case, to define a metric on $\spc{X}'$, it is sufficient to specify the distance function from $p$; that is, a function $f\:\spc{X}\to\RR$ defined by 
\[f(x)\df\dist{p}{x}{\spc{X}'}.\]
Any function $f$ of that type will be called an \index{extension function}\emph{extension function}\label{page:extension function} or {}\emph{$d$-extension function} respectively.

The extension function $f$ cannot be taken arbitrarily --- the triangle inequality implies that 
\[f(x)+f(y)\ge \dist{x}{y}{\spc{X}}\ge |f(x)-f(y)|\]
for any $x,y\in \spc{X}$.
In particular, $f$ is a non-negative 1-Lipschitz function on $\spc{X}$.
For a $d$-extension, we need to assume in addition that $\diam\spc{X}\z\le d$ and $f(x)\le d$ for any $x\in \spc{X}$.
A straightforward check shows that these conditions are necessary and sufficient.

\begin{thm}{Exercise}\label{ex:extension-of-extension}
Let $\spc{X}$ be a subspace of metric space $\spc{Y}$.
Assume $f$ is an extension function on $\spc{X}$.

\begin{subthm}{ex:extension-of-extension:a}
Show that 
\[\bar f(y)
\df
\inf_{x\in \spc{X}} \{\,f(x)+\dist{x}{y}{\spc{Y}}\,\}\]
defines an extension function on $\spc{Y}$.
\end{subthm}

\begin{subthm}{}
Assume that $\diam \spc{Y}\le d$ and $f(x)\le d$ for any $x\in  \spc{X}$.
Show that 
\[\bar f_d
\df
\min \{\, \bar f,d\,\}\]
is a $d$-extension function on $\spc{Y}$.
\end{subthm}

\end{thm}

The functions $\bar f$ and $\bar f_d$ in the above exercise are called \index{Katětov extensions}\emph{Katětov extensions} of $f$ and the minimal possible $\spc{X}$ is called its \index{support of extension function}\emph{support}, briefly \index{$\supp$}$\supp \bar f=\spc{X}$.

\begin{thm}{Definition}\label{def:finite+1}
A metric space $\spc{U}$ meets the \index{extension property}\emph{extension property}  if for any finite subspace $\spc{F}\subset\spc{U}$ and any extension function $f\:\spc{F}\to\RR$ there is a point $p\in \spc{U}$ such that $\dist{p}{x}{}=f(x)$ for any $x\in \spc{F}$.

If we assume in addition that $\diam \spc{U}\le d$ and instead of extension functions we consider only $d$-extension functions, then we arrive at a definition of {}\emph{$d$-extension property}.

If in addition, $\spc{U}$ is separable and complete, then it is called \index{Urysohn space}\emph{Urysohn space} or {}\emph{$d$-Urysohn space} respectively.
\end{thm}


\begin{thm}{Proposition}\label{prop:univeral-separable}
There is a separable metric space with the ($d$-) extension property (for any $d\ge 0$).
\end{thm}

\parit{Proof.}
Choose $d\ge 0$.
Let us construct a separable metric space with  the $d$-extension property.

Let $\spc{X}$ be a metric space such that $\diam\spc{X}\le d$.
Denote by $\spc{X}^d$ the space of all $d$-extension functions on $\spc{X}$ equipped with the metric defined by the sup-norm.
Note that the map $\spc{X} \to \spc{X}^d$ defined by $x\mapsto\distfun_x$ is a distance-preserving embedding,
so we can (and will) treat $\spc{X}$ as a subspace of $\spc{X}^d$; equivalently, $\spc{X}^d$ is an extension of $\spc{X}$.

Let us iterate this construction.
Start with a one-point space $\spc{X}_0$ and consider a sequence of spaces $(\spc{X}_n)$ defined by $\spc{X}_{n+1}\z\df\spc{X}_n^d$.
Note that the sequence is nested;
that is, $\spc{X}_0\subset \spc{X}_1\subset\dots$
and the union
\[\spc{X}_\infty=\bigcup_n\spc{X}_n;\]
comes with metric such that
$\dist{x}{y}{\spc{X}_\infty} = \dist{x}{y}{\spc{X}_n}$
if $x,y\in\spc{X}_n$.

Note that if $\spc{X}$ is compact, then so is $\spc{X}^d$.
It follows that each space $\spc{X}_n$ is compact.
In particular, $\spc{X}_\infty$ is a countable union of compact spaces;
therefore $\spc{X}_\infty$ is separable.

Any finite subspace $\spc{F}$ of $\spc{X}_\infty$ lies in some $\spc{X}_n$ for $n<\infty$.
By construction, given an extension function $f\:\spc{F}\to\RR$,
there is a point $p\in \spc{X}_{n+1}$ that meets the condition in \ref{def:finite+1}.
That is, $\spc{X}_\infty$ has the $d$-extension property.

The construction of a separable metric space with the extension property requires only two changes.
First, the sequence should be defined by $\spc{X}_{n+1}\z\df\spc{X}_n^{d_n}$, where $d_n$ is an increasing sequence such that $d_n\to\infty$.
Second, the point $p$ should be taken in $\spc{X}_{n+k}$ for sufficiently large $k$, so that $d_{n+k}>\max\{f(x)\}$
(here one has to apply \ref{ex:extension-of-extension:a}).%

(Alternatively, one can start with any separable space $\spc{X}_0$ and consider a nested sequence $\spc{X}_0\subset \spc{X}_1\subset{}\dots$ where $\spc{X}_{n+1}$ is the space of all extension functions on $\spc{X}_{n}$ with at most $n+1$ points in its support.
The last condition is needed to keep $\spc{X}_{n}$ separable.)
\qeds

Given a metric space $\spc{X}$, denote by $\spc{X}^\infty$ the space of all extension functions on $\spc{X}$ equipped with the metric defined by the sup-norm.

\begin{thm}{Exercise}\label{ex:inf-extension}
Construct a proper length space $\spc{X}$ such that $\spc{X}^\infty$ is not separable.
\end{thm}


\begin{thm}{Proposition}\label{prop:completion-univeral}
If a metric space $\spc{V}$ meets the ($d$-) extension property, then so does its completion.
\end{thm}

\parit{Proof.} 
Let us assume $\spc{V}$ meets the extension property.
We will show that its completion $\spc{U}=\bar{\spc{V}}$ meets the extension property as well.
The $d$-extension case can be proved along the same lines.

Note that $\spc{V}$ is a dense subset in a complete space $\spc{U}$.
Observe that $\spc{U}$ has the {}\emph{approximate extension property};
that is, if $\spc{F}\z\subset\spc{U}$ is a finite set, $\eps>0$, and $f\:\spc{F}\to \RR$ is an extension function, then
there exists $p\in \spc{U}$ such that
\[\dist{p}{x}{}\lg f(x)\pm\eps\eqlbl{eq:|p-x|><f(x)}\]
for any $x\in\spc{F}$.
Indeed, consider the Katětov extension $\bar f\:\spc{U}\to\RR$ of~$f$.
Since $\spc{V}$ is dense in $\spc{U}$, we can choose a finite set $\spc{F}'\in \spc{V}$ such that for any $x\in \spc{F}$ there is $x'\in \spc{F}'$ with $\dist{x}{x'}{}<\tfrac\eps2$.
Let $p$ be the point provided by the extension property for the restriction $\bar f|_{\spc{F}'}$.
It remains to observe $p$ meets \ref{eq:|p-x|><f(x)}.

It follows that there is a sequence of points $p_n\in \spc{U}$ such that for any $x\in \spc{F}$, 
\[\dist{p_n}{x}{}\lg f(x)\pm\tfrac1{2^n}.\]

Moreover, we can assume that 
\[\dist{p_n}{p_{n+1}}{} < \tfrac1{2^n}\eqlbl{eq:|pn-pn|}\]
for all large $n$.
Indeed, consider the sets $\spc{F}_n=\spc{F}\cup\{p_n\}$ and the functions $f_n\:\spc{F}_n\to\RR$ defined by $f_n(x)\df f(x)$ and
\[f_n(p_n)
\df
\max\set{\bigl|\dist{p_n}{x}{}- f(x)\bigr|}{x\in \spc{F}}\]
 if $x\ne p_n$.
Observe that $f_n$ is an extension function for large $n$ and
$f_n(p_n)\z<\tfrac1{2^n}$.
Therefore, applying the approximate extension property recursively we get~\ref{eq:|pn-pn|}.

Therefore, the sequence $p_n$ is Cauchy.
Note that its limit meets the condition in the definition of extension property (\ref{def:finite+1}).
\qeds

Note that \ref{prop:univeral-separable} and \ref{prop:completion-univeral} imply the following:

\begin{thm}{Theorem}\label{thm:urysohn-exists}
Urysohn space and $d$-Urysohn space exist for any $d>0$.
\end{thm}

Here is a slightly stronger statement:

\begin{thm}{Theorem}\label{thm:urysohn-exists+}
Any separable metric space $\spc{X}$ admits a distance-preserving embedding into an Urysohn space $\spc{U}$ such that any isometry of $\spc{X}$ can be extended to an isometry of $\spc{U}$.
\end{thm}

\parit{Sketch of proof.}
Start with $\spc{X}_0=\spc{X}$ and construct a nested sequence of spaces $\spc{X}_0\subset\spc{X}_1 \subset{}\dots$ as at the alternative end of the proof of~\ref{prop:univeral-separable}.
Note that 
any isometry $\spc{X}_n\to \spc{X}_n$ can be extended to a unique isometry $\spc{X}_{n+1}\to \spc{X}_{n+1}$.
It follows that any isometry of $\spc{X}$ can be extended to an isometry of $\spc{X}'=\bigcup_n\spc{X}_n$.

Now, consider new nested sequence $\spc{X}\subset \spc{X}'\subset \spc{X}''\subset \dots$;
denote its union by $\spc{Y}$.
Arguing as in \ref{prop:univeral-separable} and \ref{prop:completion-univeral} we get that the completion of $\spc{Y}$ is an Urysohn space, say $\spc{U}$, that comes with a distance-preserving inclusion $\spc{X}\hookrightarrow \spc{U}$.

From above, for any isometry of $\spc{X}$ can be extended to isometries of $\spc{X}'$, $\spc{X}''$ and so on.
They all define an isometry of $\spc{Y}$;
passing to its continuous extension, we get an isometry of $\spc{U}$.
\qeds


\section{Universality}

A metric space will be called \index{universal space}\emph{universal} if it has a subspace isometric to any given separable metric space.
In \ref{ex:frechet}, we proved that $\ell^\infty$ is a universal space. 
The following proposition shows that an Urysohn space is universal as well.
Unlike $\ell^\infty$, Urysohn spaces are separable;
so it might be considered as a \textit{better} universal space.
Theorem \ref{thm:compact-homogeneous} will give another reason why Urysohn spaces are better.

\begin{thm}{Proposition}\label{prop:sep-in-urys}
An Urysohn space is universal.
That is, if $\spc{U}$ is an Urysohn space, then any separable metric space $\spc{S}$ admits a distance-preserving embedding $\spc{S}\hookrightarrow\spc{U}$.

Moreover, for any finite subspace $\spc{F}\subset \spc{S}$,
any distance-preserving embedding $\spc{F}\hookrightarrow \spc{U}$ can be extended to a distance-preserving embedding $\spc{S}\hookrightarrow\spc{U}$.

A $d$-Urysohn space is $d$-universal;
that is, the above statements hold provided that $\diam\spc{S}\le d$.  
\end{thm}

\parit{Proof.}
We will prove the second statement;
the first statement is its partial case for $\spc{F}=\emptyset$.

The required isometry will be denoted by $x\mapsto x'$.

Choose a dense sequence of points $s_1,s_2,\dotsc\in\spc{S}$.
We may assume that $\spc{F}=\{s_1,\dots,s_n\}$, so $s_i'\in \spc{U}$ are defined for $i\le n$.

The sequence $s_i'$ for $i>n$ can be defined recursively using the extension property in $\spc{U}$.
Namely, suppose that $s_1',\dots,s_{i-1}'$ are already defined.
Since $\spc{U}$ meets the extension property, there is a point $s_i'\in \spc{U}$ such that
\[\dist{s_i'}{s_j'}{\spc{U}}=\dist{s_i}{s_j}{\spc{S}}\]
for any $j<i$.

The constructed map $s_i\mapsto s_i'$ is distance-preserving.
Therefore it can be continuously extended to the whole $\spc{S}$.
It remains to observe that the constructed map $\spc{S}\hookrightarrow\spc{U}$ is distance-preserving.
\qeds

\begin{thm}{Exercise}\label{ex:geodesics-urysohn}
Show that any two distinct points in an Urysohn space can be joined by an infinite number of distinct geodesics.
\end{thm}

\begin{thm}{Exercise}\label{ex:compact-extension}
Modify the proofs of \ref{prop:completion-univeral} and \ref{prop:sep-in-urys} to prove the following theorem.
\end{thm}

\begin{thm}{Theorem}\label{thm:compact-extension}
Let $K$ be a compact set in a separable space $\spc{S}$.
Then any distance-preserving map from $K$ to an Urysohn space can be extended to 
a distance-preserving map of the whole $\spc{S}$.
\end{thm}

\begin{thm}{Exercise}\label{ex:sc-urysohn}
Show that ($d$-) Urysohn space is simply-connected.
\end{thm}



\section{Uniqueness and homogeneity}

\begin{thm}{Theorem}\label{thm:urysohn-unique}
Suppose $\spc{F}\subset \spc{U}$ and $\spc{F}'\subset \spc{U}'$ be finite isometric subspaces in a pair of ($d$-)Urysohn spaces $\spc{U}$ and $\spc{U}'$.
Then any isometry $\iota\:\spc{F}\leftrightarrow \spc{F}'$ can be extended to an isometry $\spc{U}\leftrightarrow \spc{U}'$.

In particular, ($d$-)Urysohn space is unique up to isometry.
\end{thm}

Note that \ref{prop:sep-in-urys} implies that there are distance-preserving maps $\spc{U}\z\to \spc{U}'$ and $\spc{U}'\to \spc{U}$.
The next exercise shows that it does not solely imply the existence of an isometry $\spc{U}\leftrightarrow \spc{U}'$.

\begin{thm}{Exercise}\label{ex:no-isom}
Construct two metric spaces $\spc{X}$ and $\spc{Y}$ such that 
there are distance-preserving maps $\spc{X}\to \spc{Y}$ and $\spc{Y}\to \spc{X}$, but no isometry $\spc{X}\leftrightarrow \spc{Y}$.
\end{thm}


The following construction uses the idea of \ref{prop:sep-in-urys}, but it is applied \index{back-and-forth}\emph{back-and-forth} to ensure that the obtained distance-preserving map is onto.

\parit{Proof.}
Choose dense sequences $a_1,a_2,\dots{}\in \spc{U}$ and $b'_1,b'_2,\dots{}\in \spc{U}'$.
We can assume that $\spc{F}=\{a_1,\dots,a_n\}$, $\spc{F}'=\{b_1',\dots,b_n'\}$ and $\iota(a_i)=b_i'$ for $i\le n$.

The required isometry $\spc{U}\leftrightarrow \spc{U}'$ will be denoted by $u \leftrightarrow u'$.
Set $a_i=b_i$ and $a'_i=b'_i$ if $i\le n$.

Let us define recursively $a_{n+1}',b_{n+1}, a_{n+2}', b_{n+2},\dots$ --- on the odd step we define the images of $a_{n+1},a_{n+2},\dots$ and on the even steps we define inverse images of $b'_{n+1},b'_{n+2},\dots$
The same argument as in the proof of \ref{prop:sep-in-urys} shows that we can construct two sequences $a_1',a_2',\dots{}\in \spc{U}'$ and $b_1,b_2,\dots\in \spc{U}$ such that
\begin{align*}
\dist{a_i}{a_j}{\spc{U}}&=\dist{a_i'}{a_j'}{\spc{U}'},
\\
\dist{a_i}{b_j}{\spc{U}}&=\dist{a_i'}{b_j'}{\spc{U}'},
\\
\dist{b_i}{b_j}{\spc{U}}&=\dist{b_i'}{b_j'}{\spc{U}'}
\end{align*}
for all $i$ and $j$.

It remains to observe that the constructed distance-preserving bijection defined by $a_i\leftrightarrow a_i'$ and $b_i\leftrightarrow b_i'$ extends
continuously to an isometry $\spc{U}\leftrightarrow \spc{U}'$. 
\qeds

Observe that \ref{thm:urysohn-unique} implies that the Urysohn space (as well as the $d$-Urysohn space) is \index{homogeneous}\emph{finite-set-homogeneous}; that is,
\begin{itemize}
 \item any distance-preserving map from a finite subset to the whole space can be extended to an isometry.
\end{itemize}

Recall that $S(p,r)_{\spc{X}}$ denotes the sphere of radius $r$ centered at $p$ in a metric space $\spc{X}$;
that is, 
$$S(p,r)_{\spc{X}}=\set{x\in \spc{X}}{\dist{p}{x}{\spc{X}}=r}.$$

\begin{thm}{Exercise}\label{ex:sphere-in-urysohn}
Choose $d\in [0,\infty]$.
Denote by $\spc{U}_d$ the $d$-Urysohn space,
so $\spc{U}_\infty$ is the Urysohn space.

\begin{subthm}{ex:sphere-in-urysohn:sphere}
Assume that $L=S(p,r)_{\spc{U}_d}\ne \emptyset$.
Show that $L$ is isometric to $\spc{U}_{\ell}$; find $\ell$ in terms of $r$ and $d$.
\end{subthm}

\begin{subthm}{ex:sphere-in-urysohn:midpoint}
Let $\ell=\dist{p}{q}{\spc{U}_d}$.
Show that the subset $M\subset\spc{U}_d$ of midpoints between $p$ and $q$ is isometric to $\spc{U}_\ell$.
\end{subthm}

\begin{subthm}{ex:sphere-in-urysohn:homogeneous}
Show that $\spc{U}_d$ is \emph{not} countable-set-homogeneous;
that is, there is a distance-preserving map from a countable subset of $\spc{U}_d$ to $\spc{U}_d$ that cannot be extended to an isometry of $\spc{U}_d$.
\end{subthm}

\end{thm}

In fact, the Urysohn space is compact-set-homogeneous; more precisely the following theorem holds.

\begin{thm}{Theorem}\label{thm:compact-homogeneous}
Let $K$ be a compact set in a ($d$-)Urysohn space~$\spc{U}$.
Then any distance-preserving map $K\to \spc{U}$ can be extended to an isometry of $\spc{U}$.
\end{thm}

A proof can be obtained by modifying the proofs of \ref{prop:completion-univeral} and \ref{thm:urysohn-unique}
the same way as it is done in \ref{ex:compact-extension}.

\begin{thm}{Exercise}\label{ex:shere}
Let $S$ be a unit sphere in the Urysohn space $\spc{U}$.
Show that for any two distinct points $x,y\in \spc{U}$ there is a point $z\in S$ such that 
$\dist{x}{z}{}\ne \dist{y}{z}{}$.

Conclude that two isometries of $\spc{U}$ coincide if they coincide on $S$.
\end{thm}

\begin{thm}{Exercise}\label{ex:ext(shere)}
Let $B$ be an open unit ball in the Urysohn space $\spc{U}$.
Show that $\spc{U}\setminus B$ is isometric to $\spc{U}$.

Use it to construct an isometry of a unit sphere $S$ in $\spc{U}$ that cannot be extended to an isometry of $\spc{U}$.
\end{thm}

\begin{thm}{Exercise}\label{ex:katetov}

\begin{subthm}{ex:katetov:inclusion}
Show that there is a distance-preserving inclusion of the Urysohn space $\iota\:\spc{U}\hookrightarrow \spc{U}$ 
such that $\spc{U}'=\iota(\spc{U})$ is nowhere dense in $\spc{U}$ and any isometry of $\spc{U}'$ 
can be extended to an isometry of the whole~$\spc{U}$.
\end{subthm}

\begin{subthm}{ex:katetov:sol}
Consider a nested sequence $\spc{U}_0\subset \spc{U}_1\subset\dots$ of Urysohn spaces 
with each inclusion $\spc{U}_n\hookrightarrow \spc{U}_{n+1}$ as in \ref{SHORT.ex:katetov:inclusion}.
Show that the union $\bigcup_n\spc{U}_n$ is a noncomplete finite-set-homogeneous metric space that meets the extension property.
\end{subthm}

\end{thm}

{\sloppy

\begin{thm}{Exercise}\label{ex:homogeneous}
Which of the following metric spaces are 
one-point-homogeneous, finite-set-homogeneous, compact-set-homogeneous, countable-set-homogeneous?

\begin{subthm}{ex:homogeneous:euclidean}
Euclidean plane,
\end{subthm}

\begin{subthm}{ex:homogeneous:hilbert}
 Hilbert space $\ell^2$,
\end{subthm}

\begin{subthm}{ex:homogeneous:ell-infty}
 $\ell^\infty$,
\end{subthm}

\begin{subthm}{ex:homogeneous:ell-1}
 \index{$\ell^1$}$\ell^1$ --- the space of all real absolutely converging series $\bm{a}\z=(a_1,a_2,\dots)$ with the norm $|\bm{a}|_{\ell^1}=\sum_i|a_i|$.
 
\end{subthm}
\end{thm}

}

\begin{thm}{Exercise}\label{ex:homogeneous-tree}
Show that any separable one-point-homogeneous metric tree is isometric to the real line $\RR$ or the one-point space.
\end{thm}


\section{Remarks}

The statement in \ref{ex:frechet} was proved by Maurice René Fréchet in the paper where he first defined metric spaces \cite{frechet};
its extension \ref{lem:kuratowski} was given by Kazimierz Kuratowski~\cite{kuratowski}.

The following two exercises show that in this respect $\ell^\infty$ is very different from $\ell^1$.
For more on the subject, see \cite{deza-laurent}.

Let $S$ be a subset of $X$.
The \index{cut metric}\emph{cut metric} $\delta_S$ on $X$ is a semimetric such that $\delta_S(x, y) = 1$ if $x$ and $y$ are separated
by $S$ and otherwise $\delta_S(x, y) = 0$.


\begin{thm}{Exercise}\label{ex:cut}
Show that a finite metric space $\spc{F}$ admits a distance-preserving embedding into $\ell^1$ if and only if the metric of $\spc{F}$ can be written as a nonnegative linear combination%
\footnote{that is, linear combination with nonnegative coefficients.} of cut metrics on $\spc{F}$.
\end{thm}

Recall that the vertex set of any graph $\Gamma$ comes with the shortest-path distance ---
the distance between two vertices is the minimal number of edges in a path connecting them.

\begin{wrapfigure}{r}{15mm}
\vskip-8mm
\centering
\includegraphics{mppics/pic-210}
\end{wrapfigure}

\begin{thm}{Exercise}\label{ex:K23}
Use \ref{ex:cut} to show that the metric for complete biparted graph $K_{2,3}$ (see the diagram) does not admit a distance-preserving embedding into $\ell^1$.
\end{thm}

The question about existence of a separable universal space was posed by Maurice René Fréchet and answered by
Pavel Urysohn~\cite{urysohn}.
Exercise \ref{ex:katetov} answers a question posed by Pavel Urysohn \cite[§$2(6)$]{urysohn}.
It was solved by Miroslav Katětov \cite{katetov},
but long after that, it was again mentioned as an open problem \cite[p. 83]{gromov-2007}.

The idea of Urysohn's construction was reused in graph theory; it produces the so-called \index{Rado graph}\emph{Rado graph},
also known as {}\emph{Erd\H{o}s--Rényi graph} or \emph{random graph}; see \cite{cameron}.
In fact, the Urysohn space is the random metric space in \textit{certain sense} \cite{vershik}.

\textit{The ($d$-) Urysohn space is homeomorphic to the Hilbert space};
the latter was proved by Vladimir Uspenskij \cite{uspenskij} using the so-called Toruńczyk criterion.

The finite-set-homogeneous spaces include Euclidean spaces, hyperbolic spaces, and spheres all with standard length metrics and arbitrary finite dimensions.
In fact, these are the only examples of locally compact three-point-homogeneous length spaces.
The latter was proved by Herbert Busemann \cite{busemann-1942}; it also follows from the more general result of Jacques Tits about two-point-homogeneous spaces \cite{tits}.
The same conclusion holds for complete all-set-homogeneous geodesic spaces with local uniqueness of geodesics;
it was proved by Garrett Birkhoff \cite{birkhoff}.
The answer might be the same for complete separable all-set-homogeneous length spaces.
Without the separability condition, we also get the so-called \emph{universal metric trees} with finite valence \cite{dyubina-polterovich}; no other examples seem to be known \cite{lebedeva-petrunin2211.09671}.  

{\sloppy

\begin{thm}{Exercise}\label{ex:RP-not}
Show that the real projective plane $\RP^2$ with the standard metric is two-point-homogeneous, but not three-point-homogeneous.
\end{thm}

}

\begin{thm}{Exercise}\label{ex:hom-cube}
Let $Q$ be the set of vertices on the $n$-dimensional cube;
assume $n$ is large.
Show that $Q$ is three-point-homogeneous, but not four-point-homogeneous.
\end{thm}

I do not know examples of metric spaces that are $n$-point-homogeneous, but not $(n+1)$-point-homogeneous for large $n$ \cite{petrunin-431426}.

\chapter{Injective spaces}\label{chap:injective}


Injective hull is a useful construction that provides a canonical choice of a specially nice (injective) space that includes a given metric space. 
This construction is similar to the convex hull in Euclidean space.
The following exercise gives a bridge from the latter to the former.

\begin{thm}{Advanced exercise}\label{ex:conv-short}
Show that $A\subset \RR^n$ is a closed convex set if and only if for any  $B\subset \RR^n$ any short map $B\to A$ can be extended to a short map $\RR^n\to A$.
\end{thm}

\section{Definition}

\begin{thm}{Definition}\label{def:injective}
A metric space $\spc{Y}$ is called \index{injective space}\emph{injective} if for any metric space $\spc{X}$ and any of its subspace $\spc{A}$,
any short map $f\:\spc{A}\to \spc{Y}$ can be extended to a short map $F\:\spc{X}\to \spc{Y}$;
that is, $f=F|_{\spc{A}}$.
\end{thm}

\begin{thm}{Exercise}\label{ex:inj=complete-geodesic-contractible}
Show that any injective space is 
\begin{multicols}{3}

\begin{subthm}{ex:inj=complete-geodesic-contractible:complete}
complete,
\end{subthm}

\begin{subthm}{ex:inj=complete-geodesic-contractible:geodesic}
geodesic, and
\end{subthm}

\begin{subthm}{ex:inj=complete-geodesic-contractible:contractible}
contractible.
\end{subthm}

\end{multicols}

\end{thm}

\begin{thm}{Exercise}\label{ex:bicombing}
Show that for any injective space $\spc{Y}$ there is a map $m\:\spc{Y}\times\spc{Y}\to\spc{Y}$ (the \index{midpoint map}\emph{midpoint map}) such that the inequality
\[2\cdot \dist{p}{m(x,y)}{\spc{Y}}\le\dist{p}{x}{\spc{Y}}+\dist{p}{y}{\spc{Y}}\]
holds for any $p,x,y\in \spc{Y}$.
\end{thm}

\begin{thm}{Exercise}\label{ex:injective-spaces}
Show that the following spaces are injective:
\begin{subthm}{ex:injective-spaces:R}
the real line;
\end{subthm}

\begin{subthm}{ex:injective-spaces:tree}
complete metric tree;
\end{subthm}

\begin{subthm}{ex:injective-spaces:ell-infty}
The space $\ell^\infty(\spc{S})$ for any set $\spc{S}$ (defined in \ref{lem:kuratowski}).
In particular, the coordinate plane with the metric induced by the $\ell^\infty$-norm.
\end{subthm}

\end{thm}

\begin{thm}{Exercise}\label{ex:extr-ball}
Let $\spc{Y}$ be an injective space.

\begin{subthm}{ex:extr-ball:one}
Show that any closed ball in $\spc{Y}$ is injective.
\end{subthm}

\begin{subthm}{ex:extr-ball:many}
Show that the intersection of an arbitrary collection of closed balls in $\spc{Y}$ is injective.
\end{subthm}

\end{thm}

\begin{thm}{Advanced exercise}\label{ex:extr-fixed}
Let $\spc{Y}$ be a bounded injective space.
Show that any short map $s\:\spc{Y}\to\spc{Y}$ has a fixed point. 
\end{thm}


\section{Admissible and extremal functions}

Let $\spc{X}$ be a metric space.
A function $r\:\spc{X}\to(-\infty,\infty]$ is called \label{page:admissible function}\index{admissible function}\emph{admissible} if the following inequality
\[r(x)+r(y)\ge \dist{x}{y}{\spc{X}}\eqlbl{eq:admissible}\]
holds for any $x,y\in \spc{X}$.

\begin{thm}{Observation}\label{obs:admissible}

\begin{subthm}{obs:admissible:nonnegative}
Any admissible function is nonnegative.
\end{subthm}

\begin{subthm}{obs:admissible:balls}
If $\spc{X}$ is a geodesic space, then a function $r\:\spc{X}\to\RR$ is admissible if and only if 
\[\cBall[x,r(x)]\cap\cBall[y,r(y)]\ne \emptyset\]
for any $x,y\in \spc{X}$.
\end{subthm}
 
\end{thm}

\parit{Proof; \ref{SHORT.obs:admissible:nonnegative}.} Apply \ref{eq:admissible} for $x=y$.

\parit{\ref{SHORT.obs:admissible:balls}.} Apply the triangle inequality and the existence of a geodesic $[xy]$.
\qeds

A minimal admissible function will be called \label{page:extremal function}\index{extremal function}\emph{extremal}.
More precisely, an admissible function $r\:\spc{X}\to\RR$ is extremal 
if for any admissible function $s\:\spc{X}\to\RR$ we have
\[s\le r\quad\Longrightarrow\quad s=r.\]

Applying Zorn's lemma, we get the following.

\begin{thm}{Observation}\label{obs:extremal:below}
For any admissible function $s$ there is an extremal function $r$ such that $r\le s$.
\end{thm}

\begin{thm}{Lemma}\label{lem:+-c}
Let $r$ be an extremal function and $s$ an admissible function on a metric space $\spc{X}$.
Suppose that $r\ge s-c$ for some constant~$c$.
Then $r\le s+c$; in particular, $c\ge 0$.
\end{thm}

\parit{Proof.}
Note that if $c<0$, then $r>s$.
The latter is impossible since $r$ is extremal and $s$ is admissible.

Observe that the function $\bar r=\min\{\,r,s+c\,\}$ is admissible.
Indeed, choose $x,y\in \spc{X}$.
If $\bar r(x)=r(x)$ and $\bar r(y)=r(y)$, then 
\[\bar r(x)+\bar r(y)=r(x)+ r(y)\ge \dist{x}{y}{}.\]
Further, if $\bar r(x)=s(x)+c$, then 
\begin{align*}
\bar r(x)+\bar r(y)&\ge [s(x)+c]+ [s(y)-c]= 
\\
&=s(x)+s(y) \ge 
\\
&\ge\dist{x}{y}{}.
\end{align*}

Since $r$ is extremal, we have $r=\bar r$;
that is, $r\le s+c$.
\qeds

\begin{thm}{Observations}\label{obs:extremal}
Let $\spc{X}$ be a metric space.

\begin{subthm}{obs:extremal:distfun}
For any point $p\in\spc{X}$ the distance function $r\z=\distfun_p$ is extremal.
\end{subthm}

\begin{subthm}{lem:extremal-lipschitz}
Any extremal function $r$ on $\spc{X}$ is \index{1-Lipschitz function}\emph{1-Lipschitz};
that is,
\[|r(p)-r(q)|\le \dist{p}{q}{}\]
for any $p,q\in\spc{X}$.
In other words, any extremal function is an extension function [see \ref{sec:Extension property}].
\end{subthm}

\begin{subthm}{lem:opposite}
An admissible function $r$ on $\spc{X}$ is extremal if and only if
for any point $p\in\spc{X}$ and any $\delta>0$, there is a point $q\in \spc{X}$
such that 
\[r(p)+r(q)<\dist{p}{q}{\spc{X}}+\delta.\]
\end{subthm}

\begin{subthm}{lem:opposite-compact}
Suppose $\spc{X}$ is compact.
Then an admissible function $r$ on $\spc{X}$ is extremal if and only if
for any point $p\in\spc{X}$ there is a point $q\in \spc{X}$
such that 
\[r(p)+r(q)=\dist{p}{q}{\spc{X}}.\]
\end{subthm}

\end{thm}

\parit{Proof; \ref{SHORT.obs:extremal:distfun}.}
By the triangle inequality, \ref{eq:admissible} holds;
that is, $r=\distfun_p$ is an admissible function.

Further, if $s\le r$ is another admissible function, then $s(p)=0$ and \ref{eq:admissible} implies that $s(x)\z\ge\dist{p}{x}{}$.
Whence $s=r$.

\parit{\ref{SHORT.lem:extremal-lipschitz}.}
By \ref{SHORT.obs:extremal:distfun}, $\distfun_p$ is admissible.
Since $r$ is admissible, we have that
\[r\ge \distfun_p-r(p).\]
Since $r$ is extremal, \ref{lem:+-c} implies that
\[r\le \distfun_p+r(p),\]
or, equivalently,
\[r(q)-r(p)\le \dist{p}{q}{}\]
for any $p,q\in\spc{X}$.
Whence the statement follows.

\parit{\ref{SHORT.lem:opposite}.}
Assume $r$ is extremal.
Arguing by contradiction, assume there is $\delta>0$ such that
\[r(q)\ge \distfun_p(q)-r(p)+\delta\]
for any $q$.
By \ref{SHORT.obs:extremal:distfun}, $\distfun_p$ is extremal; in particular, admissible.
Therefore \ref{lem:+-c} implies that
\[r(q)\le \distfun_p(q)+r(p)-\delta\]
for any $q$.
Taking $q=p$, we get $r(p)\le r(p)-\delta$, a contradiction.

Now suppose $r$ is not extremal; that is, there is an admissible function $s\le r$ such that $r(p)-s(p)=\delta>0$ for some $p$.
Then, for any $q$, we have
\[r(p)+r(q)\ge s(p)+s(q)+\delta\ge \dist{p}{q}{\spc{X}}+\delta\]
--- a contradiction.

\parit{\ref{SHORT.lem:opposite-compact}.}
The if part follows from \ref{SHORT.lem:opposite}.

Denote by $q_n$ the point provided by \ref{SHORT.lem:opposite} for $\delta=\tfrac1n$.
Let $q$ be a partial limit of $q_n$. 
Then 
\[r(p)+r(q)\le\dist{p}{q}{\spc{X}}.\]
Since $r$ is admissible, the opposite inequality holds;
whence the only-if part follows.
\qeds

\begin{thm}{Exercise}\label{ex:circle}
Consider the unit circle 
\[\mathbb{S}^1=\set{(x,y)}{x^2+y^2=1}\]
in the plane with induced length metric.
Show that $r\:\mathbb{S}^1\to\RR$ is extremal if and only if it is 1-Lipschitz and 
\[r(p)+r(-p)=\pi\] for any $p\in\mathbb{S}^1$.
\end{thm}

\begin{thm}{Exercise}\label{ex:retraction}
Given a real-valued function $s$ on a metric space $\spc{X}$,
consider the function
\[s^*(x)=\sup\set{\dist{x}{y}{\spc{X}}-s(y)}{y\in \spc{X}}\]
Show that the function $\tfrac12\cdot(s+s^*)$ is admissible for any $s$.
\end{thm}

\section{Equivalent conditions}

\begin{thm}{Theorem}\label{thm:injective=hyperconvex}
For any metric space $\spc{Y}$ the following conditions are equivalent:

\begin{subthm}{thm:injective=hyperconvex:injective}
$\spc{Y}$ is injective
\end{subthm}


\begin{subthm}{thm:injective=hyperconvex:extremal}
If $r\:\spc{Y}\to\RR$ is an extremal function, then there is a point $p\in \spc{Y}$ such that 
\[\dist{p}{x}{}= r(x)\]
for any $x\in \spc{Y}$.
\end{subthm}

\begin{subthm}{thm:injective=hyperconvex:balls}
$\spc{Y}$ is \index{hyperconvex space}\emph{hyperconvex};
that is, if $\set{\cBall[x_\alpha,r_\alpha]}{\alpha\in\IndexSet}$ is a family of closed balls in $\spc{Y}$ such that 
 \[r_\alpha+r_\beta\ge \dist{x_\alpha}{x_\beta}{}\]
 for any $\alpha,\beta\in \IndexSet$, then all the balls in the family $\{\cBall[x_\alpha,r_\alpha]\}_{\alpha\in\IndexSet}$ have a common point.
\end{subthm}

\end{thm}

\parit{Proof.} We will prove implications 
\ref{SHORT.thm:injective=hyperconvex:injective}$\Rightarrow$\ref{SHORT.thm:injective=hyperconvex:extremal}$\Rightarrow$\ref{SHORT.thm:injective=hyperconvex:balls}$\Rightarrow$\ref{SHORT.thm:injective=hyperconvex:injective}.

\parit{\ref{SHORT.thm:injective=hyperconvex:injective}$\Rightarrow$\ref{SHORT.thm:injective=hyperconvex:extremal}.}
By \ref{lem:extremal-lipschitz}, $r$ is an extension function.
Applying the definition of injective space to a one-point extension of $\spc{Y}$, we get a point $p\in \spc{Y}$ such that 
\[\dist{p}{x}{}=\distfun_p(x)\le r(x)\]
for any $x\in \spc{Y}$.
By \ref{obs:extremal:distfun}, the distance function $\distfun_p$ is extremal.
Since  $r$ is extremal, we get $\distfun_p= r$.


\parit{\ref{SHORT.thm:injective=hyperconvex:extremal}$\Rightarrow$\ref{SHORT.thm:injective=hyperconvex:balls}.}
By \ref{obs:admissible:balls}, part \ref{SHORT.thm:injective=hyperconvex:balls} is equivalent to the following statement:
\begin{itemize}
 \item If $r\:\spc{Y}\to\RR$ is an admissible function, then there is a point $p\in \spc{Y}$ such that 
\[\dist{p}{x}{}\le r(x)\eqlbl{eq:|p-x|=<r(x)}\]
for any $x\in \spc{Y}$.
\end{itemize}
Indeed, set $r(x)\df\inf\set{r_\alpha}{x_\alpha=x}$.
(If $x_\alpha\ne x$ for any $\alpha$, then $r(x)=\infty$.)
The condition in \ref{SHORT.thm:injective=hyperconvex:balls} implies that $r$ is admissible.
It remains to observe that $p\in \cBall[x_\alpha,r_\alpha]$ for every $\alpha$ if and only if \ref{eq:|p-x|=<r(x)} holds.

By \ref{obs:extremal:below}, for any admissible function $r$ there is an extremal function $\bar r\le r$;
hence \ref{SHORT.thm:injective=hyperconvex:extremal}$\Rightarrow$\ref{SHORT.thm:injective=hyperconvex:balls}.

\parit{\ref{SHORT.thm:injective=hyperconvex:balls}$\Rightarrow$\ref{SHORT.thm:injective=hyperconvex:injective}.}
Arguing by contradiction, suppose $\spc{Y}$ is not injective;
that is, there is a metric space $\spc{X}$ with a subset $\spc{A}$
such that a short map $f\:\spc{A}\to \spc{Y}$ cannot be extended to a short map $F\:\spc{X}\to \spc{Y}$.
By Zorn's lemma, we may assume that $\spc{A}$ is a maximal subset; that is, the domain of $f$ cannot be enlarged by a single point.%
\footnote{In this case, $\spc{A}$ must be closed, but we will not use it.}

Fix a point $p$ in the complement $\spc{X}\setminus \spc{A}$.
To extend $f$ to $p$, we need to choose $f(p)$ in the intersection of the balls 
$\cBall[f(x),r(x)]$, where $r(x)=\dist{p}{x}{}$.
Therefore, this intersection for all $x\in \spc{A}$ has to be empty.

Since $f$ is short, we have that 
\begin{align*}
r(x)+r(y)&\ge \dist{x}{y}{\spc{X}}\ge
\\
&\ge \dist{f(x)}{f(y)}{\spc{Y}}.
\end{align*}
By \ref{SHORT.thm:injective=hyperconvex:balls} the balls 
$\cBall[f(x),r(x)]$ have a common point --- a contradiction. 
\qeds

\begin{thm}{Exercise}\label{ex:one-point-gluing}
Suppose a length space $\spc{W}$ has two subspaces $\spc{X}$ and $\spc{Y}$ such that $\spc{X}\cup\spc{Y}=\spc{W}$ and $\spc{X}\cap\spc{Y}$ is a one-point set.
Assume $\spc{X}$ and $\spc{Y}$ are injective.
Show that  $\spc{W}$ is injective
\end{thm}

\begin{thm}{Exercise}\label{ex:Rm-ell-infty}
Show that an $m$-dimensional normed space is injective if and only if it is isometric to $\RR^m$ with $\ell^\infty$-norm; that is,
\[|(x_1,\dots,x_m)|=\max_i\{\,|x_i|\,\}.\]
\end{thm}

A metric space $\spc{Y}$ is called \index{finitely hyperconvex}\emph{finitely hyperconvex} or \index{countably hyperconvex}\emph{countably hyperconvex} if the condition in \ref{thm:injective=hyperconvex:balls} holds only for any finite or respectively countable family of balls.

\begin{thm}{Exercise}\label{ex:compact-hyperconvex}
Show that any proper finitely hyperconvex metric space is hyperconvex.
\end{thm}


\begin{thm}{Exercise}\label{ex:urysohn-hyperconvex}
Show that the $d$-Urysohn space is finitely hyperconvex, but not countably hyperconvex.
Conclude that the $d$-Urysohn space is not injective.

Try to do the same for the Urysohn space.
\end{thm}

\begin{thm}{Exercise}\label{ex:almost-hyperconvex}
Let $\spc{Y}$ be a complete metric space.
Suppose $\spc{Y}$ is \index{almost hyperconvex}\emph{almost hyperconvex},
that is, for any $\eps>0$ any family of closed balls $\set{\cBall[x_\alpha,r_\alpha+\eps]}{\alpha\in\IndexSet}$ has a common point if 
$r_\alpha+r_\beta\ge \dist{x_\alpha}{x_\beta}{}$ for all $\alpha,\beta\in \IndexSet$.
Show that $\spc{Y}$ is hyperconvex.
\end{thm}


\section{Space of extremal functions}
\label{sec:extremal-functions}

Let $\spc{X}$ be a metric space.
Consider the space $\Inj \spc{X}$ of extremal functions on $\spc{X}$ equipped with sup-norm; \label{page:InjX}
that is,
\[\dist{f}{g}{\Inj \spc{X}}\df\sup\set{|f(x)-g(x)|}{x\in \spc{X}}.\]

Recall that by \ref{obs:extremal:distfun}, any distance function is extremal.
It follows that the map $x\mapsto \distfun_x$ produces a distance-preserving embedding $\spc{X}\hookrightarrow\Inj \spc{X}$.
So we can (and will) treat $\spc{X}$ as a subspace of $\Inj \spc{X}$,
or, equivalently, $\Inj \spc{X}$ as an extension of $\spc{X}$.
In particular, from now on, a point $x\in\spc{X}$ can refer to the function $\distfun_x\:\spc{X}\to\RR$ and the other way around.

Since any extremal function is 1-Lipschitz, for any $f\in \Inj \spc{X}$ and $p\in \spc{X}$, we have that
$f(x)\le f(p)+\distfun_p(x)$.
By \ref{lem:+-c}, we also get $f(x)\ge -f(p)+\distfun_p(x)$.
Therefore
\[
\begin{aligned}
\dist{f}{p}{\Inj \spc{X}}&=\sup\set{|f(x)-\distfun_p(x)|}{x\in \spc{X}}=
\\
&=f(p).
\end{aligned}
\eqlbl{eq:f(p)=|f-p|}
\]
In particular, the statement in \ref{lem:opposite} can be written as 
\[\dist{f}{p}{\Inj\spc{X}}+\dist{f}{q}{\Inj\spc{X}}<\dist{p}{q}{\Inj\spc{X}}+\delta.\]

\begin{thm}{Exercise}\label{ex:Inj(compact)}
Show that $\Inj\spc{X}$ is compact if and only if so is $\spc{X}$.
\end{thm}

\begin{thm}{Exercise}\label{ex:tripod+square}
Describe the set of all extremal functions on a metric space $\spc{X}$ and the metric space $\Inj \spc{X}$ in each of the following cases:

\begin{subthm}{ex:tripod+square:2}
$\spc{X}$ is a metric space with exactly two points $v,w$ on distance 1 from each other.
\end{subthm}


\begin{subthm}{ex:tripod+square:tripod} 
$\spc{X}$ is a metric space with exactly three points $a,b,c$ such that 
\[\dist{a}{b}{\spc{X}}=\dist{b}{c}{\spc{X}}=\dist{c}{a}{\spc{X}}=1.\]
\end{subthm}

\begin{subthm}{ex:tripod+square:square}
$\spc{X}$ is  a metric space with exactly four points $p,q,x,y$ such that 
\[\dist{p}{x}{\spc{X}}=\dist{p}{y}{\spc{X}}=\dist{q}{x}{\spc{X}}=\dist{q}{y}{\spc{X}}=1\]
and
\[\dist{p}{q}{\spc{X}}=\dist{x}{y}{\spc{X}}=2.\]
\end{subthm}

\end{thm}

\begin{thm}{Exercise}\label{ex:kur-inj}
Assume $\spc{X}$ is a compact metric space.
Recall that the map $x\mapsto \distfun_x$ gives an isometric embedding $\spc{X}\hookrightarrow\ell^\infty(\spc{X})$; so we can think that $\spc{X}$ is a subset of $\ell^\infty(\spc{X})$.

Given two points $x,y\in \spc{X}$, denote by $G_{x,y}$ the union of all geodesics from $x$ to $y$ in $\ell^\infty(\spc{X})$.
Show that $\Inj\spc{X}$ is isometric to
\[G=\bigcap_{x\in \spc{X}}\left(\bigcup_{y\in \spc{X}}G_{x,y}\right).\]

\end{thm}


\begin{thm}{Proposition}\label{prop:InjX-is-injective}
$\Inj\spc{X}$ is injective for any metric space $\spc{X}$. 
\end{thm}

\begin{thm}{Lemma}\label{lem:r|X-extremal}
Let $\spc{X}$ be a metric space.
Then 
\[\sigma\in \Inj(\Inj \spc{X})
\quad\Longrightarrow\quad
\sigma|_\spc{X}\in \Inj \spc{X}.\]
\end{thm}

In other words, if $\sigma$ is an extremal function on $\Inj \spc{X}$,
then the restriction of $\sigma$ to $\spc{X}$ is an extremal function on $\spc{X}$.

\parit{Proof.}
Arguing by contradiction, suppose that there is an admissible function $s\:\spc{X}\to \RR$ such that $s(x)\le \sigma(x)$ for any $x\in\spc{X}$ and $s(p)\z< \sigma(p)$ for some point $p\in\spc{X}$.
Consider another function $\bar \sigma\:\Inj \spc{X}\to\RR$ such that $\bar \sigma(f)\df \sigma(f)$ if $f\ne p$ and $\bar \sigma(p)\df s(p)$.

Let us show that $\bar \sigma$ is admissible; that is, 
\[\dist{f}{g}{\Inj \spc{X}}\le\bar \sigma(f)+\bar \sigma(g)
\eqlbl{r-admissible}\]
for any $f,g\in \Inj \spc{X}$.

Since $\sigma$ is admissible and $\bar \sigma= \sigma$ on $(\Inj \spc{X})\setminus \{p\}$, it is sufficient to prove \ref{r-admissible} assuming $f\ne g=p$.
By \ref{eq:f(p)=|f-p|}, we have $\dist{f}{p}{\Inj \spc{X}}=f(p)$.
Therefore, \ref{r-admissible} boils down to the following inequality
\[\sigma(f)+s(p)\ge f(p).\eqlbl{eq:r(f)+s(p)>=f(p)}\]
for any $f\in\Inj \spc{X}$.

Fix small $\delta>0$. 
Let $q\in\spc{X}$ be the point provided by \ref{lem:opposite}.
Then
\begin{align*}
\sigma(f)+s(p)&\ge [\sigma(f)-\sigma(q)]+[\sigma(q)+s(p)]\ge
\intertext{since $\sigma$ is 1-Lipschitz, and $\sigma(q)\ge s(q)$, we can continue}
&\ge -\dist{q}{f}{\Inj \spc{X}}+[s(q)+s(p)]\ge
\intertext{by \ref{eq:f(p)=|f-p|} and since $s$ is admissible}
&\ge -f(q)+\dist{p}{q}{}>
\intertext{and by \ref{lem:opposite}}
&> f(p)-\delta.
\end{align*}
Since $\delta>0$ is arbitrary, \ref{eq:r(f)+s(p)>=f(p)} and \ref{r-admissible} follow.

Summarizing: the function $\bar \sigma$ is admissible, $\bar \sigma\le \sigma$ and $\bar \sigma(p)<\sigma(p)$;
that is, $\sigma$ is not extremal --- a contradiction.
\qeds

\parit{Proof of \ref{prop:InjX-is-injective}.}
Choose a function $\sigma\in\Inj(\Inj\spc{X})$.
By \ref{lem:r|X-extremal}, $s\z\df \sigma|_{\spc{X}}\in \Inj\spc{X}$;
that is, $s$ is extremal.
By \ref{thm:injective=hyperconvex:extremal},
it is sufficient to show that  
\[\sigma(f)\ge\dist{s}{f}{\Inj\spc{X}}
\eqlbl{eq:r(f)>=| r-f|}\]
for any $f\in\Inj\spc{X}$.

Since $\sigma$ is $1$-Lipschitz (\ref{lem:extremal-lipschitz}) we have that
\[
s(x)-f(x)=\sigma(x)-\dist{f}{x}{\Inj \spc{X}}\le \sigma(f).
\]
for any $x\in\spc{X}$.
By \ref{lem:+-c},
$
s(x)-f(x)\ge -\sigma(f)
$
for any $x\in\spc{X}$.
Whence \ref{eq:r(f)>=| r-f|} follows.
\qeds

\begin{thm}{Exercise}\label{ex:4-on-a-line}
Let $\spc{X}$ be a compact metric space.
Show that for any two points $f,g\in\Inj \spc{X}$ lie on a geodesic $[pq]$ with $p,q\in \spc{X}$.
\end{thm}

A metric space $\spc{X}$ is called \index{$\delta$-hyperbolic}\emph{$\delta$-hyperbolic} if 
\[\dist{p}{q}{}+\dist{x}{y}{}\le
\max\{\,\dist{p}{x}{}+\dist{q}{y}{},
\,
\dist{p}{y}{}+\dist{q}{x}{}\,\}+2\cdot\delta\]
for any $p,q,x,y\in \spc{X}$.

\begin{thm}{Advanced exercise}\label{ex:delta-hyp}
Show that $\Inj \spc{X}$ is $\delta$-hyperbolic if and only if so is $\spc{X}$.
\end{thm}


\section{Injective envelope}

An extension $\spc{E}$ of a metric space $\spc{X}$ will be called its \index{injective envelope}\emph{injective envelope} if $\spc{E}$ is an injective space, and there is no proper injective subspace of $\spc{E}$ that contains $\spc{X}$.

Two injective envelopes $e\:\spc{X}\hookrightarrow \spc{E}$ and $f\:\spc{X}\hookrightarrow \spc{F}$ are called  equivalent if there is an isometry $\iota\: \spc{E}\to\spc{F}$ such that $f=\iota\circ e$.

\begin{thm}{Theorem}\label{thm:inj-envelope}
For any metric space $\spc{X}$, its extension $\Inj\spc{X}$ is an injective envelope.

Moreover, any other injective envelope of $\spc{X}$ is equivalent to $\Inj\spc{X}$.
\end{thm}

\parit{Proof.} 
Suppose $S\subset \Inj\spc{X}$ is an injective subspace containing $\spc{X}$.
Since $S$ is injective, there is a short map $w\:\Inj\spc{X}\to S$ that fixes all points in $\spc{X}$.

Suppose that $w\:f\mapsto f'$; observe that $f(x)\ge f'(x)$ for any $x\in \spc{X}$.
Since $f$ is extremal, $f=f'$;
that is, $w$ is the identity map, and therefore $S=\Inj\spc{X}$.

Assume we have another injective envelope $e\:\spc{X}\hookrightarrow \spc{E}$.
Then there are short maps $v\:\spc{E}\to \Inj\spc{X}$ and $w\:\Inj\spc{X}\to \spc{E}$ such that $x=v\circ e(x)$ and $e(x)=w(x)$ for any $x\in\spc{X}$.
From above, the composition $v\circ w$ is the identity on $\Inj\spc{X}$.
In particular, $w$ is distance-preserving.

The composition $w\circ v\:\spc{E}\to \spc{E}$ is a short map that fixes points in $e(\spc{X})$.
Since $e\:\spc{X}\hookrightarrow \spc{E}$ is an injective envelope, the composition $w\circ v$ and therefore $w$ are onto.
Whence $w$ is an isometry.
\qeds

\begin{thm}{Exercise}\label{ex:inj-envelope}
Suppose $e\:\spc{X}\hookrightarrow \spc{E}$ and $f\:\spc{X}\hookrightarrow \spc{F}$ are two injective envelopes of $\spc{X}$.
Show that there is a unique isometry $\iota\:\spc{E}\to \spc{F}$ such that $\iota\circ e=f$.
\end{thm}

\begin{thm}{Exercise}\label{ex:d-p-inclusion}
Suppose $\spc{X}$ is a subspace of a metric space $\spc{U}$.
Show that the inclusion $\spc{X}\hookrightarrow\spc{U}$ can be extended to a distance-preserving inclusion $\Inj\spc{X}\hookrightarrow\Inj\spc{U}$.
\end{thm}

\begin{thm}{Exercise}\label{ex:hemisphere-inj}
Consider the hemisphere 
\begin{align*}
\mathbb{S}^2_+&=\set{(x,y,z)\in\RR^3}{x^2+y^2+z^2=1,\quad z\ge0}
\intertext{and its boundary}
\mathbb{S}^1&=\set{(x,y,z)\in\RR^3}{x^2+y^2+z^2=1,\quad z=0};
\end{align*}
 both with induced length metrics.
 
Show that there is unique isometric embedding $\iota\:\mathbb{S}^2_+\hookrightarrow\Inj\mathbb{S}^1$ such that $\iota(u)=u$ for any $u\in \mathbb{S}^1$.
\end{thm}


\section{Remarks}

Injective spaces were introduced by Nachman Aronszajn and Prom Panitchpakdi \cite{aronszajn-panitchpakdi}.
The injective envelope was introduced by John Isbell \cite{isbell}; it is also known as \index{tight span}\emph{tight span} and \index{hyperconvex hull}\emph{hyperconvex hull}.

It was observed by John Isbell \cite{isbell2} that \textit{if $\spc{X}$ is a Banach space, then its injective hull $\Inj\spc{X}$ has a natural structure of Banach space} (which is unique by the Mazur--Ulam theorem).
Moreover, $\spc{X}$ is a linear subspace of $\Inj\spc{X}$.
 
Let us mention that a metric space $\spc{X}$ is called \index{convex space}\emph{convex} if for any pair of points $x_1,x_2\in \spc{X}$ and any $r_1,r_2\ge 0$ we have 
\[r_1+r_2\ge \dist{x_1}{x_2}{\spc{X}}\qquad\Longrightarrow\qquad\cBall[x,r_1]_\spc{X}\cap \cBall[y,r_2]_\spc{X}\ne\emptyset;\]
in other words, a pair of balls intersect if the triangle inequality does not forbid it.
Clearly, hyperconvexity (\ref{thm:injective=hyperconvex:balls}) is stronger than convexity.
Note that \textit{any geodesic space is convex}.
The converse does not hold in general, but by \ref{lem:mid>geod:geod} \textit{any complete convex space is geodesic}.

More generally, a metric space $\spc{X}$ is called \index{$n$-heperconvex space}\emph{$n$-heperconvex} if the condition in \ref{thm:injective=hyperconvex:balls} holds only for families with at most $n$ balls; so \textit{convex means $2$-hyperconvex}.

The following striking result was proved by Benjamin Miesch and Maël Pavón \cite{miesch-pavon2016}.

\begin{thm}{Theorem}
Any complete $4$-hyperconvex space is finitely hyperconvex.
\end{thm}

So, by \ref{ex:compact-hyperconvex}, it follows that \textit{any proper $4$-hyperconvex space is hyperconvex}.

\begin{thm}{Exercise}\label{ex:3-4-hypreconvex}
Show that $\ell^1$ is $3$- but not $4$-hyperconvex.
\end{thm}
 

Recall that if the following inequality
\[\dist{x}{z}{\spc{X}}
\le
\max\{\,\dist{x}{y}{\spc{X}},\dist{y}{z}{\spc{X}}\,\}\]
holds for any three points $x,y,z$ in a metric space $\spc{X}$,
then $\spc{X}$ is called an \index{ultrametric space}\emph{ultrametric space}.
In some sense, ultrametric spaces are dual to injective spaces.

\begin{thm}{Exercise}\label{ex:ultrametric}
Suppose that a metric space $\spc{X}$ satisfies the following property:
For any subspace $\spc{A}$ in $\spc{X}$ and any other metric space $\spc{Y}$, any short map $f\:\spc{A}\to \spc{Y}$ can be extended to a short map $F\:\spc{X}\to \spc{Y}$.

Show that $\spc{X}$ is an ultrametric space.
\end{thm}

A subspace $\spc{S}$ of a metric space $\spc{X}$ is called its \index{short retract}\emph{short retract} if there is a short map $\spc{X}\to \spc{S}$ that is the identity on $\spc{S}$.

\begin{thm}{Exercise}\label{ex:ultrametric-converse}
Show that any compact subspace $\spc{K}$ of an ultrametric space $\spc{X}$ is its short retract.

Construct an example of a complete ultrametric space $\spc{X}$ with a closed subspace $\spc{Q}$ that is not its short retract.
\end{thm}

The following exercise gives a sufficient condition for the existence of a short extension.

\begin{thm}{Exercise}\label{ex:petrunin-stadler}
Let $\spc{X}$ and $\spc{Y}$ be metric spaces, $A\subset \spc{X}$, and $f\:A\z\to \spc{Y}$ be a short map.
Assume $\spc{Y}$ is compact and for any finite set $F\subset \spc{X}$ there is a short map $F\to \spc{Y}$ that agrees with $f$ on $F\cap A$.
Show that there is a short map $\spc{X}\to \spc{Y}$ that agrees with $f$ on $A$.
\end{thm}

\chapter{Space of sets}

\section{Hausdorff distance}

Let $\spc{X}$ be a metric space.
Given a subset $A\subset \spc{X}$,
consider the distance function to $A$
$$\distfun_A: \spc{X} \to [0,\infty)$$
defined as 
$$\distfun_A(x)
\df
\inf_{a\in A}\{\,\dist ax{\spc{X}}\,\}.$$

\begin{thm}{Definition}\label{def:hausdorff-convergence}
Let $A$ and $B$ be two compact subsets of a metric space $\spc{X}$.
Then the \index{Hausdorff distance}\emph{Hausdorff distance} between $A$ and $B$ is defined as 
$$|A-B|_{\Haus\spc{X}}
\df
\sup_{x\in \spc{X}}\{\,|\distfun_A(x)-\distfun_B(x)|\,\}.
$$

\end{thm}

The following observation gives a useful reformulation of the definition:

\begin{thm}{Observation}\label{obs:Haus-nbhds}
Suppose $A$ and $B$ be two compact subsets of a metric space $\spc{X}$.
Then $|A-B|_{\Haus\spc{X}}< R$ if and only if and only if 
$B$ lies in an $R$-neighborhood of $A$, 
and 
$A$ lies in an $R$-neighborhood of~$B$.
\end{thm}



Note that the set of all nonempty compact subsets of a metric space $\spc{X}$ equipped with the Hausdorff metric forms a metric space.
This new metric space will be denoted as $\Haus\spc{X}$.


\begin{thm}{Exercise}\label{ex:diam}
Let $\spc{X}$ be a metric space.
Given a subset $A\subset \spc{X}$ define its \index{diameter}\emph{diameter} as 
$$\diam A\df\sup_{a,b\in A} |a-b|.$$

Show that 
$$\diam\:\Haus\spc{X}\to \RR$$ 
is a \index{Lipschitz function}\emph{$2$-Lipschitz function};
that is,
\[|\diam A-\diam B|\le 2\cdot\dist{A}{B}{\Haus\spc{X}}\]
for any two compact nonempty sets $A,B\subset\spc{X}$.
\end{thm}


\begin{thm}{Exercise}\label{ex:Hausdorff-bry}
Let $A$ and $B$ be two compact subsets in the Euclidean plane $\RR^2$.
Assume $|A-B|_{\Haus\RR^2}<\eps$.

\begin{subthm}{ex:Hausdorff-bry:conv}
Show that $|\Conv A-\Conv B|_{\Haus\RR^2}<\eps$, where $\Conv A$ denoted the convex hull of $A$.
\end{subthm}
\begin{subthm}{ex:Hausdorff-bry:bry}
Is it true that
$|\partial A-\partial B|_{\Haus\RR^2}<\eps$,
where $\partial A$ denotes the boundary of $A$.

Does the converse hold? That is, assume $A$ and $B$ be two compact subsets in $\RR^2$
and $|\partial A-\partial B|_{\Haus\RR^2}<\eps$; 
is it true that $|A-B|_{\Haus\RR^2}\z<\eps$?
\end{subthm}

\end{thm}

Note that part \ref{SHORT.ex:Hausdorff-bry:conv} implies that $A\mapsto \Conv A$ defines a short map $\Haus\RR^2\to \Haus\RR^2$. 

\begin{thm}{Exercise}\label{ex:Haus-func}
Let $A$ and $B$ be two compact subsets in metric space $\spc{X}$.
Show that 
\[\dist{A}{B}{\Haus\spc{X}}=\sup_f\, \{\,\max_{a\in A}\{f(a)\}-\max_{b\in B}\{f(b)\,\},\]
where the least upper bound is taken for all $1$-Lipschitz functions $f$.

\end{thm}


\section{Hausdorff convergence}

\begin{thm}{Blaschke selection theorem}\label{thm:compact+Hausdorff}
A metric space $\spc{X}$ is compact if and only if
so is $\Haus\spc{X}$.
\end{thm}

The Hausdorff metric can be used to define convergence.
Namely, suppose $K_1,K_2,\dots$, and $K_\infty$ are compact sets in a metric space $\spc{X}$.
If $|K_\infty-K_n|_{\Haus\spc{X}}\to0$ as $n\to\infty$, then we say that 
the sequence $K_n$ {}\emph{converges} to $K_\infty$ \index{convergence in the sense of Hausdorff}\emph{in the sense of Hausdorff};
or we can say that $K_\infty$ is \emph{Hausdorff limit} of the sequence $K_n$.

Note that the theorem implies that from any sequence of compact sets in $\spc{X}$ one can select a subsequence that converges in the sense of Hausdorff; 
for that reason, it is called a \emph{selection} theorem. 

\parit{Proof; ``only if'' part.}
Consider the map $\iota$ that sends point $x\in \spc{X}$ to the one-point subset $\{x\}$ of $\spc{X}$.
Note that $\iota\:\spc{X}\to \Haus\spc{X}$ is distance-preserving.

Suppose that $A\subset \spc{X}$.
Note that $\diam A=0$ if and only if $A$ is a one-point set.
Therefore, from Exercise~\ref{ex:diam}, it follows 
that $\iota(\spc{X})$ is a closed subset of the compact space $\Haus\spc{X}$.
Whence $\iota(\spc{X})$, and therefore $\spc{X}$, are compact.
\qeds

To prove the ``if'' part we will need the following two lemmas.

\begin{thm}{Monotone convergence}\label{lem:decreasing-converges}
Let $K_1\supset K_2\supset\dots$ be a nested sequence of nonempty compact sets in a metric space $\spc{X}$.
Then $K_\infty\z=\bigcap_n K_n$ is the Hausdorff limit of $K_n$;
that is, $|K_\infty-K_n|_{\Haus\spc{X}}\to0$ as $n\to\infty$.
\end{thm}

\parit{Proof.}
By finite intersection property, $K_\infty$ is a nonempty compact set.

If the assertion were false, then there is $\eps>0$ such that for each $n$ 
one can choose $x_n\in K_n$
such that $\distfun_{K_\infty}(x_n)\ge\eps$.
Note that $x_n\in K_1$ for each $n$.
Since $K_1$ is compact, 
there is 
a \index{partial limit}\emph{partial limit}%
\footnote{Partial limit is a limit of a subsequence.}
 $x_\infty$ of $x_n$.
Clearly, $\distfun_{K_\infty}(x_\infty)\ge \eps$.

On the other hand, since $K_n$ is closed and $x_m\in K_n$ for $m\ge n$,
we get $x_\infty\in K_n$ for each $n$.
It follows that $x_\infty\in K_\infty$ and therefore $\distfun_{K_\infty}(x_\infty)=0$ ---
a contradiction.\qeds


\begin{thm}{Lemma}\label{lem:complete+Hausdorff}
If $\spc{X}$ is a compact metric space, then $\Haus\spc{X}$
is complete.
\end{thm}

\parit{Proof.}
Let $(Q_n)$ be a Cauchy sequence in $\Haus\spc{X}$.
Passing to a subsequence of $Q_n$ we may assume that 
$$|Q_n-Q_{n+1}|_{\Haus\spc{X}}\le \tfrac1{10^n}\eqlbl{eq:eps=1/10}$$
for each $n$.

Denote by $K_n$ the closed $\tfrac1{10^n}$-neighborhood of $Q_n$;
that is,
\begin{align*}
K_n&= \set{x\in \spc{X}}{\distfun_{Q_n}(x)\le \tfrac1{10^n}}
\end{align*}
Since $\spc{X}$ is compact so is each $K_n$.

By \ref{obs:Haus-nbhds}, $|Q_n-K_n|_{\Haus\spc{X}}\le \tfrac1{10^n}$.
From \ref{eq:eps=1/10}, we get
$K_n\supset K_{n+1}$ 
for each $n$.
Set 
$$K_\infty=\bigcap_{n=1}^\infty K_n.$$
By the monotone convergence (\ref{lem:decreasing-converges}),
 $|K_n-K_\infty|_{\Haus\spc{X}}\to 0$ as $n\to\infty$.
Since $|Q_n-K_n|_{\Haus\spc{X}}\le \tfrac1{10^n}$, we get $|Q_n-K_\infty|_{\Haus\spc{X}}\to 0$ as $n\to\infty$ --- hence the lemma.
\qeds

\begin{thm}{Exercise}\label{ex:closure-union}
Let $\spc{X}$ be a complete metric space and $K_1,K_2,\dots$ be a sequence of compact sets 
that converges in the sense of Hausdorff.
Show that the union $K_1\cup K_2\cup\dots$ is a compact closure.

Use this statement to show that in Lemma~\ref{lem:complete+Hausdorff} compactness of $\spc{X}$ can be exchanged to completeness.
\end{thm}

\parit{Proof of ``if'' part in \ref{thm:compact+Hausdorff}.}
According to Lemma~\ref{lem:complete+Hausdorff},
$\Haus\spc{X}$ is complete.
It remains to show that $\Haus\spc{X}$ is totally bounded (\ref{totally-bounded});
that is, given $\eps>0$ there is a finite $\eps$-net in $\Haus\spc{X}$.

Choose a finite $\eps$-net $A$ in $\spc{X}$.
Denote by $B$ the set of all subsets of $A$.
Note that  $B$ is a finite set in $\Haus\spc{X}$.
For each compact set $K\subset \spc{X}$, consider the subset $K'$ of all points $a\in A$
such that $\distfun_K(a)\le \eps$.
Observe that $K' \in B$ and $|K-K'|_{\Haus\spc{X}}\le\eps$.
In other words, $B$ is a finite $\eps$-net in $\Haus\spc{X}$.
\qeds

\begin{thm}{Exercise}\label{ex:Haus-length}
Let $\spc{X}$ be a complete metric space.
Show that $\spc{X}$ is a length space if and only if so is $\Haus\spc{X}$.
\end{thm}

\section{An application}

The following statement is called \index{isoperimetric inequality}\emph{isoperimetric inequality in the plane}.

\begin{thm}{Theorem}\label{thm:isoperimetric}
Among the plane figures bounded by closed curves of length at most $\ell$ the round disk has the maximal area.
\end{thm}

In this section, we will sketch a proof of the isoperimetric inequality that uses the Hausdorff convergence.
It is based on the following exercise.

\begin{thm}{Exercise}\label{ex:Huas-perimeter-area}
Let $\spc{C}$ be a subspace of $\Haus\RR^2$ formed by all compact convex subsets in $\RR^2$.
Show that perimeter\footnote{If the set degenerates to a line segment of length $\ell$, then its perimeter is defined as $2\cdot \ell$.} and area are continuous on~$\spc{C}$.
That is, if a sequence of convex compact plane sets $X_n$ converges to $X_\infty$ in the sense of Hausdorff, then 
\[\perim X_n\to \perim X_\infty\quad\text{and}\quad\area X_n\to\area X_\infty\]
as $n\to\infty$.
\end{thm}

\parit{Semiproof of \ref{thm:isoperimetric}.}
It is sufficient to consider only convex figures of the given perimeter; if a figure is not convex, pass to its convex hull and observe that it has a larger area and smaller perimeter.


Note that the selection theorem (\ref{thm:compact+Hausdorff}) together with the exercise imply the existence of figure $D$ with perimeter $\ell$ and maximal area.

It remains to show that $D$ is a round disk.
This is a problem in elementary geometry.

Let us cut $D$ along a chord $[ab]$ into two lenses, $L_1$ and $L_2$.
Denote by $L_1'$ the reflection of $L_1$ across the perpendicular bisector of $[ab]$.
Note that $D$ and $D'=L_1'\cup L_2$ have the same perimeter and area.
That is, $D'$ has perimeter $\ell$ and maximal possible area;
in particular, $D'$ is convex.

The following exercise will finish the proof.
\qeds

{

\begin{wrapfigure}{o}{57 mm}
\vskip-5mm
\centering
\includegraphics{mppics/pic-405}
\end{wrapfigure}

\begin{thm}{Exercise}\label{ex:round-disc}
Suppose $D$ is a convex figure such that for any chord $[ab]$ of $D$ the above construction produces a convex figure $D'$.
Show that $D$ is a round disk.
\end{thm}


}

Another popular way to prove that $D$ is a round disk is given by the so-called {}\emph{Steiner's 4-joint method} \cite{blaschke}.

\section{Remarks}\label{sec:H-variation}

It seems that Hausdorff convergence was first introduced by Felix Hausdorff~\cite{hausdorff}.
A couple of years later an equivalent definition was given by Wilhelm Blaschke~\cite{blaschke}.

The following refinement of the definition was introduced by  Zdeněk Frolík \cite{frolik},
later it was rediscovered by Robert Wijsman~\cite{wijsman}.  
This refinement is also called \index{Hausdorff convergence}\emph{Hausdorff convergence};
in fact, it takes an intermediate place between the original Hausdorff convergence and {}\emph{closed convergence}, also introduced by Hausdorff in \cite{hausdorff}.

\begin{thm}{Definition}\label{def:gen-Haus-conv}
Let $A_1,A_2,\dots$ be a sequence of closed sets in a metric space $\spc{X}$.
We say that the sequence $A_n$ converges to a closed set $A_\infty$ in the sense of Hausdorff if for any $x\in\spc{X}$, we have
$\distfun_{A_n}(x)\z\to \distfun_{A_\infty}(x)$ as $n\to\infty$.
\end{thm}

For example, suppose $\spc{X}$ is the Euclidean plane and $A_n$ is the circle with radius $n$ and center at the point $(n,0)$.
If we use the standard definition (\ref{def:hausdorff-convergence}), then the sequence $(A_n)$ diverges, but it converges to the $y$-axis in the sense of Definition~\ref{def:gen-Haus-conv}.

The following exercise is analogous to the Blaschke selection theorem (\ref{thm:compact+Hausdorff}) for the modified Hausdorff convergence.

\begin{thm}{Exercise}\label{ex:generalized-selection}
Let $\spc{X}$ be a proper metric space
and $A_1,A_2,\dots$ be a sequence of closed sets in~$\spc{X}$.
Assume that for some (and therefore any) point  $x\in\spc{X}$, 
the sequence $a_n=\distfun_{A_n}(x)$ is bounded.
Show that the sequence  $A_1,A_2,\dots$ has a convergent subsequence in the sense of Definition~\ref{def:gen-Haus-conv}.
\end{thm}

\chapter{Space of spaces}

\section{Gromov--Hausdorff metric}

The goal of this section is to cook up a metric space out of metric spaces.
More precisely, we want to define the so-called  Gromov--Hausdorff metric on the set of {}\emph{isometry classes} of compact metric spaces.
(Being isometric is an equivalence relation, 
and an isometry class is an equivalence class with respect to this equivalence relation.)

The obtained metric space will be denoted by $\GH$.
Given two metric spaces $\spc{X}$ and $\spc{Y}$,
denote by $[\spc{X}]$ and $[\spc{Y}]$ their isometry classes;
that is, $\spc{X}'\in [\spc{X}]$ if and only if $\spc{X}'\iso \spc{X}$.
Pedantically, the Gromov--Hausdorff distance from $[\spc{X}]$ 
to $[\spc{Y}]$ should be denoted as $|[\spc{X}]-[\spc{Y}]|_{\GH}$;
but we will write it as $|\spc{X}\z-\spc{Y}|_{\GH}$ and say (not quite correctly) 
``$|\spc{X}\z-\spc{Y}|_{\GH}$ is the Gromov--Hausdorff distance from  $\spc{X}$ 
to  $\spc{Y}$''.
In other words, from now on the term {}\emph{metric space} might also stand for its {}\emph{isometry class}.

The metric on $\GH$ is defined as the maximal metric such that \textit{the distance between subspaces in a metric space is not greater than the Hausdorff distance between them}.
Here is a formal definition:

\begin{thm}{Definition}\label{def:GH}
Let $\spc{X}$ and $\spc{Y}$ be compact metric spaces. 
The Gromov--Hausdorff distance $|\spc{X}-\spc{Y}|_{\GH}$ is defined by the following
relation.
 
Given  $r > 0$, we have that $|\spc{X}-\spc{Y}|_{\GH} < r$ if and only if there exist a metric
space $\spc{Z}$ and subspaces $\spc{X}'$ and $\spc{Y}'$ in $\spc{Z}$ that are isometric to $\spc{X}$ and $\spc{Y}$
respectively and such that $|\spc{X}'-\spc{Y}'|_{\Haus\spc{Z}} < r$. 
(Here $|\spc{X}'-\spc{Y}'|_{\Haus\spc{Z}}$ denotes the Hausdorff distance between sets $\spc{X}'$ and $\spc{Y}'$ in $\spc{Z}$.)
\end{thm}

Note that passing to the subspace $\spc{X}'\cup\spc{Y}'$ of $\spc{Z}$ does not affect the definition.
Therefore we can always assume that $\spc{Z}$ is compact.

\begin{thm}{Theorem}\label{thm:GH-is-a-metric}
The set of isometry classes of compact metric spaces equipped with Gromov--Hausdorff metric forms a metric space (which is denoted by $\GH$).

In other words, for arbitrary  compact metric spaces $\spc{X}$, $\spc{Y}$ and $\spc{Z}$ the following conditions hold:

\begin{subthm}{GH-1} $|\spc{X}-\spc{Y}|_{\GH}\ge 0$;
\end{subthm}

\begin{subthm}{GH-2} $|\spc{X}-\spc{Y}|_{\GH}=0$ if and only if $\spc{X}$ is isometric to $\spc{Y}$;
\end{subthm}

\begin{subthm}{GH-3} $|\spc{X}-\spc{Y}|_{\GH}=|\spc{Y}-\spc{X}|_{\GH}$;
\end{subthm}

\begin{subthm}{GH-4} $|\spc{X}-\spc{Y}|_{\GH}+|\spc{Y}-\spc{Z}|_{\GH}\ge |\spc{X}-\spc{Z}|_{\GH}$.
\end{subthm}
\end{thm}


Note that \ref{SHORT.GH-1}, \ref{SHORT.GH-3},
and the ``if''-part of \ref{SHORT.GH-2} follow directly from Definition \ref{def:GH}.
Part \ref{SHORT.GH-4} will be proved in Section~\ref{sec:GH-approx}.
The ``only-if''-part of \ref{SHORT.GH-2} will be proved in Section~\ref{sec:alm-isom}.

Recall that $a\cdot\spc{X}$ denotes $\spc{X}$ \index{scaled space}\emph{scaled} by factor $a>0$;
that is, $a\cdot\spc{X}$ is a metric space with the underlying set of $\spc{X}$ and the metric defined by
\[\dist{x}{y}{a\cdot\spc{X}}\df a\cdot\dist{x}{y}{\spc{X}}.\]

\begin{thm}{Exercise}\label{ex:d_GH-and-diam}
Let $\spc{X}$ be a compact metric space,
$\spc{P}$ be the one-point metric space.

Prove that 
\begin{subthm}{ex:d_GH-and-diam:point}
\[|\spc{X}-\spc{P}|_{\GH}=\tfrac12\cdot \diam \spc{X}.\]

\end{subthm}

\begin{subthm}{ex:d_GH-and-diam:scale}
\[|a\cdot\spc{X}-b\cdot \spc{X}|_{\GH}=\tfrac12\cdot|a-b|\cdot\diam\spc{X}.\]
\end{subthm}


\end{thm}

\begin{thm}{Exercise}\label{ex:rectangle}
Let $\spc{A}_r$ be a rectangle $1$ by $r$ in the Euclidean plane 
and $\spc{B}_r$ be a closed line interval of length $r$.
Show that 
\[|\spc{A}_r-\spc{B}_r|_{\GH}>\tfrac1{10}\]
for all large $r$.
\end{thm}

\begin{thm}{Advanced exercise}\label{ex:GH-inj}
Let $\spc{X}$ and $\spc{Y}$ be compact metric spaces;
denote by $\hat{\spc{X}}$ and $\hat{\spc{Y}}$ their injective envelopes (see \ref{sec:extremal-functions}).
Show that 
\[|\hat{\spc{X}}-\hat{\spc{Y}}|_{\GH}\le 2\cdot|\spc{X}- \spc{Y}|_{\GH}.\] 

\end{thm}

\section{Approximations}\label{sec:GH-approx}

\begin{thm}{Definition}\label{ex:defGHR}
Let $\spc{X}$ and $\spc{Y}$ be two metric spaces.
A relation $\approx$ between points in $\spc{X}$ and $\spc{Y}$ is called $\eps$-approximation if the following conditions hold:
\begin{itemize}
\item For any $x\in  \spc{X}$ there is $y\in \spc{Y}$ such that $x\approx y$.
\item For any $y\in  \spc{Y}$ there is $x\in \spc{X}$ such that $x\approx y$.
\item If for some $x, x'\in  \spc{X}$ and $y,y'\in \spc{Y}$ we have $x\approx y$ and $x'\approx y'$, then 
\[\bigl|\dist{x}{x'}{\spc{X}}-\dist{y}{y'}{\spc{Y}}\bigr|<2\cdot\eps.\]
\end{itemize}

\end{thm}

\begin{thm}{Exercise}\label{ex:H-R}
Let $\spc{X}$ and $\spc{Y}$ be two compact metric spaces.
Show that
\[\dist{\spc{X}}{\spc{Y}}{\GH}<\eps\]
if and only if there is an $\eps$-approximation between $\spc{X}$ and $\spc{Y}$.

In other words $\dist{\spc{X}}{\spc{Y}}{\GH}$ is the greatest lower bound of values $\eps>0$ such that  there is an $\eps$-approximation between $\spc{X}$ and $\spc{Y}$.
\end{thm}

\parit{Proof of \ref{GH-4}.}
Suppose that 
\begin{itemize}
\item $\approx_1$ is a relation between points in $\spc{X}$ and $\spc{Y}$,
\item $\approx_2$ is a relation between points in $\spc{Y}$ and $\spc{Z}$.
\end{itemize}
Consider the relation $\approx_3$ between points in $\spc{X}$ and $\spc{Z}$ such that
$x\approx_3 z$ if and only if there is $y\in  \spc{Y}$ such that 
$x\approx_1 y$ and $y\approx_2 z$.

It is straightforward to check that if $\approx_1$ is an $\eps_1$-approximation and $\approx_2$ is an $\eps_2$-approximation, then $\approx_3$ is an $(\eps_1+\eps_2)$-approximation.

Applying \ref{ex:H-R}, we get that if 
\[|\spc{X}-\spc{Y}|_{\GH}<\eps_1
\quad\text{and}\quad
|\spc{Y}-\spc{Z}|_{\GH}<\eps_2,
\]
then 
\[|\spc{X}-\spc{Z}|_{\GH}<\eps_1+\eps_2.\]
Hence \ref{GH-4} follows.
\qeds


\section{Almost isometries}\label{sec:alm-isom}

\begin{thm}{Definition} Let $\spc{X}$ and $\spc{Y}$ be metric spaces and $\eps>0$. 
A  map\footnote{possibly noncontinuous} $f\: \spc{X} \z\to \spc{Y}$ is called an \index{almost isometry}\emph{$\eps$-isometry} 
if $f(\spc{X})$ is an $\eps$-net in $\spc{Y}$ and
\[\bigl|\dist{x}{x'}{\spc{X}}-\dist{f(x)}{f(x')}{\spc{Y}}\bigr|<\eps.\]
for any $x,x'\in \spc{X}$.
\end{thm}

\begin{thm}{Exercise}\label{ex:eps-isom}
Let $\spc{X}$ and $\spc{Y}$ be compact metric spaces.

\begin{subthm}{ex:eps-isom:GH>isom}
If $\dist{\spc{X}}{\spc{Y}}{\GH}<\eps$, then there is a $2\cdot\eps$-isometry $f\:\spc{X}\to\spc{Y}$.
\end{subthm}

\begin{subthm}{ex:eps-isom:isom>GH}
If there is an $\eps$-isometry $f\:\spc{X}\to\spc{Y}$, then $\dist{\spc{X}}{\spc{Y}}{\GH}<\eps$.
\end{subthm}

\end{thm}

\parit{Proof of the ``only if''-part in \ref{GH-2}.}
\label{page:GH-2-proof}
Let $\spc{X}$ and $\spc{Y}$ be compact metric spaces.
Suppose that $\dist{\spc{X}}{\spc{Y}}{\GH}<\eps$ for any $\eps>0$;
we need to show that there is an isometry $\spc{X}\to\spc{Y}$.

By \ref{ex:eps-isom:GH>isom}, for each positive integer $n$, we can choose a $\tfrac1n$-isometry $f_n\:\spc{X}\to\spc{Y}$.

Since $\spc{X}$ is compact, 
we can choose a countable dense set
$S$ in~$\spc{X}$.
Applying the diagonal procedure if necessary, we can assume that for every $x \in S$ the sequence $f_n(x)$ 
converges in $\spc{Y}$. 
Consider the pointwise limit map  $f_\infty \: S \to \spc{Y}$,
 $$f_\infty(x) \df \lim_{n\to\infty} f_n (x)$$ for every $x \in S$. 
Since $$|f_n (x)- f_n (x')|_{\spc{Y}}\lg |x- x'|_\spc{X} \pm\tfrac1n,$$ 
we have 
$$|f_\infty(x)-f_\infty (x')|_{\spc{Y}} 
= \lim_{n\to\infty} |f_n(x)-f_n (x')|_{\spc{Y}} 
= |x -x'|_\spc{X}$$ for all
$x, x' \in S$; 
that is, the map $f_\infty\:S\to \spc{Y}$ is distance-preserving. 
Therefore, $f_\infty$ can be extended to a distance-preserving map from the whole $\spc{X}$ to $\spc{Y}$.

The latter can be done by setting 
$$f_\infty(x)=\lim_{n\to\infty} f_\infty(x_n)$$ 
for some sequence $x_n$ of points  in $S$
that converges to $x$ in $\spc{X}$.
Indeed, if $x_n\to x$, then the sequence $x_n$ is Cauchy.
Since $f_\infty$ is distance-preserving, $y_n=f_\infty(x_n)$ is also a Cauchy sequence in $\spc{Y}$;
therefore it converges.
It remains to observe that this construction does not depend on the choice of the sequence $x_n$.

This way we obtain a distance-preserving map $f_\infty\:\spc{X}\to \spc{Y}$. 
It remains to show that $f_\infty$ is surjective; that is, $f_\infty(\spc{X})=\spc{Y}$.

The same argument produces a distance-preserving map $g_\infty\:\spc{Y}\z\to \spc{X}$.
If $f_\infty$ is not surjective, then neither is the composition $f_\infty\z\circ g_\infty\:\spc{Y}\to \spc{Y}$.
So $f_\infty \z\circ g_\infty$ is a distance-preserving map from a compact space to itself which is not an isometry.
The latter contradicts \ref{ex:non-contracting-map}. 
\qeds

\section{Convergence}

The Gromov--Hausdorff metric is used to define Gromov--Hausdorff convergence.
Namely, a sequence of compact metric spaces $\spc{X}_n$ converges to compact metric spaces $\spc{X}_\infty$ in the sense of Gromov--Hausdorff if 
\[\dist{\spc{X}_n}{\spc{X}_\infty}{\GH}\to 0\quad\text{as}\quad n\to\infty.\]

This convergence is more important than the metric ---
in all applications, we use only the topology on $\GH$
and we do not care about the particular value of Gromov--Hausdorff distance between spaces.
The following observation follows from \ref{ex:eps-isom}:

\begin{thm}{Observation}\label{obs:GH-e-isom}
A sequence of compact metric spaces $(\spc{X}_n)$ converges to  $\spc{X}_\infty$ in the sense of Gromov--Hausdorff if and only if there is a sequence $\eps_n\to0+$
and an $\eps_n$-isometry $f_n\:\spc{X}_n\to \spc{X}_\infty$ for each $n$.
\end{thm}

In the following exercises \textit{converge} means in the sense of Gromov--Hausdorff.

\begin{thm}{Exercise}\label{ex:GH-SC}
\begin{subthm}{ex:GH-SC:circle}
Show that a sequence of compact simply connected length spaces cannot converge to a circle.
\end{subthm}

\begin{subthm}{ex:GH-SC:nonsc-limit}
Construct a sequence of compact simply connected length spaces that converges to a compact non-simply connected space.
\end{subthm}
\end{thm}

\begin{thm}{Exercise}\label{ex:sphere-to-ball}
\begin{subthm}{ex:sphere-to-ball:2}
Show that a sequence of length metrics on the 2-sphere cannot converge to the unit disk.
\end{subthm}

\begin{subthm}{ex:sphere-to-ball:3}
Construct a sequence of length metrics on the 3-sphere that converges to a unit 3-ball.
\end{subthm}

\end{thm}

Given two metric spaces $\spc{X}$ and $\spc{Y}$, we will write $\spc{X}\le \spc{Y}$ if there is a noncontracting map $f\:\spc{X}\to \spc{Y}$;
that is, if 
$$ |x-x'|_{\spc{X}}\le|f(x)-f(x')|_{\spc{Y}}$$
for any $x,x'\in \spc{X}$.

Further, given $\eps>0$, we will write $\spc{X}\le \spc{Y}+\eps$
if there is a map $f\:\spc{X}\to \spc{Y}$ such that 
$$|x-x'|_{\spc{X}}\le|f(x)-f(x')|_{\spc{Y}}+\eps$$
for any $x,x'\in \spc{X}$.

\section{Uniformly totally bonded families}

\begin{thm}{Definition}\label{def:utb}
A family $\spc{Q}$ of (isometry classes) of compact metric spaces is called  \index{uniformly totally bonded family}\emph{uniformly totally bonded} if it meets the following two conditions:

\begin{subthm}{}
spaces in $\spc{Q}$ have uniformly bounded diameters; that is, there is $D\in\RR$ such that
\[\diam\spc{X}\le D\]
for any space $\spc{X}$ in $\spc{Q}$.
\end{subthm}

\begin{subthm}{}
For any $\eps>0$ there is $n\in\NN$ such that any space $\spc{X}$ in $\spc{Q}$ admits an $\eps$-net with at most $n$ points.
\end{subthm}
\end{thm}

\begin{thm}{Exercise}\label{ex:utb+pack}
Let $\spc{Q}$ be a family of compact spaces with uniformly bounded diameters.
Show that $\spc{Q}$ is uniformly totally bonded if for any $\eps>0$ there is $n\in\NN$ such that 
\[\pack_\eps\spc{X}\le n\]
for any space $\spc{X}$ in $\spc{Q}$.
\end{thm}


Fix a real constant $C$.
A Borel measure $\mu$ on a metric space $\spc{X}$ is called \index{doubling space}\emph{$C$-doubling} if
\[\mu[\oBall(p,2\cdot r)]< C\cdot\mu[\oBall(p,r)]\]
for any point $p\in \spc{X}$ and any $r>0$.
A Borel measure is called \index{doubling measure}\emph{doubling} if it is {}\emph{$C$-doubling} for some real constant $C$.

\begin{thm}{Exercise}\label{pr:doubling}
Let $\spc{Q}(C,D)$ be the set of all the compact metric spaces with diameter at most $D$ that admit a $C$-doubling measure.
Show that $\spc{Q}(C,D)$ is totally bounded.
\end{thm}

Recall that we write $\spc{X}\le\spc{Y}$ if there is a distance-nondecreasing map $\spc{X}\to\spc{Y}$.

\begin{thm}{Exercise}\label{pr:under}

\begin{subthm}{pr:under:if}
Let $\spc{Y}$ be a compact metric space.
Show that the set of all spaces $\spc{X}$ such that $\spc{X}\le\spc{Y}$
is uniformly totally bounded.
\end{subthm}

\begin{subthm}{pr:under:only-if}
Show that for any uniformly totally bounded set $\spc{Q}\subset\GH$ there is a compact space $\spc{Y}$
such that $\spc{X}\le\spc{Y}$ for any $\spc{X}$ in $\spc{Q}$.
\end{subthm}

\end{thm}

\section{Gromov's selection theorem}

The following theorem is analogous to Blaschke selection theorems (\ref{thm:compact+Hausdorff}).

\begin{thm}{Gromov selection theorem}\label{thm:gromov-compactness}
Let $\spc{Q}$ be a closed subset of $\GH$.
Then $\spc{Q}$ is compact if and only if the elements of $Q$ are uniformly totally bounded.
\end{thm}

\begin{thm}{Lemma}\label{lem:GH-complete}
The space $\GH$ is complete.
\end{thm}


Let us define gluing of metric spaces that will be used in the proof of the lemma.

Suppose 
$\spc{U}$ and $\spc{V}$ are metric spaces 
with isometric closed sets $A\subset\spc{U}$ and $A'\subset\spc{V}$;
let $\iota\:A\to A'$ be an isometry.
Consider the space $\spc{W}$ of all equivalence classes in $\spc{U}\sqcup\spc{V}$ with the equivalence relation given by $a\sim\iota(a)$ for any $a\in A$.

It is straightforward to check that the following defines a metric on~$\spc{W}$:
\begin{align*}
\dist{u}{u'}{\spc{W}}&\df\dist{u}{u'}{\spc{U}}
\\
\dist{v}{v'}{\spc{W}}&\df\dist{v}{v'}{\spc{V}}
\\
\dist{u}{v}{\spc{W}}&\df\min\set{\dist{u}{a}{\spc{U}}+\dist{v}{\iota(a)}{\spc{V}}}{a\in A}
\end{align*}
where $u,u'\in \spc{U}$ and $v,v'\in \spc{V}$.

The  space $\spc{W}$ is called the \index{gluing}\emph{gluing} of $\spc{U}$ and  $\spc{V}$ along~$\iota$; briefly, we can write
$\spc{W}=\spc{U}\sqcup_\iota\spc{V}$.
If one applies this construction to two copies of one space $\spc{U}$ with a set $A\subset \spc{U}$ and the identity map $\iota\:A\to A$, then the obtained space is called the \index{double}\emph{double} of $\spc{U}$ along~$A$; this space can be denoted by $\sqcup_A^2\spc{U}$.

Note that the inclusions $\spc{U}\hookrightarrow \spc{W}$ and $\spc{V}\hookrightarrow \spc{W}$ are distance preserving.
Therefore we can and will conside $\spc{U}$ and $\spc{V}$ as the subspaces of $\spc{W}$;
this way the subsets $A$ and $A'$ will be identified and denoted further by~$A$.
Note that $A=\spc{U}\cap \spc{V}\subset \spc{W}$.

\parit{Proof.}
Let $\spc{X}_1,\spc{X}_2,\dots$ be a Cauchy sequence in $\GH$.
Passing to a subsequence if necessary, 
we can assume that $|\spc{X}_n-\spc{X}_{n+1}|_{\GH}<\tfrac1{2^n}$ for each~$n$.
In particular, for each $n$ there is a metric space $\spc{V}_n$ with distance preserving inclusions $\spc{X}_n\hookrightarrow \spc{V}_n$ and $\spc{X}_{n+1}\hookrightarrow \spc{V}_n$ such that
\[|\spc{X}_n-\spc{X}_{n+1}|_{\Haus\spc{V}_n}<\tfrac1{2^n}\]
for each $n$.
Moreover, we may assume that $\spc{V}_n=\spc{X}_n\cup\spc{X}_{n+1}$.

Let us glue $\spc{V}_1$ to $\spc{V}_2$ along $\spc{X}_2$;
to the obtained space glue $\spc{V}_3$ along $\spc{X}_3$, and so on.
The obtained metric space $\spc{W}$
has an underlying set formed by the disjoint union of all $\spc{X}_n$ such that each inclusion $\spc{X}_n\z\hookrightarrow\spc{W}$ is distance preserving and
\[|\spc{X}_n-\spc{X}_{n+1}|_{\Haus\spc{W}}<\tfrac1{2^n}\]
for each $n$.
In particular,
\[|\spc{X}_m-\spc{X}_n|_{\Haus\spc{W}}<\tfrac1{2^{n-1}}\eqlbl{eq:|x_m-X_n|}\] 
if $m>n$.

Denote by $\bar{\spc{W}}$ the completion of $\spc{W}$.
Observe that the union $\spc{X}_1\z\cup \spc{X}_2\cup\z\dots\cup \spc{X}_n$ is compact and \ref{eq:|x_m-X_n|} implies that it forms a $\tfrac1{2^{n-1}}$-net in $\bar{\spc{W}}$.
Whence $\bar{\spc{W}}$ is compact; see \ref{totally-bounded} and \ref{ex:compact-net}.

Applying Blaschke selection theorem (\ref{thm:compact+Hausdorff}),
we can pass to a subsequence of $\spc{X}_n$ that converges in $\Haus\bar{\spc{W}}$; denote its limit by $\spc{X}_\infty$.
It remains to observe that $\spc{X}_\infty$ is the Gromov--Hausdorff limit of $(\spc{X}_n)$.
\qeds

\parit{Proof of \ref{thm:gromov-compactness}; ``only if'' part.}
Suppose that there is no sequence $\eps_n\to0$ as described in \ref{def:utb}.
Observe that in this case
there is a sequence of spaces $\spc{X}_n\in\spc{Q}$ such that 
$$\pack_\delta \spc{X}_n\to\infty
\quad\text{as}\quad
n\to\infty$$
for some fixed $\delta>0$.

Since $\spc{Q}$ is compact, 
this sequence has a partial limit, say $\spc{X}_\infty\in\spc{Q}$.
Observe that $\pack_{\delta} \spc{X}_\infty=\infty$.
Therefore, $\spc{X}_\infty$ is not compact --- a contradiction.

\parit{``If'' part.}
Let $\eps_n$ be a sequence as in the definition of uniformly totally bonded families (\ref{def:utb}).

Note that $\diam \spc{X}\le \eps_1$ for any $\spc{X}\in \spc{Q}$.
Given a positive integer $n$ consider the set of all metric spaces $\spc{W}_n$
with the number of points at most $n$ and diameter $\le \eps_1$.
Note that $\spc{W}_n$ is a compact set in $\GH$ for each $n$.

Further, a subspace formed by a maximal $\eps_n$-net of any $\spc{X}\in\spc{Q}$ belongs to $\spc{W}_n$.
Therefore, $\spc{W}_n\cap\spc{Q}$ is a compact $\eps_n$-net in  $\spc{Q}$.
That is, $\spc{Q}$ has a compact $\eps$-net for any $\eps>0$.
Since $\spc{Q}$ is closed in a complete space $\GH$, it implies that $\spc{Q}$ is compact.
\qeds

\begin{thm}{Exercise}\label{ex-GH-length}
Show that the space $\GH$ is 

\begin{subthm}{ex-GH-length:length}
length,
\end{subthm}

\begin{subthm}{ex-GH-length:geodesic}
geodesic.
\end{subthm}

\end{thm}

\begin{thm}{Exercise}\label{ex:GH-po}
For two metric spaces $\spc{X}$ and $\spc{Y}$,
we write $\spc{X}\le \spc{Y}+\eps$ if
there is a map $f\:\spc{X}\to \spc{Y}$ such that 
\[\dist{x}{x'}{\spc{X}}\le \dist{f(x)}{f(x')}{\spc{Y}}+\eps\]
for any $x,x'\in \spc{X}$.

\begin{subthm}{ex:GH-po:a}
Show that 
$$\dist{\spc{X}}{\spc{Y}}{\GH'}=\inf\set{\eps>0}{\spc{X}\le \spc{Y}+\eps
\quad\text{and}\quad
\spc{Y}\le \spc{X}+\eps}$$
defines a metric on the space of (isometry classes) of compact metric spaces.
\end{subthm}

\begin{subthm}{ex:GH-po:b}
Moreover $\dist{*}{*}{\GH'}$ is equivalent to the Gromov--Hausdorff metric;
that is,
$$|\spc{X}_n-\spc{X}_\infty|_{\GH}\to 0 
\quad\iff\quad 
\dist{\spc{X}_n}{\spc{X}_\infty}{\GH'}\to 0$$ 
as $n\to\infty$.
\end{subthm}
\end{thm}

\section{Universal ambient space}

Recall that a metric space is called universal if it contains an isometric copy of any separable metric space (in particular, any compact metric space).
Examples of universal spaces include Urysohn space and $\ell^\infty$ --- the space of bounded infinite sequences with the metric defined by $\sup$-norm; see \ref{prop:sep-in-urys} and \ref{ex:frechet}.

The following proposition says that the space $\spc{W}$ in Definition~\ref{def:GH} can be exchanged to a fixed universal space.

\begin{thm}{Proposition}\label{prop:GH-with-fixed-Z}
Let $\spc{U}$ be a universal space.
Then for any compact metric spaces $\spc{X}$ and $\spc{Y}$ we have
$$|\spc{X}-\spc{Y}|_{\GH} = \inf \{|\spc{X}'-\spc{Y}'|_{\Haus\spc{U}}\}$$ 
where the greatest lower bound is taken over all pairs of sets $\spc{X}'$ and $\spc{Y}'$ in $\spc{U}$
which isometric to  $\spc{X}$ and $\spc{Y}$ respectively.  
\end{thm}




\parit{Proof of \ref{prop:GH-with-fixed-Z}.}
By the definition (\ref{def:GH}), we have that 
\[|\spc{X}-\spc{Y}|_{\GH} \le \inf \{|\spc{X}'-\spc{Y}'|_{\Haus\spc{U}}\};\]
it remains to prove the opposite inequality.

Suppse $|\spc{X}-\spc{Y}|_{\GH}<\eps$;
let $\spc{X}'$, $\spc{Y}'$ and $\spc{Z}$ be as in \ref{def:GH}.
We can assume that $\spc{Z}=\spc{X}'\cup\spc{Y}'$;
otherwise pass to the subspace $\spc{X}'\cup\spc{Y}'$ of~$\spc{Z}$.
In this case, $\spc{Z}$ is compact;
in particular, it is separable.

Since $\spc{U}$ is universal, there is a distance-preserving embedding of $\spc{Z}$ in $\spc{U}$;
let us keep the same notation for $\spc{X}'$, $\spc{Y}'$, and their images.
It follows that 
\[|\spc{X}'-\spc{Y}'|_{\Haus\spc{U}}<\eps,\]
--- hence the result.
\qeds

\begin{thm}{Exercise}\label{ex:GH-urysohn}
Let $\spc{U}_\infty$ be the Urysohn space.
Given two compact set $A$ and $B$ in $\spc{U}_\infty$ define 
\[\|A-B\|=\inf\{|A-\iota(B)|_{\Haus\spc{U}_\infty}\},\]
where the greatest lower bound is taken for all isometrics $\iota$ of $\spc{U}_\infty$.
Show that $\|{*}\z-{*}\|$ defines a pseudometric%
\footnote{The value $\|A-B\|$ is called Hausdorff distance \emph{up to isometry} from $A$ to $B$ in $\spc{U}_\infty$.}
on nonempty compact subsets of $\spc{U}_\infty$ and its corresponding metric space is isometric to $\GH$.
\end{thm}

\section{Remarks}

Suppose $\spc{X}_n\GHto \spc{X}_\infty$, then there is a metric on the disjoint union 
\[\bm{X}=\bigsqcup_{n\in \NN\cup\{\infty\}} \spc{X}_n\] 
that satisfies the following property:

\begin{thm}{Property}\label{propery:GH}
The restriction of metric on each $\spc{X}_n$ and $\spc{X}_\infty$ coincides with its original metric 
and $\spc{X}_n\Hto \spc{X}_\infty$ as subsets in $\bm{X}$.
\end{thm}


Indeed, since $\spc{X}_n\GHto \spc{X}_\infty$, there is a metric on $\spc{V}_n=\spc{X}_n\sqcup \spc{X}_\infty$ such that the restriction of metric on each $\spc{X}_n$ and $\spc{X}_\infty$ coincides with its original metric and $\dist{\spc{X}_n}{\spc{X}_\infty}{\Haus\spc{V}_n}<\eps_n$ for some sequence $\eps_n\to 0$.
Gluing all $\spc{V}_n$ along $\spc{X}_\infty$, we obtain the required space $\bm{X}$.

In other words, the metric on $\bm{X}$ defines the convergence $\spc{X}_n\z\GHto \spc{X}_\infty$.
This metric makes it possible to talk about limits of sequences $x_n\in \spc{X}_n$ as $n\to\infty$, as well as weak limits of a sequence of Borel measures $\mu_n$ on $\spc{X}_n$ and so on.

For that reason, it is useful to define convergence by specifying the metric on $\bm{X}$ that satisfies the property
for the variation of Hausdorff convergence described in Section~\ref{sec:H-variation}.
This approach is very flexible;
in particular, it can be used to define Gromov--Hausdorff convergence of arbitrary metric spaces (net necessarily compact).

In this case, a limit space for this generalized convergence is not uniquely defined.
\begin{figure}[h!]
\vskip-0mm
\centering
\includegraphics{mppics/pic-500}
\end{figure}
For example, if each space $\spc{X}_n$ in the sequence is isometric to the half-line, then its limit might be isometric to the half-line or the whole line.
The first convergence is evident and the second could be guessed from the diagram.



Often the isometry class of the limit can be fixed by marking a point $p_n$ in each space $\spc{X}_n$, it is called \index{pointed convergence}\emph{pointed Gromov--Hausdorff convergence} --- we say that $(\spc{X}_n,p_n)$ converges to $(\spc{X}_\infty,p_\infty)$ if there is a metric on $\bm{X}$ such that $\spc{X}_n\Hto \spc{X}_\infty$ and $p_n\to p_\infty$.
For example, the sequence $(\spc{X}_n,p_n)=(\RR_+,0)$ converges to $(\RR_+,0)$, while $(\spc{X}_n,p_n)=(\RR_+,n)$ converges to $(\RR,0)$.

The pointed convergence works nicely only for proper metric spaces;
the following theorem is an analog of Gromov's selection theorem for this convergence.

\begin{thm}{Theorem}\label{thm:pointed-gromov-compactness}%
Let $\spc{Q}$ be a set of isometry classes of pointed proper metric spaces
$(\spc{X},p)$.
Assume that for any $R>0$, the $R$-balls in the spaces centered at the marked points form a uniformly totally bounded family of spaces.
Then $\spc{Q}$ is precompact with respect to pointed Gromov--Hausdorff convergence. 
\end{thm}







\chapter{Ultralimits}

Ultralimits provide a very general way to pass to a limit.
This procedure works for \textit{any} sequence of metric spaces, its result reminds limit in the sense of Gromov--Hausdorff, but has some strange features; for example, the limit of a constant sequence of spaces $\spc{X}_n=\spc{X}$ is \textit{not} $\spc{X}$ (see \ref{ex:ultrapower:compact}).

In geometry, ultralimits are used only as a canonical way to pass to a convergent subsequence.
It is useful in the proofs where one needs to repeat ``pass to convergent subsequence'' too many times.

This lecture is based on the introductory part of the paper by Bruce Kleiner and Bernhard Leeb \cite{kleiner-leeb}.

\section{Faces of ultrafilters}

Recall that $\NN$ denotes the set of natural numbers, $\NN=\{1,2,\dots\}$

\begin{thm}{Definition}
A finitely additive measure $\omega$ 
on $\NN$ 
is called an \index{ultrafilter}\emph{ultrafilter} if it satisfies the following condition:
\begin{subthm}{}
$\omega(\NN)=1$ and 
$\omega(S)=0$ or $1$ for any subset $S\subset \NN$.
\end{subthm}
An ultrafilter $\omega$ is called 
\emph{nonprincipal}\index{ultrafilter!nonprincipal ultrafilter}\index{nonprincipal ultrafilter} if in addition 
\begin{subthm}{}
$\omega(F)=0$ for any finite subset $F\subset \NN$.
\end{subthm}
\end{thm}

If $\omega(S)=0$ for some subset $S\subset \NN$,
we say that $S$ is \index{$\omega$-small}\emph{$\omega$-small}. 
If $\omega(S)=1$, we say that $S$ contains \index{$\omega$-almost all}\emph{$\omega$-almost all} elements of $\NN$.

\begin{thm}{Advanced exercise}\label{ex:ultrakatetov}
Let $\omega$ be an ultrafilter on $\NN$ and $f\:\NN\z\to \NN$.
Suppose that $\omega(S)\le \omega(f^{-1}(S))$ for any set $S\subset \NN$.
Show that $f(n)=n$ for $\omega$-almost all $n\in\NN$.
\end{thm}


\parbf{Classical definition.}
More commonly, a nonprincipal ultrafilter is defined as a collection, say $\mathfrak{F}$, of sets in $\NN$ such that
\begin{enumerate}
\item\label{filter:supset} if $P\in \mathfrak{F}$ and $Q\supset P$, then $Q\in \mathfrak{F}$,
\item\label{filter:cap} if $P, Q\in \mathfrak{F}$, then $P\cap Q\in \mathfrak{F}$,
\item\label{filter:ultra} for any subset $P\subset\NN$, either $P$ or its complement is an element of $\mathfrak{F}$.
\item\label{filter:non-prin} if $F\subset \NN $ is finite, then $F\notin \mathfrak{F}$.
\end{enumerate}
Setting $P\in\mathfrak{F}\Leftrightarrow\omega(P)=1$ makes these two definitions equivalent.

A nonempty collection of sets $\mathfrak{F}$ that does not include the empty set and satisfies only conditions \ref{filter:supset} and \ref{filter:cap} is called a \index{filter}\emph{filter}; 
if in addition $\mathfrak{F}$ satisfies condition \ref{filter:ultra} it is called an \index{ultrafilter}\emph{ultrafilter}.
From Zorn's lemma, it follows that every filter contains an ultrafilter.
Thus there is an ultrafilter $\mathfrak{F}$ contained in the filter of all complements of finite sets; clearly, this ultrafilter $\mathfrak{F}$ is nonprincipal.


\parbf{Stone--\v{C}ech compactification.}
Given a set $S\subset \NN$, consider subset $\Omega_S$ of all ultrafilters $\omega$ such that $\omega(S)=1$.
It is straightforward to check that the sets $\Omega_S$ for all $S\subset \NN$ form a topology on the set of ultrafilters on $\NN$. 
The obtained space was first considered by Andrey Tikhonov and called \index{Stone--\v{C}ech compactification}\emph{Stone--\v{C}ech compactification} of $\NN$;
it is usually denoted as $\beta\NN$.

Let $\omega_n$ denotes the principal ultrafilter such that $\omega_n(\{n\})=1$; that is, $\omega_n(S)=1$ if and only if $n\in S$.
Note that $n\mapsto\omega_n$ defines a natural embedding $\NN\hookrightarrow\beta\NN$. 
Using the described embedding, we can (and will) consider $\NN$ as a subset of $\beta\NN$.

The space $\beta\NN$ is the maximal compact Hausdorff space that contains $\NN$  as an everywhere dense subset.
More precisely, for any compact Hausdorff space $\spc{X}$ 
and a map $f\:\NN\to \spc{X}$ there is a unique continuous map $\bar f\:\beta\NN\to X$ such that the restriction $\bar f|_\NN$ coincides with $f$. 

\section{Ultralimits of points}
\label{ultralimits}\index{ultralimit}

Let us fix a nonprincipal  ultrafilter $\omega$ once and for all.

Assume $x_n$ is a sequence of points in a metric space $\spc{X}$. 
Let us define the \index{$\omega$-limit}\emph{$\omega$-limit} of a sequence $x_1,x_2,\dots$ as the point $x_\omega$ 
such that for any $\eps>0$, point $x_n$ lie in $\oBall(x_\omega,\eps)$ for $\omega$-almost all $n$; 
that is,
\[\omega\set{n\in\NN}{\dist{x_\omega}{x_n}{}<\eps}=1.\]
In this case, we will write 
\[x_\omega=\lim_{n\to\omega} x_n
\ \ \text{or}\ \ 
x_n\to x_\omega\ \text{as}\ n\to\omega.\]

For example, if $\omega_n$ is the \textit{principal} ultrafilter such that $\omega_n\{n\}=1$ for some $n\in\NN$, then
$x_{\omega_n}=x_n$.

The sequence $x_n$ can be regarded as a map $\NN\to\spc{X}$ defined by $n\mapsto x_n$.
If $\spc{X}$ is compact, then the map $\NN\to\spc{X}$ can be extended to a continuous map $\beta\NN\to\spc{X}$ from the Stone--\v{C}ech compactification $\beta\NN$ of $\NN$.
Then the $\omega$-limit $x_\omega$ can be regarded as the image of $\omega$.

Note that the $\omega$-limits of a sequence and its subsequence may differ.
For example, sequence $y_n=-(-1)^n$ is a subsequence of $x_n=(-1)^n$, but for any ultrafilter $\omega$, we have
\[\lim_{n\to\omega}x_n
\ne
\lim_{n\to\omega}y_n.\] 

\begin{thm}{Proposition}\label{prop:ultra/partial}
Let $\omega$ be a nonprincipal ultrafilter.
Assume $x_n$ is a sequence of points in a metric space $\spc{X}$
and $x_n\to  x_\omega$ as $n\to\omega$.
Then $x_\omega$ is a partial limit of the sequence $x_n$;
that is, there is a subsequence $(x_n)_{n\in S}$ that converges to $x_\omega$ in the usual sense.
\end{thm}

\parit{Proof.}
Given $\eps>0$, 
set $S_\eps=\set{n\in\NN}{\dist{x_n}{x_\omega}{}<\eps}$.

Note that $\omega(S_\eps)=1$ for any $\eps>0$.
Since $\omega$ is nonprincipal, the set $S_\eps$ is infinite.
Therefore, we can choose an increasing sequence $n_k$
such that $n_k\in S_{\frac1k}$ for each $k\in \NN$.
Clearly, $x_{n_k}\to x_\omega$ as $k\to\infty$.
\qeds

The following proposition 
is analogous to the statement that any sequence in a compact metric space 
has a convergent subsequence;
it can be proved the same way.

\begin{thm}{Proposition}\label{prop:ultra/compact}
Let $\spc{X}$ be a compact metric space.
Then
any sequence $x_n$ of points in $\spc{X}$ has a unique $\omega$-limit $x_\omega$.

In particular, a bounded sequence of real numbers has a unique $\omega$-limit.
\end{thm}

The following lemma is an ultralimit analog of the Cauchy convergence test.

\begin{thm}{Lemma}\label{lem:X-X^w}
Let $x_n$ be a sequence of points in a complete space~$\spc{X}$. 
Assume for each subsequence $y_n$ of $x_n$, 
the $\omega$-limit 
\[y_\omega=\lim_{n\to\omega}y_{n}\in \spc{X}\]
is defined and does not depend on the choice of subsequence, then the sequence $x_n$ converges in the usual sense.
\end{thm}

\parit{Proof.} If $x_n$ is not a Cauchy sequence, then for some $\eps>0$, there is a subsequence $y_n$ of $x_n$ such that $\dist{x_n}{y_n}{}\ge\eps$ for all $n$.

It follows that $\dist{x_\omega}{y_\omega}{}\ge \eps$ --- a contradiction.\qeds

\begin{thm}{Exercise}\label{ex:linear}
Recall that $\ell^\infty$ denotes the space of bounded sequences of real numbers.
Show that there is a linear functional $L\:\ell^\infty\to\RR$ such that
for any sequence $\bm{s}=(s_1,s_2,\dots)\in \ell^\infty$ the image $L(\bm{s})$ is a partial limit of $s_1,s_2,\dots$
\end{thm}

\begin{thm}{Exercise}\label{ex:ultrakatetov+}
Suppose that $f\:\NN\to\NN$ is a map such that 
\[\lim_{n\to\omega}x_n=\lim_{n\to\omega}x_{f(n)}\]
for any bounded sequence $x_n$ of real numbers.
Show that $f(n)=n$ for $\omega$-almost all $n\in\NN$.
\end{thm}

\section{An illustration}

\begin{thm}{Claim}
Let $\spc{X}$ and $\spc{Y}$ be compact spaces.
Suppose that for every $n\in\NN$ there is a $\tfrac1n$-isometry $f_n\:\spc{X}\to \spc{Y}$.
Then there is an isometry $\spc{X}\to \spc{Y}$.
\end{thm}

We give a proof of this claim only as an illustration for ulralimits.

\parit{Proof.}
Consider the $\omega$-limit $f_\omega$ of~$f_n$;
according to \ref{prop:ultra/compact}, $f_\omega$ is defined.
Since 
\[|f_n(x)-f_n(x')|\lege |x-x'|\pm\tfrac1n\]
we get that 
\[|f_\omega(x)-f_\omega(x')|= |x-x'|\]
for any $x,x'\in \spc{X}$;
that is, $f_\omega$ is distance-preserving.

Further, since $f_n$ is a $\tfrac1n$-isometry,
for any $y\in \spc{Y}$ there is a sequence $x_n\in \spc{X}$ such that $|f_n(x_n)-y|\le \tfrac1n$.
Therefore,
\[f_\omega(x_\omega)=y,\]
where $x_\omega$ is the $\omega$-limit of $x_n$;
that is, $f_\omega$ is onto.

It follows that $f_\omega\:\spc{X}\to\spc{Y}$ is an isometry.
\qeds

\section{Ultralimits of spaces}\label{sec:Ultralimit of spaces}

Recall that $\omega$ denotes a nonprincipal ultrafilter on the set of natural numbers.

Let $\spc{X}_n$ be a sequence of metric spaces.
Consider all sequences of points $x_n\in \spc{X}_n$.
On the set of all such sequences,
define a semimetric by
\[\dist{(x_n)}{(y_n)}{}
=
\lim_{n\to\omega} \dist{x_n}{y_n}{\spc{X}_n}.
\eqlbl{eq:olim-dist}\]
Note that the $\omega$-limit on the right-hand side is always defined 
and takes a value in $[0,\infty]$. 
(The $\omega$-convergence to $\infty$ is defined analogously to the usual convergence to $\infty$).

Set $\spc{X}_\omega$ to be the corresponding metric space; 
that is, the underlying set of $\spc{X}_\omega$ is formed by classes of equivalence of sequences of points $x_n\in\spc{X}_n$ 
defined by 
\[(x_n)\sim(y_n)
\ \Leftrightarrow\ 
\lim_{n\to\omega} \dist{x_n}{y_n}{}=0\]
and the distance is defined by \ref{eq:olim-dist}.

The space $\spc{X}_\omega$ is called the \index{$\omega$-limit space}\emph{$\omega$-limit} of $\spc{X}_n$.
(It is also called \index{$\omega$-product}\emph{$\omega$-product}; this term is motivated by the fact that   
$\spc{X}_\omega$ is a quotient of the product $\prod\spc{X}_n$)
Typically  $\spc{X}_\omega$ will denote the  
$\omega$-limit of sequence $\spc{X}_n$;
we may also write  
\[\spc{X}_n\to\spc{X}_\omega\ \ \text{as}\ \  n\to\omega\ \ \text{or}\ \ \spc{X}_\omega=\lim_{n\to\omega}\spc{X}_n.\]

Given a sequence $x_n\in \spc{X}_n$,
we will denote by $x_\omega$ its equivalence class which is a point in $\spc{X}_\omega$;
it can be written as
\[x_n\to x_\omega \ \ \text{as}\ \  n\to\omega,\ \ \text{or}\ \ x_\omega=\lim_{n\to\omega} x_n.\]

\begin{thm}{Observation}\label{obs:ultralimit-is-complete}
The $\omega$-limit of any sequence of metric spaces is complete. 
\end{thm}

We will repeat the proof of \ref{ex:complete-completion} using a slightly different language.

\parit{Proof.}
Let $\spc{X}_n$ be a sequence of metric spaces and $\spc{X}_n\to\spc{X}_\omega$ as $n\to\omega$.
Choose a Cauchy sequence $x_1,x_2,\dots{}\in\spc{X}_\omega$.
Passing to a subsequence, we can assume that $\dist{x_k}{x_{m}}{\spc{X}_\omega}<\tfrac1{k}$ if $k<m$.

Choose a double sequence $x_{n,m}\in \spc{X}_n$ such that for any fixed $m$ we have $x_{n,m}\to x_m$ as $n\to\omega$.
Note that for any $k<m$ the inequality $\dist{x_{n,k}}{x_{n,m}}{}<\tfrac1{k}$ holds for $\omega$-almost all $n$.

Given $m\in\NN$, consider the subset $S_m\subset\NN$ defined by
\[S_m=\set{n\ge m}{\dist{x_{n,k}}{x_{n,l}}{}<\tfrac1{k} \quad\text{for all}\quad k<l\le m}.\]
Note that 
\begin{itemize}
\item $\NN= S_1\supset S_2\supset\dots$
\item $\omega(S_m)=1$ for each $m$, and
\item $\min S_m\ge m$.
\end{itemize}

Consider the sequence $y_n=x_{n,m(n)}$, where $m(n)$ is the largest value such that $n\in S_{m(n)}$;
from above, $m(n)\le n$.
Denote by $y_\omega\in \spc{X}_\omega$ the $\omega$-limit of $y_n$.

Observe that $|y_m-x_{n,m}|<\tfrac1{m}$ for $\omega$-almost all $n$.
It follows that $|x_m-y_\omega|\le \tfrac1{m}$ for any $m$.
Therefore, $x_n\to y_\omega$ as $n\to \infty$.
That is, any Cauchy sequence in $\spc{X}_\omega$ converges.
\qeds

\begin{thm}{Observation}\label{obs:ultralimit-is-geodesic}
The $\omega$-limit of any sequence of length spaces is geodesic. 
\end{thm}

\parit{Proof.}
If $\spc{X}_n$ is a sequence of length spaces, then for any sequence of pairs $x_n, y_n\in X_n$ there is a sequence of $\tfrac1n$-midpoints $z_n$.

Let $x_n\to x_\omega$, $y_n\to y_\omega$ and $z_n\to z_\omega$ as $n\to \omega$.
Note that $z_\omega$ is a midpoint of $x_\omega$ and $y_\omega$ in $\spc{X}_\omega$.

By Observation~\ref{obs:ultralimit-is-complete}, $\spc{X}_\omega$ is complete.
Applying Lemma~\ref{lem:mid>geod} we get the statement.
\qeds


\begin{thm}{Exercise}\label{ex:lim(tree)}
Show that an ultralimit of metric trees is a metric tree. 
\end{thm}

\begin{thm}{Exercise}\label{ex:ultracompact}
Suppose that $\spc{X}_\infty$ and $\spc{X}_1,\spc{X}_2,\dots$ are compact metric spaces.
Assume $\spc{X}_n\GHto\spc{X}_\infty$.
Show that $\spc{X}_\omega\iso\spc{X}_\infty$.
\end{thm}


\section{Ultrapower}

If all the metric spaces in the sequence are identical $\spc{X}_n=\spc{X}$, 
its $\omega$-limit 
$\lim_{n\to\omega}\spc{X}_n$
is denoted by $\spc{X}^\omega$
and called \index{ultrapower}\index{$\omega$-power}\emph{$\omega$-power} of $\spc{X}$ (also known as \index{ultracompletion}\index{$\omega$-completion}\emph{$\omega$-completion}).



\begin{thm}{Exercise}\label{ex:ultrapower}
For any point $x\in \spc{X}$, consider the constant sequence $x_n=x$
and set $\iota(x)=\lim_{n\to\omega}x_n\in\spc{X}^\omega$.

\begin{subthm}{ex:ultrapower:a}
Show that $\iota\:\spc{X}\to\spc{X}^\omega$ is distance-preserving embedding. (So we can and will consider $\spc{X}$ as a subset of $\spc{X}^\omega$.)
\end{subthm}

\begin{subthm}{ex:ultrapower:compact} 
Show that $\iota$ is onto if and only if $\spc{X}$ is compact.
\end{subthm}

\begin{subthm}{ex:ultrapower:proper} 
Show that if $\spc{X}$ is proper, then $\iota(\spc{X})$ forms a metric component of $\spc{X}^\omega$; that is, a subset of $\spc{X}^\omega$ that lies at a finite distance from a given point.
\end{subthm}

\end{thm}

Note that \ref{SHORT.ex:ultrapower:compact} implies that the inclusion $\spc{X}\hookrightarrow\spc{X}^\omega$ is not onto if the space $\spc{X}$ is not compact.
However, the spaces $\spc{X}$ and $\spc{X}^\omega$ might be isometric; here is an example:

\begin{thm}{Exercise}\label{ex:isom-ultrapower}
Let $\spc{X}$ be an infinite countable set with discrete metric;
that is $\dist{x}{y}{\spc{X}}=1$ if $x\ne y$.
Show that 

\begin{subthm}{ex:isom-ultrapower:no}
$\spc{X}^\omega$ is not isometric to $\spc{X}$.
\end{subthm}

\begin{subthm}{ex:isom-ultrapower:yes}
$\spc{X}^\omega$ is  isometric to $(\spc{X}^\omega)^\omega$.
\end{subthm}

\end{thm}

\begin{thm}{Exercise}\label{ex:ultrapower(ultrapower)}
Given a nonprincipal ultrafilter $\omega$, construct an ultrafilter $\omega_1$ such that 
\[\spc{X}^{\omega_1}\iso(\spc{X}^\omega)^\omega\]
for any metric space~$\spc{X}$.

\end{thm}


\begin{thm}{Observation}\label{obs:ultrapower-is-geodesic}
Let $\spc{X}$ be a complete metric space. 
Then $\spc{X}^\omega$ is geodesic space if and only if $\spc{X}$ is a length space.
\end{thm}

\parit{Proof.}
The if part follows from \ref{obs:ultralimit-is-geodesic}; it remains to prove the only-if part

Assume $\spc{X}^\omega$ is geodesic space.
Then any pair of points $x,y\in \spc{X}$ has a midpoint $z_\omega\in\spc{X}^\omega$.
Fix a sequence of points $z_n\in  \spc{X}$ such that $z_n\to z_\omega$ as $n\to \omega$.

Note that 
$\dist{x}{z_n}{\spc{X}}\to \tfrac12\cdot \dist{x}{y}{\spc{X}}$
and 
$\dist{y}{z_n}{\spc{X}}\to \tfrac12\cdot \dist{x}{y}{\spc{X}}$
as 
$n\to\omega$.
In particular, for any $\eps>0$, the point $z_n$ is an $\eps$-midpoint of $x$ and $y$ for $\omega$-almost all $n$.
It remains to apply \ref{lem:mid>geod}.
\qeds

\begin{thm}{Exercise}\label{ex:two-geodesics-in-ultrapower}
Assume $\spc{X}$ is a complete length space 
and $p,q\in\spc{X}$ cannot be joined by a geodesic in $\spc{X}$.  
Show that there are at least continuum of distinct geodesics between $p$ and $q$ 
in the ultrapower $\spc{X}^\omega$.
\end{thm}

\begin{thm}{Exercise}\label{ex:notproper-limit}
Construct a proper metric space $\spc{X}$ such that $\spc{X}^\omega$ is not proper;
that is, there is a point $p\in \spc{X}^\omega$ and $R<\infty$ such that the closed ball $\cBall[p,R]_{\spc{X}^\omega}$ is not compact.
\end{thm}

\section{Tangent and asymptotic spaces}
\label{sec:tan+asymptotic}

Choose a space $\spc{X}$ and a sequence $\lambda_n$ of positive numbers.
Consider the sequence of \index{rescaled space}\emph{rescalings} $\spc{X}_n=\lambda_n\cdot\spc{X}=(\spc{X},\lambda_n\cdot\dist{*}{*}{\spc{X}})$.

Choose a point $p\in \spc{X}$ and denote by $p_n$ the corresponding point in $\spc{X}_n$.
Consider the $\omega$-limit $\spc{X}_\omega$ of $\spc{X}_n$ (one may denote it by $\lambda_\omega\cdot \spc{X}$);
set $p_\omega$ to be the $\omega$-limit of $p_n$.

If $\lambda_n\to \infty$ as $n\to\omega$, then the metric component of $p_\omega$ in $\spc{X}_\omega$ is called \index{$\lambda_\omega$-tangent space}\emph{$\lambda_\omega$-tangent space} at $p$ and denoted by $\T_p^{\lambda_\omega}\spc{X}$ (or $\T_p^{\omega}\spc{X}$ if $\lambda_n=n$).\label{page:ultratangent space}

If $\lambda_n\to 0$ as $n\to\omega$, then the metric component of $p_\omega$ is called \index{$\lambda_\omega$-asymptotic space}\emph{$\lambda_\omega$-asymptotic space}%
\footnote{Often it is called an {}\emph{asymptotic cone} despite that it is not a cone in general; this name is used since in good cases it has a cone structure.} and denoted by $\Asym\spc{X}$ or $\Asym^{\lambda_\omega}\spc{X}$.
Note that the space $\Asym\spc{X}$ and its point $p_\omega$ do not depend on the choice of $p\in \spc{X}$.

The following exercise states that the constructions above depend on the sequence $\lambda_n$ and a nonprincipal ultrafilter $\omega$.

\begin{thm}{Exercise}\label{ex:ultraT}
Construct a metric space $\spc{X}$ with a point $p$ such that the tangent space
$\T_p^{\lambda_\omega}\spc{X}$ (or its asymptotic cone $\Asym^{\lambda_\omega}\spc{X}$) depends on the sequence $\lambda_n$ and/or ultrafilter~$\omega$.
\end{thm}

For nice spaces, different choices of the sequence of coefficients and ultrafilter may give the same space; 
some examples are given in the following exercise.

\begin{thm}{Exercise}\label{ex:Asym(Lob)}
Let $\spc{T}=\Asym\spc{L}$, where $\spc{L}$ is the Lobachevsky plane, or Lobachevsky space, or 3-regular%
\footnote{that is, the degree of any vertex is 3.}
metric tree with unit edge length (choose your favorite space from these three).

\begin{subthm}{ex:Asym(Lob):metric-tree}
Show that $\spc{T}$ is a complete metric tree.
\end{subthm}

\begin{subthm}{ex:Asym(Lob):homogeneous}
Show that $\spc{T}$ is one-point-homogeneous; that is, given two points $s,t\in \spc{T}$ there is an isometry of $\spc{T}$ that maps $s$ to $t$.
\end{subthm}

\begin{subthm}{ex:Asym(Lob):continuum}
Show that $\spc{T}$ has \index{degree}\emph{continuum degree} at any point;
that is, for any point $t\in \spc{T}$ the set of connected components of the complement $\spc{T}\setminus\{t\}$ has cardinality continuum.
\end{subthm}

\end{thm}


\begin{thm}{Exercise}\label{ex:T(Sx[0,1]/Sx0)}
Consider the quotient space $\spc{X}=\mathbb{S}^1\times[0,1]/\mathbb{S}^1\times\{0\}$;
that is, 
$\dist{(u_1,t_1)}{(u_2,t_2)}{\spc{X}}
=
\min\{\,\dist{(u_1,t_1)}{(u_2,t_2)}{\mathbb{S}^1\times[0,1]},t_1+t_2\,\}$.
Describe the ultratangent space $\T_o^{\omega}\spc{X}$, where $o\in\spc{X}$ is the point that corresponds to $\mathbb{S}^1\times\{0\}$.
\end{thm}


\section{Remarks}

A nonprincipal ultrafilter $\omega$ is called 
\emph{selective}\index{ultrafilter!selective ultrafilter}\index{selective ultrafilter} if for any partition of $\NN$ into sets $\{C_\alpha\}_{\alpha\in\IndexSet}$ such that $\omega(C_\alpha)\z=0$ for each $\alpha$, 
there is a set $S\subset \NN$ such that $\omega(S)=1$ and $S\cap C_\alpha$ is a one-point set for each $\alpha\in\IndexSet$.

The existence of a selective ultrafilter follows from the continuum hypothesis \cite{rudin}.

If needed, we may assume that the chosen ultrafilter $\omega$ is selective.
In this case \textit{the subsequence $(x_n)_{n\in S}$ in \ref{prop:ultra/partial} can be chosen so that $\omega(S)=1$}.

%\chapter{Metric plus measure}

\section{Borel sets}

Let us remind few definitions assuming knowleage of basic measure theory;
comprehensive treatments can be found in \cite{billingsley} and \cite{bogachev}.

Let $\spc{X}$ be a metric space.
\index{Borel set}\emph{Borel set} is any subset of $\spc{X}$ that can be formed from open sets using the countable union, countable intersection, and complement.
In other words, Borel sets form the minimal sigma-algebra that included open sets.

A measure on metric space will be always assumed to be \index{Borel measure}\emph{Borel};
that is, it is defined on the sigma-algebra of Borel sets.
A Borel measure can be uniquely determined by its values on all open (or closed) sets.

A measure $\mu$ on $\spc{X}$ is called \index{probability measure}\emph{probability measure} if $\mu\spc{X}=1$.

Recall that \index{delta-measure}\emph{delta-measure} is a probability measure with support at one point.
Delta-measure with support in $\{x\}$ will be denoted by~\index{$\delta_{x}$}$\delta_{x}$; so
\[\text{if}\quad x\in A,\quad\text{then}\quad \delta_x(A)=1,\quad\text{otherwise}\quad\delta_x(A)=0.\]

Let $\mu_n$ be a sequence of Borel measures on $\spc{X}$.
A measure $\mu_\infty$ is a \index{weak limit}\emph{weak limit} of $\mu_n$ if 
\[\int_{\spc{X}}f\cdot(\mu_n-\mu_\infty)\to0\gamma
\quad\text{as}\quad
n\to\infty
\]
for any continuous function $f\:\spc{X}\to \RR$.

Suppose $\mu$ is a measure on a metric space $\spc{X}$ and $f\:\spc{X}\to \spc{Y}$ is a measurable map;
that is, for any Borel set $B\subset \spc{Y}$, its inverse image $f^{-1}B$ is a Borel set in $\spc{Y}$.

Consider the unit interval with its Lebesgue mesure.
If $\spc{X}$ is a complete separable metric space with probability measure $\mu$, then there is a measurable map $[0,1]\to \spc{X}$

\section{Metric on measures}

Imagine that we need to transport dirt from one pile of a given shape to make another pile of a needed shape.
Suppose that cost of transporting a unit of dirt equals to the traveled distance.%
\footnote{This is the simplest cost function one can imagine.
One may consider other cost functions; for example, if the cost proportional to the square of the distance, then the problem has more applications.}
We are free to choose a destination point for dirt from a given place.
How to minimize the total cost?

To formalize this question,
suppose that the piles of dirt are described by Borel probability measures $\mu$ and $\nu$ on a metric space~$\spc{X}$.

To describe where each piece of dirt goes, we will use the so-called \index{plan}\emph{plan} for $\mu$ and $\nu$.
It is a probability measure $\pi$ on the product $\spc{X}\times\spc{X}$ such that 
for all measurable sets $A \subset \spc{X}$, we have 
\[\mu A= \pi [A \times \spc{X}],
\quad\text{and}\quad
\nu A=\pi [\spc{X}\times A].
\eqlbl{eq:marginals}\]
Equivalently it can be described as a measure that satisfies the following identity
\begin{align*}
\int_{(x,y)\in \spc{X}\times\spc{X}}f(x)\cdot g(y) \cdot \pi
&=
\int_{x\in \spc{X}}f(x)\cdot \mu
\oldcdot \int_{y\in \spc{X}}g(y)\cdot \nu,
\end{align*}
for any continuous functions $f,g\:\spc{X}\to \RR$.

Given a measure $\pi$ on $\spc{X}\times\spc{X}$, the measures $\mu$ and $\nu$ defined by \ref{eq:marginals} are called first and second \index{marginal}\emph{marginals} of $\pi$;
so the statement \textit{$\pi$ is a plan for $\mu$ and $\nu$} is equivalent to \textit{$\mu$ and $\nu$ are the first and second marginals of $\pi$ respectively}.

\begin{thm}{Claim}\label{clm:plan-exists}
There is a plan $\pi$ for any two given Borel probability measures $\mu$ and $\nu$.
\end{thm}

The plan constructed in the proof distributes equally each piece of dirt in the new pile.
As we will see this plan is usually far from optimum.

\parit{Proof.}
Consider the measure $\pi$ that is uniquely defined  defined by the identity
\[\pi(A\times B)=\mu A\cdot \mu B\]
for any Borel subsets $A,B\subset\spc{X}$.
Observe that $\pi$ is a plan for $\mu$ and~$\nu$.
\qeds

Denote by $\Pi(\mu,\nu)$ the set of all plans for $\mu$ and $\nu$;
by \ref{clm:plan-exists}, $\Pi(\mu,\nu)\z\ne\emptyset$.
It is straightforwrd to check that the following formula defines a metric on the space of probability measures on $\spc{X}$.
\[\dist{\mu}{\nu}{\Wass_1\spc{X}}
\df
\inf_{\pi\in\Pi(\mu,\nu)}
\left\{\,\int_{(x,y)\in\spc{X}\times\spc{X}}\dist{x}{y}{\spc{X}}\cdot\pi\,\right\}.\]
This metric is called \index{Wasserstein distance}\emph{Wasserstein distance of order 1} between $\mu$ and $\nu$.

In genereral, the Wasserstein distance $\dist{\mu}{\nu}{}$ might take infinite value, but all measures with compact support lie on finite distance from each other in the obtained $\infty$-metric space.
The metric component of these measures is called \index{Wasserstein space}\emph{Wasserstein space} of order 1 over $\spc{X}$; 
it is denoted by $\Wass_1\spc{X}$.
In other words, $\Wass_1\spc{X}$ is the space of all Borel probability measures $\mu$ such that 
$\int\distfun_p\cdot\mu<\infty$ for some (and therefore any) point $p\in \spc{X}$.

\begin{thm}{Exercise}\label{ex:wasserstein-infty}
Construct two Borel probability measures $\mu$ and $\nu$ on $\RR$ with Wasserstein distance $\dist{\mu}{\nu}{}=\infty$.
\end{thm}


\begin{thm}{Exercise}\label{ex:wasserstein-compact}
Show that $\Wass_1\spc{X}$ is a compact if and only if so is~$\spc{X}$.
\end{thm}

\begin{thm}{Exercise}\label{ex:wasserstein-length}
Show that the Wasserstein space $\Wass_1\spc{X}$ is a geodesic space for any metric space $\spc{X}$.
\end{thm}

\section{Optimal plan}

A plan $\pi$ for given measures $\mu$ and $\nu$ is called \index{optimal plan}\emph{optimal} if 
\[\dist{\mu}{\nu}{\Wass_1\spc{X}}
=\int_{(x,y)\in\spc{X}\times\spc{X}}\dist{x}{y}{\spc{X}}\cdot\pi.\]

\begin{thm}{Theorem} %Vilani:Theorem 1.4
Let $\mu$ and $\nu$ be probability Borel measures on a compact metric space $\spc{X}$.
Then there is an optimal plan $\pi$ for $\mu$ and~$\nu$.
\end{thm}

\parit{Proof.}
By the definition of Wasserstein distance, we can choose a sequence of plans $\pi_n$ for $\mu$ and $\nu$ such that 
\[\int_{(x,y)\in\spc{X}\times\spc{X}}\dist{x}{y}{\spc{X}}\cdot\pi_n\to \dist{\mu}{\nu}{\Wass_1\spc{X}}\]
as $n\to \infty$.

Observe that $\pi_n$ has a weak partial limit, say $\pi$.
Moreover, $\pi$ is an optimal plan for $\mu$ and $\nu$.
\qeds

\begin{thm}{Theorem}
Any optimal plan $\pi$ is \index{cyclic monotonicity}\emph{cyclically monotonic}.
That is, suppose $\pi$ is an optimal plan for probability measures $\mu$ and $\nu$ on a metric space $\spc{X}$.
Then any sequence of pairs $(x_1,y_1),\dots,(x_n,y_n)\in\supp\pi\subset\spc{X}\times\spc{X}$ we have
\[\sum_i\dist{x_i}{y_i}{}
\le
\sum_i\dist{x_{i+1}}{y_i}{},\]
here the index $i$ in the sum is taken modulo $n$; in particular $x_{n+1}\z=x_1$.
\end{thm}

\parit{Proof.}
Assume that the cyclic monotonicity does not hold;
that is,
\[R=\sum_i\dist{x_i}{y_i}{}
-
\sum_i\dist{x_{i+1}}{y_i}{}>0,\]
for some $(x_0,y_0),\dots,(x_n,y_n)\in\supp\pi$.
We need to show that $\pi$ is not optimal;
in other words we need to construct another plan $\pi'$ for $\mu$ and $\nu$ such that 
\[\int_{(x,y)\in\spc{X}\times\spc{X}}\dist{x}{y}{\spc{X}}\cdot(\pi'-\pi)<0.\eqlbl{pi'<pi}\]

Assume $\spc{X}$ is finite.
In this case we can choose $\eps>0$ such that 
$\pi\{(x_i,y_i)\}>\eps$ for each $i$.
Let
\[\pi'=\pi-\eps\cdot\sum_i(\sigma_i-\sigma_i')\eqlbl{eq:pi'}\]
where $\sigma_i=\delta_{(x_i,y_i)}$ and $\sigma_i'=\delta_{(x_{i+1},y_i)}$.
It remains to observe that $\pi'$ is a plan for $\mu$ and $\nu$ that satisfies \ref{pi'<pi}.

The general case is similar, we only need to redefine $\eps$, $\sigma_i$, and~$\sigma_i'$.
Note that given $r>0$, we can choose a probability measures $\sigma_i$ with support in $\oBall((x_i,y_i),r)_{\spc{X}\times\spc{X}}$ such that $\eps\cdot \sigma_i<\pi$ for some fixed $\eps>0$ and every $i$.
Further, denote by $\zeta_i$ and $\eta_i$ the first and second marginals of $\sigma_i$.
Observe that $\supp\zeta_i\subset\oBall(x_i,r)$ and $\supp\eta_i\subset\oBall(y_i,r)$ for all $i$.
Let $\sigma_i'$ be a plan for $\zeta_{i+1}$ and $\eta_i$.
Evidently 
\begin{align*}
\int_{(x,y)\in\spc{X}\times\spc{X}}\dist{x}{y}{}\cdot \sigma_i
&\lessgtr
\dist{x_i}{y_i}{}\pm 2\cdot r,
\\
\int_{(x,y)\in\spc{X}\times\spc{X}}\dist{x}{y}{}\cdot \sigma_i'
&\lessgtr
\dist{x_{i+1}}{y_i}{}\pm 2\cdot r.
\end{align*}
Taking $r<\tfrac R{10\cdot n}$, we get  \ref{pi'<pi}. 
\qeds




\section{Capitalistic approach}

Imagine that measures $\mu$ and $\nu$ describe the production and consumer of beer in the space.
A transportaition company transports beer from $\mu$ to $\nu$ and want to maximize its profit by adjusting price $f(x)$ of beer the point $x$; they buy beer at price $f(x)$ per unit, move it to an other point $y$ and sale it with (presumably higher) price $f(y)$.
However, the function $f$ is 1-Lipschitz condition;
otherwise the profit goes to second-hand dealers, or maybe it is a government regulation.
In other words, we need to maximize the following expression
\[\int_{\spc{X}} f\cdot(\mu-\nu)\]
for all $1$-Lipschitz functions $f$.
The maximal profit defines a metric

\begin{thm}{Theorem}
Let $\mu$ and $\nu$ be probability Borel measures on a compact metric space $\spc{X}$.
Then
\[\dist{\mu}{\nu}{\Wass_1\spc{X}}=\sup\int_{\spc{X}} f\cdot(\mu-\nu),\]
where the least upper bound is taken for all $1$-Lipschitz functions $f\:\spc{X}\z\to\RR$.
\end{thm}

The definition of Wassershtein metric described in the previous section reminds communist's planed economy.
The right-hand side in the above equation reminds capitalistic system.
Indeed, think that measures $\mu$ and $\nu$ describe the production and consumer of beer in the space.
A transportaition company trnasports beer from $\mu$ to $\nu$ and want to maximize its profit by adjusting price $f(x)$ of beer the point $x$.
However, the function $f$ is 1-Lipschitz condition --- this is a government regulation.




\parit{Proof.}
By the definition of Wasserstein metric, we can choose a sequence $\pi_n$ of plans  

Let us choose an optimal plan $\pi$ for $\mu$ and $\nu$; it exists by ???.
We need to find a 1-Lipschitz function $f\:\spc{X}\to\RR$ such that 
\[
\int_{\spc{X}} f\cdot(\mu-\nu)=\int_{(x,y)\in\spc{X}\times\spc{X}}\dist{x}{y}{\spc{X}}\cdot \pi.
\eqlbl{eq:f(mu-nu)}
\]

Choose $x_0\in \supp\mu$.
Note that adding a constant to $f$ does not change the left hand side in \ref{eq:f(mu-nu)}.
Therefore we can assume assume that $f(x_0)=0$ and set
\[f(x)=\sup\{\,|x_0-y_0|+\dots+|x_n-y_n|-(|x_1-y_0|+\dots+|x_n-y_{n-1}|)-|x-y_n|\,\}\]
where the least upper bound is taken for all sequences $(x_0,y_0),\z\dots,(x_n,y_n)\z\in\supp\pi$.

\qeds

\section{Metric-measure space}

A metric measure space is a metric $\spc{X}$ space with a choice of Borel probability measure $\vol$ on it.
In a metric-measure we ignore sets with vanishing volume; in other words, passing from $\spc{X}$ to the support of $\vol$ does not change the metric-measure space.

Alternatively we may start with unit interval $[0,1]$ equipped with Lebesgue measure and equip it with measurable semimetric $[0,1]\times [0,1]\to \RR$.





\section{Space of measures}


It can be equipped with the \index{Wasserstein metric}\emph{Wasserstein metric}
\[\dist{\mu}{\nu}{}\df\sup\left\{\,\int_{\spc{X}} f\cdot(\mu-\nu)\,\right\},\]
where the least upper bound is taken for all $1$-Lipschitz functions $f\:\spc{X}\to\RR$.

The Wasserstein distance $\dist{\mu}{\nu}{}$ might take infinite value, but all measures with compact support lie on finite distance from each other in the obtained $\infty$-metric space.
The metric component of these measures is called \index{Wasserstein space}\emph{Wasserstein space} of order 1 over $\spc{X}$; 
it is denoted by $\Wass_1\spc{X}$.



\section{Misc}

Suppose $\pi_n$ is a sequence of plans for $\mu$ and $\nu$.
Assume that $\pi_n$ weakly converges to a probability measure $\pi$ on $\spc{X}\times\spc{X}$.

is a weak limit of a sequence of plans $\pi_n$, then $\pi$ is a plan for $\mu$ and $\nu$ if for each $n$ $\pi_n$ is a plane for $\mu$ and $\nu$ 

Suppose that $f\:\spc{X}\to \RR$ is a 1-Lipschitz function,
so $f(x)-f(y)\le\dist{x}{y}{\spc{X}}$ for any $x,y\in \spc{X}$.
It follows that 
\begin{align*}
\int_{\spc{X}} f\cdot(\mu-\nu)&=\int_{x\in\spc{X}}f(x)\cdot\mu-\int_{y\in\spc{X}}f(y)\cdot\nu=
\\
&=\int_{(x,y)\in\spc{X}\times\spc{X}}[f(x)-f(y)]\cdot \pi\le
\\
&\le\int_{(x,y)\in\spc{X}\times\spc{X}}\dist{x}{y}{\spc{X}}\cdot \pi,
\end{align*}
where $\pi$ is a plan for $\mu$ and $\nu$.
By the definition of Wasserstein metric, we get  
\[\dist{\mu}{\nu}{\Wass_1\spc{X}}\le \int_{(x,y)\in\spc{X}\times\spc{X}}\dist{x}{y}{\spc{X}}\cdot\pi\eqlbl{wass=<int.plan}\]
for any plan $\pi$.

Next we want to show that equality holds in \ref{wass=<int.plan} for some plan $\pi$; such plans will be called \index{optimal plan}\emph{optimal}.


\parit{Proof.}
Choose a point $x_0\in \supp\mu$.
Given  $p\in \spc{X}$,
let
\[f(p)=\inf\left\{\sum_{i=0}^n\dist{x_i}{y_i}{}-\sum_{i=0}^n\dist{x_{i+1}}{y_i}{}-\dist{y_n}{p}{}\right\},
\eqlbl{eq:f(p)}\]
where the least upper bound is taken for all sequences of pairs 
\[(x_0,y_0),\z\dots,(x_n,y_n)\in \supp\pi.\eqlbl{eq:sequence}\]

Fix a sequence as in \ref{eq:sequence} and  denote by $F_\sigma(p)$ the expression under infimum in \ref{eq:f(p)}.

Let us show that 
\[F_\sigma(x_0)\ge 0.\]
Indeed, suppose $F_\sigma(x_0)<-\eps<0$.
Since $(x_i,y_i)\in \supp\pi$, we have $x_i\in\supp\mu$ and $y_i\in\supp\nu$ for any $i$.
Therefore we can choose sets $X_i\subset \oBall(x_i,\tfrac{\eps}{10\cdot n})$ and $Y_i\subset \oBall(y_i,\tfrac{\eps}{10\cdot n})$ such that 
$\mu(X_0)=\nu(Y_0)=\dots=\mu(X_n)=\nu(Y_n)$



Let us denote by $F(p)$ the expression under infimum in \ref{eq:f(p)}.
By the triangle inequality, 
\[F(q)\le F(p)+\dist{p}{q}{}.\]
Passing to the least upper bound in this inequality, we get
\[f(q)\le f(p)+\dist{p}{q}{}\]
for any $p,q\in\spc{X}$.
Hence $f$ is a 1-Lipschitz function.

Further, let us show that
\[(x,y)\in\supp\pi
\quad\Longrightarrow\quad
f(y)-f(x)=\dist{x}{y}{}\]





Suppose that cyclic monotonicity fails;
that is, there is a sequence of pairs $(x_1,y_1),\dots,(x_n,y_n)\in\spc{X}\times\spc{X}$ such that
\[\dist{x_1}{y_1}{}+\dots+\dist{x_n}{y_n}{}
>
\dist{x_1}{y_2}{}+\dots+\dist{x_{n-1}}{y_n}{}+\dist{x_{n}}{y_1}{}.\]
In this case, it would be more optimal to transport measure from a neighborhood of $x_i$ to a neighborhood of $y_{i+1}$ (
here and further we assume that the indexes are taken modulo $n$, so $n+1=1$).
The latter contradicts optimality of $\pi$.

The following argument makes it precise.
Choose small $\eps>0$.
For each $n$,
choose disjoint sets $X_i$ and $Y_i$ in $\eps$-neighborhood of $x_i$ and $y_i$
such that for some $\delta>0$ we have 
\[\pi [X_i\times Y_i]=\delta\]
for each $i$.

Let us modify the plan $\pi$ in the union $X_1\times Y_1 \cup\dots\cup X_n\times Y_n$ and such that 
$\pi'(X_i\times Y_{i+1})=\delta$ for each $i$;


Observe that
\[\int_{(x,y)\in\spc{X}\times\spc{X}}\dist{x}{y}{\spc{X}}\cdot(\pi'-\pi)>\]
\qeds


\backmatter

\newgeometry{top=0.9in, bottom=0.9in,inner=0.5in, outer=0.5in}
\chapter{Semisolutions}

{

\footnotesize
\begin{multicols}{2}

\refstepcounter{chapter}
\setcounter{eqtn}{0}

\parbf{\ref{ex:quad-inq}.}
Add four triangle inequalities (\ref{metric:triangle}).

\parbf{\ref{ex:normal}.}
Consider the function 
\[f(x)=\frac{\distfun_Ax}{\distfun_Ax+\distfun_Bx},\]
where $\distfun_Ax\z\df\inf_{a\in A}\dist{a}{x}{}$.
Show that $f$ is continuous and satisfies the needed property.

\parbf{\ref{ex:tietze}.}
Use \ref{ex:normal} to construct an approximation of the needed function and pass to a limit or find a proof of the \index{Tietze extension theorem}\emph{Tietze extension theorem}.

\parbf{\ref{ex:pseudo-infty-metric}};\ref{SHORT.ex:pseudo-infty-metric:pseudo}.
Note that if $\mu(A)=\mu(B)=0$, then $|A-B|=0$.
Therefore, \ref{metric=0} does not hold for bounded closed subsets.
It is straightforward to check the remaining conditions in~\ref{def:metric} hold true.

\parit{\ref{SHORT.ex:pseudo-infty-metric:infty}.}
Note that the distance from the empty set to the whole plane is infinite; so the value $|A-B|$ might be infinite.
It is straightforward to check the remaining conditions in~\ref{def:metric}.

\parit{Remark.}
Metrics of the form $\dist{A}{B}{}=\mu(A\bigtriangleup B)$ are very special.
In particular, they satisfy the so-called \index{hypermetric inequality}\emph{hypermetric inequalities}; that is, for any sequence of sets $A_1,\dots, A_n$ and any sequence of integers $b_1,\dots,b_n$ such that $\sum_ib_i=1$ we have
\[\sum_{i,j}b_i\cdot b_j\cdot \dist{A_i}{A_j}{}\le 0.\]
Note that for $n=3$ and $b_1=b_2=-b_3=1$ we get the usual triangle inequality.
For more on the subject, see \cite{deza-laurent}.

\parbf{\ref{ex:gluing}.}
Choose $\delta>0$ and an increasing linear bijection $\ell\:[a,b]\to [c,d]$.
Show that $\ell$ has arbitrarily close increasing piecewise-linear bijection $s\:[a,b]\to [c,d]$ such that derivative at any point is either $<\delta$ or $>\tfrac1\delta$.

Start with the identity map $[0,1]\to [0,1]$;
iterate the above construction for smaller and smaller $\delta$ and pass to the limit.
This way we obtain an increasing  bijection $x\leftrightarrow x'$ from $[0,1]$ to itself
such that for any $\eps>0$ there is a partition $0=t_0<t_1<\dots <t_{2\cdot n}=1$ of $[0,1]$ with 
\begin{align*}
\eps&>|t_0-t_1|+|t_1'-t_2'|+|t_2-t_3|+\dots
\\
&\dots+|t_{2\cdot n-2}-t_{2\cdot n-1}|+|t_{2\cdot n-1}'-t_{2\cdot n}'|.
\end{align*}
Make a conclusion.

\parbf{\ref{ex:almost-min}.}
Assume the statement is wrong. 
Then for any point $x\in \spc{X}$, there is a point $x'\in \spc{X}$ such that 
\begin{align*}
\dist{x}{x'}{}&<\rho(x)
\intertext{and}
\rho(x')&\le\frac{\rho(x)}{1+\eps}.
\end{align*}
Consider a sequence $x_1,x_2,\dots\in \spc{X}$ such that $x_{n+1}\z=x_n'$.
Show that this is a Cauchy sequence.
Since $\spc{X}$ is complete, $x_n$ converges;
denote its limit by $x_\infty$.
Since $\rho$ is a continuous function we get
\begin{align*}
\rho(x_\infty)&=\lim_{n\to\infty}\rho(x_n)=0.
\end{align*}

The latter contradicts that $\rho>0$.

\parbf{\ref{ex:complete-completion}.}
Let $\bar {\spc{X}}$ be the completion of $\spc{X}$.
By the definition, for any $y\in \bar {\spc{X}}$ there is a Cauchy sequence $x_n$ in  $\spc{X}$ that converges to $y$.

Choose a Cauchy sequence $y_m$ in $\bar {\spc{X}}$.
From above, we can choose points $x_{n,m}\in \spc{X}$ such that $x_{n,m}\to y_m$ for any $m$.
Choose $z_m=x_{n_m,m}$ such that $|y_m-z_m|<\tfrac1m$.
Observe that $z_m$ is Cauchy.
Therefore, its limit $z_\infty$ lie in $\bar{\spc{X}}$.
Finally, show that $x_m\to z_\infty$.

\parbf{\ref{ex:compact-net}.}
A compact $\eps$-net $N$ in $\spc{K}$ contains a finite $\eps$ net $F$.
Show and use that $F$ is a $2\cdot\eps$-net of $\spc{K}$.

\parbf{\ref{ex:non-contracting-map}.}
Given a pair of points $x_0,y_0\in \spc{K}$, 
consider two sequences $x_0,x_1,\dots$ and $y_0,y_1,\dots$
such that $x_{n+1}=f(x_n)$ and $y_{n+1}\z=f(y_n)$ for each $n$.

Since $\spc{K}$ is compact, 
we can choose an increasing sequence of integers $n_k$
such that both sequences $(x_{n_i})_{i=1}^\infty$ and $(y_{n_i})_{i=1}^\infty$
converge.
In particular, both are Cauchy;
that is,
\[
|x_{n_i}-x_{n_j}|_{\spc{K}}\to 0 
\quad\text{and}\quad
|y_{n_i}-y_{n_j}|_{\spc{K}}\to 0
\]
as $\min\{i,j\}\to\infty$.

Since $f$ is distance-noncontracting, 
\[
|x_0-x_{|n_i-n_j|}|
\le 
|x_{n_i}-x_{n_j}|
\]
for any $i$ and $j$.
Therefore, there is a sequence $m_i\to\infty$ such that
\[
x_{m_i}\to x_0\quad\text{and}\quad y_{m_i}\to y_0
\leqno({*})\]
as $i\to\infty$.

Since $f$ is distance-noncontracting, the sequence $\ell_n=|x_n-y_n|_{\spc{K}}$ is nondecreasing.
By $({*})$,  $\ell_{m_i}\to\ell_0$ as $m_i\to\infty$.
It follows that 
\[\ell_0=\ell_1=\dots\]
In particular, 
\[|x_0-y_0|_{\spc{K}}=\ell_0=\ell_1=|f(x_0)-f(y_0)|_{\spc{K}}\]
for any pair of points $(x_0,y_0)$ in $\spc{K}$.
That is, the map $f$ is distance-preserving; hence $f$ is injective.
From $({*})$, we also get that $f(\spc{K})$ is everywhere dense.
Since $\spc{K}$ is compact $f\:\spc{K}\to \spc{K}$ is surjective --- hence the result.

\parit{Remarks.}
This is a basic lemma in the introduction to Gromov--Hausdorff distance \cite[see 7.3.30 in][]{burago-burago-ivanov}.
The presented proof is not quite standard,
I learned it from Travis Morrison, 
a student in my MASS class at Penn State, Fall 2011.

Note that this exercise implies that \textit{any surjective non-expanding map from a compact metric space to itself is an isometry}. 

\parbf{\ref{ex:loc-compact-not-proper}.}
Check an infinite set with a discrete metric.

\parbf{\ref{ex:pogorelov}.}
Set $B_p=B(x,\tfrac \pi2)_{\mathbb{S}^2}$.
The triangle inequality follows since
\[
(B_x\setminus B_y)
\cup 
(B_y\setminus B_z)
\supseteq
B_x\setminus B_z.
\leqno(*)\]
The remaining conditions in Definition \ref{def:metric} are evident.

Observe that
$B_x\setminus B_y$
does not overlap with
$B_y\z\setminus B_z$ and  we get equality in $(*)$ if and only if $y$ lies on the great circle arc from $x$ to $z$.
Therefore, the second statement follows.


\begin{wrapfigure}{r}{24 mm}
\vskip-0mm
\centering
\includegraphics{mppics/pic-29}
\end{wrapfigure}

\parit{Remarks.}
This construction is due to 
Aleksei Pogorelov \cite{pogorelov}.
It is closely related to the construction given 
by David Hilbert \cite{hilbert}
which was the motivating example for his fourth problem. 
See also the remark after the solution of~\ref{ex:pseudo-infty-metric}.

\parbf{\ref{ex:4-point-trees}.}
We may assume that none of the points $p,x,y,z$ lies on a geodesic between the other two.

Let $K$ be the set in the tree covered by all six geodesics with the endpoints $p,x,y,z$.
Observe that $K$ looks like an H or like an X; make a conclusion.

\parit{Remarks.}The value $\tfrac12\cdot(|p-x|+|p-y|-|x-y|)$ is called \index{Gromov's product}\emph{Gromov's product} of $x$ and $y$ with the origin at $p$;
usually it is denoted by $(x|y)_p$.

Note that a four-point metric space admits an isometric embedding into a metric tree if and only if one of these two equivalent conditions holds.
Moreover, a metric space admits an isometric embedding into a metric tree if every its four-point subspace admits such embedding.


\parbf{\ref{ex:spheres-in-trees}.}
Apply \ref{ex:4-point-trees}.

\parbf{\ref{ex:1-Lip-G-delta}.}
Note that $\spc{P}$ is complete.
Choose $\eps>0$.
Use \ref{thm:length-semicont} to show that the set of paths of length $>1-\eps$ is open in~$\spc{P}$;
show that this set is also dense in~$\spc{P}$.
Apply Baire's theorem (\ref{thm:baire}).

\parit{Remark.}
You might find it surprising that \textit{most of the short maps from the sphere to the plane are \index{length-preserving map}\emph{length-preserving}}; that is, they preserve lengths of all curves.
The latter follows from the result of 
Bernd Kirchheim, 
Emanuele Spadaro,
and 
L{\'a}szl{\'o} Sz{\'e}kelyhidi \cite{KSS}.
(While most of the maps have this property, it is not at all easy to construct a single such example.) 


\parbf{\ref{ex:no-geod}.}
\textit{Formally speaking, a one-point space is a solution,
but we will construct a nontrivial example.}

Recall that $c_0\subset\ell^\infty$ denotes the space of all real sequences converging to zero.
Consider the unit ball $B$ in $c_0$;
denote by $\rho_0$ the metric on $B$.

Let \[\phi(\bm{x})=2+\tfrac{1}2\cdot x_1+\tfrac{1}4\cdot x_2+\tfrac{1}8\cdot x_3+\dots,\]
where $\bm{x}=(x_1,x_2\,\dots)\in B$.
Consider another length metric $\rho_1$ on $B$ that is different from $\rho_0$ by the conformal factor $\phi$;
that is, if $t\mapsto\bm{x}(t)$ for $t\in[0,\ell]$ is a curve parametrized by $\rho_0$-length,
then its $\rho_1$-length is defined by
\[\length_{\rho_1}\bm{x}\df\int\limits_0^\ell\phi\circ\bm{x}(t)\cdot dt.\]
Note that the metric $\rho_1$ is bilipschitz to~$\rho_0$.

Assume $t\mapsto \bm{x}(t)$ and $t\mapsto \bm{x}'(t)$ are two curves parametrized by $\rho_0$-length that differ only in the $m$-th coordinate; denote them by $x_m(t)$ and $x_m'(t)$ respectively.
Show that if $x'_m(t)\le x_m(t)$ for any $t$ and 
the function $x'_m(t)$ is locally $1$-Lipschitz at all $t$ such that $x'_m(t)< x_m(t)$, then 
\[\length_{\rho_1}\bm{x}'\le \length_{\rho_1}\bm{x}.\]
Moreover, this inequality is strict if $x'_m(t)\z< x_m(t)$ for some~$t$.

Fix a curve $\bm{x}(t)$, $t\in[0,\ell]$, parametrized by  $\rho_0$-length.
We can choose large $m$ so that $x_m(t)$ is sufficiently close to $0$ for any~$t$.
In this case, it is easy to construct a function $t\mapsto x'_m$ that meets the above conditions.
It follows that for any curve $\bm{x}(t)$ in $(B,\rho_1)$, we can find a shorter curve $\bm{x}'(t)$ with the same endpoints.
In particular, $(B,\rho_1)$ has no geodesics.

\parit{Remark.}
This solution was suggested by Fedor Nazarov~\cite{nazarov}.

\parbf{\ref{ex:compact+connceted}.}
Choose a sequence of positive numbers $\varepsilon_n\to 0$ and a finite $\varepsilon_n$-net $N_n$ of $K$ for each $n$.
We can assume that $\eps_0>\diam K$, and $N_0$ is a one-point set.
If $\dist{x}{y}{}<\eps_k$ for some $x\in N_{k+1}$ and $y\in N_{k}$, then connect them by a curve of length at most $\eps_k$.

Let $K'$ be the union of all these curves and $K$.
Show that $K'$ is compact and path-connected.

\parit{Source:} This problem is due to Eugene Bilokopytov \cite{bilokopytov}.

\parbf{\ref{ex:compact=>complete}.}
Choose a Cauchy sequence $x_n$ in $(\spc{X},\|*\z-*\|)$; it is sufficient to show that a subsequence of $x_n$ converges.

Observe that the sequence $x_n$ is Cauchy in $(\spc{X},|*-*|)$;
denote its limit by $x_\infty$.

Passing to a subsequence, we can assume that $\|x_n-x_{n+1}\|\z<\tfrac1{2^n}$.
It follows that there is a 1-Lipschitz path $\gamma$ in $(\spc{X},\|*-*\|)$ such that $x_n=\gamma(\tfrac1{2^n})$ for each $n$ and $x_\infty=\gamma(0)$.
Therefore,
\begin{align*}
\|x_\infty-x_n\|&\le \length\gamma|_{[0,\frac1{2^n}]}\le \tfrac1{2^n}.
\end{align*}
In particular, $x_n$ converges to $x_\infty$ in $(\spc{X},\|*\z-*\|)$.

\parit{Source:} \cite[Corollary]{hu-kirk}; see also \cite[Lemma 2.3]{petrunin-stadler}.

\parbf{\ref{ex:menger}.} Choose two points $x,y\in \spc{X}$;
let $\ell\z=\dist{x}{y}{}$.
Suppose $f\:E\to \spc{X}$ is a distance-preserving map such that $0,\ell\in E\subset [0,\ell]$,
$f(0)=x$, and  $f(\ell)=y$.

Show that we can choose $f$ so that $E$ is maximal;
that is, $f$ cannot be extended to a distance-preserving map on a larger subset of $[0,\ell]$.

Show that there is no open interval $(a,b)$ in the complement of $E$ such that $a,b\in E$.

Apply the completeness of $\spc{X}$ to show that $E$ is closed.
Conclude that $E=[0,\ell]$.

\parbf{\ref{ex:eps-nbhd(ball)}.}
Let $U$ be the $\eps$-neighborhood of $\oBall(x,R)_{\spc{X}}$.
By the triangle inequality, $U\z\subset \oBall(x,R+\eps)_{\spc{X}}$;
this inclusion holds in any metric space.

Choose $y\in \oBall(x,R+\eps)_{\spc{X}}$, so $\dist{x}{y}{{\spc{X}}}\z<R+\eps$.
Since ${\spc{X}}$ is a length space, there is a curve $\gamma$ from $x$ to $y$ with length less than $R+\eps$.
Show and use that $\gamma$ contains a point $m$ such that $\dist{x}{m}{{\spc{X}}}<R$ and $\dist{y}{m}{{\spc{X}}}<\eps$.

\parbf{\ref{exercise from BH}.}
Consider the following subset of $\RR^2$ equipped with the induced length metric
\[
\spc{X}
=
\bigl((0,1]\times\{0,1\}\bigr)
\cup
\bigl(\{1,\tfrac12,\tfrac13,\dots\}\times[0,1]\bigr)
\]
Note that $\spc{X}$ is locally compact and geodesic.

Its completion $\bar{\spc{X}}$ is isometric to the closure of $\spc{X}$ equipped with the induced length metric.
Note that $\bar{\spc{X}}$ is obtained from $\spc{X}$ by adding two points $p=(0,0)$ and $q\z=(0,1)$.

{

\begin{wrapfigure}{r}{20 mm}
\vskip-4mm
\centering
\includegraphics{mppics/pic-1}
\end{wrapfigure}

Observe that $p$ admits no compact neighborhood in $\bar{\spc{X}}$ and there is no geodesic connecting $p$ to $q$ in~$\bar{\spc{X}}$. 


\parit{Source:} \cite[I.3.6(4)]{bridson-haefliger}.

}

\parbf{\ref{ex:gross}.}
Suppose this number does not exist.
Show that there are two point-arrays $(x_1,\z\dots,x_n)$ and $(y_1,\dots,y_m)$
such that
\[
\min_{z\in \spc{X}}\{\,f(z)\,\}>\max_{z\in \spc{X}}\{\,h(z)\,\},
\leqno({*})
\]
where
\begin{align*}
f(z)&=\tfrac1n\cdot\sum_i|x_i-z|_{\spc{X}}
\intertext{and}
h(z)&=\tfrac1m\cdot\sum_j|y_j-z|_{\spc{X}}.
\end{align*}


Note that
\begin{align*}\tfrac1m\cdot\sum_j f(y_j)&=\tfrac1{m\cdot n}\cdot\sum_{i,j}|x_i-y_j|_{\spc{X}}=
\\
&=\tfrac1n\cdot\sum_i h(x_i);
\end{align*}
that is, the average value of $f(y_j)$ coincides with the average value of $h(x_i)$.
The latter contradicts~$({*})$.

\parit{Remark.}
The value $\ell$ is uniquely defined;
it is called the \index{rendezvous value}\emph{rendezvous value} of ${\spc{X}}$.
This is a result of Oliver Gross \cite{gross}.

%%%%%%%%%%%%%%%%%%%%%%%%%%%%%%

%%%%%%%%%%%%%%%%%%%%%%%%%
\refstepcounter{chapter}
\setcounter{eqtn}{0}

\parbf{\ref{ex:compact-length}.}
By the Fréchet lemma (\ref{lem:frechet}) we can identify $\spc{K}$ with a compact subset in $\ell^\infty$.

Denote by $\spc{L}$ the \index{closed convex hull}\emph{closed convex hull} of $\spc{K}$;
that is, $\spc{L}$ is the minimal convex closed set in $\ell^\infty$ that contains $\spc{K}$.
(In other words, $\spc{L}$ is the minimal closed set containing $\spc{K}$ such that if $x,y\in \spc{L}$, then 
$t\cdot x+(1-t)\cdot y\in \spc{L}$ for any $t\in[0,1]$.)

Observe that $\spc{L}$ is a length space.
It remains to show that $\spc{L}$ is compact.

By construction, $\spc{L}$ is a closed subset of $\ell^\infty$; in particular, it is complete.
By \ref{totally-bounded}, it remains to show that $\spc{L}$ is totally bounded.

Recall that Minkowski sum $A + B$ of two sets $A$ and $B$ in a vector space is defined by
\[A + B 
\df
\set{a+b}{a\in A,\ b\in B}.\]
Observe that the Minkowski sum of two convex sets is convex.

Denote by $\bar B_\eps$ the closed $\eps$-ball in $\ell^\infty$ centered at the origin.
Choose a finite $\eps$-net $N$ in $\spc{K}$ for some $\eps>0$.
Note that $P=\Conv N$ is a convex polyhedron; in particular, $\Conv N$ is compact.

Observe that $N+\bar B_\eps$ is a closed $\eps$-neighborhood of $N$.
It follows that $N+\bar B_\eps\supset K$ and therefore $P+\bar B_\eps\supset \spc{L}$.
In particular, $P$ is a $2\cdot\eps$-net in $\spc{L}$;
since $P$ is compact and $\eps>0$ is arbitrary, $\spc{L}$ is totally bounded (see \ref{ex:compact-net}).

\parit{Remark.}
Alternatively, one may use that \textit{the injective envelope of a compact space is compact}; see \ref{ex:inj=complete-geodesic-contractible:geodesic}, \ref{ex:Inj(compact)}, and \ref{prop:InjX-is-injective}.

\parbf{\ref{ex:frechet}.}
Modify the proof of \ref{lem:frechet}.

\begin{wrapfigure}{r}{23mm}
\vskip-6mm
\centering
\includegraphics{mppics/pic-200}
\end{wrapfigure}

\parbf{\ref{ex:inf-extension}.}
Consider the metric tree $\spc{T}$ shown on the diagram;
it is a half-line $[0,\infty)$ with attached an interval of length $n+1$ to each integer~$n\ge 0$.
Denote by $o$ the origin of the half-line
and by $x_n$ the endpoint of $n^{\text{th}}$ interval.

Observe that if $m\ne n$, then
\[|x_m-x_n|_{\spc{T}}\ge |o-x_n|_{\spc{T}}+1.\]
Show and use that for any binary sequence $\eps_n$ there is an extension function $f$ such that 
\[f(x_n)=|o-x_n|_{\spc{T}}+\eps_n.\]


\parit{Remark.}
An if-and-only-if condition on $\spc{X}$ that have separable $\spc{X}^\infty$ was found by Julien Melleray \cite[2.8]{melleray}.
A similar condition was used by Herbert Federer to describe metric spaces where Besicovitch covering lemma holds \cite[2.8.9]{federer}.

\parbf{\ref{ex:geodesics-urysohn}.}
Choose a separable space $\spc{X}$ that has an infinite number of geodesics between a pair of points with the given distance between them;
say a square in $\RR^2$ with $\ell^\infty$-metric will do.
Apply to $\spc{X}$ universality of Urysohn space (\ref{prop:sep-in-urys}).

\parbf{\ref{ex:compact-extension}.} 
First let us prove the following claim:

\begin{itemize}
\item 
Suppose $f\: K\to\RR$ is an extension function defined on a compact subset $K$ of the Urysohn space $\spc{U}$.
Then there is a point $p\in \spc{U}$ such that 
$\dist{p}{x}{}=f(x)$ for any $x\in K$.
\end{itemize}

Without loss of generality, we may assume that $f>0$.
Since $K$ is compact, we may fix $\eps>0$ such that $f(x)>\eps$ for any $x\in K$.

Consider the sequence $\eps_n=\tfrac\eps{100\cdot 2^n}$.
Choose a sequence of $\eps_n$-nets $N_n\subset K$.
Applying the universality of $\spc{U}$ recursively, we may choose a point $p_n$ such that $\dist{p_n}{x}{}=f(x)$ for any $x\in N_n$ and $\dist{p_n}{p_{n-1}}{}\z=10\cdot\eps_{n-1}$.
Observe that the sequence $p_n$ is Cauchy and its limit $p$ meets 
$\dist{p}{x}{}=f(x)$ for any $x\in K$.

Now, choose a sequence $x_n$ of points that is dense in $\spc{S}$.
Applying the claim, we may extend the map from $K$ to $K\cup\{x_1\}$, further to $K\cup\{x_1,x_2\}$, and so on.
As a result, we extend the distance-preserving map $f$ to the whole sequence $x_n$.
It remains to extend it continuously to the whole space~$\spc{S}$.

\parbf{\ref{ex:sc-urysohn}.}
It is sufficient to show that any compact subspace $\spc{K}$ of the Urysohn space $\spc{U}$ can be contracted to a point.

Note that any compact space $\spc{K}$ can be extended to a contractible compact space $\spc{K}'$; for example, we may embed $\spc{K}$ into $\ell^\infty$ and pass to its convex hull, as it was done in \ref{ex:compact-length}.

By \ref{thm:compact-homogeneous}, there is an isometric embedding of $\spc{K}'$ that agrees with the inclusion $\spc{K}\hookrightarrow\spc{U}$.
Since $\spc{K}$ is contractible in $\spc{K}'$, it is contractible in $\spc{U}$.

\parit{A better way.}
One can contract the whole Urysohn space using the following construction.

Note that points in $\spc{X}_\infty$ constructed in the proof of \ref{prop:univeral-separable} can be multiplied by $t\in [0,1]$ --- simply multiply each function by $t$.
That defines a map 
\[\lambda_t\:\spc{X}_\infty\to \spc{X}_\infty\]
that rescales all distances by factor $t$.
The map $\lambda_t$ can be extended to the completion of $\spc{X}_\infty$, which is isometric to $\spc{U}_d$ (or $\spc{U}$).

Observe that 
the map $\lambda_1$ is the identity  and $\lambda_0$ maps the whole space to a single point, say $x_0$ --- this is the only point of $\spc{X}_0$.
Further, note that $(t,p)\mapsto \lambda_t(p)$ is a continuous map; in particular, $\spc{U}_d$ and $\spc{U}$ are contractible.

As a bonus, observe that for any point $p\in \spc{U}_d$ the curve $t\mapsto \lambda_t(p)$ is a geodesic path from $p$ to $x_0$.

\parit{Source:} \cite[$\text{(d)}$ on page 82]{gromov-2007}.

\parbf{\ref{ex:no-isom}.}
Consider two infinite metric trees as on the diagram. 

\begin{Figure}
\vskip-0mm
\centering
\includegraphics{mppics/pic-205}
\end{Figure}

\parit{Remark.}
A more sophisticated example: $\spc{X}\z=\ell^\infty$ and $\spc{Y}=L^\infty([0,1])$.
Try to prove that it qualifies; see also \cite{buehler}.

%Given a bounded sequence $\bm{a}=(a_1,a_2,\dots)$, consider the function $f$ such that $f(0)=0$ and $f(x)=a_n$ if $\tfrac1{n+1}<x\le \tfrac1n$.
%Note that $\bm{a}\mapsto f$ is a distance-preserving map $\ell^\infty\to L^\infty([0,1])$.

%Further, enumerate all subintervals of $[0,1]$ with rational ends, $I_1,I_2,\dots$
%Given a function $f\in L^\infty([0,1])$ consider sequence $\bm{a}\z=(a_1,a_2,\dots)$ such that $a_n$ is the mean value of $f$ on $I_n$.
%Observe that $f\mapsto \bm{a}$ is a distance-preserving map $L^\infty([0,1])\to \ell^\infty$.

%It remains to show that $\spc{X}=\ell^\infty$ and $\spc{Y}=L^\infty([0,1])$ are not isometric???



\parbf{\ref{ex:sphere-in-urysohn}}; \ref{SHORT.ex:sphere-in-urysohn:sphere} and \ref{SHORT.ex:sphere-in-urysohn:midpoint}.
Observe that $L$ and $M$ satisfy the definition of $d$-Urysohn space and apply the uniqueness (\ref{thm:urysohn-unique}).
Note that
\[\ell=\diam L=\min\{2\cdot r, d\}.\]

\parit{\ref{SHORT.ex:sphere-in-urysohn:homogeneous}.} 
Use \ref{SHORT.ex:sphere-in-urysohn:sphere}, maybe twice.

\parbf{\ref{ex:shere}.}
Let $p$ be the center of the sphere;
without loss of generality, we can assume that $\dist{p}{x}{}\le \dist{p}{y}{}$.

Consider function $f\:\{p,x,y\}\to\RR$ defined by $f(p)=1$, $f(x)=1+\dist{p}{x}{}$, and $f(y)=1+\dist{p}{y}{}-\eps$.
Suppose $\eps>0$ is sufficiently small;
show that $f$ is an extension function on $\{p,x,y\}$.

By the extension property, there is a point $z\in \spc{U}$ such that $\dist{p}{z}{}=f(p)$, $\dist{x}{z}{}=f(x)$, and $\dist{y}{z}{}=f(y)$.
Whence the statement follows.

\parit{Source:} This problem is taken from a survey of Julien Melleray
 \cite[Prop. 4.3]{melleray}, where it was attributed to Matatyahu Rubin.


\parbf{\ref{ex:ext(shere)}.} 
Observe that the complement $\spc{V}=\spc{U}\setminus B$ is complete.
Show that it $\spc{V}$ satisfies the extension property.
Conclude that $\spc{V}$ is an Urysohn space and apply \ref{thm:urysohn-unique}.

For the second part, observe that there is an isometry $\iota\:\spc{U}\to \spc{V}$.
Moreover, if $p$ is the center of $B$, then we can assume that $\iota$ has a fixed point $x$ such that $\dist{p}{x}{}>2$.

Consider the unit sphere $S$ centered at $x$.
The restriction of $\iota$ to $S$ is an isometry of $S$.
Use \ref{ex:shere} to show that it cannot be extended to an isometry of $\spc{U}$.

\parit{Source:} \cite[Sec. 4.4]{melleray}.

\parbf{\ref{ex:katetov}.}
Apply \ref{thm:urysohn-unique} and the construction in \ref{thm:urysohn-exists+}.

\parbf{\ref{ex:homogeneous}}; \ref{SHORT.ex:homogeneous:euclidean}.
The euclidean plane is homogeneous in every sense.

\parit{\ref{SHORT.ex:homogeneous:hilbert}.}
The Hilbert space $\ell^2$ is finite-set-homogeneous, but not compact-set-homogeneous, nor countable-set-homogeneous.

\parit{\ref{SHORT.ex:homogeneous:ell-infty}.}
$\ell^\infty$ is one-point-homogeneous, but not two-point-homogeneous.
Try to show that there is no isometry of $\ell^\infty$ such that
\begin{align*}
(0,0,0,\dots)&\mapsto (0,0,0,\dots),
\\
(1,1,1,\dots)&\mapsto (1,0,0,\dots).
\end{align*}

\parit{\ref{SHORT.ex:homogeneous:ell-1}.}
$\ell^1$ is one-point-homogeneous, but not two-point-homogeneous.
Try to show that there is no isometry of $\ell^\infty$ such that
\begin{align*}
(0,0,0,\dots)&\mapsto (0,0,0,\dots),
\\
(2,0,0\dots)&\mapsto (1,1,0,\dots).
\end{align*}

\parbf{\ref{ex:homogeneous-tree}.}
Let $\spc{T}$ be a one-point-homogeneous metric tree.
Note that all points in $\spc{T}$ have the same degree $d$;
that is, for any point $t\in \spc{T}$ the set of connected components of the complement $\spc{T}\setminus\{t\}$ has the same cardinality $d$.

Show that if $d=0$, then $\spc{T}$ is a one-point space;
there is no tree with $d=1$,
and if $d=2$, then $\spc{T}\iso\RR$.

Suppose $d\ge 3$.
Choose a geodesic $\gamma$ in $\spc{T}$.
Show that number of connected components of $\spc{T}\setminus\gamma$ has cardinality continuum.
Observe and use that one can choose a point $p_\alpha$ in each connected component such that $\dist{p_\alpha}{p_\beta}{\spc{T}}>1$ if $\alpha\ne\beta$.

\parbf{\ref{ex:horobry}.}
Assume $F_{\spc{X}}$ is not an embedding.
That is, there is a sequence of points $x_1,x_2,\dots$ 
and a point $x_\infty$ such that $f_{x_n}\to f_{x_\infty}$ in $C(\spc{X},\RR)$
as $n\to \infty$, 
while $|x_n-x_\infty|_{\spc{X}}>\eps$ 
for some fixed $\eps>0$ and all~$n$.

By \ref{prop:length+proper=>geodesic}, any pair of points $x,y\in \spc{X}$ can be connected by a minimizing geodesic $[xy]$.
Choose $\bar x_n$ on a geodesic $[x_\infty x_n]$ such that $|x_\infty-\bar x_n|=\eps$.
Note that 
\begin{align*}
f_{x_n}(x_\infty)-f_{x_n}(\bar x_n)&=\eps,
\\
f_{x_\infty}(x_\infty)-f_{x_n}(\bar x_n)&=-\eps
\end{align*}
for all $n$.

Since $\spc{X}$ is proper, we can pass to a subsequence of $x_n$ so that the sequence  $\bar x_n$ converges;
denote its limit by $\bar x_\infty$.
The above identities imply that
\begin{align*}
f_{x_n}(\bar x_\infty)&\not\to f_{x_\infty}(\bar x_\infty)
\quad
\text{or}
\\
f_{x_n}(x_\infty)&\not\to f_{x_\infty}( x_\infty)
\end{align*}
--- a contradiction.

For the second part, take $\spc{Y}$ to be the set of non-negative integers with the metric $\rho$ defined by
\[\rho(m,n)=m+n\] 
for $m\ne n$.

\medskip

\parit{Source:}
I learned this example from Linus Kramer and Alexander Lytchak;
it was also mentioned in the lectures of Anders Karlsson
and attributed to Uri Bader \cite[2.3]{karlsson}.

\parbf{\ref{ex:cut}.}
Suppose that our metric is $\sum a_S\cdot\delta_S$ with $a_S\ge 0$ for any $S\subset F$.
Enumerate all the subsets $S_1,\dots,S_{2^n}$;
set $S_i=F$ for all $i>2^n$. 
Consider the maps $x\mapsto (a_1,a_2,\dots)$ where $a_i=0$ if $x\in S_i$ and otherwise $a_i=1$.
Observe that it defines a distance-preserving map $F\to \ell^1$. 

The if part is proved.
For the only-if part, check the statement for subsets of the real line, and use it.

\parbf{\ref{ex:K23}.}
Show that for any proper subset $S$ in the vertex set there are three vertices $x,y,z$ such that $\dist{x}{y}{} +\dist{y}{z}{}=\dist{x}{z}{}$ and either 
$x,z\in S$ and $y\notin S$, or $x,z\notin S$ and $y\in S$.
Then apply \ref{ex:cut}.

\parbf{\ref{ex:RP-not}.}
For the first part, show and use that the quotient of $\RP^2$ by the isotropy group of one point is isometric to a line segment.

For the second part, choose three points on a closed geodesic at equal distances from each other.
Show and use that there is an isometric three-point set in $\RP^2$ that does not lie on a closed geodesic.

\parit{Source:} \cite[V \S 2]{busemann-1942}.

\parbf{\ref{ex:hom-cube}.}
Denote by $\dim(x_1,\dots,x_m)$ the dimension of the minimal face of the cube that contains all the points $x_1,\dots,x_m\in Q$.
Show and use that 
\[\dim(x_1,\dots,x_m)=\dim(x_1',\dots,x_m')\]
for any isometry $x\mapsto x'$ of $Q$.

\parit{Source:} \cite[prop. 6 and 7]{berestovskii-nikonorov}.

\parbf{\ref{ex:conv-short};} \textit{only-if part}.
To check convexity, assume that $B$ is a two-point subset.
For closeness, assume that $B$ is a countable set of $A$.

\parit{If part.}
Learn about the Kirszbraun theorem and apply it together with the closest-point projection.

\refstepcounter{chapter}
\setcounter{eqtn}{0}


\parbf{\ref{ex:inj=complete-geodesic-contractible}.}
Choose an injective space $\spc{Y}$.

\textit{\ref{SHORT.ex:inj=complete-geodesic-contractible:complete}.}
Fix a Cauchy sequence $x_n$ in $\spc{Y}$;
we need to show that it has a limit $x_\infty\in \spc{Y}$.
Consider metric on $\spc{X}=\NN\cup\{\infty\}$ defined by 
\begin{align*}
\dist{m}{n}{\spc{X}}&\df\dist{x_m}{x_n}{\spc{Y}},
\\
\dist{m}{\infty}{\spc{X}}&\df\lim_{n\to\infty}\dist{x_m}{x_n}{\spc{Y}}.
\end{align*}
Since the sequence is Cauchy, so is the sequence $\ell_n=\dist{x_m}{x_n}{\spc{Y}}$ for any $m$.
Therefore, the last limit is defined.

By construction, the map $n\mapsto x_n$ is distance-preserving on $\NN\subset \spc{X}$.
Since $\spc{Y}$ is injective, this map can be extended to $\infty$ as a short map; set $\infty\mapsto x_\infty$.
Since $\dist{x_n}{x_\infty}{\spc{Y}}\le \dist{n}{\infty}{\spc{X}}$ 
and $\dist{n}{\infty}{\spc{X}}\to 0$, we get that
$x_n\to x_\infty$ as $n\to\infty$.

\textit{\ref{SHORT.ex:inj=complete-geodesic-contractible:geodesic}.}
Applying the definition of injective space, we get a midpoint for any pair of points in $\spc{Y}$.
By \ref{SHORT.ex:inj=complete-geodesic-contractible:complete},
$\spc{Y}$ is a complete space.
It remains to apply \ref{lem:mid>geod:geod}.

\textit{\ref{SHORT.ex:inj=complete-geodesic-contractible:contractible}.}
Let $k\:\spc{Y}\hookrightarrow \ell^\infty(\spc{Y})$ be the Kuratowski embedding (\ref{lem:kuratowski}).
Observe that $\ell^\infty(\spc{Y})$ is contractible;
in particular, there is a homotopy $k_t\:\spc{Y}\hookrightarrow \ell^\infty(\spc{Y})$ such that $k_0=k$ and $k_1$ is a constant map.
(In fact, one can take $k_t=(1-t)\cdot k$.)

Since $k$ is distance-preserving and $\spc{Y}$ is injective,
there is a short map $f\:\ell^\infty(\spc{Y})\to \spc{Y}$ such that the composition $f\circ k$ is the identity map on $\spc{Y}$.
The composition $f\circ k_t\:\spc{Y}\hookrightarrow \spc{Y}$ provides the needed homotopy. 

\parbf{\ref{ex:bicombing}.}
By \ref{lem:kuratowski}, the space $\spc{Y}$ can be considered as a subset in $\ell^\infty(\spc{Y})$.
Given $x,y\in \spc{Y}$, let $\tilde\gamma_{x,y}(t)=(1-t)\cdot x+t\cdot y\in \ell^\infty(\spc{Y})$.
Observe that $\tilde\gamma_{x,y}$ meets all the conditions.
Apply the definition of injective space to $\ell^\infty(\spc{Y})$.

\parit{Remark.} The choice of geodesic paths as in the exercise is called \index{geodesic bicombing}\emph{geodesic bicombing}; it was introduced by Urs Lang \cite[3.6]{lang-2013}.

\parbf{\ref{ex:injective-spaces}.}
Suppose that a short map $f\:A\to\spc{Y}$ is defined on a subset $A$ of a metric space $\spc{X}$.
We need to construct a short extension $F$ of $f$.
Without loss of generality, we may assume that $A\ne\emptyset$;
otherwise, map the whole $\spc{X}$ to a single point.
By Zorn's lemma, it is sufficient to enlarge $A$ by a single point $x\notin A$.

\parit{\ref{SHORT.ex:injective-spaces:R}.}
Suppose $\spc{Y}=\RR$.
Set 
\[F(x)=\inf\set{f(a)-\dist{a}{x}{}}{a\in A}.\] 
Observe that $F$ is short and $F(a)=f(a)$ for any $a\in A$.

\parit{\ref{SHORT.ex:injective-spaces:tree}.}
Suppose  $\spc{Y}$ is a complete metric tree.
Fix points $p\in \spc{X}$ and $q\in\spc{Y}$.
Given a point $a\in A$,
let $x_a\in\cBall[f(a),\dist{a}{p}{}]$ be the point closest to $f(x)$.
Note that $x_a\in[q\,f(a)]$ and either $x_a=q$ or $x_a$ lies on distance $\dist{a}{p}{}$ from $f(a)$.

Note that the geodesics $[q\,x_a]$ are nested;
that is, for any $a,b\in A$ we have either $[q\,x_a]\z\subset [q\,x_b]$ or $[q\,x_b]\z\subset [q\,x_a]$.
Moreover, in the first case, we have $\dist{x_b}{f(a)}{}\le \dist{p}{a}{}$ and in the second $\dist{x_a}{f(b)}{}\le \dist{p}{b}{}$.

It follows that the closure of the union of all geodesics $[q\,x_a]$ for $a\in\spc{A}$ is a geodesic.
Denote by $x$ its endpoint; it exists since $\spc{Y}$ is complete.
It remains to observe that $\dist{x}{f(a)}{}\le \dist{p}{a}{}$ for any $a\in\spc{A}$;
that is, one can take $f(p)=x$.

\parit{\ref{SHORT.ex:injective-spaces:ell-infty}.}
Suppose $\spc{Y}=(\RR^2,\ell^\infty)$.
Note that $\spc{X}\z\to (\RR^2,\ell^\infty)$ is a short map if and only if both of its coordinate projections are short.
It remains to apply \ref{SHORT.ex:injective-spaces:R}.
The general case of $\ell^\infty(\spc{S})$ can be done the same way.

More generally, \textit{any $\ell^\infty$-product of injective spaces is injective};
in particular, if $\spc{Y}$ and $\spc{Z}$ are injective then the product $\spc{Y}\times\spc{Z}$ equipped with the metric 
\[\dist{(y,z)}{(y',z')}{\spc{Y}\times\spc{Z}}=\max\{\,\dist{y}{y'}{\spc{Y}},\dist{z}{z'}{\spc{Z}}\,\}\]
is injective as well.

\parbf{\ref{ex:extr-ball}}; \ref{SHORT.ex:extr-ball:one}.
Let $\spc{B}=\cBall[o,R]_{\spc{Y}}$.
Choose a metric space $\spc{X}$ with a subset $A$.
Given a short map $f\:A\to \spc{B}$ we need to find its short extension $\spc{X}\to \spc{B}$.

Since $\diam\spc{B}\le 2\cdot R$, we may assume that  $\diam \spc{X}\le 2\cdot R$;
if not pass to the metric defined by $\dist{x}{y}{}=\max\{\,\dist{x}{y}{\spc{X}},2\cdot R\,\}$.

Let us add point $w$ to $\spc{X}$ such that $\dist{w}{x}{}=R$ for any $x\in\spc{X}$;
denote the obtained space $\spc{X}'$.
Let $f'\:A\cup\{w\}\to \spc{B}$ be an extension of $f$ by $w\mapsto o$; note that $f'$ is short.

Since $\spc{Y}$ is injective, there is a short extension $F\:\spc{X}'\to \spc{Y}$ of $f'$.
Show and use that $F(\spc{X}')\subset \spc{B}$.

\parit{\ref{SHORT.ex:extr-ball:many}.}
Let $\spc{B}=\cap_\alpha\cBall[o_\alpha,R_\alpha]_{\spc{Y}}$.
Try to modify the argument in \ref{SHORT.ex:extr-ball:one}.

(Note that one may assume that $\diam \spc{X}\z\le 2\cdot \inf_\alpha\{\,R_\alpha\,\}$.
Consider the space $\spc{X}'\z=\spc{X}\cup\{w_\alpha\}$ such that $\dist{w_\alpha}{x}{}=R_\alpha$ for any $x\in \spc{X}$ and $\dist{w_\alpha}{w_\beta}{}=R_\alpha+R_\beta$ if $\alpha\ne\beta$.
Further, consider an extension of $f$ by $w_\alpha\mapsto o_\alpha$.)

\parbf{\ref{ex:extr-fixed}.}
Let $\diam \spc{Y}=2\cdot R$.
We can assume that $R>0$; otherwise there is nothing to prove.
Denote by $\spc{Z}$ a minimal (with respect to inclusion) intersection of closed $R$-balls in $\spc{Y}$ such that $s(\spc{Z})\subset\spc{Z}$.

Consider 
the intersection 
\[\spc{Y}'=\spc{Z}\cap\left(\bigcap_{p\in \spc{Z}} \cBall[p,R]_{\spc{Y}}\right).\]
By \ref{ex:extr-ball:many}, $\spc{Y}'$ is injective.
Use that $\spc{Z}$ is minimal to show that $s(\spc{Y}')\subset \spc{Y}'$.
Show that $\diam \spc{Y}'\le \tfrac12\cdot\diam \spc{Y}$.

Consider a sequence of nested injective spaces $\spc{Y}=\spc{Y}_0\supset \spc{Y}_1\supset\dots$ such that $\spc{Y}_{n+1}\z=\spc{Y}_{n}'$.
Choose a point $y_n\in \spc{Y}_{n}$ for each $n$.
Show that the sequence $y_n$ is Cauchy.
By \ref{ex:inj=complete-geodesic-contractible:complete}, $y_n$ converges, say to $y_\infty$.
Observe that $y_\infty$ is a fixed point of $s$.

\parbf{\ref{ex:circle};} \textit{only-if part}.
Suppose $r$ is extremal.
By \ref{lem:extremal-lipschitz}, $r$ is $1$-Lipschitz.
Since $\mathbb{S}^1$ is compact, \ref{lem:opposite-compact} implies that for any $p\in \mathbb{S}^1$ there is $q\in \mathbb{S}^1$ such that 
\[r(p)+r (q) = \dist{p}{q}{\mathbb{S}^1}.\]
Therefore
\begin{align*}
\pi&=\dist{p}{(-p)}{\mathbb{S}^1}\le 
\\
&\le 
r(p)+r(-p)=
\\
&=
r(p)+r(q) +r(-p) -r(q)\le
\\
&\le
\dist{p}{q}{\mathbb{S}^1}+\dist{q}{(-p)}{\mathbb{S}^1}=
\\
&=\pi.
\end{align*}
So, we have equality in both places, and the only-if part follows.

\parit{If part.}
Assume $r$ is a 1-Lipschitz function such that $r(p)+r(-p)=\pi$.
Then 
\begin{align*}
\dist pq{\mathbb{S}^1}&=
\dist{p}{(-p)}{\mathbb{S}^1}-\dist{q}{(-p)}{\mathbb{S}^1}\ge
\\
&\ge\pi -(r(-p)-r(q))=
\\
&=r(p)+r(q).
\end{align*}
Therefore $r$ is admissible.

Finally, if $r$ is not extremal, then there is an admissible function $s\le r$ such that $s(p)<r(p)$ for some $p$.
The latter contradicts the equality $r(p)+r(-p)=\pi$.

\parit{Source:} \cite[Proposition 2.7]{zuest}.

\parbf{\ref{ex:retraction}.}
Show and use that
$s^*(x)+s(y)\ge \dist{x}{y}{}$
for any $x,y\in \spc{X}$.

\parit{Remarks.}
It is easy to check that $q\:s\z\mapsto \tfrac12\cdot(s+s^*)$ is a short map on the space of admissible functions (with sup-norm).
Moreover, iterating $q$ and passing to the limit, we get a short retraction from the space of admissible functions to the space of extremal functions on $\spc{X}$ \cite[see 3.1 in][]{lang-2013}.
The existence of such a map will also follow from \ref{thm:inj-envelope}.

\parbf{\ref{ex:one-point-gluing}.}
Apply \ref{thm:injective=hyperconvex:balls}.

\parit{Comment.}
Conditions under which gluings of injective spaces is injective were studied by Benjamin Miesch and Maël Pavón \cite{miesch,miesch-pavon}.

\parbf{\ref{ex:Rm-ell-infty}.}
Let $B=\cBall[0,1]$ and $P\supset B$ be a parallelepiped of minimal volume.
Choose the basis $e_1,\dots,e_m$ parallel to the edges of $P$ so that in the corresponding coordinates the parallelepiped is described by inequalities
$|x_i|\le 1$ for all $i$.

Let $B_i=\cBall[(1+R)\cdot e_i,R]$ for some $R>0$.
Show that $ e_i\in B$ for any $i$; in particular $B\cap B_i\ne\emptyset$.
Show $P$ can be chosen so that $B_i\cap B_j\ne \emptyset$ for all $i$ and~$j$ and all large $R>0$.
Apply hyperconvexity to show that $e_1+\dots+ e_m\in B$.
The same way, show that $\pm e_1\pm \dots\pm e_m\in B$ for all choices of signs.
Conclude that $B=P$.

\parbf{\ref{ex:compact-hyperconvex}.}
Observe that closed balls are compact and
apply the finite intersection property.

\parbf{\ref{ex:urysohn-hyperconvex}.}
Denote by $\spc{U}_d$ the $d$-Urysohn space,
so $\spc{U}_\infty$ is the Urysohn space.

The extension property implies finite hyperconvexity.
It remains to show that $\spc{U}_d$ is not countably hyperconvex.

Suppose that $d<\infty$.
Then $\diam\spc{U}_d=d$ and for any point $x\in\spc{U}_d$ there is a point $y\in\spc{U}_d$ such that $\dist{x}{y}{\spc{U}_d}=d$.
It follows that there is no point $z\in\spc{U}_d$ such that $\dist{z}{x}{\spc{U}_d}\le \tfrac d2$ for any $x\in\spc{U}_d$.
Whence $\spc{U}_d$ is not countably hyperconvex.

Use \ref{ex:sphere-in-urysohn:midpoint} to reduce the case $d=\infty$ to the case $d<\infty$.

\parbf{\ref{ex:almost-hyperconvex}.}
Let $p_0$ be a point provided by the definition of almost hyperconvexity;
that is $\dist{x_\alpha}{p_0}{}\le r_\alpha+\eps_0$ for a given $\eps_0>0$.
We may assume that $\delta_0=\sup\{\,\dist{x_\alpha}{p_0}{}- r_\alpha\,\}>0$; otherwise the problem is solved.
Clearly, $\delta_0\le \eps_0$.

Let $p_1$ be a point provided by the definition for $\eps_1<\tfrac1{10}\cdot\delta_0$ we get a point 
$p_1$ such that $\dist{x_\alpha}{p_1}{}\le r_\alpha+\eps_1$ and $\dist{p_0}{p_1}{}\le \delta_0+\eps_1$.
Again, we may assume that $\delta_1=\sup\{\,\dist{x_\alpha}{p_1}{}- r_\alpha,\dist{p_0}{p_1}{}\,\}>0$, and we have $\delta_1\le \eps_1$.

Continuing this way, we get a sequence $p_0,p_1,\dots$ that either terminates and in this case the problem is solved, or it is an infinte Cauchy sequence.
In the latter case, its limit $p_\infty$ satisfies $\dist{x_\alpha}{p_\infty}{}\le r_\alpha$ for any $\alpha$.

\parit{Comment.}
This solution reminds the proof of \ref{prop:completion-univeral};
a more exact statement was proved by Benjamin Miesch and Maël Pavón \cite[2.2]{miesch-pavon2016};
namely they show that almost $n$-hyperconvexity implies $(n-1)$-hyperconvexity.


\parbf{\ref{ex:Inj(compact)}.}
Observe and use that the functions in $\Inj\spc{X}$ are 1-Lipschitz and uniformly bounded.

\parbf{\ref{ex:tripod+square}}; \ref{SHORT.ex:tripod+square:2}.
Use \ref{lem:opposite-compact} to show that if $f$ is extremal if and only if $f(v)=x$ and $f(w)=1-x$ for some $x\in [0,1]$.
Conclude that $\Inj\spc{X}$ is isometric to the unit interval $[0,1]$.

\parit{\ref{SHORT.ex:tripod+square:tripod}.}
Let $f$ be an extremal function.
By \ref{lem:opposite-compact}, at least two of the numbers $f(a)+f(b)$, $f(b)+f(c)$, and $f(c)+f(a)$ are $1$.
It follows that for some $x\in[0,\tfrac12]$, we have 
\begin{align*}
f(a)&=1\pm x,&
f(b)&=1\pm x,&
f(c)&=1\pm x,
\end{align*}
where we have one ``minus'' and two ``pluses'' in these three formulas.

Suppose that
\begin{align*}
g(a)&=1\pm y,& g(b)&=1\pm y,& g(c)&=1\pm y
\end{align*}
is another extremal function.
Then $|f-g|\z=|x-y|$ if $g$ has ``minus'' at the same place as $f$ and $|f-g|=|x+y|$ otherwise.

It follows that $\Inj\spc{X}$ is isometric to a {}\emph{tripod} --- three segments of length $\tfrac12$ glued at one end.

\begin{Figure}
\begin{minipage}{.48\textwidth}
\centering
\includegraphics{mppics/pic-3}
\end{minipage}\hfill
\begin{minipage}{.48\textwidth}
\centering
\includegraphics{mppics/pic-4}
\end{minipage}
\vskip-4mm
\end{Figure}

\parit{\ref{SHORT.ex:tripod+square:square}.}
Assume $f$ is an extremal function.
Use \ref{lem:opposite-compact} to show that
\begin{align*}
2&=f(x)+f(y)=
\\
&=f(p)+f(q);
\end{align*}
in particular, two values $a=f(x)-1$ and $b\z=f(p)-1$ completely describe the function $f$.
Since $f$ is extremal, we also have that 
\[(1\pm a)+(1\pm b)\ge 1\]
for all 4 choices of signs;
equivalently, 
\[|a|+|b|\le 1.\]

It follows that $\Inj\spc{X}$ is isometric to the rhombus $|a|+|b|\le 1$ in the $(a,b)$-plane with the metric induced by the $\ell^\infty$-norm.

\parit{Remarks.}
If $\spc{X}$ has $n$-points, then (evidently) $\Inj\spc{X}$ is a polyhedral complex in $(\RR^n,\ell^\infty) \z=\ell^\infty(\spc{X})$;
each face of the complex is defined by equalities and inequalities of the following type: $x_i+x_j\ge \const$ and  $x_i+x_j= \const$.
It is easy to see (and follows from \ref{ex:Rm-ell-infty}) that each face is isometric to a convex polyhedron in  $(\RR^k,\ell^\infty)$ for some $k\le n$;
in fact $k\le n/2$.
The structure of the complex can be encoded by certain graphs with the vertex set $\spc{X}$ \cite[see Section 4 in][]{lang-2013}.

\parbf{\ref{ex:kur-inj}.}
Recall that $x\mapsto \distfun_x$ gives an isometric embedding $\spc{X}\z\hookrightarrow\ell^\infty(\spc{X})$;
so we can identify $\spc{X}$ with a subset of $\ell^\infty(\spc{X})$.
Further, $\Inj\spc{X}$ is a subset of $\ell^\infty(\spc{X})$.
It is sufficient to show that $\Inj\spc{X}=G$.

Use \ref{lem:opposite-compact} to show that $\Inj\spc{X}\subset G$.

Given $g\in G$, show that $g(x)=\dist{g}{x}{\ell^\infty(\spc{X})}$.
Conclude that $g$ is admissible and apply \ref{lem:opposite-compact}.

\parit{Source:} Private communications with Rostislav Matveyev.

\parbf{\ref{ex:4-on-a-line}.}
Recall that 
\[\dist{f}{g}{\Inj\spc{X}}=\sup\set{|f(x)-g(x)|}{x\in\spc{X}}\]
and 
\[\dist{f}{p}{\Inj\spc{X}}=f(p)\]
for any $f,g\in \Inj\spc{X}$ and $p\in \spc{X}$.

Since $\spc{X}$ is compact we can find a point $p\in\spc{X}$ such that 
\begin{align*}
\dist{f}{g}{\Inj\spc{X}}&=|f(p)-g(p)|=
\\
&=\left|\dist{f}{p}{\Inj\spc{X}}-\dist{g}{p}{\Inj\spc{X}}\right|.
\end{align*}
Without loss of generality, we may assume that 
\[\dist{f}{p}{\Inj\spc{X}}
=
\dist{g}{p}{\Inj\spc{X}}
+
\dist{f}{g}{\Inj\spc{X}}.\]
Applying \ref{lem:opposite-compact}, we can find a point $q\in\spc{X}$ such that 
\[\dist{q}{p}{\Inj\spc{X}}
=
\dist{f}{p}{\Inj\spc{X}}
+
\dist{f}{q}{\Inj\spc{X}},\]
whence the result.

Since $\Inj\spc{X}$ is injective (\ref{prop:InjX-is-injective}), by \ref{ex:inj=complete-geodesic-contractible:geodesic} it has to be geodesic. It remains to note that the concatenation of geodesics $[pq]$, $[gf]$, and $[fq]$ is a required geodesic $[pq]$.

\parbf{\ref{ex:delta-hyp}.} The only-if part follows since $\spc{X}$ is isometric to a subset of $\Inj\spc{X}$.

The if part means that 
\[
\begin{aligned}
\dist{f}{g}{}+\dist{v}{w}{}\le
\max\{\,
&\dist{f}{v}{}+\dist{g}{w}{},\\
\dist{f}{w}{}+&\dist{g}{v}{}
\,\}+2\cdot\delta
\end{aligned}
\eqlbl{eq:fgvw-hyp}\]
for any $f,g,v,w\in \Inj\spc{X}$.

Suppose $\spc{X}$ is compact. 
Applying \ref{ex:4-on-a-line}, we can choose $p,q,x,y\in \spc{X}$  such that 
\[
\begin{aligned}
\dist{p}{f}{}+\dist{f}{g}{}+\dist{g}{q}{}&=\dist{p}{q}{}
\\
\dist{x}{v}{}+\dist{v}{w}{}+\dist{w}{y}{}&=\dist{x}{y}{}
\end{aligned}
\eqlbl{eq:pfgq+xvwy}
\]

Since $\spc{X}$ is $\delta$-hyperbolic, we have
\[\begin{aligned}
\dist{p}{q}{}+\dist{x}{y}{}\le
\max\{\,&\dist{p}{x}{}+\dist{q}{y}{},
\\
\dist{p}{y}{}+&\dist{q}{x}{}\,\}+2\cdot\delta.
\end{aligned}\]
Show that this inequality, together with the triangle inequality and \ref{eq:pfgq+xvwy} imply \ref{eq:fgvw-hyp}.

For the noncompact case, prove an approximate version of \ref{eq:pfgq+xvwy} and apply it the same way.

\parbf{\ref{ex:inj-envelope}.}
Show that there is unique isometry of $\Inj\spc{X}$ that is indentity of $\spc{X}$.
Use it together with \ref{thm:inj-envelope}.


\parbf{\ref{ex:d-p-inclusion}.}
Show that there is a pair of short maps 
$\Inj\spc{X}\to\Inj\spc{U}\to\Inj\spc{X}$ 
such that their composition is the identity on $\spc{X}$.
Make a conclusion.

\parbf{\ref{ex:hemisphere-inj}.}
Apply \ref{lem:opposite-compact} to show that for any $u\in\mathbb{S}^2_+$ the restriction $f_u\z\df\distfun_u|_{\mathbb{S}^1}$ is extremal function on $\mathbb{S}^1$.
Moreover, the function $f_u$ uniquely determines $u$. 
Make a conclusion.

\parbf{\ref{ex:3-4-hypreconvex}.}
Observe that coordinate functions are monotonic on any geodesic in $\ell^1$.
Use it to show that $\ell^1$ is a \emph{median space};
that is, for any three points $x,y,z$ there is a {}\emph{unique} point $m$ (it is called \index{median}\emph{median} of $x$, $y$, and $z$) that lies on {}\emph{some} geodesics $[xy]$, $[xz]$ and $[yz]$.
Apply it to show that $\ell^1$ is 3-hyperconvex.

The 4-hyperconvexity fails for the unit balls centered at four even vertices of the cube $([0,1]^3,\ell^1)$.


\parbf{\ref{ex:ultrametric}.}
Choose three points $x,y,z\in\spc{X}$ and set $\spc{A}=\{x,z\}$.
Let $f\:\spc{A}\z\to \spc{A}$ be the identity map.
Then $F(y)=x$ or $F(y)=z$.
The strong triangle inequality easily follows in both cases.

\parbf{\ref{ex:ultrametric-converse}}; \textit{main part.}
Choose a maximal subset $A\z\supset K$ that admits a short retraction $f\:A\to K$;
it exists by Zorn's lemma.
If $A$ is the whole space, then the problem is solved.
Otherwise, choose $p\notin A$.

Choose a sequence of points $a_n\in A$ such that $\dist{a_n}{p}{}$ converge to the exact lower bound on the distances from points in $A$ to $p$.
Since $K$ is compact, we can pass to a subsequence of $a_n$ such that $f(a_n)$ converges.
Let 
\[f(p)=\lim f(a_n).\]

It remains to check that 
\[\dist{f(a)}{f(p)}{}\le\dist{a}{p}{}\eqlbl{eq:short-retract}\]
for any $a\in A$.
Choose $\eps>0$; note that 
\begin{align*}
\dist{a_n}{p}{}&<\dist{a}{p}{}+\eps
\intertext{and}
\dist{f(a_n)}{f(p)}{}&<\dist{f(a)}{f(a_n)}{}+\eps
\end{align*}
for all large~$n$.
Therefore, 
\begin{align*}
\dist{f(a)}{f(p)}{}&\le \max\{\,\dist{f(a)}{f(a_n)}{},
\\
&\qquad\dist{f(a_n)}{f(p)}{}\,\}\le
\\
&\le \dist{f(a)}{f(a_n)}{}+\eps\le
\\
&\le \dist{a}{a_n}{} +\eps\le 
\\
&\le \max\{\,\dist{a}{p}{},\dist{a_n}{p}{}\,\}+\eps< 
\\
&< \dist{a}{p}{}+2\cdot\eps.
\end{align*}
Since $\eps>0$ is arbitrary, we get \ref{eq:short-retract}.

\parit{Example.}
Consider set of $\{\infty,1,2,\dots\}$ with metric defined by 
\[|m-n|=1+\frac1{\min\{m,n\}}\]
for $m\ne n$.
Observe that the space is complete, the subset $\{1,2,\dots\}$ is closed, but it is not a short retract of the ambient space.

\parbf{\ref{ex:petrunin-stadler}.} Consider the space $\spc{Y}^{\spc{X}}$ of all maps $\spc{X}\z\to \spc{Y}$ equipped with the product topology.

Denote by $\mathfrak{S}_F$ the set of maps $h\in \spc{Y}^\spc{X}$ such that the restriction $h|_F$  is short and agrees with $f$ in $F\cap A$.
Note that the sets $\mathfrak{S}_F\subset \spc{Y}^\spc{X}$ are closed and any finite intersection of these sets is nonempty.

According to Tikhonov's theorem, $\spc{Y}^{\spc{X}}$ is compact.
By the finite intersection property, the intersection $\bigcap_F\mathfrak{S}_F$ for all finite sets $F\subset X$ is nonempty.
Hence the statement follows.

\parit{Source:} \cite{petrunin-stadler}.

\parbf{\ref{ex:diam}.}
Suppose that $\dist{A}{B}{\Haus\spc{X}}<r$.
Choose a pair of points $a,a'\in A$ on maximal distance from each other.
Observe that there are points $b,b'\in B$ such that 
$\dist{a}{b}{\spc{X}},\dist{a'}{b'}{\spc{X}}<r$.
Whence 
\[\dist{a}{a'}{\spc{X}}-\dist{b}{b'}{\spc{X}}\le 2\cdot r\]
and therefore
\[\diam A-\diam B\le 2\cdot\dist{A}{B}{\Haus\spc{X}}.\]

It remains to swap $A$ and $B$ and repeat the argument.


\parbf{\ref{ex:Hausdorff-bry}}; \ref{SHORT.ex:Hausdorff-bry:conv}.
Denote by $A^r$ the closed $r$-neighborhood of a set $A\z\subset \RR^2$.
Observe  that 
\[(\Conv A)^r=\Conv(A^r),\]
and try to use it.

\parit{\ref{SHORT.ex:Hausdorff-bry:bry}.}
The answer is ``no'' in both parts.

For the first part let $A$ be a unit disk and $B$ a finite $\eps$-net in $A$.
Evidently, $|A-B|_{\Haus\RR^2}<\eps$, 
but
$|\partial A-\partial B|_{\Haus\RR^2}\approx 1$.

For the second part take $A$ to be a unit disk and $B=\partial A$ to be its boundary circle.
Note that $\partial A=\partial B$; in particular, $\dist{\partial A}{\partial B}{\Haus\RR^2}=0$ while $\dist{ A}{B}{\Haus\RR^2}=1$.

\parit{Remark.}
There is the so-called {}\emph{lakes of Wada} --- an example of three (and more) open bounded topological disks in the plane that have identical boundaries.
It can be used to construct more interesting examples for \ref{SHORT.ex:Hausdorff-bry:bry}.

\parbf{\ref{ex:closure-union}.} 
Show that for any $\eps>0$ there is a positive integer $N$ such that $\bigcup_{n\le N} K_n$ is an $\eps$-net in the union $\bigcup_{n} K_n$.
Observe that $\bigcup_{n\le N} K_n$
is compact and apply \ref{ex:compact-net}.

\parbf{\ref{ex:Haus-length}}; \textit{``if'' part.}
Choose two compact sets $A,B\subset \spc{X}$;
suppose that $\dist{A}{B}{\Haus\spc{X}}<r$.

Choose finite $\eps$-nets $\{a_1,\dots a_m\}\subset A$ and $\{b_1,\dots b_n\}\subset B$.
For each pair $a_i,b_j$ construct a constant-speed path $\gamma_{i,j}$ from $a_i$ to $b_j$ such that 
\[\length \gamma_{i,j}<\dist{a_i}{b_j}{}+\eps.\]
Set 
\[C(t)=\set{\gamma_{i,j}(t)}{\dist{a_i}{b_j}{\spc{X}}<r+\eps}.\]
Observe that $C(t)$ is finite; in particular, it is compact.

Show and use that 
\begin{align*}
\dist{A}{C(t)}{\spc{X}}&<t\cdot r+10\cdot\eps,
\\
\dist{C(t)}{B}{\spc{X}}&<(1-t)\cdot r+10\cdot\eps.
\end{align*}
Apply \ref{ex:closure-union} and \ref{lem:mid>geod}.

\parit{``only-if'' part.}
Choose points $p,q\in\spc{X}$. 
Show that the existence of $\eps$-midpoints between $\{p\}$ and $\{q\}$ in $\Haus\spc{X}$ implies the existence of $\eps$-midpoints between $p$ and $q$ in $\spc{X}$.
Apply \ref{lem:mid>geod}.


\parbf{\ref{ex:Huas-perimeter-area}.}
Let $A$ be a compact convex set in the plane.
Denote by $A^r$ the closed $r$-neighborhood of $A$.
Recall that by Steiner's formula we have
\[\area A^r=\area A+r\cdot\perim A+\pi\cdot r^2.\]
Taking derivative and applying the coarea formula, we get 
\[\perim A^r=\perim A+2\cdot\pi\cdot r.\]

Observe that if $A$ lies in a compact set $B$ bounded by a closed curve, then 
\[\perim A\le \perim B.\]
Indeed the closest-point projection $\RR^2\to A$ is short and it maps $\partial B$ onto $\partial A$.

It remains to use the following observation: if $A_n\to A_\infty$, then for any $r>0$ we have that
\[A_\infty^r\supset A_n
\quad\text{and}\quad
A_\infty\subset A_n^r\]
for all large $n$.

\begin{wrapfigure}{r}{27 mm}
\vskip-6mm
\centering
\includegraphics{mppics/pic-410}
\end{wrapfigure}

\parbf{\ref{ex:round-disc}.}
Note that almost all points on $\partial D$ have a defined tangent line.
In particular, for almost all pairs of points $a,b\z\in\partial D$ the two angles $\alpha$ and $\beta$ between the chord $[ab]$ and $ \partial D$ are defined.

The convexity of $D'$ implies that $\alpha=\beta$;
here we measure the angles $\alpha$ and $\beta$ on one side from $[ab]$.
Show that if the identity $\alpha=\beta$ holds for almost all chords, then $D$ is a round disk. 


\parbf{\ref{ex:generalized-selection}.}
Observe that all functions $\distfun_{A_n}$ are Lipschitz and uniformly bounded on compact sets.
Therefore, passing to a subsequence, we may assume that the sequence $\distfun_{A_n}$ converges to some function $f$.

Set $A_\infty=f^{-1}\{0\}$.
It remains to show that $f=\distfun_{A_\infty}$.

%%%%%%%%%%%%%%%%%%%%%%%%%%%%%%

\parbf{\ref{ex:d_GH-and-diam}};
\ref{SHORT.ex:d_GH-and-diam:point}.
Apply the definition for space $\spc{Z}$ obtained from $\spc{X}$ by adding one point on distance $\tfrac12\cdot\diam \spc{X}$ to each point of $\spc{X}$.

\parit{\ref{SHORT.ex:d_GH-and-diam:scale}.}
Given a point $x\in\spc{X}$, denote by $a\cdot x$ and $b\cdot x$ the corresponding points in $a\cdot\spc{X}$ and $b\cdot \spc{X}$ respectively.
Show that there is a metric on $\spc{Z}\z=a\cdot\spc{X}\sqcup b\cdot\spc{X}$ such that 
\[|a\cdot x-b\cdot x|_{\spc{Z}}=\tfrac{|b-a|}2\cdot\diam\spc{X}\]
for any $x$ and the inclusions
$a\cdot\spc{X}\hookrightarrow\spc{Z}$,
$b\cdot\spc{X}\hookrightarrow\spc{Z}$ are distance preserving.

\parbf{\ref{ex:rectangle}.}
Arguing by contradiction,
we can identify $\spc{A}_r$ and $\spc{B}_r$ with subspaces of a space $\spc{Z}$
such that 
\[|\spc{A}_r-\spc{B}_r|_{\Haus \spc{Z}}<\tfrac1{10}\]
for large $r$; see the definition of Gromov--Hausdorff metric (\ref{def:GH}).

Set $n=\lceil r \rceil$.
Note that there are $2\cdot n$ integer points in~$\spc{A}_r$: 
$a_1\z=(0,0)$, $a_2=(1,0),\dots,a_{2\cdot n}=(n,1)$.
Choose a point $b_i\in \spc{B}_r$ that lies at the minimal distance from $a_i$.
Note that $|b_i-b_j|>\tfrac 45$ if $i\ne j$.
It follows that $r>\tfrac 45\cdot (2\cdot n-1)$.
The latter contradicts $n=\lceil r \rceil$ for large~$r$.

\parit{Remark.}
Try to show that $|\spc{A}_r-\spc{B}_r|_{\GH}=\tfrac12$ for all large $r$.

\parbf{\ref{ex:GH-inj}.}
Suppose that $|\spc{X}-\spc{Y}|_{\spc{U}}<\eps$;
we need to show that 
\[|\hat{\spc{X}}-\hat{\spc{Y}}|_{\GH}<2\cdot \eps.\]

Denote by $\hat{\spc{U}}$ the injective envelope of $\spc{U}$.
Recall that $\spc{U}$, $\spc{X}$, and $\spc{Y}$ can be considered as subspaces of $\hat{\spc{U}}$, $\hat{\spc{X}}$, and $\hat{\spc{Y}}$ respectively.

According to \ref{ex:d-p-inclusion}, the inclusions $\spc{X}\hookrightarrow\spc{U}$ and $\spc{Y}\hookrightarrow\spc{U}$ can be extended to a distance-preserving inclusions $\hat{\spc{X}}\hookrightarrow\hat{\spc{U}}$ and $\hat{\spc{Y}}\hookrightarrow\hat{\spc{U}}$.
Therefore, we can and will consider  $\hat{\spc{X}}$ and $\hat{\spc{Y}}$ as subspaces of $\hat{\spc{U}}$.

Given $f\in \hat{\spc{U}}$,
let us find $g\in \hat{\spc{X}}$ such that 
\[|f(u)-g(u)|<2\cdot\eps\eqlbl{|g-f|}\]
for any $u\in\spc{U}$.
Note that the restriction $f|_{\spc{X}}$ is admissible on ${\spc{X}}$.
By \ref{obs:extremal:below}, there is $g\in \hat{\spc{X}}$ such that 
\[g(x)\le f(x)\eqlbl{g(x)=<f(x)}\]
for any $x\in\spc{X}$.

Recall that any extremal function is $1$-Lipschitz;
in particular, $f$ and $g$ are $1$-Lipschitz on $\spc{U}$.
Therefore, \ref{g(x)=<f(x)} and $|\spc{X}-\spc{Y}|_{\spc{U}}<\eps$ imply that
\[g(u)< f(u)+2\cdot \eps\]
for any $u\in\spc{U}$.
By \ref{ex:+-c}, we also have 
\[g(u)> f(u)-2\cdot \eps\]
for any $u\in\spc{U}$.
Whence \ref{|g-f|} follows.

It follows that $\hat{\spc{Y}}$ lies in a $2\cdot\eps$-neighborhood of $\hat{\spc{X}}$ in $\hat{\spc{U}}$.
The same way we show that $\hat{\spc{X}}$ lies in a $2\cdot\eps$-neighborhood of $\hat{\spc{Y}}$ in $\hat{\spc{U}}$.
The latter means that
$|\hat{\spc{X}}-\hat{\spc{Y}}|_{\Haus\hat{\spc{U}}}<2\cdot\eps$,
and therefore
$|\hat{\spc{X}}-\hat{\spc{Y}}|_{\GH}<2\cdot\eps$.

\parit{Remark.} 
This problem was discussed by Urs Lang, Maël Pavón, and Roger Züst \cite[3.1]{lang-pavon-zust}.
\begin{figure}[h!]
\vskip-0mm
\centering
\includegraphics{mppics/pic-505}
\end{figure}
They also show that the constant 2 is optimal.
To see this, look at the injective envelopes of two 4-point metric spaces shown on the diagram and observe that the Gromov--Hausdorff distance between the 4-point metric spaces is 1, while the distance between their injective envelopes approaches 2 as $s\to\infty$. 

\parbf{\ref{ex:H-R}}; \textit{only-if part.}
Let us identify $\spc{X}$ and $\spc{Y}$ with subspaces of a metric space $\spc{Z}$ such that 
\[|\spc{X}-\spc{Y}|_{\Haus \spc{Z}}<\eps.\]

Set $x\approx y$ if and only if $\dist{x}{y}{\spc{Z}}<\eps$.
It remains to check that $\approx$ is an $\eps$-approximation.

\parit{If part.}
Show that we can assume that 
\[R=\set{(x,y)\in\spc{X}\times\spc{Y}}{x\approx y}\] is a compact subset of $\spc{X}\times\spc{Y}$.
Conclude that
\[\bigl|\dist{x}{x'}{\spc{X}}-\dist{y}{y'}{\spc{Y}}\bigr|<2\cdot\eps'\]
for some $\eps'<\eps$.

Show that there is a metric on $\spc{Z}=\spc{X}\sqcup\spc{Y}$ such that the inclusions $\spc{X}\hookrightarrow\spc{Z}$ and
$\spc{Y}\hookrightarrow\spc{Z}$ are distance preserving and $\dist{x}{y}{\spc{Z}}=\eps'$ if $x\approx y$.
Conclude that 
\[|\spc{X}-\spc{Y}|_{\Haus \spc{Z}}\le\eps'<\eps.\]

\parbf{\ref{ex:eps-isom}};
\ref{SHORT.ex:eps-isom:GH>isom}.
Let $\approx$ be an $\eps$-approximation provided by \ref{ex:H-R}.
For any $x\in\spc{X}$ choose a point $f(x)\in\spc{Y}$ such that $x\approx f(x)$.
Show that $x\mapsto f(x)$ is an $2\cdot\eps$-isometry.

\parit{\ref{SHORT.ex:eps-isom:isom>GH}.}
Let $x\in\spc{X}$ and $y\in\spc{Y}$.
Set $x\approx y$ if $\dist{y}{f(x)}{\spc{Y}}<\eps$.
Show that $\approx$ is a $\eps$-approximation. 
Apply \ref{ex:H-R}.

\parbf{\ref{ex:GH-SC}}; \ref{SHORT.ex:GH-SC:circle}.
Suppose $\spc{X}_n\GHto \spc{X}$ and $\spc{X}_n$ are simply connected length metric space.
It is sufficient to show that any nontrivial covering map $f\:\tilde{\spc{X}}\to \spc{X}$ corresponds to a nontrivial covering map $f_n\:\tilde{\spc{X}}_n\to \spc{X}_n$ for large $n$.

The latter can be constructed by covering $\spc{X}_n$ by small balls that lie close to sets in $\spc{X}$ evenly covered by $f$, prepare a few copies of these sets and glue them the same way as the inverse images of the evenly covered sets in $\spc{X}$ glued to obtain $\tilde{\spc{X}}$.

\begin{wrapfigure}{r}{40 mm}
\vskip-0mm
\centering
\includegraphics{mppics/pic-2}
\end{wrapfigure}

\parit{\ref{SHORT.ex:GH-SC:nonsc-limit}.}
Let $\spc{V}$ be a cone over Hawaiian earrings.
Consider the {}\emph{doubled cone} $\spc{W}$ --- two copies of $\spc{V}$ with glued their base points (see the diagram).

The space $\spc{W}$ can be equipped with a length metric
(for example, the induced length metric from the shown embedding).

Show that $\spc{V}$ is simply connected, but $\spc{W}$ is not; the latter is a good exercise in topology.

If we delete from the earrings all small circles, then the obtained double cone becomes simply connected and it remains to be close to $\spc{W}$.
That is $\spc{W}$ is a Gromov--Hausdorff limit of simply connected spaces.

\parit{Remark.}
Note that from part \ref{SHORT.ex:GH-SC:nonsc-limit}, the limit does not admit a nontrivial covering.
So, if we define the fundamental group as the inverse image of groups of deck transformations for all the coverings of the given space, then one may say that Gromov--Hausdorff limit of simply connected length spaces is simply connected.

\parbf{\ref{ex:sphere-to-ball},}
\textit{\ref{SHORT.ex:sphere-to-ball:2}.}
Suppose that a metric on $\mathbb{S}^2$ is close to the disk $\DD^2$.
Note that $\mathbb{S}^2$ contains a circle $\gamma$ that is close to the boundary curve of $\DD^2$.
By the Jordan curve theorem, $\gamma$ divides $\mathbb{S}^2$ into two disks, say $D_1$ and $D_2$.

By \ref{ex:GH-SC:nonsc-limit}, the Gromov--Hausdorff limits of $D_1$ and $D_2$ have to contain the whole $\DD^2$, otherwise the limit would admit a nontrivial covering.

Consider points $p_1\in D_1$ and $p_2\in D_2$ that are close to the center of $\DD^2$.
If $n$ is large, the distance $\dist{p_1}{p_2}{n}$ has to be very small.
On the other hand, any curve from $p_1$ to $p_2$ must cross $\gamma$, so it has length about 2 --- a contradiction.

\parbf{\ref{ex:utb+pack}.} Apply \ref{ex:pack-net}.

\parbf{\ref{pr:doubling}.}
Choose a space $\spc{X}$ in $\spc{Q}(C,D)$, denote a $C$-doubling measure by~$\mu$.
Without loss of generality, we may assume that $\mu(\spc{X})\z=1$.

The doubling condition implies that 
\[\mu[\oBall(p,\tfrac D{2^n})]\ge\tfrac 1{C^n}\]
for any point $x\in \spc{X}$.
It follows that 
\[\pack_{\frac D{2^n}}\spc{X}\le C^n.\]

By \ref{ex:pack-net}, for any $\eps\ge\frac D{2^{n-1}}$, the space $\spc{X}$ admits an $\eps$-net with at most $C^n$ points.
Whence $\spc{Q}(C,D)$ is uniformly totally bounded.

\parbf{\ref{pr:under}}; \ref{SHORT.pr:under:if}.
Choose $\eps>0$.
Since $\spc{Y}$ is compact, we can choose a finite $\eps$-net $\{y_1,\dots,y_{n}\}$ in $\spc{Y}$.

Suppose $f\:\spc{X}\to \spc{Y}$ be a distance-nondecreasing map.
Choose one point $x_i$ in each nonempty subset $B_i=f^{-1}[\oBall(y_i,\eps)]$.
Note that the subset $B_i$ has diameter at most $2\cdot \eps$ and 
\[\spc{X}=\bigcup_iB_i.\]
Therefore, the set of points $\{x_i\}$ is a $2\cdot\eps$-net in $\spc{X}$.

\parit{\ref{SHORT.pr:under:only-if}.} Let $\spc{Q}$ be a uniformly totally bounded family of spaces. 
Suppose that each space in $\spc{Q}$ has an $\tfrac1{2^n}$-net with at most $M_n$ points; we may assume that $M_0=1$.

Consider the space $\spc{Y}$ of all infinite integer sequences $m_0,m_1,\dots$ such that $1\le m_n\le M_n$ for any $n$.
Given two sequences $(\ell_n)$, and $(m_n)$ of points in $\spc{Y}$, set 
\[\dist{(\ell_n)}{(m_n)}{\spc{Y}}=\tfrac C{2^{n}},\]
where $n$ is the minimal index such that $\ell_n\ne m_n$ and $C$ is a positive constant.

Observe that $\spc{Y}$ is compact.
Indeed it is complete and the sequences with constant tails, starting from index $n$, form a finite $\tfrac C{2^{n}}$-net in $\spc{Y}$.

Given a space $\spc{X}$ in $\spc{Q}$,
choose a sequence of $\tfrac1{2^n}$ nets 
$N_n\subset\spc{X}$ for each $n$.
We can assume that $|N_n|\le M_n$; let us label the points in $N_n$ by $\{1,\dots,M_n\}$.
Consider the map $f:\spc{X}\to\spc{Y}$ defined by $f:x\to (m_1(x),m_2(x),\dots)$ where $m_n(x)$ is the label of a point in $N_n$ that lies on the distance $<\tfrac1{2^n}$ from $x$.

If $\tfrac1{2^{n-2}}\ge \dist{x}{x'}{\spc{X}}>\tfrac1{2^{n-1}}$, then $m_n(x)\ne m_n(x')$.
It follows that $\dist{f(x)}{f(x')}{\spc{Y}}\ge \tfrac C{2^{n}}$.
In particular, if $C>10$, then 
\[\dist{f(x)}{f(x')}{\spc{Y}}\ge \dist{x}{x'}{\spc{X}}\]
for any $x,x'\in \spc{X}$.
That is, $f$ is a distance-nondecreasing map $\spc{X}\to \spc{Y}$.

\parbf{\ref{ex-GH-length}}; \ref{SHORT.ex-GH-length:length}
Apply
\ref{ex:Haus-length},
\ref{prop:GH-with-fixed-Z},
\ref{lem:GH-complete},
and \ref{lem:mid>geod}.

\parit{\ref{SHORT.ex-GH-length:geodesic}.}
Choose two compact metric spaces $\spc{X}$ and $\spc{Y}$.
Show that there are subsets $\spc{X}'$, and $\spc{Y}'$ in the Urysohn space $\spc{U}$ that isometric to $\spc{X}$ and $\spc{Y}$ respectively and such that 
\[|\spc{X}-\spc{Y}|_{\GH} = |\spc{X}'-\spc{Y}'|_{\Haus\spc{U}}.\]

Further, construct a sequence of compact sets $\spc{Z}_n\subset \spc{U}$ such that $\spc{Z}_n$ is an $\tfrac1{2^n}$-midpoint of $\spc{X}'$, and $\spc{Y}'$ in $\Haus\spc{U}$ and 
\[|\spc{Z}_n-\spc{Z}_{n+1}|_{\Haus\spc{U}}<\tfrac1{2^n}\]
for any $n$.

Observe that the sequence $\spc{Z}_n$ converges in $\GH$, and its limit by $\spc{Z}$ is a midpoint of $\spc{X}$ and $\spc{Y}$.
Finally, apply \ref{lem:GH-complete} and \ref{lem:mid>geod}.

\parit{Source:} \cite{ivanov-nikolaeva-tuzhilin}.



\parit{\ref{SHORT.ex:sphere-to-ball:3}.}
Make fine burrows in the standard 3-ball without changing its topology,
but at the same time come sufficiently close to any point in the ball.

Consider the \index{doubling}\emph{doubling} of the obtained ball along its boundary;
that is, two copies of the ball with identified corresponding points on their boundaries.
The obtained space is homeomorphic to $\mathbb{S}^3$.
Note that the burrows can be made 
so that the obtained space is sufficiently close to the original ball 
in the Gromov--Hausdorff metric.

\parit{Source:} \cite[Exercises 7.5.13 and 7.5.17]{burago-burago-ivanov}. 

\parbf{\ref{ex:GH-po}}; \ref{SHORT.ex:GH-po:a}.
To check that $\dist{*}{*}{\GH'}$ is a metric, it is sufficient to show that
\[\dist{\spc{X}}{\spc{Y}}{\GH'}=0 
\quad\Longrightarrow\quad
\spc{X}\iso\spc{Y};\]
the remaining conditions are trivial.

If $\dist{\spc{X}}{\spc{Y}}{\GH'}=0$, then there is a sequence of maps $f_n\:\spc{X}\to \spc{Y}$ such that 
\[\dist{f_n(x)}{f_n(x')}{\spc{Y}}\ge \dist{x}{x'}{\spc{X}}-\tfrac1n.\]

Arguing the same way as in the proof of the ``only if''-part in \ref{GH-2} (page~\pageref{page:GH-2-proof}),
we get a distance-nondecreasing map $f_\infty\:\spc{X}\to \spc{Y}$.

The same way we can construct a distance-nondecreasing map 
$g_\infty\:\spc{Y}\to \spc{X}$.

By \ref{ex:non-contracting-map}, the compositions $f_\infty\circ g_\infty\:\spc{Y}\to \spc{Y}$ and $g_\infty\z\circ f_\infty\:\spc{X}\to \spc{X}$ are isometries.
Therefore, $f_\infty$ and $g_\infty$ are isometries as well.

\parit{\ref{SHORT.ex:GH-po:b}.} The implication 
\[|\spc{X}_n-\spc{X}_\infty|_{\GH}\to 0 
\quad\Rightarrow\quad 
\dist{\spc{X}_n}{\spc{X}_\infty}{\GH'}\to 0\]
follows from \ref{ex:eps-isom:GH>isom}. 

Now suppose $\dist{\spc{X}_n}{\spc{X}_\infty}{\GH'}\to 0$.
Show that $\{\spc{X}_n\}$ is a uniformly totally bonded family.

If $\dist{\spc{X}_n}{\spc{X}_\infty}{\GH}\not\to 0$, then we can pass to a subsequence such that $\dist{\spc{X}_n}{\spc{X}_\infty}{\GH}\ge\eps$ for some $\eps>0$.
By Gromov selection theorem, we can assume that $\spc{X}_n$ converges in the sense of Gromov--Hausdorff.
From the first implication, the limit $\spc{X}_\infty'$ has to be isometric to $\spc{X}_\infty$;
on the other hand, $\dist{\spc{X}_\infty'}{\spc{X}_\infty}{\GH}\ge \eps$ --- a contradiction.

\parbf{\ref{ex:GH-urysohn}.}
Apply \ref{thm:compact-homogeneous} and \ref{prop:GH-with-fixed-Z}.

%%%%%%%%%%%%%%%%%%%%%%%%%%%%%%%%

%%%%%%%%%%%%%%%%%%%%%%%%%%%%%%
\refstepcounter{chapter}
\setcounter{eqtn}{0}

\parbf{\ref{ex:ultrakatetov}.} 
Let $F=\set{n\in \NN}{f(n)=n}$; we need to show that $\omega(F)=1$.

Consider an oriented graph $\Gamma$ with vertex set $\NN\setminus F$ such that $m$ is connected to $n$ if $f(m)=n$.
Show that each connected component of $\Gamma$ has at most one cycle.
Use it to subdivide vertices of $\Gamma$ into three sets $S_1$, $S_2$, and $S_3$ such that $f(S_i)\cap S_i=\emptyset$ for each $i$.

Conclude that $\omega(S_1)=\omega(S_2)=\omega(S_3)=0$ and hence \[\omega(F)=\omega(\NN\setminus(S_1\cup S_2\cup S_3))=1.\]

\parit{Source:} 
The presented proof was given by Robert Solovay \cite{solovay}, but
the key statement is due to Miroslav Katětov \cite{katetov}.

\parbf{\ref{ex:linear}.}
Choose a nonprincipal ultrafilter $\omega$ and set $L(\bm{s})=s_\omega$.
It remains to observe that $L$ is linear.

\parit{Remark.} 
This construction identifies ultrafilters with vectors in $(\ell^\infty)^*$.
Recall that $\ell^\infty=(\ell^1)^*$ and $\ell^1\subsetneq(\ell^\infty)^*$.
A principle ultrafilter is a basis vector in $\ell^1$; 
nonprincipal ultrafilters lie in $(\ell^\infty)^*\setminus\ell^1$.
The set of ultrafilters is the closure of basis vectors in $\ell^1$ with respect to weak*-topology on $(\ell^\infty)^*$.


\parbf{\ref{ex:ultrakatetov+}.}
Use \ref{ex:ultrakatetov}.

\parbf{\ref{ex:lim(tree)}.}
Let $\gamma$ be a path from $p$ to $q$ in a metric tree $\spc{T}$.
Assume that $\gamma$ passes thru a point $x$ on distance $\ell$ from $[pq]$.
Then 
\[\length\gamma\ge \dist{p}{q}{}+2\cdot \ell.
\eqlbl{eq:+ell}\]

Suppose that $\spc{T}_n$ is a sequence of metric trees that $\omega$-converges to $\spc{T}_\omega$.
By \ref{obs:ultralimit-is-geodesic}, the space $\spc{T}_\omega$ is geodesic.

The uniqueness of geodesics follows from \ref{eq:+ell}.
Indeed, if for a geodesic $[p_\omega q_\omega]$ there is another geodesic $\gamma_\omega$ connecting its ends, then it has to pass thru a point $x_\omega\notin [p_\omega q_\omega]$.
Choose sequences $p_n,q_n,x_n\in\spc{T}_n$ such that $p_n\to p_\omega$, $q_n\to q_\omega$, and $x_n\to x_\omega$ as $n\to\omega$.
Then 
\begin{align*}
\dist{p_\omega}{q_\omega}{}&=\length\gamma\ge 
\\
&\ge\lim_{n\to\omega}(\dist{p_n}{x_n}{}+\dist{q_n}{x_n}{})\ge
\\
&\ge \lim_{n\to\omega}(\dist{p_n}{q_n}{}+2\cdot\ell_n)=
\\
&=\dist{p_\omega}{q_\omega}{}+2\cdot\ell_\omega.
\end{align*}
Since $x_\omega\notin [p_\omega q_\omega]$, we have that $\ell_\omega>0$ --- a contradiction.

It remains to show that any geodesic triangle $\spc{T}_\omega$ is a tripod.
Consider the sequence of centers of tripods $m_n$ for given sequences of points $x_n,y_n,z_n\in \spc{T}_n$.
Observe that its ultralimit $m_\omega$ is the center of the tripod with ends at $x_\omega,y_\omega,z_\omega\in \spc{T}_\omega$.

\parbf{\ref{ex:ultracompact}.}
Construct $\bm{X}$ and distance-preserving embeddings $\spc{X}_n\hookrightarrow\bm{X}$ that satisfy \ref{propery:GH}.
Given $x_\infty\in \spc{X}_\infty$, choose a sequence $x_n\in \spc{X}_n$ such that $x_n\to x_\infty$ in $\bm{X}$.
Let $x_\omega$ be $\omega$-limit of the sequence $x_n$ in $\bm{X}$.
Note that $x_\omega\in \spc{X}_\infty$.
Show that the map $x_\infty\mapsto x_\omega$ is defined; that is, it does not depend on the choice of the sequence $x_n$.
Further, show that the map $x_\infty\mapsto x_\omega$ is an isometry of $\spc{X}_\infty$.
Make a conclusion.

\parbf{\ref{ex:ultrapower}.}
Further, we consider $\spc{X}$ as a subset of $\spc{X}^\omega$.

\parit{\ref{SHORT.ex:ultrapower:a}.} Follows directly from the definitions.

\parit{\ref{SHORT.ex:ultrapower:compact}.}
Suppose $\spc{X}$ compact.
Given a sequence $x_1,x_2,\dots{}\in\spc{X}$, denote its $\omega$-limit in $\spc{X}^\omega$ by $x^\omega$ and its $\omega$-limit in $\spc{X}$ by $x_\omega$.

Observe that $x^\omega=\iota(x_\omega)$.
Therefore, $\iota$ is onto.

If $\spc{X}$ is not compact, we can choose a sequence $x_n$ such that $\dist{x_m}{x_n}{}>\eps$ for fixed $\eps>0$ and all $m\ne n$.
Observe that
\[\lim_{n\to\omega}\dist{x_n}{y}{\spc{X}}\ge \tfrac\eps2\]
for any $y\in\spc{X}$.
It follows that $x_\omega$ lies at the distance $\ge\tfrac\eps2$ from~$\spc{X}$.

\parit{\ref{SHORT.ex:ultrapower:proper}.}
A sequence of points $x_n$ in $\spc{X}$ will be called $\omega$-bounded if there is a real constant $C$ such that
\[\dist{p}{x_n}{\spc{X}}\le C\] 
for $\omega$-almost all $n$.

The same argument as in \ref{SHORT.ex:ultrapower:compact} shows that any $\omega$-bounded sequence has its $\omega$-limit in $\spc{X}$.
Further, if $(x_n)$ is not  $\omega$-bounded, then 
\[\lim_{n\to\omega}\dist{p}{x_n}{\spc{X}}=\infty;\]
that is, $x_\omega$ does not lie in the metric component of $p$ in $\spc{X}^\omega$.

\parbf{\ref{ex:isom-ultrapower}.}
Let us identify points in $\spc{X}$ with nonnegative integers.
Consider the set $\mathcal{A}$ of all sequences $a_n$ such that $a_0=0$ and $a_{n+1}=a_n+\eps_n\cdot 2^n$ where $\eps_n\in\{0,1\}$ for any $n$.
Observe that $\mathcal{A}$ has cardinality continuum and distinct sequences in $\mathcal{A}$ have distinct $\omega$-limits.
Conclude that the cardinality of $\spc{X}^\omega$ is at least continuum.

Show and use that the spaces $\spc{X}^\omega$ and $(\spc{X}^\omega)^\omega$ have discrete metrics and both have cardinality at most continuum.


\parbf{\ref{ex:ultrapower(ultrapower)}.}
Choose a bijection $\iota\:\NN\to \NN\times \NN$.
Given a set $S\subset \NN$, consider the sequence $S_1$, $S_2,\dots$ of subsets in $\NN$ defined by $m\in S_n$ if $(m,n)\z=\iota(k)$ for some $k\in S$.
Set $\omega_1(S)=1$ if and only if $\omega(S_n)=1$ for $\omega$-almost all $n$.
It remains to check that $\omega_1$ meets the conditions of the exercise.

\parit{Comment.}
It turns out that $\omega_1\ne \omega$ for any $\iota$;
see the post of Andreas Blass \cite{blass}.

\parbf{\ref{ex:two-geodesics-in-ultrapower}.}
Arguing as in \ref{obs:ultrapower-is-geodesic}, we get a pair of points $x$ and $y$ in $\spc{X}$ such that
\[\dist{p}{x}{}+\dist{x}{y}{}+\dist{y}{q}{}=\dist{p}{q}{}\]
and there is no midpoint between $x$ and $y$ in $\spc{X}$
(possibly $p=x$ and $q=y$).
Note that it is sufficient to show that there is a continuum of distinct midpoints in $\spc{X}^\omega$ between $x$ and $y$ in $\spc{X}$.

Since $\spc{X}$ is a length space, we can choose a $\tfrac1n$-midpoint $m_n\in\spc{X}$ between $x$ and $y$.
Note that the sequence $m_n$ contains no converging subsequence.
Conclude that we may pass to a subsequence of $m_n$ such that $\dist{m_i}{m_j}{}>\eps$ for a fixed $\eps>0$ and any $i\ne j$.

Argue as in \ref{ex:isom-ultrapower} to show that there is a continuum of distinct $\omega$-limits of subsequences of $m_n$;
each such limit is a midpoint between $x$ and $y$.

\parit{\ref{SHORT.ex:sphere-in-urysohn:homogeneous}.} 
Use \ref{SHORT.ex:sphere-in-urysohn:sphere}, maybe twice.

\parbf{\ref{ex:notproper-limit}.} Consider the infinite metric $\spc{T}$ tree with unit edges shown
on the diagram.
Observe that $\spc{T}$ is proper.

\begin{Figure}
\vskip-0mm
\centering
\includegraphics{mppics/pic-605}
\end{Figure}

Consider the vertex $v_\omega=\lim_{n\to\omega}v_n$ in the ultrapower $\spc{T}^\omega$.
Observe that $\omega$ has an infinite degree.
Conclude that $\spc{T}^\omega$ is not locally compact.

\parbf{\ref{ex:ultraT}.}
Consider a product of an infinite sequence of two-point spaces.

\parit{Remark.}
There are such examples with cocompact isometric action of finitely generated group \cite{thomas-velickovic}.

\parbf{\ref{ex:Asym(Lob)}.} Assume $\spc{L}$ is the Lobachevsky plane.

\parbf{\ref{SHORT.ex:Asym(Lob):metric-tree}.}
Show that there is $\delta>0$ such that sides of any geodesic triangle in $\spc{L}$ intersect a disk of radius $\delta$.
Conclude that any geodesic triangle in $\Asym\spc{L}$ is a tripod.

\parit{\ref{SHORT.ex:Asym(Lob):homogeneous}.} Observe that $\spc{L}$ is one-point-homogeneous and use it.

\parit{\ref{SHORT.ex:Asym(Lob):continuum}.} 
By \ref{SHORT.ex:Asym(Lob):homogeneous}, it is sufficient to show that $p_\omega$ has a continuum degree.

Choose distinct geodesics $\gamma_1,\gamma_2\:[0,\infty)\z\to L$ that start at a point $p$.
Show that the limits of $\gamma_1$ and $\gamma_2$ run in the different connected components of $(\Asym\spc{L})\setminus \{p_\omega\}$.
Since there is a continuum of distinct geodesics starting at $p$,
we get that the degree of $p_\omega$ is at least continuum.

On the other hand, the set of sequences of points in $\spc{L}$  has cardinality continuum.
In particular, the set of points in $\Asym\spc{L}$ has cardinality at most continuum.
It follows that the degree of any vertex is at most continuum.

The proof for the Lobachevsky space goes along the same lines.

For the infinite three-regular tree, part \ref{SHORT.ex:Asym(Lob):metric-tree} follows from \ref{ex:lim(tree)}.
The three-regular tree is only vertex-homogeneous; the latter is sufficient to prove \ref{SHORT.ex:Asym(Lob):homogeneous}.
No changes are needed in~\ref{SHORT.ex:Asym(Lob):continuum}.

\parit{Remark.}
According to the result of Anna Dyubina and Iosif Polterovich \cite{dyubina-polterovich}, the properties \ref{SHORT.ex:Asym(Lob):homogeneous} and \ref{SHORT.ex:Asym(Lob):continuum} describe the tree $\spc{T}$ up to isometry.
In particular, the asymptotic space of the Lobachevsky plane does not depend on the choice of the ultrafilter and the sequence $\lambda_n\to \infty$.


\parbf{\ref{ex:T(Sx[0,1]/Sx0)}.}
Denote by $o_\omega$ the point in $\T^\omega_o\spc{X}$ that corresponds to $o$.
Argue as in \ref{ex:Asym(Lob):continuum} to show that $\T^\omega_o\spc{X}\setminus \{o_\omega\}$ has continuum connected components.
Further, show that each connected component $\spc{W}_\alpha$ is isometric to $\RR\times (0,\infty)$ with the metric described by
\begin{align*}
&\dist{(x_1,t_1)}{(x_2,t_2)}{}=
\\
&\qquad=\min\{\,\dist{(x_1,t_1)}{(x_2,t_2)}{\RR^2},t_1+t_2\,\}.
\end{align*}

Conclude that the space $\T^\omega_o\spc{X}$ can be described as follows.
Consider continuum copies $\spc{W}_\alpha$ as above;
denote by $(x,t)_\alpha$ the point in $\spc{W}_\alpha$ with coordinates $(x,t)$.
The tangent space is the disjoint union of single point $o_\omega$ and all $\spc{W}_\alpha$ 
such that $\dist{(x_1,t_1)_\alpha}{(x_2,t_2)_\alpha}{}$ is the same as in $\spc{W}_\alpha$ and for the remaining pairs, we have $\dist{o_\omega}{(x,t)_\alpha}{}=t$ and $\dist{(x_1,t_1)_\alpha}{(x_2,t_2)_\beta}{}=t_1+t_2$
if $\alpha\ne\beta$.


\end{multicols}
}

\newgeometry{top=0.9in, bottom=0.9in,left=0.9in, right=0.9in, paperwidth=6in, paperheight=9in}

%%%%%%%%%%%%%%%%%%%%%%%%%%%%
{\small\sloppy
\documentclass[twoside]{book}

\usepackage{lectures}
\usepackage[colorlinks=true,
citecolor=black,
linkcolor=black,
anchorcolor=black,
filecolor=black,
menucolor=black,
urlcolor=black,
pdftitle={Pure metric geometry: introductory lectures},
pdfsubject={Geometry},
pdfauthor={Anton Petrunin}
]{hyperref}
\makeindex

\begin{document}
%\pagestyle{empty}\renewcommand\includegraphics[2][{}]{}\def\emph{\textit}
%\overfullrule=100mm

 
\title{Pure metric geometry:\\
introductory lectures}
\author{Anton Petrunin}
\date{}
\maketitle

\section*{Preface}

This text can serve as an introductory part to a variety of courses in metric geometry.
Here is a graph of essential dependencies of the lectures; some statements (mostly exercises) add more dependencies, but they can be ignored.
\begin{figure}[!ht]
\centering
\begin{tikzpicture}[->,>=stealth',shorten >=1pt,auto,scale=1.4,
  thick,main node/.style={circle,draw,font=\sffamily\bfseries,minimum size=8mm}]

  \node[main node] (1) at (1,0) {\ref{chap:defs}};
  \node[main node] (2) at (.5,-5/6){\ref{chap:urysohn}};
  \node[main node] (3) at (1.5,-5/6) {\ref{chap:injective}};
  \node[main node] (4) at (2,0) {\ref{chap:hausdorff}};
  \node[main node] (5) at (3,0) {\ref{chap:GH}};
  \node[main node] (6) at (4,0) {\ref{chap:ultralimits}};
  

  \path[every node/.style={font=\sffamily\small}]
   (1) edge node{}(2)
   (1) edge node{}(3)
   (1) edge node{}(4)
   (4) edge node{}(5)
   (5) edge node{}(6);
\end{tikzpicture}
\end{figure}
The necessary definitions introduced in (\ref{chap:defs}).
In (\ref{chap:urysohn}) we discuss the Urysohn space.
In (\ref{chap:injective}) we discuss injective spaces.
In (\ref{chap:hausdorff}) we introduce Hausdorff metric.
In (\ref{chap:GH}) and (\ref{chap:ultralimits}) we discuss two types of convergences of metric spaces --- the Gromov--Hausdorff limit and ultralimit.

Applications are given only as illustrations.
We stick to domestic affairs of metric spaces, keeping away from any extra structure. 
(Adding an extra structure brings an extra tool and often opens a huge field for development.
The examples include Alexandrov geometry,
geometric group theory,
metric-measure spaces and optimal transport.)

These notes are based on the minicourse given at SPbSU (Fall 2022) and the introductory part of a course at PSU (Spring 2020).
The latter included additional material from \cite{alexander-kapovitch-petrunin-2019,petrunin2020mnfld,nabutovsky}.
A part of the text is a compilation from \cite{alexander-kapovitch-petrunin-2019, alexander-kapovitch-petrunin-2025, petrunin-yashinski, petrunin-2022-PIGTIKAL, petrunin-zamorabarrera} and its drafts.

I want to thank
Sergei Ivanov,
Urs Lang,
Alexander Lytchak,
Rostislav Matveyev,
Julien Melleray,
and Sergio Zamora Barrera for help.
The present work is partially supported by NSF grant DMS-2005279
and the Simons Foundation grant \#584781.

\thispagestyle{empty}
\tableofcontents
\thispagestyle{empty}

\chapter{Definitions}

In this lecture we give some conventions used further
and remind some the definitions related to metric spaces.


\section{Metric spaces}
\label{sec:metric spaces}

The distance between two points $x$ and $y$ in a metric space $\spc{X}$ will be denoted by $\dist{x}{y}{}$ or $\dist{x}{y}{\spc{X}}$.
The latter notation is used if we need to emphasize 
that the distance is taken in the space~${\spc{X}}$.

Let us recall the definition of metric. 

\begin{thm}{Definition}\label{def:metric}
A \index{metric}\emph{metric} on a set $\spc{X}$ is a real-valued function $(x,y)\mapsto\dist{x}{y}{\spc{X}}$ that satisfies the following conditions for any three points $x,y,z\in \spc{X}$:
\begin{enumerate}[(i)]
\item $\dist{x}{y}{\spc{X}}\ge 0$,
\item\label{metric=0} $\dist{x}{y}{\spc{X}}= 0$ $\iff$ $x=y$,
\item $\dist{x}{y}{\spc{X}}=\dist{y}{x}{\spc{X}}$,
\item $\dist{x}{y}{\spc{X}}+\dist{y}{z}{\spc{X}}\ge\dist{x}{z}{\spc{X}}$,
\end{enumerate}
\end{thm}

A set $\spc{X}$ with a metric on it is called \index{metric space}\emph{metric space};
most of the time we keep the same notation for the metric space and its underlying set.

The function 
\[\distfun_x\:y\mapsto \dist{x}{y}{}\]
is called the \index{distance function}\emph{distance function} from~$x$. 

Given $R\in[0,\infty]$ and $x\in \spc{X}$, the sets
\begin{align*}
\oBall(x,R)&=\{y\in \spc{X}\mid \dist{x}{y}{}<R\},
\\
\cBall[x,R]&=\{y\in \spc{X}\mid \dist{x}{y}{}\le R\}
\end{align*}
are called, respectively, the  \index{open ball}\emph{open} and  the \index{closed ball}\emph{closed  balls}   of radius $R$ with center~$x$.
Again, if we need to emphasize that these balls are taken in the metric space $\spc{X}$,
we write 
\[\oBall(x,R)_{\spc{X}}\quad\text{and}\quad\cBall[x,R]_{\spc{X}}.\]

\begin{thm}{Exercise}
Show that
\[\dist{p}{q}{\spc{X}}+\dist{x}{y}{\spc{X}}\le\dist{p}{x}{\spc{X}}+\dist{p}{y}{\spc{X}}+\dist{q}{x}{\spc{X}}+\dist{q}{y}{\spc{X}}\]
for any points $p$, $q$, $x$, and $y$ in a metric space $\spc{X}$.
\end{thm}

\section{Variations of definition}

\parbf{Pseudometrics.}
A metric for which the distance between two distinct points can be zero is called a \index{pseudometric}\emph{pseudometric}.
In other words, to define pseudometric, we need to remove condition (\ref{metric=0}) from \ref{def:metric}.

The following observation show that
nearly any question about pseudometric spaces can be reduced to a question about genuine metric spaces.

Assume $\spc{X}$ is a pseudometric space.
Consider an equivalence relation $\sim$ on $\spc{X}$ defined by
$x\sim y$ if and only if $\dist{x}{y}{}=0$. 
Note that if $x\sim x'$, then $\dist{y}{x}{}=\dist{y}{x'}{}$ for any $y\in\spc{X}$.
Thus, $\dist{*}{*}{}$ defines a metric on the
quotient set $\spc{X}/{\sim}$.
This way we obtain a metric space $\spc{X}'$.
The space $\spc{X}'$ is called the 
\emph{corresponding metric space} for the pseudometric space $\spc{X}$.
Often we do not distinguish between $\spc{X}'$ and~$\spc{X}$. 

\parbf{$\bm{\infty}$-metrics.}
One may also consider metrics with values in $\RR\cup\{\infty\}$;
we might call them \index{metric!$\infty$-metric}\emph{$\infty$-metrics}, but most of the time we use the term {}\emph{metric}.

Again nearly any question about $\infty$-metric spaces can be reduced to a question about genuine metric spaces. 

Indeed, let us write $x\approx y$ if  $\dist{x}{y}{}<\infty$;
this is another equivalence relation on $\spc{X}$.
The equivalence class of a point $x\in\spc{X}$ will be called the \index{metric component}\emph{metric component} 
 of $x$; it will be denoted by $\spc{X}_x$.
One could think of $\spc{X}_x$ as  $\oBall(x,\infty)_{\spc{X}}$ --- the open ball centered at $x$ and radius $\infty$ in $\spc{X}$.

It follows that any $\infty$-metric space is a {}\emph{disjoint union} of genuine metric spaces --- the metric components of the original $\infty$-metric space.

\begin{thm}{Exercise}
Given two sets $A$ and $B$ on the plane, set 
\[\dist{A}{B}{}=\mu(A\backslash B)+\mu(B\backslash A),\]
where $\mu$ denotes the Lebesgue measure.
\begin{subthm}{}
Show that $\dist{*}{*}{}$ is a pseudometric on the set of bounded measurable sets of the plane.
\end{subthm}

\begin{subthm}{}
Show that $\dist{*}{*}{}$ is an $\infty$-metric on the set of all open sets of the plane.
\end{subthm}
\end{thm}

\section{Completeness}

A metric space $\spc{X}$ is called \index{complete space}\emph{complete} if every Cauchy sequence of points in $\spc{X}$ converges in $\spc{X}$.

\begin{thm}{Exercise}\label{ex:almost-min}
Suppose that $\rho$ is a positive continuous function on a complete metric space $\spc{X}$.
Show that for any $\eps>0$ there is a point $x\in \spc{X}$ such that 
\[\rho(x)<(1+\eps)\cdot\rho(y)\]
for any point $y\in \oBall(x,\rho(x))$.
\end{thm}

Most of the time we will assume that a metric space is complete.
The following construction produces a complete metric space $\bar{\spc{X}}$ for any given metric space $\spc{X}$.


\parbf{Completion.}
Given a metric space $\spc{X}$, 
consider the set $\spc{C}$ of all Cauchy sequences in $\spc{X}$.
Note that for any two Cauchy sequences $(x_n)$ and $(y_n)$ the right hand side in \ref{eq:cauchy-dist} is defined; moreover it defines a pseudometric on~$\spc{C}$
\[\dist{(x_n)}{(y_n)}{\spc{C}}\df\lim_{n\to\infty}\dist{x_n}{y_n}{\spc{X}}.\eqlbl{eq:cauchy-dist}\]
The corresponding metric space $\bar{\spc{X}}$ is called a \index{completion}\emph{completion} of $\spc{X}$.

Note that the original space $\spc{X}$ forms a dense subset in its completion $\bar{\spc{X}}$.
More precisely,  for each point $x\in\spc{X}$ one can consider a constant sequence $x_n=x$ which is Cauchy.
It defines a natural map $\spc{X}\to \bar{\spc{X}}$.
It is easy to check that this map is distance-preserving.
In partucular we can (and will) consider $\spc{X}$ as a subset of $\bar{\spc{X}}$.

\begin{thm}{Exercise}
Show that completion of a metric space is complete.
\end{thm}


\section{Compact spaces}

Let us recall few equivalent definitions of compact metric spaces.

\begin{thm}{Definition}\label{def:compact}
A metric space $\spc{K}$ is compact if and only if one of the following equivalent condition holds:

\begin{subthm}{}
 Every open cover of $\spc{K}$ has a finite subcover.
\end{subthm}

\begin{subthm}{}
 For any open cover of $\spc{K}$ there is $\eps>0$ such that any $\eps$-ball in $\spc{K}$ lie in one element of the cover. (The value $\eps$ is called a \index{Lebesgue number}\emph{Lebesgue number} of the covering.)
\end{subthm}

\begin{subthm}{}
 Every sequence of points in $\spc{K}$ has a subsequence that converges in $\spc{K}$.
\end{subthm}

\begin{subthm}{totally-bounded}
The space $\spc{K}$ is complete and \index{totally bounded space}\emph{totally bounded}; that is, for any $\eps>0$, the space $\spc{K}$ admits a finite cover by open $\eps$-balls.
\end{subthm}

\end{thm}

A subset $N$ of a metric space $\spc{K}$ is called \index{net}\emph{$\eps$-net} if any other point $x$ lies on the distance less than $\eps$ from a point in $N$.
Note that totally bounded spaces can be defined as spaces that admit a finite $\eps$-net for any $\eps>0$.

\begin{thm}{Exercise}\label{ex:compact-net}
Show that a space $\spc{K}$ is totally bounded if and only if it contains a compact $\eps$-net for any $\eps>0$. 
\end{thm}


Let $\pack_\eps\spc{X}$ be exact upper bound on the number of points $x_1,\z\dots,x_n\in \spc{X}$ such that $\dist{x_i}{x_j}{}\ge\eps$ if $i\ne j$.

If $n=\pack_\eps\spc{X}<\infty$, then
the collection of points $x_1,\dots,x_n$ is called a \index{maximal packing}\emph{maximal $\eps$-packing}.
Note that $n$ is the maximal number of open disjoint $\tfrac\eps2$-balls in $\spc{X}$.

\begin{thm}{Exercise}\label{ex:pack-net}
Show that a complete space $\spc{X}$ is compact if and only of $\pack_\eps\spc{X}\z<\infty$ for any $\eps>0$.

Show that any maximal $\eps$-packing is an $\eps$-net.
\end{thm}


\begin{thm}{Exercise}\label{ex:non-contracting-map}
Let $\spc{K}$  be a compact metric space and
\[f\:\spc{K}\z\to \spc{K}\] 
be a distance-nondecreasing map.
Prove that $f$ is an \index{isometry}\emph{isometry};
that is, $f$ is a distance-preserving bijection.
\end{thm}

A metric space $\spc{X}$ is called \index{locally compact space}\emph{locally compact} if any point in $\spc{X}$ admits a compact neighborhood;
in other words, for any point $x\in\spc{X}$ a closed ball $\cBall[x,r]$ is compact for some $r>0$.

\section{Proper spaces}

A metric space $\spc{X}$ is called \index{proper space}\emph{proper} if all closed bounded sets in $\spc{X}$ are compact. 
This condition is equivalent to each of the following statements:
\begin{itemize}
\item For some (and therefore any) point $p\in \spc{X}$ and any $R<\infty$, 
the closed ball $\cBall[p,R]_{\spc{X}}$ is compact. 
\item The function $\distfun_p\:\spc{X}\to\RR$ is \index{proper function}\emph{proper} for some (and therefore any) point $p\in \spc{X}$;
that is, for any compact set $K\subset \RR$, its inverse image 
\[\distfun_p^{-1}(K)=\set{x\in \spc{X}}{\dist{p}{x}{\spc{X}}\in K}\]
is compact.
\end{itemize}

\begin{thm}{Exercise}\label{ex:loc-compact-not-proper}
Give an example of space which is locally compact but not proper.
\end{thm}

\section{Geodesics}
\label{sec:geods}

Let $\spc{X}$ be a metric space 
and $\II$\index{$\II$} a real interval. 
A~globally isometric map $\gamma\:\II\to \spc{X}$ is called a \index{geodesic}\emph{geodesic}%
\footnote{Various authors call it differently: {}\emph{shortest path}, {}\emph{minimizing geodesic}.
Also note that the meaning of the term \emph{geodesic} is different from what is used in Riemannian geometry, altho they are closely related.}; 
in other words, $\gamma\:\II\to \spc{X}$ is a geodesic if 
\[\dist{\gamma(s)}{\gamma(t)}{\spc{X}}=|s-t|\]
for any pair $s,t\in \II$.

If $\gamma\:[a,b]\to \spc{X}$ is a geodesic and $p=\gamma(a)$, $q=\gamma(b)$, then we say that $\gamma$ is a geodesic from point $p$ to point $q$.
In this case the image of $\gamma$ is denoted by $[p q]$\index{$[{*}{*}]$} and with an abuse of notations  we also call it a \index{geodesic}\emph{geodesic}.


We may write $[p q]_{\spc{X}}$ 
to emphasize that the geodesic $[p q]$ is in the space  ${\spc{X}}$.
We also use the following shortcut notation:
\begin{align*}
\left] p q \right[&=[pq]\backslash\{p,q\},
&
\left] p q \right]&=[pq]\backslash\{p\},
&
\left[ p q \right[&=[pq]\backslash\{q\}.
\end{align*}

In general, a geodesic from $p$ to $q$ need not exist and if it exists, it need not  be unique.  
However, once we write $[p q]$ we assume mean that we have made a choice of geodesic.

A \index{geodesic path}\emph{geodesic path} is a geodesic with constant-speed parametrization by $[0,1]$.

A curve $\gamma\:\II\to \spc{X}$  is called a \index{geodesic!local geodesic}\emph{local geodesic} if for any $t\in\II$ there is a neighborhood $U$ of $t$ in $\II$ such that the restriction $\gamma|_U$ is a  geodesic.
A constant-speed parametrization of a local geodesic by the unit interval $[0,1]$ is called a \index{geodesic!local geodesic}\emph{local geodesic path}. 

\section{Geodesic spaces and metric trees}

A metric space is called \index{geodesic}\emph{geodesic} if any pair of its points can be joined by a geodesic.

A geodesic space $\spc{T}$ is called a \index{metric tree}\emph{metric tree} if any pair of points in $\spc{T}$ are connected by a unique geodesic,
and the union of any two geodesics $[xy]$, and $[yz]$ contain the geodesic $[xz]_{\spc{T}}$.
In other words any triangle in $\spc{T}$ is a tripod;
that is, for any three geodesics $[xy]$, $[yz]$, and $[zx]$ have a common point.

\begin{thm}{Exercise}
Show that spheres in metric trees are ultrametric spaces;
that is, if $\Sigma$ is a sphere in a metric tree $\spc{T}$, then
\[\dist{x}{z}{\spc{T}}
\le
\max\{\,\dist{x}{y}{\spc{T}},\dist{y}{z}{\spc{T}}\,\}\]
for any $x,y,z\in\Sigma$.
\end{thm}

\section{Length}

A \index{curve}\emph{curve} is defined as a continuous map from a real interval to a metric space.
If the real interval is $[0,1]$, then the curve is called a \index{path}\emph{path}.

\begin{thm}{Definition}
Let $\spc{X}$ be a metric space and
$\alpha\: \II\to \spc{X}$ be a curve.
We define the \index{length}\emph{length} of $\alpha$ as 
\[
\length \alpha \df \sup_{t_0\le t_1\le\ldots\le t_n}\sum_i \dist{\alpha(t_i)}{\alpha(t_{i-1})}{}.
\]

A curve $\alpha$ is called \index{rectifiable curve}\emph{rectifiable} if $\length \alpha<\infty$.
\end{thm}



\begin{thm}{Theorem}\label{thm:length-semicont}
Length is a lower semi-continuous with respect to pointwise convergence of curves. 

More precisely, assume that a sequence
of curves $\gamma_n\:\II\to \spc{X}$ in a metric space $\spc{X}$ converges pointwise 
to a curve $\gamma_\infty\:\II\to \spc{X}$;
that is, for any fixed $t \in \II$, $\gamma_n(t)\z\to\gamma_\infty(t)$ as $n\to\infty$. 
Then 
$$\liminf_{n\to\infty} \length\gamma_n \ge \length\gamma_\infty.\eqlbl{eq:semicont-length}$$
\end{thm}


\begin{wrapfigure}{o}{20 mm}
\vskip-0mm
\centering
\includegraphics{mppics/pic-100}
\end{wrapfigure}


Note that the inequality \ref{eq:semicont-length} might be strict.
For example the diagonal $\gamma_\infty$ of the unit square 
can be  approximated by a stairs-like
polygonal curves $\gamma_n$
with sides parallel to the sides of the square ($\gamma_6$ is on the picture).
In this case
\[\length\gamma_\infty=\sqrt{2}\quad
\text{and}\quad \length\gamma_n=2\]
for any $n$.

\parit{Proof.}
Fix a sequence $t_0<t_1<\dots<t_k$ in $\II$.
Set 
\begin{align*}\Sigma_n
&\df
|\gamma_n(t_0)-\gamma_n(t_1)|+\dots+|\gamma_n(t_{k-1})-\gamma_n(t_k)|.
\\
\Sigma_\infty
&\df
|\gamma_\infty(t_0)-\gamma_\infty(t_1)|+\dots+|\gamma_\infty(t_{k-1})-\gamma_\infty(t_k)|.
\end{align*}

Note that for each $i$ we have 
\[|\gamma_n(t_{i-1})-\gamma_n(t_i)|\to|\gamma_\infty(t_{i-1})-\gamma_\infty(t_i)|\]
and therefore
\[\Sigma_n\to \Sigma_\infty\] 
as $n\to\infty$.
Note that 
\[\Sigma_n\le\length\gamma_n\]
for each $n$.
Hence
$$\liminf_{n\to\infty} \length\gamma_n \ge \Sigma_\infty.\eqlbl{>=Sigma-infty}$$

If $\gamma_\infty$ is rectifiable, we can assume that 
\begin{align*}
\length\gamma_\infty<\Sigma_\infty+\eps.
\end{align*}
for any given $\eps>0$.
By \ref{>=Sigma-infty} it follows that 
$$\liminf_{n\to\infty} \length\gamma_n > \length\gamma_\infty-\eps$$
for any $\eps>0$; whence \ref{eq:semicont-length} follows.

It remains to consider the case when $\gamma_\infty$ is not rectifiable; 
that is, $\length\gamma_\infty=\infty$.
In this case we can choose a partition so that $\Sigma_\infty>L$ for any real number $L$.
By \ref{>=Sigma-infty} it follows that 
$$\liminf_{n\to\infty} \length\gamma_n > L$$
for any given $L$; whence 
\[\liminf_{n\to\infty}\length\gamma_n=\infty\]
and \ref{eq:semicont-length} follows.
\qeds

\section{Length spaces}\label{sec:intrinsic}

If for any $\eps>0$ and any pair of points $x$ and $y$ in a metric space $\spc{X}$, there is a path $\alpha$ connecting $x$ to $y$ such that
\[\length\alpha< \dist{x}{y}{}+\eps,\]
then $\spc{X}$ is called a \index{length space}\emph{length space} and the metric on $\spc{X}$ is called a \index{length metric}\emph{length metric}.\label{page:length metric}

If $\spc{X}$ is an $\infty$-metric space, then in the above definition we assume in addition that $x$ and $y$ lie in one metric component; that is, $\dist{x}{y}{\spc{X}}<\infty$.
In other words an $\infty$-metric space $\spc{X}$ is a length space if each metric component of $\spc{X}$ is a length space.

Note that any geodesic space is a length space.
The following example shows that the converse does not hold.


\begin{thm}{Example}
Suppose a space $\spc{X}$ is obtained by gluing a countable collection of disjoint intervals $\{\II_n\}$ of length $1+\tfrac1n$, where for each $\II_n$ the left end is glued to $p$ and the right end to~$q$.

Observe that the space $\spc{X}$ carries a natural complete length metric with respect to which $\dist{p}{q}{}=1$ but there is no geodesic connecting $p$ to~$q$.
\end{thm}



\begin{thm}{Exercise}\label{ex:no-geod}
Give an example of a complete length space $\spc{X}$ such that no pair of distinct points in $\spc{X}$ can be joined by a geodesic.
\end{thm}

Directly from the definition, it follows that if a path $\alpha\:[0,1]\to\spc{X}$ connects two points $x$ and $y$ 
(that is, if $\alpha(0)=x$ and $\alpha(1)=y$), then 
\[\length\alpha\ge \dist{x}{y}{}.\]
Set 
\[\yetdist{x}{y}{}=\inf\{\length\alpha\}\]
where the greatest lower bound is taken for all paths connecing $x$ and~$y$.
It is straightforward to check that $(x,y)\mapsto \yetdist{x}{y}{}$ is an $\infty$-metric; moreover $(\spc{X},\yetdist{*}{*}{})$ is a length space.
The metric $\yetdist{*}{*}{}$ is called \index{induced length metric}\emph{induced length metric}.

\begin{thm}{Exercise}\label{ex:compact+connceted}
Let $\spc{X}$ be a complete length space.
Show that for any compact subset $K$ in $\spc{X}$
there is a compact path connected subset $K'$ that contains $K$.  
\end{thm}

\begin{thm}{Exercise}\label{ex:compact=>complete}
Suppose $(\spc{X},\dist{*}{*}{})$ is a complete metric space.
Show that $(\spc{X},\yetdist{*}{*}{})$ is complete.
\end{thm}

Let $A$ be a subset of a metric space $\spc{X}$.
Given two points $x,y\in A$,
consider the value
\[\dist{x}{y}{A}=\inf_{\alpha}\{\length\alpha\},\]
where the greatest lower bound is taken for all paths $\alpha$ from $x$ to $y$ in~$A$.
In other words $\dist{*}{*}{A}$ denotes the induced length metric on the subspace $A$.%
\footnote{The notation $\dist{*}{*}{A}$ conflicts with the previously defined notation for distance $\dist{x}{y}{\spc{X}}$ in a metric space $\spc{X}$. However, most of the time we will work with ambient length spaces where the meaning will be unambiguous.}

Let $\spc{X}$ be a metric space and $x,y\in\spc{X}$.

\begin{enumerate}[(i)]
\item A point $z\in \spc{X}$ is called a \index{midpoint}\emph{midpoint} between $x$ and $y$
if 
\[\dist{x}{z}{}=\dist{y}{z}{}=\tfrac12\cdot\dist[{{}}]{x}{y}{}.\]
\item Assume $\eps\ge 0$.
A point $z\in \spc{X}$ is called an \index{$\eps$-midpoint}\emph{$\eps$-midpoint} between $x$ and $y$
if 
\[\dist{x}{z}{},\quad\dist{y}{z}{}\le\tfrac12\cdot\dist[{{}}]{x}{y}{}+\eps.\]
\end{enumerate}


Note that a $0$-midpoint is the same as a midpoint.


\begin{thm}{Lemma}\label{lem:mid>geod}
Let $\spc{X}$ be a complete metric space.
\begin{subthm}{lem:mid>length}
Assume that for any pair of points $x,y\in \spc{X}$  
 and any $\eps>0$
there is an $\eps$-midpoint~$z$.
Then $\spc{X}$ is a length space.
\end{subthm}

\begin{subthm}{lem:mid>geod:geod}
Assume that for any pair of points $x,y\in \spc{X}$, 
there is a midpoint~$z$.
Then $\spc{X}$ is a geodesic space.
\end{subthm}
\end{thm}

\parit{Proof.}
We first prove \ref{SHORT.lem:mid>length}.
Let $x,y\in \spc{X}$ be a pair of points.

Set $\eps_n=\frac\eps{4^n}$, $\alpha(0)=x$ and $\alpha(1)=y$.

Let $\alpha(\tfrac12)$ be an $\eps_1$-midpoint between $\alpha(0)$ and $\alpha(1)$.
Further, let $\alpha(\frac14)$ 
and $\alpha(\frac34)$ be $\eps_2$-midpoints between the pairs $(\alpha(0),\alpha(\tfrac12))$ 
and $(\alpha(\tfrac12),\alpha(1))$ respectively.
Applying the above procedure recursively,
on the $n$-th step we define $\alpha(\tfrac{k}{2^n})$,
for every odd integer $k$ such that $0<\tfrac k{2^n}<1$, 
as an $\eps_{n}$-midpoint between the already defined
$\alpha(\tfrac{k-1}{2^n})$ and $\alpha(\tfrac{k+1}{2^n})$.


In this way we define $\alpha(t)$ for $t\in W$,
where $W$ denotes the set of dyadic rationals in $[0,1]$.
Since $\spc{X}$ is complete, the map $\alpha$ can be extended continuously to $[0,1]$.
Moreover,
\[\begin{aligned}
\length\alpha&\le \dist{x}{y}{}+\sum_{n=1}^\infty 2^{n-1}\cdot\eps_n\le
\\
&\le \dist{x}{y}{}+\tfrac\eps2.
\end{aligned}
\eqlbl{eq:eps-midpoint}
\]
Since $\eps>0$ is arbitrary, we get \ref{SHORT.lem:mid>length}.

To prove \ref{SHORT.lem:mid>geod:geod}, 
one should repeat the same argument 
taking midpoints instead of $\eps_n$-midpoints.
In this case \ref{eq:eps-midpoint} holds for $\eps_n=\eps=0$.
\qeds

Since in a compact space a sequence of $\tfrac1n$-midpoints $z_n$ contains a convergent subsequence, Lemma~\ref{lem:mid>geod} immediately implies

\begin{thm}{Proposition}\label{prop:length+proper=>geodesic}
Any proper length space is geodesic.
\end{thm}

\begin{thm}{Hopf--Rinow theorem}\label{thm:Hopf-Rinow}
Any complete, locally compact length space is proper.
\end{thm}

Before reading the proof, it is instructive to solve \ref{ex:loc-compact-not-proper}.

\parit{Proof.}
Let $\spc{X}$ be a locally compact length space.
Given $x\in \spc{X}$, denote by $\rho(x)$ the supremum of all $R>0$ such that
the closed ball $\cBall[x,R]$ is compact.
Since $\spc{X}$ is locally compact, 
$$\rho(x)>0
\quad\text{for any}\quad
x\in \spc{X}.\eqlbl{eq:rho>0}$$
It is sufficient to show that $\rho(x)=\infty$ for some (and therefore any) point $x\in \spc{X}$.

\begin{clm}{} If $\rho(x)<\infty$, then $B=\cBall[x,\rho(x)]$ is compact.
\end{clm}

Indeed, $\spc{X}$ is a length space;
therefore for any $\eps>0$, 
the set $\cBall[x,\rho(x)-\eps]$ is a compact $\eps$-net in~$B$.
Since $B$ is closed and hence complete, it must be compact.
\claimqeds
Next we claim that
\begin{clm}{} $|\rho(x)-\rho(y)|\le \dist{x}{y}{\spc{X}}$ for any $x,y\in \spc{X}$;
in particular $\rho\:\spc{X}\to\RR$ is a continuous function.
\end{clm}

Indeed, 
assume the contrary; that is, $\rho(x)+|x-y|<\rho(y)$ for some $x,y\in \spc{X}$. 
Then 
$\cBall[x,\rho(x)+\eps]$ is a closed subset of $\cBall[y,\rho(y)]$ for some $\eps>0$.
Then  compactness of $\cBall[y,\rho(y)]$ implies compactness of $\cBall[x,\rho(x)+\eps]$, a contradiction.\claimqeds

Set $\eps=\min\set{\rho(y)}{y\in B}$; the minimum is defined since $B$ is compact and $\rho$ is continuous.
From \ref{eq:rho>0}, we have $\eps>0$.

Choose a finite $\tfrac\eps{10}$-net $\{a_1,a_2,\dots,a_n\}$ in $B=\cBall[x,\rho(x)]$.
The union $W$ of the closed balls $\cBall[a_i,\eps]$ is compact.
Clearly 
$\cBall[x,\rho(x)+\frac\eps{10}]\subset W$.
Therefore $\cBall[x,\rho(x)+\frac\eps{10}]$ is compact,
a contradiction.
\qeds

\begin{thm}{Exercise}\label{exercise from BH}
Construct a geodesic space $\spc{X}$ that is locally compact,
but whose completion $\bar{\spc{X}}$ is neither geodesic nor locally compact.
\end{thm}

\begin{thm}{Advanced exercise}\label{ex:gross}
Show that for any compact length-metric space $X$ there is number $\ell=\ell(X)$ such that for any finite collection of points there is a point $z$ that lies of average distance $\ell$ from the collection;
that is, for any $x_1,\dots,x_n\in X$ there is $z\in X$ such that
\[\tfrac1n\cdot\sum_i|x_i-z|_X=\ell.\]
\end{thm}







\chapter{Universal spaces}\label{chap:urysohn}

The Urysohn space is the main hero of this lecture.
It shares some fundamental properties with classical spaces (spheres, Euclidean, and Lobachevsky spaces),
but also has many counterintuitive properties.

This space often serves as a counterexample to plausible conjectures,
so it is worth to know it.
In addition, this space is beautiful.



\section{Embedding in a normed space}

Recall that a function $v\mapsto |v|$ on a vector space $\spc{V}$ is called \index{norm}\emph{norm} if it satisfies the following condition for any two vectors $v,w\in \spc{V}$ and a scalar $\alpha$:
\begin{itemize}
\item $|v|\ge 0$;
\item $|\alpha\cdot v|=|\alpha|\cdot |v|$;
\item $|v|+|w|\ge|v+w|$.
\end{itemize}

As an example, consider \index{$\ell^\infty$}$\ell^\infty$ --- the space of real sequences equipped with \index{sup-norm}\emph{sup-norm}; that is, the norm of $\bm{a}=(a_1,a_2,\dots)$ is defined by
\[|\bm{a}|_{\ell^\infty}
\df
\sup_n\{\,|a_n|\,\}.\]


It is straightforward to check that for any normed space the function $(v,w)\mapsto |v-w|$ defines a metric on it.
Therefore, any normed space is an example of metric space;
moreover, it is a geodesic space.
Often we do not distinguish normed space from the corresponding metric space.
(By the Mazur--Ulam theorem, the metric remembers the affine structure of the space; so, to recover the original normed space we only need to specify the origin.
A slick proof of this theorem was given by Jussi V\"{a}is\"{a}l\"{a} \cite{vaisala}.)

Recall that \index{diameter}\emph{diameter} of a metric space $\spc{X}$ (briefly $\diam \spc{X}$) is defined as the least upper bound on the distances between pairs of its points;
that is,
\[\diam \spc{X}
\df
\sup\set{\dist{x}{y}{\spc{X}}}{x,y\in \spc{X}}.\]
If $\diam\spc{X}<\infty$, then the space $\spc{X}$ is called \index{bounded space}\emph{bounded}.



\begin{thm}{Lemma}\label{lem:frechet}
Suppose $\spc{X}$ is a bounded \index{separable space}\emph{separable} metric space;
that is, $\spc{X}$ contains a countable, dense set, say $\{w_n\}$.
Given $x\in \spc{X}$, set $a_n(x)=\dist{w_n}{x}{\spc{X}}$.
Then 
\[\iota\:x\mapsto (a_1(x), a_2(x),\dots)\]
defines a distance-preserving embedding $\iota\:\spc{X}\hookrightarrow \ell^\infty$.
\end{thm}

\parit{Proof.} 
By the triangle inequality 
\[|a_n(x)-a_n(y)|\le \dist{x}{y}{\spc{X}}.\eqlbl{eq:a-a=<dist}\]
Therefore, $\iota$ is \index{short map}\emph{short} (in other words, $\iota$ is distance-expanding).

Again by triangle inequality we have 
\[|a_n(x)-a_n(y)|\ge \dist{x}{y}{\spc{X}}-2\cdot\dist{w_n}{x}{\spc{X}}.\]
Since the set $\{w_n\}$ is dense, we can choose $w_n$ arbitrarily close to $x$.
Whence 
\[\sup_n\{\,|a_n(x)-a_n(y)|\,\}\ge \dist{x}{y}{\spc{X}};\eqlbl{eq:a-a>=dist}\]
that is, $\iota$ is distance-noncontracting.

Finally, observe that \ref{eq:a-a=<dist} and \ref{eq:a-a>=dist} imply the lemma.
\qeds

\begin{thm}{Exercise}\label{ex:compact-length}
Show that any compact metric space $\spc{K}$ is isometric to a subspace of a compact geodesic space. 
\end{thm}

The following exercise generalizes the lemma to arbitrary separable spaces.

\begin{thm}{Exercise}\label{ex:frechet}
Suppose $\{w_n\}$ is a countable, dense set in a metric space $\spc{X}$.
Choose $x_0\in \spc{X}$;
given $x\in \spc{X}$, set 
\[a_n(x)=\dist{w_n}{x}{\spc{X}}-\dist{w_n}{x_0}{\spc{X}}.\]
Show that $\iota\:x\mapsto (a_1(x), a_2(x),\dots)$ defines a distance-preserving embedding $\iota\:\spc{X}\hookrightarrow \ell^\infty$.

Conclude that any separable metric space $\spc{X}$ admits a distance-preserving embedding $\iota\:\spc{X}\hookrightarrow \ell^\infty$.
\end{thm}

The following lemma implies that {}\textit{any metric space is isometric to a subset of a normed vector space};
its proof is nearly identical to the proof of \ref{ex:frechet}.
Given a set $\spc{X}$, denote by \index{$\ell^\infty(\spc{X})$}$\ell^\infty(\spc{X})$ the space of all bounded functions on $\spc{X}$ equipped with sup-norm; 
that is,
\[|f-g|_{\ell^\infty}=\sup\set{|f(x)-f(x)}{x\in \spc{X}}.\]

\begin{thm}{Lemma}\label{lem:kuratowski}
Let $x_0$ be a point in a metric space $\spc{X}$.
Then the map $\iota\:\spc{X}\to \ell^\infty(\spc{X})$ defined by 
\[\iota\:x\mapsto (\distfun_x-\distfun_{x_0})\]
is distance-preserving.

In particular, any metric space $\spc{X}$ admits a distance-preserving into $\ell^\infty(\spc{X})$.
\end{thm}

\section{Extension property}
\label{sec:Extension property}

If a metric space $\spc{X}$ is a subspace of a semimetric space $\spc{X}'$, then we say that $\spc{X}'$ is an \index{extension}\emph{extension} of $\spc{X}$.
If in addition, $\diam\spc{X}'\le d$, then we say that $\spc{X}'$ is a {}\emph{$d$-extension}.

If the complement $\spc{X}'\setminus \spc{X}$ contains a single point, say $p$, then $\spc{X}'$ is called a \index{one-point extension}\emph{one-point extension} of $\spc{X}$.
In this case, to define a metric on $\spc{X}'$, it is sufficient to specify the distance function from $p$; that is, a function $f\:\spc{X}\to\RR$ defined by 
\[f(x)\df\dist{p}{x}{\spc{X}'}.\]
Any function $f$ of that type will be called an \index{extension function}\emph{extension function}\label{page:extension function} or {}\emph{$d$-extension function} respectively.

The extension function $f$ cannot be taken arbitrarily --- the triangle inequality implies that 
\[f(x)+f(y)\ge \dist{x}{y}{\spc{X}}\ge |f(x)-f(y)|\]
for any $x,y\in \spc{X}$.
In particular, $f$ is a non-negative 1-Lipschitz function on $\spc{X}$.
For a $d$-extension, we need to assume in addition that $\diam\spc{X}\z\le d$ and $f(x)\le d$ for any $x\in \spc{X}$.
A straightforward check shows that these conditions are necessary and sufficient.

\begin{thm}{Exercise}\label{ex:extension-of-extension}
Let $\spc{X}$ be a subspace of metric space $\spc{Y}$.
Assume $f$ is an extension function on $\spc{X}$.

\begin{subthm}{ex:extension-of-extension:a}
Show that 
\[\bar f(y)
\df
\inf_{x\in \spc{X}} \{\,f(x)+\dist{x}{y}{\spc{Y}}\,\}\]
defines an extension function on $\spc{Y}$.
\end{subthm}

\begin{subthm}{}
Assume that $\diam \spc{Y}\le d$ and $f(x)\le d$ for any $x\in  \spc{X}$.
Show that 
\[\bar f_d
\df
\min \{\, \bar f,d\,\}\]
is a $d$-extension function on $\spc{Y}$.
\end{subthm}

\end{thm}

The functions $\bar f$ and $\bar f_d$ in the above exercise are called \index{Katětov extensions}\emph{Katětov extensions} of $f$ and the minimal possible $\spc{X}$ is called its \index{support of extension function}\emph{support}, briefly \index{$\supp$}$\supp \bar f=\spc{X}$.

\begin{thm}{Definition}\label{def:finite+1}
A metric space $\spc{U}$ meets the \index{extension property}\emph{extension property}  if for any finite subspace $\spc{F}\subset\spc{U}$ and any extension function $f\:\spc{F}\to\RR$ there is a point $p\in \spc{U}$ such that $\dist{p}{x}{}=f(x)$ for any $x\in \spc{F}$.

If we assume in addition that $\diam \spc{U}\le d$ and instead of extension functions we consider only $d$-extension functions, then we arrive at a definition of {}\emph{$d$-extension property}.

If in addition, $\spc{U}$ is separable and complete, then it is called \index{Urysohn space}\emph{Urysohn space} or {}\emph{$d$-Urysohn space} respectively.
\end{thm}


\begin{thm}{Proposition}\label{prop:univeral-separable}
There is a separable metric space with the ($d$-) extension property (for any $d\ge 0$).
\end{thm}

\parit{Proof.}
Choose $d\ge 0$.
Let us construct a separable metric space with  the $d$-extension property.

Let $\spc{X}$ be a metric space such that $\diam\spc{X}\le d$.
Denote by $\spc{X}^d$ the space of all $d$-extension functions on $\spc{X}$ equipped with the metric defined by the sup-norm.
Note that the map $\spc{X} \to \spc{X}^d$ defined by $x\mapsto\distfun_x$ is a distance-preserving embedding,
so we can (and will) treat $\spc{X}$ as a subspace of $\spc{X}^d$; equivalently, $\spc{X}^d$ is an extension of $\spc{X}$.

Let us iterate this construction.
Start with a one-point space $\spc{X}_0$ and consider a sequence of spaces $(\spc{X}_n)$ defined by $\spc{X}_{n+1}\z\df\spc{X}_n^d$.
Note that the sequence is nested;
that is, $\spc{X}_0\subset \spc{X}_1\subset\dots$
and the union
\[\spc{X}_\infty=\bigcup_n\spc{X}_n;\]
comes with metric such that
$\dist{x}{y}{\spc{X}_\infty} = \dist{x}{y}{\spc{X}_n}$
if $x,y\in\spc{X}_n$.

Note that if $\spc{X}$ is compact, then so is $\spc{X}^d$.
It follows that each space $\spc{X}_n$ is compact.
In particular, $\spc{X}_\infty$ is a countable union of compact spaces;
therefore $\spc{X}_\infty$ is separable.

Any finite subspace $\spc{F}$ of $\spc{X}_\infty$ lies in some $\spc{X}_n$ for $n<\infty$.
By construction, given an extension function $f\:\spc{F}\to\RR$,
there is a point $p\in \spc{X}_{n+1}$ that meets the condition in \ref{def:finite+1}.
That is, $\spc{X}_\infty$ has the $d$-extension property.

The construction of a separable metric space with the extension property requires only two changes.
First, the sequence should be defined by $\spc{X}_{n+1}\z\df\spc{X}_n^{d_n}$, where $d_n$ is an increasing sequence such that $d_n\to\infty$.
Second, the point $p$ should be taken in $\spc{X}_{n+k}$ for sufficiently large $k$, so that $d_{n+k}>\max\{f(x)\}$
(here one has to apply \ref{ex:extension-of-extension:a}).%

(Alternatively, one can start with any separable space $\spc{X}_0$ and consider a nested sequence $\spc{X}_0\subset \spc{X}_1\subset{}\dots$ where $\spc{X}_{n+1}$ is the space of all extension functions on $\spc{X}_{n}$ with at most $n+1$ points in its support.
The last condition is needed to keep $\spc{X}_{n}$ separable.)
\qeds

Given a metric space $\spc{X}$, denote by $\spc{X}^\infty$ the space of all extension functions on $\spc{X}$ equipped with the metric defined by the sup-norm.

\begin{thm}{Exercise}\label{ex:inf-extension}
Construct a proper length space $\spc{X}$ such that $\spc{X}^\infty$ is not separable.
\end{thm}


\begin{thm}{Proposition}\label{prop:completion-univeral}
If a metric space $\spc{V}$ meets the ($d$-) extension property, then so does its completion.
\end{thm}

\parit{Proof.} 
Let us assume $\spc{V}$ meets the extension property.
We will show that its completion $\spc{U}=\bar{\spc{V}}$ meets the extension property as well.
The $d$-extension case can be proved along the same lines.

Note that $\spc{V}$ is a dense subset in a complete space $\spc{U}$.
Observe that $\spc{U}$ has the {}\emph{approximate extension property};
that is, if $\spc{F}\z\subset\spc{U}$ is a finite set, $\eps>0$, and $f\:\spc{F}\to \RR$ is an extension function, then
there exists $p\in \spc{U}$ such that
\[\dist{p}{x}{}\lg f(x)\pm\eps\eqlbl{eq:|p-x|><f(x)}\]
for any $x\in\spc{F}$.
Indeed, consider the Katětov extension $\bar f\:\spc{U}\to\RR$ of~$f$.
Since $\spc{V}$ is dense in $\spc{U}$, we can choose a finite set $\spc{F}'\in \spc{V}$ such that for any $x\in \spc{F}$ there is $x'\in \spc{F}'$ with $\dist{x}{x'}{}<\tfrac\eps2$.
Let $p$ be the point provided by the extension property for the restriction $\bar f|_{\spc{F}'}$.
It remains to observe $p$ meets \ref{eq:|p-x|><f(x)}.

It follows that there is a sequence of points $p_n\in \spc{U}$ such that for any $x\in \spc{F}$, 
\[\dist{p_n}{x}{}\lg f(x)\pm\tfrac1{2^n}.\]

Moreover, we can assume that 
\[\dist{p_n}{p_{n+1}}{} < \tfrac1{2^n}\eqlbl{eq:|pn-pn|}\]
for all large $n$.
Indeed, consider the sets $\spc{F}_n=\spc{F}\cup\{p_n\}$ and the functions $f_n\:\spc{F}_n\to\RR$ defined by $f_n(x)\df f(x)$ and
\[f_n(p_n)
\df
\max\set{\bigl|\dist{p_n}{x}{}- f(x)\bigr|}{x\in \spc{F}}\]
 if $x\ne p_n$.
Observe that $f_n$ is an extension function for large $n$ and
$f_n(p_n)\z<\tfrac1{2^n}$.
Therefore, applying the approximate extension property recursively we get~\ref{eq:|pn-pn|}.

Therefore, the sequence $p_n$ is Cauchy.
Note that its limit meets the condition in the definition of extension property (\ref{def:finite+1}).
\qeds

Note that \ref{prop:univeral-separable} and \ref{prop:completion-univeral} imply the following:

\begin{thm}{Theorem}\label{thm:urysohn-exists}
Urysohn space and $d$-Urysohn space exist for any $d>0$.
\end{thm}

Here is a slightly stronger statement:

\begin{thm}{Theorem}\label{thm:urysohn-exists+}
Any separable metric space $\spc{X}$ admits a distance-preserving embedding into an Urysohn space $\spc{U}$ such that any isometry of $\spc{X}$ can be extended to an isometry of $\spc{U}$.
\end{thm}

\parit{Sketch of proof.}
Start with $\spc{X}_0=\spc{X}$ and construct a nested sequence of spaces $\spc{X}_0\subset\spc{X}_1 \subset{}\dots$ as at the alternative end of the proof of~\ref{prop:univeral-separable}.
Note that 
any isometry $\spc{X}_n\to \spc{X}_n$ can be extended to a unique isometry $\spc{X}_{n+1}\to \spc{X}_{n+1}$.
It follows that any isometry of $\spc{X}$ can be extended to an isometry of $\spc{X}'=\bigcup_n\spc{X}_n$.

Now, consider new nested sequence $\spc{X}\subset \spc{X}'\subset \spc{X}''\subset \dots$;
denote its union by $\spc{Y}$.
Arguing as in \ref{prop:univeral-separable} and \ref{prop:completion-univeral} we get that the completion of $\spc{Y}$ is an Urysohn space, say $\spc{U}$, that comes with a distance-preserving inclusion $\spc{X}\hookrightarrow \spc{U}$.

From above, for any isometry of $\spc{X}$ can be extended to isometries of $\spc{X}'$, $\spc{X}''$ and so on.
They all define an isometry of $\spc{Y}$;
passing to its continuous extension, we get an isometry of $\spc{U}$.
\qeds


\section{Universality}

A metric space will be called \index{universal space}\emph{universal} if it has a subspace isometric to any given separable metric space.
In \ref{ex:frechet}, we proved that $\ell^\infty$ is a universal space. 
The following proposition shows that an Urysohn space is universal as well.
Unlike $\ell^\infty$, Urysohn spaces are separable;
so it might be considered as a \textit{better} universal space.
Theorem \ref{thm:compact-homogeneous} will give another reason why Urysohn spaces are better.

\begin{thm}{Proposition}\label{prop:sep-in-urys}
An Urysohn space is universal.
That is, if $\spc{U}$ is an Urysohn space, then any separable metric space $\spc{S}$ admits a distance-preserving embedding $\spc{S}\hookrightarrow\spc{U}$.

Moreover, for any finite subspace $\spc{F}\subset \spc{S}$,
any distance-preserving embedding $\spc{F}\hookrightarrow \spc{U}$ can be extended to a distance-preserving embedding $\spc{S}\hookrightarrow\spc{U}$.

A $d$-Urysohn space is $d$-universal;
that is, the above statements hold provided that $\diam\spc{S}\le d$.  
\end{thm}

\parit{Proof.}
We will prove the second statement;
the first statement is its partial case for $\spc{F}=\emptyset$.

The required isometry will be denoted by $x\mapsto x'$.

Choose a dense sequence of points $s_1,s_2,\dotsc\in\spc{S}$.
We may assume that $\spc{F}=\{s_1,\dots,s_n\}$, so $s_i'\in \spc{U}$ are defined for $i\le n$.

The sequence $s_i'$ for $i>n$ can be defined recursively using the extension property in $\spc{U}$.
Namely, suppose that $s_1',\dots,s_{i-1}'$ are already defined.
Since $\spc{U}$ meets the extension property, there is a point $s_i'\in \spc{U}$ such that
\[\dist{s_i'}{s_j'}{\spc{U}}=\dist{s_i}{s_j}{\spc{S}}\]
for any $j<i$.

The constructed map $s_i\mapsto s_i'$ is distance-preserving.
Therefore it can be continuously extended to the whole $\spc{S}$.
It remains to observe that the constructed map $\spc{S}\hookrightarrow\spc{U}$ is distance-preserving.
\qeds

\begin{thm}{Exercise}\label{ex:geodesics-urysohn}
Show that any two distinct points in an Urysohn space can be joined by an infinite number of distinct geodesics.
\end{thm}

\begin{thm}{Exercise}\label{ex:compact-extension}
Modify the proofs of \ref{prop:completion-univeral} and \ref{prop:sep-in-urys} to prove the following theorem.
\end{thm}

\begin{thm}{Theorem}\label{thm:compact-extension}
Let $K$ be a compact set in a separable space $\spc{S}$.
Then any distance-preserving map from $K$ to an Urysohn space can be extended to 
a distance-preserving map of the whole $\spc{S}$.
\end{thm}

\begin{thm}{Exercise}\label{ex:sc-urysohn}
Show that ($d$-) Urysohn space is simply-connected.
\end{thm}



\section{Uniqueness and homogeneity}

\begin{thm}{Theorem}\label{thm:urysohn-unique}
Suppose $\spc{F}\subset \spc{U}$ and $\spc{F}'\subset \spc{U}'$ be finite isometric subspaces in a pair of ($d$-)Urysohn spaces $\spc{U}$ and $\spc{U}'$.
Then any isometry $\iota\:\spc{F}\leftrightarrow \spc{F}'$ can be extended to an isometry $\spc{U}\leftrightarrow \spc{U}'$.

In particular, ($d$-)Urysohn space is unique up to isometry.
\end{thm}

Note that \ref{prop:sep-in-urys} implies that there are distance-preserving maps $\spc{U}\z\to \spc{U}'$ and $\spc{U}'\to \spc{U}$.
The next exercise shows that it does not solely imply the existence of an isometry $\spc{U}\leftrightarrow \spc{U}'$.

\begin{thm}{Exercise}\label{ex:no-isom}
Construct two metric spaces $\spc{X}$ and $\spc{Y}$ such that 
there are distance-preserving maps $\spc{X}\to \spc{Y}$ and $\spc{Y}\to \spc{X}$, but no isometry $\spc{X}\leftrightarrow \spc{Y}$.
\end{thm}


The following construction uses the idea of \ref{prop:sep-in-urys}, but it is applied \index{back-and-forth}\emph{back-and-forth} to ensure that the obtained distance-preserving map is onto.

\parit{Proof.}
Choose dense sequences $a_1,a_2,\dots{}\in \spc{U}$ and $b'_1,b'_2,\dots{}\in \spc{U}'$.
We can assume that $\spc{F}=\{a_1,\dots,a_n\}$, $\spc{F}'=\{b_1',\dots,b_n'\}$ and $\iota(a_i)=b_i'$ for $i\le n$.

The required isometry $\spc{U}\leftrightarrow \spc{U}'$ will be denoted by $u \leftrightarrow u'$.
Set $a_i=b_i$ and $a'_i=b'_i$ if $i\le n$.

Let us define recursively $a_{n+1}',b_{n+1}, a_{n+2}', b_{n+2},\dots$ --- on the odd step we define the images of $a_{n+1},a_{n+2},\dots$ and on the even steps we define inverse images of $b'_{n+1},b'_{n+2},\dots$
The same argument as in the proof of \ref{prop:sep-in-urys} shows that we can construct two sequences $a_1',a_2',\dots{}\in \spc{U}'$ and $b_1,b_2,\dots\in \spc{U}$ such that
\begin{align*}
\dist{a_i}{a_j}{\spc{U}}&=\dist{a_i'}{a_j'}{\spc{U}'},
\\
\dist{a_i}{b_j}{\spc{U}}&=\dist{a_i'}{b_j'}{\spc{U}'},
\\
\dist{b_i}{b_j}{\spc{U}}&=\dist{b_i'}{b_j'}{\spc{U}'}
\end{align*}
for all $i$ and $j$.

It remains to observe that the constructed distance-preserving bijection defined by $a_i\leftrightarrow a_i'$ and $b_i\leftrightarrow b_i'$ extends
continuously to an isometry $\spc{U}\leftrightarrow \spc{U}'$. 
\qeds

Observe that \ref{thm:urysohn-unique} implies that the Urysohn space (as well as the $d$-Urysohn space) is \index{homogeneous}\emph{finite-set-homogeneous}; that is,
\begin{itemize}
 \item any distance-preserving map from a finite subset to the whole space can be extended to an isometry.
\end{itemize}

Recall that $S(p,r)_{\spc{X}}$ denotes the sphere of radius $r$ centered at $p$ in a metric space $\spc{X}$;
that is, 
$$S(p,r)_{\spc{X}}=\set{x\in \spc{X}}{\dist{p}{x}{\spc{X}}=r}.$$

\begin{thm}{Exercise}\label{ex:sphere-in-urysohn}
Choose $d\in [0,\infty]$.
Denote by $\spc{U}_d$ the $d$-Urysohn space,
so $\spc{U}_\infty$ is the Urysohn space.

\begin{subthm}{ex:sphere-in-urysohn:sphere}
Assume that $L=S(p,r)_{\spc{U}_d}\ne \emptyset$.
Show that $L$ is isometric to $\spc{U}_{\ell}$; find $\ell$ in terms of $r$ and $d$.
\end{subthm}

\begin{subthm}{ex:sphere-in-urysohn:midpoint}
Let $\ell=\dist{p}{q}{\spc{U}_d}$.
Show that the subset $M\subset\spc{U}_d$ of midpoints between $p$ and $q$ is isometric to $\spc{U}_\ell$.
\end{subthm}

\begin{subthm}{ex:sphere-in-urysohn:homogeneous}
Show that $\spc{U}_d$ is \emph{not} countable-set-homogeneous;
that is, there is a distance-preserving map from a countable subset of $\spc{U}_d$ to $\spc{U}_d$ that cannot be extended to an isometry of $\spc{U}_d$.
\end{subthm}

\end{thm}

In fact, the Urysohn space is compact-set-homogeneous; more precisely the following theorem holds.

\begin{thm}{Theorem}\label{thm:compact-homogeneous}
Let $K$ be a compact set in a ($d$-)Urysohn space~$\spc{U}$.
Then any distance-preserving map $K\to \spc{U}$ can be extended to an isometry of $\spc{U}$.
\end{thm}

A proof can be obtained by modifying the proofs of \ref{prop:completion-univeral} and \ref{thm:urysohn-unique}
the same way as it is done in \ref{ex:compact-extension}.

\begin{thm}{Exercise}\label{ex:shere}
Let $S$ be a unit sphere in the Urysohn space $\spc{U}$.
Show that for any two distinct points $x,y\in \spc{U}$ there is a point $z\in S$ such that 
$\dist{x}{z}{}\ne \dist{y}{z}{}$.

Conclude that two isometries of $\spc{U}$ coincide if they coincide on $S$.
\end{thm}

\begin{thm}{Exercise}\label{ex:ext(shere)}
Let $B$ be an open unit ball in the Urysohn space $\spc{U}$.
Show that $\spc{U}\setminus B$ is isometric to $\spc{U}$.

Use it to construct an isometry of a unit sphere $S$ in $\spc{U}$ that cannot be extended to an isometry of $\spc{U}$.
\end{thm}

\begin{thm}{Exercise}\label{ex:katetov}

\begin{subthm}{ex:katetov:inclusion}
Show that there is a distance-preserving inclusion of the Urysohn space $\iota\:\spc{U}\hookrightarrow \spc{U}$ 
such that $\spc{U}'=\iota(\spc{U})$ is nowhere dense in $\spc{U}$ and any isometry of $\spc{U}'$ 
can be extended to an isometry of the whole~$\spc{U}$.
\end{subthm}

\begin{subthm}{ex:katetov:sol}
Consider a nested sequence $\spc{U}_0\subset \spc{U}_1\subset\dots$ of Urysohn spaces 
with each inclusion $\spc{U}_n\hookrightarrow \spc{U}_{n+1}$ as in \ref{SHORT.ex:katetov:inclusion}.
Show that the union $\bigcup_n\spc{U}_n$ is a noncomplete finite-set-homogeneous metric space that meets the extension property.
\end{subthm}

\end{thm}

{\sloppy

\begin{thm}{Exercise}\label{ex:homogeneous}
Which of the following metric spaces are 
one-point-homogeneous, finite-set-homogeneous, compact-set-homogeneous, countable-set-homogeneous?

\begin{subthm}{ex:homogeneous:euclidean}
Euclidean plane,
\end{subthm}

\begin{subthm}{ex:homogeneous:hilbert}
 Hilbert space $\ell^2$,
\end{subthm}

\begin{subthm}{ex:homogeneous:ell-infty}
 $\ell^\infty$,
\end{subthm}

\begin{subthm}{ex:homogeneous:ell-1}
 \index{$\ell^1$}$\ell^1$ --- the space of all real absolutely converging series $\bm{a}\z=(a_1,a_2,\dots)$ with the norm $|\bm{a}|_{\ell^1}=\sum_i|a_i|$.
 
\end{subthm}
\end{thm}

}

\begin{thm}{Exercise}\label{ex:homogeneous-tree}
Show that any separable one-point-homogeneous metric tree is isometric to the real line $\RR$ or the one-point space.
\end{thm}


\section{Remarks}

The statement in \ref{ex:frechet} was proved by Maurice René Fréchet in the paper where he first defined metric spaces \cite{frechet};
its extension \ref{lem:kuratowski} was given by Kazimierz Kuratowski~\cite{kuratowski}.

The following two exercises show that in this respect $\ell^\infty$ is very different from $\ell^1$.
For more on the subject, see \cite{deza-laurent}.

Let $S$ be a subset of $X$.
The \index{cut metric}\emph{cut metric} $\delta_S$ on $X$ is a semimetric such that $\delta_S(x, y) = 1$ if $x$ and $y$ are separated
by $S$ and otherwise $\delta_S(x, y) = 0$.


\begin{thm}{Exercise}\label{ex:cut}
Show that a finite metric space $\spc{F}$ admits a distance-preserving embedding into $\ell^1$ if and only if the metric of $\spc{F}$ can be written as a nonnegative linear combination%
\footnote{that is, linear combination with nonnegative coefficients.} of cut metrics on $\spc{F}$.
\end{thm}

Recall that the vertex set of any graph $\Gamma$ comes with the shortest-path distance ---
the distance between two vertices is the minimal number of edges in a path connecting them.

\begin{wrapfigure}{r}{15mm}
\vskip-8mm
\centering
\includegraphics{mppics/pic-210}
\end{wrapfigure}

\begin{thm}{Exercise}\label{ex:K23}
Use \ref{ex:cut} to show that the metric for complete biparted graph $K_{2,3}$ (see the diagram) does not admit a distance-preserving embedding into $\ell^1$.
\end{thm}

The question about existence of a separable universal space was posed by Maurice René Fréchet and answered by
Pavel Urysohn~\cite{urysohn}.
Exercise \ref{ex:katetov} answers a question posed by Pavel Urysohn \cite[§$2(6)$]{urysohn}.
It was solved by Miroslav Katětov \cite{katetov},
but long after that, it was again mentioned as an open problem \cite[p. 83]{gromov-2007}.

The idea of Urysohn's construction was reused in graph theory; it produces the so-called \index{Rado graph}\emph{Rado graph},
also known as {}\emph{Erd\H{o}s--Rényi graph} or \emph{random graph}; see \cite{cameron}.
In fact, the Urysohn space is the random metric space in \textit{certain sense} \cite{vershik}.

\textit{The ($d$-) Urysohn space is homeomorphic to the Hilbert space};
the latter was proved by Vladimir Uspenskij \cite{uspenskij} using the so-called Toruńczyk criterion.

The finite-set-homogeneous spaces include Euclidean spaces, hyperbolic spaces, and spheres all with standard length metrics and arbitrary finite dimensions.
In fact, these are the only examples of locally compact three-point-homogeneous length spaces.
The latter was proved by Herbert Busemann \cite{busemann-1942}; it also follows from the more general result of Jacques Tits about two-point-homogeneous spaces \cite{tits}.
The same conclusion holds for complete all-set-homogeneous geodesic spaces with local uniqueness of geodesics;
it was proved by Garrett Birkhoff \cite{birkhoff}.
The answer might be the same for complete separable all-set-homogeneous length spaces.
Without the separability condition, we also get the so-called \emph{universal metric trees} with finite valence \cite{dyubina-polterovich}; no other examples seem to be known \cite{lebedeva-petrunin2211.09671}.  

{\sloppy

\begin{thm}{Exercise}\label{ex:RP-not}
Show that the real projective plane $\RP^2$ with the standard metric is two-point-homogeneous, but not three-point-homogeneous.
\end{thm}

}

\begin{thm}{Exercise}\label{ex:hom-cube}
Let $Q$ be the set of vertices on the $n$-dimensional cube;
assume $n$ is large.
Show that $Q$ is three-point-homogeneous, but not four-point-homogeneous.
\end{thm}

I do not know examples of metric spaces that are $n$-point-homogeneous, but not $(n+1)$-point-homogeneous for large $n$ \cite{petrunin-431426}.

\chapter{Injective spaces}\label{chap:injective}


Injective hull is a useful construction that provides a canonical choice of a specially nice (injective) space that includes a given metric space. 
This construction is similar to the convex hull in Euclidean space.
The following exercise gives a bridge from the latter to the former.

\begin{thm}{Advanced exercise}\label{ex:conv-short}
Show that $A\subset \RR^n$ is a closed convex set if and only if for any  $B\subset \RR^n$ any short map $B\to A$ can be extended to a short map $\RR^n\to A$.
\end{thm}

\section{Definition}

\begin{thm}{Definition}\label{def:injective}
A metric space $\spc{Y}$ is called \index{injective space}\emph{injective} if for any metric space $\spc{X}$ and any of its subspace $\spc{A}$,
any short map $f\:\spc{A}\to \spc{Y}$ can be extended to a short map $F\:\spc{X}\to \spc{Y}$;
that is, $f=F|_{\spc{A}}$.
\end{thm}

\begin{thm}{Exercise}\label{ex:inj=complete-geodesic-contractible}
Show that any injective space is 
\begin{multicols}{3}

\begin{subthm}{ex:inj=complete-geodesic-contractible:complete}
complete,
\end{subthm}

\begin{subthm}{ex:inj=complete-geodesic-contractible:geodesic}
geodesic, and
\end{subthm}

\begin{subthm}{ex:inj=complete-geodesic-contractible:contractible}
contractible.
\end{subthm}

\end{multicols}

\end{thm}

\begin{thm}{Exercise}\label{ex:bicombing}
Show that for any injective space $\spc{Y}$ there is a map $m\:\spc{Y}\times\spc{Y}\to\spc{Y}$ (the \index{midpoint map}\emph{midpoint map}) such that the inequality
\[2\cdot \dist{p}{m(x,y)}{\spc{Y}}\le\dist{p}{x}{\spc{Y}}+\dist{p}{y}{\spc{Y}}\]
holds for any $p,x,y\in \spc{Y}$.
\end{thm}

\begin{thm}{Exercise}\label{ex:injective-spaces}
Show that the following spaces are injective:
\begin{subthm}{ex:injective-spaces:R}
the real line;
\end{subthm}

\begin{subthm}{ex:injective-spaces:tree}
complete metric tree;
\end{subthm}

\begin{subthm}{ex:injective-spaces:ell-infty}
The space $\ell^\infty(\spc{S})$ for any set $\spc{S}$ (defined in \ref{lem:kuratowski}).
In particular, the coordinate plane with the metric induced by the $\ell^\infty$-norm.
\end{subthm}

\end{thm}

\begin{thm}{Exercise}\label{ex:extr-ball}
Let $\spc{Y}$ be an injective space.

\begin{subthm}{ex:extr-ball:one}
Show that any closed ball in $\spc{Y}$ is injective.
\end{subthm}

\begin{subthm}{ex:extr-ball:many}
Show that the intersection of an arbitrary collection of closed balls in $\spc{Y}$ is injective.
\end{subthm}

\end{thm}

\begin{thm}{Advanced exercise}\label{ex:extr-fixed}
Let $\spc{Y}$ be a bounded injective space.
Show that any short map $s\:\spc{Y}\to\spc{Y}$ has a fixed point. 
\end{thm}


\section{Admissible and extremal functions}

Let $\spc{X}$ be a metric space.
A function $r\:\spc{X}\to(-\infty,\infty]$ is called \label{page:admissible function}\index{admissible function}\emph{admissible} if the following inequality
\[r(x)+r(y)\ge \dist{x}{y}{\spc{X}}\eqlbl{eq:admissible}\]
holds for any $x,y\in \spc{X}$.

\begin{thm}{Observation}\label{obs:admissible}

\begin{subthm}{obs:admissible:nonnegative}
Any admissible function is nonnegative.
\end{subthm}

\begin{subthm}{obs:admissible:balls}
If $\spc{X}$ is a geodesic space, then a function $r\:\spc{X}\to\RR$ is admissible if and only if 
\[\cBall[x,r(x)]\cap\cBall[y,r(y)]\ne \emptyset\]
for any $x,y\in \spc{X}$.
\end{subthm}
 
\end{thm}

\parit{Proof; \ref{SHORT.obs:admissible:nonnegative}.} Apply \ref{eq:admissible} for $x=y$.

\parit{\ref{SHORT.obs:admissible:balls}.} Apply the triangle inequality and the existence of a geodesic $[xy]$.
\qeds

A minimal admissible function will be called \label{page:extremal function}\index{extremal function}\emph{extremal}.
More precisely, an admissible function $r\:\spc{X}\to\RR$ is extremal 
if for any admissible function $s\:\spc{X}\to\RR$ we have
\[s\le r\quad\Longrightarrow\quad s=r.\]

Applying Zorn's lemma, we get the following.

\begin{thm}{Observation}\label{obs:extremal:below}
For any admissible function $s$ there is an extremal function $r$ such that $r\le s$.
\end{thm}

\begin{thm}{Lemma}\label{lem:+-c}
Let $r$ be an extremal function and $s$ an admissible function on a metric space $\spc{X}$.
Suppose that $r\ge s-c$ for some constant~$c$.
Then $r\le s+c$; in particular, $c\ge 0$.
\end{thm}

\parit{Proof.}
Note that if $c<0$, then $r>s$.
The latter is impossible since $r$ is extremal and $s$ is admissible.

Observe that the function $\bar r=\min\{\,r,s+c\,\}$ is admissible.
Indeed, choose $x,y\in \spc{X}$.
If $\bar r(x)=r(x)$ and $\bar r(y)=r(y)$, then 
\[\bar r(x)+\bar r(y)=r(x)+ r(y)\ge \dist{x}{y}{}.\]
Further, if $\bar r(x)=s(x)+c$, then 
\begin{align*}
\bar r(x)+\bar r(y)&\ge [s(x)+c]+ [s(y)-c]= 
\\
&=s(x)+s(y) \ge 
\\
&\ge\dist{x}{y}{}.
\end{align*}

Since $r$ is extremal, we have $r=\bar r$;
that is, $r\le s+c$.
\qeds

\begin{thm}{Observations}\label{obs:extremal}
Let $\spc{X}$ be a metric space.

\begin{subthm}{obs:extremal:distfun}
For any point $p\in\spc{X}$ the distance function $r\z=\distfun_p$ is extremal.
\end{subthm}

\begin{subthm}{lem:extremal-lipschitz}
Any extremal function $r$ on $\spc{X}$ is \index{1-Lipschitz function}\emph{1-Lipschitz};
that is,
\[|r(p)-r(q)|\le \dist{p}{q}{}\]
for any $p,q\in\spc{X}$.
In other words, any extremal function is an extension function [see \ref{sec:Extension property}].
\end{subthm}

\begin{subthm}{lem:opposite}
An admissible function $r$ on $\spc{X}$ is extremal if and only if
for any point $p\in\spc{X}$ and any $\delta>0$, there is a point $q\in \spc{X}$
such that 
\[r(p)+r(q)<\dist{p}{q}{\spc{X}}+\delta.\]
\end{subthm}

\begin{subthm}{lem:opposite-compact}
Suppose $\spc{X}$ is compact.
Then an admissible function $r$ on $\spc{X}$ is extremal if and only if
for any point $p\in\spc{X}$ there is a point $q\in \spc{X}$
such that 
\[r(p)+r(q)=\dist{p}{q}{\spc{X}}.\]
\end{subthm}

\end{thm}

\parit{Proof; \ref{SHORT.obs:extremal:distfun}.}
By the triangle inequality, \ref{eq:admissible} holds;
that is, $r=\distfun_p$ is an admissible function.

Further, if $s\le r$ is another admissible function, then $s(p)=0$ and \ref{eq:admissible} implies that $s(x)\z\ge\dist{p}{x}{}$.
Whence $s=r$.

\parit{\ref{SHORT.lem:extremal-lipschitz}.}
By \ref{SHORT.obs:extremal:distfun}, $\distfun_p$ is admissible.
Since $r$ is admissible, we have that
\[r\ge \distfun_p-r(p).\]
Since $r$ is extremal, \ref{lem:+-c} implies that
\[r\le \distfun_p+r(p),\]
or, equivalently,
\[r(q)-r(p)\le \dist{p}{q}{}\]
for any $p,q\in\spc{X}$.
Whence the statement follows.

\parit{\ref{SHORT.lem:opposite}.}
Assume $r$ is extremal.
Arguing by contradiction, assume there is $\delta>0$ such that
\[r(q)\ge \distfun_p(q)-r(p)+\delta\]
for any $q$.
By \ref{SHORT.obs:extremal:distfun}, $\distfun_p$ is extremal; in particular, admissible.
Therefore \ref{lem:+-c} implies that
\[r(q)\le \distfun_p(q)+r(p)-\delta\]
for any $q$.
Taking $q=p$, we get $r(p)\le r(p)-\delta$, a contradiction.

Now suppose $r$ is not extremal; that is, there is an admissible function $s\le r$ such that $r(p)-s(p)=\delta>0$ for some $p$.
Then, for any $q$, we have
\[r(p)+r(q)\ge s(p)+s(q)+\delta\ge \dist{p}{q}{\spc{X}}+\delta\]
--- a contradiction.

\parit{\ref{SHORT.lem:opposite-compact}.}
The if part follows from \ref{SHORT.lem:opposite}.

Denote by $q_n$ the point provided by \ref{SHORT.lem:opposite} for $\delta=\tfrac1n$.
Let $q$ be a partial limit of $q_n$. 
Then 
\[r(p)+r(q)\le\dist{p}{q}{\spc{X}}.\]
Since $r$ is admissible, the opposite inequality holds;
whence the only-if part follows.
\qeds

\begin{thm}{Exercise}\label{ex:circle}
Consider the unit circle 
\[\mathbb{S}^1=\set{(x,y)}{x^2+y^2=1}\]
in the plane with induced length metric.
Show that $r\:\mathbb{S}^1\to\RR$ is extremal if and only if it is 1-Lipschitz and 
\[r(p)+r(-p)=\pi\] for any $p\in\mathbb{S}^1$.
\end{thm}

\begin{thm}{Exercise}\label{ex:retraction}
Given a real-valued function $s$ on a metric space $\spc{X}$,
consider the function
\[s^*(x)=\sup\set{\dist{x}{y}{\spc{X}}-s(y)}{y\in \spc{X}}\]
Show that the function $\tfrac12\cdot(s+s^*)$ is admissible for any $s$.
\end{thm}

\section{Equivalent conditions}

\begin{thm}{Theorem}\label{thm:injective=hyperconvex}
For any metric space $\spc{Y}$ the following conditions are equivalent:

\begin{subthm}{thm:injective=hyperconvex:injective}
$\spc{Y}$ is injective
\end{subthm}


\begin{subthm}{thm:injective=hyperconvex:extremal}
If $r\:\spc{Y}\to\RR$ is an extremal function, then there is a point $p\in \spc{Y}$ such that 
\[\dist{p}{x}{}= r(x)\]
for any $x\in \spc{Y}$.
\end{subthm}

\begin{subthm}{thm:injective=hyperconvex:balls}
$\spc{Y}$ is \index{hyperconvex space}\emph{hyperconvex};
that is, if $\set{\cBall[x_\alpha,r_\alpha]}{\alpha\in\IndexSet}$ is a family of closed balls in $\spc{Y}$ such that 
 \[r_\alpha+r_\beta\ge \dist{x_\alpha}{x_\beta}{}\]
 for any $\alpha,\beta\in \IndexSet$, then all the balls in the family $\{\cBall[x_\alpha,r_\alpha]\}_{\alpha\in\IndexSet}$ have a common point.
\end{subthm}

\end{thm}

\parit{Proof.} We will prove implications 
\ref{SHORT.thm:injective=hyperconvex:injective}$\Rightarrow$\ref{SHORT.thm:injective=hyperconvex:extremal}$\Rightarrow$\ref{SHORT.thm:injective=hyperconvex:balls}$\Rightarrow$\ref{SHORT.thm:injective=hyperconvex:injective}.

\parit{\ref{SHORT.thm:injective=hyperconvex:injective}$\Rightarrow$\ref{SHORT.thm:injective=hyperconvex:extremal}.}
By \ref{lem:extremal-lipschitz}, $r$ is an extension function.
Applying the definition of injective space to a one-point extension of $\spc{Y}$, we get a point $p\in \spc{Y}$ such that 
\[\dist{p}{x}{}=\distfun_p(x)\le r(x)\]
for any $x\in \spc{Y}$.
By \ref{obs:extremal:distfun}, the distance function $\distfun_p$ is extremal.
Since  $r$ is extremal, we get $\distfun_p= r$.


\parit{\ref{SHORT.thm:injective=hyperconvex:extremal}$\Rightarrow$\ref{SHORT.thm:injective=hyperconvex:balls}.}
By \ref{obs:admissible:balls}, part \ref{SHORT.thm:injective=hyperconvex:balls} is equivalent to the following statement:
\begin{itemize}
 \item If $r\:\spc{Y}\to\RR$ is an admissible function, then there is a point $p\in \spc{Y}$ such that 
\[\dist{p}{x}{}\le r(x)\eqlbl{eq:|p-x|=<r(x)}\]
for any $x\in \spc{Y}$.
\end{itemize}
Indeed, set $r(x)\df\inf\set{r_\alpha}{x_\alpha=x}$.
(If $x_\alpha\ne x$ for any $\alpha$, then $r(x)=\infty$.)
The condition in \ref{SHORT.thm:injective=hyperconvex:balls} implies that $r$ is admissible.
It remains to observe that $p\in \cBall[x_\alpha,r_\alpha]$ for every $\alpha$ if and only if \ref{eq:|p-x|=<r(x)} holds.

By \ref{obs:extremal:below}, for any admissible function $r$ there is an extremal function $\bar r\le r$;
hence \ref{SHORT.thm:injective=hyperconvex:extremal}$\Rightarrow$\ref{SHORT.thm:injective=hyperconvex:balls}.

\parit{\ref{SHORT.thm:injective=hyperconvex:balls}$\Rightarrow$\ref{SHORT.thm:injective=hyperconvex:injective}.}
Arguing by contradiction, suppose $\spc{Y}$ is not injective;
that is, there is a metric space $\spc{X}$ with a subset $\spc{A}$
such that a short map $f\:\spc{A}\to \spc{Y}$ cannot be extended to a short map $F\:\spc{X}\to \spc{Y}$.
By Zorn's lemma, we may assume that $\spc{A}$ is a maximal subset; that is, the domain of $f$ cannot be enlarged by a single point.%
\footnote{In this case, $\spc{A}$ must be closed, but we will not use it.}

Fix a point $p$ in the complement $\spc{X}\setminus \spc{A}$.
To extend $f$ to $p$, we need to choose $f(p)$ in the intersection of the balls 
$\cBall[f(x),r(x)]$, where $r(x)=\dist{p}{x}{}$.
Therefore, this intersection for all $x\in \spc{A}$ has to be empty.

Since $f$ is short, we have that 
\begin{align*}
r(x)+r(y)&\ge \dist{x}{y}{\spc{X}}\ge
\\
&\ge \dist{f(x)}{f(y)}{\spc{Y}}.
\end{align*}
By \ref{SHORT.thm:injective=hyperconvex:balls} the balls 
$\cBall[f(x),r(x)]$ have a common point --- a contradiction. 
\qeds

\begin{thm}{Exercise}\label{ex:one-point-gluing}
Suppose a length space $\spc{W}$ has two subspaces $\spc{X}$ and $\spc{Y}$ such that $\spc{X}\cup\spc{Y}=\spc{W}$ and $\spc{X}\cap\spc{Y}$ is a one-point set.
Assume $\spc{X}$ and $\spc{Y}$ are injective.
Show that  $\spc{W}$ is injective
\end{thm}

\begin{thm}{Exercise}\label{ex:Rm-ell-infty}
Show that an $m$-dimensional normed space is injective if and only if it is isometric to $\RR^m$ with $\ell^\infty$-norm; that is,
\[|(x_1,\dots,x_m)|=\max_i\{\,|x_i|\,\}.\]
\end{thm}

A metric space $\spc{Y}$ is called \index{finitely hyperconvex}\emph{finitely hyperconvex} or \index{countably hyperconvex}\emph{countably hyperconvex} if the condition in \ref{thm:injective=hyperconvex:balls} holds only for any finite or respectively countable family of balls.

\begin{thm}{Exercise}\label{ex:compact-hyperconvex}
Show that any proper finitely hyperconvex metric space is hyperconvex.
\end{thm}


\begin{thm}{Exercise}\label{ex:urysohn-hyperconvex}
Show that the $d$-Urysohn space is finitely hyperconvex, but not countably hyperconvex.
Conclude that the $d$-Urysohn space is not injective.

Try to do the same for the Urysohn space.
\end{thm}

\begin{thm}{Exercise}\label{ex:almost-hyperconvex}
Let $\spc{Y}$ be a complete metric space.
Suppose $\spc{Y}$ is \index{almost hyperconvex}\emph{almost hyperconvex},
that is, for any $\eps>0$ any family of closed balls $\set{\cBall[x_\alpha,r_\alpha+\eps]}{\alpha\in\IndexSet}$ has a common point if 
$r_\alpha+r_\beta\ge \dist{x_\alpha}{x_\beta}{}$ for all $\alpha,\beta\in \IndexSet$.
Show that $\spc{Y}$ is hyperconvex.
\end{thm}


\section{Space of extremal functions}
\label{sec:extremal-functions}

Let $\spc{X}$ be a metric space.
Consider the space $\Inj \spc{X}$ of extremal functions on $\spc{X}$ equipped with sup-norm; \label{page:InjX}
that is,
\[\dist{f}{g}{\Inj \spc{X}}\df\sup\set{|f(x)-g(x)|}{x\in \spc{X}}.\]

Recall that by \ref{obs:extremal:distfun}, any distance function is extremal.
It follows that the map $x\mapsto \distfun_x$ produces a distance-preserving embedding $\spc{X}\hookrightarrow\Inj \spc{X}$.
So we can (and will) treat $\spc{X}$ as a subspace of $\Inj \spc{X}$,
or, equivalently, $\Inj \spc{X}$ as an extension of $\spc{X}$.
In particular, from now on, a point $x\in\spc{X}$ can refer to the function $\distfun_x\:\spc{X}\to\RR$ and the other way around.

Since any extremal function is 1-Lipschitz, for any $f\in \Inj \spc{X}$ and $p\in \spc{X}$, we have that
$f(x)\le f(p)+\distfun_p(x)$.
By \ref{lem:+-c}, we also get $f(x)\ge -f(p)+\distfun_p(x)$.
Therefore
\[
\begin{aligned}
\dist{f}{p}{\Inj \spc{X}}&=\sup\set{|f(x)-\distfun_p(x)|}{x\in \spc{X}}=
\\
&=f(p).
\end{aligned}
\eqlbl{eq:f(p)=|f-p|}
\]
In particular, the statement in \ref{lem:opposite} can be written as 
\[\dist{f}{p}{\Inj\spc{X}}+\dist{f}{q}{\Inj\spc{X}}<\dist{p}{q}{\Inj\spc{X}}+\delta.\]

\begin{thm}{Exercise}\label{ex:Inj(compact)}
Show that $\Inj\spc{X}$ is compact if and only if so is $\spc{X}$.
\end{thm}

\begin{thm}{Exercise}\label{ex:tripod+square}
Describe the set of all extremal functions on a metric space $\spc{X}$ and the metric space $\Inj \spc{X}$ in each of the following cases:

\begin{subthm}{ex:tripod+square:2}
$\spc{X}$ is a metric space with exactly two points $v,w$ on distance 1 from each other.
\end{subthm}


\begin{subthm}{ex:tripod+square:tripod} 
$\spc{X}$ is a metric space with exactly three points $a,b,c$ such that 
\[\dist{a}{b}{\spc{X}}=\dist{b}{c}{\spc{X}}=\dist{c}{a}{\spc{X}}=1.\]
\end{subthm}

\begin{subthm}{ex:tripod+square:square}
$\spc{X}$ is  a metric space with exactly four points $p,q,x,y$ such that 
\[\dist{p}{x}{\spc{X}}=\dist{p}{y}{\spc{X}}=\dist{q}{x}{\spc{X}}=\dist{q}{y}{\spc{X}}=1\]
and
\[\dist{p}{q}{\spc{X}}=\dist{x}{y}{\spc{X}}=2.\]
\end{subthm}

\end{thm}

\begin{thm}{Exercise}\label{ex:kur-inj}
Assume $\spc{X}$ is a compact metric space.
Recall that the map $x\mapsto \distfun_x$ gives an isometric embedding $\spc{X}\hookrightarrow\ell^\infty(\spc{X})$; so we can think that $\spc{X}$ is a subset of $\ell^\infty(\spc{X})$.

Given two points $x,y\in \spc{X}$, denote by $G_{x,y}$ the union of all geodesics from $x$ to $y$ in $\ell^\infty(\spc{X})$.
Show that $\Inj\spc{X}$ is isometric to
\[G=\bigcap_{x\in \spc{X}}\left(\bigcup_{y\in \spc{X}}G_{x,y}\right).\]

\end{thm}


\begin{thm}{Proposition}\label{prop:InjX-is-injective}
$\Inj\spc{X}$ is injective for any metric space $\spc{X}$. 
\end{thm}

\begin{thm}{Lemma}\label{lem:r|X-extremal}
Let $\spc{X}$ be a metric space.
Then 
\[\sigma\in \Inj(\Inj \spc{X})
\quad\Longrightarrow\quad
\sigma|_\spc{X}\in \Inj \spc{X}.\]
\end{thm}

In other words, if $\sigma$ is an extremal function on $\Inj \spc{X}$,
then the restriction of $\sigma$ to $\spc{X}$ is an extremal function on $\spc{X}$.

\parit{Proof.}
Arguing by contradiction, suppose that there is an admissible function $s\:\spc{X}\to \RR$ such that $s(x)\le \sigma(x)$ for any $x\in\spc{X}$ and $s(p)\z< \sigma(p)$ for some point $p\in\spc{X}$.
Consider another function $\bar \sigma\:\Inj \spc{X}\to\RR$ such that $\bar \sigma(f)\df \sigma(f)$ if $f\ne p$ and $\bar \sigma(p)\df s(p)$.

Let us show that $\bar \sigma$ is admissible; that is, 
\[\dist{f}{g}{\Inj \spc{X}}\le\bar \sigma(f)+\bar \sigma(g)
\eqlbl{r-admissible}\]
for any $f,g\in \Inj \spc{X}$.

Since $\sigma$ is admissible and $\bar \sigma= \sigma$ on $(\Inj \spc{X})\setminus \{p\}$, it is sufficient to prove \ref{r-admissible} assuming $f\ne g=p$.
By \ref{eq:f(p)=|f-p|}, we have $\dist{f}{p}{\Inj \spc{X}}=f(p)$.
Therefore, \ref{r-admissible} boils down to the following inequality
\[\sigma(f)+s(p)\ge f(p).\eqlbl{eq:r(f)+s(p)>=f(p)}\]
for any $f\in\Inj \spc{X}$.

Fix small $\delta>0$. 
Let $q\in\spc{X}$ be the point provided by \ref{lem:opposite}.
Then
\begin{align*}
\sigma(f)+s(p)&\ge [\sigma(f)-\sigma(q)]+[\sigma(q)+s(p)]\ge
\intertext{since $\sigma$ is 1-Lipschitz, and $\sigma(q)\ge s(q)$, we can continue}
&\ge -\dist{q}{f}{\Inj \spc{X}}+[s(q)+s(p)]\ge
\intertext{by \ref{eq:f(p)=|f-p|} and since $s$ is admissible}
&\ge -f(q)+\dist{p}{q}{}>
\intertext{and by \ref{lem:opposite}}
&> f(p)-\delta.
\end{align*}
Since $\delta>0$ is arbitrary, \ref{eq:r(f)+s(p)>=f(p)} and \ref{r-admissible} follow.

Summarizing: the function $\bar \sigma$ is admissible, $\bar \sigma\le \sigma$ and $\bar \sigma(p)<\sigma(p)$;
that is, $\sigma$ is not extremal --- a contradiction.
\qeds

\parit{Proof of \ref{prop:InjX-is-injective}.}
Choose a function $\sigma\in\Inj(\Inj\spc{X})$.
By \ref{lem:r|X-extremal}, $s\z\df \sigma|_{\spc{X}}\in \Inj\spc{X}$;
that is, $s$ is extremal.
By \ref{thm:injective=hyperconvex:extremal},
it is sufficient to show that  
\[\sigma(f)\ge\dist{s}{f}{\Inj\spc{X}}
\eqlbl{eq:r(f)>=| r-f|}\]
for any $f\in\Inj\spc{X}$.

Since $\sigma$ is $1$-Lipschitz (\ref{lem:extremal-lipschitz}) we have that
\[
s(x)-f(x)=\sigma(x)-\dist{f}{x}{\Inj \spc{X}}\le \sigma(f).
\]
for any $x\in\spc{X}$.
By \ref{lem:+-c},
$
s(x)-f(x)\ge -\sigma(f)
$
for any $x\in\spc{X}$.
Whence \ref{eq:r(f)>=| r-f|} follows.
\qeds

\begin{thm}{Exercise}\label{ex:4-on-a-line}
Let $\spc{X}$ be a compact metric space.
Show that for any two points $f,g\in\Inj \spc{X}$ lie on a geodesic $[pq]$ with $p,q\in \spc{X}$.
\end{thm}

A metric space $\spc{X}$ is called \index{$\delta$-hyperbolic}\emph{$\delta$-hyperbolic} if 
\[\dist{p}{q}{}+\dist{x}{y}{}\le
\max\{\,\dist{p}{x}{}+\dist{q}{y}{},
\,
\dist{p}{y}{}+\dist{q}{x}{}\,\}+2\cdot\delta\]
for any $p,q,x,y\in \spc{X}$.

\begin{thm}{Advanced exercise}\label{ex:delta-hyp}
Show that $\Inj \spc{X}$ is $\delta$-hyperbolic if and only if so is $\spc{X}$.
\end{thm}


\section{Injective envelope}

An extension $\spc{E}$ of a metric space $\spc{X}$ will be called its \index{injective envelope}\emph{injective envelope} if $\spc{E}$ is an injective space, and there is no proper injective subspace of $\spc{E}$ that contains $\spc{X}$.

Two injective envelopes $e\:\spc{X}\hookrightarrow \spc{E}$ and $f\:\spc{X}\hookrightarrow \spc{F}$ are called  equivalent if there is an isometry $\iota\: \spc{E}\to\spc{F}$ such that $f=\iota\circ e$.

\begin{thm}{Theorem}\label{thm:inj-envelope}
For any metric space $\spc{X}$, its extension $\Inj\spc{X}$ is an injective envelope.

Moreover, any other injective envelope of $\spc{X}$ is equivalent to $\Inj\spc{X}$.
\end{thm}

\parit{Proof.} 
Suppose $S\subset \Inj\spc{X}$ is an injective subspace containing $\spc{X}$.
Since $S$ is injective, there is a short map $w\:\Inj\spc{X}\to S$ that fixes all points in $\spc{X}$.

Suppose that $w\:f\mapsto f'$; observe that $f(x)\ge f'(x)$ for any $x\in \spc{X}$.
Since $f$ is extremal, $f=f'$;
that is, $w$ is the identity map, and therefore $S=\Inj\spc{X}$.

Assume we have another injective envelope $e\:\spc{X}\hookrightarrow \spc{E}$.
Then there are short maps $v\:\spc{E}\to \Inj\spc{X}$ and $w\:\Inj\spc{X}\to \spc{E}$ such that $x=v\circ e(x)$ and $e(x)=w(x)$ for any $x\in\spc{X}$.
From above, the composition $v\circ w$ is the identity on $\Inj\spc{X}$.
In particular, $w$ is distance-preserving.

The composition $w\circ v\:\spc{E}\to \spc{E}$ is a short map that fixes points in $e(\spc{X})$.
Since $e\:\spc{X}\hookrightarrow \spc{E}$ is an injective envelope, the composition $w\circ v$ and therefore $w$ are onto.
Whence $w$ is an isometry.
\qeds

\begin{thm}{Exercise}\label{ex:inj-envelope}
Suppose $e\:\spc{X}\hookrightarrow \spc{E}$ and $f\:\spc{X}\hookrightarrow \spc{F}$ are two injective envelopes of $\spc{X}$.
Show that there is a unique isometry $\iota\:\spc{E}\to \spc{F}$ such that $\iota\circ e=f$.
\end{thm}

\begin{thm}{Exercise}\label{ex:d-p-inclusion}
Suppose $\spc{X}$ is a subspace of a metric space $\spc{U}$.
Show that the inclusion $\spc{X}\hookrightarrow\spc{U}$ can be extended to a distance-preserving inclusion $\Inj\spc{X}\hookrightarrow\Inj\spc{U}$.
\end{thm}

\begin{thm}{Exercise}\label{ex:hemisphere-inj}
Consider the hemisphere 
\begin{align*}
\mathbb{S}^2_+&=\set{(x,y,z)\in\RR^3}{x^2+y^2+z^2=1,\quad z\ge0}
\intertext{and its boundary}
\mathbb{S}^1&=\set{(x,y,z)\in\RR^3}{x^2+y^2+z^2=1,\quad z=0};
\end{align*}
 both with induced length metrics.
 
Show that there is unique isometric embedding $\iota\:\mathbb{S}^2_+\hookrightarrow\Inj\mathbb{S}^1$ such that $\iota(u)=u$ for any $u\in \mathbb{S}^1$.
\end{thm}


\section{Remarks}

Injective spaces were introduced by Nachman Aronszajn and Prom Panitchpakdi \cite{aronszajn-panitchpakdi}.
The injective envelope was introduced by John Isbell \cite{isbell}; it is also known as \index{tight span}\emph{tight span} and \index{hyperconvex hull}\emph{hyperconvex hull}.

It was observed by John Isbell \cite{isbell2} that \textit{if $\spc{X}$ is a Banach space, then its injective hull $\Inj\spc{X}$ has a natural structure of Banach space} (which is unique by the Mazur--Ulam theorem).
Moreover, $\spc{X}$ is a linear subspace of $\Inj\spc{X}$.
 
Let us mention that a metric space $\spc{X}$ is called \index{convex space}\emph{convex} if for any pair of points $x_1,x_2\in \spc{X}$ and any $r_1,r_2\ge 0$ we have 
\[r_1+r_2\ge \dist{x_1}{x_2}{\spc{X}}\qquad\Longrightarrow\qquad\cBall[x,r_1]_\spc{X}\cap \cBall[y,r_2]_\spc{X}\ne\emptyset;\]
in other words, a pair of balls intersect if the triangle inequality does not forbid it.
Clearly, hyperconvexity (\ref{thm:injective=hyperconvex:balls}) is stronger than convexity.
Note that \textit{any geodesic space is convex}.
The converse does not hold in general, but by \ref{lem:mid>geod:geod} \textit{any complete convex space is geodesic}.

More generally, a metric space $\spc{X}$ is called \index{$n$-heperconvex space}\emph{$n$-heperconvex} if the condition in \ref{thm:injective=hyperconvex:balls} holds only for families with at most $n$ balls; so \textit{convex means $2$-hyperconvex}.

The following striking result was proved by Benjamin Miesch and Maël Pavón \cite{miesch-pavon2016}.

\begin{thm}{Theorem}
Any complete $4$-hyperconvex space is finitely hyperconvex.
\end{thm}

So, by \ref{ex:compact-hyperconvex}, it follows that \textit{any proper $4$-hyperconvex space is hyperconvex}.

\begin{thm}{Exercise}\label{ex:3-4-hypreconvex}
Show that $\ell^1$ is $3$- but not $4$-hyperconvex.
\end{thm}
 

Recall that if the following inequality
\[\dist{x}{z}{\spc{X}}
\le
\max\{\,\dist{x}{y}{\spc{X}},\dist{y}{z}{\spc{X}}\,\}\]
holds for any three points $x,y,z$ in a metric space $\spc{X}$,
then $\spc{X}$ is called an \index{ultrametric space}\emph{ultrametric space}.
In some sense, ultrametric spaces are dual to injective spaces.

\begin{thm}{Exercise}\label{ex:ultrametric}
Suppose that a metric space $\spc{X}$ satisfies the following property:
For any subspace $\spc{A}$ in $\spc{X}$ and any other metric space $\spc{Y}$, any short map $f\:\spc{A}\to \spc{Y}$ can be extended to a short map $F\:\spc{X}\to \spc{Y}$.

Show that $\spc{X}$ is an ultrametric space.
\end{thm}

A subspace $\spc{S}$ of a metric space $\spc{X}$ is called its \index{short retract}\emph{short retract} if there is a short map $\spc{X}\to \spc{S}$ that is the identity on $\spc{S}$.

\begin{thm}{Exercise}\label{ex:ultrametric-converse}
Show that any compact subspace $\spc{K}$ of an ultrametric space $\spc{X}$ is its short retract.

Construct an example of a complete ultrametric space $\spc{X}$ with a closed subspace $\spc{Q}$ that is not its short retract.
\end{thm}

The following exercise gives a sufficient condition for the existence of a short extension.

\begin{thm}{Exercise}\label{ex:petrunin-stadler}
Let $\spc{X}$ and $\spc{Y}$ be metric spaces, $A\subset \spc{X}$, and $f\:A\z\to \spc{Y}$ be a short map.
Assume $\spc{Y}$ is compact and for any finite set $F\subset \spc{X}$ there is a short map $F\to \spc{Y}$ that agrees with $f$ on $F\cap A$.
Show that there is a short map $\spc{X}\to \spc{Y}$ that agrees with $f$ on $A$.
\end{thm}

\chapter{Space of sets}

\section{Hausdorff distance}

Let $\spc{X}$ be a metric space.
Given a subset $A\subset \spc{X}$,
consider the distance function to $A$
$$\distfun_A: \spc{X} \to [0,\infty)$$
defined as 
$$\distfun_A(x)
\df
\inf_{a\in A}\{\,\dist ax{\spc{X}}\,\}.$$

\begin{thm}{Definition}\label{def:hausdorff-convergence}
Let $A$ and $B$ be two compact subsets of a metric space $\spc{X}$.
Then the \index{Hausdorff distance}\emph{Hausdorff distance} between $A$ and $B$ is defined as 
$$|A-B|_{\Haus\spc{X}}
\df
\sup_{x\in \spc{X}}\{\,|\distfun_A(x)-\distfun_B(x)|\,\}.
$$

\end{thm}

The following observation gives a useful reformulation of the definition:

\begin{thm}{Observation}\label{obs:Haus-nbhds}
Suppose $A$ and $B$ be two compact subsets of a metric space $\spc{X}$.
Then $|A-B|_{\Haus\spc{X}}< R$ if and only if and only if 
$B$ lies in an $R$-neighborhood of $A$, 
and 
$A$ lies in an $R$-neighborhood of~$B$.
\end{thm}



Note that the set of all nonempty compact subsets of a metric space $\spc{X}$ equipped with the Hausdorff metric forms a metric space.
This new metric space will be denoted as $\Haus\spc{X}$.


\begin{thm}{Exercise}\label{ex:diam}
Let $\spc{X}$ be a metric space.
Given a subset $A\subset \spc{X}$ define its \index{diameter}\emph{diameter} as 
$$\diam A\df\sup_{a,b\in A} |a-b|.$$

Show that 
$$\diam\:\Haus\spc{X}\to \RR$$ 
is a \index{Lipschitz function}\emph{$2$-Lipschitz function};
that is,
\[|\diam A-\diam B|\le 2\cdot\dist{A}{B}{\Haus\spc{X}}\]
for any two compact nonempty sets $A,B\subset\spc{X}$.
\end{thm}


\begin{thm}{Exercise}\label{ex:Hausdorff-bry}
Let $A$ and $B$ be two compact subsets in the Euclidean plane $\RR^2$.
Assume $|A-B|_{\Haus\RR^2}<\eps$.

\begin{subthm}{ex:Hausdorff-bry:conv}
Show that $|\Conv A-\Conv B|_{\Haus\RR^2}<\eps$, where $\Conv A$ denoted the convex hull of $A$.
\end{subthm}
\begin{subthm}{ex:Hausdorff-bry:bry}
Is it true that
$|\partial A-\partial B|_{\Haus\RR^2}<\eps$,
where $\partial A$ denotes the boundary of $A$.

Does the converse hold? That is, assume $A$ and $B$ be two compact subsets in $\RR^2$
and $|\partial A-\partial B|_{\Haus\RR^2}<\eps$; 
is it true that $|A-B|_{\Haus\RR^2}\z<\eps$?
\end{subthm}

\end{thm}

Note that part \ref{SHORT.ex:Hausdorff-bry:conv} implies that $A\mapsto \Conv A$ defines a short map $\Haus\RR^2\to \Haus\RR^2$. 

\begin{thm}{Exercise}\label{ex:Haus-func}
Let $A$ and $B$ be two compact subsets in metric space $\spc{X}$.
Show that 
\[\dist{A}{B}{\Haus\spc{X}}=\sup_f\, \{\,\max_{a\in A}\{f(a)\}-\max_{b\in B}\{f(b)\,\},\]
where the least upper bound is taken for all $1$-Lipschitz functions $f$.

\end{thm}


\section{Hausdorff convergence}

\begin{thm}{Blaschke selection theorem}\label{thm:compact+Hausdorff}
A metric space $\spc{X}$ is compact if and only if
so is $\Haus\spc{X}$.
\end{thm}

The Hausdorff metric can be used to define convergence.
Namely, suppose $K_1,K_2,\dots$, and $K_\infty$ are compact sets in a metric space $\spc{X}$.
If $|K_\infty-K_n|_{\Haus\spc{X}}\to0$ as $n\to\infty$, then we say that 
the sequence $K_n$ {}\emph{converges} to $K_\infty$ \index{convergence in the sense of Hausdorff}\emph{in the sense of Hausdorff};
or we can say that $K_\infty$ is \emph{Hausdorff limit} of the sequence $K_n$.

Note that the theorem implies that from any sequence of compact sets in $\spc{X}$ one can select a subsequence that converges in the sense of Hausdorff; 
for that reason, it is called a \emph{selection} theorem. 

\parit{Proof; ``only if'' part.}
Consider the map $\iota$ that sends point $x\in \spc{X}$ to the one-point subset $\{x\}$ of $\spc{X}$.
Note that $\iota\:\spc{X}\to \Haus\spc{X}$ is distance-preserving.

Suppose that $A\subset \spc{X}$.
Note that $\diam A=0$ if and only if $A$ is a one-point set.
Therefore, from Exercise~\ref{ex:diam}, it follows 
that $\iota(\spc{X})$ is a closed subset of the compact space $\Haus\spc{X}$.
Whence $\iota(\spc{X})$, and therefore $\spc{X}$, are compact.
\qeds

To prove the ``if'' part we will need the following two lemmas.

\begin{thm}{Monotone convergence}\label{lem:decreasing-converges}
Let $K_1\supset K_2\supset\dots$ be a nested sequence of nonempty compact sets in a metric space $\spc{X}$.
Then $K_\infty\z=\bigcap_n K_n$ is the Hausdorff limit of $K_n$;
that is, $|K_\infty-K_n|_{\Haus\spc{X}}\to0$ as $n\to\infty$.
\end{thm}

\parit{Proof.}
By finite intersection property, $K_\infty$ is a nonempty compact set.

If the assertion were false, then there is $\eps>0$ such that for each $n$ 
one can choose $x_n\in K_n$
such that $\distfun_{K_\infty}(x_n)\ge\eps$.
Note that $x_n\in K_1$ for each $n$.
Since $K_1$ is compact, 
there is 
a \index{partial limit}\emph{partial limit}%
\footnote{Partial limit is a limit of a subsequence.}
 $x_\infty$ of $x_n$.
Clearly, $\distfun_{K_\infty}(x_\infty)\ge \eps$.

On the other hand, since $K_n$ is closed and $x_m\in K_n$ for $m\ge n$,
we get $x_\infty\in K_n$ for each $n$.
It follows that $x_\infty\in K_\infty$ and therefore $\distfun_{K_\infty}(x_\infty)=0$ ---
a contradiction.\qeds


\begin{thm}{Lemma}\label{lem:complete+Hausdorff}
If $\spc{X}$ is a compact metric space, then $\Haus\spc{X}$
is complete.
\end{thm}

\parit{Proof.}
Let $(Q_n)$ be a Cauchy sequence in $\Haus\spc{X}$.
Passing to a subsequence of $Q_n$ we may assume that 
$$|Q_n-Q_{n+1}|_{\Haus\spc{X}}\le \tfrac1{10^n}\eqlbl{eq:eps=1/10}$$
for each $n$.

Denote by $K_n$ the closed $\tfrac1{10^n}$-neighborhood of $Q_n$;
that is,
\begin{align*}
K_n&= \set{x\in \spc{X}}{\distfun_{Q_n}(x)\le \tfrac1{10^n}}
\end{align*}
Since $\spc{X}$ is compact so is each $K_n$.

By \ref{obs:Haus-nbhds}, $|Q_n-K_n|_{\Haus\spc{X}}\le \tfrac1{10^n}$.
From \ref{eq:eps=1/10}, we get
$K_n\supset K_{n+1}$ 
for each $n$.
Set 
$$K_\infty=\bigcap_{n=1}^\infty K_n.$$
By the monotone convergence (\ref{lem:decreasing-converges}),
 $|K_n-K_\infty|_{\Haus\spc{X}}\to 0$ as $n\to\infty$.
Since $|Q_n-K_n|_{\Haus\spc{X}}\le \tfrac1{10^n}$, we get $|Q_n-K_\infty|_{\Haus\spc{X}}\to 0$ as $n\to\infty$ --- hence the lemma.
\qeds

\begin{thm}{Exercise}\label{ex:closure-union}
Let $\spc{X}$ be a complete metric space and $K_1,K_2,\dots$ be a sequence of compact sets 
that converges in the sense of Hausdorff.
Show that the union $K_1\cup K_2\cup\dots$ is a compact closure.

Use this statement to show that in Lemma~\ref{lem:complete+Hausdorff} compactness of $\spc{X}$ can be exchanged to completeness.
\end{thm}

\parit{Proof of ``if'' part in \ref{thm:compact+Hausdorff}.}
According to Lemma~\ref{lem:complete+Hausdorff},
$\Haus\spc{X}$ is complete.
It remains to show that $\Haus\spc{X}$ is totally bounded (\ref{totally-bounded});
that is, given $\eps>0$ there is a finite $\eps$-net in $\Haus\spc{X}$.

Choose a finite $\eps$-net $A$ in $\spc{X}$.
Denote by $B$ the set of all subsets of $A$.
Note that  $B$ is a finite set in $\Haus\spc{X}$.
For each compact set $K\subset \spc{X}$, consider the subset $K'$ of all points $a\in A$
such that $\distfun_K(a)\le \eps$.
Observe that $K' \in B$ and $|K-K'|_{\Haus\spc{X}}\le\eps$.
In other words, $B$ is a finite $\eps$-net in $\Haus\spc{X}$.
\qeds

\begin{thm}{Exercise}\label{ex:Haus-length}
Let $\spc{X}$ be a complete metric space.
Show that $\spc{X}$ is a length space if and only if so is $\Haus\spc{X}$.
\end{thm}

\section{An application}

The following statement is called \index{isoperimetric inequality}\emph{isoperimetric inequality in the plane}.

\begin{thm}{Theorem}\label{thm:isoperimetric}
Among the plane figures bounded by closed curves of length at most $\ell$ the round disk has the maximal area.
\end{thm}

In this section, we will sketch a proof of the isoperimetric inequality that uses the Hausdorff convergence.
It is based on the following exercise.

\begin{thm}{Exercise}\label{ex:Huas-perimeter-area}
Let $\spc{C}$ be a subspace of $\Haus\RR^2$ formed by all compact convex subsets in $\RR^2$.
Show that perimeter\footnote{If the set degenerates to a line segment of length $\ell$, then its perimeter is defined as $2\cdot \ell$.} and area are continuous on~$\spc{C}$.
That is, if a sequence of convex compact plane sets $X_n$ converges to $X_\infty$ in the sense of Hausdorff, then 
\[\perim X_n\to \perim X_\infty\quad\text{and}\quad\area X_n\to\area X_\infty\]
as $n\to\infty$.
\end{thm}

\parit{Semiproof of \ref{thm:isoperimetric}.}
It is sufficient to consider only convex figures of the given perimeter; if a figure is not convex, pass to its convex hull and observe that it has a larger area and smaller perimeter.


Note that the selection theorem (\ref{thm:compact+Hausdorff}) together with the exercise imply the existence of figure $D$ with perimeter $\ell$ and maximal area.

It remains to show that $D$ is a round disk.
This is a problem in elementary geometry.

Let us cut $D$ along a chord $[ab]$ into two lenses, $L_1$ and $L_2$.
Denote by $L_1'$ the reflection of $L_1$ across the perpendicular bisector of $[ab]$.
Note that $D$ and $D'=L_1'\cup L_2$ have the same perimeter and area.
That is, $D'$ has perimeter $\ell$ and maximal possible area;
in particular, $D'$ is convex.

The following exercise will finish the proof.
\qeds

{

\begin{wrapfigure}{o}{57 mm}
\vskip-5mm
\centering
\includegraphics{mppics/pic-405}
\end{wrapfigure}

\begin{thm}{Exercise}\label{ex:round-disc}
Suppose $D$ is a convex figure such that for any chord $[ab]$ of $D$ the above construction produces a convex figure $D'$.
Show that $D$ is a round disk.
\end{thm}


}

Another popular way to prove that $D$ is a round disk is given by the so-called {}\emph{Steiner's 4-joint method} \cite{blaschke}.

\section{Remarks}\label{sec:H-variation}

It seems that Hausdorff convergence was first introduced by Felix Hausdorff~\cite{hausdorff}.
A couple of years later an equivalent definition was given by Wilhelm Blaschke~\cite{blaschke}.

The following refinement of the definition was introduced by  Zdeněk Frolík \cite{frolik},
later it was rediscovered by Robert Wijsman~\cite{wijsman}.  
This refinement is also called \index{Hausdorff convergence}\emph{Hausdorff convergence};
in fact, it takes an intermediate place between the original Hausdorff convergence and {}\emph{closed convergence}, also introduced by Hausdorff in \cite{hausdorff}.

\begin{thm}{Definition}\label{def:gen-Haus-conv}
Let $A_1,A_2,\dots$ be a sequence of closed sets in a metric space $\spc{X}$.
We say that the sequence $A_n$ converges to a closed set $A_\infty$ in the sense of Hausdorff if for any $x\in\spc{X}$, we have
$\distfun_{A_n}(x)\z\to \distfun_{A_\infty}(x)$ as $n\to\infty$.
\end{thm}

For example, suppose $\spc{X}$ is the Euclidean plane and $A_n$ is the circle with radius $n$ and center at the point $(n,0)$.
If we use the standard definition (\ref{def:hausdorff-convergence}), then the sequence $(A_n)$ diverges, but it converges to the $y$-axis in the sense of Definition~\ref{def:gen-Haus-conv}.

The following exercise is analogous to the Blaschke selection theorem (\ref{thm:compact+Hausdorff}) for the modified Hausdorff convergence.

\begin{thm}{Exercise}\label{ex:generalized-selection}
Let $\spc{X}$ be a proper metric space
and $A_1,A_2,\dots$ be a sequence of closed sets in~$\spc{X}$.
Assume that for some (and therefore any) point  $x\in\spc{X}$, 
the sequence $a_n=\distfun_{A_n}(x)$ is bounded.
Show that the sequence  $A_1,A_2,\dots$ has a convergent subsequence in the sense of Definition~\ref{def:gen-Haus-conv}.
\end{thm}

\chapter{Space of spaces}

\section{Gromov--Hausdorff metric}

The goal of this section is to cook up a metric space out of metric spaces.
More precisely, we want to define the so-called  Gromov--Hausdorff metric on the set of {}\emph{isometry classes} of compact metric spaces.
(Being isometric is an equivalence relation, 
and an isometry class is an equivalence class with respect to this equivalence relation.)

The obtained metric space will be denoted by $\GH$.
Given two metric spaces $\spc{X}$ and $\spc{Y}$,
denote by $[\spc{X}]$ and $[\spc{Y}]$ their isometry classes;
that is, $\spc{X}'\in [\spc{X}]$ if and only if $\spc{X}'\iso \spc{X}$.
Pedantically, the Gromov--Hausdorff distance from $[\spc{X}]$ 
to $[\spc{Y}]$ should be denoted as $|[\spc{X}]-[\spc{Y}]|_{\GH}$;
but we will write it as $|\spc{X}\z-\spc{Y}|_{\GH}$ and say (not quite correctly) 
``$|\spc{X}\z-\spc{Y}|_{\GH}$ is the Gromov--Hausdorff distance from  $\spc{X}$ 
to  $\spc{Y}$''.
In other words, from now on the term {}\emph{metric space} might also stand for its {}\emph{isometry class}.

The metric on $\GH$ is defined as the maximal metric such that \textit{the distance between subspaces in a metric space is not greater than the Hausdorff distance between them}.
Here is a formal definition:

\begin{thm}{Definition}\label{def:GH}
Let $\spc{X}$ and $\spc{Y}$ be compact metric spaces. 
The Gromov--Hausdorff distance $|\spc{X}-\spc{Y}|_{\GH}$ is defined by the following
relation.
 
Given  $r > 0$, we have that $|\spc{X}-\spc{Y}|_{\GH} < r$ if and only if there exist a metric
space $\spc{Z}$ and subspaces $\spc{X}'$ and $\spc{Y}'$ in $\spc{Z}$ that are isometric to $\spc{X}$ and $\spc{Y}$
respectively and such that $|\spc{X}'-\spc{Y}'|_{\Haus\spc{Z}} < r$. 
(Here $|\spc{X}'-\spc{Y}'|_{\Haus\spc{Z}}$ denotes the Hausdorff distance between sets $\spc{X}'$ and $\spc{Y}'$ in $\spc{Z}$.)
\end{thm}

Note that passing to the subspace $\spc{X}'\cup\spc{Y}'$ of $\spc{Z}$ does not affect the definition.
Therefore we can always assume that $\spc{Z}$ is compact.

\begin{thm}{Theorem}\label{thm:GH-is-a-metric}
The set of isometry classes of compact metric spaces equipped with Gromov--Hausdorff metric forms a metric space (which is denoted by $\GH$).

In other words, for arbitrary  compact metric spaces $\spc{X}$, $\spc{Y}$ and $\spc{Z}$ the following conditions hold:

\begin{subthm}{GH-1} $|\spc{X}-\spc{Y}|_{\GH}\ge 0$;
\end{subthm}

\begin{subthm}{GH-2} $|\spc{X}-\spc{Y}|_{\GH}=0$ if and only if $\spc{X}$ is isometric to $\spc{Y}$;
\end{subthm}

\begin{subthm}{GH-3} $|\spc{X}-\spc{Y}|_{\GH}=|\spc{Y}-\spc{X}|_{\GH}$;
\end{subthm}

\begin{subthm}{GH-4} $|\spc{X}-\spc{Y}|_{\GH}+|\spc{Y}-\spc{Z}|_{\GH}\ge |\spc{X}-\spc{Z}|_{\GH}$.
\end{subthm}
\end{thm}


Note that \ref{SHORT.GH-1}, \ref{SHORT.GH-3},
and the ``if''-part of \ref{SHORT.GH-2} follow directly from Definition \ref{def:GH}.
Part \ref{SHORT.GH-4} will be proved in Section~\ref{sec:GH-approx}.
The ``only-if''-part of \ref{SHORT.GH-2} will be proved in Section~\ref{sec:alm-isom}.

Recall that $a\cdot\spc{X}$ denotes $\spc{X}$ \index{scaled space}\emph{scaled} by factor $a>0$;
that is, $a\cdot\spc{X}$ is a metric space with the underlying set of $\spc{X}$ and the metric defined by
\[\dist{x}{y}{a\cdot\spc{X}}\df a\cdot\dist{x}{y}{\spc{X}}.\]

\begin{thm}{Exercise}\label{ex:d_GH-and-diam}
Let $\spc{X}$ be a compact metric space,
$\spc{P}$ be the one-point metric space.

Prove that 
\begin{subthm}{ex:d_GH-and-diam:point}
\[|\spc{X}-\spc{P}|_{\GH}=\tfrac12\cdot \diam \spc{X}.\]

\end{subthm}

\begin{subthm}{ex:d_GH-and-diam:scale}
\[|a\cdot\spc{X}-b\cdot \spc{X}|_{\GH}=\tfrac12\cdot|a-b|\cdot\diam\spc{X}.\]
\end{subthm}


\end{thm}

\begin{thm}{Exercise}\label{ex:rectangle}
Let $\spc{A}_r$ be a rectangle $1$ by $r$ in the Euclidean plane 
and $\spc{B}_r$ be a closed line interval of length $r$.
Show that 
\[|\spc{A}_r-\spc{B}_r|_{\GH}>\tfrac1{10}\]
for all large $r$.
\end{thm}

\begin{thm}{Advanced exercise}\label{ex:GH-inj}
Let $\spc{X}$ and $\spc{Y}$ be compact metric spaces;
denote by $\hat{\spc{X}}$ and $\hat{\spc{Y}}$ their injective envelopes (see \ref{sec:extremal-functions}).
Show that 
\[|\hat{\spc{X}}-\hat{\spc{Y}}|_{\GH}\le 2\cdot|\spc{X}- \spc{Y}|_{\GH}.\] 

\end{thm}

\section{Approximations}\label{sec:GH-approx}

\begin{thm}{Definition}\label{ex:defGHR}
Let $\spc{X}$ and $\spc{Y}$ be two metric spaces.
A relation $\approx$ between points in $\spc{X}$ and $\spc{Y}$ is called $\eps$-approximation if the following conditions hold:
\begin{itemize}
\item For any $x\in  \spc{X}$ there is $y\in \spc{Y}$ such that $x\approx y$.
\item For any $y\in  \spc{Y}$ there is $x\in \spc{X}$ such that $x\approx y$.
\item If for some $x, x'\in  \spc{X}$ and $y,y'\in \spc{Y}$ we have $x\approx y$ and $x'\approx y'$, then 
\[\bigl|\dist{x}{x'}{\spc{X}}-\dist{y}{y'}{\spc{Y}}\bigr|<2\cdot\eps.\]
\end{itemize}

\end{thm}

\begin{thm}{Exercise}\label{ex:H-R}
Let $\spc{X}$ and $\spc{Y}$ be two compact metric spaces.
Show that
\[\dist{\spc{X}}{\spc{Y}}{\GH}<\eps\]
if and only if there is an $\eps$-approximation between $\spc{X}$ and $\spc{Y}$.

In other words $\dist{\spc{X}}{\spc{Y}}{\GH}$ is the greatest lower bound of values $\eps>0$ such that  there is an $\eps$-approximation between $\spc{X}$ and $\spc{Y}$.
\end{thm}

\parit{Proof of \ref{GH-4}.}
Suppose that 
\begin{itemize}
\item $\approx_1$ is a relation between points in $\spc{X}$ and $\spc{Y}$,
\item $\approx_2$ is a relation between points in $\spc{Y}$ and $\spc{Z}$.
\end{itemize}
Consider the relation $\approx_3$ between points in $\spc{X}$ and $\spc{Z}$ such that
$x\approx_3 z$ if and only if there is $y\in  \spc{Y}$ such that 
$x\approx_1 y$ and $y\approx_2 z$.

It is straightforward to check that if $\approx_1$ is an $\eps_1$-approximation and $\approx_2$ is an $\eps_2$-approximation, then $\approx_3$ is an $(\eps_1+\eps_2)$-approximation.

Applying \ref{ex:H-R}, we get that if 
\[|\spc{X}-\spc{Y}|_{\GH}<\eps_1
\quad\text{and}\quad
|\spc{Y}-\spc{Z}|_{\GH}<\eps_2,
\]
then 
\[|\spc{X}-\spc{Z}|_{\GH}<\eps_1+\eps_2.\]
Hence \ref{GH-4} follows.
\qeds


\section{Almost isometries}\label{sec:alm-isom}

\begin{thm}{Definition} Let $\spc{X}$ and $\spc{Y}$ be metric spaces and $\eps>0$. 
A  map\footnote{possibly noncontinuous} $f\: \spc{X} \z\to \spc{Y}$ is called an \index{almost isometry}\emph{$\eps$-isometry} 
if $f(\spc{X})$ is an $\eps$-net in $\spc{Y}$ and
\[\bigl|\dist{x}{x'}{\spc{X}}-\dist{f(x)}{f(x')}{\spc{Y}}\bigr|<\eps.\]
for any $x,x'\in \spc{X}$.
\end{thm}

\begin{thm}{Exercise}\label{ex:eps-isom}
Let $\spc{X}$ and $\spc{Y}$ be compact metric spaces.

\begin{subthm}{ex:eps-isom:GH>isom}
If $\dist{\spc{X}}{\spc{Y}}{\GH}<\eps$, then there is a $2\cdot\eps$-isometry $f\:\spc{X}\to\spc{Y}$.
\end{subthm}

\begin{subthm}{ex:eps-isom:isom>GH}
If there is an $\eps$-isometry $f\:\spc{X}\to\spc{Y}$, then $\dist{\spc{X}}{\spc{Y}}{\GH}<\eps$.
\end{subthm}

\end{thm}

\parit{Proof of the ``only if''-part in \ref{GH-2}.}
\label{page:GH-2-proof}
Let $\spc{X}$ and $\spc{Y}$ be compact metric spaces.
Suppose that $\dist{\spc{X}}{\spc{Y}}{\GH}<\eps$ for any $\eps>0$;
we need to show that there is an isometry $\spc{X}\to\spc{Y}$.

By \ref{ex:eps-isom:GH>isom}, for each positive integer $n$, we can choose a $\tfrac1n$-isometry $f_n\:\spc{X}\to\spc{Y}$.

Since $\spc{X}$ is compact, 
we can choose a countable dense set
$S$ in~$\spc{X}$.
Applying the diagonal procedure if necessary, we can assume that for every $x \in S$ the sequence $f_n(x)$ 
converges in $\spc{Y}$. 
Consider the pointwise limit map  $f_\infty \: S \to \spc{Y}$,
 $$f_\infty(x) \df \lim_{n\to\infty} f_n (x)$$ for every $x \in S$. 
Since $$|f_n (x)- f_n (x')|_{\spc{Y}}\lg |x- x'|_\spc{X} \pm\tfrac1n,$$ 
we have 
$$|f_\infty(x)-f_\infty (x')|_{\spc{Y}} 
= \lim_{n\to\infty} |f_n(x)-f_n (x')|_{\spc{Y}} 
= |x -x'|_\spc{X}$$ for all
$x, x' \in S$; 
that is, the map $f_\infty\:S\to \spc{Y}$ is distance-preserving. 
Therefore, $f_\infty$ can be extended to a distance-preserving map from the whole $\spc{X}$ to $\spc{Y}$.

The latter can be done by setting 
$$f_\infty(x)=\lim_{n\to\infty} f_\infty(x_n)$$ 
for some sequence $x_n$ of points  in $S$
that converges to $x$ in $\spc{X}$.
Indeed, if $x_n\to x$, then the sequence $x_n$ is Cauchy.
Since $f_\infty$ is distance-preserving, $y_n=f_\infty(x_n)$ is also a Cauchy sequence in $\spc{Y}$;
therefore it converges.
It remains to observe that this construction does not depend on the choice of the sequence $x_n$.

This way we obtain a distance-preserving map $f_\infty\:\spc{X}\to \spc{Y}$. 
It remains to show that $f_\infty$ is surjective; that is, $f_\infty(\spc{X})=\spc{Y}$.

The same argument produces a distance-preserving map $g_\infty\:\spc{Y}\z\to \spc{X}$.
If $f_\infty$ is not surjective, then neither is the composition $f_\infty\z\circ g_\infty\:\spc{Y}\to \spc{Y}$.
So $f_\infty \z\circ g_\infty$ is a distance-preserving map from a compact space to itself which is not an isometry.
The latter contradicts \ref{ex:non-contracting-map}. 
\qeds

\section{Convergence}

The Gromov--Hausdorff metric is used to define Gromov--Hausdorff convergence.
Namely, a sequence of compact metric spaces $\spc{X}_n$ converges to compact metric spaces $\spc{X}_\infty$ in the sense of Gromov--Hausdorff if 
\[\dist{\spc{X}_n}{\spc{X}_\infty}{\GH}\to 0\quad\text{as}\quad n\to\infty.\]

This convergence is more important than the metric ---
in all applications, we use only the topology on $\GH$
and we do not care about the particular value of Gromov--Hausdorff distance between spaces.
The following observation follows from \ref{ex:eps-isom}:

\begin{thm}{Observation}\label{obs:GH-e-isom}
A sequence of compact metric spaces $(\spc{X}_n)$ converges to  $\spc{X}_\infty$ in the sense of Gromov--Hausdorff if and only if there is a sequence $\eps_n\to0+$
and an $\eps_n$-isometry $f_n\:\spc{X}_n\to \spc{X}_\infty$ for each $n$.
\end{thm}

In the following exercises \textit{converge} means in the sense of Gromov--Hausdorff.

\begin{thm}{Exercise}\label{ex:GH-SC}
\begin{subthm}{ex:GH-SC:circle}
Show that a sequence of compact simply connected length spaces cannot converge to a circle.
\end{subthm}

\begin{subthm}{ex:GH-SC:nonsc-limit}
Construct a sequence of compact simply connected length spaces that converges to a compact non-simply connected space.
\end{subthm}
\end{thm}

\begin{thm}{Exercise}\label{ex:sphere-to-ball}
\begin{subthm}{ex:sphere-to-ball:2}
Show that a sequence of length metrics on the 2-sphere cannot converge to the unit disk.
\end{subthm}

\begin{subthm}{ex:sphere-to-ball:3}
Construct a sequence of length metrics on the 3-sphere that converges to a unit 3-ball.
\end{subthm}

\end{thm}

Given two metric spaces $\spc{X}$ and $\spc{Y}$, we will write $\spc{X}\le \spc{Y}$ if there is a noncontracting map $f\:\spc{X}\to \spc{Y}$;
that is, if 
$$ |x-x'|_{\spc{X}}\le|f(x)-f(x')|_{\spc{Y}}$$
for any $x,x'\in \spc{X}$.

Further, given $\eps>0$, we will write $\spc{X}\le \spc{Y}+\eps$
if there is a map $f\:\spc{X}\to \spc{Y}$ such that 
$$|x-x'|_{\spc{X}}\le|f(x)-f(x')|_{\spc{Y}}+\eps$$
for any $x,x'\in \spc{X}$.

\section{Uniformly totally bonded families}

\begin{thm}{Definition}\label{def:utb}
A family $\spc{Q}$ of (isometry classes) of compact metric spaces is called  \index{uniformly totally bonded family}\emph{uniformly totally bonded} if it meets the following two conditions:

\begin{subthm}{}
spaces in $\spc{Q}$ have uniformly bounded diameters; that is, there is $D\in\RR$ such that
\[\diam\spc{X}\le D\]
for any space $\spc{X}$ in $\spc{Q}$.
\end{subthm}

\begin{subthm}{}
For any $\eps>0$ there is $n\in\NN$ such that any space $\spc{X}$ in $\spc{Q}$ admits an $\eps$-net with at most $n$ points.
\end{subthm}
\end{thm}

\begin{thm}{Exercise}\label{ex:utb+pack}
Let $\spc{Q}$ be a family of compact spaces with uniformly bounded diameters.
Show that $\spc{Q}$ is uniformly totally bonded if for any $\eps>0$ there is $n\in\NN$ such that 
\[\pack_\eps\spc{X}\le n\]
for any space $\spc{X}$ in $\spc{Q}$.
\end{thm}


Fix a real constant $C$.
A Borel measure $\mu$ on a metric space $\spc{X}$ is called \index{doubling space}\emph{$C$-doubling} if
\[\mu[\oBall(p,2\cdot r)]< C\cdot\mu[\oBall(p,r)]\]
for any point $p\in \spc{X}$ and any $r>0$.
A Borel measure is called \index{doubling measure}\emph{doubling} if it is {}\emph{$C$-doubling} for some real constant $C$.

\begin{thm}{Exercise}\label{pr:doubling}
Let $\spc{Q}(C,D)$ be the set of all the compact metric spaces with diameter at most $D$ that admit a $C$-doubling measure.
Show that $\spc{Q}(C,D)$ is totally bounded.
\end{thm}

Recall that we write $\spc{X}\le\spc{Y}$ if there is a distance-nondecreasing map $\spc{X}\to\spc{Y}$.

\begin{thm}{Exercise}\label{pr:under}

\begin{subthm}{pr:under:if}
Let $\spc{Y}$ be a compact metric space.
Show that the set of all spaces $\spc{X}$ such that $\spc{X}\le\spc{Y}$
is uniformly totally bounded.
\end{subthm}

\begin{subthm}{pr:under:only-if}
Show that for any uniformly totally bounded set $\spc{Q}\subset\GH$ there is a compact space $\spc{Y}$
such that $\spc{X}\le\spc{Y}$ for any $\spc{X}$ in $\spc{Q}$.
\end{subthm}

\end{thm}

\section{Gromov's selection theorem}

The following theorem is analogous to Blaschke selection theorems (\ref{thm:compact+Hausdorff}).

\begin{thm}{Gromov selection theorem}\label{thm:gromov-compactness}
Let $\spc{Q}$ be a closed subset of $\GH$.
Then $\spc{Q}$ is compact if and only if the elements of $Q$ are uniformly totally bounded.
\end{thm}

\begin{thm}{Lemma}\label{lem:GH-complete}
The space $\GH$ is complete.
\end{thm}


Let us define gluing of metric spaces that will be used in the proof of the lemma.

Suppose 
$\spc{U}$ and $\spc{V}$ are metric spaces 
with isometric closed sets $A\subset\spc{U}$ and $A'\subset\spc{V}$;
let $\iota\:A\to A'$ be an isometry.
Consider the space $\spc{W}$ of all equivalence classes in $\spc{U}\sqcup\spc{V}$ with the equivalence relation given by $a\sim\iota(a)$ for any $a\in A$.

It is straightforward to check that the following defines a metric on~$\spc{W}$:
\begin{align*}
\dist{u}{u'}{\spc{W}}&\df\dist{u}{u'}{\spc{U}}
\\
\dist{v}{v'}{\spc{W}}&\df\dist{v}{v'}{\spc{V}}
\\
\dist{u}{v}{\spc{W}}&\df\min\set{\dist{u}{a}{\spc{U}}+\dist{v}{\iota(a)}{\spc{V}}}{a\in A}
\end{align*}
where $u,u'\in \spc{U}$ and $v,v'\in \spc{V}$.

The  space $\spc{W}$ is called the \index{gluing}\emph{gluing} of $\spc{U}$ and  $\spc{V}$ along~$\iota$; briefly, we can write
$\spc{W}=\spc{U}\sqcup_\iota\spc{V}$.
If one applies this construction to two copies of one space $\spc{U}$ with a set $A\subset \spc{U}$ and the identity map $\iota\:A\to A$, then the obtained space is called the \index{double}\emph{double} of $\spc{U}$ along~$A$; this space can be denoted by $\sqcup_A^2\spc{U}$.

Note that the inclusions $\spc{U}\hookrightarrow \spc{W}$ and $\spc{V}\hookrightarrow \spc{W}$ are distance preserving.
Therefore we can and will conside $\spc{U}$ and $\spc{V}$ as the subspaces of $\spc{W}$;
this way the subsets $A$ and $A'$ will be identified and denoted further by~$A$.
Note that $A=\spc{U}\cap \spc{V}\subset \spc{W}$.

\parit{Proof.}
Let $\spc{X}_1,\spc{X}_2,\dots$ be a Cauchy sequence in $\GH$.
Passing to a subsequence if necessary, 
we can assume that $|\spc{X}_n-\spc{X}_{n+1}|_{\GH}<\tfrac1{2^n}$ for each~$n$.
In particular, for each $n$ there is a metric space $\spc{V}_n$ with distance preserving inclusions $\spc{X}_n\hookrightarrow \spc{V}_n$ and $\spc{X}_{n+1}\hookrightarrow \spc{V}_n$ such that
\[|\spc{X}_n-\spc{X}_{n+1}|_{\Haus\spc{V}_n}<\tfrac1{2^n}\]
for each $n$.
Moreover, we may assume that $\spc{V}_n=\spc{X}_n\cup\spc{X}_{n+1}$.

Let us glue $\spc{V}_1$ to $\spc{V}_2$ along $\spc{X}_2$;
to the obtained space glue $\spc{V}_3$ along $\spc{X}_3$, and so on.
The obtained metric space $\spc{W}$
has an underlying set formed by the disjoint union of all $\spc{X}_n$ such that each inclusion $\spc{X}_n\z\hookrightarrow\spc{W}$ is distance preserving and
\[|\spc{X}_n-\spc{X}_{n+1}|_{\Haus\spc{W}}<\tfrac1{2^n}\]
for each $n$.
In particular,
\[|\spc{X}_m-\spc{X}_n|_{\Haus\spc{W}}<\tfrac1{2^{n-1}}\eqlbl{eq:|x_m-X_n|}\] 
if $m>n$.

Denote by $\bar{\spc{W}}$ the completion of $\spc{W}$.
Observe that the union $\spc{X}_1\z\cup \spc{X}_2\cup\z\dots\cup \spc{X}_n$ is compact and \ref{eq:|x_m-X_n|} implies that it forms a $\tfrac1{2^{n-1}}$-net in $\bar{\spc{W}}$.
Whence $\bar{\spc{W}}$ is compact; see \ref{totally-bounded} and \ref{ex:compact-net}.

Applying Blaschke selection theorem (\ref{thm:compact+Hausdorff}),
we can pass to a subsequence of $\spc{X}_n$ that converges in $\Haus\bar{\spc{W}}$; denote its limit by $\spc{X}_\infty$.
It remains to observe that $\spc{X}_\infty$ is the Gromov--Hausdorff limit of $(\spc{X}_n)$.
\qeds

\parit{Proof of \ref{thm:gromov-compactness}; ``only if'' part.}
Suppose that there is no sequence $\eps_n\to0$ as described in \ref{def:utb}.
Observe that in this case
there is a sequence of spaces $\spc{X}_n\in\spc{Q}$ such that 
$$\pack_\delta \spc{X}_n\to\infty
\quad\text{as}\quad
n\to\infty$$
for some fixed $\delta>0$.

Since $\spc{Q}$ is compact, 
this sequence has a partial limit, say $\spc{X}_\infty\in\spc{Q}$.
Observe that $\pack_{\delta} \spc{X}_\infty=\infty$.
Therefore, $\spc{X}_\infty$ is not compact --- a contradiction.

\parit{``If'' part.}
Let $\eps_n$ be a sequence as in the definition of uniformly totally bonded families (\ref{def:utb}).

Note that $\diam \spc{X}\le \eps_1$ for any $\spc{X}\in \spc{Q}$.
Given a positive integer $n$ consider the set of all metric spaces $\spc{W}_n$
with the number of points at most $n$ and diameter $\le \eps_1$.
Note that $\spc{W}_n$ is a compact set in $\GH$ for each $n$.

Further, a subspace formed by a maximal $\eps_n$-net of any $\spc{X}\in\spc{Q}$ belongs to $\spc{W}_n$.
Therefore, $\spc{W}_n\cap\spc{Q}$ is a compact $\eps_n$-net in  $\spc{Q}$.
That is, $\spc{Q}$ has a compact $\eps$-net for any $\eps>0$.
Since $\spc{Q}$ is closed in a complete space $\GH$, it implies that $\spc{Q}$ is compact.
\qeds

\begin{thm}{Exercise}\label{ex-GH-length}
Show that the space $\GH$ is 

\begin{subthm}{ex-GH-length:length}
length,
\end{subthm}

\begin{subthm}{ex-GH-length:geodesic}
geodesic.
\end{subthm}

\end{thm}

\begin{thm}{Exercise}\label{ex:GH-po}
For two metric spaces $\spc{X}$ and $\spc{Y}$,
we write $\spc{X}\le \spc{Y}+\eps$ if
there is a map $f\:\spc{X}\to \spc{Y}$ such that 
\[\dist{x}{x'}{\spc{X}}\le \dist{f(x)}{f(x')}{\spc{Y}}+\eps\]
for any $x,x'\in \spc{X}$.

\begin{subthm}{ex:GH-po:a}
Show that 
$$\dist{\spc{X}}{\spc{Y}}{\GH'}=\inf\set{\eps>0}{\spc{X}\le \spc{Y}+\eps
\quad\text{and}\quad
\spc{Y}\le \spc{X}+\eps}$$
defines a metric on the space of (isometry classes) of compact metric spaces.
\end{subthm}

\begin{subthm}{ex:GH-po:b}
Moreover $\dist{*}{*}{\GH'}$ is equivalent to the Gromov--Hausdorff metric;
that is,
$$|\spc{X}_n-\spc{X}_\infty|_{\GH}\to 0 
\quad\iff\quad 
\dist{\spc{X}_n}{\spc{X}_\infty}{\GH'}\to 0$$ 
as $n\to\infty$.
\end{subthm}
\end{thm}

\section{Universal ambient space}

Recall that a metric space is called universal if it contains an isometric copy of any separable metric space (in particular, any compact metric space).
Examples of universal spaces include Urysohn space and $\ell^\infty$ --- the space of bounded infinite sequences with the metric defined by $\sup$-norm; see \ref{prop:sep-in-urys} and \ref{ex:frechet}.

The following proposition says that the space $\spc{W}$ in Definition~\ref{def:GH} can be exchanged to a fixed universal space.

\begin{thm}{Proposition}\label{prop:GH-with-fixed-Z}
Let $\spc{U}$ be a universal space.
Then for any compact metric spaces $\spc{X}$ and $\spc{Y}$ we have
$$|\spc{X}-\spc{Y}|_{\GH} = \inf \{|\spc{X}'-\spc{Y}'|_{\Haus\spc{U}}\}$$ 
where the greatest lower bound is taken over all pairs of sets $\spc{X}'$ and $\spc{Y}'$ in $\spc{U}$
which isometric to  $\spc{X}$ and $\spc{Y}$ respectively.  
\end{thm}




\parit{Proof of \ref{prop:GH-with-fixed-Z}.}
By the definition (\ref{def:GH}), we have that 
\[|\spc{X}-\spc{Y}|_{\GH} \le \inf \{|\spc{X}'-\spc{Y}'|_{\Haus\spc{U}}\};\]
it remains to prove the opposite inequality.

Suppse $|\spc{X}-\spc{Y}|_{\GH}<\eps$;
let $\spc{X}'$, $\spc{Y}'$ and $\spc{Z}$ be as in \ref{def:GH}.
We can assume that $\spc{Z}=\spc{X}'\cup\spc{Y}'$;
otherwise pass to the subspace $\spc{X}'\cup\spc{Y}'$ of~$\spc{Z}$.
In this case, $\spc{Z}$ is compact;
in particular, it is separable.

Since $\spc{U}$ is universal, there is a distance-preserving embedding of $\spc{Z}$ in $\spc{U}$;
let us keep the same notation for $\spc{X}'$, $\spc{Y}'$, and their images.
It follows that 
\[|\spc{X}'-\spc{Y}'|_{\Haus\spc{U}}<\eps,\]
--- hence the result.
\qeds

\begin{thm}{Exercise}\label{ex:GH-urysohn}
Let $\spc{U}_\infty$ be the Urysohn space.
Given two compact set $A$ and $B$ in $\spc{U}_\infty$ define 
\[\|A-B\|=\inf\{|A-\iota(B)|_{\Haus\spc{U}_\infty}\},\]
where the greatest lower bound is taken for all isometrics $\iota$ of $\spc{U}_\infty$.
Show that $\|{*}\z-{*}\|$ defines a pseudometric%
\footnote{The value $\|A-B\|$ is called Hausdorff distance \emph{up to isometry} from $A$ to $B$ in $\spc{U}_\infty$.}
on nonempty compact subsets of $\spc{U}_\infty$ and its corresponding metric space is isometric to $\GH$.
\end{thm}

\section{Remarks}

Suppose $\spc{X}_n\GHto \spc{X}_\infty$, then there is a metric on the disjoint union 
\[\bm{X}=\bigsqcup_{n\in \NN\cup\{\infty\}} \spc{X}_n\] 
that satisfies the following property:

\begin{thm}{Property}\label{propery:GH}
The restriction of metric on each $\spc{X}_n$ and $\spc{X}_\infty$ coincides with its original metric 
and $\spc{X}_n\Hto \spc{X}_\infty$ as subsets in $\bm{X}$.
\end{thm}


Indeed, since $\spc{X}_n\GHto \spc{X}_\infty$, there is a metric on $\spc{V}_n=\spc{X}_n\sqcup \spc{X}_\infty$ such that the restriction of metric on each $\spc{X}_n$ and $\spc{X}_\infty$ coincides with its original metric and $\dist{\spc{X}_n}{\spc{X}_\infty}{\Haus\spc{V}_n}<\eps_n$ for some sequence $\eps_n\to 0$.
Gluing all $\spc{V}_n$ along $\spc{X}_\infty$, we obtain the required space $\bm{X}$.

In other words, the metric on $\bm{X}$ defines the convergence $\spc{X}_n\z\GHto \spc{X}_\infty$.
This metric makes it possible to talk about limits of sequences $x_n\in \spc{X}_n$ as $n\to\infty$, as well as weak limits of a sequence of Borel measures $\mu_n$ on $\spc{X}_n$ and so on.

For that reason, it is useful to define convergence by specifying the metric on $\bm{X}$ that satisfies the property
for the variation of Hausdorff convergence described in Section~\ref{sec:H-variation}.
This approach is very flexible;
in particular, it can be used to define Gromov--Hausdorff convergence of arbitrary metric spaces (net necessarily compact).

In this case, a limit space for this generalized convergence is not uniquely defined.
\begin{figure}[h!]
\vskip-0mm
\centering
\includegraphics{mppics/pic-500}
\end{figure}
For example, if each space $\spc{X}_n$ in the sequence is isometric to the half-line, then its limit might be isometric to the half-line or the whole line.
The first convergence is evident and the second could be guessed from the diagram.



Often the isometry class of the limit can be fixed by marking a point $p_n$ in each space $\spc{X}_n$, it is called \index{pointed convergence}\emph{pointed Gromov--Hausdorff convergence} --- we say that $(\spc{X}_n,p_n)$ converges to $(\spc{X}_\infty,p_\infty)$ if there is a metric on $\bm{X}$ such that $\spc{X}_n\Hto \spc{X}_\infty$ and $p_n\to p_\infty$.
For example, the sequence $(\spc{X}_n,p_n)=(\RR_+,0)$ converges to $(\RR_+,0)$, while $(\spc{X}_n,p_n)=(\RR_+,n)$ converges to $(\RR,0)$.

The pointed convergence works nicely only for proper metric spaces;
the following theorem is an analog of Gromov's selection theorem for this convergence.

\begin{thm}{Theorem}\label{thm:pointed-gromov-compactness}%
Let $\spc{Q}$ be a set of isometry classes of pointed proper metric spaces
$(\spc{X},p)$.
Assume that for any $R>0$, the $R$-balls in the spaces centered at the marked points form a uniformly totally bounded family of spaces.
Then $\spc{Q}$ is precompact with respect to pointed Gromov--Hausdorff convergence. 
\end{thm}







\chapter{Ultralimits}

Ultralimits provide a very general way to pass to a limit.
This procedure works for \textit{any} sequence of metric spaces, its result reminds limit in the sense of Gromov--Hausdorff, but has some strange features; for example, the limit of a constant sequence of spaces $\spc{X}_n=\spc{X}$ is \textit{not} $\spc{X}$ (see \ref{ex:ultrapower:compact}).

In geometry, ultralimits are used only as a canonical way to pass to a convergent subsequence.
It is useful in the proofs where one needs to repeat ``pass to convergent subsequence'' too many times.

This lecture is based on the introductory part of the paper by Bruce Kleiner and Bernhard Leeb \cite{kleiner-leeb}.

\section{Faces of ultrafilters}

Recall that $\NN$ denotes the set of natural numbers, $\NN=\{1,2,\dots\}$

\begin{thm}{Definition}
A finitely additive measure $\omega$ 
on $\NN$ 
is called an \index{ultrafilter}\emph{ultrafilter} if it satisfies the following condition:
\begin{subthm}{}
$\omega(\NN)=1$ and 
$\omega(S)=0$ or $1$ for any subset $S\subset \NN$.
\end{subthm}
An ultrafilter $\omega$ is called 
\emph{nonprincipal}\index{ultrafilter!nonprincipal ultrafilter}\index{nonprincipal ultrafilter} if in addition 
\begin{subthm}{}
$\omega(F)=0$ for any finite subset $F\subset \NN$.
\end{subthm}
\end{thm}

If $\omega(S)=0$ for some subset $S\subset \NN$,
we say that $S$ is \index{$\omega$-small}\emph{$\omega$-small}. 
If $\omega(S)=1$, we say that $S$ contains \index{$\omega$-almost all}\emph{$\omega$-almost all} elements of $\NN$.

\begin{thm}{Advanced exercise}\label{ex:ultrakatetov}
Let $\omega$ be an ultrafilter on $\NN$ and $f\:\NN\z\to \NN$.
Suppose that $\omega(S)\le \omega(f^{-1}(S))$ for any set $S\subset \NN$.
Show that $f(n)=n$ for $\omega$-almost all $n\in\NN$.
\end{thm}


\parbf{Classical definition.}
More commonly, a nonprincipal ultrafilter is defined as a collection, say $\mathfrak{F}$, of sets in $\NN$ such that
\begin{enumerate}
\item\label{filter:supset} if $P\in \mathfrak{F}$ and $Q\supset P$, then $Q\in \mathfrak{F}$,
\item\label{filter:cap} if $P, Q\in \mathfrak{F}$, then $P\cap Q\in \mathfrak{F}$,
\item\label{filter:ultra} for any subset $P\subset\NN$, either $P$ or its complement is an element of $\mathfrak{F}$.
\item\label{filter:non-prin} if $F\subset \NN $ is finite, then $F\notin \mathfrak{F}$.
\end{enumerate}
Setting $P\in\mathfrak{F}\Leftrightarrow\omega(P)=1$ makes these two definitions equivalent.

A nonempty collection of sets $\mathfrak{F}$ that does not include the empty set and satisfies only conditions \ref{filter:supset} and \ref{filter:cap} is called a \index{filter}\emph{filter}; 
if in addition $\mathfrak{F}$ satisfies condition \ref{filter:ultra} it is called an \index{ultrafilter}\emph{ultrafilter}.
From Zorn's lemma, it follows that every filter contains an ultrafilter.
Thus there is an ultrafilter $\mathfrak{F}$ contained in the filter of all complements of finite sets; clearly, this ultrafilter $\mathfrak{F}$ is nonprincipal.


\parbf{Stone--\v{C}ech compactification.}
Given a set $S\subset \NN$, consider subset $\Omega_S$ of all ultrafilters $\omega$ such that $\omega(S)=1$.
It is straightforward to check that the sets $\Omega_S$ for all $S\subset \NN$ form a topology on the set of ultrafilters on $\NN$. 
The obtained space was first considered by Andrey Tikhonov and called \index{Stone--\v{C}ech compactification}\emph{Stone--\v{C}ech compactification} of $\NN$;
it is usually denoted as $\beta\NN$.

Let $\omega_n$ denotes the principal ultrafilter such that $\omega_n(\{n\})=1$; that is, $\omega_n(S)=1$ if and only if $n\in S$.
Note that $n\mapsto\omega_n$ defines a natural embedding $\NN\hookrightarrow\beta\NN$. 
Using the described embedding, we can (and will) consider $\NN$ as a subset of $\beta\NN$.

The space $\beta\NN$ is the maximal compact Hausdorff space that contains $\NN$  as an everywhere dense subset.
More precisely, for any compact Hausdorff space $\spc{X}$ 
and a map $f\:\NN\to \spc{X}$ there is a unique continuous map $\bar f\:\beta\NN\to X$ such that the restriction $\bar f|_\NN$ coincides with $f$. 

\section{Ultralimits of points}
\label{ultralimits}\index{ultralimit}

Let us fix a nonprincipal  ultrafilter $\omega$ once and for all.

Assume $x_n$ is a sequence of points in a metric space $\spc{X}$. 
Let us define the \index{$\omega$-limit}\emph{$\omega$-limit} of a sequence $x_1,x_2,\dots$ as the point $x_\omega$ 
such that for any $\eps>0$, point $x_n$ lie in $\oBall(x_\omega,\eps)$ for $\omega$-almost all $n$; 
that is,
\[\omega\set{n\in\NN}{\dist{x_\omega}{x_n}{}<\eps}=1.\]
In this case, we will write 
\[x_\omega=\lim_{n\to\omega} x_n
\ \ \text{or}\ \ 
x_n\to x_\omega\ \text{as}\ n\to\omega.\]

For example, if $\omega_n$ is the \textit{principal} ultrafilter such that $\omega_n\{n\}=1$ for some $n\in\NN$, then
$x_{\omega_n}=x_n$.

The sequence $x_n$ can be regarded as a map $\NN\to\spc{X}$ defined by $n\mapsto x_n$.
If $\spc{X}$ is compact, then the map $\NN\to\spc{X}$ can be extended to a continuous map $\beta\NN\to\spc{X}$ from the Stone--\v{C}ech compactification $\beta\NN$ of $\NN$.
Then the $\omega$-limit $x_\omega$ can be regarded as the image of $\omega$.

Note that the $\omega$-limits of a sequence and its subsequence may differ.
For example, sequence $y_n=-(-1)^n$ is a subsequence of $x_n=(-1)^n$, but for any ultrafilter $\omega$, we have
\[\lim_{n\to\omega}x_n
\ne
\lim_{n\to\omega}y_n.\] 

\begin{thm}{Proposition}\label{prop:ultra/partial}
Let $\omega$ be a nonprincipal ultrafilter.
Assume $x_n$ is a sequence of points in a metric space $\spc{X}$
and $x_n\to  x_\omega$ as $n\to\omega$.
Then $x_\omega$ is a partial limit of the sequence $x_n$;
that is, there is a subsequence $(x_n)_{n\in S}$ that converges to $x_\omega$ in the usual sense.
\end{thm}

\parit{Proof.}
Given $\eps>0$, 
set $S_\eps=\set{n\in\NN}{\dist{x_n}{x_\omega}{}<\eps}$.

Note that $\omega(S_\eps)=1$ for any $\eps>0$.
Since $\omega$ is nonprincipal, the set $S_\eps$ is infinite.
Therefore, we can choose an increasing sequence $n_k$
such that $n_k\in S_{\frac1k}$ for each $k\in \NN$.
Clearly, $x_{n_k}\to x_\omega$ as $k\to\infty$.
\qeds

The following proposition 
is analogous to the statement that any sequence in a compact metric space 
has a convergent subsequence;
it can be proved the same way.

\begin{thm}{Proposition}\label{prop:ultra/compact}
Let $\spc{X}$ be a compact metric space.
Then
any sequence $x_n$ of points in $\spc{X}$ has a unique $\omega$-limit $x_\omega$.

In particular, a bounded sequence of real numbers has a unique $\omega$-limit.
\end{thm}

The following lemma is an ultralimit analog of the Cauchy convergence test.

\begin{thm}{Lemma}\label{lem:X-X^w}
Let $x_n$ be a sequence of points in a complete space~$\spc{X}$. 
Assume for each subsequence $y_n$ of $x_n$, 
the $\omega$-limit 
\[y_\omega=\lim_{n\to\omega}y_{n}\in \spc{X}\]
is defined and does not depend on the choice of subsequence, then the sequence $x_n$ converges in the usual sense.
\end{thm}

\parit{Proof.} If $x_n$ is not a Cauchy sequence, then for some $\eps>0$, there is a subsequence $y_n$ of $x_n$ such that $\dist{x_n}{y_n}{}\ge\eps$ for all $n$.

It follows that $\dist{x_\omega}{y_\omega}{}\ge \eps$ --- a contradiction.\qeds

\begin{thm}{Exercise}\label{ex:linear}
Recall that $\ell^\infty$ denotes the space of bounded sequences of real numbers.
Show that there is a linear functional $L\:\ell^\infty\to\RR$ such that
for any sequence $\bm{s}=(s_1,s_2,\dots)\in \ell^\infty$ the image $L(\bm{s})$ is a partial limit of $s_1,s_2,\dots$
\end{thm}

\begin{thm}{Exercise}\label{ex:ultrakatetov+}
Suppose that $f\:\NN\to\NN$ is a map such that 
\[\lim_{n\to\omega}x_n=\lim_{n\to\omega}x_{f(n)}\]
for any bounded sequence $x_n$ of real numbers.
Show that $f(n)=n$ for $\omega$-almost all $n\in\NN$.
\end{thm}

\section{An illustration}

\begin{thm}{Claim}
Let $\spc{X}$ and $\spc{Y}$ be compact spaces.
Suppose that for every $n\in\NN$ there is a $\tfrac1n$-isometry $f_n\:\spc{X}\to \spc{Y}$.
Then there is an isometry $\spc{X}\to \spc{Y}$.
\end{thm}

We give a proof of this claim only as an illustration for ulralimits.

\parit{Proof.}
Consider the $\omega$-limit $f_\omega$ of~$f_n$;
according to \ref{prop:ultra/compact}, $f_\omega$ is defined.
Since 
\[|f_n(x)-f_n(x')|\lege |x-x'|\pm\tfrac1n\]
we get that 
\[|f_\omega(x)-f_\omega(x')|= |x-x'|\]
for any $x,x'\in \spc{X}$;
that is, $f_\omega$ is distance-preserving.

Further, since $f_n$ is a $\tfrac1n$-isometry,
for any $y\in \spc{Y}$ there is a sequence $x_n\in \spc{X}$ such that $|f_n(x_n)-y|\le \tfrac1n$.
Therefore,
\[f_\omega(x_\omega)=y,\]
where $x_\omega$ is the $\omega$-limit of $x_n$;
that is, $f_\omega$ is onto.

It follows that $f_\omega\:\spc{X}\to\spc{Y}$ is an isometry.
\qeds

\section{Ultralimits of spaces}\label{sec:Ultralimit of spaces}

Recall that $\omega$ denotes a nonprincipal ultrafilter on the set of natural numbers.

Let $\spc{X}_n$ be a sequence of metric spaces.
Consider all sequences of points $x_n\in \spc{X}_n$.
On the set of all such sequences,
define a semimetric by
\[\dist{(x_n)}{(y_n)}{}
=
\lim_{n\to\omega} \dist{x_n}{y_n}{\spc{X}_n}.
\eqlbl{eq:olim-dist}\]
Note that the $\omega$-limit on the right-hand side is always defined 
and takes a value in $[0,\infty]$. 
(The $\omega$-convergence to $\infty$ is defined analogously to the usual convergence to $\infty$).

Set $\spc{X}_\omega$ to be the corresponding metric space; 
that is, the underlying set of $\spc{X}_\omega$ is formed by classes of equivalence of sequences of points $x_n\in\spc{X}_n$ 
defined by 
\[(x_n)\sim(y_n)
\ \Leftrightarrow\ 
\lim_{n\to\omega} \dist{x_n}{y_n}{}=0\]
and the distance is defined by \ref{eq:olim-dist}.

The space $\spc{X}_\omega$ is called the \index{$\omega$-limit space}\emph{$\omega$-limit} of $\spc{X}_n$.
(It is also called \index{$\omega$-product}\emph{$\omega$-product}; this term is motivated by the fact that   
$\spc{X}_\omega$ is a quotient of the product $\prod\spc{X}_n$)
Typically  $\spc{X}_\omega$ will denote the  
$\omega$-limit of sequence $\spc{X}_n$;
we may also write  
\[\spc{X}_n\to\spc{X}_\omega\ \ \text{as}\ \  n\to\omega\ \ \text{or}\ \ \spc{X}_\omega=\lim_{n\to\omega}\spc{X}_n.\]

Given a sequence $x_n\in \spc{X}_n$,
we will denote by $x_\omega$ its equivalence class which is a point in $\spc{X}_\omega$;
it can be written as
\[x_n\to x_\omega \ \ \text{as}\ \  n\to\omega,\ \ \text{or}\ \ x_\omega=\lim_{n\to\omega} x_n.\]

\begin{thm}{Observation}\label{obs:ultralimit-is-complete}
The $\omega$-limit of any sequence of metric spaces is complete. 
\end{thm}

We will repeat the proof of \ref{ex:complete-completion} using a slightly different language.

\parit{Proof.}
Let $\spc{X}_n$ be a sequence of metric spaces and $\spc{X}_n\to\spc{X}_\omega$ as $n\to\omega$.
Choose a Cauchy sequence $x_1,x_2,\dots{}\in\spc{X}_\omega$.
Passing to a subsequence, we can assume that $\dist{x_k}{x_{m}}{\spc{X}_\omega}<\tfrac1{k}$ if $k<m$.

Choose a double sequence $x_{n,m}\in \spc{X}_n$ such that for any fixed $m$ we have $x_{n,m}\to x_m$ as $n\to\omega$.
Note that for any $k<m$ the inequality $\dist{x_{n,k}}{x_{n,m}}{}<\tfrac1{k}$ holds for $\omega$-almost all $n$.

Given $m\in\NN$, consider the subset $S_m\subset\NN$ defined by
\[S_m=\set{n\ge m}{\dist{x_{n,k}}{x_{n,l}}{}<\tfrac1{k} \quad\text{for all}\quad k<l\le m}.\]
Note that 
\begin{itemize}
\item $\NN= S_1\supset S_2\supset\dots$
\item $\omega(S_m)=1$ for each $m$, and
\item $\min S_m\ge m$.
\end{itemize}

Consider the sequence $y_n=x_{n,m(n)}$, where $m(n)$ is the largest value such that $n\in S_{m(n)}$;
from above, $m(n)\le n$.
Denote by $y_\omega\in \spc{X}_\omega$ the $\omega$-limit of $y_n$.

Observe that $|y_m-x_{n,m}|<\tfrac1{m}$ for $\omega$-almost all $n$.
It follows that $|x_m-y_\omega|\le \tfrac1{m}$ for any $m$.
Therefore, $x_n\to y_\omega$ as $n\to \infty$.
That is, any Cauchy sequence in $\spc{X}_\omega$ converges.
\qeds

\begin{thm}{Observation}\label{obs:ultralimit-is-geodesic}
The $\omega$-limit of any sequence of length spaces is geodesic. 
\end{thm}

\parit{Proof.}
If $\spc{X}_n$ is a sequence of length spaces, then for any sequence of pairs $x_n, y_n\in X_n$ there is a sequence of $\tfrac1n$-midpoints $z_n$.

Let $x_n\to x_\omega$, $y_n\to y_\omega$ and $z_n\to z_\omega$ as $n\to \omega$.
Note that $z_\omega$ is a midpoint of $x_\omega$ and $y_\omega$ in $\spc{X}_\omega$.

By Observation~\ref{obs:ultralimit-is-complete}, $\spc{X}_\omega$ is complete.
Applying Lemma~\ref{lem:mid>geod} we get the statement.
\qeds


\begin{thm}{Exercise}\label{ex:lim(tree)}
Show that an ultralimit of metric trees is a metric tree. 
\end{thm}

\begin{thm}{Exercise}\label{ex:ultracompact}
Suppose that $\spc{X}_\infty$ and $\spc{X}_1,\spc{X}_2,\dots$ are compact metric spaces.
Assume $\spc{X}_n\GHto\spc{X}_\infty$.
Show that $\spc{X}_\omega\iso\spc{X}_\infty$.
\end{thm}


\section{Ultrapower}

If all the metric spaces in the sequence are identical $\spc{X}_n=\spc{X}$, 
its $\omega$-limit 
$\lim_{n\to\omega}\spc{X}_n$
is denoted by $\spc{X}^\omega$
and called \index{ultrapower}\index{$\omega$-power}\emph{$\omega$-power} of $\spc{X}$ (also known as \index{ultracompletion}\index{$\omega$-completion}\emph{$\omega$-completion}).



\begin{thm}{Exercise}\label{ex:ultrapower}
For any point $x\in \spc{X}$, consider the constant sequence $x_n=x$
and set $\iota(x)=\lim_{n\to\omega}x_n\in\spc{X}^\omega$.

\begin{subthm}{ex:ultrapower:a}
Show that $\iota\:\spc{X}\to\spc{X}^\omega$ is distance-preserving embedding. (So we can and will consider $\spc{X}$ as a subset of $\spc{X}^\omega$.)
\end{subthm}

\begin{subthm}{ex:ultrapower:compact} 
Show that $\iota$ is onto if and only if $\spc{X}$ is compact.
\end{subthm}

\begin{subthm}{ex:ultrapower:proper} 
Show that if $\spc{X}$ is proper, then $\iota(\spc{X})$ forms a metric component of $\spc{X}^\omega$; that is, a subset of $\spc{X}^\omega$ that lies at a finite distance from a given point.
\end{subthm}

\end{thm}

Note that \ref{SHORT.ex:ultrapower:compact} implies that the inclusion $\spc{X}\hookrightarrow\spc{X}^\omega$ is not onto if the space $\spc{X}$ is not compact.
However, the spaces $\spc{X}$ and $\spc{X}^\omega$ might be isometric; here is an example:

\begin{thm}{Exercise}\label{ex:isom-ultrapower}
Let $\spc{X}$ be an infinite countable set with discrete metric;
that is $\dist{x}{y}{\spc{X}}=1$ if $x\ne y$.
Show that 

\begin{subthm}{ex:isom-ultrapower:no}
$\spc{X}^\omega$ is not isometric to $\spc{X}$.
\end{subthm}

\begin{subthm}{ex:isom-ultrapower:yes}
$\spc{X}^\omega$ is  isometric to $(\spc{X}^\omega)^\omega$.
\end{subthm}

\end{thm}

\begin{thm}{Exercise}\label{ex:ultrapower(ultrapower)}
Given a nonprincipal ultrafilter $\omega$, construct an ultrafilter $\omega_1$ such that 
\[\spc{X}^{\omega_1}\iso(\spc{X}^\omega)^\omega\]
for any metric space~$\spc{X}$.

\end{thm}


\begin{thm}{Observation}\label{obs:ultrapower-is-geodesic}
Let $\spc{X}$ be a complete metric space. 
Then $\spc{X}^\omega$ is geodesic space if and only if $\spc{X}$ is a length space.
\end{thm}

\parit{Proof.}
The if part follows from \ref{obs:ultralimit-is-geodesic}; it remains to prove the only-if part

Assume $\spc{X}^\omega$ is geodesic space.
Then any pair of points $x,y\in \spc{X}$ has a midpoint $z_\omega\in\spc{X}^\omega$.
Fix a sequence of points $z_n\in  \spc{X}$ such that $z_n\to z_\omega$ as $n\to \omega$.

Note that 
$\dist{x}{z_n}{\spc{X}}\to \tfrac12\cdot \dist{x}{y}{\spc{X}}$
and 
$\dist{y}{z_n}{\spc{X}}\to \tfrac12\cdot \dist{x}{y}{\spc{X}}$
as 
$n\to\omega$.
In particular, for any $\eps>0$, the point $z_n$ is an $\eps$-midpoint of $x$ and $y$ for $\omega$-almost all $n$.
It remains to apply \ref{lem:mid>geod}.
\qeds

\begin{thm}{Exercise}\label{ex:two-geodesics-in-ultrapower}
Assume $\spc{X}$ is a complete length space 
and $p,q\in\spc{X}$ cannot be joined by a geodesic in $\spc{X}$.  
Show that there are at least continuum of distinct geodesics between $p$ and $q$ 
in the ultrapower $\spc{X}^\omega$.
\end{thm}

\begin{thm}{Exercise}\label{ex:notproper-limit}
Construct a proper metric space $\spc{X}$ such that $\spc{X}^\omega$ is not proper;
that is, there is a point $p\in \spc{X}^\omega$ and $R<\infty$ such that the closed ball $\cBall[p,R]_{\spc{X}^\omega}$ is not compact.
\end{thm}

\section{Tangent and asymptotic spaces}
\label{sec:tan+asymptotic}

Choose a space $\spc{X}$ and a sequence $\lambda_n$ of positive numbers.
Consider the sequence of \index{rescaled space}\emph{rescalings} $\spc{X}_n=\lambda_n\cdot\spc{X}=(\spc{X},\lambda_n\cdot\dist{*}{*}{\spc{X}})$.

Choose a point $p\in \spc{X}$ and denote by $p_n$ the corresponding point in $\spc{X}_n$.
Consider the $\omega$-limit $\spc{X}_\omega$ of $\spc{X}_n$ (one may denote it by $\lambda_\omega\cdot \spc{X}$);
set $p_\omega$ to be the $\omega$-limit of $p_n$.

If $\lambda_n\to \infty$ as $n\to\omega$, then the metric component of $p_\omega$ in $\spc{X}_\omega$ is called \index{$\lambda_\omega$-tangent space}\emph{$\lambda_\omega$-tangent space} at $p$ and denoted by $\T_p^{\lambda_\omega}\spc{X}$ (or $\T_p^{\omega}\spc{X}$ if $\lambda_n=n$).\label{page:ultratangent space}

If $\lambda_n\to 0$ as $n\to\omega$, then the metric component of $p_\omega$ is called \index{$\lambda_\omega$-asymptotic space}\emph{$\lambda_\omega$-asymptotic space}%
\footnote{Often it is called an {}\emph{asymptotic cone} despite that it is not a cone in general; this name is used since in good cases it has a cone structure.} and denoted by $\Asym\spc{X}$ or $\Asym^{\lambda_\omega}\spc{X}$.
Note that the space $\Asym\spc{X}$ and its point $p_\omega$ do not depend on the choice of $p\in \spc{X}$.

The following exercise states that the constructions above depend on the sequence $\lambda_n$ and a nonprincipal ultrafilter $\omega$.

\begin{thm}{Exercise}\label{ex:ultraT}
Construct a metric space $\spc{X}$ with a point $p$ such that the tangent space
$\T_p^{\lambda_\omega}\spc{X}$ (or its asymptotic cone $\Asym^{\lambda_\omega}\spc{X}$) depends on the sequence $\lambda_n$ and/or ultrafilter~$\omega$.
\end{thm}

For nice spaces, different choices of the sequence of coefficients and ultrafilter may give the same space; 
some examples are given in the following exercise.

\begin{thm}{Exercise}\label{ex:Asym(Lob)}
Let $\spc{T}=\Asym\spc{L}$, where $\spc{L}$ is the Lobachevsky plane, or Lobachevsky space, or 3-regular%
\footnote{that is, the degree of any vertex is 3.}
metric tree with unit edge length (choose your favorite space from these three).

\begin{subthm}{ex:Asym(Lob):metric-tree}
Show that $\spc{T}$ is a complete metric tree.
\end{subthm}

\begin{subthm}{ex:Asym(Lob):homogeneous}
Show that $\spc{T}$ is one-point-homogeneous; that is, given two points $s,t\in \spc{T}$ there is an isometry of $\spc{T}$ that maps $s$ to $t$.
\end{subthm}

\begin{subthm}{ex:Asym(Lob):continuum}
Show that $\spc{T}$ has \index{degree}\emph{continuum degree} at any point;
that is, for any point $t\in \spc{T}$ the set of connected components of the complement $\spc{T}\setminus\{t\}$ has cardinality continuum.
\end{subthm}

\end{thm}


\begin{thm}{Exercise}\label{ex:T(Sx[0,1]/Sx0)}
Consider the quotient space $\spc{X}=\mathbb{S}^1\times[0,1]/\mathbb{S}^1\times\{0\}$;
that is, 
$\dist{(u_1,t_1)}{(u_2,t_2)}{\spc{X}}
=
\min\{\,\dist{(u_1,t_1)}{(u_2,t_2)}{\mathbb{S}^1\times[0,1]},t_1+t_2\,\}$.
Describe the ultratangent space $\T_o^{\omega}\spc{X}$, where $o\in\spc{X}$ is the point that corresponds to $\mathbb{S}^1\times\{0\}$.
\end{thm}


\section{Remarks}

A nonprincipal ultrafilter $\omega$ is called 
\emph{selective}\index{ultrafilter!selective ultrafilter}\index{selective ultrafilter} if for any partition of $\NN$ into sets $\{C_\alpha\}_{\alpha\in\IndexSet}$ such that $\omega(C_\alpha)\z=0$ for each $\alpha$, 
there is a set $S\subset \NN$ such that $\omega(S)=1$ and $S\cap C_\alpha$ is a one-point set for each $\alpha\in\IndexSet$.

The existence of a selective ultrafilter follows from the continuum hypothesis \cite{rudin}.

If needed, we may assume that the chosen ultrafilter $\omega$ is selective.
In this case \textit{the subsequence $(x_n)_{n\in S}$ in \ref{prop:ultra/partial} can be chosen so that $\omega(S)=1$}.

%\chapter{Metric plus measure}

\section{Borel sets}

Let us remind few definitions assuming knowleage of basic measure theory;
comprehensive treatments can be found in \cite{billingsley} and \cite{bogachev}.

Let $\spc{X}$ be a metric space.
\index{Borel set}\emph{Borel set} is any subset of $\spc{X}$ that can be formed from open sets using the countable union, countable intersection, and complement.
In other words, Borel sets form the minimal sigma-algebra that included open sets.

A measure on metric space will be always assumed to be \index{Borel measure}\emph{Borel};
that is, it is defined on the sigma-algebra of Borel sets.
A Borel measure can be uniquely determined by its values on all open (or closed) sets.

A measure $\mu$ on $\spc{X}$ is called \index{probability measure}\emph{probability measure} if $\mu\spc{X}=1$.

Recall that \index{delta-measure}\emph{delta-measure} is a probability measure with support at one point.
Delta-measure with support in $\{x\}$ will be denoted by~\index{$\delta_{x}$}$\delta_{x}$; so
\[\text{if}\quad x\in A,\quad\text{then}\quad \delta_x(A)=1,\quad\text{otherwise}\quad\delta_x(A)=0.\]

Let $\mu_n$ be a sequence of Borel measures on $\spc{X}$.
A measure $\mu_\infty$ is a \index{weak limit}\emph{weak limit} of $\mu_n$ if 
\[\int_{\spc{X}}f\cdot(\mu_n-\mu_\infty)\to0\gamma
\quad\text{as}\quad
n\to\infty
\]
for any continuous function $f\:\spc{X}\to \RR$.

Suppose $\mu$ is a measure on a metric space $\spc{X}$ and $f\:\spc{X}\to \spc{Y}$ is a measurable map;
that is, for any Borel set $B\subset \spc{Y}$, its inverse image $f^{-1}B$ is a Borel set in $\spc{Y}$.

Consider the unit interval with its Lebesgue mesure.
If $\spc{X}$ is a complete separable metric space with probability measure $\mu$, then there is a measurable map $[0,1]\to \spc{X}$

\section{Metric on measures}

Imagine that we need to transport dirt from one pile of a given shape to make another pile of a needed shape.
Suppose that cost of transporting a unit of dirt equals to the traveled distance.%
\footnote{This is the simplest cost function one can imagine.
One may consider other cost functions; for example, if the cost proportional to the square of the distance, then the problem has more applications.}
We are free to choose a destination point for dirt from a given place.
How to minimize the total cost?

To formalize this question,
suppose that the piles of dirt are described by Borel probability measures $\mu$ and $\nu$ on a metric space~$\spc{X}$.

To describe where each piece of dirt goes, we will use the so-called \index{plan}\emph{plan} for $\mu$ and $\nu$.
It is a probability measure $\pi$ on the product $\spc{X}\times\spc{X}$ such that 
for all measurable sets $A \subset \spc{X}$, we have 
\[\mu A= \pi [A \times \spc{X}],
\quad\text{and}\quad
\nu A=\pi [\spc{X}\times A].
\eqlbl{eq:marginals}\]
Equivalently it can be described as a measure that satisfies the following identity
\begin{align*}
\int_{(x,y)\in \spc{X}\times\spc{X}}f(x)\cdot g(y) \cdot \pi
&=
\int_{x\in \spc{X}}f(x)\cdot \mu
\oldcdot \int_{y\in \spc{X}}g(y)\cdot \nu,
\end{align*}
for any continuous functions $f,g\:\spc{X}\to \RR$.

Given a measure $\pi$ on $\spc{X}\times\spc{X}$, the measures $\mu$ and $\nu$ defined by \ref{eq:marginals} are called first and second \index{marginal}\emph{marginals} of $\pi$;
so the statement \textit{$\pi$ is a plan for $\mu$ and $\nu$} is equivalent to \textit{$\mu$ and $\nu$ are the first and second marginals of $\pi$ respectively}.

\begin{thm}{Claim}\label{clm:plan-exists}
There is a plan $\pi$ for any two given Borel probability measures $\mu$ and $\nu$.
\end{thm}

The plan constructed in the proof distributes equally each piece of dirt in the new pile.
As we will see this plan is usually far from optimum.

\parit{Proof.}
Consider the measure $\pi$ that is uniquely defined  defined by the identity
\[\pi(A\times B)=\mu A\cdot \mu B\]
for any Borel subsets $A,B\subset\spc{X}$.
Observe that $\pi$ is a plan for $\mu$ and~$\nu$.
\qeds

Denote by $\Pi(\mu,\nu)$ the set of all plans for $\mu$ and $\nu$;
by \ref{clm:plan-exists}, $\Pi(\mu,\nu)\z\ne\emptyset$.
It is straightforwrd to check that the following formula defines a metric on the space of probability measures on $\spc{X}$.
\[\dist{\mu}{\nu}{\Wass_1\spc{X}}
\df
\inf_{\pi\in\Pi(\mu,\nu)}
\left\{\,\int_{(x,y)\in\spc{X}\times\spc{X}}\dist{x}{y}{\spc{X}}\cdot\pi\,\right\}.\]
This metric is called \index{Wasserstein distance}\emph{Wasserstein distance of order 1} between $\mu$ and $\nu$.

In genereral, the Wasserstein distance $\dist{\mu}{\nu}{}$ might take infinite value, but all measures with compact support lie on finite distance from each other in the obtained $\infty$-metric space.
The metric component of these measures is called \index{Wasserstein space}\emph{Wasserstein space} of order 1 over $\spc{X}$; 
it is denoted by $\Wass_1\spc{X}$.
In other words, $\Wass_1\spc{X}$ is the space of all Borel probability measures $\mu$ such that 
$\int\distfun_p\cdot\mu<\infty$ for some (and therefore any) point $p\in \spc{X}$.

\begin{thm}{Exercise}\label{ex:wasserstein-infty}
Construct two Borel probability measures $\mu$ and $\nu$ on $\RR$ with Wasserstein distance $\dist{\mu}{\nu}{}=\infty$.
\end{thm}


\begin{thm}{Exercise}\label{ex:wasserstein-compact}
Show that $\Wass_1\spc{X}$ is a compact if and only if so is~$\spc{X}$.
\end{thm}

\begin{thm}{Exercise}\label{ex:wasserstein-length}
Show that the Wasserstein space $\Wass_1\spc{X}$ is a geodesic space for any metric space $\spc{X}$.
\end{thm}

\section{Optimal plan}

A plan $\pi$ for given measures $\mu$ and $\nu$ is called \index{optimal plan}\emph{optimal} if 
\[\dist{\mu}{\nu}{\Wass_1\spc{X}}
=\int_{(x,y)\in\spc{X}\times\spc{X}}\dist{x}{y}{\spc{X}}\cdot\pi.\]

\begin{thm}{Theorem} %Vilani:Theorem 1.4
Let $\mu$ and $\nu$ be probability Borel measures on a compact metric space $\spc{X}$.
Then there is an optimal plan $\pi$ for $\mu$ and~$\nu$.
\end{thm}

\parit{Proof.}
By the definition of Wasserstein distance, we can choose a sequence of plans $\pi_n$ for $\mu$ and $\nu$ such that 
\[\int_{(x,y)\in\spc{X}\times\spc{X}}\dist{x}{y}{\spc{X}}\cdot\pi_n\to \dist{\mu}{\nu}{\Wass_1\spc{X}}\]
as $n\to \infty$.

Observe that $\pi_n$ has a weak partial limit, say $\pi$.
Moreover, $\pi$ is an optimal plan for $\mu$ and $\nu$.
\qeds

\begin{thm}{Theorem}
Any optimal plan $\pi$ is \index{cyclic monotonicity}\emph{cyclically monotonic}.
That is, suppose $\pi$ is an optimal plan for probability measures $\mu$ and $\nu$ on a metric space $\spc{X}$.
Then any sequence of pairs $(x_1,y_1),\dots,(x_n,y_n)\in\supp\pi\subset\spc{X}\times\spc{X}$ we have
\[\sum_i\dist{x_i}{y_i}{}
\le
\sum_i\dist{x_{i+1}}{y_i}{},\]
here the index $i$ in the sum is taken modulo $n$; in particular $x_{n+1}\z=x_1$.
\end{thm}

\parit{Proof.}
Assume that the cyclic monotonicity does not hold;
that is,
\[R=\sum_i\dist{x_i}{y_i}{}
-
\sum_i\dist{x_{i+1}}{y_i}{}>0,\]
for some $(x_0,y_0),\dots,(x_n,y_n)\in\supp\pi$.
We need to show that $\pi$ is not optimal;
in other words we need to construct another plan $\pi'$ for $\mu$ and $\nu$ such that 
\[\int_{(x,y)\in\spc{X}\times\spc{X}}\dist{x}{y}{\spc{X}}\cdot(\pi'-\pi)<0.\eqlbl{pi'<pi}\]

Assume $\spc{X}$ is finite.
In this case we can choose $\eps>0$ such that 
$\pi\{(x_i,y_i)\}>\eps$ for each $i$.
Let
\[\pi'=\pi-\eps\cdot\sum_i(\sigma_i-\sigma_i')\eqlbl{eq:pi'}\]
where $\sigma_i=\delta_{(x_i,y_i)}$ and $\sigma_i'=\delta_{(x_{i+1},y_i)}$.
It remains to observe that $\pi'$ is a plan for $\mu$ and $\nu$ that satisfies \ref{pi'<pi}.

The general case is similar, we only need to redefine $\eps$, $\sigma_i$, and~$\sigma_i'$.
Note that given $r>0$, we can choose a probability measures $\sigma_i$ with support in $\oBall((x_i,y_i),r)_{\spc{X}\times\spc{X}}$ such that $\eps\cdot \sigma_i<\pi$ for some fixed $\eps>0$ and every $i$.
Further, denote by $\zeta_i$ and $\eta_i$ the first and second marginals of $\sigma_i$.
Observe that $\supp\zeta_i\subset\oBall(x_i,r)$ and $\supp\eta_i\subset\oBall(y_i,r)$ for all $i$.
Let $\sigma_i'$ be a plan for $\zeta_{i+1}$ and $\eta_i$.
Evidently 
\begin{align*}
\int_{(x,y)\in\spc{X}\times\spc{X}}\dist{x}{y}{}\cdot \sigma_i
&\lessgtr
\dist{x_i}{y_i}{}\pm 2\cdot r,
\\
\int_{(x,y)\in\spc{X}\times\spc{X}}\dist{x}{y}{}\cdot \sigma_i'
&\lessgtr
\dist{x_{i+1}}{y_i}{}\pm 2\cdot r.
\end{align*}
Taking $r<\tfrac R{10\cdot n}$, we get  \ref{pi'<pi}. 
\qeds




\section{Capitalistic approach}

Imagine that measures $\mu$ and $\nu$ describe the production and consumer of beer in the space.
A transportaition company transports beer from $\mu$ to $\nu$ and want to maximize its profit by adjusting price $f(x)$ of beer the point $x$; they buy beer at price $f(x)$ per unit, move it to an other point $y$ and sale it with (presumably higher) price $f(y)$.
However, the function $f$ is 1-Lipschitz condition;
otherwise the profit goes to second-hand dealers, or maybe it is a government regulation.
In other words, we need to maximize the following expression
\[\int_{\spc{X}} f\cdot(\mu-\nu)\]
for all $1$-Lipschitz functions $f$.
The maximal profit defines a metric

\begin{thm}{Theorem}
Let $\mu$ and $\nu$ be probability Borel measures on a compact metric space $\spc{X}$.
Then
\[\dist{\mu}{\nu}{\Wass_1\spc{X}}=\sup\int_{\spc{X}} f\cdot(\mu-\nu),\]
where the least upper bound is taken for all $1$-Lipschitz functions $f\:\spc{X}\z\to\RR$.
\end{thm}

The definition of Wassershtein metric described in the previous section reminds communist's planed economy.
The right-hand side in the above equation reminds capitalistic system.
Indeed, think that measures $\mu$ and $\nu$ describe the production and consumer of beer in the space.
A transportaition company trnasports beer from $\mu$ to $\nu$ and want to maximize its profit by adjusting price $f(x)$ of beer the point $x$.
However, the function $f$ is 1-Lipschitz condition --- this is a government regulation.




\parit{Proof.}
By the definition of Wasserstein metric, we can choose a sequence $\pi_n$ of plans  

Let us choose an optimal plan $\pi$ for $\mu$ and $\nu$; it exists by ???.
We need to find a 1-Lipschitz function $f\:\spc{X}\to\RR$ such that 
\[
\int_{\spc{X}} f\cdot(\mu-\nu)=\int_{(x,y)\in\spc{X}\times\spc{X}}\dist{x}{y}{\spc{X}}\cdot \pi.
\eqlbl{eq:f(mu-nu)}
\]

Choose $x_0\in \supp\mu$.
Note that adding a constant to $f$ does not change the left hand side in \ref{eq:f(mu-nu)}.
Therefore we can assume assume that $f(x_0)=0$ and set
\[f(x)=\sup\{\,|x_0-y_0|+\dots+|x_n-y_n|-(|x_1-y_0|+\dots+|x_n-y_{n-1}|)-|x-y_n|\,\}\]
where the least upper bound is taken for all sequences $(x_0,y_0),\z\dots,(x_n,y_n)\z\in\supp\pi$.

\qeds

\section{Metric-measure space}

A metric measure space is a metric $\spc{X}$ space with a choice of Borel probability measure $\vol$ on it.
In a metric-measure we ignore sets with vanishing volume; in other words, passing from $\spc{X}$ to the support of $\vol$ does not change the metric-measure space.

Alternatively we may start with unit interval $[0,1]$ equipped with Lebesgue measure and equip it with measurable semimetric $[0,1]\times [0,1]\to \RR$.





\section{Space of measures}


It can be equipped with the \index{Wasserstein metric}\emph{Wasserstein metric}
\[\dist{\mu}{\nu}{}\df\sup\left\{\,\int_{\spc{X}} f\cdot(\mu-\nu)\,\right\},\]
where the least upper bound is taken for all $1$-Lipschitz functions $f\:\spc{X}\to\RR$.

The Wasserstein distance $\dist{\mu}{\nu}{}$ might take infinite value, but all measures with compact support lie on finite distance from each other in the obtained $\infty$-metric space.
The metric component of these measures is called \index{Wasserstein space}\emph{Wasserstein space} of order 1 over $\spc{X}$; 
it is denoted by $\Wass_1\spc{X}$.



\section{Misc}

Suppose $\pi_n$ is a sequence of plans for $\mu$ and $\nu$.
Assume that $\pi_n$ weakly converges to a probability measure $\pi$ on $\spc{X}\times\spc{X}$.

is a weak limit of a sequence of plans $\pi_n$, then $\pi$ is a plan for $\mu$ and $\nu$ if for each $n$ $\pi_n$ is a plane for $\mu$ and $\nu$ 

Suppose that $f\:\spc{X}\to \RR$ is a 1-Lipschitz function,
so $f(x)-f(y)\le\dist{x}{y}{\spc{X}}$ for any $x,y\in \spc{X}$.
It follows that 
\begin{align*}
\int_{\spc{X}} f\cdot(\mu-\nu)&=\int_{x\in\spc{X}}f(x)\cdot\mu-\int_{y\in\spc{X}}f(y)\cdot\nu=
\\
&=\int_{(x,y)\in\spc{X}\times\spc{X}}[f(x)-f(y)]\cdot \pi\le
\\
&\le\int_{(x,y)\in\spc{X}\times\spc{X}}\dist{x}{y}{\spc{X}}\cdot \pi,
\end{align*}
where $\pi$ is a plan for $\mu$ and $\nu$.
By the definition of Wasserstein metric, we get  
\[\dist{\mu}{\nu}{\Wass_1\spc{X}}\le \int_{(x,y)\in\spc{X}\times\spc{X}}\dist{x}{y}{\spc{X}}\cdot\pi\eqlbl{wass=<int.plan}\]
for any plan $\pi$.

Next we want to show that equality holds in \ref{wass=<int.plan} for some plan $\pi$; such plans will be called \index{optimal plan}\emph{optimal}.


\parit{Proof.}
Choose a point $x_0\in \supp\mu$.
Given  $p\in \spc{X}$,
let
\[f(p)=\inf\left\{\sum_{i=0}^n\dist{x_i}{y_i}{}-\sum_{i=0}^n\dist{x_{i+1}}{y_i}{}-\dist{y_n}{p}{}\right\},
\eqlbl{eq:f(p)}\]
where the least upper bound is taken for all sequences of pairs 
\[(x_0,y_0),\z\dots,(x_n,y_n)\in \supp\pi.\eqlbl{eq:sequence}\]

Fix a sequence as in \ref{eq:sequence} and  denote by $F_\sigma(p)$ the expression under infimum in \ref{eq:f(p)}.

Let us show that 
\[F_\sigma(x_0)\ge 0.\]
Indeed, suppose $F_\sigma(x_0)<-\eps<0$.
Since $(x_i,y_i)\in \supp\pi$, we have $x_i\in\supp\mu$ and $y_i\in\supp\nu$ for any $i$.
Therefore we can choose sets $X_i\subset \oBall(x_i,\tfrac{\eps}{10\cdot n})$ and $Y_i\subset \oBall(y_i,\tfrac{\eps}{10\cdot n})$ such that 
$\mu(X_0)=\nu(Y_0)=\dots=\mu(X_n)=\nu(Y_n)$



Let us denote by $F(p)$ the expression under infimum in \ref{eq:f(p)}.
By the triangle inequality, 
\[F(q)\le F(p)+\dist{p}{q}{}.\]
Passing to the least upper bound in this inequality, we get
\[f(q)\le f(p)+\dist{p}{q}{}\]
for any $p,q\in\spc{X}$.
Hence $f$ is a 1-Lipschitz function.

Further, let us show that
\[(x,y)\in\supp\pi
\quad\Longrightarrow\quad
f(y)-f(x)=\dist{x}{y}{}\]





Suppose that cyclic monotonicity fails;
that is, there is a sequence of pairs $(x_1,y_1),\dots,(x_n,y_n)\in\spc{X}\times\spc{X}$ such that
\[\dist{x_1}{y_1}{}+\dots+\dist{x_n}{y_n}{}
>
\dist{x_1}{y_2}{}+\dots+\dist{x_{n-1}}{y_n}{}+\dist{x_{n}}{y_1}{}.\]
In this case, it would be more optimal to transport measure from a neighborhood of $x_i$ to a neighborhood of $y_{i+1}$ (
here and further we assume that the indexes are taken modulo $n$, so $n+1=1$).
The latter contradicts optimality of $\pi$.

The following argument makes it precise.
Choose small $\eps>0$.
For each $n$,
choose disjoint sets $X_i$ and $Y_i$ in $\eps$-neighborhood of $x_i$ and $y_i$
such that for some $\delta>0$ we have 
\[\pi [X_i\times Y_i]=\delta\]
for each $i$.

Let us modify the plan $\pi$ in the union $X_1\times Y_1 \cup\dots\cup X_n\times Y_n$ and such that 
$\pi'(X_i\times Y_{i+1})=\delta$ for each $i$;


Observe that
\[\int_{(x,y)\in\spc{X}\times\spc{X}}\dist{x}{y}{\spc{X}}\cdot(\pi'-\pi)>\]
\qeds


\backmatter

\newgeometry{top=0.9in, bottom=0.9in,inner=0.5in, outer=0.5in}
\chapter{Semisolutions}

{

\footnotesize
\begin{multicols}{2}

\refstepcounter{chapter}
\setcounter{eqtn}{0}

\parbf{\ref{ex:quad-inq}.}
Add four triangle inequalities (\ref{metric:triangle}).

\parbf{\ref{ex:normal}.}
Consider the function 
\[f(x)=\frac{\distfun_Ax}{\distfun_Ax+\distfun_Bx},\]
where $\distfun_Ax\z\df\inf_{a\in A}\dist{a}{x}{}$.
Show that $f$ is continuous and satisfies the needed property.

\parbf{\ref{ex:tietze}.}
Use \ref{ex:normal} to construct an approximation of the needed function and pass to a limit or find a proof of the \index{Tietze extension theorem}\emph{Tietze extension theorem}.

\parbf{\ref{ex:pseudo-infty-metric}};\ref{SHORT.ex:pseudo-infty-metric:pseudo}.
Note that if $\mu(A)=\mu(B)=0$, then $|A-B|=0$.
Therefore, \ref{metric=0} does not hold for bounded closed subsets.
It is straightforward to check the remaining conditions in~\ref{def:metric} hold true.

\parit{\ref{SHORT.ex:pseudo-infty-metric:infty}.}
Note that the distance from the empty set to the whole plane is infinite; so the value $|A-B|$ might be infinite.
It is straightforward to check the remaining conditions in~\ref{def:metric}.

\parit{Remark.}
Metrics of the form $\dist{A}{B}{}=\mu(A\bigtriangleup B)$ are very special.
In particular, they satisfy the so-called \index{hypermetric inequality}\emph{hypermetric inequalities}; that is, for any sequence of sets $A_1,\dots, A_n$ and any sequence of integers $b_1,\dots,b_n$ such that $\sum_ib_i=1$ we have
\[\sum_{i,j}b_i\cdot b_j\cdot \dist{A_i}{A_j}{}\le 0.\]
Note that for $n=3$ and $b_1=b_2=-b_3=1$ we get the usual triangle inequality.
For more on the subject, see \cite{deza-laurent}.

\parbf{\ref{ex:gluing}.}
Choose $\delta>0$ and an increasing linear bijection $\ell\:[a,b]\to [c,d]$.
Show that $\ell$ has arbitrarily close increasing piecewise-linear bijection $s\:[a,b]\to [c,d]$ such that derivative at any point is either $<\delta$ or $>\tfrac1\delta$.

Start with the identity map $[0,1]\to [0,1]$;
iterate the above construction for smaller and smaller $\delta$ and pass to the limit.
This way we obtain an increasing  bijection $x\leftrightarrow x'$ from $[0,1]$ to itself
such that for any $\eps>0$ there is a partition $0=t_0<t_1<\dots <t_{2\cdot n}=1$ of $[0,1]$ with 
\begin{align*}
\eps&>|t_0-t_1|+|t_1'-t_2'|+|t_2-t_3|+\dots
\\
&\dots+|t_{2\cdot n-2}-t_{2\cdot n-1}|+|t_{2\cdot n-1}'-t_{2\cdot n}'|.
\end{align*}
Make a conclusion.

\parbf{\ref{ex:almost-min}.}
Assume the statement is wrong. 
Then for any point $x\in \spc{X}$, there is a point $x'\in \spc{X}$ such that 
\begin{align*}
\dist{x}{x'}{}&<\rho(x)
\intertext{and}
\rho(x')&\le\frac{\rho(x)}{1+\eps}.
\end{align*}
Consider a sequence $x_1,x_2,\dots\in \spc{X}$ such that $x_{n+1}\z=x_n'$.
Show that this is a Cauchy sequence.
Since $\spc{X}$ is complete, $x_n$ converges;
denote its limit by $x_\infty$.
Since $\rho$ is a continuous function we get
\begin{align*}
\rho(x_\infty)&=\lim_{n\to\infty}\rho(x_n)=0.
\end{align*}

The latter contradicts that $\rho>0$.

\parbf{\ref{ex:complete-completion}.}
Let $\bar {\spc{X}}$ be the completion of $\spc{X}$.
By the definition, for any $y\in \bar {\spc{X}}$ there is a Cauchy sequence $x_n$ in  $\spc{X}$ that converges to $y$.

Choose a Cauchy sequence $y_m$ in $\bar {\spc{X}}$.
From above, we can choose points $x_{n,m}\in \spc{X}$ such that $x_{n,m}\to y_m$ for any $m$.
Choose $z_m=x_{n_m,m}$ such that $|y_m-z_m|<\tfrac1m$.
Observe that $z_m$ is Cauchy.
Therefore, its limit $z_\infty$ lie in $\bar{\spc{X}}$.
Finally, show that $x_m\to z_\infty$.

\parbf{\ref{ex:compact-net}.}
A compact $\eps$-net $N$ in $\spc{K}$ contains a finite $\eps$ net $F$.
Show and use that $F$ is a $2\cdot\eps$-net of $\spc{K}$.

\parbf{\ref{ex:non-contracting-map}.}
Given a pair of points $x_0,y_0\in \spc{K}$, 
consider two sequences $x_0,x_1,\dots$ and $y_0,y_1,\dots$
such that $x_{n+1}=f(x_n)$ and $y_{n+1}\z=f(y_n)$ for each $n$.

Since $\spc{K}$ is compact, 
we can choose an increasing sequence of integers $n_k$
such that both sequences $(x_{n_i})_{i=1}^\infty$ and $(y_{n_i})_{i=1}^\infty$
converge.
In particular, both are Cauchy;
that is,
\[
|x_{n_i}-x_{n_j}|_{\spc{K}}\to 0 
\quad\text{and}\quad
|y_{n_i}-y_{n_j}|_{\spc{K}}\to 0
\]
as $\min\{i,j\}\to\infty$.

Since $f$ is distance-noncontracting, 
\[
|x_0-x_{|n_i-n_j|}|
\le 
|x_{n_i}-x_{n_j}|
\]
for any $i$ and $j$.
Therefore, there is a sequence $m_i\to\infty$ such that
\[
x_{m_i}\to x_0\quad\text{and}\quad y_{m_i}\to y_0
\leqno({*})\]
as $i\to\infty$.

Since $f$ is distance-noncontracting, the sequence $\ell_n=|x_n-y_n|_{\spc{K}}$ is nondecreasing.
By $({*})$,  $\ell_{m_i}\to\ell_0$ as $m_i\to\infty$.
It follows that 
\[\ell_0=\ell_1=\dots\]
In particular, 
\[|x_0-y_0|_{\spc{K}}=\ell_0=\ell_1=|f(x_0)-f(y_0)|_{\spc{K}}\]
for any pair of points $(x_0,y_0)$ in $\spc{K}$.
That is, the map $f$ is distance-preserving; hence $f$ is injective.
From $({*})$, we also get that $f(\spc{K})$ is everywhere dense.
Since $\spc{K}$ is compact $f\:\spc{K}\to \spc{K}$ is surjective --- hence the result.

\parit{Remarks.}
This is a basic lemma in the introduction to Gromov--Hausdorff distance \cite[see 7.3.30 in][]{burago-burago-ivanov}.
The presented proof is not quite standard,
I learned it from Travis Morrison, 
a student in my MASS class at Penn State, Fall 2011.

Note that this exercise implies that \textit{any surjective non-expanding map from a compact metric space to itself is an isometry}. 

\parbf{\ref{ex:loc-compact-not-proper}.}
Check an infinite set with a discrete metric.

\parbf{\ref{ex:pogorelov}.}
Set $B_p=B(x,\tfrac \pi2)_{\mathbb{S}^2}$.
The triangle inequality follows since
\[
(B_x\setminus B_y)
\cup 
(B_y\setminus B_z)
\supseteq
B_x\setminus B_z.
\leqno(*)\]
The remaining conditions in Definition \ref{def:metric} are evident.

Observe that
$B_x\setminus B_y$
does not overlap with
$B_y\z\setminus B_z$ and  we get equality in $(*)$ if and only if $y$ lies on the great circle arc from $x$ to $z$.
Therefore, the second statement follows.


\begin{wrapfigure}{r}{24 mm}
\vskip-0mm
\centering
\includegraphics{mppics/pic-29}
\end{wrapfigure}

\parit{Remarks.}
This construction is due to 
Aleksei Pogorelov \cite{pogorelov}.
It is closely related to the construction given 
by David Hilbert \cite{hilbert}
which was the motivating example for his fourth problem. 
See also the remark after the solution of~\ref{ex:pseudo-infty-metric}.

\parbf{\ref{ex:4-point-trees}.}
We may assume that none of the points $p,x,y,z$ lies on a geodesic between the other two.

Let $K$ be the set in the tree covered by all six geodesics with the endpoints $p,x,y,z$.
Observe that $K$ looks like an H or like an X; make a conclusion.

\parit{Remarks.}The value $\tfrac12\cdot(|p-x|+|p-y|-|x-y|)$ is called \index{Gromov's product}\emph{Gromov's product} of $x$ and $y$ with the origin at $p$;
usually it is denoted by $(x|y)_p$.

Note that a four-point metric space admits an isometric embedding into a metric tree if and only if one of these two equivalent conditions holds.
Moreover, a metric space admits an isometric embedding into a metric tree if every its four-point subspace admits such embedding.


\parbf{\ref{ex:spheres-in-trees}.}
Apply \ref{ex:4-point-trees}.

\parbf{\ref{ex:1-Lip-G-delta}.}
Note that $\spc{P}$ is complete.
Choose $\eps>0$.
Use \ref{thm:length-semicont} to show that the set of paths of length $>1-\eps$ is open in~$\spc{P}$;
show that this set is also dense in~$\spc{P}$.
Apply Baire's theorem (\ref{thm:baire}).

\parit{Remark.}
You might find it surprising that \textit{most of the short maps from the sphere to the plane are \index{length-preserving map}\emph{length-preserving}}; that is, they preserve lengths of all curves.
The latter follows from the result of 
Bernd Kirchheim, 
Emanuele Spadaro,
and 
L{\'a}szl{\'o} Sz{\'e}kelyhidi \cite{KSS}.
(While most of the maps have this property, it is not at all easy to construct a single such example.) 


\parbf{\ref{ex:no-geod}.}
\textit{Formally speaking, a one-point space is a solution,
but we will construct a nontrivial example.}

Recall that $c_0\subset\ell^\infty$ denotes the space of all real sequences converging to zero.
Consider the unit ball $B$ in $c_0$;
denote by $\rho_0$ the metric on $B$.

Let \[\phi(\bm{x})=2+\tfrac{1}2\cdot x_1+\tfrac{1}4\cdot x_2+\tfrac{1}8\cdot x_3+\dots,\]
where $\bm{x}=(x_1,x_2\,\dots)\in B$.
Consider another length metric $\rho_1$ on $B$ that is different from $\rho_0$ by the conformal factor $\phi$;
that is, if $t\mapsto\bm{x}(t)$ for $t\in[0,\ell]$ is a curve parametrized by $\rho_0$-length,
then its $\rho_1$-length is defined by
\[\length_{\rho_1}\bm{x}\df\int\limits_0^\ell\phi\circ\bm{x}(t)\cdot dt.\]
Note that the metric $\rho_1$ is bilipschitz to~$\rho_0$.

Assume $t\mapsto \bm{x}(t)$ and $t\mapsto \bm{x}'(t)$ are two curves parametrized by $\rho_0$-length that differ only in the $m$-th coordinate; denote them by $x_m(t)$ and $x_m'(t)$ respectively.
Show that if $x'_m(t)\le x_m(t)$ for any $t$ and 
the function $x'_m(t)$ is locally $1$-Lipschitz at all $t$ such that $x'_m(t)< x_m(t)$, then 
\[\length_{\rho_1}\bm{x}'\le \length_{\rho_1}\bm{x}.\]
Moreover, this inequality is strict if $x'_m(t)\z< x_m(t)$ for some~$t$.

Fix a curve $\bm{x}(t)$, $t\in[0,\ell]$, parametrized by  $\rho_0$-length.
We can choose large $m$ so that $x_m(t)$ is sufficiently close to $0$ for any~$t$.
In this case, it is easy to construct a function $t\mapsto x'_m$ that meets the above conditions.
It follows that for any curve $\bm{x}(t)$ in $(B,\rho_1)$, we can find a shorter curve $\bm{x}'(t)$ with the same endpoints.
In particular, $(B,\rho_1)$ has no geodesics.

\parit{Remark.}
This solution was suggested by Fedor Nazarov~\cite{nazarov}.

\parbf{\ref{ex:compact+connceted}.}
Choose a sequence of positive numbers $\varepsilon_n\to 0$ and a finite $\varepsilon_n$-net $N_n$ of $K$ for each $n$.
We can assume that $\eps_0>\diam K$, and $N_0$ is a one-point set.
If $\dist{x}{y}{}<\eps_k$ for some $x\in N_{k+1}$ and $y\in N_{k}$, then connect them by a curve of length at most $\eps_k$.

Let $K'$ be the union of all these curves and $K$.
Show that $K'$ is compact and path-connected.

\parit{Source:} This problem is due to Eugene Bilokopytov \cite{bilokopytov}.

\parbf{\ref{ex:compact=>complete}.}
Choose a Cauchy sequence $x_n$ in $(\spc{X},\|*\z-*\|)$; it is sufficient to show that a subsequence of $x_n$ converges.

Observe that the sequence $x_n$ is Cauchy in $(\spc{X},|*-*|)$;
denote its limit by $x_\infty$.

Passing to a subsequence, we can assume that $\|x_n-x_{n+1}\|\z<\tfrac1{2^n}$.
It follows that there is a 1-Lipschitz path $\gamma$ in $(\spc{X},\|*-*\|)$ such that $x_n=\gamma(\tfrac1{2^n})$ for each $n$ and $x_\infty=\gamma(0)$.
Therefore,
\begin{align*}
\|x_\infty-x_n\|&\le \length\gamma|_{[0,\frac1{2^n}]}\le \tfrac1{2^n}.
\end{align*}
In particular, $x_n$ converges to $x_\infty$ in $(\spc{X},\|*\z-*\|)$.

\parit{Source:} \cite[Corollary]{hu-kirk}; see also \cite[Lemma 2.3]{petrunin-stadler}.

\parbf{\ref{ex:menger}.} Choose two points $x,y\in \spc{X}$;
let $\ell\z=\dist{x}{y}{}$.
Suppose $f\:E\to \spc{X}$ is a distance-preserving map such that $0,\ell\in E\subset [0,\ell]$,
$f(0)=x$, and  $f(\ell)=y$.

Show that we can choose $f$ so that $E$ is maximal;
that is, $f$ cannot be extended to a distance-preserving map on a larger subset of $[0,\ell]$.

Show that there is no open interval $(a,b)$ in the complement of $E$ such that $a,b\in E$.

Apply the completeness of $\spc{X}$ to show that $E$ is closed.
Conclude that $E=[0,\ell]$.

\parbf{\ref{ex:eps-nbhd(ball)}.}
Let $U$ be the $\eps$-neighborhood of $\oBall(x,R)_{\spc{X}}$.
By the triangle inequality, $U\z\subset \oBall(x,R+\eps)_{\spc{X}}$;
this inclusion holds in any metric space.

Choose $y\in \oBall(x,R+\eps)_{\spc{X}}$, so $\dist{x}{y}{{\spc{X}}}\z<R+\eps$.
Since ${\spc{X}}$ is a length space, there is a curve $\gamma$ from $x$ to $y$ with length less than $R+\eps$.
Show and use that $\gamma$ contains a point $m$ such that $\dist{x}{m}{{\spc{X}}}<R$ and $\dist{y}{m}{{\spc{X}}}<\eps$.

\parbf{\ref{exercise from BH}.}
Consider the following subset of $\RR^2$ equipped with the induced length metric
\[
\spc{X}
=
\bigl((0,1]\times\{0,1\}\bigr)
\cup
\bigl(\{1,\tfrac12,\tfrac13,\dots\}\times[0,1]\bigr)
\]
Note that $\spc{X}$ is locally compact and geodesic.

Its completion $\bar{\spc{X}}$ is isometric to the closure of $\spc{X}$ equipped with the induced length metric.
Note that $\bar{\spc{X}}$ is obtained from $\spc{X}$ by adding two points $p=(0,0)$ and $q\z=(0,1)$.

{

\begin{wrapfigure}{r}{20 mm}
\vskip-4mm
\centering
\includegraphics{mppics/pic-1}
\end{wrapfigure}

Observe that $p$ admits no compact neighborhood in $\bar{\spc{X}}$ and there is no geodesic connecting $p$ to $q$ in~$\bar{\spc{X}}$. 


\parit{Source:} \cite[I.3.6(4)]{bridson-haefliger}.

}

\parbf{\ref{ex:gross}.}
Suppose this number does not exist.
Show that there are two point-arrays $(x_1,\z\dots,x_n)$ and $(y_1,\dots,y_m)$
such that
\[
\min_{z\in \spc{X}}\{\,f(z)\,\}>\max_{z\in \spc{X}}\{\,h(z)\,\},
\leqno({*})
\]
where
\begin{align*}
f(z)&=\tfrac1n\cdot\sum_i|x_i-z|_{\spc{X}}
\intertext{and}
h(z)&=\tfrac1m\cdot\sum_j|y_j-z|_{\spc{X}}.
\end{align*}


Note that
\begin{align*}\tfrac1m\cdot\sum_j f(y_j)&=\tfrac1{m\cdot n}\cdot\sum_{i,j}|x_i-y_j|_{\spc{X}}=
\\
&=\tfrac1n\cdot\sum_i h(x_i);
\end{align*}
that is, the average value of $f(y_j)$ coincides with the average value of $h(x_i)$.
The latter contradicts~$({*})$.

\parit{Remark.}
The value $\ell$ is uniquely defined;
it is called the \index{rendezvous value}\emph{rendezvous value} of ${\spc{X}}$.
This is a result of Oliver Gross \cite{gross}.

%%%%%%%%%%%%%%%%%%%%%%%%%%%%%%

%%%%%%%%%%%%%%%%%%%%%%%%%
\refstepcounter{chapter}
\setcounter{eqtn}{0}

\parbf{\ref{ex:compact-length}.}
By the Fréchet lemma (\ref{lem:frechet}) we can identify $\spc{K}$ with a compact subset in $\ell^\infty$.

Denote by $\spc{L}$ the \index{closed convex hull}\emph{closed convex hull} of $\spc{K}$;
that is, $\spc{L}$ is the minimal convex closed set in $\ell^\infty$ that contains $\spc{K}$.
(In other words, $\spc{L}$ is the minimal closed set containing $\spc{K}$ such that if $x,y\in \spc{L}$, then 
$t\cdot x+(1-t)\cdot y\in \spc{L}$ for any $t\in[0,1]$.)

Observe that $\spc{L}$ is a length space.
It remains to show that $\spc{L}$ is compact.

By construction, $\spc{L}$ is a closed subset of $\ell^\infty$; in particular, it is complete.
By \ref{totally-bounded}, it remains to show that $\spc{L}$ is totally bounded.

Recall that Minkowski sum $A + B$ of two sets $A$ and $B$ in a vector space is defined by
\[A + B 
\df
\set{a+b}{a\in A,\ b\in B}.\]
Observe that the Minkowski sum of two convex sets is convex.

Denote by $\bar B_\eps$ the closed $\eps$-ball in $\ell^\infty$ centered at the origin.
Choose a finite $\eps$-net $N$ in $\spc{K}$ for some $\eps>0$.
Note that $P=\Conv N$ is a convex polyhedron; in particular, $\Conv N$ is compact.

Observe that $N+\bar B_\eps$ is a closed $\eps$-neighborhood of $N$.
It follows that $N+\bar B_\eps\supset K$ and therefore $P+\bar B_\eps\supset \spc{L}$.
In particular, $P$ is a $2\cdot\eps$-net in $\spc{L}$;
since $P$ is compact and $\eps>0$ is arbitrary, $\spc{L}$ is totally bounded (see \ref{ex:compact-net}).

\parit{Remark.}
Alternatively, one may use that \textit{the injective envelope of a compact space is compact}; see \ref{ex:inj=complete-geodesic-contractible:geodesic}, \ref{ex:Inj(compact)}, and \ref{prop:InjX-is-injective}.

\parbf{\ref{ex:frechet}.}
Modify the proof of \ref{lem:frechet}.

\begin{wrapfigure}{r}{23mm}
\vskip-6mm
\centering
\includegraphics{mppics/pic-200}
\end{wrapfigure}

\parbf{\ref{ex:inf-extension}.}
Consider the metric tree $\spc{T}$ shown on the diagram;
it is a half-line $[0,\infty)$ with attached an interval of length $n+1$ to each integer~$n\ge 0$.
Denote by $o$ the origin of the half-line
and by $x_n$ the endpoint of $n^{\text{th}}$ interval.

Observe that if $m\ne n$, then
\[|x_m-x_n|_{\spc{T}}\ge |o-x_n|_{\spc{T}}+1.\]
Show and use that for any binary sequence $\eps_n$ there is an extension function $f$ such that 
\[f(x_n)=|o-x_n|_{\spc{T}}+\eps_n.\]


\parit{Remark.}
An if-and-only-if condition on $\spc{X}$ that have separable $\spc{X}^\infty$ was found by Julien Melleray \cite[2.8]{melleray}.
A similar condition was used by Herbert Federer to describe metric spaces where Besicovitch covering lemma holds \cite[2.8.9]{federer}.

\parbf{\ref{ex:geodesics-urysohn}.}
Choose a separable space $\spc{X}$ that has an infinite number of geodesics between a pair of points with the given distance between them;
say a square in $\RR^2$ with $\ell^\infty$-metric will do.
Apply to $\spc{X}$ universality of Urysohn space (\ref{prop:sep-in-urys}).

\parbf{\ref{ex:compact-extension}.} 
First let us prove the following claim:

\begin{itemize}
\item 
Suppose $f\: K\to\RR$ is an extension function defined on a compact subset $K$ of the Urysohn space $\spc{U}$.
Then there is a point $p\in \spc{U}$ such that 
$\dist{p}{x}{}=f(x)$ for any $x\in K$.
\end{itemize}

Without loss of generality, we may assume that $f>0$.
Since $K$ is compact, we may fix $\eps>0$ such that $f(x)>\eps$ for any $x\in K$.

Consider the sequence $\eps_n=\tfrac\eps{100\cdot 2^n}$.
Choose a sequence of $\eps_n$-nets $N_n\subset K$.
Applying the universality of $\spc{U}$ recursively, we may choose a point $p_n$ such that $\dist{p_n}{x}{}=f(x)$ for any $x\in N_n$ and $\dist{p_n}{p_{n-1}}{}\z=10\cdot\eps_{n-1}$.
Observe that the sequence $p_n$ is Cauchy and its limit $p$ meets 
$\dist{p}{x}{}=f(x)$ for any $x\in K$.

Now, choose a sequence $x_n$ of points that is dense in $\spc{S}$.
Applying the claim, we may extend the map from $K$ to $K\cup\{x_1\}$, further to $K\cup\{x_1,x_2\}$, and so on.
As a result, we extend the distance-preserving map $f$ to the whole sequence $x_n$.
It remains to extend it continuously to the whole space~$\spc{S}$.

\parbf{\ref{ex:sc-urysohn}.}
It is sufficient to show that any compact subspace $\spc{K}$ of the Urysohn space $\spc{U}$ can be contracted to a point.

Note that any compact space $\spc{K}$ can be extended to a contractible compact space $\spc{K}'$; for example, we may embed $\spc{K}$ into $\ell^\infty$ and pass to its convex hull, as it was done in \ref{ex:compact-length}.

By \ref{thm:compact-homogeneous}, there is an isometric embedding of $\spc{K}'$ that agrees with the inclusion $\spc{K}\hookrightarrow\spc{U}$.
Since $\spc{K}$ is contractible in $\spc{K}'$, it is contractible in $\spc{U}$.

\parit{A better way.}
One can contract the whole Urysohn space using the following construction.

Note that points in $\spc{X}_\infty$ constructed in the proof of \ref{prop:univeral-separable} can be multiplied by $t\in [0,1]$ --- simply multiply each function by $t$.
That defines a map 
\[\lambda_t\:\spc{X}_\infty\to \spc{X}_\infty\]
that rescales all distances by factor $t$.
The map $\lambda_t$ can be extended to the completion of $\spc{X}_\infty$, which is isometric to $\spc{U}_d$ (or $\spc{U}$).

Observe that 
the map $\lambda_1$ is the identity  and $\lambda_0$ maps the whole space to a single point, say $x_0$ --- this is the only point of $\spc{X}_0$.
Further, note that $(t,p)\mapsto \lambda_t(p)$ is a continuous map; in particular, $\spc{U}_d$ and $\spc{U}$ are contractible.

As a bonus, observe that for any point $p\in \spc{U}_d$ the curve $t\mapsto \lambda_t(p)$ is a geodesic path from $p$ to $x_0$.

\parit{Source:} \cite[$\text{(d)}$ on page 82]{gromov-2007}.

\parbf{\ref{ex:no-isom}.}
Consider two infinite metric trees as on the diagram. 

\begin{Figure}
\vskip-0mm
\centering
\includegraphics{mppics/pic-205}
\end{Figure}

\parit{Remark.}
A more sophisticated example: $\spc{X}\z=\ell^\infty$ and $\spc{Y}=L^\infty([0,1])$.
Try to prove that it qualifies; see also \cite{buehler}.

%Given a bounded sequence $\bm{a}=(a_1,a_2,\dots)$, consider the function $f$ such that $f(0)=0$ and $f(x)=a_n$ if $\tfrac1{n+1}<x\le \tfrac1n$.
%Note that $\bm{a}\mapsto f$ is a distance-preserving map $\ell^\infty\to L^\infty([0,1])$.

%Further, enumerate all subintervals of $[0,1]$ with rational ends, $I_1,I_2,\dots$
%Given a function $f\in L^\infty([0,1])$ consider sequence $\bm{a}\z=(a_1,a_2,\dots)$ such that $a_n$ is the mean value of $f$ on $I_n$.
%Observe that $f\mapsto \bm{a}$ is a distance-preserving map $L^\infty([0,1])\to \ell^\infty$.

%It remains to show that $\spc{X}=\ell^\infty$ and $\spc{Y}=L^\infty([0,1])$ are not isometric???



\parbf{\ref{ex:sphere-in-urysohn}}; \ref{SHORT.ex:sphere-in-urysohn:sphere} and \ref{SHORT.ex:sphere-in-urysohn:midpoint}.
Observe that $L$ and $M$ satisfy the definition of $d$-Urysohn space and apply the uniqueness (\ref{thm:urysohn-unique}).
Note that
\[\ell=\diam L=\min\{2\cdot r, d\}.\]

\parit{\ref{SHORT.ex:sphere-in-urysohn:homogeneous}.} 
Use \ref{SHORT.ex:sphere-in-urysohn:sphere}, maybe twice.

\parbf{\ref{ex:shere}.}
Let $p$ be the center of the sphere;
without loss of generality, we can assume that $\dist{p}{x}{}\le \dist{p}{y}{}$.

Consider function $f\:\{p,x,y\}\to\RR$ defined by $f(p)=1$, $f(x)=1+\dist{p}{x}{}$, and $f(y)=1+\dist{p}{y}{}-\eps$.
Suppose $\eps>0$ is sufficiently small;
show that $f$ is an extension function on $\{p,x,y\}$.

By the extension property, there is a point $z\in \spc{U}$ such that $\dist{p}{z}{}=f(p)$, $\dist{x}{z}{}=f(x)$, and $\dist{y}{z}{}=f(y)$.
Whence the statement follows.

\parit{Source:} This problem is taken from a survey of Julien Melleray
 \cite[Prop. 4.3]{melleray}, where it was attributed to Matatyahu Rubin.


\parbf{\ref{ex:ext(shere)}.} 
Observe that the complement $\spc{V}=\spc{U}\setminus B$ is complete.
Show that it $\spc{V}$ satisfies the extension property.
Conclude that $\spc{V}$ is an Urysohn space and apply \ref{thm:urysohn-unique}.

For the second part, observe that there is an isometry $\iota\:\spc{U}\to \spc{V}$.
Moreover, if $p$ is the center of $B$, then we can assume that $\iota$ has a fixed point $x$ such that $\dist{p}{x}{}>2$.

Consider the unit sphere $S$ centered at $x$.
The restriction of $\iota$ to $S$ is an isometry of $S$.
Use \ref{ex:shere} to show that it cannot be extended to an isometry of $\spc{U}$.

\parit{Source:} \cite[Sec. 4.4]{melleray}.

\parbf{\ref{ex:katetov}.}
Apply \ref{thm:urysohn-unique} and the construction in \ref{thm:urysohn-exists+}.

\parbf{\ref{ex:homogeneous}}; \ref{SHORT.ex:homogeneous:euclidean}.
The euclidean plane is homogeneous in every sense.

\parit{\ref{SHORT.ex:homogeneous:hilbert}.}
The Hilbert space $\ell^2$ is finite-set-homogeneous, but not compact-set-homogeneous, nor countable-set-homogeneous.

\parit{\ref{SHORT.ex:homogeneous:ell-infty}.}
$\ell^\infty$ is one-point-homogeneous, but not two-point-homogeneous.
Try to show that there is no isometry of $\ell^\infty$ such that
\begin{align*}
(0,0,0,\dots)&\mapsto (0,0,0,\dots),
\\
(1,1,1,\dots)&\mapsto (1,0,0,\dots).
\end{align*}

\parit{\ref{SHORT.ex:homogeneous:ell-1}.}
$\ell^1$ is one-point-homogeneous, but not two-point-homogeneous.
Try to show that there is no isometry of $\ell^\infty$ such that
\begin{align*}
(0,0,0,\dots)&\mapsto (0,0,0,\dots),
\\
(2,0,0\dots)&\mapsto (1,1,0,\dots).
\end{align*}

\parbf{\ref{ex:homogeneous-tree}.}
Let $\spc{T}$ be a one-point-homogeneous metric tree.
Note that all points in $\spc{T}$ have the same degree $d$;
that is, for any point $t\in \spc{T}$ the set of connected components of the complement $\spc{T}\setminus\{t\}$ has the same cardinality $d$.

Show that if $d=0$, then $\spc{T}$ is a one-point space;
there is no tree with $d=1$,
and if $d=2$, then $\spc{T}\iso\RR$.

Suppose $d\ge 3$.
Choose a geodesic $\gamma$ in $\spc{T}$.
Show that number of connected components of $\spc{T}\setminus\gamma$ has cardinality continuum.
Observe and use that one can choose a point $p_\alpha$ in each connected component such that $\dist{p_\alpha}{p_\beta}{\spc{T}}>1$ if $\alpha\ne\beta$.

\parbf{\ref{ex:horobry}.}
Assume $F_{\spc{X}}$ is not an embedding.
That is, there is a sequence of points $x_1,x_2,\dots$ 
and a point $x_\infty$ such that $f_{x_n}\to f_{x_\infty}$ in $C(\spc{X},\RR)$
as $n\to \infty$, 
while $|x_n-x_\infty|_{\spc{X}}>\eps$ 
for some fixed $\eps>0$ and all~$n$.

By \ref{prop:length+proper=>geodesic}, any pair of points $x,y\in \spc{X}$ can be connected by a minimizing geodesic $[xy]$.
Choose $\bar x_n$ on a geodesic $[x_\infty x_n]$ such that $|x_\infty-\bar x_n|=\eps$.
Note that 
\begin{align*}
f_{x_n}(x_\infty)-f_{x_n}(\bar x_n)&=\eps,
\\
f_{x_\infty}(x_\infty)-f_{x_n}(\bar x_n)&=-\eps
\end{align*}
for all $n$.

Since $\spc{X}$ is proper, we can pass to a subsequence of $x_n$ so that the sequence  $\bar x_n$ converges;
denote its limit by $\bar x_\infty$.
The above identities imply that
\begin{align*}
f_{x_n}(\bar x_\infty)&\not\to f_{x_\infty}(\bar x_\infty)
\quad
\text{or}
\\
f_{x_n}(x_\infty)&\not\to f_{x_\infty}( x_\infty)
\end{align*}
--- a contradiction.

For the second part, take $\spc{Y}$ to be the set of non-negative integers with the metric $\rho$ defined by
\[\rho(m,n)=m+n\] 
for $m\ne n$.

\medskip

\parit{Source:}
I learned this example from Linus Kramer and Alexander Lytchak;
it was also mentioned in the lectures of Anders Karlsson
and attributed to Uri Bader \cite[2.3]{karlsson}.

\parbf{\ref{ex:cut}.}
Suppose that our metric is $\sum a_S\cdot\delta_S$ with $a_S\ge 0$ for any $S\subset F$.
Enumerate all the subsets $S_1,\dots,S_{2^n}$;
set $S_i=F$ for all $i>2^n$. 
Consider the maps $x\mapsto (a_1,a_2,\dots)$ where $a_i=0$ if $x\in S_i$ and otherwise $a_i=1$.
Observe that it defines a distance-preserving map $F\to \ell^1$. 

The if part is proved.
For the only-if part, check the statement for subsets of the real line, and use it.

\parbf{\ref{ex:K23}.}
Show that for any proper subset $S$ in the vertex set there are three vertices $x,y,z$ such that $\dist{x}{y}{} +\dist{y}{z}{}=\dist{x}{z}{}$ and either 
$x,z\in S$ and $y\notin S$, or $x,z\notin S$ and $y\in S$.
Then apply \ref{ex:cut}.

\parbf{\ref{ex:RP-not}.}
For the first part, show and use that the quotient of $\RP^2$ by the isotropy group of one point is isometric to a line segment.

For the second part, choose three points on a closed geodesic at equal distances from each other.
Show and use that there is an isometric three-point set in $\RP^2$ that does not lie on a closed geodesic.

\parit{Source:} \cite[V \S 2]{busemann-1942}.

\parbf{\ref{ex:hom-cube}.}
Denote by $\dim(x_1,\dots,x_m)$ the dimension of the minimal face of the cube that contains all the points $x_1,\dots,x_m\in Q$.
Show and use that 
\[\dim(x_1,\dots,x_m)=\dim(x_1',\dots,x_m')\]
for any isometry $x\mapsto x'$ of $Q$.

\parit{Source:} \cite[prop. 6 and 7]{berestovskii-nikonorov}.

\parbf{\ref{ex:conv-short};} \textit{only-if part}.
To check convexity, assume that $B$ is a two-point subset.
For closeness, assume that $B$ is a countable set of $A$.

\parit{If part.}
Learn about the Kirszbraun theorem and apply it together with the closest-point projection.

\refstepcounter{chapter}
\setcounter{eqtn}{0}


\parbf{\ref{ex:inj=complete-geodesic-contractible}.}
Choose an injective space $\spc{Y}$.

\textit{\ref{SHORT.ex:inj=complete-geodesic-contractible:complete}.}
Fix a Cauchy sequence $x_n$ in $\spc{Y}$;
we need to show that it has a limit $x_\infty\in \spc{Y}$.
Consider metric on $\spc{X}=\NN\cup\{\infty\}$ defined by 
\begin{align*}
\dist{m}{n}{\spc{X}}&\df\dist{x_m}{x_n}{\spc{Y}},
\\
\dist{m}{\infty}{\spc{X}}&\df\lim_{n\to\infty}\dist{x_m}{x_n}{\spc{Y}}.
\end{align*}
Since the sequence is Cauchy, so is the sequence $\ell_n=\dist{x_m}{x_n}{\spc{Y}}$ for any $m$.
Therefore, the last limit is defined.

By construction, the map $n\mapsto x_n$ is distance-preserving on $\NN\subset \spc{X}$.
Since $\spc{Y}$ is injective, this map can be extended to $\infty$ as a short map; set $\infty\mapsto x_\infty$.
Since $\dist{x_n}{x_\infty}{\spc{Y}}\le \dist{n}{\infty}{\spc{X}}$ 
and $\dist{n}{\infty}{\spc{X}}\to 0$, we get that
$x_n\to x_\infty$ as $n\to\infty$.

\textit{\ref{SHORT.ex:inj=complete-geodesic-contractible:geodesic}.}
Applying the definition of injective space, we get a midpoint for any pair of points in $\spc{Y}$.
By \ref{SHORT.ex:inj=complete-geodesic-contractible:complete},
$\spc{Y}$ is a complete space.
It remains to apply \ref{lem:mid>geod:geod}.

\textit{\ref{SHORT.ex:inj=complete-geodesic-contractible:contractible}.}
Let $k\:\spc{Y}\hookrightarrow \ell^\infty(\spc{Y})$ be the Kuratowski embedding (\ref{lem:kuratowski}).
Observe that $\ell^\infty(\spc{Y})$ is contractible;
in particular, there is a homotopy $k_t\:\spc{Y}\hookrightarrow \ell^\infty(\spc{Y})$ such that $k_0=k$ and $k_1$ is a constant map.
(In fact, one can take $k_t=(1-t)\cdot k$.)

Since $k$ is distance-preserving and $\spc{Y}$ is injective,
there is a short map $f\:\ell^\infty(\spc{Y})\to \spc{Y}$ such that the composition $f\circ k$ is the identity map on $\spc{Y}$.
The composition $f\circ k_t\:\spc{Y}\hookrightarrow \spc{Y}$ provides the needed homotopy. 

\parbf{\ref{ex:bicombing}.}
By \ref{lem:kuratowski}, the space $\spc{Y}$ can be considered as a subset in $\ell^\infty(\spc{Y})$.
Given $x,y\in \spc{Y}$, let $\tilde\gamma_{x,y}(t)=(1-t)\cdot x+t\cdot y\in \ell^\infty(\spc{Y})$.
Observe that $\tilde\gamma_{x,y}$ meets all the conditions.
Apply the definition of injective space to $\ell^\infty(\spc{Y})$.

\parit{Remark.} The choice of geodesic paths as in the exercise is called \index{geodesic bicombing}\emph{geodesic bicombing}; it was introduced by Urs Lang \cite[3.6]{lang-2013}.

\parbf{\ref{ex:injective-spaces}.}
Suppose that a short map $f\:A\to\spc{Y}$ is defined on a subset $A$ of a metric space $\spc{X}$.
We need to construct a short extension $F$ of $f$.
Without loss of generality, we may assume that $A\ne\emptyset$;
otherwise, map the whole $\spc{X}$ to a single point.
By Zorn's lemma, it is sufficient to enlarge $A$ by a single point $x\notin A$.

\parit{\ref{SHORT.ex:injective-spaces:R}.}
Suppose $\spc{Y}=\RR$.
Set 
\[F(x)=\inf\set{f(a)-\dist{a}{x}{}}{a\in A}.\] 
Observe that $F$ is short and $F(a)=f(a)$ for any $a\in A$.

\parit{\ref{SHORT.ex:injective-spaces:tree}.}
Suppose  $\spc{Y}$ is a complete metric tree.
Fix points $p\in \spc{X}$ and $q\in\spc{Y}$.
Given a point $a\in A$,
let $x_a\in\cBall[f(a),\dist{a}{p}{}]$ be the point closest to $f(x)$.
Note that $x_a\in[q\,f(a)]$ and either $x_a=q$ or $x_a$ lies on distance $\dist{a}{p}{}$ from $f(a)$.

Note that the geodesics $[q\,x_a]$ are nested;
that is, for any $a,b\in A$ we have either $[q\,x_a]\z\subset [q\,x_b]$ or $[q\,x_b]\z\subset [q\,x_a]$.
Moreover, in the first case, we have $\dist{x_b}{f(a)}{}\le \dist{p}{a}{}$ and in the second $\dist{x_a}{f(b)}{}\le \dist{p}{b}{}$.

It follows that the closure of the union of all geodesics $[q\,x_a]$ for $a\in\spc{A}$ is a geodesic.
Denote by $x$ its endpoint; it exists since $\spc{Y}$ is complete.
It remains to observe that $\dist{x}{f(a)}{}\le \dist{p}{a}{}$ for any $a\in\spc{A}$;
that is, one can take $f(p)=x$.

\parit{\ref{SHORT.ex:injective-spaces:ell-infty}.}
Suppose $\spc{Y}=(\RR^2,\ell^\infty)$.
Note that $\spc{X}\z\to (\RR^2,\ell^\infty)$ is a short map if and only if both of its coordinate projections are short.
It remains to apply \ref{SHORT.ex:injective-spaces:R}.
The general case of $\ell^\infty(\spc{S})$ can be done the same way.

More generally, \textit{any $\ell^\infty$-product of injective spaces is injective};
in particular, if $\spc{Y}$ and $\spc{Z}$ are injective then the product $\spc{Y}\times\spc{Z}$ equipped with the metric 
\[\dist{(y,z)}{(y',z')}{\spc{Y}\times\spc{Z}}=\max\{\,\dist{y}{y'}{\spc{Y}},\dist{z}{z'}{\spc{Z}}\,\}\]
is injective as well.

\parbf{\ref{ex:extr-ball}}; \ref{SHORT.ex:extr-ball:one}.
Let $\spc{B}=\cBall[o,R]_{\spc{Y}}$.
Choose a metric space $\spc{X}$ with a subset $A$.
Given a short map $f\:A\to \spc{B}$ we need to find its short extension $\spc{X}\to \spc{B}$.

Since $\diam\spc{B}\le 2\cdot R$, we may assume that  $\diam \spc{X}\le 2\cdot R$;
if not pass to the metric defined by $\dist{x}{y}{}=\max\{\,\dist{x}{y}{\spc{X}},2\cdot R\,\}$.

Let us add point $w$ to $\spc{X}$ such that $\dist{w}{x}{}=R$ for any $x\in\spc{X}$;
denote the obtained space $\spc{X}'$.
Let $f'\:A\cup\{w\}\to \spc{B}$ be an extension of $f$ by $w\mapsto o$; note that $f'$ is short.

Since $\spc{Y}$ is injective, there is a short extension $F\:\spc{X}'\to \spc{Y}$ of $f'$.
Show and use that $F(\spc{X}')\subset \spc{B}$.

\parit{\ref{SHORT.ex:extr-ball:many}.}
Let $\spc{B}=\cap_\alpha\cBall[o_\alpha,R_\alpha]_{\spc{Y}}$.
Try to modify the argument in \ref{SHORT.ex:extr-ball:one}.

(Note that one may assume that $\diam \spc{X}\z\le 2\cdot \inf_\alpha\{\,R_\alpha\,\}$.
Consider the space $\spc{X}'\z=\spc{X}\cup\{w_\alpha\}$ such that $\dist{w_\alpha}{x}{}=R_\alpha$ for any $x\in \spc{X}$ and $\dist{w_\alpha}{w_\beta}{}=R_\alpha+R_\beta$ if $\alpha\ne\beta$.
Further, consider an extension of $f$ by $w_\alpha\mapsto o_\alpha$.)

\parbf{\ref{ex:extr-fixed}.}
Let $\diam \spc{Y}=2\cdot R$.
We can assume that $R>0$; otherwise there is nothing to prove.
Denote by $\spc{Z}$ a minimal (with respect to inclusion) intersection of closed $R$-balls in $\spc{Y}$ such that $s(\spc{Z})\subset\spc{Z}$.

Consider 
the intersection 
\[\spc{Y}'=\spc{Z}\cap\left(\bigcap_{p\in \spc{Z}} \cBall[p,R]_{\spc{Y}}\right).\]
By \ref{ex:extr-ball:many}, $\spc{Y}'$ is injective.
Use that $\spc{Z}$ is minimal to show that $s(\spc{Y}')\subset \spc{Y}'$.
Show that $\diam \spc{Y}'\le \tfrac12\cdot\diam \spc{Y}$.

Consider a sequence of nested injective spaces $\spc{Y}=\spc{Y}_0\supset \spc{Y}_1\supset\dots$ such that $\spc{Y}_{n+1}\z=\spc{Y}_{n}'$.
Choose a point $y_n\in \spc{Y}_{n}$ for each $n$.
Show that the sequence $y_n$ is Cauchy.
By \ref{ex:inj=complete-geodesic-contractible:complete}, $y_n$ converges, say to $y_\infty$.
Observe that $y_\infty$ is a fixed point of $s$.

\parbf{\ref{ex:circle};} \textit{only-if part}.
Suppose $r$ is extremal.
By \ref{lem:extremal-lipschitz}, $r$ is $1$-Lipschitz.
Since $\mathbb{S}^1$ is compact, \ref{lem:opposite-compact} implies that for any $p\in \mathbb{S}^1$ there is $q\in \mathbb{S}^1$ such that 
\[r(p)+r (q) = \dist{p}{q}{\mathbb{S}^1}.\]
Therefore
\begin{align*}
\pi&=\dist{p}{(-p)}{\mathbb{S}^1}\le 
\\
&\le 
r(p)+r(-p)=
\\
&=
r(p)+r(q) +r(-p) -r(q)\le
\\
&\le
\dist{p}{q}{\mathbb{S}^1}+\dist{q}{(-p)}{\mathbb{S}^1}=
\\
&=\pi.
\end{align*}
So, we have equality in both places, and the only-if part follows.

\parit{If part.}
Assume $r$ is a 1-Lipschitz function such that $r(p)+r(-p)=\pi$.
Then 
\begin{align*}
\dist pq{\mathbb{S}^1}&=
\dist{p}{(-p)}{\mathbb{S}^1}-\dist{q}{(-p)}{\mathbb{S}^1}\ge
\\
&\ge\pi -(r(-p)-r(q))=
\\
&=r(p)+r(q).
\end{align*}
Therefore $r$ is admissible.

Finally, if $r$ is not extremal, then there is an admissible function $s\le r$ such that $s(p)<r(p)$ for some $p$.
The latter contradicts the equality $r(p)+r(-p)=\pi$.

\parit{Source:} \cite[Proposition 2.7]{zuest}.

\parbf{\ref{ex:retraction}.}
Show and use that
$s^*(x)+s(y)\ge \dist{x}{y}{}$
for any $x,y\in \spc{X}$.

\parit{Remarks.}
It is easy to check that $q\:s\z\mapsto \tfrac12\cdot(s+s^*)$ is a short map on the space of admissible functions (with sup-norm).
Moreover, iterating $q$ and passing to the limit, we get a short retraction from the space of admissible functions to the space of extremal functions on $\spc{X}$ \cite[see 3.1 in][]{lang-2013}.
The existence of such a map will also follow from \ref{thm:inj-envelope}.

\parbf{\ref{ex:one-point-gluing}.}
Apply \ref{thm:injective=hyperconvex:balls}.

\parit{Comment.}
Conditions under which gluings of injective spaces is injective were studied by Benjamin Miesch and Maël Pavón \cite{miesch,miesch-pavon}.

\parbf{\ref{ex:Rm-ell-infty}.}
Let $B=\cBall[0,1]$ and $P\supset B$ be a parallelepiped of minimal volume.
Choose the basis $e_1,\dots,e_m$ parallel to the edges of $P$ so that in the corresponding coordinates the parallelepiped is described by inequalities
$|x_i|\le 1$ for all $i$.

Let $B_i=\cBall[(1+R)\cdot e_i,R]$ for some $R>0$.
Show that $ e_i\in B$ for any $i$; in particular $B\cap B_i\ne\emptyset$.
Show $P$ can be chosen so that $B_i\cap B_j\ne \emptyset$ for all $i$ and~$j$ and all large $R>0$.
Apply hyperconvexity to show that $e_1+\dots+ e_m\in B$.
The same way, show that $\pm e_1\pm \dots\pm e_m\in B$ for all choices of signs.
Conclude that $B=P$.

\parbf{\ref{ex:compact-hyperconvex}.}
Observe that closed balls are compact and
apply the finite intersection property.

\parbf{\ref{ex:urysohn-hyperconvex}.}
Denote by $\spc{U}_d$ the $d$-Urysohn space,
so $\spc{U}_\infty$ is the Urysohn space.

The extension property implies finite hyperconvexity.
It remains to show that $\spc{U}_d$ is not countably hyperconvex.

Suppose that $d<\infty$.
Then $\diam\spc{U}_d=d$ and for any point $x\in\spc{U}_d$ there is a point $y\in\spc{U}_d$ such that $\dist{x}{y}{\spc{U}_d}=d$.
It follows that there is no point $z\in\spc{U}_d$ such that $\dist{z}{x}{\spc{U}_d}\le \tfrac d2$ for any $x\in\spc{U}_d$.
Whence $\spc{U}_d$ is not countably hyperconvex.

Use \ref{ex:sphere-in-urysohn:midpoint} to reduce the case $d=\infty$ to the case $d<\infty$.

\parbf{\ref{ex:almost-hyperconvex}.}
Let $p_0$ be a point provided by the definition of almost hyperconvexity;
that is $\dist{x_\alpha}{p_0}{}\le r_\alpha+\eps_0$ for a given $\eps_0>0$.
We may assume that $\delta_0=\sup\{\,\dist{x_\alpha}{p_0}{}- r_\alpha\,\}>0$; otherwise the problem is solved.
Clearly, $\delta_0\le \eps_0$.

Let $p_1$ be a point provided by the definition for $\eps_1<\tfrac1{10}\cdot\delta_0$ we get a point 
$p_1$ such that $\dist{x_\alpha}{p_1}{}\le r_\alpha+\eps_1$ and $\dist{p_0}{p_1}{}\le \delta_0+\eps_1$.
Again, we may assume that $\delta_1=\sup\{\,\dist{x_\alpha}{p_1}{}- r_\alpha,\dist{p_0}{p_1}{}\,\}>0$, and we have $\delta_1\le \eps_1$.

Continuing this way, we get a sequence $p_0,p_1,\dots$ that either terminates and in this case the problem is solved, or it is an infinte Cauchy sequence.
In the latter case, its limit $p_\infty$ satisfies $\dist{x_\alpha}{p_\infty}{}\le r_\alpha$ for any $\alpha$.

\parit{Comment.}
This solution reminds the proof of \ref{prop:completion-univeral};
a more exact statement was proved by Benjamin Miesch and Maël Pavón \cite[2.2]{miesch-pavon2016};
namely they show that almost $n$-hyperconvexity implies $(n-1)$-hyperconvexity.


\parbf{\ref{ex:Inj(compact)}.}
Observe and use that the functions in $\Inj\spc{X}$ are 1-Lipschitz and uniformly bounded.

\parbf{\ref{ex:tripod+square}}; \ref{SHORT.ex:tripod+square:2}.
Use \ref{lem:opposite-compact} to show that if $f$ is extremal if and only if $f(v)=x$ and $f(w)=1-x$ for some $x\in [0,1]$.
Conclude that $\Inj\spc{X}$ is isometric to the unit interval $[0,1]$.

\parit{\ref{SHORT.ex:tripod+square:tripod}.}
Let $f$ be an extremal function.
By \ref{lem:opposite-compact}, at least two of the numbers $f(a)+f(b)$, $f(b)+f(c)$, and $f(c)+f(a)$ are $1$.
It follows that for some $x\in[0,\tfrac12]$, we have 
\begin{align*}
f(a)&=1\pm x,&
f(b)&=1\pm x,&
f(c)&=1\pm x,
\end{align*}
where we have one ``minus'' and two ``pluses'' in these three formulas.

Suppose that
\begin{align*}
g(a)&=1\pm y,& g(b)&=1\pm y,& g(c)&=1\pm y
\end{align*}
is another extremal function.
Then $|f-g|\z=|x-y|$ if $g$ has ``minus'' at the same place as $f$ and $|f-g|=|x+y|$ otherwise.

It follows that $\Inj\spc{X}$ is isometric to a {}\emph{tripod} --- three segments of length $\tfrac12$ glued at one end.

\begin{Figure}
\begin{minipage}{.48\textwidth}
\centering
\includegraphics{mppics/pic-3}
\end{minipage}\hfill
\begin{minipage}{.48\textwidth}
\centering
\includegraphics{mppics/pic-4}
\end{minipage}
\vskip-4mm
\end{Figure}

\parit{\ref{SHORT.ex:tripod+square:square}.}
Assume $f$ is an extremal function.
Use \ref{lem:opposite-compact} to show that
\begin{align*}
2&=f(x)+f(y)=
\\
&=f(p)+f(q);
\end{align*}
in particular, two values $a=f(x)-1$ and $b\z=f(p)-1$ completely describe the function $f$.
Since $f$ is extremal, we also have that 
\[(1\pm a)+(1\pm b)\ge 1\]
for all 4 choices of signs;
equivalently, 
\[|a|+|b|\le 1.\]

It follows that $\Inj\spc{X}$ is isometric to the rhombus $|a|+|b|\le 1$ in the $(a,b)$-plane with the metric induced by the $\ell^\infty$-norm.

\parit{Remarks.}
If $\spc{X}$ has $n$-points, then (evidently) $\Inj\spc{X}$ is a polyhedral complex in $(\RR^n,\ell^\infty) \z=\ell^\infty(\spc{X})$;
each face of the complex is defined by equalities and inequalities of the following type: $x_i+x_j\ge \const$ and  $x_i+x_j= \const$.
It is easy to see (and follows from \ref{ex:Rm-ell-infty}) that each face is isometric to a convex polyhedron in  $(\RR^k,\ell^\infty)$ for some $k\le n$;
in fact $k\le n/2$.
The structure of the complex can be encoded by certain graphs with the vertex set $\spc{X}$ \cite[see Section 4 in][]{lang-2013}.

\parbf{\ref{ex:kur-inj}.}
Recall that $x\mapsto \distfun_x$ gives an isometric embedding $\spc{X}\z\hookrightarrow\ell^\infty(\spc{X})$;
so we can identify $\spc{X}$ with a subset of $\ell^\infty(\spc{X})$.
Further, $\Inj\spc{X}$ is a subset of $\ell^\infty(\spc{X})$.
It is sufficient to show that $\Inj\spc{X}=G$.

Use \ref{lem:opposite-compact} to show that $\Inj\spc{X}\subset G$.

Given $g\in G$, show that $g(x)=\dist{g}{x}{\ell^\infty(\spc{X})}$.
Conclude that $g$ is admissible and apply \ref{lem:opposite-compact}.

\parit{Source:} Private communications with Rostislav Matveyev.

\parbf{\ref{ex:4-on-a-line}.}
Recall that 
\[\dist{f}{g}{\Inj\spc{X}}=\sup\set{|f(x)-g(x)|}{x\in\spc{X}}\]
and 
\[\dist{f}{p}{\Inj\spc{X}}=f(p)\]
for any $f,g\in \Inj\spc{X}$ and $p\in \spc{X}$.

Since $\spc{X}$ is compact we can find a point $p\in\spc{X}$ such that 
\begin{align*}
\dist{f}{g}{\Inj\spc{X}}&=|f(p)-g(p)|=
\\
&=\left|\dist{f}{p}{\Inj\spc{X}}-\dist{g}{p}{\Inj\spc{X}}\right|.
\end{align*}
Without loss of generality, we may assume that 
\[\dist{f}{p}{\Inj\spc{X}}
=
\dist{g}{p}{\Inj\spc{X}}
+
\dist{f}{g}{\Inj\spc{X}}.\]
Applying \ref{lem:opposite-compact}, we can find a point $q\in\spc{X}$ such that 
\[\dist{q}{p}{\Inj\spc{X}}
=
\dist{f}{p}{\Inj\spc{X}}
+
\dist{f}{q}{\Inj\spc{X}},\]
whence the result.

Since $\Inj\spc{X}$ is injective (\ref{prop:InjX-is-injective}), by \ref{ex:inj=complete-geodesic-contractible:geodesic} it has to be geodesic. It remains to note that the concatenation of geodesics $[pq]$, $[gf]$, and $[fq]$ is a required geodesic $[pq]$.

\parbf{\ref{ex:delta-hyp}.} The only-if part follows since $\spc{X}$ is isometric to a subset of $\Inj\spc{X}$.

The if part means that 
\[
\begin{aligned}
\dist{f}{g}{}+\dist{v}{w}{}\le
\max\{\,
&\dist{f}{v}{}+\dist{g}{w}{},\\
\dist{f}{w}{}+&\dist{g}{v}{}
\,\}+2\cdot\delta
\end{aligned}
\eqlbl{eq:fgvw-hyp}\]
for any $f,g,v,w\in \Inj\spc{X}$.

Suppose $\spc{X}$ is compact. 
Applying \ref{ex:4-on-a-line}, we can choose $p,q,x,y\in \spc{X}$  such that 
\[
\begin{aligned}
\dist{p}{f}{}+\dist{f}{g}{}+\dist{g}{q}{}&=\dist{p}{q}{}
\\
\dist{x}{v}{}+\dist{v}{w}{}+\dist{w}{y}{}&=\dist{x}{y}{}
\end{aligned}
\eqlbl{eq:pfgq+xvwy}
\]

Since $\spc{X}$ is $\delta$-hyperbolic, we have
\[\begin{aligned}
\dist{p}{q}{}+\dist{x}{y}{}\le
\max\{\,&\dist{p}{x}{}+\dist{q}{y}{},
\\
\dist{p}{y}{}+&\dist{q}{x}{}\,\}+2\cdot\delta.
\end{aligned}\]
Show that this inequality, together with the triangle inequality and \ref{eq:pfgq+xvwy} imply \ref{eq:fgvw-hyp}.

For the noncompact case, prove an approximate version of \ref{eq:pfgq+xvwy} and apply it the same way.

\parbf{\ref{ex:inj-envelope}.}
Show that there is unique isometry of $\Inj\spc{X}$ that is indentity of $\spc{X}$.
Use it together with \ref{thm:inj-envelope}.


\parbf{\ref{ex:d-p-inclusion}.}
Show that there is a pair of short maps 
$\Inj\spc{X}\to\Inj\spc{U}\to\Inj\spc{X}$ 
such that their composition is the identity on $\spc{X}$.
Make a conclusion.

\parbf{\ref{ex:hemisphere-inj}.}
Apply \ref{lem:opposite-compact} to show that for any $u\in\mathbb{S}^2_+$ the restriction $f_u\z\df\distfun_u|_{\mathbb{S}^1}$ is extremal function on $\mathbb{S}^1$.
Moreover, the function $f_u$ uniquely determines $u$. 
Make a conclusion.

\parbf{\ref{ex:3-4-hypreconvex}.}
Observe that coordinate functions are monotonic on any geodesic in $\ell^1$.
Use it to show that $\ell^1$ is a \emph{median space};
that is, for any three points $x,y,z$ there is a {}\emph{unique} point $m$ (it is called \index{median}\emph{median} of $x$, $y$, and $z$) that lies on {}\emph{some} geodesics $[xy]$, $[xz]$ and $[yz]$.
Apply it to show that $\ell^1$ is 3-hyperconvex.

The 4-hyperconvexity fails for the unit balls centered at four even vertices of the cube $([0,1]^3,\ell^1)$.


\parbf{\ref{ex:ultrametric}.}
Choose three points $x,y,z\in\spc{X}$ and set $\spc{A}=\{x,z\}$.
Let $f\:\spc{A}\z\to \spc{A}$ be the identity map.
Then $F(y)=x$ or $F(y)=z$.
The strong triangle inequality easily follows in both cases.

\parbf{\ref{ex:ultrametric-converse}}; \textit{main part.}
Choose a maximal subset $A\z\supset K$ that admits a short retraction $f\:A\to K$;
it exists by Zorn's lemma.
If $A$ is the whole space, then the problem is solved.
Otherwise, choose $p\notin A$.

Choose a sequence of points $a_n\in A$ such that $\dist{a_n}{p}{}$ converge to the exact lower bound on the distances from points in $A$ to $p$.
Since $K$ is compact, we can pass to a subsequence of $a_n$ such that $f(a_n)$ converges.
Let 
\[f(p)=\lim f(a_n).\]

It remains to check that 
\[\dist{f(a)}{f(p)}{}\le\dist{a}{p}{}\eqlbl{eq:short-retract}\]
for any $a\in A$.
Choose $\eps>0$; note that 
\begin{align*}
\dist{a_n}{p}{}&<\dist{a}{p}{}+\eps
\intertext{and}
\dist{f(a_n)}{f(p)}{}&<\dist{f(a)}{f(a_n)}{}+\eps
\end{align*}
for all large~$n$.
Therefore, 
\begin{align*}
\dist{f(a)}{f(p)}{}&\le \max\{\,\dist{f(a)}{f(a_n)}{},
\\
&\qquad\dist{f(a_n)}{f(p)}{}\,\}\le
\\
&\le \dist{f(a)}{f(a_n)}{}+\eps\le
\\
&\le \dist{a}{a_n}{} +\eps\le 
\\
&\le \max\{\,\dist{a}{p}{},\dist{a_n}{p}{}\,\}+\eps< 
\\
&< \dist{a}{p}{}+2\cdot\eps.
\end{align*}
Since $\eps>0$ is arbitrary, we get \ref{eq:short-retract}.

\parit{Example.}
Consider set of $\{\infty,1,2,\dots\}$ with metric defined by 
\[|m-n|=1+\frac1{\min\{m,n\}}\]
for $m\ne n$.
Observe that the space is complete, the subset $\{1,2,\dots\}$ is closed, but it is not a short retract of the ambient space.

\parbf{\ref{ex:petrunin-stadler}.} Consider the space $\spc{Y}^{\spc{X}}$ of all maps $\spc{X}\z\to \spc{Y}$ equipped with the product topology.

Denote by $\mathfrak{S}_F$ the set of maps $h\in \spc{Y}^\spc{X}$ such that the restriction $h|_F$  is short and agrees with $f$ in $F\cap A$.
Note that the sets $\mathfrak{S}_F\subset \spc{Y}^\spc{X}$ are closed and any finite intersection of these sets is nonempty.

According to Tikhonov's theorem, $\spc{Y}^{\spc{X}}$ is compact.
By the finite intersection property, the intersection $\bigcap_F\mathfrak{S}_F$ for all finite sets $F\subset X$ is nonempty.
Hence the statement follows.

\parit{Source:} \cite{petrunin-stadler}.

\parbf{\ref{ex:diam}.}
Suppose that $\dist{A}{B}{\Haus\spc{X}}<r$.
Choose a pair of points $a,a'\in A$ on maximal distance from each other.
Observe that there are points $b,b'\in B$ such that 
$\dist{a}{b}{\spc{X}},\dist{a'}{b'}{\spc{X}}<r$.
Whence 
\[\dist{a}{a'}{\spc{X}}-\dist{b}{b'}{\spc{X}}\le 2\cdot r\]
and therefore
\[\diam A-\diam B\le 2\cdot\dist{A}{B}{\Haus\spc{X}}.\]

It remains to swap $A$ and $B$ and repeat the argument.


\parbf{\ref{ex:Hausdorff-bry}}; \ref{SHORT.ex:Hausdorff-bry:conv}.
Denote by $A^r$ the closed $r$-neighborhood of a set $A\z\subset \RR^2$.
Observe  that 
\[(\Conv A)^r=\Conv(A^r),\]
and try to use it.

\parit{\ref{SHORT.ex:Hausdorff-bry:bry}.}
The answer is ``no'' in both parts.

For the first part let $A$ be a unit disk and $B$ a finite $\eps$-net in $A$.
Evidently, $|A-B|_{\Haus\RR^2}<\eps$, 
but
$|\partial A-\partial B|_{\Haus\RR^2}\approx 1$.

For the second part take $A$ to be a unit disk and $B=\partial A$ to be its boundary circle.
Note that $\partial A=\partial B$; in particular, $\dist{\partial A}{\partial B}{\Haus\RR^2}=0$ while $\dist{ A}{B}{\Haus\RR^2}=1$.

\parit{Remark.}
There is the so-called {}\emph{lakes of Wada} --- an example of three (and more) open bounded topological disks in the plane that have identical boundaries.
It can be used to construct more interesting examples for \ref{SHORT.ex:Hausdorff-bry:bry}.

\parbf{\ref{ex:closure-union}.} 
Show that for any $\eps>0$ there is a positive integer $N$ such that $\bigcup_{n\le N} K_n$ is an $\eps$-net in the union $\bigcup_{n} K_n$.
Observe that $\bigcup_{n\le N} K_n$
is compact and apply \ref{ex:compact-net}.

\parbf{\ref{ex:Haus-length}}; \textit{``if'' part.}
Choose two compact sets $A,B\subset \spc{X}$;
suppose that $\dist{A}{B}{\Haus\spc{X}}<r$.

Choose finite $\eps$-nets $\{a_1,\dots a_m\}\subset A$ and $\{b_1,\dots b_n\}\subset B$.
For each pair $a_i,b_j$ construct a constant-speed path $\gamma_{i,j}$ from $a_i$ to $b_j$ such that 
\[\length \gamma_{i,j}<\dist{a_i}{b_j}{}+\eps.\]
Set 
\[C(t)=\set{\gamma_{i,j}(t)}{\dist{a_i}{b_j}{\spc{X}}<r+\eps}.\]
Observe that $C(t)$ is finite; in particular, it is compact.

Show and use that 
\begin{align*}
\dist{A}{C(t)}{\spc{X}}&<t\cdot r+10\cdot\eps,
\\
\dist{C(t)}{B}{\spc{X}}&<(1-t)\cdot r+10\cdot\eps.
\end{align*}
Apply \ref{ex:closure-union} and \ref{lem:mid>geod}.

\parit{``only-if'' part.}
Choose points $p,q\in\spc{X}$. 
Show that the existence of $\eps$-midpoints between $\{p\}$ and $\{q\}$ in $\Haus\spc{X}$ implies the existence of $\eps$-midpoints between $p$ and $q$ in $\spc{X}$.
Apply \ref{lem:mid>geod}.


\parbf{\ref{ex:Huas-perimeter-area}.}
Let $A$ be a compact convex set in the plane.
Denote by $A^r$ the closed $r$-neighborhood of $A$.
Recall that by Steiner's formula we have
\[\area A^r=\area A+r\cdot\perim A+\pi\cdot r^2.\]
Taking derivative and applying the coarea formula, we get 
\[\perim A^r=\perim A+2\cdot\pi\cdot r.\]

Observe that if $A$ lies in a compact set $B$ bounded by a closed curve, then 
\[\perim A\le \perim B.\]
Indeed the closest-point projection $\RR^2\to A$ is short and it maps $\partial B$ onto $\partial A$.

It remains to use the following observation: if $A_n\to A_\infty$, then for any $r>0$ we have that
\[A_\infty^r\supset A_n
\quad\text{and}\quad
A_\infty\subset A_n^r\]
for all large $n$.

\begin{wrapfigure}{r}{27 mm}
\vskip-6mm
\centering
\includegraphics{mppics/pic-410}
\end{wrapfigure}

\parbf{\ref{ex:round-disc}.}
Note that almost all points on $\partial D$ have a defined tangent line.
In particular, for almost all pairs of points $a,b\z\in\partial D$ the two angles $\alpha$ and $\beta$ between the chord $[ab]$ and $ \partial D$ are defined.

The convexity of $D'$ implies that $\alpha=\beta$;
here we measure the angles $\alpha$ and $\beta$ on one side from $[ab]$.
Show that if the identity $\alpha=\beta$ holds for almost all chords, then $D$ is a round disk. 


\parbf{\ref{ex:generalized-selection}.}
Observe that all functions $\distfun_{A_n}$ are Lipschitz and uniformly bounded on compact sets.
Therefore, passing to a subsequence, we may assume that the sequence $\distfun_{A_n}$ converges to some function $f$.

Set $A_\infty=f^{-1}\{0\}$.
It remains to show that $f=\distfun_{A_\infty}$.

%%%%%%%%%%%%%%%%%%%%%%%%%%%%%%

\parbf{\ref{ex:d_GH-and-diam}};
\ref{SHORT.ex:d_GH-and-diam:point}.
Apply the definition for space $\spc{Z}$ obtained from $\spc{X}$ by adding one point on distance $\tfrac12\cdot\diam \spc{X}$ to each point of $\spc{X}$.

\parit{\ref{SHORT.ex:d_GH-and-diam:scale}.}
Given a point $x\in\spc{X}$, denote by $a\cdot x$ and $b\cdot x$ the corresponding points in $a\cdot\spc{X}$ and $b\cdot \spc{X}$ respectively.
Show that there is a metric on $\spc{Z}\z=a\cdot\spc{X}\sqcup b\cdot\spc{X}$ such that 
\[|a\cdot x-b\cdot x|_{\spc{Z}}=\tfrac{|b-a|}2\cdot\diam\spc{X}\]
for any $x$ and the inclusions
$a\cdot\spc{X}\hookrightarrow\spc{Z}$,
$b\cdot\spc{X}\hookrightarrow\spc{Z}$ are distance preserving.

\parbf{\ref{ex:rectangle}.}
Arguing by contradiction,
we can identify $\spc{A}_r$ and $\spc{B}_r$ with subspaces of a space $\spc{Z}$
such that 
\[|\spc{A}_r-\spc{B}_r|_{\Haus \spc{Z}}<\tfrac1{10}\]
for large $r$; see the definition of Gromov--Hausdorff metric (\ref{def:GH}).

Set $n=\lceil r \rceil$.
Note that there are $2\cdot n$ integer points in~$\spc{A}_r$: 
$a_1\z=(0,0)$, $a_2=(1,0),\dots,a_{2\cdot n}=(n,1)$.
Choose a point $b_i\in \spc{B}_r$ that lies at the minimal distance from $a_i$.
Note that $|b_i-b_j|>\tfrac 45$ if $i\ne j$.
It follows that $r>\tfrac 45\cdot (2\cdot n-1)$.
The latter contradicts $n=\lceil r \rceil$ for large~$r$.

\parit{Remark.}
Try to show that $|\spc{A}_r-\spc{B}_r|_{\GH}=\tfrac12$ for all large $r$.

\parbf{\ref{ex:GH-inj}.}
Suppose that $|\spc{X}-\spc{Y}|_{\spc{U}}<\eps$;
we need to show that 
\[|\hat{\spc{X}}-\hat{\spc{Y}}|_{\GH}<2\cdot \eps.\]

Denote by $\hat{\spc{U}}$ the injective envelope of $\spc{U}$.
Recall that $\spc{U}$, $\spc{X}$, and $\spc{Y}$ can be considered as subspaces of $\hat{\spc{U}}$, $\hat{\spc{X}}$, and $\hat{\spc{Y}}$ respectively.

According to \ref{ex:d-p-inclusion}, the inclusions $\spc{X}\hookrightarrow\spc{U}$ and $\spc{Y}\hookrightarrow\spc{U}$ can be extended to a distance-preserving inclusions $\hat{\spc{X}}\hookrightarrow\hat{\spc{U}}$ and $\hat{\spc{Y}}\hookrightarrow\hat{\spc{U}}$.
Therefore, we can and will consider  $\hat{\spc{X}}$ and $\hat{\spc{Y}}$ as subspaces of $\hat{\spc{U}}$.

Given $f\in \hat{\spc{U}}$,
let us find $g\in \hat{\spc{X}}$ such that 
\[|f(u)-g(u)|<2\cdot\eps\eqlbl{|g-f|}\]
for any $u\in\spc{U}$.
Note that the restriction $f|_{\spc{X}}$ is admissible on ${\spc{X}}$.
By \ref{obs:extremal:below}, there is $g\in \hat{\spc{X}}$ such that 
\[g(x)\le f(x)\eqlbl{g(x)=<f(x)}\]
for any $x\in\spc{X}$.

Recall that any extremal function is $1$-Lipschitz;
in particular, $f$ and $g$ are $1$-Lipschitz on $\spc{U}$.
Therefore, \ref{g(x)=<f(x)} and $|\spc{X}-\spc{Y}|_{\spc{U}}<\eps$ imply that
\[g(u)< f(u)+2\cdot \eps\]
for any $u\in\spc{U}$.
By \ref{ex:+-c}, we also have 
\[g(u)> f(u)-2\cdot \eps\]
for any $u\in\spc{U}$.
Whence \ref{|g-f|} follows.

It follows that $\hat{\spc{Y}}$ lies in a $2\cdot\eps$-neighborhood of $\hat{\spc{X}}$ in $\hat{\spc{U}}$.
The same way we show that $\hat{\spc{X}}$ lies in a $2\cdot\eps$-neighborhood of $\hat{\spc{Y}}$ in $\hat{\spc{U}}$.
The latter means that
$|\hat{\spc{X}}-\hat{\spc{Y}}|_{\Haus\hat{\spc{U}}}<2\cdot\eps$,
and therefore
$|\hat{\spc{X}}-\hat{\spc{Y}}|_{\GH}<2\cdot\eps$.

\parit{Remark.} 
This problem was discussed by Urs Lang, Maël Pavón, and Roger Züst \cite[3.1]{lang-pavon-zust}.
\begin{figure}[h!]
\vskip-0mm
\centering
\includegraphics{mppics/pic-505}
\end{figure}
They also show that the constant 2 is optimal.
To see this, look at the injective envelopes of two 4-point metric spaces shown on the diagram and observe that the Gromov--Hausdorff distance between the 4-point metric spaces is 1, while the distance between their injective envelopes approaches 2 as $s\to\infty$. 

\parbf{\ref{ex:H-R}}; \textit{only-if part.}
Let us identify $\spc{X}$ and $\spc{Y}$ with subspaces of a metric space $\spc{Z}$ such that 
\[|\spc{X}-\spc{Y}|_{\Haus \spc{Z}}<\eps.\]

Set $x\approx y$ if and only if $\dist{x}{y}{\spc{Z}}<\eps$.
It remains to check that $\approx$ is an $\eps$-approximation.

\parit{If part.}
Show that we can assume that 
\[R=\set{(x,y)\in\spc{X}\times\spc{Y}}{x\approx y}\] is a compact subset of $\spc{X}\times\spc{Y}$.
Conclude that
\[\bigl|\dist{x}{x'}{\spc{X}}-\dist{y}{y'}{\spc{Y}}\bigr|<2\cdot\eps'\]
for some $\eps'<\eps$.

Show that there is a metric on $\spc{Z}=\spc{X}\sqcup\spc{Y}$ such that the inclusions $\spc{X}\hookrightarrow\spc{Z}$ and
$\spc{Y}\hookrightarrow\spc{Z}$ are distance preserving and $\dist{x}{y}{\spc{Z}}=\eps'$ if $x\approx y$.
Conclude that 
\[|\spc{X}-\spc{Y}|_{\Haus \spc{Z}}\le\eps'<\eps.\]

\parbf{\ref{ex:eps-isom}};
\ref{SHORT.ex:eps-isom:GH>isom}.
Let $\approx$ be an $\eps$-approximation provided by \ref{ex:H-R}.
For any $x\in\spc{X}$ choose a point $f(x)\in\spc{Y}$ such that $x\approx f(x)$.
Show that $x\mapsto f(x)$ is an $2\cdot\eps$-isometry.

\parit{\ref{SHORT.ex:eps-isom:isom>GH}.}
Let $x\in\spc{X}$ and $y\in\spc{Y}$.
Set $x\approx y$ if $\dist{y}{f(x)}{\spc{Y}}<\eps$.
Show that $\approx$ is a $\eps$-approximation. 
Apply \ref{ex:H-R}.

\parbf{\ref{ex:GH-SC}}; \ref{SHORT.ex:GH-SC:circle}.
Suppose $\spc{X}_n\GHto \spc{X}$ and $\spc{X}_n$ are simply connected length metric space.
It is sufficient to show that any nontrivial covering map $f\:\tilde{\spc{X}}\to \spc{X}$ corresponds to a nontrivial covering map $f_n\:\tilde{\spc{X}}_n\to \spc{X}_n$ for large $n$.

The latter can be constructed by covering $\spc{X}_n$ by small balls that lie close to sets in $\spc{X}$ evenly covered by $f$, prepare a few copies of these sets and glue them the same way as the inverse images of the evenly covered sets in $\spc{X}$ glued to obtain $\tilde{\spc{X}}$.

\begin{wrapfigure}{r}{40 mm}
\vskip-0mm
\centering
\includegraphics{mppics/pic-2}
\end{wrapfigure}

\parit{\ref{SHORT.ex:GH-SC:nonsc-limit}.}
Let $\spc{V}$ be a cone over Hawaiian earrings.
Consider the {}\emph{doubled cone} $\spc{W}$ --- two copies of $\spc{V}$ with glued their base points (see the diagram).

The space $\spc{W}$ can be equipped with a length metric
(for example, the induced length metric from the shown embedding).

Show that $\spc{V}$ is simply connected, but $\spc{W}$ is not; the latter is a good exercise in topology.

If we delete from the earrings all small circles, then the obtained double cone becomes simply connected and it remains to be close to $\spc{W}$.
That is $\spc{W}$ is a Gromov--Hausdorff limit of simply connected spaces.

\parit{Remark.}
Note that from part \ref{SHORT.ex:GH-SC:nonsc-limit}, the limit does not admit a nontrivial covering.
So, if we define the fundamental group as the inverse image of groups of deck transformations for all the coverings of the given space, then one may say that Gromov--Hausdorff limit of simply connected length spaces is simply connected.

\parbf{\ref{ex:sphere-to-ball},}
\textit{\ref{SHORT.ex:sphere-to-ball:2}.}
Suppose that a metric on $\mathbb{S}^2$ is close to the disk $\DD^2$.
Note that $\mathbb{S}^2$ contains a circle $\gamma$ that is close to the boundary curve of $\DD^2$.
By the Jordan curve theorem, $\gamma$ divides $\mathbb{S}^2$ into two disks, say $D_1$ and $D_2$.

By \ref{ex:GH-SC:nonsc-limit}, the Gromov--Hausdorff limits of $D_1$ and $D_2$ have to contain the whole $\DD^2$, otherwise the limit would admit a nontrivial covering.

Consider points $p_1\in D_1$ and $p_2\in D_2$ that are close to the center of $\DD^2$.
If $n$ is large, the distance $\dist{p_1}{p_2}{n}$ has to be very small.
On the other hand, any curve from $p_1$ to $p_2$ must cross $\gamma$, so it has length about 2 --- a contradiction.

\parbf{\ref{ex:utb+pack}.} Apply \ref{ex:pack-net}.

\parbf{\ref{pr:doubling}.}
Choose a space $\spc{X}$ in $\spc{Q}(C,D)$, denote a $C$-doubling measure by~$\mu$.
Without loss of generality, we may assume that $\mu(\spc{X})\z=1$.

The doubling condition implies that 
\[\mu[\oBall(p,\tfrac D{2^n})]\ge\tfrac 1{C^n}\]
for any point $x\in \spc{X}$.
It follows that 
\[\pack_{\frac D{2^n}}\spc{X}\le C^n.\]

By \ref{ex:pack-net}, for any $\eps\ge\frac D{2^{n-1}}$, the space $\spc{X}$ admits an $\eps$-net with at most $C^n$ points.
Whence $\spc{Q}(C,D)$ is uniformly totally bounded.

\parbf{\ref{pr:under}}; \ref{SHORT.pr:under:if}.
Choose $\eps>0$.
Since $\spc{Y}$ is compact, we can choose a finite $\eps$-net $\{y_1,\dots,y_{n}\}$ in $\spc{Y}$.

Suppose $f\:\spc{X}\to \spc{Y}$ be a distance-nondecreasing map.
Choose one point $x_i$ in each nonempty subset $B_i=f^{-1}[\oBall(y_i,\eps)]$.
Note that the subset $B_i$ has diameter at most $2\cdot \eps$ and 
\[\spc{X}=\bigcup_iB_i.\]
Therefore, the set of points $\{x_i\}$ is a $2\cdot\eps$-net in $\spc{X}$.

\parit{\ref{SHORT.pr:under:only-if}.} Let $\spc{Q}$ be a uniformly totally bounded family of spaces. 
Suppose that each space in $\spc{Q}$ has an $\tfrac1{2^n}$-net with at most $M_n$ points; we may assume that $M_0=1$.

Consider the space $\spc{Y}$ of all infinite integer sequences $m_0,m_1,\dots$ such that $1\le m_n\le M_n$ for any $n$.
Given two sequences $(\ell_n)$, and $(m_n)$ of points in $\spc{Y}$, set 
\[\dist{(\ell_n)}{(m_n)}{\spc{Y}}=\tfrac C{2^{n}},\]
where $n$ is the minimal index such that $\ell_n\ne m_n$ and $C$ is a positive constant.

Observe that $\spc{Y}$ is compact.
Indeed it is complete and the sequences with constant tails, starting from index $n$, form a finite $\tfrac C{2^{n}}$-net in $\spc{Y}$.

Given a space $\spc{X}$ in $\spc{Q}$,
choose a sequence of $\tfrac1{2^n}$ nets 
$N_n\subset\spc{X}$ for each $n$.
We can assume that $|N_n|\le M_n$; let us label the points in $N_n$ by $\{1,\dots,M_n\}$.
Consider the map $f:\spc{X}\to\spc{Y}$ defined by $f:x\to (m_1(x),m_2(x),\dots)$ where $m_n(x)$ is the label of a point in $N_n$ that lies on the distance $<\tfrac1{2^n}$ from $x$.

If $\tfrac1{2^{n-2}}\ge \dist{x}{x'}{\spc{X}}>\tfrac1{2^{n-1}}$, then $m_n(x)\ne m_n(x')$.
It follows that $\dist{f(x)}{f(x')}{\spc{Y}}\ge \tfrac C{2^{n}}$.
In particular, if $C>10$, then 
\[\dist{f(x)}{f(x')}{\spc{Y}}\ge \dist{x}{x'}{\spc{X}}\]
for any $x,x'\in \spc{X}$.
That is, $f$ is a distance-nondecreasing map $\spc{X}\to \spc{Y}$.

\parbf{\ref{ex-GH-length}}; \ref{SHORT.ex-GH-length:length}
Apply
\ref{ex:Haus-length},
\ref{prop:GH-with-fixed-Z},
\ref{lem:GH-complete},
and \ref{lem:mid>geod}.

\parit{\ref{SHORT.ex-GH-length:geodesic}.}
Choose two compact metric spaces $\spc{X}$ and $\spc{Y}$.
Show that there are subsets $\spc{X}'$, and $\spc{Y}'$ in the Urysohn space $\spc{U}$ that isometric to $\spc{X}$ and $\spc{Y}$ respectively and such that 
\[|\spc{X}-\spc{Y}|_{\GH} = |\spc{X}'-\spc{Y}'|_{\Haus\spc{U}}.\]

Further, construct a sequence of compact sets $\spc{Z}_n\subset \spc{U}$ such that $\spc{Z}_n$ is an $\tfrac1{2^n}$-midpoint of $\spc{X}'$, and $\spc{Y}'$ in $\Haus\spc{U}$ and 
\[|\spc{Z}_n-\spc{Z}_{n+1}|_{\Haus\spc{U}}<\tfrac1{2^n}\]
for any $n$.

Observe that the sequence $\spc{Z}_n$ converges in $\GH$, and its limit by $\spc{Z}$ is a midpoint of $\spc{X}$ and $\spc{Y}$.
Finally, apply \ref{lem:GH-complete} and \ref{lem:mid>geod}.

\parit{Source:} \cite{ivanov-nikolaeva-tuzhilin}.



\parit{\ref{SHORT.ex:sphere-to-ball:3}.}
Make fine burrows in the standard 3-ball without changing its topology,
but at the same time come sufficiently close to any point in the ball.

Consider the \index{doubling}\emph{doubling} of the obtained ball along its boundary;
that is, two copies of the ball with identified corresponding points on their boundaries.
The obtained space is homeomorphic to $\mathbb{S}^3$.
Note that the burrows can be made 
so that the obtained space is sufficiently close to the original ball 
in the Gromov--Hausdorff metric.

\parit{Source:} \cite[Exercises 7.5.13 and 7.5.17]{burago-burago-ivanov}. 

\parbf{\ref{ex:GH-po}}; \ref{SHORT.ex:GH-po:a}.
To check that $\dist{*}{*}{\GH'}$ is a metric, it is sufficient to show that
\[\dist{\spc{X}}{\spc{Y}}{\GH'}=0 
\quad\Longrightarrow\quad
\spc{X}\iso\spc{Y};\]
the remaining conditions are trivial.

If $\dist{\spc{X}}{\spc{Y}}{\GH'}=0$, then there is a sequence of maps $f_n\:\spc{X}\to \spc{Y}$ such that 
\[\dist{f_n(x)}{f_n(x')}{\spc{Y}}\ge \dist{x}{x'}{\spc{X}}-\tfrac1n.\]

Arguing the same way as in the proof of the ``only if''-part in \ref{GH-2} (page~\pageref{page:GH-2-proof}),
we get a distance-nondecreasing map $f_\infty\:\spc{X}\to \spc{Y}$.

The same way we can construct a distance-nondecreasing map 
$g_\infty\:\spc{Y}\to \spc{X}$.

By \ref{ex:non-contracting-map}, the compositions $f_\infty\circ g_\infty\:\spc{Y}\to \spc{Y}$ and $g_\infty\z\circ f_\infty\:\spc{X}\to \spc{X}$ are isometries.
Therefore, $f_\infty$ and $g_\infty$ are isometries as well.

\parit{\ref{SHORT.ex:GH-po:b}.} The implication 
\[|\spc{X}_n-\spc{X}_\infty|_{\GH}\to 0 
\quad\Rightarrow\quad 
\dist{\spc{X}_n}{\spc{X}_\infty}{\GH'}\to 0\]
follows from \ref{ex:eps-isom:GH>isom}. 

Now suppose $\dist{\spc{X}_n}{\spc{X}_\infty}{\GH'}\to 0$.
Show that $\{\spc{X}_n\}$ is a uniformly totally bonded family.

If $\dist{\spc{X}_n}{\spc{X}_\infty}{\GH}\not\to 0$, then we can pass to a subsequence such that $\dist{\spc{X}_n}{\spc{X}_\infty}{\GH}\ge\eps$ for some $\eps>0$.
By Gromov selection theorem, we can assume that $\spc{X}_n$ converges in the sense of Gromov--Hausdorff.
From the first implication, the limit $\spc{X}_\infty'$ has to be isometric to $\spc{X}_\infty$;
on the other hand, $\dist{\spc{X}_\infty'}{\spc{X}_\infty}{\GH}\ge \eps$ --- a contradiction.

\parbf{\ref{ex:GH-urysohn}.}
Apply \ref{thm:compact-homogeneous} and \ref{prop:GH-with-fixed-Z}.

%%%%%%%%%%%%%%%%%%%%%%%%%%%%%%%%

%%%%%%%%%%%%%%%%%%%%%%%%%%%%%%
\refstepcounter{chapter}
\setcounter{eqtn}{0}

\parbf{\ref{ex:ultrakatetov}.} 
Let $F=\set{n\in \NN}{f(n)=n}$; we need to show that $\omega(F)=1$.

Consider an oriented graph $\Gamma$ with vertex set $\NN\setminus F$ such that $m$ is connected to $n$ if $f(m)=n$.
Show that each connected component of $\Gamma$ has at most one cycle.
Use it to subdivide vertices of $\Gamma$ into three sets $S_1$, $S_2$, and $S_3$ such that $f(S_i)\cap S_i=\emptyset$ for each $i$.

Conclude that $\omega(S_1)=\omega(S_2)=\omega(S_3)=0$ and hence \[\omega(F)=\omega(\NN\setminus(S_1\cup S_2\cup S_3))=1.\]

\parit{Source:} 
The presented proof was given by Robert Solovay \cite{solovay}, but
the key statement is due to Miroslav Katětov \cite{katetov}.

\parbf{\ref{ex:linear}.}
Choose a nonprincipal ultrafilter $\omega$ and set $L(\bm{s})=s_\omega$.
It remains to observe that $L$ is linear.

\parit{Remark.} 
This construction identifies ultrafilters with vectors in $(\ell^\infty)^*$.
Recall that $\ell^\infty=(\ell^1)^*$ and $\ell^1\subsetneq(\ell^\infty)^*$.
A principle ultrafilter is a basis vector in $\ell^1$; 
nonprincipal ultrafilters lie in $(\ell^\infty)^*\setminus\ell^1$.
The set of ultrafilters is the closure of basis vectors in $\ell^1$ with respect to weak*-topology on $(\ell^\infty)^*$.


\parbf{\ref{ex:ultrakatetov+}.}
Use \ref{ex:ultrakatetov}.

\parbf{\ref{ex:lim(tree)}.}
Let $\gamma$ be a path from $p$ to $q$ in a metric tree $\spc{T}$.
Assume that $\gamma$ passes thru a point $x$ on distance $\ell$ from $[pq]$.
Then 
\[\length\gamma\ge \dist{p}{q}{}+2\cdot \ell.
\eqlbl{eq:+ell}\]

Suppose that $\spc{T}_n$ is a sequence of metric trees that $\omega$-converges to $\spc{T}_\omega$.
By \ref{obs:ultralimit-is-geodesic}, the space $\spc{T}_\omega$ is geodesic.

The uniqueness of geodesics follows from \ref{eq:+ell}.
Indeed, if for a geodesic $[p_\omega q_\omega]$ there is another geodesic $\gamma_\omega$ connecting its ends, then it has to pass thru a point $x_\omega\notin [p_\omega q_\omega]$.
Choose sequences $p_n,q_n,x_n\in\spc{T}_n$ such that $p_n\to p_\omega$, $q_n\to q_\omega$, and $x_n\to x_\omega$ as $n\to\omega$.
Then 
\begin{align*}
\dist{p_\omega}{q_\omega}{}&=\length\gamma\ge 
\\
&\ge\lim_{n\to\omega}(\dist{p_n}{x_n}{}+\dist{q_n}{x_n}{})\ge
\\
&\ge \lim_{n\to\omega}(\dist{p_n}{q_n}{}+2\cdot\ell_n)=
\\
&=\dist{p_\omega}{q_\omega}{}+2\cdot\ell_\omega.
\end{align*}
Since $x_\omega\notin [p_\omega q_\omega]$, we have that $\ell_\omega>0$ --- a contradiction.

It remains to show that any geodesic triangle $\spc{T}_\omega$ is a tripod.
Consider the sequence of centers of tripods $m_n$ for given sequences of points $x_n,y_n,z_n\in \spc{T}_n$.
Observe that its ultralimit $m_\omega$ is the center of the tripod with ends at $x_\omega,y_\omega,z_\omega\in \spc{T}_\omega$.

\parbf{\ref{ex:ultracompact}.}
Construct $\bm{X}$ and distance-preserving embeddings $\spc{X}_n\hookrightarrow\bm{X}$ that satisfy \ref{propery:GH}.
Given $x_\infty\in \spc{X}_\infty$, choose a sequence $x_n\in \spc{X}_n$ such that $x_n\to x_\infty$ in $\bm{X}$.
Let $x_\omega$ be $\omega$-limit of the sequence $x_n$ in $\bm{X}$.
Note that $x_\omega\in \spc{X}_\infty$.
Show that the map $x_\infty\mapsto x_\omega$ is defined; that is, it does not depend on the choice of the sequence $x_n$.
Further, show that the map $x_\infty\mapsto x_\omega$ is an isometry of $\spc{X}_\infty$.
Make a conclusion.

\parbf{\ref{ex:ultrapower}.}
Further, we consider $\spc{X}$ as a subset of $\spc{X}^\omega$.

\parit{\ref{SHORT.ex:ultrapower:a}.} Follows directly from the definitions.

\parit{\ref{SHORT.ex:ultrapower:compact}.}
Suppose $\spc{X}$ compact.
Given a sequence $x_1,x_2,\dots{}\in\spc{X}$, denote its $\omega$-limit in $\spc{X}^\omega$ by $x^\omega$ and its $\omega$-limit in $\spc{X}$ by $x_\omega$.

Observe that $x^\omega=\iota(x_\omega)$.
Therefore, $\iota$ is onto.

If $\spc{X}$ is not compact, we can choose a sequence $x_n$ such that $\dist{x_m}{x_n}{}>\eps$ for fixed $\eps>0$ and all $m\ne n$.
Observe that
\[\lim_{n\to\omega}\dist{x_n}{y}{\spc{X}}\ge \tfrac\eps2\]
for any $y\in\spc{X}$.
It follows that $x_\omega$ lies at the distance $\ge\tfrac\eps2$ from~$\spc{X}$.

\parit{\ref{SHORT.ex:ultrapower:proper}.}
A sequence of points $x_n$ in $\spc{X}$ will be called $\omega$-bounded if there is a real constant $C$ such that
\[\dist{p}{x_n}{\spc{X}}\le C\] 
for $\omega$-almost all $n$.

The same argument as in \ref{SHORT.ex:ultrapower:compact} shows that any $\omega$-bounded sequence has its $\omega$-limit in $\spc{X}$.
Further, if $(x_n)$ is not  $\omega$-bounded, then 
\[\lim_{n\to\omega}\dist{p}{x_n}{\spc{X}}=\infty;\]
that is, $x_\omega$ does not lie in the metric component of $p$ in $\spc{X}^\omega$.

\parbf{\ref{ex:isom-ultrapower}.}
Let us identify points in $\spc{X}$ with nonnegative integers.
Consider the set $\mathcal{A}$ of all sequences $a_n$ such that $a_0=0$ and $a_{n+1}=a_n+\eps_n\cdot 2^n$ where $\eps_n\in\{0,1\}$ for any $n$.
Observe that $\mathcal{A}$ has cardinality continuum and distinct sequences in $\mathcal{A}$ have distinct $\omega$-limits.
Conclude that the cardinality of $\spc{X}^\omega$ is at least continuum.

Show and use that the spaces $\spc{X}^\omega$ and $(\spc{X}^\omega)^\omega$ have discrete metrics and both have cardinality at most continuum.


\parbf{\ref{ex:ultrapower(ultrapower)}.}
Choose a bijection $\iota\:\NN\to \NN\times \NN$.
Given a set $S\subset \NN$, consider the sequence $S_1$, $S_2,\dots$ of subsets in $\NN$ defined by $m\in S_n$ if $(m,n)\z=\iota(k)$ for some $k\in S$.
Set $\omega_1(S)=1$ if and only if $\omega(S_n)=1$ for $\omega$-almost all $n$.
It remains to check that $\omega_1$ meets the conditions of the exercise.

\parit{Comment.}
It turns out that $\omega_1\ne \omega$ for any $\iota$;
see the post of Andreas Blass \cite{blass}.

\parbf{\ref{ex:two-geodesics-in-ultrapower}.}
Arguing as in \ref{obs:ultrapower-is-geodesic}, we get a pair of points $x$ and $y$ in $\spc{X}$ such that
\[\dist{p}{x}{}+\dist{x}{y}{}+\dist{y}{q}{}=\dist{p}{q}{}\]
and there is no midpoint between $x$ and $y$ in $\spc{X}$
(possibly $p=x$ and $q=y$).
Note that it is sufficient to show that there is a continuum of distinct midpoints in $\spc{X}^\omega$ between $x$ and $y$ in $\spc{X}$.

Since $\spc{X}$ is a length space, we can choose a $\tfrac1n$-midpoint $m_n\in\spc{X}$ between $x$ and $y$.
Note that the sequence $m_n$ contains no converging subsequence.
Conclude that we may pass to a subsequence of $m_n$ such that $\dist{m_i}{m_j}{}>\eps$ for a fixed $\eps>0$ and any $i\ne j$.

Argue as in \ref{ex:isom-ultrapower} to show that there is a continuum of distinct $\omega$-limits of subsequences of $m_n$;
each such limit is a midpoint between $x$ and $y$.

\parit{\ref{SHORT.ex:sphere-in-urysohn:homogeneous}.} 
Use \ref{SHORT.ex:sphere-in-urysohn:sphere}, maybe twice.

\parbf{\ref{ex:notproper-limit}.} Consider the infinite metric $\spc{T}$ tree with unit edges shown
on the diagram.
Observe that $\spc{T}$ is proper.

\begin{Figure}
\vskip-0mm
\centering
\includegraphics{mppics/pic-605}
\end{Figure}

Consider the vertex $v_\omega=\lim_{n\to\omega}v_n$ in the ultrapower $\spc{T}^\omega$.
Observe that $\omega$ has an infinite degree.
Conclude that $\spc{T}^\omega$ is not locally compact.

\parbf{\ref{ex:ultraT}.}
Consider a product of an infinite sequence of two-point spaces.

\parit{Remark.}
There are such examples with cocompact isometric action of finitely generated group \cite{thomas-velickovic}.

\parbf{\ref{ex:Asym(Lob)}.} Assume $\spc{L}$ is the Lobachevsky plane.

\parbf{\ref{SHORT.ex:Asym(Lob):metric-tree}.}
Show that there is $\delta>0$ such that sides of any geodesic triangle in $\spc{L}$ intersect a disk of radius $\delta$.
Conclude that any geodesic triangle in $\Asym\spc{L}$ is a tripod.

\parit{\ref{SHORT.ex:Asym(Lob):homogeneous}.} Observe that $\spc{L}$ is one-point-homogeneous and use it.

\parit{\ref{SHORT.ex:Asym(Lob):continuum}.} 
By \ref{SHORT.ex:Asym(Lob):homogeneous}, it is sufficient to show that $p_\omega$ has a continuum degree.

Choose distinct geodesics $\gamma_1,\gamma_2\:[0,\infty)\z\to L$ that start at a point $p$.
Show that the limits of $\gamma_1$ and $\gamma_2$ run in the different connected components of $(\Asym\spc{L})\setminus \{p_\omega\}$.
Since there is a continuum of distinct geodesics starting at $p$,
we get that the degree of $p_\omega$ is at least continuum.

On the other hand, the set of sequences of points in $\spc{L}$  has cardinality continuum.
In particular, the set of points in $\Asym\spc{L}$ has cardinality at most continuum.
It follows that the degree of any vertex is at most continuum.

The proof for the Lobachevsky space goes along the same lines.

For the infinite three-regular tree, part \ref{SHORT.ex:Asym(Lob):metric-tree} follows from \ref{ex:lim(tree)}.
The three-regular tree is only vertex-homogeneous; the latter is sufficient to prove \ref{SHORT.ex:Asym(Lob):homogeneous}.
No changes are needed in~\ref{SHORT.ex:Asym(Lob):continuum}.

\parit{Remark.}
According to the result of Anna Dyubina and Iosif Polterovich \cite{dyubina-polterovich}, the properties \ref{SHORT.ex:Asym(Lob):homogeneous} and \ref{SHORT.ex:Asym(Lob):continuum} describe the tree $\spc{T}$ up to isometry.
In particular, the asymptotic space of the Lobachevsky plane does not depend on the choice of the ultrafilter and the sequence $\lambda_n\to \infty$.


\parbf{\ref{ex:T(Sx[0,1]/Sx0)}.}
Denote by $o_\omega$ the point in $\T^\omega_o\spc{X}$ that corresponds to $o$.
Argue as in \ref{ex:Asym(Lob):continuum} to show that $\T^\omega_o\spc{X}\setminus \{o_\omega\}$ has continuum connected components.
Further, show that each connected component $\spc{W}_\alpha$ is isometric to $\RR\times (0,\infty)$ with the metric described by
\begin{align*}
&\dist{(x_1,t_1)}{(x_2,t_2)}{}=
\\
&\qquad=\min\{\,\dist{(x_1,t_1)}{(x_2,t_2)}{\RR^2},t_1+t_2\,\}.
\end{align*}

Conclude that the space $\T^\omega_o\spc{X}$ can be described as follows.
Consider continuum copies $\spc{W}_\alpha$ as above;
denote by $(x,t)_\alpha$ the point in $\spc{W}_\alpha$ with coordinates $(x,t)$.
The tangent space is the disjoint union of single point $o_\omega$ and all $\spc{W}_\alpha$ 
such that $\dist{(x_1,t_1)_\alpha}{(x_2,t_2)_\alpha}{}$ is the same as in $\spc{W}_\alpha$ and for the remaining pairs, we have $\dist{o_\omega}{(x,t)_\alpha}{}=t$ and $\dist{(x_1,t_1)_\alpha}{(x_2,t_2)_\beta}{}=t_1+t_2$
if $\alpha\ne\beta$.


\end{multicols}
}

\newgeometry{top=0.9in, bottom=0.9in,left=0.9in, right=0.9in, paperwidth=6in, paperheight=9in}

%%%%%%%%%%%%%%%%%%%%%%%%%%%%
{\small\sloppy
\documentclass[twoside]{book}

\usepackage{lectures}
\usepackage[colorlinks=true,
citecolor=black,
linkcolor=black,
anchorcolor=black,
filecolor=black,
menucolor=black,
urlcolor=black,
pdftitle={Pure metric geometry: introductory lectures},
pdfsubject={Geometry},
pdfauthor={Anton Petrunin}
]{hyperref}
\makeindex

\begin{document}
%\pagestyle{empty}\renewcommand\includegraphics[2][{}]{}\def\emph{\textit}
%\overfullrule=100mm

 
\title{Pure metric geometry:\\
introductory lectures}
\author{Anton Petrunin}
\date{}
\maketitle

\section*{Preface}

This text can serve as an introductory part to a variety of courses in metric geometry.
Here is a graph of essential dependencies of the lectures; some statements (mostly exercises) add more dependencies, but they can be ignored.
\begin{figure}[!ht]
\centering
\begin{tikzpicture}[->,>=stealth',shorten >=1pt,auto,scale=1.4,
  thick,main node/.style={circle,draw,font=\sffamily\bfseries,minimum size=8mm}]

  \node[main node] (1) at (1,0) {\ref{chap:defs}};
  \node[main node] (2) at (.5,-5/6){\ref{chap:urysohn}};
  \node[main node] (3) at (1.5,-5/6) {\ref{chap:injective}};
  \node[main node] (4) at (2,0) {\ref{chap:hausdorff}};
  \node[main node] (5) at (3,0) {\ref{chap:GH}};
  \node[main node] (6) at (4,0) {\ref{chap:ultralimits}};
  

  \path[every node/.style={font=\sffamily\small}]
   (1) edge node{}(2)
   (1) edge node{}(3)
   (1) edge node{}(4)
   (4) edge node{}(5)
   (5) edge node{}(6);
\end{tikzpicture}
\end{figure}
The necessary definitions introduced in (\ref{chap:defs}).
In (\ref{chap:urysohn}) we discuss the Urysohn space.
In (\ref{chap:injective}) we discuss injective spaces.
In (\ref{chap:hausdorff}) we introduce Hausdorff metric.
In (\ref{chap:GH}) and (\ref{chap:ultralimits}) we discuss two types of convergences of metric spaces --- the Gromov--Hausdorff limit and ultralimit.

Applications are given only as illustrations.
We stick to domestic affairs of metric spaces, keeping away from any extra structure. 
(Adding an extra structure brings an extra tool and often opens a huge field for development.
The examples include Alexandrov geometry,
geometric group theory,
metric-measure spaces and optimal transport.)

These notes are based on the minicourse given at SPbSU (Fall 2022) and the introductory part of a course at PSU (Spring 2020).
The latter included additional material from \cite{alexander-kapovitch-petrunin-2019,petrunin2020mnfld,nabutovsky}.
A part of the text is a compilation from \cite{alexander-kapovitch-petrunin-2019, alexander-kapovitch-petrunin-2025, petrunin-yashinski, petrunin-2022-PIGTIKAL, petrunin-zamorabarrera} and its drafts.

I want to thank
Sergei Ivanov,
Urs Lang,
Alexander Lytchak,
Rostislav Matveyev,
Julien Melleray,
and Sergio Zamora Barrera for help.
The present work is partially supported by NSF grant DMS-2005279
and the Simons Foundation grant \#584781.

\thispagestyle{empty}
\tableofcontents
\thispagestyle{empty}

\include{metric}
\include{uryson}
\include{injective}
\include{converge}
\include{ultralimit}
%\include{wasserstein}

\backmatter

\newgeometry{top=0.9in, bottom=0.9in,inner=0.5in, outer=0.5in}
\chapter{Semisolutions}

{

\footnotesize
\begin{multicols}{2}

\input{metric-sol}
\input{uryson-sol}
\input{injective-sol}
\input{converge-sol}
\input{ultralimit-sol}

\end{multicols}
}

\newgeometry{top=0.9in, bottom=0.9in,left=0.9in, right=0.9in, paperwidth=6in, paperheight=9in}

%%%%%%%%%%%%%%%%%%%%%%%%%%%%
{\small\sloppy
\input{pure-metric.ind}

\def\emph{\textit}

\printbibliography[heading=bibintoc]
\fussy
}


\end{document}


\def\emph{\textit}

\printbibliography[heading=bibintoc]
\fussy
}


\end{document}


\def\emph{\textit}

\printbibliography[heading=bibintoc]
\fussy
}


\end{document}


\printbibliography[heading=bibintoc]
\fussy
}


\end{document}
