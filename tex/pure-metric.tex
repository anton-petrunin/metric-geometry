\documentclass[twoside]{book}

\usepackage{lectures}
\usepackage[colorlinks=true,
citecolor=black,
linkcolor=black,
anchorcolor=black,
filecolor=black,
menucolor=black,
urlcolor=black,
pdftitle={Pure metric geometry: introductory lectures},
pdfsubject={Geometry},
pdfauthor={Anton Petrunin}
]{hyperref}
\makeindex

\begin{document}
%\pagestyle{empty}\renewcommand\includegraphics[2][{}]{}\def\emph{\textit}
%\overfullrule=100mm

 
\title{Pure metric geometry:\\
introductory lectures}
\author{Anton Petrunin}
\date{}
\maketitle

We discuss domestic affairs of metric spaces,
keeping away from any extra structure. 
Applications are given only as illustrations.

A part of the text is a compilation from \cite{alexander-kapovitch-petrunin-2019, alexander-kapovitch-petrunin-2025, petrunin-yashinski, petrunin-2022-PIGTIKAL, petrunin-zamorabarrera} and its drafts.
These notes are based on a minicourse I gave at SPbSU (Fall 2022) and the introductory part of a course at PSU (Spring 2020).
The latter included an introduction to Alexandrov geometry based on \cite{alexander-kapovitch-petrunin-2019} and metric geometry on manifolds \cite{petrunin2020mnfld} based on a simplified proof of Gromov's systolic inequality given by Alexander Nabutovsky~\cite{nabutovsky}.

I want to thank
Urs Lang,
Alexander Lytchak,
Rostislav Matveyev,
Julien Melleray,
and Sergio Zamora Barrera for help.
The present work is partially supported by NSF grant DMS-2005279
and the Simons Foundation grant \#584781.

\thispagestyle{empty}
\tableofcontents
\thispagestyle{empty}

\chapter{Definitions}

\section{Metric spaces}
\label{sec:metric spaces}


The distance between two points $x$ and $y$ in a metric space $\spc{X}$ will be denoted by $\dist{x}{y}{}$ or $\dist{x}{y}{\spc{X}}$.
The latter notation is used if we need to emphasize 
that the distance is taken in the space~${\spc{X}}$.

The function 
\[\distfun_x\:y\mapsto \dist{x}{y}{}\]
is called the \index{distance function}\emph{distance function} from~$x$. 

Given $R\in[0,\infty]$ and $x\in \spc{X}$, the sets
\begin{align*}
\oBall(x,R)&=\{y\in \spc{X}\mid \dist{x}{y}{}<R\},
\\
\cBall[x,R]&=\{y\in \spc{X}\mid \dist{x}{y}{}\le R\}
\end{align*}
are called, respectively, the  \index{open ball}\emph{open} and  the \index{closed ball}\emph{closed  balls}   of radius $R$ with center~$x$.
Again, if we need to emphasize that these balls are taken in the metric space $\spc{X}$,
we write 
\[\oBall(x,R)_{\spc{X}}\quad\text{and}\quad\cBall[x,R]_{\spc{X}}.\]


\section{Variations of definition}

Recall that a metric is a real-valued function $(x,y)\mapsto\dist{x}{y}{\spc{X}}$ that satisfies the following conditions for any three points $x,y,z\in \spc{X}$:
\begin{enumerate}[(i)]
\item $\dist{x}{y}{\spc{X}}\ge 0$,
\item\label{metric=0} $\dist{x}{y}{\spc{X}}= 0$ $\iff$ $x=y$,
\item $\dist{x}{y}{\spc{X}}=\dist{y}{x}{\spc{X}}$,
\item $\dist{x}{y}{\spc{X}}+\dist{y}{z}{\spc{X}}\ge\dist{x}{z}{\spc{X}}$,
\end{enumerate}

\parbf{Pseudometrics.}
A generalization of a metric in which the distance between two distinct points can be zero is called \emph{pseudometric}.
In other words, to define pseudometric, we need to remove condition (\ref{metric=0}) from the list.

The following two observations show that
nearly any question about pseudometric spaces can be reduced to a question about genuine metric spaces.

Assume $\spc{X}$ is a pseudometric space.
Set
$x\sim y$ if $\dist{x}{y}{}=0$. 
Note that if $x\sim x'$, then $\dist{y}{x}{}=\dist{y}{x'}{}$ for any $y\in\spc{X}$.
Thus, $\dist{*}{*}{}$ defines a metric on the
quotient set $\spc{X}/{\sim}$.
In this way we obtain a metric space $\spc{X}'$.
The space $\spc{X}'$ is called the 
\emph{corresponding metric space} for the pseudometric space $\spc{X}$.
Often we do not distinguish between $\spc{X}'$ and~$\spc{X}$. 

\parbf{$\bm{\infty}$-metrics.}
One may also consider metrics with values in $\RR\cup\{\infty\}$;
we might call them $\infty$-metrics or simply metrics.

Again nearly any question about $\infty$-metric spaces can be reduced to a question about genuine metric spaces. 

Indeed, set $x\approx y$ if and only if $\dist{x}{y}{}<\infty$;
this is an other equivalence relation on $\spc{X}$.
The equivalence class of a point $x\in\spc{X}$ will be called the \emph{metric component}\index{metric component} 
 of $x$; it will be denoted as $\spc{X}_x$.
One could think of $\spc{X}_x$ as  $\oBall(x,\infty)_{\spc{X}}$ --- the open ball centered at $x$ and radius $\infty$ in $\spc{X}$.

It follows that any $\infty$-metric space is a \emph{disjoint union} of genuine metric spaces --- the metric components of the original $\infty$-metric space.

\begin{thm}{Exercise}
Given two sets $A$ and $B$ on the plane, set 
\[\dist{A}{B}{}=\mu(A\backslash B)+\mu(B\backslash A),\]
where $\mu$ denotes the Lebesgue measure.
\begin{subthm}{}
Show that $\dist{*}{*}{}$ is a pseudometric on the set of bounded measurable sets of the plane.
\end{subthm}

\begin{subthm}{}
Show that $\dist{*}{*}{}$ is an $\infty$-metric on the set of all open sets of the plane.
\end{subthm}
\end{thm}

\section{Completeness}

Recall that a metric space $\spc{X}$ is called \emph{complete} if every Cauchy sequence of points in $\spc{X}$ converges in $\spc{X}$.

\begin{thm}{Exercise}\label{ex:almost-min}
Suppose that $\rho$ is a positive continuous function on a complete metric space $\spc{X}$.
Show that for any $\eps>0$ there is a point $x\in \spc{X}$ such that 
\[\rho(x)<(1+\eps)\cdot\rho(y)\]
for any point $y\in \oBall(x,\rho(x))$.
\end{thm}

Most of the time we will assume that a metric space is complete.
The following construction produces a complete metric space $\bar{\spc{X}}$ for any given metric space $\spc{X}$.
The space $\bar{\spc{X}}$ is called \emph{completion} of $\spc{X}$;
the original space $\spc{X}$ forms a dense subset in $\bar{\spc{X}}$.

\parbf{Completion.}
Given metric space $\spc{X}$, 
consider the set of all Cauchy sequences in $\spc{X}$.
Note that for any two Cauchy sequences $(x_n)$ and $(y_n)$ the right hand side in \ref{eq:cauchy-dist} is defined; moreover it defines a pseudometric on the set $\spc{C}$ of all Cauchy sequences
\[\dist{(x_n)}{(y_n)}{\spc{C}}\df\lim_{n\to\infty}\dist{x_n}{y_n}{\spc{X}}.\eqlbl{eq:cauchy-dist}\]
The corresponding metric space is called a completion of $\spc{X}$.

It is left as an exercise that completion of $\spc{X}$ is complete.

Note that for each point $x\in\spc{X}$ one can consider a constant sequence $x_n=x$ which is Cauchy.
It defines a natural map $\spc{X}\to \bar{\spc{X}}$.
It is easy to check that this map is distance preserving.
In partucular we can (and will) consider $\spc{X}$ as a subset of $\bar{\spc{X}}$.

\section{Compactness}

Let us recall few equivalent definitions of compact metric spaces.

\begin{thm}{Definition}\label{def:compact}
A metric space $\spc{K}$ is compact if and only if one of the following equivalent condition holds:

\begin{subthm}{}
 Every open cover of $\spc{K}$ has a finite subcover.
\end{subthm}

\begin{subthm}{}
 For any open cover of $\spc{K}$ there is $\eps>0$ such that any $\eps$-ball in $\spc{K}$ lie in one element of the cover. (The value $\eps$ is called Lebesgue number of the covering.)
\end{subthm}

\begin{subthm}{}
 Every sequence in $\spc{K}$ has a convergent subsequence.
\end{subthm}

\begin{subthm}{totally-bounded}
The space $\spc{K}$ is complete and \emph{totally bounded}; that is, for any $\eps>0$, the space $\spc{K}$ admits a finite cover by open $\eps$-balls.\footnote{Equivalently, for any $\eps>0$ there is a finite \emph{$\eps$-net}; that is a finite set of points $x_1,\dots,x_n\in \spc{K}$ such that any other point $x$ lies on the distance less than $\eps$ from one of $x_i$.}
\end{subthm}

\end{thm}

Let $\pack_\eps\spc{X}$ be exact upper bound on the number of points $x_1,\z\dots,x_n\in \spc{X}$ such that $\dist{x_i}{x_j}{}\ge\eps$ for any $i\ne j$.

If $n=\pack_\eps\spc{X}<\infty$, then
the collection of points $x_1,\dots,x_n$ is called a \emph{maximal $\eps$-packing}.
Note that $n$ is the maximal number of open disjoint $\tfrac\eps2$-balls in $\spc{X}$.

\begin{thm}{Exercise}\label{ex:pack-net}
Show that a complete space $\spc{X}$ is compact if and only of $\pack_\eps\spc{X}\z<\infty$ for any $\eps>0$.

Show that any maximal $\eps$-packing is an $\eps$-net.
\end{thm}


\begin{thm}{Exercise}\label{ex:non-contracting-map}
Let $\spc{K}$  be a compact metric space and
\[f\:\spc{K}\z\to \spc{K}\] 
be a distance non-decreasing map.
Prove that $f$ is an isometry.
\end{thm}


A metric space $\spc{X}$ is called \index{proper space}\emph{proper} if all closed bounded sets in $\spc{X}$ are compact. 
This condition is equivalent to each of the following statements:
\begin{itemize}
\item For some (and therefore any) point $p\in \spc{X}$ and any $R<\infty$, 
the closed ball $\cBall[p,R]_{\spc{X}}$ is compact. 
\item The function $\distfun_p\:\spc{X}\to\RR$ is proper for some (and therefore any) point $p\in \spc{X}$;
that is, for any compact set $K\subset \RR$, its inverse image 
\[\distfun_p^{-1}(K)=\set{x\in \spc{X}}{\dist{p}{x}{\spc{X}}\in K}\]
is compact.
\end{itemize}

A metric space $\spc{X}$ is called \emph{locally compact} if any point in $\spc{X}$ admits a compact neighborhood;
in other words, for any point $x\in\spc{X}$ a closed ball $\cBall[x,r]$ is compact for some $r>0$.

\section{Geodesics}
\label{sec:geods}

Let $\spc{X}$ be a metric space 
and $\II$\index{$\II$} a real interval. 
A~globally isometric map $\gamma\:\II\to \spc{X}$ is called a \index{geodesic}\emph{geodesic}%
\footnote{Various authors call it differently: {}\emph{shortest path}, {}\emph{minimizing geodesic}.}; 
in other words, $\gamma\:\II\to \spc{X}$ is a geodesic if 
\[\dist{\gamma(s)}{\gamma(t)}{\spc{X}}=|s-t|\]
for any pair $s,t\in \II$.

We say that  $\gamma\:\II\to \spc{X}$ is a geodesic from point $p$ to point $q$ if 
$\II=[a,b]$ and $p=\gamma(a)$, $q=\gamma(b)$.
In this case the image of $\gamma$ is denoted by $[p q]$\index{$[{*}{*}]$} and with an abuse of notations  we also call it a \index{geodesic}\emph{geodesic}.
Given a geodesic $[pq]$, we can parametrize it by distance to $p$;
this parametrization will be denoted by $\geod_{[p q]}(t)$.


We may write $[p q]_{\spc{X}}$ 
to emphasize that the geodesic $[p q]$ is in the space  ${\spc{X}}$.
We also use the following shortcut notation:
\begin{align*}
\left] p q \right[&=[pq]\backslash\{p,q\},
&
\left] p q \right]&=[pq]\backslash\{p\},
&
\left[ p q \right[&=[pq]\backslash\{q\}.
\end{align*}

In general, a geodesic from $p$ to $q$ need not exist and if it exists, it need not  be unique.  
However, once we write $[p q]$ we assume mean that we have made a choice of geodesic.

A metric space is called \index{geodesic}\emph{geodesic} if any pair of its points can be joined by a geodesic. 


A \index{geodesic path}\emph{geodesic path} is a geodesic with constant-speed parametrization by $[0,1]$.
Given a geodesic $[p q]$,
we denote by $\geodpath_{[pq]}$ the corresponding geodesic path;
that is,
$$\geodpath_{[pq]}(t)\z\df\geod_{[pq]}(t\cdot\dist[{{}}]{p}{q}{}).$$

A curve $\gamma\:\II\to \spc{X}$  is called a \index{geodesic!local geodesic}\emph{local geodesic} if for any $t\in\II$ there is a neighborhood $U$ of $t$ in $\II$ such that the restriction $\gamma|_U$ is a  geodesic.
A constant-speed parametrization of a local geodesic by the unit interval $[0,1]$ is called a \index{geodesic!local geodesic}\emph{local geodesic path}. 



\section{Length}

A \emph{curve} is defined as a continuous map from a real interval to a space.
If the real interval is $[0,1]$, then the curve is called a \emph{path}.

\begin{thm}{Definition}
Let $\spc{X}$ be a metric space and
$\alpha\: \II\to \spc{X}$ be a curve.
We define the \index{length}\emph{length} of $\alpha$ as 
\[
\length \alpha \df \sup_{t_0\le t_1\le\ldots\le t_n}\sum_i \dist{\alpha(t_i)}{\alpha(t_{i-1})}{}.
\]

A curve $\alpha$ is called \emph{rectifiable} if $\length \alpha<\infty$.
\end{thm}



\begin{thm}{Theorem}\label{thm:length-semicont}
Length is a lower semi-continuous with respect to pointwise convergence of curves. 

More precisely, assume that a sequence
of curves $\gamma_n\:\II\to \spc{X}$ in a metric space $\spc{X}$ converges pointwise 
to a curve $\gamma_\infty\:\II\to \spc{X}$;
that is, for any fixed $t \in \II$, $\gamma_n(t)\z\to\gamma_\infty(t)$ as $n\to\infty$. 
Then 
$$\liminf_{n\to\infty} \length\gamma_n \ge \length\gamma_\infty.\eqlbl{eq:semicont-length}$$
\end{thm}


\begin{wrapfigure}{o}{20 mm}
\vskip-0mm
\centering
\includegraphics{mppics/pic-10}
\end{wrapfigure}


Note that the inequality \ref{eq:semicont-length} might be strict.
For example the diagonal $\gamma_\infty$ of the unit square 
can be  approximated by a stairs-like
polygonal curves $\gamma_n$
with sides parallel to the sides of the square ($\gamma_6$ is on the picture).
In this case
\[\length\gamma_\infty=\sqrt{2}\quad
\text{and}\quad \length\gamma_n=2\]
for any $n$.

\parit{Proof.}
Fix a sequence $t_0<t_1<\dots<t_k$ in $\II$.
Set 
\begin{align*}\Sigma_n
&\df
|\gamma_n(t_0)-\gamma_n(t_1)|+\dots+|\gamma_n(t_{k-1})-\gamma_n(t_k)|.
\\
\Sigma_\infty
&\df
|\gamma_\infty(t_0)-\gamma_\infty(t_1)|+\dots+|\gamma_\infty(t_{k-1})-\gamma_\infty(t_k)|.
\end{align*}

Note that for each $i$ we have 
\[|\gamma_n(t_{i-1})-\gamma_n(t_i)|\to|\gamma_\infty(t_{i-1})-\gamma_\infty(t_i)|\]
and therefore
\[\Sigma_n\to \Sigma_\infty\] 
as $n\to\infty$.
Note that 
\[\Sigma_n\le\length\gamma_n\]
for each $n$.
Hence
$$\liminf_{n\to\infty} \length\gamma_n \ge \Sigma_\infty.\eqlbl{>=Sigma-infty}$$

If $\gamma_\infty$ is rectifiable, we can assume that 
\begin{align*}
\length\gamma_\infty<\Sigma_\infty+\eps.
\end{align*}
for any given $\eps>0$.
By \ref{>=Sigma-infty} it follows that 
$$\liminf_{n\to\infty} \length\gamma_n > \length\gamma_\infty-\eps$$
for any $\eps>0$; whence \ref{eq:semicont-length} follows.

It remains to consider the case when $\gamma_\infty$ is not rectifiable; 
that is, $\length\gamma_\infty=\infty$.
In this case we can choose a partition so that $\Sigma_\infty>L$ for any real number $L$.
By \ref{>=Sigma-infty} it follows that 
$$\liminf_{n\to\infty} \length\gamma_n > L$$
for any given $L$; whence 
\[\liminf_{n\to\infty}\length\gamma_n=\infty\]
and \ref{eq:semicont-length} follows.
\qeds

\section{Length spaces}\label{sec:intrinsic}

If for any $\eps>0$ and any pair of points $x$ and $y$ in a metric space $\spc{X}$, there is a path $\alpha$ connecting $x$ to $y$ such that
\[\length\alpha< \dist{x}{y}{}+\eps,\]
then $\spc{X}$ is called a \index{length space}\emph{length space} and the metric on $\spc{X}$ is called a \index{length metric}\emph{length metric}.\label{page:length metric}

Note that any geodesic space is a length space.
As can be seen from the following example, the converse does not hold.


\begin{thm}{Example}
Let $\spc{X}$ be obtained by gluing a countable collection of disjoint intervals $\{\II_n\}$ of length $1+\tfrac1n$, where for each $\II_n$ the left end is glued to $p$ and the right end to~$q$.

Observe that the space $\spc{X}$ carries a natural complete length metric with respect to which $\dist{p}{q}{}=1$ but there is no geodesic connecting $p$ to~$q$.
\end{thm}



\begin{thm}{Exercise}\label{ex:no-geod}
Give an example of a complete length space for which no pair of distinct points can be joined by a geodesic.
\end{thm}

Directly from the definition, it follows that if a path $\alpha\:[0,1]\to\spc{X}$ connects two points $x$ and $y$ 
(that is, if $\alpha(0)=x$ and $\alpha(1)=y$), then 
\[\length\alpha\ge \dist{x}{y}{}.\]
Set 
\[\yetdist{x}{y}{}=\inf\{\length\alpha\}\]
where the greatest lower bound is taken for all paths connecing $x$ and $y$.
It is straightforward to check that $(x,y)\mapsto \yetdist{x}{y}{}$ is an $\infty$-metric; moreover $(\spc{X},\yetdist{*}{*}{})$ is a length space.
The metric $\yetdist{*}{*}{}$ is called \emph{induced length metric}.

\begin{thm}{Exercise}\label{ex:compact=>complete}
Suppose $(\spc{X},\dist{*}{*}{})$ is a complete metric space.
Show that $(\spc{X},\yetdist{*}{*}{})$ is complete.
\end{thm}


Let $A$ be a subset of a metric space $\spc{X}$.
Given two points $x,y\in A$,
consider the value
\[\dist{x}{y}{A}=\inf_{\alpha}\{\length\alpha\},\]
where the greatest lower bound is taken for all paths $\alpha$ from $x$ to $y$ in $A$.%
\footnote{This notation slightly conflicts with the previously defined notation for distance $\dist{x}{y}{\spc{X}}$ in a metric space $\spc{X}$. However, most of the time we will work with ambient length spaces where the meaning will be unambiguous.}

Let $\spc{X}$ be a metric space and $x,y\in\spc{X}$.

\begin{enumerate}[(i)]
\item A point $z\in \spc{X}$ is called a \index{midpoint}\emph{midpoint} between $x$ and $y$
if 
\[\dist{x}{z}{}=\dist{y}{z}{}=\tfrac12\cdot\dist[{{}}]{x}{y}{}.\]
\item Assume $\eps\ge 0$.
A point $z\in \spc{X}$ is called an \index{$\eps$-midpoint}\emph{$\eps$-midpoint} between $x$ and $y$
if 
\[\dist{x}{z}{},\quad\dist{y}{z}{}\le\tfrac12\cdot\dist[{{}}]{x}{y}{}+\eps.\]
\end{enumerate}


Note that a $0$-midpoint is the same as a midpoint.


\begin{thm}{Lemma}\label{lem:mid>geod}
Let $\spc{X}$ be a complete metric space.
\begin{subthm}{lem:mid>length}
Assume that for any pair of points $x,y\in \spc{X}$  
 and any $\eps>0$
there is an $\eps$-midpoint~$z$.
Then $\spc{X}$ is a length space.
\end{subthm}

\begin{subthm}{lem:mid>geod:geod}
Assume that for any pair of points $x,y\in \spc{X}$, 
there is a midpoint~$z$.
Then $\spc{X}$ is a geodesic space.
\end{subthm}
\end{thm}

\parit{Proof.}
We first prove (\ref{SHORT.lem:mid>length}).
Let $x,y\in \spc{X}$ be a pair of points.

Set $\eps_n=\frac\eps{4^n}$, $\alpha(0)=x$ and $\alpha(1)=y$.

Let $\alpha(\tfrac12)$ be an $\eps_1$-midpoint between $\alpha(0)$ and $\alpha(1)$.
Further, let $\alpha(\frac14)$ 
and $\alpha(\frac34)$ be $\eps_2$-midpoints between the pairs $(\alpha(0),\alpha(\tfrac12))$ 
and $(\alpha(\tfrac12),\alpha(1))$ respectively.
Applying the above procedure recursively,
on the $n$-th step we define $\alpha(\tfrac{k}{2^n})$,
for every odd integer $k$ such that $0<\tfrac k{2^n}<1$, 
as an $\eps_{n}$-midpoint between the already defined
$\alpha(\tfrac{k-1}{2^n})$ and $\alpha(\tfrac{k+1}{2^n})$.


In this way we define $\alpha(t)$ for $t\in W$,
where $W$ denotes the set of dyadic rationals in $[0,1]$.
Since $\spc{X}$ is complete, the map $\alpha$ can be extended continuously to $[0,1]$.
Moreover,
\[\begin{aligned}
\length\alpha&\le \dist{x}{y}{}+\sum_{n=1}^\infty 2^{n-1}\cdot\eps_n\le
\\
&\le \dist{x}{y}{}+\tfrac\eps2.
\end{aligned}
\eqlbl{eq:eps-midpoint}
\]
Since $\eps>0$ is arbitrary, we get (\ref{SHORT.lem:mid>length}).

To prove (\ref{SHORT.lem:mid>geod:geod}), 
one should repeat the same argument 
taking midpoints instead of $\eps_n$-midpoints.
In this case \ref{eq:eps-midpoint} holds for $\eps_n=\eps=0$.
\qeds

Since in a compact space a sequence of $\tfrac1n$-midpoints $z_n$ contains a convergent subsequence, Lemma~\ref{lem:mid>geod} immediately implies

\begin{thm}{Proposition}\label{prop:length+proper=>geodesic}
A proper length space is geodesic.
\end{thm}

\begin{thm}{Hopf--Rinow theorem}\label{thm:Hopf-Rinow}
Any complete, locally compact length space is proper.
\end{thm}

It is instructive to solve the following exercise before reading the proof.

\begin{thm}{Exercise}
Give an example of space which is locally compact but not proper.
\end{thm}

\parit{Proof.}
Let $\spc{X}$ be a locally compact length space.
Given $x\in \spc{X}$, denote by $\rho(x)$ the supremum of all $R>0$ such that
the closed ball $\cBall[x,R]$ is compact.
Since $\spc{X}$ is locally compact, 
$$\rho(x)>0
\quad\text{for any}\quad
x\in \spc{X}.\eqlbl{eq:rho>0}$$
It is sufficient to show that $\rho(x)=\infty$ for some (and therefore any) point $x\in \spc{X}$.

Assume the contrary; that is, $\rho(x)<\infty$. We claim that

\begin{clm}{} $B=\cBall[x,\rho(x)]$ is compact for any~$x$.
\end{clm}

Indeed, $\spc{X}$ is a length space;
therefore for any $\eps>0$, 
the set $\cBall[x,\rho(x)-\eps]$ is a compact $\eps$-net in~$B$.
Since $B$ is closed and hence complete, it must be compact.
\claimqeds
Next we claim that
\begin{clm}{} $|\rho(x)-\rho(y)|\le \dist{x}{y}{\spc{X}}$ for any $x,y\in \spc{X}$;
in particular $\rho\:\spc{X}\to\RR$ is a continuous function.
\end{clm}

Indeed, 
assume the contrary; that is, $\rho(x)+|x-y|<\rho(y)$ for some $x,y\in \spc{X}$. 
Then 
$\cBall[x,\rho(x)+\eps]$ is a closed subset of $\cBall[y,\rho(y)]$ for some $\eps>0$.
Then  compactness of $\cBall[y,\rho(y)]$ implies compactness of $\cBall[x,\rho(x)+\eps]$, a contradiction.\claimqeds

Set $\eps=\min\set{\rho(y)}{y\in B}$; the minimum is defined since $B$ is compact.
From \ref{eq:rho>0}, we have $\eps>0$.

Choose a finite $\tfrac\eps{10}$-net $\{a_1,a_2,\dots,a_n\}$ in $B$.
The union $W$ of the closed balls $\cBall[a_i,\eps]$ is compact.
Clearly 
$\cBall[x,\rho(x)+\frac\eps{10}]\subset W$.
Therefore $\cBall[x,\rho(x)+\frac\eps{10}]$ is compact,
a contradiction.
\qeds

\begin{thm}{Exercise}\label{exercise from BH}
Construct a geodesic space that is locally compact,
but whose completion is neither geodesic nor locally compact.
\end{thm}

\section{Subsets in normed spaces}

Recall that a function $v\mapsto |v|$ on a vector space $\spc{V}$ is called \emph{norm} if it satisfies the following condition for any two vectors $v,w\in \spc{V}$ and a scalar $\alpha$:
\begin{itemize}
\item $|v|\ge 0$;
\item $|\alpha\cdot v|=|\alpha|\cdot |v|$;
\item $|v|+|w|\ge|v+w|$.
\end{itemize}

It is straightforward to check that for any normed space the function $(v,w)\mapsto |v-w|$ defines a metric on it.
Therefore any normed space is an example of metric space (which is in fact geodesic).
The following lemma says in particular that any metric space is isometric to a subset of a normed space.

\begin{thm}{Lemma}\label{lem:frechet}
Suppose $\spc{X}$ is a bounded separable space;
that is, $\diam\spc{X}$ is finite and $\spc{X}$ contains a countable, dense set $\{w_n\}$.
Given $x\in \spc{X}$, set $a_n(x)=\dist{w_n}{x}{\spc{X}}$.
Then 
\[\iota\:x\mapsto (a_1(x), a_2(x),\dots)\]
defines a distance preserving embedding $\iota\:\spc{X}\hookrightarrow \ell^\infty$.
\end{thm}

\parit{Proof.}
By the triangle inequality 
\[|a_n(x)-a_n(y)|\le \dist{x}{y}{\spc{X}}.\]
Therefore $\iota$ is short.

Again by triangle inequality we have 
\[|a_n(x)-a_n(y)|\ge \dist{x}{y}{\spc{X}}-2\cdot\dist{w_n}{x}{\spc{X}}.\]
Since the set $\{w_n\}$ is dense, we can choose $w_n$ arbitrary close to $x$.
Whence the value $|a_n(x)-a_n(y)|$ can be chosen arbitrary close to $\dist{x}{y}{\spc{X}}$.
In other words 
\[\sup_n\{\,|\dist{w_n}{x}{\spc{X}}-\dist{w_n}{y}{\spc{X}}|\,\}\ge \dist{x}{y}{\spc{X}};\]
hence $\iota$ is distance non-decreasing.
\qeds

The following exercise generalizes the lemma to arbitrary separable spaces.

\begin{thm}{Exercise}
Suppose $\{w_n\}$ is a countable, dense set in a metric space $\spc{X}$.
Choose $x_0\in \spc{X}$;
given $x\in \spc{X}$, set 
\[a_n(x)=\dist{w_n}{x}{\spc{X}}-\dist{w_n}{x_0}{\spc{X}}.\]
Show that $\iota\:x\mapsto (a_1(x), a_2(x),\dots)$ defines a distance preserving embedding $\iota\:\spc{X}\hookrightarrow \ell^\infty$.
\end{thm}


\begin{thm}{Exercise}\label{ex:compact-length}
Show that any compact metric space is isometric to a subspace of a compact geodesic space. 
\end{thm}

The lemma above was proved by Maurice René Fréchet in the paper where he defined metric space \cite{frechet}.
Nearly identical construction was rediscovered later by Kazimierz Kuratowski~\cite{kuratowski}.
Namely he made the following claim:

\begin{thm}{Lemma}\label{lem:kuratowski}
Let $\spc{X}$ be arbitrary metric space.
Denote by $\ell^\infty(\spc{X})$ the space of all bounded functions of $\spc{X}$ equipped with sup-norm.

Then for any point $x_0\in \spc{X}$, the map $\iota\:\spc{X}\to \ell^\infty(\spc{X})$ defied by 
\[\iota\:x\mapsto (\distfun_x-\distfun_{x_0})\]
is distance preserving.
\end{thm}

Note that this claim implies that \emph{any metric space is isometric to a subset of a normed vector space}.








\chapter{Urysohn space}

We discuss a construction introduced by Pavel Urysohn~\cite{urysohn}.
Our presentation is very close to the one in \cite{gromov-2007}.

This subject is closely related to the so called \emph{Rado graph},
also known as \emph{Erd\H{o}s–R\'enyi graph} or \emph{random graph}; a good survey this subject is written by Peter Cameron~\cite{cameron}.

\section{Existance}
Suppose a metric space $\spc{X}$ is a subspace of a pseudometric space $\spc{X}'$.
In this case we may say that $\spc{X}'$ is an \emph{extension} of $\spc{X}$.
If $\diam\spc{X}'\le d$, then we say that $\spc{X}'$ is a \emph{$d$-extension}.

If the complement $\spc{X}'\backslash \spc{X}$ contains a single point, say $p$, we say that $\spc{X}'$ is a \emph{one-point extension} of $\spc{X}$.
In this case, to define metric on $\spc{X}'$, it is sufficient to specify the distance function from $p$; that is, a function $f\:\spc{X}\to\RR$ defined by 
\[f(x)=\dist{p}{x}{\spc{X}'}.\]

The function $f$ can not be taken arbitrary --- the triangle inequality implies that 
\[f(x)+f(y)\ge \dist{x}{y}{\spc{X}}\ge |f(x)-f(y)|\]
for any $x,y\in \spc{X}$.
In particular $f$ is a non-negative 1-Lipschitz function on $\spc{X}$.
For a $d$-extension we need to assume in addition that $\diam\spc{X}\le d$ and $f(x)\le d$ for any $x\in \spc{X}$.

Any function $f$ of that type will be called \emph{extension function} or \emph{$d$-extension function} correspondingly.

\begin{thm}{Definition}\label{def:universal}
A metric space $\spc{U}$ is called \emph{universal}  if for any finite subspace $\spc{F}\subset\spc{U}$ and any extension function $f\:\spc{F}\to\RR$ there is a point $p\in \spc{U}$ such that $\dist{p}{x}{}=f(x)$ for any $x\in \spc{F}$.

If instead of extension functions we consider only $d$-extension functions and assume in addition that $\diam \spc{U}\le d$, then we arrive to a definition of \emph{$d$-universal space}.

If in addition $\spc{U}$ is separable and complete, then it is called \emph{Urysohn space} or \emph{$d$-Urysohn space}.
\end{thm}


\begin{thm}{Proposition}\label{prop:univeral-separable}
Given a positive $d$, there is a separable $d$-universal metric space.

Moreover, a separable universal space metric exists.
\end{thm}

\parit{Proof.}
Let $\spc{X}$ be a compact metric space such that $\diam\spc{X}\le d$.
Denote by $\spc{X}^d$ the space of all $d$-extension functions on $\spc{X}$ equipped with the metric defined by sup-norm.
Note that the map $\spc{X} \to \spc{X}^d$ defined by $x\mapsto\distfun_x$ is a distance preserving embedding,
so we can (and will) treat $\spc{X}$ as a subspace of $\spc{X}^d$, or, equivalently, $\spc{X}^d$ is an extension of $\spc{X}$.

Let us iterate this construction.
Start with a one-point space $\spc{X}_0$ and consider a sequence of spaces $(\spc{X}_n)$ defined by $\spc{X}_{n+1}\z=\spc{X}_n^d$.
Note that the sequence is nested, that is $\spc{X}_0\subset \spc{X}_1\subset\dots$
and the union
\[\spc{X}_\infty=\bigcup_n\spc{X}_n;\]
comes with metric such that
$\dist{x}{y}{\spc{X}_\infty} = \dist{x}{y}{\spc{X}_n}$
if $x,y\in\spc{X}_n$.

Note that if $\spc{X}$ is compact, then so is $\spc{X}^d$.
It follows that each space $\spc{X}_n$ is compact.
Since $\spc{X}_\infty$ is a countable union of compact spaces, it is separable.

Any finite subspace $\spc{F}$ of $\spc{X}_\infty$ lies in some $\spc{X}_n$ for $n<\infty$.
By construction, there is a point $p\in \spc{X}_{n+1}$ that meets the condition in Definiton~\ref{def:universal}.
That is, $\spc{X}_\infty$ is $d$-universal.

A construction of a universal separable metric space is done along the same lines, but one has the sequence should be defined by $\spc{X}_{n+1}\z=\spc{X}_n^{d_n}$ for some sequence $d_n\to\infty$.
\qeds

\begin{thm}{Proposition}\label{prop:completion-univeral}
A completion of $d$-universlal space is $d$-universal.

A completion of universal space universal.
\end{thm}

Note that \ref{prop:univeral-separable} and \ref{prop:completion-univeral} imply the following:

\begin{thm}{Theorem}\label{thm:urysohn-exists}
Urysohn space, and $d$-Urysohn space for any $d>0$, exist.
\end{thm}


\parit{Proof.} Suppose $\spc{V}$ be a $d$-universal space;
denote by $\spc{U}$ its completion; so $\spc{V}$ is a dense subset in a complete space $\spc{U}$.

Observe that $\spc{U}$ is \emph{approximately $d$-universal};
that is, if $\spc{F}\subset\spc{U}$ is a finite set, and $f\:\spc{F}\to \RR$ is a $d$-extension function, then
there exists $p\in \spc{U}$ such that
\[\dist{p}{x}{}\lg f(x)\pm\eps.\]
for any $x\in\spc{F}$.

Therefore there is a sequence of points $p_n\in \spc{U}$ such that for any $x\in \spc{F}$, 
\[\dist{p_n}{x}{}\lg f(x)\pm\tfrac1{2^n}.\]

Moreover, we can assume that 
\[\dist{p_n}{p_{n+1}}{} < \tfrac1{2^n}\eqlbl{eq:|pn-pn|}\]
for all large $n$.
Indeed, consider the sets $\spc{F}_n=\spc{F}\cup\{p_n\}$ and the functions $f_n$ defined by $f_n(x)=f(x)$ for any $x\in \spc{F}$, and
\[f_n(p_n)=\max\set{\bigl|\dist{p_n}{x}{}- f(x)\bigr|}{x\in \spc{F}}.\]
Observe that $f_n$ is a an $d$-extension function for large $n$ and
$f_n(p_n)\z<\tfrac1{2^n}$.
By applying approximate universal property recursively we get~\ref{eq:|pn-pn|}.

By \ref{eq:|pn-pn|}, $(p_n)$ is a Cauchy sequence and its limit meets the condition in the definition of universal space (\ref{def:universal}).
\qeds

\section{Universality}

\begin{thm}{Proposition}\label{prop:sep-in-urys}
Let $\spc{U}$ a Urysohn space.
Then any separable metric space $\spc{S}$ admits a distance preserving embedding $\spc{S}\hookrightarrow\spc{U}$.

Moreover, for any finite subspace $\spc{F}\subset \spc{S}$,
any distance preserving embedding $\spc{F}\hookrightarrow \spc{U}$ can be extended to an distance preserving embedding $\spc{S}\hookrightarrow\spc{U}$.

If $\spc{U}$ is $d$-Urysohn,
then the statements hold provided $\diam\spc{S}\le d$.  
\end{thm}

\parit{Proof.}
The required isometry will be denoted by $x\mapsto x'$.

Choose a dense sequence of points $s_1,s_2,\dotsc\in\spc{S}$.
We may assume that $\spc{F}=\{s_1,\dots,s_n\}$, so $s_i'\in \spc{U}$ are defined for $i\le n$.

The sequence $s_i'$ for $i>n$ can be defined recursively using universality of $\spc{U}$.
Namely that $s_1',\dots,s_{i-1}'$ are already defined.
Since $\spc{U}$ is universal, there is a point $s_i'\in \spc{U}$ such that
\[\dist{s_i'}{s_j'}{\spc{U}}=\dist{s_i}{s_j}{\spc{S}}\]
for any $j<i$.

We constructed a distance preserving map $s_i\mapsto s_i'$, it remains to extend it to a continuous map on whole $\spc{S}$.

The first statement follows if $\spc{F}=\emptyset$.\qeds

\begin{thm}{Exercise}\label{ex:geodesics-urysohn}
Show that any two distinct points in an Urysohn space can be jointed by infinite number of geodesics.
\end{thm}

\begin{thm}{Exercise}\label{ex:sc-urysohn}
Show that Urysohn space is simply connected.
\end{thm}

\section{Uniqueness}

\begin{thm}{Theorem}\label{thm:urysohn-unique}
Suppose $\spc{F}\subset \spc{U}$ and $\spc{F}'\subset \spc{U}'$ be finite isometric subspaces in a pair of ($d$-)Urysohn spaces $\spc{U}$ and $\spc{U}'$.
Then any isometry $\spc{F}\to \spc{F}'$ can be extended to an isometry $\spc{U}\to \spc{U}'$.

In particular ($d$-)Urysohn space is unique up to isometry.
\end{thm}

Note that \ref{prop:sep-in-urys} implies that there are distance-preserving maps $\spc{U}\to \spc{U}'$ and $\spc{U}'\to \spc{U}$,
but it does not immideately imply existence of an isometry.
The following construction use the same idea as in the proof of \ref{prop:sep-in-urys}, but we need to apply it \emph{back and forth} to ensure that the constructed distance-preserving map is onto.

\parit{Proof.}
The required isometry $\spc{U}\leftrightarrow \spc{U}'$ will be denoted by $u \leftrightarrow u'$.

Choose a dense sequences $a_1,a_2,\dots\in \spc{U}$ and $b'_1,b'_2,\dots\in \spc{U}$.
Let us define recursively $a_1',b_1, a_2', b_2,\dots$ --- on the odd step we define the images of $a_1,a_2,\dots$ and on the even steps we define invese images of $b'_1,b'_2,\dots$.
The same argument as in the proof of \ref{prop:sep-in-urys} shows that we can construct two sequences $a_1',a_2',\dots\in \spc{U}'$ and $b_1,b_2,\dots\in \spc{U}$ such that
\begin{align*}
\dist{a_i}{a_j}{\spc{U}}&=\dist{a_i'}{a_j'}{\spc{U}'}
\\
\dist{a_i}{b_j}{\spc{U}}&=\dist{a_i'}{b_j'}{\spc{U}'}
\\
\dist{b_i}{b_j}{\spc{U}}&=\dist{b_i'}{b_j'}{\spc{U}'}
\end{align*}
for all $i$ and $j$.

Let us extend the constructed distance preserving bijection defined by $a_i\leftrightarrow a_i'$ and $b_i\leftrightarrow b_i'$ continuousely to whole $\spc{U}$.
Observe that the image of this bijection is dense in $\spc{U}'$ therefore the constructed map $\spc{U}\to \spc{U}'$ is a bijection.
\qeds

Further the Urysohn space will be denoted by $\spc{U}$, and the $d$-Urysohn space will be denoted by $\spc{U}_d$.
Observe that \ref{thm:urysohn-unique} implies that the spaces $\spc{U}$ and $\spc{U}_d$ are finite-set homogeneous; that is,
\begin{itemize}
 \item any distance preserving map from a finite subset to to the whole space can be extended to an isometry.
\end{itemize}
It is unknown if there is a separable universal space that is finite-set homogeneous (this question appeared already in \cite{urysohn} and reappeared in \cite[p. 83]{gromov-2007} with a missing key word). 


\begin{thm}{Exercise}\label{ex:sphere-in-urysohn}
Let $S$ be a sphere of radius $\tfrac d2$ in $\spc{U}_d$;
that is, 
\[S=\set{x\in \spc{U}_d}{\dist{p}{x}{\spc{U}_d}=\tfrac d2}\]
for some point $p\in \spc{U}_d$.
Show that $S$ is isometric to $\spc{U}_d$.

Use it to show that $\spc{U}_d$ is not countable-set homogeneous;
that is there is an distance preserving map from a countable subset of $\spc{U}_d$ to $\spc{U}_d$ that can not be extended to an isometry $\spc{U}_d\to \spc{U}_d$.
\end{thm}


\begin{thm}{Exercise}
Modify the proofs of \ref{prop:completion-univeral} and \ref{thm:urysohn-unique} to prove the following theorem.
\end{thm}

\begin{thm}{Theorem}\label{thm:compact-homogeneous}
Let $K\subset \spc{U}$ be a compact set.
Show that any distance-preserving map $f\:K\to\spc{U}$ can be extended to 
an isometry of~$\spc{U}$.
\end{thm}











\chapter{Injective spaces}

\textit{Injective spaces} (also known as \textit{hyperconvex spaces}) are the metric analog of convex sets in the following sense:

\begin{thm}{Advanced exercise}\label{ex:conv-short}
Show that $A\subset \RR^n$ is a closed convex set if and only if for any  $B\subset \RR^n$ any short map $B\to A$ can be extended to a short map $\RR^n\to A$.
\end{thm}

\section{Definition}

\begin{thm}{Definition}\label{def:injective}
A metric space $\spc{Y}$ is called \index{injective space}\emph{injective} if for any metric space $\spc{X}$ and any of its subspaces $\spc{A}$
any short map $f\:\spc{A}\to \spc{Y}$ can be extended to a short map $F\:\spc{X}\to \spc{Y}$;
that is, $f=F|_{\spc{A}}$.
\end{thm}

\begin{thm}{Exercise}\label{ex:inj=complete-geodesic-contractible}
Show that any injective space is 
\begin{multicols}{3}

\begin{subthm}{ex:inj=complete-geodesic-contractible:complete}
complete,
\end{subthm}

\begin{subthm}{ex:inj=complete-geodesic-contractible:geodesic}
geodesic, and
\end{subthm}

\begin{subthm}{ex:inj=complete-geodesic-contractible:contractible}
contractible.
\end{subthm}

\end{multicols}

\end{thm}

\begin{thm}{Exercise}\label{ex:bicombing}
Let $\spc{Y}$ be an injective space.
Show that one can choose a geodesic path $\gamma_{x,y}\:[0,1]\to \spc{Y}$ from any $x\in \spc{Y}$ to any $y\in \spc{Y}$ such that
$\gamma_{x,y}(t)\equiv\gamma_{y,x}(1-t)$ and
\[\dist{\gamma_{x,y}(t)}{\gamma_{p,q}(t)}{\spc{Y}}\le (1-t)\cdot\dist{p}{x}{\spc{Y}}+t\cdot\dist{q}{y}{\spc{Y}}\]
for any $x,y,p,q\in \spc{Y}$.
\end{thm}

\begin{thm}{Exercise}\label{ex:injective-spaces}
Show that the following spaces are injective:
\begin{subthm}{ex:injective-spaces:R}
the real line;
\end{subthm}


\begin{subthm}{ex:injective-spaces:tree}
complete metric tree;
\end{subthm}

\begin{subthm}{ex:injective-spaces:ell-infty}
coordinate plane with the metric induced by the $\ell^\infty$-norm.
\end{subthm}

%\begin{subthm}{ex:injective-spaces:L-infty}$L^\infty([0,1])$.\end{subthm}%%%???

\end{thm}

\begin{thm}{Exercise}\label{ex:extr-ball}
Let $\spc{Y}$ be an injective space.

\begin{subthm}{ex:extr-ball:one}
Show that any closed ball in $\spc{Y}$ is injective.
\end{subthm}

\begin{subthm}{ex:extr-ball:many}
Show that intersection of an arbitrary collection of closed ball in $\spc{Y}$ is injective.
\end{subthm}

\end{thm}

\begin{thm}{Advanced exercise}\label{ex:extr-fixed}
Let $\spc{Y}$ be a bounded injective space.
Show that any short map $s\:\spc{Y}\to\spc{Y}$ has a fixed point. 
\end{thm}


\section{Admissible and extremal functions}

Let $\spc{X}$ be a metric space.
A function $r\:\spc{X}\to\RR$ is called \label{page:admissible function}\index{admissible function}\emph{admissible} if the following inequality
\[r(x)+r(y)\ge \dist{x}{y}{\spc{X}}\eqlbl{eq:admissible}\]
holds for any $x,y\in \spc{X}$.

\begin{thm}{Observation}\label{obs:admissible}

\begin{subthm}{obs:admissible:nonnegative}
Any admissible function is nonnegative.
\end{subthm}

\begin{subthm}{obs:admissible:balls}
If $\spc{X}$ is a geodesic space, then a function $r\:\spc{X}\to\RR$ is admissible if and only if 
\[\cBall[x,r(x)]\cap\cBall[y,r(y)]\ne \emptyset\]
for any $x,y\in \spc{X}$.
\end{subthm}
 
\end{thm}

\parit{Proof.} For \ref{SHORT.obs:admissible:nonnegative}, take $x=y$ in \ref{eq:admissible}.

Part \ref{SHORT.obs:admissible:balls} follows from the triangle inequality and the existence of a geodesic $[xy]$.
\qeds

A minimal admissible function will be called \label{page:extremal function}\index{extremal function}\emph{extremal}.
More precisely, an admissible function $r\:\spc{X}\to\RR$ is extremal 
if for any admissible function $s\:\spc{X}\to\RR$ we have
\[s\le r\quad\Longrightarrow\quad s=r.\]

Applying Zorn's lemma, we get the following.

\begin{thm}{Observation}\label{obs:extremal:below}
For any admissible function $s$ there is an extremal function $r$ such that $r\le s$.
\end{thm}

\begin{thm}{Lemma}\label{lem:+-c}
Let $r$ be an extremal function and $s$ an admissible function on a metric space $\spc{X}$.
Suppose that $r\ge s-c$ for some constant~$c$.
Then $r\le s+c$; in particular, $c\ge 0$.
\end{thm}

\parit{Proof.}
Note that if $c<0$, then $r>s$.
The latter is impossible since $r$ is extremal and $s$ is admissible.

Observe that the function $\bar r=\min\{\,r,s+c\,\}$ is admissible.
Indeed, choose $x,y\in \spc{X}$.
If $\bar r(x)=r(x)$ and $\bar r(y)=r(y)$, then 
\[\bar r(x)+\bar r(y)=r(x)+ r(y)\ge \dist{x}{y}{}.\]
Further, if $\bar r(x)=s(x)+c$, then 
\begin{align*}
\bar r(x)+\bar r(y)&\ge [s(x)+c]+ [s(y)-c]= 
\\
&=s(x)+s(y) \ge 
\\
&\ge\dist{x}{y}{}.
\end{align*}

Since $r$ is extremal, we have $r=\bar r$;
that is, $r\le s+c$.
\qeds

\begin{thm}{Observations}\label{obs:extremal}
Let $\spc{X}$ be a metric space.

\begin{subthm}{obs:extremal:distfun}
For any point $p\in\spc{X}$
the distance function $r\z=\distfun_p$ is extremal.
\end{subthm}

\begin{subthm}{lem:extremal-lipschitz}
Any extremal function $r$ on $\spc{X}$ is \index{1-Lipschitz function}\emph{1-Lipschitz};
that is,
\[|r(p)-r(q)|\le \dist{p}{q}{}\]
for any $p,q\in\spc{X}$.
In other words, any extremal function is an extension function; see the definition in \ref{sec:Extension property}.
\end{subthm}

\begin{subthm}{lem:opposite}
An admissible function $r$ on $\spc{X}$ is extremal if and only if
for any point $p\in\spc{X}$ and any $\delta>0$, there is a point $q\in \spc{X}$
such that 
\[r(p)+r(q)<\dist{p}{q}{\spc{X}}+\delta.\]
\end{subthm}

\begin{subthm}{lem:opposite-compact}
Suppose $\spc{X}$ is compact.
Then an admissible function $r$ on $\spc{X}$ is extremal if and only if
for any point $p\in\spc{X}$ there is a point $q\in \spc{X}$
such that 
\[r(p)+r(q)=\dist{p}{q}{\spc{X}}.\]
\end{subthm}

\end{thm}

\parit{Proof; \ref{SHORT.obs:extremal:distfun}.}
By the triangle inequality, \ref{eq:admissible} holds;
that is, $r=\distfun_p$ is an admissible function.

Further, if $s\le r$ is another admissible function, then $s(p)=0$ and \ref{eq:admissible} implies that $s(x)\z\ge\dist{p}{x}{}$.
Whence $s=r$.

\parit{\ref{SHORT.lem:extremal-lipschitz}.}
By \ref{SHORT.obs:extremal:distfun}, $\distfun_p$ is admissible.
Since $r$ is admissible, we have that
\[r\ge \distfun_p-r(p).\]
Since $r$ is extremal, \ref{lem:+-c} implies that
\[r\le \distfun_p+r(p),\]
or, equivalently,
\[r(q)-r(p)\le \dist{p}{q}{}\]
for any $p,q\in\spc{X}$.
The same way we can show that
$r(p)-r(q)\le \dist{p}{q}{}$.
Whence the statement follows.

\parit{\ref{SHORT.lem:opposite}.}
Assume $r$ is extremal.
Arguing by contradiction, assume there is $\delta>0$ such that
\[r(q)\ge \distfun_p(q)-r(p)+\delta\]
for any $q$.
By \ref{SHORT.obs:extremal:distfun}, $\distfun_p$ is extremal; in particular, admissible.
Therefore \ref{lem:+-c} implies that
\[r(q)\le \distfun_p(q)+r(p)-\delta\]
for any $q$.
Taking $q=p$, we get $r(p)\le r(p)-\delta$, a contradiction.

Now suppose $r$ is not extremal; that is, there is an admissible function $s\le r$ such that $r(p)-s(p)=\delta>0$ for some $p$.
Then, for any $q$, we have
\[r(p)+r(q)\ge s(p)+s(q)+\delta\ge \dist{p}{q}{\spc{X}}+\delta\]
--- a contradiction.

\parit{\ref{SHORT.lem:opposite-compact}.}
The if part follows from \ref{SHORT.lem:opposite}.

Denote by $q_n$ the point provided by \ref{SHORT.lem:opposite} for $\delta=\tfrac1n$.
Let $q$ be a partial limit of $q_n$. 
Then 
\[r(p)+r(q)\le\dist{p}{q}{\spc{X}}.\]
Since $r$ is admissible, the opposite inequality holds;
whence the only-if part follows.
\qeds

\begin{thm}{Exercise}\label{ex:circle}
Consider the unit circle $\mathbb{S}^1=\set{(x,y)}{x^2+y^2=1}$ in the plane with induced length metric.
Show that $r\:\mathbb{S}^1\to\RR$ is extremal if and only if it is 1-Lipschitz and 
\[r(p)+r(-p)=\pi\] for any $p\in\mathbb{S}^1$.
\end{thm}

\begin{thm}{Exercise}\label{ex:retraction}
Given a real-valued function $s$ on a metric space $\spc{X}$,
consider the function
\[s^*(x)=\sup\set{\dist{z}{y}{\spc{X}}-s(y)}{y\in \spc{X}}\]
Show that if $s$ is admissible then so is $\tfrac12\cdot(s+s^*)$.
\end{thm}

\section{Equivalent conditions}

\begin{thm}{Theorem}\label{thm:injective=hyperconvex}
For any metric space $\spc{Y}$ the following condition are equivalent:

\begin{subthm}{thm:injective=hyperconvex:injective}
$\spc{Y}$ is injective
\end{subthm}


\begin{subthm}{thm:injective=hyperconvex:extremal}
If $r\:\spc{Y}\to\RR$ is an extremal function, then there is a point $p\in \spc{Y}$ such that 
\[\dist{p}{x}{}\le r(x)\]
for any $x\in \spc{Y}$.
\end{subthm}

\begin{subthm}{thm:injective=hyperconvex:balls}
$\spc{Y}$ is \index{hyperconvex space}\emph{hyperconvex};
that is, if $\set{\cBall[x_\alpha,r_\alpha]}{\alpha\in\IndexSet}$ is a family of closed balls in $\spc{Y}$ such that 
 \[r_\alpha+r_\beta\ge \dist{x_\alpha}{x_\beta}{}\]
 for any $\alpha,\beta\in \IndexSet$, then all the balls in the family $\{\cBall[x_\alpha,r_\alpha]\}_{\alpha\in\IndexSet}$ have a common point.
\end{subthm}

\end{thm}

\parit{Proof.} We will prove implications 
\ref{SHORT.thm:injective=hyperconvex:injective}$\Rightarrow$\ref{SHORT.thm:injective=hyperconvex:extremal}$\Rightarrow$\ref{SHORT.thm:injective=hyperconvex:balls}$\Rightarrow$\ref{SHORT.thm:injective=hyperconvex:injective}.

\parit{\ref{SHORT.thm:injective=hyperconvex:injective}$\Rightarrow$\ref{SHORT.thm:injective=hyperconvex:extremal}.}
Let us apply the definition of injective space to a one-point extension of $\spc{Y}$.
It follows that for any extension function $r\:\spc{Y}\to\RR$ there is a point $p\in \spc{Y}$ such that 
\[\dist{p}{x}{}\le r(x)\]
for any $x\in \spc{Y}$.
By \ref{lem:extremal-lipschitz}, any extremal function is an extension function, whence the implication follows.

\parit{\ref{SHORT.thm:injective=hyperconvex:extremal}$\Rightarrow$\ref{SHORT.thm:injective=hyperconvex:balls}.}
By \ref{obs:admissible:balls}, part \ref{SHORT.thm:injective=hyperconvex:balls} is equivalent to the following statement:
\begin{itemize}
 \item If $r\:\spc{Y}\to\RR$ is an admissible function, then there is a point $p\in \spc{Y}$ such that 
\[\dist{p}{x}{}\le r(x)\eqlbl{eq:|p-x|=<r(x)}\]
for any $x\in \spc{Y}$.
\end{itemize}
Indeed, set $r(x)\df\inf\set{r_\alpha}{x_\alpha=x}$.
(If $x_\alpha\ne x$ for any $\alpha$, then $r(x)=\infty$.)
The condition in \ref{SHORT.thm:injective=hyperconvex:balls} implies that $r$ is admissible.
It remains to observe that $p\in \cBall[x_\alpha,r_\alpha]$ for every $\alpha$ if and only if \ref{eq:|p-x|=<r(x)} holds.

By \ref{obs:extremal:below}, for any admissible function $r$ there is an extremal function $\bar r\le r$;
hence \ref{SHORT.thm:injective=hyperconvex:extremal}$\Rightarrow$\ref{SHORT.thm:injective=hyperconvex:balls}.

\parit{\ref{SHORT.thm:injective=hyperconvex:balls}$\Rightarrow$\ref{SHORT.thm:injective=hyperconvex:injective}.}
Arguing by contradiction, suppose $\spc{Y}$ is not injective;
that is, there is a metric space $\spc{X}$ with a subset $\spc{A}$
such that a short map $f\:\spc{A}\to \spc{Y}$ cannot be extended to a short map $F\:\spc{X}\to \spc{Y}$.
By Zorn's lemma, we may assume that $\spc{A}$ is a maximal subset; that is, the domain of $f$ cannot be enlarged by a single point.%
\footnote{In this case, $\spc{A}$ must be closed, but we will not use it.}

Fix a point $p$ in the complement $\spc{X}\setminus \spc{A}$.
To extend $f$ to $p$, we need to choose $f(p)$ in the intersection of the balls 
$\cBall[f(x),r(x)]$, where $r(x)=\dist{p}{x}{}$.
Therefore, this intersection for all $x\in \spc{A}$ has to be empty.

Since $f$ is short, we have that 
\begin{align*}
r(x)+r(y)&\ge \dist{x}{y}{\spc{X}}\ge
\\
&\ge \dist{f(x)}{f(y)}{\spc{Y}}.
\end{align*}
Therefore, by \ref{SHORT.thm:injective=hyperconvex:balls} the balls 
$\cBall[f(x),r(x)]$ have a common point --- a contradiction. 
\qeds

\begin{thm}{Exercise}\label{ex:one-point-gluing}
Suppose a length space $\spc{W}$ has two subspaces $\spc{X}$ and $\spc{Y}$ such that $\spc{X}\cup\spc{Y}=\spc{W}$ and $\spc{X}\cap\spc{Y}$ is a one-point set.
Assume $\spc{X}$ and $\spc{Y}$ are injective.
Show that  $\spc{W}$ is injective
\end{thm}

\begin{thm}{Exercise}\label{ex:Rm-ell-infty}
Show that a $m$-dimensional normed space is injective if and only if it is isometric to $\RR^m$ with the norm
\[|(x_1,\dots,x_m)|=\max_i\{\,|x_i|\,\}.\]
\end{thm}


\begin{thm}{Exercise}\label{ex:urysohn-hyperconvex}
Show that the $d$-Urysohn space is {}\emph{finitely hyperconvex} but not {}\emph{countably hyperconvex};
that is, the condition in \ref{thm:injective=hyperconvex:balls} holds for any finite family of balls, but may not hold for a countable family.
Conclude that the $d$-Urysohn space is not injective.

Try to do the same for the Urysohn space.
\end{thm}

\section{Space of extremal functions}
\label{sec:extremal-functions}

Let $\spc{X}$ be a metric space.
Consider the space $\Inj \spc{X}$ of extremal functions on $\spc{X}$ equipped with sup-norm; \label{page:InjX}
that is,
\[\dist{f}{g}{\Inj \spc{X}}\df\sup\set{|f(x)-g(x)|}{x\in \spc{X}}.\]

Recall that by \ref{obs:extremal:distfun}, any distance function is extremal.
It follows that the map $x\mapsto \distfun_x$ produces a distance-preserving embedding $\spc{X}\hookrightarrow\Inj \spc{X}$.
So we can (and will) treat $\spc{X}$ as a subspace of $\Inj \spc{X}$,
or, equivalently, $\Inj \spc{X}$ as an extension of $\spc{X}$.

Since any extremal function is 1-Lipschitz, for any $f\in \Inj \spc{X}$ and $p\in \spc{X}$, we have that
$f(x)\le f(p)+\distfun_p(x)$.
By \ref{lem:+-c}, we also get $f(x)\ge -f(p)+\distfun_p(x)$.
Therefore
\[
\begin{aligned}
\dist{f}{p}{\Inj \spc{X}}&=\sup\set{|f(x)-\distfun_p(x)|}{x\in \spc{X}}=
\\
&=f(p).
\end{aligned}
\eqlbl{eq:f(p)=|f-p|}
\]
In particular, the statement in \ref{lem:opposite} can be written as 
\[\dist{f}{p}{\Inj\spc{X}}+\dist{f}{q}{\Inj\spc{X}}<\dist{p}{q}{\Inj\spc{X}}+\delta.\]

\begin{thm}{Exercise}\label{ex:Inj(compact)}
Let $\spc{X}$ be a metric space.
Show that $\Inj\spc{X}$ is compact if and only if so is $\spc{X}$.
\end{thm}

\begin{thm}{Exercise}\label{ex:tripod+square}
Describe the set of all extremal functions on a metric space $\spc{X}$ and the metric space $\Inj \spc{X}$ in each of the following cases:

\begin{subthm}{ex:tripod+square:2}
$\spc{X}$ is a metric space with exactly two points $v,w$ on distance 1 from each other.
\end{subthm}


\begin{subthm}{ex:tripod+square:tripod} 
$\spc{X}$ is a metric space with exactly three points $a,b,c$ such that 
\[\dist{a}{b}{\spc{X}}=\dist{b}{c}{\spc{X}}=\dist{c}{a}{\spc{X}}=1.\]
\end{subthm}

\begin{subthm}{ex:tripod+square:square}
$\spc{X}$ is  a metric space with exactly four points $p,q,x,y$ such that 
\[\dist{p}{x}{\spc{X}}=\dist{p}{y}{\spc{X}}=\dist{q}{x}{\spc{X}}=\dist{q}{y}{\spc{X}}=1\]
and
\[\dist{p}{q}{\spc{X}}=\dist{x}{y}{\spc{X}}=2.\]
\end{subthm}

\end{thm}

\begin{thm}{Exercise}\label{ex:kur-inj}
Assume $\spc{X}$ is a compact metric space.
Recall that the map $x\mapsto \distfun_x$ gives an isometric embedding $\spc{X}\hookrightarrow\ell^\infty(\spc{X})$; so we can think that $\spc{X}$ is a subset of $\ell^\infty(\spc{X})$.

Given two points $x,y\in \spc{X}$, denote by $G_{x,y}$ the union of all geodesics from $x$ to $y$ in $\ell^\infty(\spc{X})$.
Show that $\Inj\spc{X}$ is isometric to
\[G=\bigcap_{x\in \spc{X}}\left(\bigcup_{y\in \spc{X}}G_{x,y}\right).\]

\end{thm}


\begin{thm}{Proposition}\label{prop:InjX-is-injective}
For any metric space $\spc{X}$, its extension $\Inj\spc{X}$ is  injective.
\end{thm}

\begin{thm}{Lemma}\label{lem:r|X-extremal}
Let $\spc{X}$ be a metric space.
Suppose $r\in \Inj(\Inj \spc{X})$;
that is, $r$ is an extremal function on $\Inj \spc{X}$.
Then $r|_\spc{X}\in \Inj \spc{X}$;
that is, the restriction of $r$ to $\spc{X}$ is an extremal function.
\end{thm}

\parit{Proof.}
Arguing by contradiction, suppose that there is an admissible function $s\:\spc{X}\to \RR$ such that $s(x)\le r(x)$ for any $x\in\spc{X}$ and $s(p)\z< r(p)$ for some point $p\in\spc{X}$.
Consider another function $\bar r\:\Inj \spc{X}\to\RR$ such that $\bar r(f)\df r(f)$ if $f\ne p$ and $\bar r(p)\df s(p)$.

Let us show that $\bar r$ is admissible; that is, 
\[\dist{f}{g}{\Inj \spc{X}}\le\bar r(f)+\bar r(g)
\eqlbl{r-admissible}\]
for any $f,g\in \Inj \spc{X}$.

Since $r$ is admissible and $\bar r= r$ on $(\Inj \spc{X})\setminus \{p\}$, it is sufficient to prove \ref{r-admissible} if $f\ne g=p$.
By \ref{eq:f(p)=|f-p|}, we have $\dist{f}{p}{\Inj \spc{X}}=f(p)$.
Therefore, \ref{r-admissible} boils down to the following inequality
\[r(f)+s(p)\ge f(p).\eqlbl{eq:r(f)+s(p)>=f(p)}\]
for any $f\in\Inj \spc{X}$.

Fix small $\delta>0$. 
Let $q\in\spc{X}$ be the point provided by \ref{lem:opposite}.
Then
\begin{align*}
r(f)+s(p)&\ge [r(f)-r(q)]+[r(q)+s(p)]\ge
\intertext{since $r$ is 1-Lipschitz, and $r(q)\ge s(q)$, we can continue}
&\ge -\dist{q}{f}{\Inj \spc{X}}+[s(q)+s(p)]\ge
\intertext{by \ref{eq:f(p)=|f-p|} and since $s$ is admissible}
&\ge -f(q)+\dist{p}{q}{}>
\intertext{and by \ref{lem:opposite}}
&> f(p)-\delta.
\end{align*}
Since $\delta>0$ is arbitrary, \ref{eq:r(f)+s(p)>=f(p)} and \ref{r-admissible} follow.

Summarizing: the function $\bar r$ is admissible, $\bar r\le r$ and $\bar r(p)<r(p)$;
that is, $r$ is not extremal --- a contradiction.
\qeds

\parit{Proof of \ref{prop:InjX-is-injective}.}
Choose a function $r\in\Inj(\Inj\spc{X})$.
By \ref{lem:r|X-extremal}, $s\z\df r|_{\spc{X}}\in \Inj\spc{X}$;
that is, $s$ is extremal.
By \ref{thm:injective=hyperconvex:extremal},
it is sufficient to show that  
\[r(f)\ge\dist{s}{f}{\Inj\spc{X}}
\eqlbl{eq:r(f)>=| r-f|}\]
for any $f\in\Inj\spc{X}$.

Since $r$ is $1$-Lipschitz (\ref{lem:extremal-lipschitz}) we have that
\[
s(x)-f(x)=r(x)-\dist{f}{x}{\Inj \spc{X}}\le r(f).
\]
for any $x\in\spc{X}$.
By \ref{lem:+-c},
$
s(x)-f(x)\ge -r(f)
$
for any $x\in\spc{X}$.
Whence \ref{eq:r(f)>=| r-f|} follows.
\qeds

\begin{thm}{Exercise}\label{ex:4-on-a-line}
Let $\spc{X}$ be a compact metric space.
Show that for any two points $f,g\in\Inj \spc{X}$ lie on a geodesic $[pq]$ with $p,q\in \spc{X}$.
\end{thm}

A metric space $\spc{X}$ is called \index{$\delta$-hyperbolic}\emph{$\delta$-hyperbolic} if 
\[\dist{p}{q}{}+\dist{x}{y}{}\le
\max\{\,\dist{p}{x}{}+\dist{q}{y}{},
\,
\dist{p}{y}{}+\dist{q}{x}{}\,\}+2\cdot\delta\]
for any $p,q,x,y\in \spc{X}$.

\begin{thm}{Advanced exercise}\label{ex:delta-hyp}
Show that $\Inj \spc{X}$ is $\delta$-hyperbolic if and only if $\spc{X}$ is.
\end{thm}


\section{Injective envelope}

An extension $\spc{E}$ of a metric space $\spc{X}$ will be called its \index{injective envelope}\emph{injective envelope} if $\spc{E}$ is an injective space, and there is no proper injective subspace of $\spc{E}$ that contains $\spc{X}$.

Two injective envelopes $e\:\spc{X}\hookrightarrow \spc{E}$ and $f\:\spc{X}\hookrightarrow \spc{F}$ are called  equivalent if there is an isometry $\iota\: \spc{E}\to\spc{F}$ such that $f=\iota\circ e$.

\begin{thm}{Theorem}\label{thm:inj-envelope}
For any metric space $\spc{X}$, its extension $\Inj\spc{X}$ is an injective envelope.

Moreover, any other injective envelope of $\spc{X}$ is equivalent to $\Inj\spc{X}$.
\end{thm}

\parit{Proof.} 
Suppose $S\subset \Inj\spc{X}$ is an injective subspace containing $\spc{X}$.
Since $S$ is injective, there is a short map $w\:\Inj\spc{X}\to S$ that fixes all points in $\spc{X}$.

Suppose that $w\:f\mapsto f'$; observe that $f(x)\ge f'(x)$ for any $x\in \spc{X}$.
Since $f$ is extremal, $f=f'$;
that is, $w$ is the identity map, and therefore $S=\Inj\spc{X}$.

Assume we have another injective envelope $e\:\spc{X}\hookrightarrow \spc{E}$.
Then there are short maps $v\:\spc{E}\to \Inj\spc{X}$ and $w\:\Inj\spc{X}\to \spc{E}$ such that $x=v\circ e(x)$ and $e(x)=w(x)$ for any $x\in\spc{X}$.
From above, the composition $v\circ w$ is the identity on $\Inj\spc{X}$.
In particular, $w$ is distance-preserving.

The composition $w\circ v\:\spc{E}\to \spc{E}$ is a short map that fixes points in $e(\spc{X})$.
Since $e\:\spc{X}\hookrightarrow \spc{E}$ is an injective envelope, the composition $w\circ v$ and therefore $w$ are onto.
Whence $w$ is an isometry.
\qeds

\begin{thm}{Exercise}\label{ex:d-p-inclusion}
Suppose $\spc{X}$ is a subspace of a metric space $\spc{U}$.
Show that the inclusion $\spc{X}\hookrightarrow\spc{U}$ can be extended to a distance-preserving inclusion $\Inj\spc{X}\hookrightarrow\Inj\spc{U}$.
\end{thm}


\section{Remarks}

Injective spaces were introduced by Nachman Aronszajn and Prom Panitchpakdi \cite{aronszajn-panitchpakdi}.
The injective envelope was introduced by John Isbell \cite{isbell}.
It was rediscovered a couple of times since then;
as a result, the injective envelope has many other names including \index{tight span}\emph{tight span} and \index{hyperconvex hull}\emph{hyperconvex hull}.

The following two exercise deals with ultrametric spaces which in some sense are dual to the injective spaces. 

Recall that if the following inequality
\[\dist{x}{z}{\spc{X}}
\le
\max\{\,\dist{x}{y}{\spc{X}},\dist{y}{z}{\spc{X}}\,\}\]
holds for any three points $x,y,z$ in a metric space $\spc{X}$,
then $\spc{X}$ is called an \index{ultrametric space}\emph{ultrametric space}.

\begin{thm}{Exercise}\label{ex:ultrametric}
Suppose that a metric space $\spc{X}$ satisfies the following property:
For any subspace $\spc{A}$ in $\spc{X}$ and any other metric space $\spc{Y}$, any short map $f\:\spc{A}\to \spc{Y}$ can be extended to a short map $F\:\spc{X}\to \spc{Y}$.

Show that $\spc{X}$ is an ultrametric space.
\end{thm}

A subspace $\spc{S}$ of a metric space $\spc{X}$ is called its \index{short retract}\emph{short retract} if there is a short map $\spc{X}\to \spc{S}$ that is the identity on $\spc{S}$.

\begin{thm}{Exercise}\label{ex:ultrametric-converse}
Show that any compact subspace $\spc{K}$ of an ultrametric space $\spc{X}$ is its short retract.

Construct an example of a complete ultrametric space $\spc{X}$ with a closed subset $Q$ that is not its short retract.
\end{thm}

The following exercise gives a sufficient condition for existence of a short extension.

\begin{thm}{Exercise}\label{ex:petrunin-stadler}
Let $\spc{X}$ and $\spc{Y}$ be metric spaces, $A\subset \spc{X}$, and $f\:A\z\to \spc{Y}$ be a short map.
Assume $\spc{Y}$ is compact and for any finite set $F\subset \spc{X}$ there is a short map $F\to \spc{Y}$ that agrees with $f$ on $F\cap A$.
Show that there is a short map $\spc{X}\to \spc{Y}$ that agrees with $f$ on $A$.
\end{thm}

\chapter{Space of sets}

\section{Hausdorff distance}

Let $\spc{X}$ be a metric space.
Given a subset $A\subset \spc{X}$,
consider the distance function to $A$
$$\distfun_A: \spc{X} \to [0,\infty)$$
defined as 
$$\distfun_A(x)
\df
\inf_{a\in A}\{\,\dist ax{\spc{X}}\,\}.$$

\begin{thm}{Definition}\label{def:hausdorff-convergence}
Let $A$ and $B$ be two compact subsets of a metric space $\spc{X}$.
Then the \index{Hausdorff distance}\emph{Hausdorff distance} between $A$ and $B$ is defined as 
$$|A-B|_{\Haus\spc{X}}
\df
\sup_{x\in \spc{X}}\{\,|\distfun_A(x)-\distfun_B(x)|\,\}.
$$

\end{thm}

The following observation gives a useful reformulation of the definition:

\begin{thm}{Observation}\label{obs:Haus-nbhds}
Suppose $A$ and $B$ be two compact subsets of a metric space $\spc{X}$.
Then $|A-B|_{\Haus\spc{X}}< R$ if and only if and only if 
$B$ lies in an $R$-neighborhood of $A$, 
and 
$A$ lies in an $R$-neighborhood of~$B$.
\end{thm}



Note that the set of all nonempty compact subsets of a metric space $\spc{X}$ equipped with the Hausdorff metric forms a metric space.
This new metric space will be denoted as $\Haus\spc{X}$.


\begin{thm}{Exercise}\label{ex:diam}
Let $\spc{X}$ be a metric space.
Given a subset $A\subset \spc{X}$ define its \index{diameter}\emph{diameter} as 
$$\diam A\df\sup_{a,b\in A} |a-b|.$$

Show that 
$$\diam\:\Haus\spc{X}\to \RR$$ 
is a \index{Lipschitz function}\emph{$2$-Lipschitz function};
that is,
\[|\diam A-\diam B|\le 2\cdot\dist{A}{B}{\Haus\spc{X}}\]
for any two compact nonempty sets $A,B\subset\spc{X}$.
\end{thm}


\begin{thm}{Exercise}\label{ex:Hausdorff-bry}
Let $A$ and $B$ be two compact subsets in the Euclidean plane $\RR^2$.
Assume $|A-B|_{\Haus\RR^2}<\eps$.

\begin{subthm}{ex:Hausdorff-bry:conv}
Show that $|\Conv A-\Conv B|_{\Haus\RR^2}<\eps$, where $\Conv A$ denoted the convex hull of $A$.
\end{subthm}
\begin{subthm}{ex:Hausdorff-bry:bry}
Is it true that
$|\partial A-\partial B|_{\Haus\RR^2}<\eps$,
where $\partial A$ denotes the boundary of $A$.

Does the converse hold? That is, assume $A$ and $B$ be two compact subsets in $\RR^2$
and $|\partial A-\partial B|_{\Haus\RR^2}<\eps$; 
is it true that $|A-B|_{\Haus\RR^2}\z<\eps$?
\end{subthm}

\end{thm}

Note that part \ref{SHORT.ex:Hausdorff-bry:conv} implies that $A\mapsto \Conv A$ defines a short map $\Haus\RR^2\to \Haus\RR^2$. 

\begin{thm}{Exercise}\label{ex:Haus-func}
Let $A$ and $B$ be two compact subsets in metric space~$\spc{X}$.
Show that 
\[\dist{A}{B}{\Haus\spc{X}}=\sup_f\, \{\,\max_{a\in A}\{f(a)\}-\max_{b\in B}\{f(b)\,\},\]
where the least upper bound is taken for all $1$-Lipschitz functions $f$.

\end{thm}

\begin{thm}{Advanced exercise}\label{ex:H-sections}
\begin{subthm}{ex:H-sections:S}
Construct a family of compact sets $C_t\subset\mathbb{S}^1$, $t\z\in [0,1]$ that is continuous in the Hausdorff topology, 
but does not admit a {}\emph{section}.
That is, there is no path $c\:[0,1]\to \mathbb{S}^1$ such that $c(t)\in C_t$ for all $t$.
\end{subthm}

\begin{subthm}{ex:H-sections:R}
Show that any family of compact sets $C_t\subset\RR^1$, $t\z\in [0,1]$ that is continuous in the Hausdorff topology, 
admits a {}\emph{section}.
That is, there is path $c\:[0,1]\to \RR^1$ such that $c(t)\in C_t$ for all $t$.
\end{subthm}

\end{thm}

\section{Hausdorff convergence}

\begin{thm}{Blaschke selection theorem}\label{thm:compact+Hausdorff}
A metric space $\spc{X}$ is compact if and only if
so is $\Haus\spc{X}$.
\end{thm}

The Hausdorff metric can be used to define convergence.
Namely, suppose $K_1,K_2,\dots$, and $K_\infty$ are compact sets in a metric space $\spc{X}$.
If $|K_\infty-K_n|_{\Haus\spc{X}}\to0$ as $n\to\infty$, then we say that 
the sequence $K_n$ {}\emph{converges} to $K_\infty$ \index{convergence in the sense of Hausdorff}\emph{in the sense of Hausdorff};
or we can say that $K_\infty$ is {}\emph{Hausdorff limit} of the sequence $K_n$.

Note that the theorem implies that from any sequence of compact sets in $\spc{X}$ one can select a subsequence that converges in the sense of Hausdorff; 
for that reason, it is called a \textit{selection} theorem. 

\parit{Proof; if part.}
Consider the map $\iota$ that sends each point $x\in \spc{X}$ to the one-point subset $\{x\}$ of $\spc{X}$.
Note that $\iota\:\spc{X}\to \Haus\spc{X}$ is distance-preserving.

Suppose that $A\subset \spc{X}$.
Note that $\diam A=0$ if and only if $A$ is a one-point set.
By \ref{ex:diam}, $\iota(\spc{X})$ is a closed subset of the compact space $\Haus\spc{X}$.
It follows that $\iota(\spc{X})$, and therefore $\spc{X}$, are compact.
\qeds

Since the map $\iota$ above is distance-preserving, we can and will consider $\spc{X}$ as a subspace of $\Haus\spc{X}$.

\begin{thm}{Exercise}\label{ex:haus-contractible}
Let $\spc{X}$ be a bounded length space with.
Suppose that there is a short retraction $\Haus\spc{X}\to \spc{X}$.
Show that $\spc{X}$ is contractible.
\end{thm}


To prove the only-if part we will need the following two lemmas.

\begin{thm}{Monotone convergence}\label{lem:decreasing-converges}
Let $K_1\supset K_2\supset\dots$ be a nested sequence of nonempty compact sets in a metric space $\spc{X}$.
Then $K_\infty\z=\bigcap_n K_n$ is the Hausdorff limit of $K_n$;
that is, $|K_\infty-K_n|_{\Haus\spc{X}}\to0$ as $n\to\infty$.
\end{thm}

\parit{Proof.}
By finite intersection property, $K_\infty$ is a nonempty compact set.

If the assertion were false, then there is $\eps>0$ such that for each $n$ 
one can choose $x_n\in K_n$
such that $\distfun_{K_\infty}(x_n)\ge\eps$.
Note that $x_n\in K_1$ for each $n$.
Since $K_1$ is compact, 
there is 
a \index{partial limit}\emph{partial limit}%
\footnote{Partial limit is a limit of a subsequence.}
 $x_\infty$ of $x_n$.
Clearly $\distfun_{K_\infty}(x_\infty)\ge \eps$.

On the other hand, since $K_n$ is closed and $x_m\in K_n$ for $m\ge n$,
we get $x_\infty\in K_n$ for each $n$.
It follows that $x_\infty\in K_\infty$ and therefore $\distfun_{K_\infty}(x_\infty)=0$ ---
a contradiction.\qeds


\begin{thm}{Lemma}\label{lem:complete+Hausdorff}
If $\spc{X}$ is a compact metric space, then $\Haus\spc{X}$
is complete.
\end{thm}

\parit{Proof.}
Let $(Q_n)$ be a Cauchy sequence in $\Haus\spc{X}$.
Passing to a subsequence of $Q_n$ we may assume that 
$$|Q_n-Q_{n+1}|_{\Haus\spc{X}}\le \tfrac1{10^n}\eqlbl{eq:eps=1/10}$$
for each $n$.

Denote by $K_n$ the closed $\tfrac1{10^n}$-neighborhood of $Q_n$;
that is,
\begin{align*}
K_n&= \set{x\in \spc{X}}{\distfun_{Q_n}(x)\le \tfrac1{10^n}}
\end{align*}
Since $\spc{X}$ is compact so is each $K_n$.

By \ref{obs:Haus-nbhds}, $|Q_n-K_n|_{\Haus\spc{X}}\le \tfrac1{10^n}$.
From \ref{eq:eps=1/10}, we get
$K_n\supset K_{n+1}$ 
for each $n$.
Set 
$$K_\infty=\bigcap_{n=1}^\infty K_n.$$
By the monotone convergence (\ref{lem:decreasing-converges}),
 $|K_n-K_\infty|_{\Haus\spc{X}}\to 0$ as $n\to\infty$.
Since $|Q_n-K_n|_{\Haus\spc{X}}\le \tfrac1{10^n}$, we get $|Q_n-K_\infty|_{\Haus\spc{X}}\to 0$ as $n\to\infty$ --- hence the lemma.
\qeds

\begin{thm}{Exercise}\label{ex:closure-union}
Let $\spc{X}$ be a complete metric space and $K_1,K_2,\dots$ be a sequence of compact sets 
that converges in the sense of Hausdorff.
Show that the union $K_1\cup K_2\cup\dots$ has compact closure.

Use this statement to show that in Lemma~\ref{lem:complete+Hausdorff} compactness of $\spc{X}$ can be exchanged to completeness.
\end{thm}

\parit{Proof of only-if part in \ref{thm:compact+Hausdorff}.}
According to Lemma~\ref{lem:complete+Hausdorff},
$\Haus\spc{X}$ is complete.
It remains to show that $\Haus\spc{X}$ is totally bounded (\ref{totally-bounded});
that is, given $\eps>0$ there is a finite $\eps$-net in $\Haus\spc{X}$.

Choose a finite $\eps$-net $A$ in $\spc{X}$.
Denote by $B$ the set of all subsets of $A$.
Note that  $B$ is a finite set in $\Haus\spc{X}$.
For each compact set $K\subset \spc{X}$, consider the subset $K'$ of all points $a\in A$
such that $\distfun_K(a)\le \eps$.
Observe that $K' \in B$ and $|K-K'|_{\Haus\spc{X}}\le\eps$.
In other words, $B$ is a finite $\eps$-net in $\Haus\spc{X}$.
\qeds

\begin{thm}{Exercise}\label{ex:Haus-length}
Let $\spc{X}$ be a complete metric space.
Show that $\spc{X}$ is a length space if and only if so is $\Haus\spc{X}$.
\end{thm}

\section{An application}

The following statement is called \index{isoperimetric inequality}\emph{isoperimetric inequality in the plane}.

\begin{thm}{Theorem}\label{thm:isoperimetric}
Among the plane figures bounded by closed curves of length at most $\ell$, the round disk has the maximal area.
\end{thm}

In this section, we will sketch a proof of the isoperimetric inequality that uses the Hausdorff convergence.
It is based on the following exercise.

\begin{thm}{Exercise}\label{ex:Huas-perimeter-area}
Let $\spc{C}$ be a subspace of $\Haus\RR^2$ formed by all compact convex subsets in $\RR^2$.
Show that perimeter\footnote{If the set degenerates to a line segment of length $\ell$, then its perimeter is defined as $2\cdot \ell$.} and area are continuous on~$\spc{C}$.
That is, if a sequence of convex compact plane sets $X_n$ converges to $X_\infty$ in the sense of Hausdorff, then 
\[\perim X_n\to \perim X_\infty\quad\text{and}\quad\area X_n\to\area X_\infty\]
as $n\to\infty$.
\end{thm}

\parit{Semiproof of \ref{thm:isoperimetric}.}
It is sufficient to consider only convex figures of the given perimeter; if a figure is not convex, pass to its convex hull and observe that it has a larger area and smaller perimeter.


Note that the selection theorem (\ref{thm:compact+Hausdorff}) together with the exercise imply the existence of figure $D$ with perimeter $\ell$ and maximal area.

It remains to show that $D$ is a round disk.
This is a problem in elementary geometry.

Let us cut $D$ along a chord $[ab]$ into two lenses, $L_1$ and $L_2$.
Denote by $L_1'$ the reflection of $L_1$ across the perpendicular bisector of $[ab]$.
Note that $D$ and $D'=L_1'\cup L_2$ have the same perimeter and area.
That is, $D'$ has perimeter $\ell$ and maximal possible area;
in particular, $D'$ is convex.

The following exercise will finish the proof.
\qeds

{

\begin{wrapfigure}{o}{57 mm}
\vskip-5mm
\centering
\includegraphics{mppics/pic-405}
\end{wrapfigure}

\begin{thm}{Exercise}\label{ex:round-disc}
Suppose $D$ is a convex figure such that for any chord $[ab]$ of $D$ the above construction produces a convex figure $D'$.
Show that $D$ is a round disk.
\end{thm}


}

Another popular way to prove that $D$ is a round disk is given by the so-called {}\emph{Steiner's 4-joint method} \cite{blaschke}.

\section{Remarks}\label{sec:H-variation}

It seems that Hausdorff convergence was first introduced by Felix Hausdorff~\cite{hausdorff}.
A couple of years later an equivalent definition was given by Wilhelm Blaschke~\cite{blaschke}.

The following refinement was introduced by  Zdeněk Frolík \cite{frolik},
later it was rediscovered by Robert Wijsman~\cite{wijsman}.  
This refinement is also called \index{Hausdorff convergence}\emph{Hausdorff convergence};
in fact, it takes an intermediate place between the original Hausdorff convergence and {}\emph{closed convergence}, also introduced by Hausdorff in \cite{hausdorff}.

\begin{thm}{Definition}\label{def:gen-Haus-conv}
Let $A_1,A_2,\dots$ be a sequence of closed sets in a metric space $\spc{X}$.
We say that the sequence $A_n$ converges to a closed set $A_\infty$ in the sense of Hausdorff if for any $x\in\spc{X}$, we have
$\distfun_{A_n}(x)\z\to \distfun_{A_\infty}(x)$ as $n\to\infty$.
\end{thm}

For example, suppose $\spc{X}$ is the Euclidean plane and $A_n$ is the circle with radius $n$ and center at the point $(n,0)$.
If we use the standard definition (\ref{def:hausdorff-convergence}), then the sequence $(A_n)$ diverges, but it converges to the $y$-axis in the sense of Definition~\ref{def:gen-Haus-conv}.

Further, consider the sequence of one-point sets $B_n=\{(n,0)\}$ in the Euclidean plane.
It diverges in the sense of the standard definition, but, in the sense of \ref{def:gen-Haus-conv}, it converges to the empty set;
indeed, for any point $x$ we have $\distfun_{B_n}(x)\to\infty$ as $n\to \infty$ and $\distfun_{\emptyset}(x)= \infty$ for any~$x$.

The following exercise is analogous to the Blaschke selection theorem (\ref{thm:compact+Hausdorff}) for the modified Hausdorff convergence.

\begin{thm}{Exercise}\label{ex:generalized-selection}
Let $\spc{X}$ be a proper metric space
and $A_1,A_2,\dots$ be a sequence of closed sets in~$\spc{X}$.
Show that the sequence  $A_1,A_2,\dots$ has a convergent subsequence in the sense of Definition~\ref{def:gen-Haus-conv}.
\end{thm}

\chapter{Space of spaces}

\section{Gromov--Hausdorff metric}

The goal of this section is to cook up a metric space out of all compact metric spaces.
More precisely, we want to define the so-called  Gromov--Hausdorff metric on the set of \textit{isometry classes} of compact metric spaces.
(Being isometric is an equivalence relation, 
and an \index{isometry class}\emph{isometry class} is an equivalence class with respect to this relation.)

The obtained metric space will be denoted by $\GH$.
Given two metric spaces $\spc{X}$ and $\spc{Y}$,
denote by $[\spc{X}]$ and $[\spc{Y}]$ their isometry classes;
that is, $\spc{X}'\in [\spc{X}]$ if and only if $\spc{X}'\iso \spc{X}$.
Pedantically, the Gromov--Hausdorff distance from $[\spc{X}]$ 
to $[\spc{Y}]$ should be denoted as $|[\spc{X}]-[\spc{Y}]|_{\GH}$;
but we will write it as $|\spc{X}\z-\spc{Y}|_{\GH}$ and say (not quite correctly) that 
\textit{$|\spc{X}\z-\spc{Y}|_{\GH}$ is the Gromov--Hausdorff distance from  $\spc{X}$ 
to  $\spc{Y}$}.
In other words, from now on the term \textit{metric space} might also stand for its \textit{isometry class}.

The metric on $\GH$ is defined as the maximal metric such that \textit{the distance between subspaces in a metric space is not greater than the Hausdorff distance between them}.
Here is a formal definition:

\begin{thm}{Definition}\label{def:GH}
The \index{Gromov--Hausdorff distance}\emph{Gromov--Hausdorff distance} $|\spc{X}-\spc{Y}|_{\GH}$ between compact metric spaces $\spc{X}$ and $\spc{Y}$
is defined by the following
relation.
 
Given  $r > 0$, we have that $|\spc{X}-\spc{Y}|_{\GH} < r$ if and only if there exists a metric
space $\spc{W}$ and subspaces $\spc{X}'$ and $\spc{Y}'$ in $\spc{W}$ that are isometric to $\spc{X}$ and $\spc{Y}$
respectively such that $|\spc{X}'-\spc{Y}'|_{\Haus\spc{W}} < r$. 
(Here $|\spc{X}'-\spc{Y}'|_{\Haus\spc{W}}$ denotes the Hausdorff distance between sets $\spc{X}'$ and $\spc{Y}'$ in $\spc{W}$.)
\end{thm}

\begin{thm}{Theorem}\label{thm:GH-is-a-metric}
The set of isometry classes of compact metric spaces equipped with Gromov--Hausdorff metric forms a metric space (which is denoted by $\GH$).

In other words, for arbitrary compact metric spaces $\spc{X}$, $\spc{Y}$ and $\spc{Z}$ the following conditions hold:

\begin{subthm}{GH-1} $|\spc{X}-\spc{Y}|_{\GH}\ge 0$;
\end{subthm}

\begin{subthm}{GH-2} $|\spc{X}-\spc{Y}|_{\GH}=0$ if and only if $\spc{X}$ is isometric to $\spc{Y}$;
\end{subthm}

\begin{subthm}{GH-3} $|\spc{X}-\spc{Y}|_{\GH}=|\spc{Y}-\spc{X}|_{\GH}$;
\end{subthm}

\begin{subthm}{GH-4} $|\spc{X}-\spc{Y}|_{\GH}+|\spc{Y}-\spc{Z}|_{\GH}\ge |\spc{X}-\spc{Z}|_{\GH}$.
\end{subthm}
\end{thm}


Note that \ref{SHORT.GH-1}, \ref{SHORT.GH-3},
and the ``if''-part of \ref{SHORT.GH-2} follow directly from Definition \ref{def:GH}.
Part \ref{SHORT.GH-4} will be proved in Section~\ref{sec:GH-approx}.
The ``only-if''-part of \ref{SHORT.GH-2} will be proved in Section~\ref{sec:extfun=GH}.

Recall that $a\cdot\spc{X}$ denotes $\spc{X}$ \index{rescaled space}\emph{rescaled} by factor $a>0$;
that is, $a\cdot\spc{X}$ is a metric space with the underlying set of $\spc{X}$ and the metric defined by
\[\dist{x}{y}{a\cdot\spc{X}}\df a\cdot\dist{x}{y}{\spc{X}}.\]

\begin{thm}{Exercise}\label{ex:d_GH-and-diam}
Let $\spc{X}$ be a compact metric space,
$\spc{O}$ be the one-point metric space.

Prove that 

\begin{subthm}{ex:d_GH-and-diam:point}
$|\spc{X}-\spc{O}|_{\GH}=\tfrac12\cdot \diam \spc{X}.$

\end{subthm}

\begin{subthm}{ex:d_GH-and-diam:scale}
$|a\cdot\spc{X}-b\cdot \spc{X}|_{\GH}=\tfrac12\cdot|a-b|\cdot\diam\spc{X}.$
\end{subthm}

\begin{subthm}{ex:d_GH-and-diam:isometry}
$\iota[\spc{O}]=[\spc{O}]$ for any isometry $\iota\:\GH\to\GH$.
\end{subthm}


\end{thm}




\begin{thm}{Exercise}\label{ex:GH<H}
Find subsets $A,B\subset\RR^2$ such that 
\[|A-B|_{\GH}>|A-\iota(B)|_{\Haus\RR^2}\]
for any isometry $\iota$ of $\RR^2$.
\end{thm}


\begin{thm}{Exercise}\label{ex:rectangle}
Let $\spc{A}_r$ be a rectangle $1$ by $r$ in the Euclidean plane 
and $\spc{B}_r$ be a closed line interval of length $r$.
Show that 
\[|\spc{A}_r-\spc{B}_r|_{\GH}>\tfrac1{10}\]
for all large $r$.
\end{thm}

\begin{thm}{Advanced exercise}\label{ex:GH-inj}
Let $\spc{X}$ and $\spc{Y}$ be compact metric spaces;
denote by $\hat{\spc{X}}$ and $\hat{\spc{Y}}$ their injective envelopes (see \ref{sec:extremal-functions}).
Show that 
\[|\hat{\spc{X}}-\hat{\spc{Y}}|_{\GH}\le 2\cdot|\spc{X}- \spc{Y}|_{\GH}.\] 
In other words $\spc{X}\mapsto \hat{\spc{X}}$ defines a $2$-Lipschitz map $\GH\to\GH$.

\end{thm}



\section{Approximations and almost isometries}\label{sec:GH-approx}

\begin{thm}{Definition}\label{ex:defGHR}
Let $\spc{X}$ and $\spc{Y}$ be two metric spaces.
A relation $\approx$ between points in $\spc{X}$ and $\spc{Y}$ is called \index{$\eps$-approximation}\emph{$\eps$-approximation} if the following conditions hold:
\begin{itemize}
\item For any $x\in  \spc{X}$ there is $y\in \spc{Y}$ such that $x\approx y$.
\item For any $y\in  \spc{Y}$ there is $x\in \spc{X}$ such that $x\approx y$.
\item If $x\approx y$ and $x'\approx y'$ for some $x, x'\in  \spc{X}$ and $y,y'\in \spc{Y}$, then 
\[\bigl|\dist{x}{x'}{\spc{X}}-\dist{y}{y'}{\spc{Y}}\bigr|<2\cdot\eps.\]
\end{itemize}

\end{thm}

\begin{thm}{Exercise}\label{ex:H-R}
Let $\spc{X}$ and $\spc{Y}$ be two compact metric spaces.
Show that
\[\dist{\spc{X}}{\spc{Y}}{\GH}<\eps\]
if and only if there is an $\eps$-approximation between $\spc{X}$ and $\spc{Y}$.

In other words $\dist{\spc{X}}{\spc{Y}}{\GH}$ is the greatest lower bound of values $\eps>0$ such that  there is an $\eps$-approximation between $\spc{X}$ and $\spc{Y}$.
\end{thm}

\parit{Proof of \ref{GH-4}.}
Suppose that 
\begin{itemize}
\item $\approx_1$ is a relation between points in $\spc{X}$ and $\spc{Y}$,
\item $\approx_2$ is a relation between points in $\spc{Y}$ and $\spc{Z}$.
\end{itemize}
Consider the relation $\approx_3$ between points in $\spc{X}$ and $\spc{Z}$ such that
$x\approx_3 z$ if and only if there is $y\in  \spc{Y}$ such that 
$x\approx_1 y$ and $y\approx_2 z$.

It is straightforward to check that if $\approx_1$ is an $\eps_1$-approximation and $\approx_2$ is an $\eps_2$-approximation, then $\approx_3$ is an $(\eps_1+\eps_2)$-approximation.

Applying \ref{ex:H-R}, we get that if 
\[|\spc{X}-\spc{Y}|_{\GH}<\eps_1
\quad\text{and}\quad
|\spc{Y}-\spc{Z}|_{\GH}<\eps_2,
\]
then 
\[|\spc{X}-\spc{Z}|_{\GH}<\eps_1+\eps_2.\]
Hence \ref{GH-4} follows.
\qeds

The following weakened version of isometry is closely related to $\eps$-approximations.

\begin{thm}{Definition} Let $\spc{X}$ and $\spc{Y}$ be metric spaces and $\eps>0$. 
A  map\footnote{possibly noncontinuous} $f\: \spc{X} \z\to \spc{Y}$ is called an \index{almost isometry}\emph{$\eps$-isometry} 
if $f(\spc{X})$ is an $\eps$-net in $\spc{Y}$ and
\[\bigl|\dist{x}{x'}{\spc{X}}-\dist{f(x)}{f(x')}{\spc{Y}}\bigr|<\eps.\]
for any $x,x'\in \spc{X}$.
\end{thm}

\begin{thm}{Exercise}\label{ex:eps-isom}
Let $\spc{X}$ and $\spc{Y}$ be compact metric spaces.

\begin{subthm}{ex:eps-isom:GH>isom}
If $\dist{\spc{X}}{\spc{Y}}{\GH}<\eps$, then there is a $2\cdot\eps$-isometry $f\:\spc{X}\to\spc{Y}$.
\end{subthm}

\begin{subthm}{ex:eps-isom:isom>GH}
If there is an $\eps$-isometry $f\:\spc{X}\to\spc{Y}$, then $\dist{\spc{X}}{\spc{Y}}{\GH}<\eps$.
\end{subthm}

\end{thm}

\section{Optimal realization}\label{sec:extfun=GH}

Note that
\[\dist{\spc{X}'}{\spc{Y}'}{\Haus\spc{W}}\ge \dist{\spc{X}}{\spc{Y}}{\GH},\]
where $\spc{X}$, $\spc{Y}$, $\spc{X}'$, $\spc{Y}'$, and $\spc{W}$ are as in \ref{def:GH}.
The following proposition states that equality holds for some choice of $\spc{X}'$, $\spc{Y}'$, and $\spc{W}$.

\begin{thm}{Proposition}\label{prop:GH=H}
For any two compact metric spaces $\spc{X}$ and $\spc{Y}$ there is a metric space $\spc{W}$
with subsets $\spc{X}'$ and $\spc{Y}'$ such that 
$\spc{X}'\iso\spc{X}$, $\spc{Y}'\iso\spc{Y}$, and 
\[\dist{\spc{X}'}{\spc{Y}'}{\Haus\spc{W}}=\dist{\spc{X}}{\spc{Y}}{\GH}.\]
\end{thm}

Let us introduce the so-called \textit{appropriate functions} and use them in a reinterpretation of Gromov--Hausdorff distance.

Suppose $\spc{X}$, $\spc{Y}$, $\spc{X}'$, $\spc{Y}'$, and $\spc{W}$ are as in \ref{def:GH}.
By passing to the subspace $\spc{X}'\cup\spc{Y}'$ in $\spc{W}$, we can assume that $\spc{W}=\spc{X}'\cup\spc{Y}'$.
Note that in this case the metric on $\spc{W}$ is completely determined by the function 
\[f(x,y)=\dist{x}{y}{\spc{W}};\]
a function $f\:\spc{X}\times \spc{Y}\to\RR$ that can appear this way will be called \index{appropriate function}\emph{appropriate}.

Note that a function $f\:\spc{X}\times\spc{Y}\to\RR$ is appropriate if and only if
$x\mapsto f(x,y)$ and $y\mapsto f(x,y)$ are extension functions;
that is, if
\[
\begin{aligned}
f(x,y)+f(x,y')
&\ge \dist{y}{y'}{\spc{Y}}\ge |f(x,y)-f(x,y')|,
\\
f(x,y)+f(x',y)
&\ge \dist{x}{x'}{\spc{X}}\ge |f(x,y)-f(x',y)|;
\end{aligned}
\eqlbl{eq:appropriate}
\]
for any $x,x',\in\spc{X}$ and  $y,y'\in\spc{X}$;
see \ref{sec:Extension property}.
In other words, the following defines a pseudometric on $\spc{X}\sqcup\spc{Y}$
\[\dist{x}{y}{\spc{X}\sqcup\spc{Y}}=
\begin{cases}
\dist{x}{y}{\spc{X}}&\text{if\ } x,y\in \spc{X},
\\
\dist{x}{y}{\spc{Y}}&\text{if\ } x,y\in \spc{Y},
\\
f(x,y)&\text{if\ } x\in \spc{X}\ \text{and}\ y\in \spc{Y},
\end{cases}
\]
and the corresponding metric space $\spc{W}$ contains isometric copies of $\spc{X}$ and $\spc{Y}$.

Given an appropriate function $f\:\spc{X}\times\spc{Y}\to\RR$, set 
\begin{align*}
a_f&=\max_{x\in \spc{X}}\{\min_{y\in\spc{Y}} \{f(x,y)\}\},
\\
b_f&=\max_{y\in \spc{Y}}\{\min_{x\in\spc{X}} \{f(x,y)\}\}.
\end{align*}

\begin{thm}{Observation}\label{obs:GH=min-appropriate}
If $\spc{X}$, $\spc{Y}$, $\spc{X}'$, $\spc{Y}'$, and $\spc{W}$ as above then
\[\dist{\spc{X}'}{\spc{Y}'}{\Haus\spc{W}}=\inf_f\{a_f,b_f\}.\]

\end{thm}

\parit{Proof of \ref{prop:GH=H}.}
By \ref{eq:appropriate}, any appropriate functions $f\:\spc{X}\times\spc{Y}\to\RR$ is $2$-Lipschitz.
Observe that the functional $f\mapsto \min\{a_f,b_f\}$ is continuous.
Applying the Arzelà--Ascoli theorem, we can get an  appropriate function $f\:\spc{X}\times\spc{Y}\to\RR$ 
with minimal possible value $\min\{a_f,b_f\}$.
It remains to apply \ref{obs:GH=min-appropriate}.
\qeds

\begin{thm}{Exercise}\label{ex:XYZ}
Construct three compact metric spaces $\spc{X}$, $\spc{Y}$, and $\spc{Z}$
such that for any metric space $\spc{W}$
with subsets $\spc{X}'$, $\spc{Y}'$, and $\spc{Z}'$ such that 
$\spc{X}'\iso\spc{X}$, $\spc{Y}'\iso\spc{Y}$, and $\spc{Z}'\iso\spc{Z}$
at least one of the following three inequalities is strict
\begin{align*}
\dist{\spc{X}'}{\spc{Y}'}{\Haus\spc{W}}&\ge \dist{\spc{X}}{\spc{Y}}{\GH},
\\
\dist{\spc{Y}'}{\spc{Z}'}{\Haus\spc{W}}&\ge\dist{\spc{Y}}{\spc{Z}}{\GH},
\\
\dist{\spc{Z}'}{\spc{X}'}{\Haus\spc{W}}&\ge\dist{\spc{Z}}{\spc{X}}{\GH}.
\end{align*}
\end{thm}

\section{Convergence}

The Gromov--Hausdorff metric is used to define \index{Gromov--Hausdorff convergence}\emph{Gromov--Hausdorff convergence}.
Namely, a sequence of compact metric spaces $\spc{X}_n$ converges to compact metric spaces $\spc{X}_\infty$ in the sense of Gromov--Hausdorff if 
\[\dist{\spc{X}_n}{\spc{X}_\infty}{\GH}\to 0\quad\text{as}\quad n\to\infty.\]

This convergence is more important than the metric ---
in all applications, we use only the topology on $\GH$
and we do not care about the particular value of Gromov--Hausdorff distance between spaces.
The following observation follows from \ref{ex:eps-isom}:

\begin{thm}{Observation}\label{obs:GH-e-isom}
A sequence of compact metric spaces $(\spc{X}_n)$ converges to  $\spc{X}_\infty$ in the sense of Gromov--Hausdorff if and only if there is a sequence $\eps_n\to0+$
and an $\eps_n$-isometry $f_n\:\spc{X}_n\to \spc{X}_\infty$ for each $n$.
\end{thm}

In the following exercises, \textit{convergence} is understood in the sense of Gromov--Hausdorff.

\begin{thm}{Exercise}\label{ex:GH-SC}
\begin{subthm}{ex:GH-SC:circle}
Show that a sequence of compact simply-connected length spaces cannot converge to a circle.
\end{subthm}

\begin{subthm}{ex:GH-SC:nonsc-limit}
Construct a sequence of compact simply-connected length spaces that converges to a compact non-simply-connected space.
\end{subthm}
\end{thm}

\begin{thm}{Exercise}\label{ex:sphere-to-ball}
\begin{subthm}{ex:sphere-to-ball:2}
Show that a sequence of length metrics on the 2-sphere cannot converge to the unit disk.
\end{subthm}

\begin{subthm}{ex:sphere-to-ball:3}
Construct a sequence of length metrics on the 3-sphere that converges to a unit 3-ball.
\end{subthm}

\end{thm}

\section{Uniformly totally bonded families}

\begin{thm}{Definition}\label{def:utb}
A family $\spc{Q}$ of (isometry classes) of compact metric spaces is called  \index{uniformly totally bonded family}\emph{uniformly totally bonded} if it meets the following two conditions:

\begin{subthm}{}
spaces in $\spc{Q}$ have uniformly bounded diameters; that is, there is $D\in\RR$ such that
\[\diam\spc{X}\le D\]
for any space $\spc{X}$ in $\spc{Q}$.
\end{subthm}

\begin{subthm}{}
For any $\eps>0$ there is $n\in\NN$ such that any space $\spc{X}$ in $\spc{Q}$ admits an $\eps$-net with at most $n$ points.
\end{subthm}
\end{thm}

\begin{thm}{Exercise}\label{ex:utb+pack}
Let $\spc{Q}$ be a family of compact spaces with uniformly bounded diameters.
Show that $\spc{Q}$ is uniformly totally bonded if for any $\eps>0$ there is $n\in\NN$ such that 
\[\pack_\eps\spc{X}\le n\]
for any space $\spc{X}$ in $\spc{Q}$.
\end{thm}


Fix a real constant $C$.
A Borel measure $\mu$ on a metric space $\spc{X}$ is called \index{doubling space}\emph{$C$-doubling} if
\[\mu[\oBall(p,2\cdot r)]< C\cdot\mu[\oBall(p,r)]\]
for any point $p\in \spc{X}$ and any $r>0$.
A Borel measure is called \index{doubling measure}\emph{doubling} if it is {}\emph{$C$-doubling} for some real constant $C$.

\begin{thm}{Exercise}\label{pr:doubling}
Let $\spc{Q}(C,D)$ be the set of all the compact metric spaces with diameter at most $D$ that admit a $C$-doubling measure.
Show that $\spc{Q}(C,D)$ is totally bounded.
\end{thm}

Given two metric spaces $\spc{X}$ and $\spc{Y}$, we will write $\spc{X}\le \spc{Y}$ if there is a distance-noncontracting map $f\:\spc{X}\to \spc{Y}$;
that is, if 
$$ |x-x'|_{\spc{X}}\le|f(x)-f(x')|_{\spc{Y}}$$
for any $x,x'\in \spc{X}$.

\begin{thm}{Exercise}\label{pr:under}

\begin{subthm}{pr:under:if}
Let $\spc{Y}$ be a compact metric space.
Show that the set of all spaces $\spc{X}$ such that $\spc{X}\le\spc{Y}$
is uniformly totally bounded.
\end{subthm}

\begin{subthm}{pr:under:only-if}
Show that for any uniformly totally bounded set $\spc{Q}\subset\GH$ there is a compact space $\spc{Y}$
such that $\spc{X}\le\spc{Y}$ for any $\spc{X}$ in $\spc{Q}$.
\end{subthm}

\end{thm}

\section{Gromov selection theorem}

The following theorem is analogous to Blaschke selection theorems (\ref{thm:compact+Hausdorff}).

\begin{thm}{Gromov selection theorem}\label{thm:gromov-compactness}
Let $\spc{Q}$ be a closed subset of $\GH$.
Then $\spc{Q}$ is compact if and only if the spaces in $Q$ are uniformly totally bounded.
\end{thm}

\begin{thm}{Lemma}\label{lem:GH-complete}
The space $\GH$ is complete.
\end{thm}


Let us define gluing of metric spaces that will be used in the proof of the lemma.

Suppose 
$\spc{U}$ and $\spc{V}$ are metric spaces 
with isometric closed sets $A\subset\spc{U}$ and $A'\subset\spc{V}$;
let $\iota\:A\to A'$ be an isometry.
Consider the space $\spc{W}$ of all equivalence classes in $\spc{U}\sqcup\spc{V}$ with the equivalence relation given by $a\sim\iota(a)$ for any $a\in A$.

It is straightforward to check that the following defines a metric on~$\spc{W}$:
\begin{align*}
\dist{u}{u'}{\spc{W}}&\df\dist{u}{u'}{\spc{U}}
\\
\dist{v}{v'}{\spc{W}}&\df\dist{v}{v'}{\spc{V}}
\\
\dist{u}{v}{\spc{W}}&\df\min\set{\dist{u}{a}{\spc{U}}+\dist{v}{\iota(a)}{\spc{V}}}{a\in A}
\end{align*}
where $u,u'\in \spc{U}$ and $v,v'\in \spc{V}$.

The  space $\spc{W}$ is called the \index{gluing}\emph{gluing} of $\spc{U}$ and  $\spc{V}$ along~$\iota$; briefly, we can write
$\spc{W}=\spc{U}\sqcup_\iota\spc{V}$.
If one applies this construction to two copies of one space $\spc{U}$ with a set $A\subset \spc{U}$ and the identity map $\iota\:A\to A$, then the obtained space is called the \index{doubling}\emph{doubling} of $\spc{U}$ along~$A$; this space can be denoted by $\sqcup_A^2\spc{U}$.

Note that the inclusions $\spc{U}\hookrightarrow \spc{W}$ and $\spc{V}\hookrightarrow \spc{W}$ are distance preserving.
Therefore we can and will consider $\spc{U}$ and $\spc{V}$ as the subspaces of $\spc{W}$;
this way the subsets $A$ and $A'$ will be identified and denoted further by~$A$.
Note that $A=\spc{U}\cap \spc{V}\subset \spc{W}$.

\parit{Proof.}
Let $\spc{X}_1,\spc{X}_2,\dots$ be a Cauchy sequence in $\GH$.
Passing to a subsequence if necessary, 
we can assume that $|\spc{X}_n-\spc{X}_{n+1}|_{\GH}<\tfrac1{2^n}$ for each~$n$.
In particular, for each $n$ there is a metric space $\spc{V}_n$ with distance preserving inclusions $\spc{X}_n\hookrightarrow \spc{V}_n$ and $\spc{X}_{n+1}\hookrightarrow \spc{V}_n$ such that
\[|\spc{X}_n-\spc{X}_{n+1}|_{\Haus\spc{V}_n}<\tfrac1{2^n}\]
for each $n$.
Moreover, we may assume that $\spc{V}_n=\spc{X}_n\cup\spc{X}_{n+1}$.

Let us glue $\spc{V}_1$ to $\spc{V}_2$ along $\spc{X}_2$;
to the obtained space glue $\spc{V}_3$ along $\spc{X}_3$, and so on.
The obtained metric space $\spc{W}$
has an underlying set formed by the disjoint union of all $\spc{X}_n$ such that each inclusion $\spc{X}_n\z\hookrightarrow\spc{W}$ is distance preserving and
\[|\spc{X}_n-\spc{X}_{n+1}|_{\Haus\spc{W}}<\tfrac1{2^n}\]
for each $n$.
In particular,
\[|\spc{X}_m-\spc{X}_n|_{\Haus\spc{W}}<\tfrac1{2^{n-1}}\eqlbl{eq:|x_m-X_n|}\] 
if $m>n$.

Denote by $\bar{\spc{W}}$ the completion of $\spc{W}$.
Observe that the union $\spc{X}_1\z\cup \spc{X}_2\cup\z\dots\cup \spc{X}_n$ is compact and \ref{eq:|x_m-X_n|} implies that it forms a $\tfrac1{2^{n-1}}$-net in $\bar{\spc{W}}$.
Whence $\bar{\spc{W}}$ is compact; see \ref{totally-bounded} and \ref{ex:compact-net}.

Applying the Blaschke selection theorem (\ref{thm:compact+Hausdorff}),
we can pass to a subsequence of $\spc{X}_n$ that converges in $\Haus\bar{\spc{W}}$; denote its limit by $\spc{X}_\infty$.
It remains to observe that $\spc{X}_\infty$ is the Gromov--Hausdorff limit of $\spc{X}_n$.
\qeds

\parit{Proof of \ref{thm:gromov-compactness}; only-if part.}
Suppose that there is no sequence $\eps_n\to0$ as described in \ref{def:utb}.
Observe that in this case
there is a sequence of spaces $\spc{X}_n\in\spc{Q}$ such that 
$$\pack_\delta \spc{X}_n\to\infty
\quad\text{as}\quad
n\to\infty$$
for some fixed $\delta>0$.

Since $\spc{Q}$ is compact, 
this sequence has a partial limit, say $\spc{X}_\infty\in\spc{Q}$.
Observe that $\pack_{\delta} \spc{X}_\infty=\infty$.
Therefore, $\spc{X}_\infty$ is not compact --- a contradiction.

\parit{If part.}
Given a positive integer $n$ consider the set of all metric spaces $\spc{W}_n$
with the number of points at most $n$ and diameter $\le D$.
Note that $\spc{W}_n$ is a compact set in $\GH$ for each $n$.

Let $D$ and $n=n(\eps)$ be as in the definition of uniformly totally bonded families (\ref{def:utb}).

Note that an $\eps$-net of any $\spc{X}\in\spc{Q}$ belongs to $\spc{W}_{n(\eps)}$.
Therefore, $\spc{W}_{n(\eps)}$ is a compact $\eps$-net of $\spc{Q}$ for any $\eps>0$.
Since $\spc{Q}$ is closed in a complete space $\GH$, it implies that $\spc{Q}$ is compact.
\qeds

\begin{thm}{Exercise}\label{ex-GH-length}
Show that the space $\GH$ is 

\begin{subthm}{ex-GH-length:length}
length,
\end{subthm}

\begin{subthm}{ex-GH-length:geodesic}
geodesic.
\end{subthm}

\end{thm}

\begin{thm}{Exercise}\label{ex:GH-po}
For two metric spaces $\spc{X}$ and $\spc{Y}$,
we write $\spc{X}\le \spc{Y}+\eps$ if
there is a map $f\:\spc{X}\to \spc{Y}$ such that 
\[\dist{x}{x'}{\spc{X}}\le \dist{f(x)}{f(x')}{\spc{Y}}+\eps\]
for any $x,x'\in \spc{X}$.

\begin{subthm}{ex:GH-po:a}
Show that 
$$\dist{\spc{X}}{\spc{Y}}{\GH'}=\inf\set{\eps>0}{\spc{X}\le \spc{Y}+\eps
\quad\text{and}\quad
\spc{Y}\le \spc{X}+\eps}$$
defines a metric on the space of (isometry classes) of compact metric spaces.
\end{subthm}

\begin{subthm}{ex:GH-po:b}
Moreover $\dist{*}{*}{\GH'}$ is equivalent to the Gromov--Hausdorff metric;
that is,
$$|\spc{X}_n-\spc{X}_\infty|_{\GH}\to 0 
\quad\iff\quad 
\dist{\spc{X}_n}{\spc{X}_\infty}{\GH'}\to 0$$ 
as $n\to\infty$.
\end{subthm}
\end{thm}

\section{Universal ambient space}

Recall that a metric space is called universal if it contains an isometric copy of any separable metric space (in particular, any compact metric space).
Examples of universal spaces include $\spc{U}_\infty$ --- the Urysohn space and $\ell^\infty$ --- the space of bounded infinite sequences with the metric defined by $\sup$-norm; see \ref{prop:sep-in-urys} and \ref{ex:frechet}.

The following proposition says that the space $\spc{W}$ in Definition~\ref{def:GH} can be exchanged to a fixed universal space.

\begin{thm}{Proposition}\label{prop:GH-with-fixed-Z}
Let $\spc{U}$ be a universal space.
Then for any compact metric spaces $\spc{X}$ and $\spc{Y}$ we have
$$|\spc{X}-\spc{Y}|_{\GH} = \inf \{|\spc{X}'-\spc{Y}'|_{\Haus\spc{U}}\}$$ 
where the greatest lower bound is taken over all pairs of sets $\spc{X}'$ and $\spc{Y}'$ in $\spc{U}$
which isometric to  $\spc{X}$ and $\spc{Y}$ respectively.  
\end{thm}




\parit{Proof of \ref{prop:GH-with-fixed-Z}.}
By the definition (\ref{def:GH}), we have that 
\[|\spc{X}-\spc{Y}|_{\GH} \le \inf \{|\spc{X}'-\spc{Y}'|_{\Haus\spc{U}}\};\]
it remains to prove the opposite inequality.

Suppse $|\spc{X}-\spc{Y}|_{\GH}<\eps$;
let $\spc{X}'$, $\spc{Y}'$ and $\spc{W}$ be as in \ref{def:GH}.
We can assume that $\spc{W}=\spc{X}'\cup\spc{Y}'$;
otherwise pass to the subspace $\spc{X}'\cup\spc{Y}'$ of~$\spc{W}$.
In this case, $\spc{W}$ is compact;
in particular, it is separable.

Since $\spc{U}$ is universal, there is a distance-preserving embedding of $\spc{W}$ in $\spc{U}$;
let us keep the same notation for $\spc{X}'$, $\spc{Y}'$, and their images.
It follows that 
\[|\spc{X}'-\spc{Y}'|_{\Haus\spc{U}}<\eps,\]
--- hence the result.
\qeds

\begin{thm}{Exercise}\label{ex:GH-urysohn}
Let $\spc{U}_\infty$ be the Urysohn space.
Given two compact sets $A$ and $B$ in $\spc{U}_\infty$ define 
\[\|A-B\|=\inf\{|A-\iota(B)|_{\Haus\spc{U}_\infty}\},\]
where the greatest lower bound is taken for all isometrics $\iota$ of $\spc{U}_\infty$.
Show that $\|{*}\z-{*}\|$ defines a pseudometric%
\footnote{The value $\|A-B\|$ is called Hausdorff distance \index{Hausdorff distance up to isometry}\emph{up to isometry} from $A$ to $B$ in $\spc{U}_\infty$.}
on nonempty compact subsets of $\spc{U}_\infty$ and its corresponding metric space is isometric to $\GH$.
\end{thm}

\section{Remarks}

Suppose $\spc{X}_n\GHto \spc{X}_\infty$, then there is a metric on the disjoint union 
\[\bm{X}=\bigsqcup_{n\in \NN\cup\{\infty\}} \spc{X}_n\] 
that satisfies the following property:

\begin{thm}{Property}\label{propery:GH}
The restriction of metric on each $\spc{X}_n$ and $\spc{X}_\infty$ coincides with its original metric, 
and $\spc{X}_n\Hto \spc{X}_\infty$ as subsets in $\bm{X}$.
\end{thm}

Indeed, since $\spc{X}_n\GHto \spc{X}_\infty$, there is a metric on $\spc{V}_n=\spc{X}_n\sqcup \spc{X}_\infty$ such that the restriction of metric on each $\spc{X}_n$ and $\spc{X}_\infty$ coincides with its original metric, and $\dist{\spc{X}_n}{\spc{X}_\infty}{\Haus\spc{V}_n}<\eps_n$ for some sequence $\eps_n\to 0$.
Gluing all $\spc{V}_n$ along $\spc{X}_\infty$, we obtain the required space $\bm{X}$.

In other words, the metric on $\bm{X}$ \textit{defines} the convergence $\spc{X}_n\z\GHto \spc{X}_\infty$.
This metric makes it possible to talk about limits of sequences $x_n\in \spc{X}_n$ as $n\to\infty$, as well as weak limits of a sequence of Borel measures $\mu_n$ on $\spc{X}_n$ and so on.

For that reason, it is useful to define \index{Hausdorff convergence}\emph{convergence} by specifying the metric on $\bm{X}$ that satisfies the property
for the variation of Hausdorff convergence described in Section~\ref{sec:H-variation}.

This approach is more flexible;
in particular, it can be used to define Gromov--Hausdorff convergence of arbitrary metric spaces (not necessarily compact).
A limit space for this generalized convergence is not uniquely defined.
For example, if each space $\spc{X}_n$ in the sequence is isometric to the half-line, then its limit might be isometric to the half-line or the whole line.
The first convergence is evident and the second could be guessed from the diagram.

\begin{figure}[ht!]
\vskip-0mm
\centering
\includegraphics{mppics/pic-500}
\end{figure}

Often the isometry class of the limit can be fixed by marking a point $p_n$ in each space $\spc{X}_n$, it is called \index{pointed convergence}\emph{pointed Gromov--Hausdorff convergence} --- we say that $(\spc{X}_n,p_n)$ converges to $(\spc{X}_\infty,p_\infty)$ if there is a metric on $\bm{X}$ as in \ref{propery:GH} such that $\spc{X}_n\Hto \spc{X}_\infty$ and $p_n\to p_\infty$.
For example, the sequence $(\spc{X}_n,p_n)=(\RR_+,0)$ converges to $(\RR_+,0)$, while $(\spc{X}_n,p_n)=(\RR_+,n)$ converges to $(\RR,0)$.

The pointed convergence works nicely for proper metric spaces;
the following theorem is an analog of Gromov's selection theorem for this convergence.

\begin{thm}{Theorem}\label{thm:pointed-gromov-compactness}%
Let $\spc{Q}$ be a set of isometry classes of pointed proper metric spaces
$(\spc{X},p)$.
Assume that for any $R>0$, the $R$-balls in the spaces centered at the marked points form a uniformly totally bounded family of spaces.
Then $\spc{Q}$ is precompact with respect to the pointed Gromov--Hausdorff convergence. 
\end{thm}

\chapter{Ultralimits}

Ultralimits provide a very general way to pass to a limit that always works.
It use a set-theoretical construction --- the so called ulrafilter.

In geometry, ultralimits are used only as a canonical way to pass to a convergent subsequence.
It is useful thing in the proofs where one needs to repeat ``pass to convergent subsequence'' too many times.

This lecture is based on the introduction to the paper of Bruce Kleiner and Bernhard Leeb \cite{kleiner-leeb}.

\section{Ultrafilters}

Recall that $\NN$ denotes the set of natural numbers, $\NN=\{1,2,\dots\}$

\begin{thm}{Definition}
A finitely additive measure $\omega$ 
on  $\NN$ 
is called an \index{ultrafilter}\emph{ultrafilter} if it satisfies 
\begin{subthm}{}
$\omega(S)=0$ or $1$ for any subset $S\subset \NN$.
\end{subthm}
An ultrafilter $\omega$ is called 
\emph{nonprinciple}\index{ultrafilter!nonprinciple ultrafilter}\index{nonprinciple ultrafilter} if in addition 
\begin{subthm}{}
$\omega(F)=0$ for any finite subset $F\subset \NN$.
\end{subthm}
\end{thm}

If $\omega(S)=0$ for some subset $S\subset \NN$,
we say that $S$ is \index{$\omega$-small}\emph{$\omega$-small}. 
If $\omega(S)=1$, we say that $S$ contains \index{$\omega$-almost all}\emph{$\omega$-almost all} elements of $\NN$.

\parbf{Classical definition.}
More commonly, a nonprinciple ultrafilter is defined as a collection, say $\mathfrak{F}$, of sets in $\NN$ such that
\begin{enumerate}
\item\label{filter:supset} if $P\in \mathfrak{F}$ and $Q\supset P$, then $Q\in \mathfrak{F}$,
\item\label{filter:cap} if $P, Q\in \mathfrak{F}$, then $P\cap Q\in \mathfrak{F}$,
\item\label{filter:ultra} for any subset $P\subset\NN$, either $P$ or its complement is an element of $\mathfrak{F}$.
\item\label{filter:non-prin} if $F\subset \NN $ is finite, then $F\notin \mathfrak{F}$.
\end{enumerate}
Setting $P\in\mathfrak{F}\Leftrightarrow\omega(P)=1$ makes these two definitions equivalent.

A nonempty collection of sets $\mathfrak{F}$ that does not include the empty set and satisfies only conditions \ref{filter:supset} and \ref{filter:cap} is called a \index{filter}\emph{filter}; 
if in addition $\mathfrak{F}$ satisfies Condition~\ref{filter:ultra} it is called an \index{ultrafilter}\emph{ultrafilter}.
From Zorn's lemma, it follows that every filter contains an ultrafilter.
Thus there is an ultrafilter $\mathfrak{F}$ contained in the filter of all complements of finite sets; clearly this $\mathfrak{F}$ is nonprinciple.


\parbf{Stone--\v{C}ech compactification.}
Given a set $S\subset \NN$, consider subset $\Omega_S$ of all ultrafilters $\omega$ such that $\omega(S)=1$.
It is straightforward to check that the sets $\Omega_S$ for all $S\subset \NN$ form a topology on the set of ultrafilters on $\NN$. 
The obtained space is called \index{Stone--\v{C}ech compactification}\emph{Stone--\v{C}ech compactification} of $\NN$;
it is usually denoted as $\beta\NN$.

Let $\omega_n$ denotes the principle ultrafilter such that $\omega_n(\{n\})=1$; that is, $\omega_n(S)=1$ if and only if $n\in S$.
Note that $n\mapsto\omega_n$ defines a natural embedding $\NN\hookrightarrow\beta\NN$. 
Using the described embedding, we can (and will) consider $\NN$ as a subset of $\beta\NN$.

The space $\beta\NN$ is the maximal compact Hausdorff space that contains $\NN$  as an everywhere dense subset.
More precisely, for any compact Hausdorff space $\spc{X}$ 
and a map $f\:\NN\to \spc{X}$ there is unique continuous map $\bar f\:\beta\NN\to X$ such that the restriction $\bar f|_\NN$ coincides with $f$. 

\section{Ultralimits of points}
\label{ultralimits}\index{ultralimit}

Further we will need the existence of a nonprinciple  ultrafilter $\omega$,
which we fix once and for all.

Assume $(x_n)$ is a sequence of points in a metric space $\spc{X}$. 
Let us define the \index{$\omega$-limit}\emph{$\omega$-limit} of $(x_n)$ as the point $x_\omega$ 
such that for any $\eps>0$, $\omega$-almost all elements of $(x_n)$ lie in $\oBall(x_\omega,\eps)$; 
that is,
\[\omega\set{n\in\NN}{\dist{x_\omega}{x_n}{}<\eps}=1.\]
In this case, we will write 
\[x_\omega=\lim_{n\to\omega} x_n
\ \ \text{or}\ \ 
x_n\to x_\omega\ \text{as}\ n\to\omega.\]

For example if $\omega$ is the principle ultrafilter such that $\omega(\{n\})=1$ for some $n\in\NN$, then
$x_\omega=x_n$.

Alternatively, the sequence $(x_n)$ can be regarded as a map $\NN\to\spc{X}$.
In this case the map $\NN\to\spc{X}$ can be extended to a continuous map $\beta\NN\to\spc{X}$ from the Stone--\v{C}ech compactification of $\NN$.
Then the $\omega$-limit $x_\omega$ can be regarded as the image of $\omega$.

Note that $\omega$-limits of a sequence and its subsequence may differ.
For example, in general
\[\lim_{n\to\omega}x_n
\ne
\lim_{n\to\omega}x_{2\cdot n}.\] 

\begin{thm}{Proposition}\label{prop:ultra/partial}
Let $\omega$ be a nonprinciple ultrafilter.
Assume $(x_n)$ is a sequence of points in a metric space $\spc{X}$
and $x_n\to  x_\omega$ as $n\to\omega$.
Then $x_\omega$ is a partial limit of the sequence $(x_n)$;
that is, there is a subsequence $(x_n)_{n\in S}$ that converges to $x_\omega$ in the usual sense.
\end{thm}

\parit{Proof.}
Given $\eps>0$, 
set $S_\eps=\set{n\in\NN}{\dist{x_n}{x_\omega}{}<\eps}$.

Note that $\omega(S_\eps)=1$ for any $\eps>0$.
Since $\omega$ is nonprinciple, the set $S_\eps$ is infinite.
Therefore we can choose an increasing sequence $(n_k)$
such that $n_k\in S_{\frac1k}$ for each $k\in \NN$.
Clearly $x_{n_k}\to x_\omega$ as $k\to\infty$.
\qeds

The following proposition 
is analogous to the statement that any sequence in a compact metric space 
has a convergent subsequence;
it can be proved the same way.

\begin{thm}{Proposition}\label{prop:ultra/compact}
Let $\spc{X}$ be a compact metric space.
Then
any sequence of points $(x_n)$ in $\spc{X}$ has unique $\omega$-limit $x_\omega$.

In particular, a bounded sequence of real numbers has a unique $\omega$-limit.
\end{thm}

The following lemma is an ultralimit analog of Cauchy convergence test.

\begin{thm}{Lemma}\label{lem:X-X^w}
Let $(x_n)$ be a sequence of points in a complete space $\spc{X}$. 
Assume for each subsequence $(y_n)$ of $(x_n)$, 
the $\omega$-limit 
\[y_\omega=\lim_{n\to\omega}y_{n}\in \spc{X}\]
is defined and does not depend on the choice of subsequence, 
then the sequence $(x_n)$ converges in the usual sense.
\end{thm}

\parit{Proof.} If $(x_n)$ is not a Cauchy sequence,
then for some $\eps>0$, there is a subsequence $(y_n)$ of $(x_n)$ such that $\dist{x_n}{y_n}{}\ge\eps$ for all $n$.

It follows that $\dist{x_\omega}{y_\omega}{}\ge \eps$, a contradiction.\qeds


\section{Ultralimits of spaces}\label{sec:Ultralimit of spaces}

Recall that $\omega$ denotes a nonprinciple ultrafilter on the set of natural numbers.

Let $\spc{X}_n$ be a sequence of metric spaces.
Consider all sequences of points $x_n\in \spc{X}_n$.
On the set of all such sequences,
define a pseudometric by
\[\dist{(x_n)}{(y_n)}{}
=
\lim_{n\to\omega} \dist{x_n}{y_n}{\spc{X}_n}.
\eqlbl{eq:olim-dist}\]
Note that the $\omega$-limit on the right hand side is always defined 
and takes a value in $[0,\infty]$. 

Set $\spc{X}_\omega$ to be the corresponding metric space; 
that is, the underlying set of $\spc{X}_\omega$ is formed by classes of equivalence of sequences of points $x_n\in\spc{X}_n$ 
defined by 
\[(x_n)\sim(y_n)
\ \Leftrightarrow\ 
\lim_{n\to\omega} \dist{x_n}{y_n}{}=0\]
and the distance is defined by \ref{eq:olim-dist}.

The space $\spc{X}_\omega$ is called \index{$\omega$-limit space}\emph{$\omega$-limit} of $\spc{X}_n$.
Typically  $\spc{X}_\omega$ will denote the  
$\omega$-limit of sequence $\spc{X}_n$;
we may also write  
\[\spc{X}_n\to\spc{X}_\omega\ \ \text{as}\ \  n\to\omega\ \ \text{or}\ \ \spc{X}_\omega=\lim_{n\to\omega}\spc{X}_n.\]

Given a sequence $x_n\in \spc{X}_n$,
we will denote by $x_\omega$ its equivalence class which is a point in $\spc{X}_\omega$;
equivalently we will write
\[x_n\to x_\omega \ \ \text{as}\ \  n\to\omega\ \ \text{or}\ \ x_\omega=\lim_{n\to\omega} x_n.\]

\begin{thm}{Observation}\label{obs:ultralimit-is-complete}
The $\omega$-limit of any sequence of metric spaces is complete. 
\end{thm}

\parit{Proof.}
Let $\spc{X}_n$ be a sequence of metric spaces and $\spc{X}_n\to\spc{X}_\omega$ as $n\to\omega$.

Fix a Cauchy sequence $x_{m}\in \spc{X}_\omega$.
Passing to a subsequence we can assume that $\dist{x_m}{x_{m-1}}{\spc{X}_\omega}<\tfrac1{2^m}$ for any $m$.

Let us choose double sequence $x_{n,m}\in \spc{X}_n$ such that for any fixed $m$ we have $x_{n,m}\to x_m$ as $n\to\omega$.
Note that $\dist{x_{n,m}}{x_{n,m-1}}{}<\tfrac1{2^m}$ for $\omega$-almost all $n$.
It follows that we can choose a nested sequence of sets 
\[\NN= S_1\supset S_2\supset\dots\] 
such that 
\begin{itemize}
\item $\omega(S_m)=1$ for each $m$, 
\item $k\ge m$ for any $k\in S_m$, and
\item if $n\in S_m$, then 
\[\dist{x_{n,m}}{x_{n,m-1}}{}<\tfrac1{2^m}\]
\end{itemize}

Consider the sequence $y_n=x_{n,m(n)}$, where $m(n)$ is the largest value such that $m(n)\in S_{m}$.
Denote by $y\in \spc{X}_\omega$ its $\omega$-limit.

Observe that by construction $x_n\to y$ as $n\to \infty$.
Hence the statement follows.
\qeds

\begin{thm}{Observation}\label{obs:ultralimit-is-geodesic}
The $\omega$-limit of any sequence of length spaces is geodesic. 
\end{thm}

\parit{Proof.}
If $\spc{X}_n$ is a sequence length spaces, then for any sequence of pairs $x_n, y_n\in X_n$ there is a sequence of $\tfrac1n$-midpoints $z_n$.

Let $x_n\to x_\omega$, $y_n\to y_\omega$ and $z_n\to z_\omega$ as $n\to \omega$.
Note that $z_\omega$ is a midpoint of $x_\omega$ and $y_\omega$ in $\spc{X}^\omega$.

By Observation~\ref{obs:ultralimit-is-complete}, $\spc{X}^\omega$ is complete.
Applying Lemma~\ref{lem:mid>geod} we get the statement.
\qeds


\begin{thm}{Exercise}\label{ex:lim(tree)}
Show that an ultralimit of metric trees is a metric tree. 
\end{thm}

\section{Ultrapower}

If all the metric spaces in the sequence are identical $\spc{X}_n=\spc{X}$, 
its $\omega$-limit 
$\lim_{n\to\omega}\spc{X}_n$
is denoted by $\spc{X}^\omega$
and called $\omega$-power of $\spc{X}$.



\begin{thm}{Exercise}\label{ex:ultrapower}
For any point $x\in \spc{X}$, consider the constant sequence $x_n=x$
and set $\iota(x)=\lim_{n\to\omega}x_n\in\spc{X}^\omega$.

\begin{subthm}{ex:ultrapower:a}
Show that $\iota\:\spc{X}\to\spc{X}^\omega$ is distance-preserving embedding. (So we can and will consider $\spc{X}$ as a subset of $\spc{X}^\omega$.)
\end{subthm}

\begin{subthm}{ex:ultrapower:compact} 
Show that $\iota$ is onto if and only if $\spc{X}$ compact.
\end{subthm}

\begin{subthm}{ex:ultrapower:proper} 
Show that if $\spc{X}$ is proper, then $\iota(\spc{X})$ forms a metric component of $\spc{X}^\omega$; that is, a subset of $\spc{X}^\omega$ that lie on finite distance from a given point.
\end{subthm}

\end{thm}

\begin{thm}{Observation}\label{obs:ultrapower-is-geodesic}
Let $\spc{X}$ be a complete metric space. 
Then $\spc{X}^\omega$ is geodesic space if and only if $\spc{X}$ is a length space.
\end{thm}

\parit{Proof.}
Assume $\spc{X}^\omega$ is geodesic space.
Then any pair of points $x,y\in \spc{X}$ has a midpoint $z_\omega\in\spc{X}^\omega$.
Fix a sequence of points $z_n\in  \spc{X}$ such that $z_n\to z_\omega$ as $n\to \omega$.

Note that 
$\dist{x}{z_n}{\spc{X}}\to \tfrac12\cdot \dist{x}{y}{\spc{X}}$
and 
$\dist{y}{z_n}{\spc{X}}\to \tfrac12\cdot \dist{x}{y}{\spc{X}}$
as 
$n\to\omega$.
In particular, for any $\eps>0$, the point $z_n$ is an $\eps$-midpoint of $x$ and $y$ for $\omega$-almost all $n$.
It remains to apply \ref{lem:mid>geod}.

The ``if''-part follows from \ref{obs:ultralimit-is-geodesic}.
\qeds

\begin{thm}{Exercise}\label{ex:two-geodesics-in-ultrapower}
Assume $\spc{X}$ is a complete length space 
and $p,q\in\spc{X}$ cannot be joined by a geodesic in $\spc{X}$.  
Then there are at least two distinct geodesics between $p$ and $q$ 
in the ultrapower $\spc{X}^\omega$.
\end{thm}

\begin{thm}{Exercise}
 Construct a proper metric space $\spc{X}$ such that $\spc{X}^\omega$ is not proper; that is, there is a point $p\in \spc{X}^\omega$ and $R<\infty$ such that the closed ball $\cBall[p,R]_{\spc{X}^\omega}$ is not compact.
\end{thm}

\section{Tangent and asymptotic spaces}

Choose a space $\spc{X}$ and a sequence of $\lambda_n>0$.
Consider the sequence of scalings $\spc{X}_n=\lambda_n\cdot\spc{X}=(\spc{X},\lambda_n\cdot\dist{*}{*}{\spc{X}})$.

Choose a point $p\in \spc{X}$ and denote by $p_n$ the corresponding point in $\spc{X}_n$.
Consider the $\omega$-limit $\spc{X}_\omega$ of $\spc{X}_n$ (one may denote it by $\lambda_\omega\cdot \spc{X}$);
set $p_\omega$ to be the $\omega$-limit of $p_n$.

If $\lambda_n\to 0$ as $n\to\omega$, then the metric component of $p_\omega$ in $\spc{X}_\omega$ is called \index{$\omega$-tangent space}\emph{$\omega$-tangent space} at $p$ and denoted by $\T_p^{\lambda_\omega}\spc{X}$ (or $\T_p^{\omega}\spc{X}$ if $\lambda_n=n$).\label{page:ultratangent space}

If $\lambda_n\to \infty$ as $n\to\omega$, then the metric component of $p_\omega$ in called \index{$\omega$-asymptotic space}\emph{$\omega$-asymptotic space}\footnote{Often it is called \emph{asymptotic cone} despite that it is not a cone in general; this name is used since in good cases it has a cone structure.}  and denoted by $\Asym\spc{X}$ or $\Asym^{\lambda_\omega}\spc{X}$.
Note that the space $\Asym\spc{X}$ and its point $p_\omega$ does not depend on the choice of $p\in \spc{X}$.

In general, the tangent and asymptotic spaces depend the sequence $\lambda_n$ and an nonprinciple ultrafiler $\omega$.
For nice spaces different choices may give the same space.

\begin{thm}{Exercise}\label{ex:ultraT}
Construct a metric space $\spc{X}$ with a point $p$ such that the tangent space
$\T_p^{\lambda_\omega}\spc{X}$ depends on the sequence $\lambda_n$ and/or ultrafilter $\omega$.
\end{thm}


\begin{thm}{Exercise}\label{ex:Asym(Lob)}
Let $\spc{L}$ be the Lobachevsky plane; $\spc{T}=\Asym\spc{L}$.

\begin{subthm}{ex:Asym(Lob):metric-tree}
Show that $\spc{T}$ is a complete metric tree.
\end{subthm}

\begin{subthm}{ex:Asym(Lob):continuum}
Show that $\spc{T}$ has {}\emph{continuum degree} at any point;
that is, for any point $t\in \spc{T}$ the set of connected components of the complement $\spc{T}\backslash\{t\}$ has cardinality continuum.
\end{subthm}

\begin{subthm}{ex:Asym(Lob):homogeneous}
Show that $\spc{T}$ is homogeneous; that is given two points $s,t\in \spc{T}$ there is an isometry of $\spc{T}$ that maps $s$ to $t$.
\end{subthm}

\begin{subthm}{ex:Asym(Lob):others}
Prove \ref{SHORT.ex:Asym(Lob):metric-tree}--\ref{SHORT.ex:Asym(Lob):homogeneous} if $\spc{L}$ is Lobachevsky space and/or for the infinite 3-regular%
\footnote{that is, degree of any vertex is 3.}
tree with unit edge. 
\end{subthm}


\end{thm}

As it shown in \cite{dyubina-polterovich}, the properties \ref{SHORT.ex:Asym(Lob):metric-tree} and \ref{SHORT.ex:Asym(Lob):continuum} describe the tree $\spc{T}$ up to isometry.
In particular, the asymptotic space of Lobachevsky plane does not depend on the choice of ultrafilter and the sequence $\lambda_n\to \infty$.


\section{Remarks}

A nonprinciple ultrafilter $\omega$ is called 
\emph{selective}\index{ultrafilter!selective ultrafilter}\index{selective ultrafilter} if for any partition of $\NN$ into sets $\{C_\alpha\}_{\alpha\in\IndexSet}$ such that $\omega(C_\alpha)\z=0$ for each $\alpha$, 
there is a set $S\subset \NN$ such that $\omega(S)=1$ and $S\cap C_\alpha$ is a one-point set for each $\alpha\in\IndexSet$.

The existence of a selective ultrafilter follows from the continuum hypothesis;
it was proved by Walter Rudin \cite{rudin}.

For a selective ultrafilter $\omega$, there is a stronger version of Proposition~\ref{prop:ultra/partial};
namely we can assume that the subsequence $(x_n)_{n\in S}$ can be chosen so that $\omega(S)=1$.
So, if needed, one may assume that the ultrafilter $\omega$ is chosen to be selective and use this stronger version of the proposition.

%\chapter{Metric plus measure}

\section{Borel sets}

Let us remind few definitions assuming knowleage of basic measure theory;
comprehensive treatments can be found in \cite{billingsley} and \cite{bogachev}.

Let $\spc{X}$ be a metric space.
\index{Borel set}\emph{Borel set} is any subset of $\spc{X}$ that can be formed from open sets using the countable union, countable intersection, and complement.
In other words, Borel sets form the minimal sigma-algebra that included open sets.

A measure on metric space will be always assumed to be \index{Borel measure}\emph{Borel};
that is, it is defined on the sigma-algebra of Borel sets.
A Borel measure can be uniquely determined by its values on all open (or closed) sets.

A measure $\mu$ on $\spc{X}$ is called \index{probability measure}\emph{probability measure} if $\mu\spc{X}=1$.

Recall that \index{delta-measure}\emph{delta-measure} is a probability measure with support at one point.
Delta-measure with support in $\{x\}$ will be denoted by~\index{$\delta_{x}$}$\delta_{x}$; so
\[\text{if}\quad x\in A,\quad\text{then}\quad \delta_x(A)=1,\quad\text{otherwise}\quad\delta_x(A)=0.\]

Let $\mu_n$ be a sequence of Borel measures on $\spc{X}$.
A measure $\mu_\infty$ is a \index{weak limit}\emph{weak limit} of $\mu_n$ if 
\[\int_{\spc{X}}f\cdot(\mu_n-\mu_\infty)\to0\gamma
\quad\text{as}\quad
n\to\infty
\]
for any continuous function $f\:\spc{X}\to \RR$.

Suppose $\mu$ is a measure on a metric space $\spc{X}$ and $f\:\spc{X}\to \spc{Y}$ is a measurable map;
that is, for any Borel set $B\subset \spc{Y}$, its inverse image $f^{-1}B$ is a Borel set in $\spc{Y}$.

Consider the unit interval with its Lebesgue mesure.
If $\spc{X}$ is a complete separable metric space with probability measure $\mu$, then there is a measurable map $[0,1]\to \spc{X}$

\section{Metric on measures}

Imagine that we need to transport dirt from one pile of a given shape to make another pile of a needed shape.
Suppose that cost of transporting a unit of dirt equals to the traveled distance.%
\footnote{This is the simplest cost function one can imagine.
One may consider other cost functions; for example, if the cost proportional to the square of the distance, then the problem has more applications.}
We are free to choose a destination point for dirt from a given place.
How to minimize the total cost?

To formalize this question,
suppose that the piles of dirt are described by Borel probability measures $\mu$ and $\nu$ on a metric space~$\spc{X}$.

To describe where each piece of dirt goes, we will use the so called \index{plan}\emph{plan} for $\mu$ and $\nu$.
It is a probability measure $\pi$ on the product $\spc{X}\times\spc{X}$ such that 
for all measurable sets $A \subset \spc{X}$, we have 
\[\mu A= \pi [A \times \spc{X}],
\quad\text{and}\quad
\nu A=\pi [\spc{X}\times A].
\eqlbl{eq:marginals}\]
Equivalently it can be described as a measure that satisfies the following identity
\begin{align*}
\int_{(x,y)\in \spc{X}\times\spc{X}}f(x)\cdot g(y) \cdot \pi
&=
\int_{x\in \spc{X}}f(x)\cdot \mu
\oldcdot \int_{y\in \spc{X}}g(y)\cdot \nu,
\end{align*}
for any continuous functions $f,g\:\spc{X}\to \RR$.

Given a measure $\pi$ on $\spc{X}\times\spc{X}$, the measures $\mu$ and $\nu$ defined by \ref{eq:marginals} are called first and second \index{marginal}\emph{marginals} of $\pi$;
so the statement \textit{$\pi$ is a plan for $\mu$ and $\nu$} is equivalent to \textit{$\mu$ and $\nu$ are the first and second marginals of $\pi$ respectively}.

\begin{thm}{Claim}\label{clm:plan-exists}
There is a plan $\pi$ for any two given Borel probability measures $\mu$ and $\nu$.
\end{thm}

The plan constructed in the proof distributes equally each piece of dirt in the new pile.
As we will see this plan is usually far from optimum.

\parit{Proof.}
Consider the measure $\pi$ that is uniquely defined  defined by the identity
\[\pi(A\times B)=\mu A\cdot \mu B\]
for any Borel subsets $A,B\subset\spc{X}$.
Observe that $\pi$ is a plan for $\mu$ and~$\nu$.
\qeds

Denote by $\Pi(\mu,\nu)$ the set of all plans for $\mu$ and $\nu$;
by \ref{clm:plan-exists}, $\Pi(\mu,\nu)\z\ne\emptyset$.
It is straightforwrd to check that the following formula defines a metric on the space of probability measures on $\spc{X}$.
\[\dist{\mu}{\nu}{\Wass_1\spc{X}}
\df
\inf_{\pi\in\Pi(\mu,\nu)}
\left\{\,\int_{(x,y)\in\spc{X}\times\spc{X}}\dist{x}{y}{\spc{X}}\cdot\pi\,\right\}.\]
This metric is called \index{Wasserstein distance}\emph{Wasserstein distance of order 1} between $\mu$ and $\nu$.

In genereral, the Wasserstein distance $\dist{\mu}{\nu}{}$ might take infinite value, but all measures with compact support lie on finite distance from each other in the obtained $\infty$-metric space.
The metric component of these measures is called \index{Wasserstein space}\emph{Wasserstein space} of order 1 over $\spc{X}$; 
it is denoted by $\Wass_1\spc{X}$.
In other words, $\Wass_1\spc{X}$ is the space of all Borel probability measures $\mu$ such that 
$\int\distfun_p\cdot\mu<\infty$ for some (and therefore any) point $p\in \spc{X}$.

\begin{thm}{Exercise}\label{ex:wasserstein-infty}
Construct two Borel probability measures $\mu$ and $\nu$ on $\RR$ with Wasserstein distance $\dist{\mu}{\nu}{}=\infty$.
\end{thm}


\begin{thm}{Exercise}\label{ex:wasserstein-compact}
Show that $\Wass_1\spc{X}$ is a compact if and only if so is~$\spc{X}$.
\end{thm}

\begin{thm}{Exercise}\label{ex:wasserstein-length}
Show that the Wasserstein space $\Wass_1\spc{X}$ is a geodesic space for any metric space $\spc{X}$.
\end{thm}

\section{Optimal plan}

A plan $\pi$ for given measures $\mu$ and $\nu$ is called \index{optimal plan}\emph{optimal} if 
\[\dist{\mu}{\nu}{\Wass_1\spc{X}}
=\int_{(x,y)\in\spc{X}\times\spc{X}}\dist{x}{y}{\spc{X}}\cdot\pi.\]

\begin{thm}{Theorem} %Vilani:Theorem 1.4
Let $\mu$ and $\nu$ be probability Borel measures on a compact metric space $\spc{X}$.
Then there is an optimal plan $\pi$ for $\mu$ and~$\nu$.
\end{thm}

\parit{Proof.}
By the definition of Wasserstein distance, we can choose a sequence of plans $\pi_n$ for $\mu$ and $\nu$ such that 
\[\int_{(x,y)\in\spc{X}\times\spc{X}}\dist{x}{y}{\spc{X}}\cdot\pi_n\to \dist{\mu}{\nu}{\Wass_1\spc{X}}\]
as $n\to \infty$.

Observe that $\pi_n$ has a weak partial limit, say $\pi$.
Moreover $\pi$ is an optimal plan for $\mu$ and $\nu$.
\qeds

\begin{thm}{Theorem}
Any optimal plan $\pi$ is \index{cyclic monotonicity}\emph{cyclically monotonic}.
That is, suppose $\pi$ is an optimal plan for probability measures $\mu$ and $\nu$ on a metric space $\spc{X}$.
Then any sequence of pairs $(x_1,y_1),\dots,(x_n,y_n)\in\supp\pi\subset\spc{X}\times\spc{X}$ we have
\[\sum_i\dist{x_i}{y_i}{}
\le
\sum_i\dist{x_{i+1}}{y_i}{},\]
here the index $i$ in the sum is taken modulo $n$; in particular $x_{n+1}\z=x_1$.
\end{thm}

\parit{Proof.}
Assume that the cyclic monotonicity does not hold;
that is,
\[R=\sum_i\dist{x_i}{y_i}{}
-
\sum_i\dist{x_{i+1}}{y_i}{}>0,\]
for some $(x_0,y_0),\dots,(x_n,y_n)\in\supp\pi$.
We need to show that $\pi$ is not optimal;
in other words we need to construct another plan $\pi'$ for $\mu$ and $\nu$ such that 
\[\int_{(x,y)\in\spc{X}\times\spc{X}}\dist{x}{y}{\spc{X}}\cdot(\pi'-\pi)<0.\eqlbl{pi'<pi}\]

Assume $\spc{X}$ is finite.
In this case we can choose $\eps>0$ such that 
$\pi\{(x_i,y_i)\}>\eps$ for each $i$.
Let
\[\pi'=\pi-\eps\cdot\sum_i(\sigma_i-\sigma_i')\eqlbl{eq:pi'}\]
where $\sigma_i=\delta_{(x_i,y_i)}$ and $\sigma_i'=\delta_{(x_{i+1},y_i)}$.
It remains to observe that $\pi'$ is a plan for $\mu$ and $\nu$ that satisfies \ref{pi'<pi}.

The general case is similar, we only need to redefine $\eps$, $\sigma_i$, and~$\sigma_i'$.
Note that given $r>0$, we can choose a probability measures $\sigma_i$ with support in $\oBall((x_i,y_i),r)_{\spc{X}\times\spc{X}}$ such that $\eps\cdot \sigma_i<\pi$ for some fixed $\eps>0$ and every $i$.
Further denote by $\zeta_i$ and $\eta_i$ the first and second marginals of $\sigma_i$.
Observe that $\supp\zeta_i\subset\oBall(x_i,r)$ and $\supp\eta_i\subset\oBall(y_i,r)$ for all $i$.
Let $\sigma_i'$ be a plan for $\zeta_{i+1}$ and $\eta_i$.
Evidently 
\begin{align*}
\int_{(x,y)\in\spc{X}\times\spc{X}}\dist{x}{y}{}\cdot \sigma_i
&\lessgtr
\dist{x_i}{y_i}{}\pm 2\cdot r,
\\
\int_{(x,y)\in\spc{X}\times\spc{X}}\dist{x}{y}{}\cdot \sigma_i'
&\lessgtr
\dist{x_{i+1}}{y_i}{}\pm 2\cdot r.
\end{align*}
Taking $r<\tfrac R{10\cdot n}$, we get  \ref{pi'<pi}. 
\qeds




\section{Capitalistic approach}

Imagine that measures $\mu$ and $\nu$ describe the production and consumer of beer in the space.
A transportaition company transports beer from $\mu$ to $\nu$ and want to maximize its profit by adjusting price $f(x)$ of beer the point $x$; they buy beer at price $f(x)$ per unit, move it to an other point $y$ and sale it with (presumably higher) price $f(y)$.
However, the function $f$ is 1-Lipschitz condition;
otherwise the profit goes to second-hand dealers, or maybe it is a government regulation.
In other words we need to maximize the following expression
\[\int_{\spc{X}} f\cdot(\mu-\nu)\]
for all $1$-Lipschitz functions $f$.
The maximal profit defines a metric

\begin{thm}{Theorem}
Let $\mu$ and $\nu$ be probability Borel measures on a compact metric space $\spc{X}$.
Then
\[\dist{\mu}{\nu}{\Wass_1\spc{X}}=\sup\int_{\spc{X}} f\cdot(\mu-\nu),\]
where the least upper bound is taken for all $1$-Lipschitz functions $f\:\spc{X}\z\to\RR$.
\end{thm}

The definition of Wassershtein metric described in the previous section reminds communist's planed economy.
The right-hand side in the above equation reminds capitalistic system.
Indeed, think that measures $\mu$ and $\nu$ describe the production and consumer of beer in the space.
A transportaition company trnasports beer from $\mu$ to $\nu$ and want to maximize its profit by adjusting price $f(x)$ of beer the point $x$.
However, the function $f$ is 1-Lipschitz condition --- this is a government regulation.




\parit{Proof.}
By the definition of Wasserstein metric, we can choose a sequence $\pi_n$ of plans  

Let us choose an optimal plan $\pi$ for $\mu$ and $\nu$; it exists by ???.
We need to find a 1-Lipschitz function $f\:\spc{X}\to\RR$ such that 
\[
\int_{\spc{X}} f\cdot(\mu-\nu)=\int_{(x,y)\in\spc{X}\times\spc{X}}\dist{x}{y}{\spc{X}}\cdot \pi.
\eqlbl{eq:f(mu-nu)}
\]

Choose $x_0\in \supp\mu$.
Note that adding a constant to $f$ does not change the left hand side in \ref{eq:f(mu-nu)}.
Therefore we can assume assume that $f(x_0)=0$ and set
\[f(x)=\sup\{\,|x_0-y_0|+\dots+|x_n-y_n|-(|x_1-y_0|+\dots+|x_n-y_{n-1}|)-|x-y_n|\,\}\]
where the least upper bound is taken for all sequences $(x_0,y_0),\z\dots,(x_n,y_n)\z\in\supp\pi$.

\qeds

\section{Metric-measure space}

A metric measure space is a metric $\spc{X}$ space with a choice of Borel probability measure $\vol$ on it.
In a metric-measure we ignore sets with vanishing volume; in other words, passing from $\spc{X}$ to the support of $\vol$ does not change the metric-measure space.

Alternatively we may start with unit interval $[0,1]$ equipped with Lebesgue measure and equip it with measurable pseudometric $[0,1]\times [0,1]\to \RR$.





\section{Space of measures}


It can be equipped with the \index{Wasserstein metric}\emph{Wasserstein metric}
\[\dist{\mu}{\nu}{}\df\sup\left\{\,\int_{\spc{X}} f\cdot(\mu-\nu)\,\right\},\]
where the least upper bound is taken for all $1$-Lipschitz functions $f\:\spc{X}\to\RR$.

The Wasserstein distance $\dist{\mu}{\nu}{}$ might take infinite value, but all measures with compact support lie on finite distance from each other in the obtained $\infty$-metric space.
The metric component of these measures is called \index{Wasserstein space}\emph{Wasserstein space} of order 1 over $\spc{X}$; 
it is denoted by $\Wass_1\spc{X}$.



\section{Misc}

Suppose $\pi_n$ is a sequence of plans for $\mu$ and $\nu$.
Assume that $\pi_n$ weakly converges to a probability measure $\pi$ on $\spc{X}\times\spc{X}$.

is a weak limit of a sequence of plans $\pi_n$, then $\pi$ is a plan for $\mu$ and $\nu$ if for each $n$ $\pi_n$ is a plane for $\mu$ and $\nu$ 

Suppose that $f\:\spc{X}\to \RR$ is a 1-Lipschitz function,
so $f(x)-f(y)\le\dist{x}{y}{\spc{X}}$ for any $x,y\in \spc{X}$.
It follows that 
\begin{align*}
\int_{\spc{X}} f\cdot(\mu-\nu)&=\int_{x\in\spc{X}}f(x)\cdot\mu-\int_{y\in\spc{X}}f(y)\cdot\nu=
\\
&=\int_{(x,y)\in\spc{X}\times\spc{X}}[f(x)-f(y)]\cdot \pi\le
\\
&\le\int_{(x,y)\in\spc{X}\times\spc{X}}\dist{x}{y}{\spc{X}}\cdot \pi,
\end{align*}
where $\pi$ is a plan for $\mu$ and $\nu$.
By the definition of Wasserstein metric, we get  
\[\dist{\mu}{\nu}{\Wass_1\spc{X}}\le \int_{(x,y)\in\spc{X}\times\spc{X}}\dist{x}{y}{\spc{X}}\cdot\pi\eqlbl{wass=<int.plan}\]
for any plan $\pi$.

Next we want to show that equality holds in \ref{wass=<int.plan} for some plan $\pi$; such plans will be called \index{optimal plan}\emph{optimal}.


\parit{Proof.}
Choose a point $x_0\in \supp\mu$.
Given  $p\in \spc{X}$,
let
\[f(p)=\inf\left\{\sum_{i=0}^n\dist{x_i}{y_i}{}-\sum_{i=0}^n\dist{x_{i+1}}{y_i}{}-\dist{y_n}{p}{}\right\},
\eqlbl{eq:f(p)}\]
where the least upper bound is taken for all sequences of pairs 
\[(x_0,y_0),\z\dots,(x_n,y_n)\in \supp\pi.\eqlbl{eq:sequence}\]

Fix a sequence as in \ref{eq:sequence} and  denote by $F_\sigma(p)$ the expression under infimum in \ref{eq:f(p)}.

Let us show that 
\[F_\sigma(x_0)\ge 0.\]
Indeed, suppose $F_\sigma(x_0)<-\eps<0$.
Since $(x_i,y_i)\in \supp\pi$, we have $x_i\in\supp\mu$ and $y_i\in\supp\nu$ for any $i$.
Therefore we can choose sets $X_i\subset \oBall(x_i,\tfrac{\eps}{10\cdot n})$ and $Y_i\subset \oBall(y_i,\tfrac{\eps}{10\cdot n})$ such that 
$\mu(X_0)=\nu(Y_0)=\dots=\mu(X_n)=\nu(Y_n)$



Let us denote by $F(p)$ the expression under infimum in \ref{eq:f(p)}.
By triangle inequality, 
\[F(q)\le F(p)+\dist{p}{q}{}.\]
Passing to the least upper bound in this inequality, we get
\[f(q)\le f(p)+\dist{p}{q}{}\]
for any $p,q\in\spc{X}$.
Hence $f$ is a 1-Lipschitz function.

Further, let us show that
\[(x,y)\in\supp\pi
\quad\Longrightarrow\quad
f(y)-f(x)=\dist{x}{y}{}\]





Suppose that cyclic monotonicity fails;
that is, there is a sequence of pairs $(x_1,y_1),\dots,(x_n,y_n)\in\spc{X}\times\spc{X}$ such that
\[\dist{x_1}{y_1}{}+\dots+\dist{x_n}{y_n}{}
>
\dist{x_1}{y_2}{}+\dots+\dist{x_{n-1}}{y_n}{}+\dist{x_{n}}{y_1}{}.\]
In this case, it would be more optimal to transport measure from a neighborhood of $x_i$ to a neighborhood of $y_{i+1}$ (
here and further we assume that the indexes are taken modulo $n$, so $n+1=1$).
The latter contradicts optimality of $\pi$.

The following argument makes it precise.
Choose small $\eps>0$.
For each $n$,
choose disjoint sets $X_i$ and $Y_i$ in $\eps$-neighborhood of $x_i$ and $y_i$
such that for some $\delta>0$ we have 
\[\pi [X_i\times Y_i]=\delta\]
for each $i$.

Let us modify the plan $\pi$ in the union $X_1\times Y_1 \cup\dots\cup X_n\times Y_n$ and such that 
$\pi'(X_i\times Y_{i+1})=\delta$ for each $i$;


Observe that
\[\int_{(x,y)\in\spc{X}\times\spc{X}}\dist{x}{y}{\spc{X}}\cdot(\pi'-\pi)>\]
\qeds


\appendix
\chapter{Semisolutions}
\parbf{\ref{ex:almost-min}.}
Assume the statement is wrong. 
Then for any point $x\in \spc{X}$, there is a point $x'\in \spc{X}$ such that 
\[\dist{x}{x'}{}< \rho(x)
\quad\text{and}\quad
\rho(x')\le\frac{\rho(x)}{1+\eps}.\]
Consider a sequence of points $(x_n)$ such that $x_{n+1}\z=x_n'$.
Clearly 
\[\dist{x_{n+1}}{x_n}{}
\le
\frac{\rho(x_0)}{\eps\cdot(1+\eps)^n}
\quad\hbox{and}\quad
\rho(x_n)\le \frac{\rho(x_0)}{(1+\eps)^n}.\] 
Therefore $(x_n)$ is a Cauchy sequence.
Since $\spc{X}$ is complete, the sequence $(x_n)$ converges;
denote its limit by $x_\infty$.
Since $\rho$ is a continuous function we get
\begin{align*}\rho(x_\infty)&=\lim_{n\to\infty}\rho(x_n)=
\\&=0.
\end{align*}

The latter contradicts that $\rho>0$.


\parbf{\ref{ex:non-contracting-map}.}
Given any pair of point $x_0,y_0\in \spc{K}$, 
consider two sequences $x_0,x_1,\dots$ and $y_0,y_1,\dots$
such that $x_{n+1}=f(x_n)$ and $y_{n+1}\z=f(y_n)$ for each $n$.

Since $\spc{K}$ is compact, 
we can choose an increasing sequence of integers $n_k$
such that both sequences $(x_{n_i})_{i=1}^\infty$ and $(y_{n_i})_{i=1}^\infty$
converge.
In particular, both are Cauchy;
that is,
\[
|x_{n_i}-x_{n_j}|_{\spc{K}}, |y_{n_i}-y_{n_j}|_{\spc{K}}\to 0
\quad
\text{as}
\quad\min\{i,j\}\to\infty.
\]


Since $f$ is non-contracting, we get
\[
|x_0-x_{|n_i-n_j|}|
\le 
|x_{n_i}-x_{n_j}|.
\]

It follows that  
there is a sequence $m_i\to\infty$ such that
\[
x_{m_i}\to x_0\quad\text{and}\quad y_{m_i}\to y_0\quad\text{as}\quad i\to\infty.
\leqno({*})\]

Set \[\ell_n=|x_n-y_n|_{\spc{K}}.\]
Since $f$ is non-contracting, the sequence $(\ell_n)$ is nondecreasing.

By $({*})$,  $\ell_{m_i}\to\ell_0$ as $m_i\to\infty$.
It follows that $(\ell_n)$ is a constant sequence.

In particular 
\[|x_0-y_0|_{\spc{K}}=\ell_0=\ell_1=|f(x_0)-f(y_0)|_{\spc{K}}\]
for any pair of points $(x_0,y_0)$ in $\spc{K}$.
That is, $f$ is distance-preserving, in particular injective.

From $({*})$, we also get that $f(\spc{K})$ is everywhere dense.
Since $\spc{K}$ is compact $f\:\spc{K}\to \spc{K}$ is surjective. Hence the result follows.

\parit{Remarks.}
This is a basic lemma in the introduction to Gromov--Hausdorff distance \cite[see 7.3.30 in][]{burago-burago-ivanov}.
This proof is not quite standard,
I learned this proof from Travis Morrison, 
a student in my MASS class at Penn State, Fall 2011.

Note that as an easy corollary one can see that any surjective non-expanding map from a compact metric space to itself is an isometry. 

\parbf{\ref{ex:pogorelov}.}
The conditions \ref{SHORT.metric>=0}--\ref{SHORT.metric:sym} in Definition \ref{def:metric} are evident.

The triangle inequality \ref{SHORT.metric:triangle} follows since
\[[B(x,\tfrac \pi2)\backslash B(y,\tfrac\pi2)]
\cup 
[B(y,\tfrac\pi2)\backslash B(z,\tfrac\pi2)]
\supseteq
B(x,\tfrac \pi2) \backslash B(z,\tfrac\pi2).
\leqno(*)\]

\begin{wrapfigure}[8]{o}{31 mm}
\vskip-2mm
\centering
\includegraphics{mppics/pic-29}
\end{wrapfigure}

Observe that
$B(x,\tfrac \pi2)\backslash B(y,\tfrac\pi2)$
does not overlap
$B(y,\tfrac\pi2)\backslash B(z,\tfrac\pi2)$ and  we get equality in $(*)$ if and only if $y$ lies on the great circle arc from $x$ to $z$.
Therefore the second statement follows.


\parit{Remarks.}
This construction was given by 
Aleksei Pogorelov \cite{pogorelov}.
It is closely related to the construction given 
by David Hilbert in \cite{hilbert}
which was the motivating example for his 4-th problem.


\parbf{\ref{ex:no-geod}.}
We assume that the space is nontrivial, otherwise a one-point space is an example.

Consider the unit ball $(B,\rho_0)$
in the space $c_0$ of all sequences converging to zero equipped with the sup-norm.

Consider another metric $\rho_1$ which is different from $\rho_0$ by the conformal factor
\[\phi(\bm{x})=2+\tfrac{1}2\cdot x_1+\tfrac{1}4\cdot x_2+\tfrac{1}8\cdot x_3+\dots,\]
where $\bm{x}=(x_1,x_2\,\dots)\in B$.
That is, if $\bm{x}(t)$, $t\in[0,\ell]$, is a curve parametrized by $\rho_0$-length 
then its $\rho_1$-length is defined by
\[\length_{\rho_1}\bm{x}\df\int\limits_0^\ell\phi\circ\bm{x}(t)\cdot dt.\]
Note that the metric $\rho_1$ is bi-Lipschitz to~$\rho_0$.

Assume $\bm{x}(t)$ and $\bm{x}'(t)$ are two curves parametrized by $\rho_0$-length that differ only in the $m$-th coordinate, denoted by $x_m(t)$ and $x_m'(t)$ respectively.
Note that if $x'_m(t)\le x_m(t)$ for any $t$ and 
the function $x'_m(t)$ is locally $1$-Lipschitz at all $t$ such that $x'_m(t)< x_m(t)$, then 
\[\length_{\rho_1}\bm{x}'\le \length_{\rho_1}\bm{x}.\]
Moreover this inequality is strict if $x'_m(t)< x_m(t)$ for some~$t$.

Fix a curve $\bm{x}(t)$, $t\in[0,\ell]$, parametrized by  $\rho_0$-length.
We can choose $m$ large, so that $x_m(t)$ is sufficiently close to $0$ for any~$t$.
In particular, for some values $t$, we have $y_m(t)<x_m(t)$, where
\[y_m(t)=(1-\tfrac t\ell)\cdot x_m(0)
+\tfrac t\ell\cdot x_m(\ell)
-\tfrac 1{100}\cdot \min\{t,\ell-t\}.\]
Consider the curve $\bm{x}'(t)$ as above with
\[x'_m(t)=\min\{x_m(t),y_m(t)\}.\]
Note that $\bm{x}'(t)$ and $\bm{x}(t)$ have the same end points, and by the above
\[\length_{\rho_1}\bm{x}'<\length_{\rho_1}\bm{x}.\]
That is, for any curve $\bm{x}(t)$ in $(B,\rho_1)$, we can find a shorter curve $\bm{x}'(t)$ with the same end points.
In particular, $(B,\rho_1)$ has no geodesics.

\parit{Remarks.}
This solution was suggested by Fedor Nazarov~\cite{nazarov}.

\parbf{\ref{ex:compact+connceted}.}
Choose a sequence $\varepsilon_n\to 0$ and a $\varepsilon_n$-net $N_n$ of $K$ for each $n$.
Assume $N_0$ is a one-point set, so $\eps_0>\diam K$.
Connect each point $x\in N_{k+1}$ to a point $y\in N_{k}$ by a curve of length at most $\eps_k$.

Consider the union $K'$ of all these curves with $K$; observe that $K'$ is compact and path connected.

\parit{Source:} This problem was suggested by Eugene Bilokopytov \cite{bilokopytov}.

\parbf{\ref{ex:compact=>complete}.}
Choose a Cauchy sequence $(x_n)$ in $(\spc{X},\|*-*\|)$; it sufficient to show that a subsequence of $(x_n)$ converges.

Note that the sequence $(x_n)$ is Cauchy in $(\spc{X},|*-*|)$;
denote its limit by $x_\infty$.

After passing to a subsequence, we can assume that $\|x_n-x_{n+1}\|\z<\tfrac1{2^n}$.
It follows that there is a 1-Lipschitz path $\gamma$ in $(\spc{X},\|*-*\|)$ such that $x_n=\gamma(\tfrac1{2^n})$ for each $n$ and $x_\infty=\gamma(0)$.

It follows that
\begin{align*}
\|x_\infty-x_n\|&\le \length\gamma|_{[0,\frac1{2^n}]}\le
\\
&\le \tfrac1{2^n}.
\end{align*}
In particular $x_n$ converges.

\parit{Source:} \cite[Lemma 2.3]{petrunin-stadler}.


\begin{wrapfigure}{r}{20 mm}
\vskip-0mm
\centering
\includegraphics{mppics/pic-1}
\end{wrapfigure}

\parbf{\ref{exercise from BH}.}
Consider the following subset of $\RR^2$ equipped with the induced length metric
\[
\spc{X}
=
\bigl((0,1]\times\{0,1\}\bigr)
\cup
\bigl(\{1,\tfrac12,\tfrac13,\dots\}\times[0,1]\bigr)
\]
Note that $\spc{X}$ is locally compact and geodesic.

Its completion $\bar{\spc{X}}$ is isometric to the closure of $\spc{X}$ equipped with the induced length metric.
Note that $\bar{\spc{X}}$ is obtained from $\spc{X}$ by adding two points $p=(0,0)$ and $q=(0,1)$.

Observe that the point $p$ admits no compact neighborhood in $\bar{\spc{X}}$ 
and there is no geodesic connecting $p$ to $q$ in~$\bar{\spc{X}}$. 

\parit{Source:} \cite[I.3.6(4)]{bridson-haefliger}.

%%%%%%%%%%%%%%%%%%%%%%%%%%%%%%



\parbf{\ref{ex:gross}.}
If such a number does not exist, then the ranges of average distance functions have empty intersection.
Since $\spc{X}$ is a compact length-metric space, the range of any continuous function on $\spc{X}$ is a closed interval.
By 1-dimensional Helly's theorem, there is a pair of such range intervals that do not intersect.
That is, for two point-arrays $(x_1,\dots,x_n)$ and $(y_1,\dots,y_m)$
and their average distance functions 
\[f(z)=\tfrac1n\cdot\sum_i|x_i-z|_{\spc{X}}\quad\text{and}\quad h(z)=\tfrac1m\cdot\sum_j|y_j-z|_{\spc{X}},\] we have 
$$\min\set{f(z)}{z\in \spc{X}}>\max\set{h(z)}{z\in \spc{X}}.\leqno({*})$$

Note that 
$$\tfrac1m\cdot\sum_j f(y_j)=\tfrac1{m\cdot n}\cdot\sum_{i,j}|x_i-y_j|_{\spc{X}}=\tfrac1n\cdot\sum_i h(x_i);$$
that is, the average value of $f(y_j)$ coincides with the average value of $h(x_i)$, 
which contradicts $({*})$.

\parit{Remarks.}
In fact the value $\ell$ is uniquely defined;
it is called the \index{rendezvous value}\emph{rendezvous value} of ${\spc{X}}$.
This is a result of Oliver Gross \cite{gross}.

\parbf{\ref{ex:wasserstein}.}
Choose a finite $\eps$-net $F\subset\spc{X}$.
Show that the space $P_F$ of probability measures with support in $F$ is a compact net in $\Wass_1\spc{X}$.
Observe that $\Wass_1\spc{X}$ is complete; 
by \ref{ex:compact-net}, it follows that $\Wass_1\spc{X}$ is compact.

Show that 

Choose an integer $n$.
Consider the set of probability measures $P_n$ of the form 
\[\tfrac1n\cdot\sum_{i=1}^n\delta_{x_i},\]
where $\delta_{x_i}$ denotes the delta-measure supported at $x_i\in\spc{X}$. 

Show that $P_n$ is a compact subset of $\Wass_1\spc{X}$.
Moreover, for any $\eps>0$ there is $n$ such that $P_n$ is 




%%%%%%%%%%%%%%%%%%%%%%%%%

\parbf{\ref{ex:compact-length}.} By Fréchet lemma (\ref{lem:frechet}) we can identify $\spc{K}$ with a compact subset of $\ell^\infty$.

Denote by $\spc{L}=\Conv\spc{K}$ --- it is defined as the minimal convex closed set in $\ell^\infty$ that contains $\spc{K}$.
(In other words, $\spc{L}$ is the intersection of all convex closed sets that contain $\spc{K}$.)

Observe that $\spc{L}$ is a length space.
It remains to show that $\spc{L}$ is compact.

By construction $\spc{L}$ is a closed subset of $\ell^\infty$, in particular it is a complete space.
By \ref{totally-bounded}, it remains to show that $\spc{L}$ is totally bounded.

Recall that Minkowski sum $A + B$ of two sets $A$ and $B$ in a vector space is defined by
\[A + B = \set{a+b}{a\in A,\ b\in B}.\]
Observe that the Minkowski sum of two convex sets is convex.

Denote by $\bar B_\eps$ the closed $\eps$-ball in $\ell^\infty$ centered at the origin.
Choose a finite $\eps$-net $N$ in $\spc{K}$ for some $\eps>0$.
Note that $P=\Conv N$ is a convex polyhedron, in particular $\Conv N$ is compact.

Observe that $N+\bar B_\eps$ is closed $\eps$-neighborhood of $N$.
It follows that $N+\bar B_\eps\supset K$ and therefore $P+\bar B_\eps\supset \spc{L}$.
In particular $P$ is a $2\cdot\eps$-net in $\spc{L}$;
since $P$ is compact and $\eps>0$ is arbitrary, $\spc{L}$ is totally bounded (see \ref{ex:compact-net}).

\parit{Remark.}
Another solution follows since the injective envelope of a compact space is compact; see \ref{ex:inj=complete-geodesic-contractible:geodesic}, \ref{ex:Inj(compact)}, and \ref{prop:InjX-is-injective}.

\parbf{\ref{ex:geodesics-urysohn}.}
Choose a separable space $\spc{X}$ that has an infinite number of geodesics between a pair of points, say a square will $\ell^\infty$-metric in $\RR^2$.
Apply to $\spc{X}$ universality of Urysohn space (\ref{prop:sep-in-urys}).

\parbf{\ref{ex:compact-extension}.} 
First let us prove the following claim:

\begin{itemize}
\item 
Suppose $f\: K\to\RR$ is an extension function defined on a compact subset $K$ of the Urysohn space $\spc{U}$.
Then there is a point $p\in \spc{U}$ such that 
$\dist{p}{x}{}=f(x)$ for any $x\in K$.
\end{itemize}

Without loss of generality we may assume that $f(x)>0$ for any $x\in K$.
Since $K$ is compact, we may fix $\eps>0$ such that $f(x)>\eps$.

Consider the sequence $\eps_n=\tfrac\eps{100\cdot 2^n}$.
Choose a sequence of $\eps_n$-nets $N_n\subset K$.
Applying universality of $\spc{U}$ recursively, we may choose a point $p_n$ such that $\dist{p_n}{x}{}=f(x)$ for any $x\in N_n$ and $\dist{p_n}{p_{n-1}}{}\z=10\cdot\eps_{n-1}$.
Observe that the sequence $(p_n)$ is Cauchy and its limit $p$ meets 
$\dist{p}{x}{}=f(x)$ for any $x\in K$.

Now, choose a sequence of points $(x_n)$ in $\spc{S}$.
Applying the claim, we may extend the map from $K$ to $K\cup\{x_1\}$, and further to $K\cup\{x_1,x_2\}$, and so on.
As a result we extend the distance-preserving map $f$ to the whole sequence $(x_n)$.
It remains to extend it continuously to the whole space~$\spc{S}$.

\parbf{\ref{ex:sc-urysohn}.}
It is sufficient to show that any compact subspace $\spc{K}$ of Urysohn space can be contracted to a point.

Note that any compact space $\spc{K}$ can be extended to a contractible compact space $\spc{K}'$; for example we may embed $\spc{K}$ into $\ell^\infty$ and pass to its convex hull, as it was done in \ref{ex:compact-length}.

By \ref{thm:compact-homogeneous}, there is an isometric embedding of $\spc{K}'$ that agrees with inclusion of $\spc{K}$.
Since $\spc{K}$ is contractible in $\spc{K}'$, it is contractible in $\spc{U}$.

\parit{A better way.}
One can contract the whole Urysohn space using the following construction.

Note that points in the space $\spc{X}_\infty$ constructed in the proof of \ref{prop:univeral-separable} can be multiplied number $t\in [0,1]$ --- simply multiply each function by $t$.
That defines a map 
\[\lambda_t\:\spc{X}_\infty\to \spc{X}_\infty\]
that scales all distances by factor $t$.
The map $\lambda_t$ can be extended to the completion of $\spc{X}_\infty$, which is isometric to $\spc{U}_d$ (or $\spc{U}$).

Observe that 
the map $\lambda_1$ is the identity  and $\lambda_0$ maps whole space to a single point, say $x_0$ --- this is the only point of $\spc{X}_0$.
Further note that the map $(t,p)\mapsto \lambda_t(p)$ is continuous ---  in particular $\spc{U}_d$ and $\spc{U}$ are contractible.

As a bonus, observe that for any point $p\in \spc{U}_d$ the curve $t\mapsto \lambda_t(p)$ is a geodesic path from $p$ to $x_0$.

\parit{Source:} \cite[(d) on page 82]{gromov-2007}.

\parbf{\ref{ex:sphere-in-urysohn}}; \ref{SHORT.ex:sphere-in-urysohn:sphere} and \ref{SHORT.ex:sphere-in-urysohn:midpoint}.
Observe that $L$ and $M$ satisfy the definition of $d$-Urysohn space and apply the uniqueness (\ref{thm:urysohn-unique}).

\parit{\ref{SHORT.ex:sphere-in-urysohn:homogeneous}.} 
Use \ref{SHORT.ex:sphere-in-urysohn:sphere}, maybe twice.

\parbf{\ref{ex:homogeneous}}; \ref{SHORT.ex:homogeneous:euclidean}.
The euclidean plane is homogeneous in every sense.

\parit{\ref{SHORT.ex:homogeneous:hilbert}.} The ilbert space $\ell^2$ is finite set homogeneous, but not compact set homogeneous, nor countable homogeneous.

\parit{\ref{SHORT.ex:homogeneous:ell-infty}.} The space $\ell^\infty$ is 1-point homogeneous, but not 2-point homogeneous.
Try to show that there is no isometry of $\ell^\infty$ such that
\begin{align*}
(0,0,0,\dots)&\mapsto (0,0,0,\dots),
\\
(1,1,1,\dots)&\mapsto (1,0,0,\dots).
\end{align*}

\parit{\ref{SHORT.ex:homogeneous:ell-1}.}
The space $\ell^1$ is 1-point homogeneous, but not 2-point homogeneous.
Try to show that there is no isometry of $\ell^\infty$ such that
\begin{align*}
(0,0,0,\dots)&\mapsto (0,0,0,\dots),
\\
(2,0,0\dots)&\mapsto (1,1,0,\dots).
\end{align*}

\parbf{\ref{ex:+-c}.}
Note that if $c<0$, then $r>s$.
The latter is impossible since $r$ is extremal and $s$ is admissible.

Observe that the function $\bar r=\min\{\,r,s+c\}$ is admissible.
Indeed if $\bar r(x)=r(x)$ and $\bar r(y)=r(y)$ then 
\[\bar r(x)+\bar r(y)=r(x)+ r(y)\ge \dist{x}{y}{}.\]
Further if $\bar r(x)=s(x)+c$ then 
\begin{align*}
\bar r(x)+\bar r(y)&\ge [s(x)+c]+ [s(y)-c]= 
\\
&=s(x)+s(y) \ge 
\\
&\ge\dist{x}{y}{}.
\end{align*}

Since $r$ is extremal, we have $r=\bar r$;
that is, $r\le s+c$.

\parbf{\ref{ex:inj=complete-geodesic-contractible}.}
Choose an injective space $\spc{Y}$.

\textit{\ref{SHORT.ex:inj=complete-geodesic-contractible:complete}.}
Fix a Cauchy sequence $(x_n)$ in $\spc{Y}$;
we need to show that it has a limit $x_\infty\in \spc{Y}$.
Consider metric on $\spc{X}=\NN\cup\{\infty\}$ defined by 
\begin{align*}
\dist{m}{n}{\spc{X}}&=\dist{x_m}{x_n}{\spc{Y}},
\\
\dist{m}{\infty}{\spc{X}}&=\lim_{n\to\infty}\dist{x_m}{x_n}{\spc{Y}}.
\end{align*}
Since the sequence is Cauchy, so is the sequence $\ell_n=\dist{p}{x_n}{\spc{Y}}$.
Therefore the last limit is defined.

By construction the map $n\mapsto x_n$ is distance-preserving on $\NN\subset \spc{X}$.
Since $\spc{Y}$ is injective, this map can be extended to $\infty$ as a short map; set $\infty\mapsto x_\infty$.
Since $\dist{x_n}{x_\infty}{\spc{Y}}\le \dist{n}{\infty}{\spc{X}}$ 
and $\dist{n}{\infty}{\spc{X}}\to 0$, we get that
$x_n\to x_\infty$ as $n\to\infty$.

\textit{\ref{SHORT.ex:inj=complete-geodesic-contractible:geodesic}.}
Applying the definition of injective space, we get a midpoint for any pair of points in $\spc{Y}$.
By \ref{SHORT.ex:inj=complete-geodesic-contractible:complete},
$\spc{Y}$ is a complete space.
It remains to apply \ref{lem:mid>geod:geod}.

\textit{\ref{SHORT.ex:inj=complete-geodesic-contractible:contractible}.}
Let $k\:\spc{Y}\hookrightarrow \ell^\infty(\spc{Y})$ be the Kuratowski embedding (\ref{lem:kuratowski}).
Observe that $\ell^\infty(\spc{Y})$ is contractible;
in particular, there is a homotopy $k_t\:\spc{Y}\hookrightarrow \ell^\infty(\spc{Y})$ such that $k_0=k$ and $k_1$ is a constant map.
(In fact one can take $k_t=(1-t)\cdot k$.)

Since $k$ is distance-preserving and $\spc{Y}$ is injective,
there is a short map $f\:\ell^\infty(\spc{Y})\to \spc{Y}$ such that the composition $f\circ k$ is the identity map on $\spc{Y}$.
The composition $f\circ k_t\:\spc{Y}\hookrightarrow \spc{Y}$ is a needed homotopy. 

\parbf{\ref{ex:injective-spaces}.}
Suppose that a short map $f\:A\to\spc{Y}$ is defined on a subset $A$ of a metric space $\spc{X}$.
We need to construct a short extension $F$ of $f$.

\parit{\ref{SHORT.ex:injective-spaces:R}.}
Suppose $\spc{Y}=\RR$.
Without loss of generality, we may assume that $A\ne\emptyset$, otherwise map whole $\spc{X}$ to a single point.
Set 
\[F(x)=\inf\set{f(a)-\dist{a}{x}{}}{a\in A}.\] 
Observe that $F$ is short and $F(a)=f(a)$ for any $a\in A$.

\parit{\ref{SHORT.ex:injective-spaces:tree}.}
Suppose  $\spc{Y}$ is a complete metric tree.
Fix points $p\in \spc{X}$ and $q\in\spc{Y}$.
Given a point $a\in A$,
let $x_a\in\cBall[f(a),\dist{a}{p}{}]$ be the point closest to $f(x)$.
Note that $x_a\in[q\,f(a)]$ and either $x_a=q$ or $x_a$ lies on distance $\dist{a}{p}{}$ from $f(a)$.

Note that the geodesics $[q\,x_a]$ are nested;
that is, for any $a,b\in A$ we have either $[q\,x_a]\subset [q\,x_b]$ or $[q\,x_b]\subset [q\,x_a]$.
Moreover, in the first case we have $\dist{x_b}{f(a)}{}\le \dist{p}{a}{}$ and in the second $\dist{x_a}{f(b)}{}\le \dist{p}{b}{}$.

It follows that the closure of the union of all geodesics $[q\,x_a]$ for $a\in\spc{A}$ is a geodesic.
Denote by $x$ its endpoint; it exists since $\spc{Y}$ is complete.
It remains to observe that $\dist{x}{f(a)}{}\le \dist{p}{a}{}$ for any $a\in\spc{A}$;
that is, one can take $f(p)=x$.

\parbf{\ref{ex:ultrametric}.}
Choose three points $x,y,z\in\spc{X}$ and set $\spc{A}=\{x,z\}$.
Let $f\:\spc{A}\to \spc{Y}$ be an isometry.
Then $F(y)=f(x)$ or $F(y)=f(z)$.
If  $f(y)=f(x)$, then
\begin{align*}
\dist{y}{z}{\spc{X}}&\ge  \dist{F(y)}{f(z)}{\spc{Y}}=
\\
 &=\dist{x}{z}{\spc{X}}.
\end{align*}
Analogously if $f(y)=f(z)$, then $\dist{x}{y}{\spc{X}}\ge\dist{x}{z}{\spc{X}}$.

It remains to observe that the strong triangle inequality holds in both cases.

\parit{\ref{SHORT.ex:injective-spaces:ell-infty}.}
In this case $\spc{Y}=(\RR^2,\ell^\infty)$.
Note that the map $\spc{X}\to (\RR^2,\ell^\infty)$ is short if and only if both of its coordinate projections are short.
It remains to apply \ref{SHORT.ex:injective-spaces:R}.

\parbf{\ref{ex:tripod+square}}; \ref{SHORT.ex:tripod+square:tripod}.
Let $f$ be an extremal function.
Observe that at least two of the numbers $f(a)+f(b)$, $f(b)+f(c)$, and $f(c)+f(a)$ are $1$.
It follows that for some $x\in[0,\tfrac12]$, we have 
\begin{align*}
f(a)&=1\pm x,&
f(b)&=1\pm x,&
f(c)&=1\pm x,
\end{align*}
where we have one ``minus'' and two ``pluses'' in these three formulas.

Suppose that
\begin{align*}
g(a)&=1\pm y,& g(b)&=1\pm y,& g(c)&=1\pm y
\end{align*}
is another extremal function.
Then $|f-g|\z=|x-y|$ if $g$ has ``minus'' at the same place as $f$ and $|f-g|=|x+y|$ otherwise.

\begin{wrapfigure}{o}{30 mm}
\vskip-0mm
\centering
\includegraphics{mppics/pic-3}
\bigskip
\includegraphics{mppics/pic-4}
\end{wrapfigure}

It follows that $\Inj\spc{X}$ is isometric to a tripod;
that is, $\Inj\spc{X}$ is formed by three segments of length $\tfrac12$ glued at one end.

\parit{\ref{SHORT.ex:tripod+square:square}.}
Assume $f$ is an extremal function.
Observe that 
$f(x)+f(y)=f(p)+f(q)=2$;
in particular, two values $a=f(x)-1$ and $b=f(p)-1$ completely describe the function $f$.
Since $f$ is extremal, we also have that 
\[(1\pm a)+(1\pm b)\ge 1\]
for all 4 choices of signs;
that is, $|a|+|b|\le 1$.

It follows that $\Inj\spc{X}$ is isometric to the rhombus $|a|+|b|\le 1$ in the $(a,b)$-plane with the metric induced by the $\ell^\infty$-norm.





\parbf{\ref{ex:4-on-a-line}.}
Recall that 
\[\dist{f}{g}{\Inj\spc{X}}=\sup\set{|f(x)-g(x)|}{x\in\spc{X}}\]
and 
\[\dist{f}{p}{\Inj\spc{X}}=f(p)\]
for any $f,g\in \Inj\spc{X}$ and $p\in \spc{X}$.

Since $\spc{X}$ is compact we can find a point $p\in\spc{X}$ such that 
\[\dist{f}{g}{\Inj\spc{X}}=|f(p)-g(p)|=\left|\dist{f}{p}{\Inj\spc{X}}-\dist{g}{p}{\Inj\spc{X}}\right|.\]
Without loss of generality we may assume that 
\[\dist{f}{p}{\Inj\spc{X}}
=
\dist{g}{p}{\Inj\spc{X}}
+
\dist{f}{g}{\Inj\spc{X}}.\]
Applying \ref{lem:opposite}, we can find a point $q\in\spc{X}$ such that 
\[\dist{q}{p}{\Inj\spc{X}}
=
\dist{f}{p}{\Inj\spc{X}}
+
\dist{f}{q}{\Inj\spc{X}},\]
whence the result.

Since $\Inj\spc{X}$ is injective (\ref{prop:InjX-is-injective}), by \ref{ex:inj=complete-geodesic-contractible:geodesic} it has to be geodesic. It remains to note that the concatenation of geodesics $[pq]$, $[gf]$, and $[fq]$ forms a required geodesic $[pq]$.



\parbf{\ref{ex:Hausdorff-bry}}; \ref{SHORT.ex:Hausdorff-bry:conv}.
Denote by $X_r$ the $r$ neighborhood of a set $X\z\subset \RR^2$.
Observe  that 
\[(\Conv X)_r=\Conv(X_r),\]
and try to use it.

\parit{\ref{SHORT.ex:Hausdorff-bry:bry}.}
The answer is ``no'' in both parts.

For the first part let $X$ be a unit disc and $Y$ a finite $\eps$-net in $X$.
Evidently $|X-Y|_{\Haus\RR^2}<\eps$, 
but
$|\partial X-\partial Y|_{\Haus\RR^2}\approx 1$.

For the second part take $X$ to be a unit disc and $Y=\partial X$ to be its boundary circle.
Note that $\partial X=\partial Y$ in particular $\dist{\partial X}{\partial Y}{\Haus\RR^2}=0$ while $\dist{ X}{ Y}{\Haus\RR^2}=1$.

\parit{Remark.}
A more interesting example for the second part can be build on {}\emph{lakes of Wada} --- and example of three open bounded topological disks in the plane that have identical boundary.

\parbf{\ref{ex:Huas-perimeter-area}.}
Let $A$ be a compact convex set in the plane.
Denote by $A^r$ the closed $r$-neighborhood of $A$.
Recall that by Steiner's formula we have
\[\area A^r=\area A+r\cdot\perim A+\pi\cdot r^2.\]
Taking derivative and applying coarea formula, we get 
\[\perim A^r=\perim A+2\cdot\pi\cdot r.\]

Observe that if $A$ lies in a compact set $B$ bounded by a closed curve, then 
\[\perim A\le \perim B.\]
Indeed the closest-point projection $\RR^2\to A$ is short and it maps $\partial B$ onto $\partial A$.

It remains to observe that if $A_n\to A_\infty$, then for any $r>0$ we have that
\[A_\infty^r\supset A_n
\quad\text{and}\quad
A_\infty\subset A_n^r\]
for all large $n$.

%%%%%%%%%%%%%%%%%%%%%%%%%%%%%%

\parbf{\ref{ex:GH-inj}.}
Let $\spc{U}$ be as in \ref{prop:GH=X+Y}.
Suppose that $|\spc{X}-\spc{Y}|_{\spc{U}}<\eps$;
we need to show that 
\[|\hat{\spc{X}}-\hat{\spc{Y}}|_{\GH}<2\cdot \eps.\]

Denote by $\hat{\spc{U}}$ the injective envelop of $\spc{U}$.
Recall that $\spc{U}$, $\spc{X}$, and $\spc{Y}$ can be considered as subspaces of $\hat{\spc{U}}$, $\hat{\spc{X}}$, and $\hat{\spc{Y}}$ respectively.

According to \ref{ex:d-p-inclusion}, the inclusions $\spc{X}\hookrightarrow\spc{U}$ and $\spc{Y}\hookrightarrow\spc{U}$ can be extended to distance preserving inclusions $\hat{\spc{X}}\hookrightarrow\hat{\spc{U}}$ and $\hat{\spc{Y}}\hookrightarrow\hat{\spc{U}}$.
Therefore we can and will consider  $\hat{\spc{X}}$ and $\hat{\spc{Y}}$ as subspaces of $\hat{\spc{U}}$.


Given $f\in \hat{\spc{U}}$,
let us find $g\in \hat{\spc{X}}$ such that 
\[|f(u)-g(u)|<2\cdot\eps\eqlbl{|g-f|}\]
for any $u\in\spc{U}$.
Note that the restriction $f|_{\spc{X}}$ is admissible on ${\spc{X}}$.
By \ref{obs:extremal:below}, there is $g\in \hat{\spc{X}}$ such that 
\[g(x)\le f(x)\eqlbl{g(x)=<f(x)}\]
for any $x\in\spc{X}$.


Recall that any extremal function is $1$-Lipschitz;
in particular $f$ and $g$ are $1$-Lipschitz on $\spc{U}$.
Therefore \ref{g(x)=<f(x)} and $|\spc{X}-\spc{Y}|_{\spc{U}}<\eps$ imply that
\[g(u)< f(u)+2\cdot \eps\]
for any $u\in\spc{U}$.
By \ref{ex:+-c}, we also have 
\[g(u)> f(u)-2\cdot \eps\]
for any $u\in\spc{U}$.
Whence \ref{|g-f|} follows.

It follows that $\hat{\spc{Y}}$ lies in a $2\cdot\eps$-neighborhood of $\hat{\spc{X}}$ in $\hat{\spc{U}}$.
The same way we show that $\hat{\spc{X}}$ lies in a $2\cdot\eps$-neighborhood of $\hat{\spc{Y}}$ in $\hat{\spc{U}}$.
The later means that
$|\hat{\spc{X}}-\hat{\spc{Y}}|_{\Haus\spc{U}}<2\cdot\eps$,
and therefore
$|\hat{\spc{X}}-\hat{\spc{Y}}|_{\GH}<2\cdot\eps$.

\parit{Comment.} 
This problem was discussed by Urs Lang, Maël Pavón, and Roger Züst \cite[3.1]{lang-pavon-zust}.
\begin{figure}[h!]
\vskip-0mm
\centering
\includegraphics{mppics/pic-505}
\end{figure}
They also show that the constant 2 is optimal.
To see this look at the injective envelops of two 4-point metric spaces shown on the diagram and observe that the Gromov--Hausdorff distance between the 4-point metric spaces is 1, while the distance between their injective envelops approaches 2 as $s\to\infty$. 




\parbf{\ref{ex:GH-po:a}.}
In order to check that $\dist{*}{*}{\GH'}$ is a metric, it is sufficient to show that
\[\dist{\spc{X}}{\spc{Y}}{\GH'}=0 
\quad\Longrightarrow\quad
\spc{X}\iso\spc{Y};\]
the remaining conditions are trivial.

If $\dist{\spc{X}}{\spc{Y}}{\GH'}=0$, then there is a sequence of maps $f_n\:\spc{X}\to \spc{Y}$ such that 
\[\dist{f_n(x)}{f_n(x')}{\spc{Y}}\ge \dist{x}{x'}{\spc{X}}-\tfrac1n.\]

Choose a countable dense set $S$ in $\spc{X}$.
Passing to a subsequence of $f_n$ we can assume that $f_n(x)$ converges for any $x\in S$ as $n\to\infty$;
denote its limit by $f_\infty(x)$.

For each point $x\in\spc{X}$ choose a sequence $x_m\in S$ converging to $x$.
Since $\spc{Y}$ is compact, we can assume in addition that $y_m=f_\infty(x_m)$ converges in $\spc{Y}$.
Set $f_\infty(x)=y$.
Note that the map $f_\infty\:\spc{X}\to \spc{Y}$ is  distance-nondecreasing.

The same way we can construct a distance-nondecreasing map 
$g_\infty\:\spc{Y}\to \spc{X}$.

By \ref{ex:non-contracting-map}, the compositions $f_\infty\circ g_\infty\:\spc{Y}\to \spc{Y}$ and $g_\infty\z\circ f_\infty\:\spc{X}\to \spc{X}$ are isometries.
Therefore $f_\infty$ and $g_\infty$ are isometries as well.

%Observe that 
%$$|\spc{X}_n-\spc{X}_\infty|_{\mathcal{M}}\to 0 
%\quad\Longrightarrow\quad 
%\dist{\spc{X}_n}{\spc{X}_\infty}{\GH'}\to 0$$
%follows from Proposition~\ref{prop:GH-e-isom} and Exercise~\ref{ex:alm-isom:inverse}.
%To prove that 
%$$|\spc{X}_n-\spc{X}_\infty|_{\mathcal{M}}\to 0 
%\quad\Longleftarrow\quad 
%\dist{\spc{X}_n}{\spc{X}_\infty}{\GH'}\to 0,$$
%Suppose that $f_n\:\spc{X}_n\to\spc{X}_\infty$ and $g_n\:\spc{X}_\infty\to\spc{X}_n$ are $\eps_n$-almost distance-nondecreasing maps for $\eps_n\to 0$.
%Arguing as above, pass to a partial limit $h$ of the sequence $f_n\circ g_n\:\spc{X}_\infty\to\spc{X}_\infty$.
%Note that $h$ is a distance non-deceasing map from a compact space to an itself.
%By Exercise~\ref{ex:non-contracting-map}, $h$ is an isometry.


\parbf{\ref{pr:doubling}.}
Choose a space $\spc{X}$ in $\spc{Q}(C,D)$, denote a $C$-doubling measure by $\mu$.
Without loss of generality we may assume that $\mu(\spc{X})\z=1$.

The doubling condition implies that 
\[\mu[\oBall(p,\tfrac D{2^n})]\ge\tfrac 1{C^n}\]
for any point $x\in \spc{X}$.
It follows that 
\[\pack_{\frac D{2^n}}\spc{X}\le C^n.\]

By \ref{ex:pack-net}, for any $\eps\ge\frac D{2^{n-1}}$, the space $\spc{X}$ admits an $\eps$-net with at most $C^n$ points.
Whence $\spc{Q}(C,D)$ is uniformly totally bounded.

\parbf{\ref{pr:under}.} 
Since $\spc{Y}$ is compact, it has a finite $\eps$-net for any $\eps>0$.
For each $\eps>0$ choose a finite $\eps$-net $\{y_1,\dots,y_{n_\eps}\}$ in $\spc{Y}$.

Suppose $f\:\spc{X}\to \spc{Y}$ be a distance-nondecreasing map.
Choose one point $x_i$ in each nonempty subset $B_i=f^{-1}[\oBall(y_i,\eps)]$.
Note that the subset $B_i$ has diameter at most $2\cdot \eps$ and 
\[\spc{X}=\bigcup_iB_i.\]
Therefore the set of points $\{x_i\}$ forms a $2\cdot\eps$ net in $\spc{X}$.
Whence \ref{SHORT.pr:under:if} follows.

\parit{\ref{SHORT.pr:under:only-if}.} Let $\spc{Q}$ be a uniformly totally bounded family of spaces. 
Suppose that each space in $\spc{Q}$ has an $\tfrac1{2^n}$-net with at most $M_n$ points; we may assume that $M_0=1$.

Consider the space $\spc{Y}$ of all infinite integer sequences $m_0,m_1,\dots$ such that $1\le m_n\le M_n$ for any $n$.
Given two sequences $(\ell_n)$, and $(m_n)$ of points in $\spc{Y}$, set 
\[\dist{(\ell_n)}{(m_n)}{\spc{Y}}=\tfrac C{2^{n}},\]
where $n$ is minimal index such that $\ell_n\ne m_n$ and $C$ is a positive constant.

Observe that $\spc{Y}$ is compact.
Indeed it is complete and the sequences constant starting from index $n$ form a finite $\tfrac C{2^{n}}$-net in $\spc{Y}$.

Given a space $\spc{X}$ in $\spc{Q}$,
choose a sequence of $\tfrac1{2^n}$ nets 
$N_n\subset\spc{X}$ for each natural $n$.
We can assume that $|N_n|\le M_n$; let us enumerate the points in $N_n$ by $\{1,\dots,M_n\}$.
Consider the map $f:\spc{X}\to\spc{Y}$ defined by $f:x\to (m_1(x),m_2(x),\dots)$ where $m_n(x)$ is a number of the point in $N_n$ that lies on the distance $<\tfrac1{2^n}$ from $x$.

If $\tfrac1{2^{n-2}}\ge \dist{x}{x'}{\spc{X}}>\tfrac1{2^{n-1}}$, then $m_n(x)\ne m_n(x')$.
It follows that $\dist{f(x)}{f(x')}{\spc{Y}}\ge \tfrac C{2^{n}}$.
In particular, if $C>10$, then 
\[\dist{f(x)}{f(x')}{\spc{Y}}\ge \dist{x}{x'}{\spc{X}}\]
for any $x,x'\in \spc{X}$.
That is, $f$ is a distance-nondecreasing map $\spc{X}\to \spc{Y}$.

\parbf{\ref{ex:GH-SC},} \ref{SHORT.ex:GH-SC:circle}.
Suppose $\spc{X}_n\GHto \spc{X}$ and $\spc{X}_n$ are simply connected length metric space.
It is sufficient to show that any nontrivial covering map $f\:\tilde{\spc{X}}\to \spc{X}$ corresponds to a nontrivial covering map $f_n\:\tilde{\spc{X}}_n\to \spc{X}_n$ for large $n$.

The latter can be constructed by covering $\spc{X}_n$ by small balls that lie close to sets in $\spc{X}$ evenly covered by $f$, prepare few copies of these sets and glue them the same way as the inverse images of the evenly covered sets in $\spc{X}$ glued to obtain $\tilde{\spc{X}}$.

\begin{wrapfigure}{r}{40 mm}
\vskip-0mm
\centering
\includegraphics{mppics/pic-2}
\end{wrapfigure}

\parit{\ref{SHORT.ex:GH-SC:nonsc-limit}.}
Let $\spc{V}$ be a cone over Hawaiian earring.
Consider the {}\emph{doubled cone} $\spc{W}$ --- two copies of $\spc{V}$ with glued base points earrings (see the diagram).

The space $\spc{W}$ can be equipped with length metric for example the induced length metric from the shown embedding.

Note that $\spc{V}$ is simply connected, but $\spc{W}$ is not --- it is a good exercise in topology.

If we delete from the earrings all small circles, then the obtained double cone becomes simply connected and it remains to be close to $\spc{W}$ in the sense of Gromov--Hausdorff.

\parit{Remark.}
Note that from part \ref{SHORT.ex:GH-SC:nonsc-limit}, the limit does not admit a nontrivial covering.
So if we define fundamental group right --- as the inverse image of groups of deck transformations for all its coverings, then one may say that Gromov--Hausdorff limit of simply connected length spaces is simply connected.

\parbf{\ref{ex:sphere-to-ball},}
\textit{\ref{SHORT.ex:sphere-to-ball:2}.}
Suppose that a metric on $\mathbb{S}^2$ is close to the disc $\DD^2$.
Note that $\mathbb{S}^2$ contains a circle $\gamma$ that is close to the boundary curve of $\DD^2$.
By Jordan curve theorem, $\gamma$ divides $\mathbb{S}^2$ into two discs, say $D_1$ and $D_2$.

By \ref{ex:GH-SC:nonsc-limit}, the Gromov--Hausdorff limit of $D_1$ and $D_2$ have to contain whole $\DD^2$, otherwise the limit would admit a nontrivial covering.
Consider points $p_1\in D_1$ and $p_2\in D_2$ that a close to the center of $\DD^2$.
On one hand the distance $\dist{p_1}{p_2}{n}$ have to be very small.
On the other hand, any curve from $p_1$ to $p_2$ must cross $\gamma$, so it has length about 2 at least --- a contradiction.



\parit{\ref{SHORT.ex:sphere-to-ball:3}.}
Make fine burrows in the standard 3-ball without changing its topology,
but at the same time come sufficiently close to any point in the ball.

Consider the \index{doubling}\emph{doubling} of the obtained ball along its boundary;
that is, two copies of the ball with identified corresponding points on their boundaries.
The obtained space is homeomorphic to $\mathbb{S}^3$.
Note that the burrows can be made 
so that the obtained space is sufficiently close to the original ball 
in the Gromov--Hausdorff metric.

\parit{Source:} \cite[Exercises 7.5.13 and 7.5.17]{burago-burago-ivanov}. 

%%%%%%%%%%%%%%%%%%%%%%%%%%%%%%%%

%%%%%%%%%%%%%%%%%%%%%%%%%%%%%%

\parbf{\ref{ex:prop:eps-isometry=isometry}.}
Suppose that  $f_n\:\spc{X}\to \spc{Y}$ is a $\tfrac1n$-isometry between compact spaces for each $n\in\NN$.
Consider the $\omega$-limit $f_\omega$ of~$f_n$,
\[f_\omega(x)=\lim_{n\to\omega}f_n(x);\]
according to \ref{prop:ultra/compact}, $f_\omega$ is defined.
Since 
\[|f_n(x)-f_n(x')|\lege |x-x'|\pm\tfrac1n\]

we get that 
\[|f_\omega(x)-f_\omega(x')|= |x-x'|\]
for any $x,x'\in \spc{X}$;
that is, $f_\omega$ is distance-preserving.
Further, since $f_n$ is a $\tfrac1n$-isometry,
for any $y\in \spc{Y}$ there is $x_m$ such that $|f_n(x_n)-y|\le \tfrac1n$.
Therefore,
\[f_\omega(x_\omega)=y,\]
where $x_\omega$ is the $\omega$-limit of $x_n$;
that is, $f_\omega$ is onto.
It follows that $f_\omega\:\spc{X}\to\spc{Y}$ is an isometry.

\parbf{\ref{ex:linear}.}
Choose a nonprincipal ultrafilter $\omega$ and set $L(\bm{s})=s_\omega$.
It remains to observe that $L$ is linear.

\parbf{\ref{ex:lim(tree)}.}
Let $\gamma$ be a path from $p$ to $q$ in a metric tree $\spc{T}$.
Assume that $\gamma$ passes thru a point $x$ on distance $\ell$ from $[pq]$.
Then 
\[\length\gamma\ge \dist{p}{q}{}+2\cdot \ell.
\eqlbl{eq:+ell}\]

Suppose that $\spc{T}_n$ is a sequence of metric trees that $\omega$-converges to $\spc{T}_\omega$.
By \ref{obs:ultralimit-is-geodesic}, the space $\spc{T}_\omega$ is geodesic.

The uniqueness of geodesics follows from \ref{eq:+ell}.
Indeed, if for a geodesic $[p_\omega q_\omega]$ there is another geodesic $\gamma_\omega$ connecting its ends, then it has to pass thru a point $x_\omega\notin [p_\omega q_\omega]$.
Choose sequences $p_n,q_n,x_n\in\spc{T}_n$ such that $p_n\to p_\omega$, $q_n\to q_\omega$, and $x_n\to x_\omega$ as $n\to\omega$.
Then 
\begin{align*}
\dist{p_\omega}{q_\omega}{}&=\length\gamma\ge \lim_{n\to\omega}(\dist{p_n}{x_n}{}+\dist{q_n}{x_n}{})\ge
\\
&\ge \lim_{n\to\omega}(\dist{p_n}{q_n}{}+2\cdot\ell_n)=
\\
&=\dist{p_\omega}{q_\omega}{}+2\cdot\ell_\omega.
\end{align*}
Since $x_\omega\notin [p_\omega q_\omega]$, we have that $\ell_\omega>0$ --- a contradiction.

It remains to show that any geodesic triangle $\spc{T}_\omega$ is a tripod.
Consider the sequence of centers of tripods $m_n$ for a given sequences of points $x_n,y_n,z_n\in \spc{T}_n$.
Observe that its ultralimit $m_\omega$ is the center of the tripod with ends at $x_\omega,y_\omega,z_\omega\in \spc{T}_\omega$.

\parbf{\ref{ex:ultrapower}.}
Further, we consider $\spc{X}$ as a subset of $\spc{X}^\omega$.

\parit{\ref{SHORT.ex:ultrapower:a}.} Follows directly from the definitions.

\parit{\ref{SHORT.ex:ultrapower:compact}.}
Suppose $\spc{X}$ compact.
Given a sequence $x_n$ in $\spc{X}$, denote its $\omega$-limit in $\spc{X}^\omega$ by $x^\omega$ and its $\omega$-limit in $\spc{X}$ by $x_\omega$.

Observe that $x^\omega=\iota(x_\omega)$.
Therefore, $\iota$ is onto.

If $\spc{X}$ is not compact, we can choose a sequence $x_n$ such that $\dist{x_m}{x_n}{}>\eps$ for fixed $\eps>0$ and all $m\ne n$.
Observe that
\[\lim_{n\to\omega}\dist{x_n}{y}{\spc{X}}\ge \tfrac\eps2\]
for any $y\in\spc{X}$.
It follows that $x_\omega$ lies on the distance at least $\tfrac\eps2$ from~$\spc{X}$.

\parit{\ref{SHORT.ex:ultrapower:proper}.}
A sequence of points $x_n$ in $\spc{X}$ will be called $\omega$-bounded if there is a real constant $C$ such that
\[\dist{p}{x_n}{\spc{X}}\le C\] 
for $\omega$-almost all $n$.

The same argument as in \ref{SHORT.ex:ultrapower:compact} shows that any $\omega$-bounded sequence has its $\omega$-limit in $\spc{X}$.
Further, if $(x_n)$ is not  $\omega$-bounded, then 
\[\lim_{n\to\omega}\dist{p}{x_n}{\spc{X}}=\infty;\]
that is, $x_\omega$ does not lie in the metric component of $p$ in $\spc{X}^\omega$.

\parbf{\ref{ex:isom-ultrapower}.} Show and use that the spaces $\spc{X}^\omega$ and $(\spc{X}^\omega)^\omega$ have discrete metric and both have cardinality continuum.

\parbf{\ref{ex:two-geodesics-in-ultrapower}.}
Apply \ref{lem:X-X^w} and \ref{obs:ultrapower-is-geodesic}.

\parbf{\ref{ex:notproper-limit}.} Consider the infinite metric $\spc{T}$ tree with unit edges shown
\begin{figure}[h!]
\vskip-0mm
\centering
\includegraphics{mppics/pic-605}
\end{figure}
on the diagram. Observe that $\spc{T}$ is proper.

Consider the vertex $v_\omega=\lim_{n\to\omega}v_n$ in the ultrapower $\spc{T}^\omega$.
Observe that $\omega$ has an infinite degree.
Conclude that $\spc{T}^\omega$ is not locally compact.

\parbf{\ref{ex:ultraT}.} Consider a product space $[0,1]\times[0,\tfrac12]\times[0,\tfrac14]\times\dots$.

\parbf{\ref{ex:Asym(Lob)}}; \ref{SHORT.ex:Asym(Lob):metric-tree}.
Show that there is $\delta>0$ such that sides of any geodesic triangle intersect a disk of radius $\delta$.
Conclude that any geodesic triangle in $\Asym\spc{L}$ is a tripod.
Make a conclusion.

\parit{\ref{SHORT.ex:Asym(Lob):homogeneous}.} Observe that $L$ is one-point homogeneous and use it.

\parit{\ref{SHORT.ex:Asym(Lob):continuum}.} 
By \ref{SHORT.ex:Asym(Lob):homogeneous}, it is sufficient to show that $p_\omega$ has a continuum degree.

Choose distinct geodesics $\gamma_1,\gamma_2\:[0,\infty)\to L$ that start at a point $p$.
Show that the limits of $\gamma_1$ and $\gamma_2$ run in the different connected components of $(\Asym\spc{L})\setminus \{p_\omega\}$.
Since there is a continuum of distinct geodesics starting at $p$,
we get that the degree of $p_\omega$ is at least continuum.

On the other hand, the set of sequences of points in $L$  has cardinality continuum.
In particular, the set of points in $\Asym\spc{L}$ has cardinality at most continuum.
It follows that the degree of any vertex is at most continuum.

\parit{\ref{SHORT.ex:Asym(Lob):others}.}
The proof for the Lobachevsky space goes along the same lines.

For the infinite 3-regular tree, part \ref{SHORT.ex:Asym(Lob):metric-tree} follows from \ref{ex:lim(tree)}.
The 3-regular tree is not one-point homogeneous, but it is vertex homogeneous; the latter is sufficient to prove \ref{SHORT.ex:Asym(Lob):homogeneous}.
No changes are needed in \ref{SHORT.ex:Asym(Lob):continuum}.

\parit{Remark.}
Anna Dyubina and Iosif Polterovich \cite{dyubina-polterovich} proved that the properties \ref{SHORT.ex:Asym(Lob):homogeneous} and \ref{SHORT.ex:Asym(Lob):continuum} describe the tree $\spc{T}$ up to isometry.
In particular, the asymptotic space of the Lobachevsky plane does not depend on the choice of ultrafilter and the sequence $\lambda_n\to \infty$.


%%%%%%%%%%%%%%%%%%%%%%%%%%%%
{\small\sloppy
\documentclass[twoside]{book}

\usepackage{lectures}
\usepackage[colorlinks=true,
citecolor=black,
linkcolor=black,
anchorcolor=black,
filecolor=black,
menucolor=black,
urlcolor=black,
pdftitle={Pure metric geometry: introductory lectures},
pdfsubject={Geometry},
pdfauthor={Anton Petrunin}
]{hyperref}
\makeindex

\begin{document}
%\pagestyle{empty}\renewcommand\includegraphics[2][{}]{}\def\emph{\textit}
%\overfullrule=100mm

 
\title{Pure metric geometry:\\
introductory lectures}
\author{Anton Petrunin}
\date{}
\maketitle

We discuss only domestic affairs of metric spaces;
applications are given only as illustrations.

These notes are based on couses at PSU (Spring 2020) and SPbSU (Fall 2022).
An extended version can be found on the author's website;
it includes an introduction to Alexandrov geometry based on \cite{alexander-kapovitch-petrunin-2019} and metric geometry on manifolds \cite{petrunin2020mnfld} based on a simplified proof of Gromov's systolic inequality given by Alexander Nabutovsky~\cite{nabutovsky}.

A part of the text is a compilation from \cite{alexander-kapovitch-petrunin-2019, alexander-kapovitch-petrunin-2025, petrunin-yashinski, petrunin-2022-PIGTIKAL, petrunin-zamorabarrera} and its drafts.

I want to thank
Alexander Lytchak,
Julien Melleray,
and Sergio Zamora Barrera for help.
The present work is partially supported by NSF grant DMS-2005279
and by the Simons Foundation under grant \#584781.

\thispagestyle{empty}
\tableofcontents
\thispagestyle{empty}

\chapter{Definitions}

\section{Metric spaces}
\label{sec:metric spaces}


The distance between two points $x$ and $y$ in a metric space $\spc{X}$ will be denoted by $\dist{x}{y}{}$ or $\dist{x}{y}{\spc{X}}$.
The latter notation is used if we need to emphasize 
that the distance is taken in the space~${\spc{X}}$.

The function 
\[\distfun_x\:y\mapsto \dist{x}{y}{}\]
is called the \index{distance function}\emph{distance function} from~$x$. 

Given $R\in[0,\infty]$ and $x\in \spc{X}$, the sets
\begin{align*}
\oBall(x,R)&=\{y\in \spc{X}\mid \dist{x}{y}{}<R\},
\\
\cBall[x,R]&=\{y\in \spc{X}\mid \dist{x}{y}{}\le R\}
\end{align*}
are called, respectively, the  \index{open ball}\emph{open} and  the \index{closed ball}\emph{closed  balls}   of radius $R$ with center~$x$.
Again, if we need to emphasize that these balls are taken in the metric space $\spc{X}$,
we write 
\[\oBall(x,R)_{\spc{X}}\quad\text{and}\quad\cBall[x,R]_{\spc{X}}.\]


\section{Variations of definition}

Recall that a metric is a real-valued function $(x,y)\mapsto\dist{x}{y}{\spc{X}}$ that satisfies the following conditions for any three points $x,y,z\in \spc{X}$:
\begin{enumerate}[(i)]
\item $\dist{x}{y}{\spc{X}}\ge 0$,
\item\label{metric=0} $\dist{x}{y}{\spc{X}}= 0$ $\iff$ $x=y$,
\item $\dist{x}{y}{\spc{X}}=\dist{y}{x}{\spc{X}}$,
\item $\dist{x}{y}{\spc{X}}+\dist{y}{z}{\spc{X}}\ge\dist{x}{z}{\spc{X}}$,
\end{enumerate}

\parbf{Pseudometrics.}
A generalization of a metric in which the distance between two distinct points can be zero is called \emph{pseudometric}.
In other words, to define pseudometric, we need to remove condition (\ref{metric=0}) from the list.

The following two observations show that
nearly any question about pseudometric spaces can be reduced to a question about genuine metric spaces.

Assume $\spc{X}$ is a pseudometric space.
Set
$x\sim y$ if $\dist{x}{y}{}=0$. 
Note that if $x\sim x'$, then $\dist{y}{x}{}=\dist{y}{x'}{}$ for any $y\in\spc{X}$.
Thus, $\dist{*}{*}{}$ defines a metric on the
quotient set $\spc{X}/{\sim}$.
In this way we obtain a metric space $\spc{X}'$.
The space $\spc{X}'$ is called the 
\emph{corresponding metric space} for the pseudometric space $\spc{X}$.
Often we do not distinguish between $\spc{X}'$ and~$\spc{X}$. 

\parbf{$\bm{\infty}$-metrics.}
One may also consider metrics with values in $\RR\cup\{\infty\}$;
we might call them $\infty$-metrics or simply metrics.

Again nearly any question about $\infty$-metric spaces can be reduced to a question about genuine metric spaces. 

Indeed, set $x\approx y$ if and only if $\dist{x}{y}{}<\infty$;
this is an other equivalence relation on $\spc{X}$.
The equivalence class of a point $x\in\spc{X}$ will be called the \emph{metric component}\index{metric component} 
 of $x$; it will be denoted as $\spc{X}_x$.
One could think of $\spc{X}_x$ as  $\oBall(x,\infty)_{\spc{X}}$ --- the open ball centered at $x$ and radius $\infty$ in $\spc{X}$.

It follows that any $\infty$-metric space is a \emph{disjoint union} of genuine metric spaces --- the metric components of the original $\infty$-metric space.

\begin{thm}{Exercise}
Given two sets $A$ and $B$ on the plane, set 
\[\dist{A}{B}{}=\mu(A\backslash B)+\mu(B\backslash A),\]
where $\mu$ denotes the Lebesgue measure.
\begin{subthm}{}
Show that $\dist{*}{*}{}$ is a pseudometric on the set of bounded measurable sets of the plane.
\end{subthm}

\begin{subthm}{}
Show that $\dist{*}{*}{}$ is an $\infty$-metric on the set of all open sets of the plane.
\end{subthm}
\end{thm}

\section{Completeness}

Recall that a metric space $\spc{X}$ is called \emph{complete} if every Cauchy sequence of points in $\spc{X}$ converges in $\spc{X}$.

\begin{thm}{Exercise}\label{ex:almost-min}
Suppose that $\rho$ is a positive continuous function on a complete metric space $\spc{X}$.
Show that for any $\eps>0$ there is a point $x\in \spc{X}$ such that 
\[\rho(x)<(1+\eps)\cdot\rho(y)\]
for any point $y\in \oBall(x,\rho(x))$.
\end{thm}

Most of the time we will assume that a metric space is complete.
The following construction produces a complete metric space $\bar{\spc{X}}$ for any given metric space $\spc{X}$.
The space $\bar{\spc{X}}$ is called \emph{completion} of $\spc{X}$;
the original space $\spc{X}$ forms a dense subset in $\bar{\spc{X}}$.

\parbf{Completion.}
Given metric space $\spc{X}$, 
consider the set of all Cauchy sequences in $\spc{X}$.
Note that for any two Cauchy sequences $(x_n)$ and $(y_n)$ the right hand side in \ref{eq:cauchy-dist} is defined; moreover it defines a pseudometric on the set $\spc{C}$ of all Cauchy sequences
\[\dist{(x_n)}{(y_n)}{\spc{C}}\df\lim_{n\to\infty}\dist{x_n}{y_n}{\spc{X}}.\eqlbl{eq:cauchy-dist}\]
The corresponding metric space is called a completion of $\spc{X}$.

It is left as an exercise that completion of $\spc{X}$ is complete.

Note that for each point $x\in\spc{X}$ one can consider a constant sequence $x_n=x$ which is Cauchy.
It defines a natural map $\spc{X}\to \bar{\spc{X}}$.
It is easy to check that this map is distance preserving.
In partucular we can (and will) consider $\spc{X}$ as a subset of $\bar{\spc{X}}$.

\section{Compactness}

Let us recall few equivalent definitions of compact metric spaces.

\begin{thm}{Definition}\label{def:compact}
A metric space $\spc{K}$ is compact if and only if one of the following equivalent condition holds:

\begin{subthm}{}
 Every open cover of $\spc{K}$ has a finite subcover.
\end{subthm}

\begin{subthm}{}
 For any open cover of $\spc{K}$ there is $\eps>0$ such that any $\eps$-ball in $\spc{K}$ lie in one element of the cover. (The value $\eps$ is called Lebesgue number of the covering.)
\end{subthm}

\begin{subthm}{}
 Every sequence in $\spc{K}$ has a convergent subsequence.
\end{subthm}

\begin{subthm}{totally-bounded}
The space $\spc{K}$ is complete and \emph{totally bounded}; that is, for any $\eps>0$, the space $\spc{K}$ admits a finite cover by open $\eps$-balls.\footnote{Equivalently, for any $\eps>0$ there is a finite \emph{$\eps$-net}; that is a finite set of points $x_1,\dots,x_n\in \spc{K}$ such that any other point $x$ lies on the distance less than $\eps$ from one of $x_i$.}
\end{subthm}

\end{thm}

Let $\pack_\eps\spc{X}$ be exact upper bound on the number of points $x_1,\z\dots,x_n\in \spc{X}$ such that $\dist{x_i}{x_j}{}\ge\eps$ for any $i\ne j$.

If $n=\pack_\eps\spc{X}<\infty$, then
the collection of points $x_1,\dots,x_n$ is called a \emph{maximal $\eps$-packing}.
Note that $n$ is the maximal number of open disjoint $\tfrac\eps2$-balls in $\spc{X}$.

\begin{thm}{Exercise}\label{ex:pack-net}
Show that a complete space $\spc{X}$ is compact if and only of $\pack_\eps\spc{X}\z<\infty$ for any $\eps>0$.

Show that any maximal $\eps$-packing is an $\eps$-net.
\end{thm}


\begin{thm}{Exercise}\label{ex:non-contracting-map}
Let $\spc{K}$  be a compact metric space and
\[f\:\spc{K}\z\to \spc{K}\] 
be a distance non-decreasing map.
Prove that $f$ is an isometry.
\end{thm}


A metric space $\spc{X}$ is called \index{proper space}\emph{proper} if all closed bounded sets in $\spc{X}$ are compact. 
This condition is equivalent to each of the following statements:
\begin{itemize}
\item For some (and therefore any) point $p\in \spc{X}$ and any $R<\infty$, 
the closed ball $\cBall[p,R]_{\spc{X}}$ is compact. 
\item The function $\distfun_p\:\spc{X}\to\RR$ is proper for some (and therefore any) point $p\in \spc{X}$;
that is, for any compact set $K\subset \RR$, its inverse image 
\[\distfun_p^{-1}(K)=\set{x\in \spc{X}}{\dist{p}{x}{\spc{X}}\in K}\]
is compact.
\end{itemize}

A metric space $\spc{X}$ is called \emph{locally compact} if any point in $\spc{X}$ admits a compact neighborhood;
in other words, for any point $x\in\spc{X}$ a closed ball $\cBall[x,r]$ is compact for some $r>0$.

\section{Geodesics}
\label{sec:geods}

Let $\spc{X}$ be a metric space 
and $\II$\index{$\II$} a real interval. 
A~globally isometric map $\gamma\:\II\to \spc{X}$ is called a \index{geodesic}\emph{geodesic}%
\footnote{Various authors call it differently: {}\emph{shortest path}, {}\emph{minimizing geodesic}.}; 
in other words, $\gamma\:\II\to \spc{X}$ is a geodesic if 
\[\dist{\gamma(s)}{\gamma(t)}{\spc{X}}=|s-t|\]
for any pair $s,t\in \II$.

We say that  $\gamma\:\II\to \spc{X}$ is a geodesic from point $p$ to point $q$ if 
$\II=[a,b]$ and $p=\gamma(a)$, $q=\gamma(b)$.
In this case the image of $\gamma$ is denoted by $[p q]$\index{$[{*}{*}]$} and with an abuse of notations  we also call it a \index{geodesic}\emph{geodesic}.
Given a geodesic $[pq]$, we can parametrize it by distance to $p$;
this parametrization will be denoted by $\geod_{[p q]}(t)$.


We may write $[p q]_{\spc{X}}$ 
to emphasize that the geodesic $[p q]$ is in the space  ${\spc{X}}$.
We also use the following shortcut notation:
\begin{align*}
\left] p q \right[&=[pq]\backslash\{p,q\},
&
\left] p q \right]&=[pq]\backslash\{p\},
&
\left[ p q \right[&=[pq]\backslash\{q\}.
\end{align*}

In general, a geodesic from $p$ to $q$ need not exist and if it exists, it need not  be unique.  
However, once we write $[p q]$ we assume mean that we have made a choice of geodesic.

A metric space is called \index{geodesic}\emph{geodesic} if any pair of its points can be joined by a geodesic. 


A \index{geodesic path}\emph{geodesic path} is a geodesic with constant-speed parametrization by $[0,1]$.
Given a geodesic $[p q]$,
we denote by $\geodpath_{[pq]}$ the corresponding geodesic path;
that is,
$$\geodpath_{[pq]}(t)\z\df\geod_{[pq]}(t\cdot\dist[{{}}]{p}{q}{}).$$

A curve $\gamma\:\II\to \spc{X}$  is called a \index{geodesic!local geodesic}\emph{local geodesic} if for any $t\in\II$ there is a neighborhood $U$ of $t$ in $\II$ such that the restriction $\gamma|_U$ is a  geodesic.
A constant-speed parametrization of a local geodesic by the unit interval $[0,1]$ is called a \index{geodesic!local geodesic}\emph{local geodesic path}. 



\section{Length}

A \emph{curve} is defined as a continuous map from a real interval to a space.
If the real interval is $[0,1]$, then the curve is called a \emph{path}.

\begin{thm}{Definition}
Let $\spc{X}$ be a metric space and
$\alpha\: \II\to \spc{X}$ be a curve.
We define the \index{length}\emph{length} of $\alpha$ as 
\[
\length \alpha \df \sup_{t_0\le t_1\le\ldots\le t_n}\sum_i \dist{\alpha(t_i)}{\alpha(t_{i-1})}{}.
\]

A curve $\alpha$ is called \emph{rectifiable} if $\length \alpha<\infty$.
\end{thm}



\begin{thm}{Theorem}\label{thm:length-semicont}
Length is a lower semi-continuous with respect to pointwise convergence of curves. 

More precisely, assume that a sequence
of curves $\gamma_n\:\II\to \spc{X}$ in a metric space $\spc{X}$ converges pointwise 
to a curve $\gamma_\infty\:\II\to \spc{X}$;
that is, for any fixed $t \in \II$, $\gamma_n(t)\z\to\gamma_\infty(t)$ as $n\to\infty$. 
Then 
$$\liminf_{n\to\infty} \length\gamma_n \ge \length\gamma_\infty.\eqlbl{eq:semicont-length}$$
\end{thm}


\begin{wrapfigure}{o}{20 mm}
\vskip-0mm
\centering
\includegraphics{mppics/pic-10}
\end{wrapfigure}


Note that the inequality \ref{eq:semicont-length} might be strict.
For example the diagonal $\gamma_\infty$ of the unit square 
can be  approximated by a stairs-like
polygonal curves $\gamma_n$
with sides parallel to the sides of the square ($\gamma_6$ is on the picture).
In this case
\[\length\gamma_\infty=\sqrt{2}\quad
\text{and}\quad \length\gamma_n=2\]
for any $n$.

\parit{Proof.}
Fix a sequence $t_0<t_1<\dots<t_k$ in $\II$.
Set 
\begin{align*}\Sigma_n
&\df
|\gamma_n(t_0)-\gamma_n(t_1)|+\dots+|\gamma_n(t_{k-1})-\gamma_n(t_k)|.
\\
\Sigma_\infty
&\df
|\gamma_\infty(t_0)-\gamma_\infty(t_1)|+\dots+|\gamma_\infty(t_{k-1})-\gamma_\infty(t_k)|.
\end{align*}

Note that for each $i$ we have 
\[|\gamma_n(t_{i-1})-\gamma_n(t_i)|\to|\gamma_\infty(t_{i-1})-\gamma_\infty(t_i)|\]
and therefore
\[\Sigma_n\to \Sigma_\infty\] 
as $n\to\infty$.
Note that 
\[\Sigma_n\le\length\gamma_n\]
for each $n$.
Hence
$$\liminf_{n\to\infty} \length\gamma_n \ge \Sigma_\infty.\eqlbl{>=Sigma-infty}$$

If $\gamma_\infty$ is rectifiable, we can assume that 
\begin{align*}
\length\gamma_\infty<\Sigma_\infty+\eps.
\end{align*}
for any given $\eps>0$.
By \ref{>=Sigma-infty} it follows that 
$$\liminf_{n\to\infty} \length\gamma_n > \length\gamma_\infty-\eps$$
for any $\eps>0$; whence \ref{eq:semicont-length} follows.

It remains to consider the case when $\gamma_\infty$ is not rectifiable; 
that is, $\length\gamma_\infty=\infty$.
In this case we can choose a partition so that $\Sigma_\infty>L$ for any real number $L$.
By \ref{>=Sigma-infty} it follows that 
$$\liminf_{n\to\infty} \length\gamma_n > L$$
for any given $L$; whence 
\[\liminf_{n\to\infty}\length\gamma_n=\infty\]
and \ref{eq:semicont-length} follows.
\qeds

\section{Length spaces}\label{sec:intrinsic}

If for any $\eps>0$ and any pair of points $x$ and $y$ in a metric space $\spc{X}$, there is a path $\alpha$ connecting $x$ to $y$ such that
\[\length\alpha< \dist{x}{y}{}+\eps,\]
then $\spc{X}$ is called a \index{length space}\emph{length space} and the metric on $\spc{X}$ is called a \index{length metric}\emph{length metric}.\label{page:length metric}

Note that any geodesic space is a length space.
As can be seen from the following example, the converse does not hold.


\begin{thm}{Example}
Let $\spc{X}$ be obtained by gluing a countable collection of disjoint intervals $\{\II_n\}$ of length $1+\tfrac1n$, where for each $\II_n$ the left end is glued to $p$ and the right end to~$q$.

Observe that the space $\spc{X}$ carries a natural complete length metric with respect to which $\dist{p}{q}{}=1$ but there is no geodesic connecting $p$ to~$q$.
\end{thm}



\begin{thm}{Exercise}\label{ex:no-geod}
Give an example of a complete length space for which no pair of distinct points can be joined by a geodesic.
\end{thm}

Directly from the definition, it follows that if a path $\alpha\:[0,1]\to\spc{X}$ connects two points $x$ and $y$ 
(that is, if $\alpha(0)=x$ and $\alpha(1)=y$), then 
\[\length\alpha\ge \dist{x}{y}{}.\]
Set 
\[\yetdist{x}{y}{}=\inf\{\length\alpha\}\]
where the greatest lower bound is taken for all paths connecing $x$ and $y$.
It is straightforward to check that $(x,y)\mapsto \yetdist{x}{y}{}$ is an $\infty$-metric; moreover $(\spc{X},\yetdist{*}{*}{})$ is a length space.
The metric $\yetdist{*}{*}{}$ is called \emph{induced length metric}.

\begin{thm}{Exercise}\label{ex:compact=>complete}
Suppose $(\spc{X},\dist{*}{*}{})$ is a complete metric space.
Show that $(\spc{X},\yetdist{*}{*}{})$ is complete.
\end{thm}


Let $A$ be a subset of a metric space $\spc{X}$.
Given two points $x,y\in A$,
consider the value
\[\dist{x}{y}{A}=\inf_{\alpha}\{\length\alpha\},\]
where the greatest lower bound is taken for all paths $\alpha$ from $x$ to $y$ in $A$.%
\footnote{This notation slightly conflicts with the previously defined notation for distance $\dist{x}{y}{\spc{X}}$ in a metric space $\spc{X}$. However, most of the time we will work with ambient length spaces where the meaning will be unambiguous.}

Let $\spc{X}$ be a metric space and $x,y\in\spc{X}$.

\begin{enumerate}[(i)]
\item A point $z\in \spc{X}$ is called a \index{midpoint}\emph{midpoint} between $x$ and $y$
if 
\[\dist{x}{z}{}=\dist{y}{z}{}=\tfrac12\cdot\dist[{{}}]{x}{y}{}.\]
\item Assume $\eps\ge 0$.
A point $z\in \spc{X}$ is called an \index{$\eps$-midpoint}\emph{$\eps$-midpoint} between $x$ and $y$
if 
\[\dist{x}{z}{},\quad\dist{y}{z}{}\le\tfrac12\cdot\dist[{{}}]{x}{y}{}+\eps.\]
\end{enumerate}


Note that a $0$-midpoint is the same as a midpoint.


\begin{thm}{Lemma}\label{lem:mid>geod}
Let $\spc{X}$ be a complete metric space.
\begin{subthm}{lem:mid>length}
Assume that for any pair of points $x,y\in \spc{X}$  
 and any $\eps>0$
there is an $\eps$-midpoint~$z$.
Then $\spc{X}$ is a length space.
\end{subthm}

\begin{subthm}{lem:mid>geod:geod}
Assume that for any pair of points $x,y\in \spc{X}$, 
there is a midpoint~$z$.
Then $\spc{X}$ is a geodesic space.
\end{subthm}
\end{thm}

\parit{Proof.}
We first prove (\ref{SHORT.lem:mid>length}).
Let $x,y\in \spc{X}$ be a pair of points.

Set $\eps_n=\frac\eps{4^n}$, $\alpha(0)=x$ and $\alpha(1)=y$.

Let $\alpha(\tfrac12)$ be an $\eps_1$-midpoint between $\alpha(0)$ and $\alpha(1)$.
Further, let $\alpha(\frac14)$ 
and $\alpha(\frac34)$ be $\eps_2$-midpoints between the pairs $(\alpha(0),\alpha(\tfrac12))$ 
and $(\alpha(\tfrac12),\alpha(1))$ respectively.
Applying the above procedure recursively,
on the $n$-th step we define $\alpha(\tfrac{k}{2^n})$,
for every odd integer $k$ such that $0<\tfrac k{2^n}<1$, 
as an $\eps_{n}$-midpoint between the already defined
$\alpha(\tfrac{k-1}{2^n})$ and $\alpha(\tfrac{k+1}{2^n})$.


In this way we define $\alpha(t)$ for $t\in W$,
where $W$ denotes the set of dyadic rationals in $[0,1]$.
Since $\spc{X}$ is complete, the map $\alpha$ can be extended continuously to $[0,1]$.
Moreover,
\[\begin{aligned}
\length\alpha&\le \dist{x}{y}{}+\sum_{n=1}^\infty 2^{n-1}\cdot\eps_n\le
\\
&\le \dist{x}{y}{}+\tfrac\eps2.
\end{aligned}
\eqlbl{eq:eps-midpoint}
\]
Since $\eps>0$ is arbitrary, we get (\ref{SHORT.lem:mid>length}).

To prove (\ref{SHORT.lem:mid>geod:geod}), 
one should repeat the same argument 
taking midpoints instead of $\eps_n$-midpoints.
In this case \ref{eq:eps-midpoint} holds for $\eps_n=\eps=0$.
\qeds

Since in a compact space a sequence of $\tfrac1n$-midpoints $z_n$ contains a convergent subsequence, Lemma~\ref{lem:mid>geod} immediately implies

\begin{thm}{Proposition}\label{prop:length+proper=>geodesic}
A proper length space is geodesic.
\end{thm}

\begin{thm}{Hopf--Rinow theorem}\label{thm:Hopf-Rinow}
Any complete, locally compact length space is proper.
\end{thm}

It is instructive to solve the following exercise before reading the proof.

\begin{thm}{Exercise}
Give an example of space which is locally compact but not proper.
\end{thm}

\parit{Proof.}
Let $\spc{X}$ be a locally compact length space.
Given $x\in \spc{X}$, denote by $\rho(x)$ the supremum of all $R>0$ such that
the closed ball $\cBall[x,R]$ is compact.
Since $\spc{X}$ is locally compact, 
$$\rho(x)>0
\quad\text{for any}\quad
x\in \spc{X}.\eqlbl{eq:rho>0}$$
It is sufficient to show that $\rho(x)=\infty$ for some (and therefore any) point $x\in \spc{X}$.

Assume the contrary; that is, $\rho(x)<\infty$. We claim that

\begin{clm}{} $B=\cBall[x,\rho(x)]$ is compact for any~$x$.
\end{clm}

Indeed, $\spc{X}$ is a length space;
therefore for any $\eps>0$, 
the set $\cBall[x,\rho(x)-\eps]$ is a compact $\eps$-net in~$B$.
Since $B$ is closed and hence complete, it must be compact.
\claimqeds
Next we claim that
\begin{clm}{} $|\rho(x)-\rho(y)|\le \dist{x}{y}{\spc{X}}$ for any $x,y\in \spc{X}$;
in particular $\rho\:\spc{X}\to\RR$ is a continuous function.
\end{clm}

Indeed, 
assume the contrary; that is, $\rho(x)+|x-y|<\rho(y)$ for some $x,y\in \spc{X}$. 
Then 
$\cBall[x,\rho(x)+\eps]$ is a closed subset of $\cBall[y,\rho(y)]$ for some $\eps>0$.
Then  compactness of $\cBall[y,\rho(y)]$ implies compactness of $\cBall[x,\rho(x)+\eps]$, a contradiction.\claimqeds

Set $\eps=\min\set{\rho(y)}{y\in B}$; the minimum is defined since $B$ is compact.
From \ref{eq:rho>0}, we have $\eps>0$.

Choose a finite $\tfrac\eps{10}$-net $\{a_1,a_2,\dots,a_n\}$ in $B$.
The union $W$ of the closed balls $\cBall[a_i,\eps]$ is compact.
Clearly 
$\cBall[x,\rho(x)+\frac\eps{10}]\subset W$.
Therefore $\cBall[x,\rho(x)+\frac\eps{10}]$ is compact,
a contradiction.
\qeds

\begin{thm}{Exercise}\label{exercise from BH}
Construct a geodesic space that is locally compact,
but whose completion is neither geodesic nor locally compact.
\end{thm}

\section{Subsets in normed spaces}

Recall that a function $v\mapsto |v|$ on a vector space $\spc{V}$ is called \emph{norm} if it satisfies the following condition for any two vectors $v,w\in \spc{V}$ and a scalar $\alpha$:
\begin{itemize}
\item $|v|\ge 0$;
\item $|\alpha\cdot v|=|\alpha|\cdot |v|$;
\item $|v|+|w|\ge|v+w|$.
\end{itemize}

It is straightforward to check that for any normed space the function $(v,w)\mapsto |v-w|$ defines a metric on it.
Therefore any normed space is an example of metric space (which is in fact geodesic).
The following lemma says in particular that any metric space is isometric to a subset of a normed space.

\begin{thm}{Lemma}\label{lem:frechet}
Suppose $\spc{X}$ is a bounded separable space;
that is, $\diam\spc{X}$ is finite and $\spc{X}$ contains a countable, dense set $\{w_n\}$.
Given $x\in \spc{X}$, set $a_n(x)=\dist{w_n}{x}{\spc{X}}$.
Then 
\[\iota\:x\mapsto (a_1(x), a_2(x),\dots)\]
defines a distance preserving embedding $\iota\:\spc{X}\hookrightarrow \ell^\infty$.
\end{thm}

\parit{Proof.}
By the triangle inequality 
\[|a_n(x)-a_n(y)|\le \dist{x}{y}{\spc{X}}.\]
Therefore $\iota$ is short.

Again by triangle inequality we have 
\[|a_n(x)-a_n(y)|\ge \dist{x}{y}{\spc{X}}-2\cdot\dist{w_n}{x}{\spc{X}}.\]
Since the set $\{w_n\}$ is dense, we can choose $w_n$ arbitrary close to $x$.
Whence the value $|a_n(x)-a_n(y)|$ can be chosen arbitrary close to $\dist{x}{y}{\spc{X}}$.
In other words 
\[\sup_n\{\,|\dist{w_n}{x}{\spc{X}}-\dist{w_n}{y}{\spc{X}}|\,\}\ge \dist{x}{y}{\spc{X}};\]
hence $\iota$ is distance non-decreasing.
\qeds

The following exercise generalizes the lemma to arbitrary separable spaces.

\begin{thm}{Exercise}
Suppose $\{w_n\}$ is a countable, dense set in a metric space $\spc{X}$.
Choose $x_0\in \spc{X}$;
given $x\in \spc{X}$, set 
\[a_n(x)=\dist{w_n}{x}{\spc{X}}-\dist{w_n}{x_0}{\spc{X}}.\]
Show that $\iota\:x\mapsto (a_1(x), a_2(x),\dots)$ defines a distance preserving embedding $\iota\:\spc{X}\hookrightarrow \ell^\infty$.
\end{thm}


\begin{thm}{Exercise}\label{ex:compact-length}
Show that any compact metric space is isometric to a subspace of a compact geodesic space. 
\end{thm}

The lemma above was proved by Maurice René Fréchet in the paper where he defined metric space \cite{frechet}.
Nearly identical construction was rediscovered later by Kazimierz Kuratowski~\cite{kuratowski}.
Namely he made the following claim:

\begin{thm}{Lemma}\label{lem:kuratowski}
Let $\spc{X}$ be arbitrary metric space.
Denote by $\ell^\infty(\spc{X})$ the space of all bounded functions of $\spc{X}$ equipped with sup-norm.

Then for any point $x_0\in \spc{X}$, the map $\iota\:\spc{X}\to \ell^\infty(\spc{X})$ defied by 
\[\iota\:x\mapsto (\distfun_x-\distfun_{x_0})\]
is distance preserving.
\end{thm}

Note that this claim implies that \emph{any metric space is isometric to a subset of a normed vector space}.








\chapter{Urysohn space}

We discuss a construction introduced by Pavel Urysohn~\cite{urysohn}.
Our presentation is very close to the one in \cite{gromov-2007}.

This subject is closely related to the so called \emph{Rado graph},
also known as \emph{Erd\H{o}s–R\'enyi graph} or \emph{random graph}; a good survey this subject is written by Peter Cameron~\cite{cameron}.

\section{Existance}
Suppose a metric space $\spc{X}$ is a subspace of a pseudometric space $\spc{X}'$.
In this case we may say that $\spc{X}'$ is an \emph{extension} of $\spc{X}$.
If $\diam\spc{X}'\le d$, then we say that $\spc{X}'$ is a \emph{$d$-extension}.

If the complement $\spc{X}'\backslash \spc{X}$ contains a single point, say $p$, we say that $\spc{X}'$ is a \emph{one-point extension} of $\spc{X}$.
In this case, to define metric on $\spc{X}'$, it is sufficient to specify the distance function from $p$; that is, a function $f\:\spc{X}\to\RR$ defined by 
\[f(x)=\dist{p}{x}{\spc{X}'}.\]

The function $f$ can not be taken arbitrary --- the triangle inequality implies that 
\[f(x)+f(y)\ge \dist{x}{y}{\spc{X}}\ge |f(x)-f(y)|\]
for any $x,y\in \spc{X}$.
In particular $f$ is a non-negative 1-Lipschitz function on $\spc{X}$.
For a $d$-extension we need to assume in addition that $\diam\spc{X}\le d$ and $f(x)\le d$ for any $x\in \spc{X}$.

Any function $f$ of that type will be called \emph{extension function} or \emph{$d$-extension function} correspondingly.

\begin{thm}{Definition}\label{def:universal}
A metric space $\spc{U}$ is called \emph{universal}  if for any finite subspace $\spc{F}\subset\spc{U}$ and any extension function $f\:\spc{F}\to\RR$ there is a point $p\in \spc{U}$ such that $\dist{p}{x}{}=f(x)$ for any $x\in \spc{F}$.

If instead of extension functions we consider only $d$-extension functions and assume in addition that $\diam \spc{U}\le d$, then we arrive to a definition of \emph{$d$-universal space}.

If in addition $\spc{U}$ is separable and complete, then it is called \emph{Urysohn space} or \emph{$d$-Urysohn space}.
\end{thm}


\begin{thm}{Proposition}\label{prop:univeral-separable}
Given a positive $d$, there is a separable $d$-universal metric space.

Moreover, a separable universal space metric exists.
\end{thm}

\parit{Proof.}
Let $\spc{X}$ be a compact metric space such that $\diam\spc{X}\le d$.
Denote by $\spc{X}^d$ the space of all $d$-extension functions on $\spc{X}$ equipped with the metric defined by sup-norm.
Note that the map $\spc{X} \to \spc{X}^d$ defined by $x\mapsto\distfun_x$ is a distance preserving embedding,
so we can (and will) treat $\spc{X}$ as a subspace of $\spc{X}^d$, or, equivalently, $\spc{X}^d$ is an extension of $\spc{X}$.

Let us iterate this construction.
Start with a one-point space $\spc{X}_0$ and consider a sequence of spaces $(\spc{X}_n)$ defined by $\spc{X}_{n+1}\z=\spc{X}_n^d$.
Note that the sequence is nested, that is $\spc{X}_0\subset \spc{X}_1\subset\dots$
and the union
\[\spc{X}_\infty=\bigcup_n\spc{X}_n;\]
comes with metric such that
$\dist{x}{y}{\spc{X}_\infty} = \dist{x}{y}{\spc{X}_n}$
if $x,y\in\spc{X}_n$.

Note that if $\spc{X}$ is compact, then so is $\spc{X}^d$.
It follows that each space $\spc{X}_n$ is compact.
Since $\spc{X}_\infty$ is a countable union of compact spaces, it is separable.

Any finite subspace $\spc{F}$ of $\spc{X}_\infty$ lies in some $\spc{X}_n$ for $n<\infty$.
By construction, there is a point $p\in \spc{X}_{n+1}$ that meets the condition in Definiton~\ref{def:universal}.
That is, $\spc{X}_\infty$ is $d$-universal.

A construction of a universal separable metric space is done along the same lines, but one has the sequence should be defined by $\spc{X}_{n+1}\z=\spc{X}_n^{d_n}$ for some sequence $d_n\to\infty$.
\qeds

\begin{thm}{Proposition}\label{prop:completion-univeral}
A completion of $d$-universlal space is $d$-universal.

A completion of universal space universal.
\end{thm}

Note that \ref{prop:univeral-separable} and \ref{prop:completion-univeral} imply the following:

\begin{thm}{Theorem}\label{thm:urysohn-exists}
Urysohn space, and $d$-Urysohn space for any $d>0$, exist.
\end{thm}


\parit{Proof.} Suppose $\spc{V}$ be a $d$-universal space;
denote by $\spc{U}$ its completion; so $\spc{V}$ is a dense subset in a complete space $\spc{U}$.

Observe that $\spc{U}$ is \emph{approximately $d$-universal};
that is, if $\spc{F}\subset\spc{U}$ is a finite set, and $f\:\spc{F}\to \RR$ is a $d$-extension function, then
there exists $p\in \spc{U}$ such that
\[\dist{p}{x}{}\lg f(x)\pm\eps.\]
for any $x\in\spc{F}$.

Therefore there is a sequence of points $p_n\in \spc{U}$ such that for any $x\in \spc{F}$, 
\[\dist{p_n}{x}{}\lg f(x)\pm\tfrac1{2^n}.\]

Moreover, we can assume that 
\[\dist{p_n}{p_{n+1}}{} < \tfrac1{2^n}\eqlbl{eq:|pn-pn|}\]
for all large $n$.
Indeed, consider the sets $\spc{F}_n=\spc{F}\cup\{p_n\}$ and the functions $f_n$ defined by $f_n(x)=f(x)$ for any $x\in \spc{F}$, and
\[f_n(p_n)=\max\set{\bigl|\dist{p_n}{x}{}- f(x)\bigr|}{x\in \spc{F}}.\]
Observe that $f_n$ is a an $d$-extension function for large $n$ and
$f_n(p_n)\z<\tfrac1{2^n}$.
By applying approximate universal property recursively we get~\ref{eq:|pn-pn|}.

By \ref{eq:|pn-pn|}, $(p_n)$ is a Cauchy sequence and its limit meets the condition in the definition of universal space (\ref{def:universal}).
\qeds

\section{Universality}

\begin{thm}{Proposition}\label{prop:sep-in-urys}
Let $\spc{U}$ a Urysohn space.
Then any separable metric space $\spc{S}$ admits a distance preserving embedding $\spc{S}\hookrightarrow\spc{U}$.

Moreover, for any finite subspace $\spc{F}\subset \spc{S}$,
any distance preserving embedding $\spc{F}\hookrightarrow \spc{U}$ can be extended to an distance preserving embedding $\spc{S}\hookrightarrow\spc{U}$.

If $\spc{U}$ is $d$-Urysohn,
then the statements hold provided $\diam\spc{S}\le d$.  
\end{thm}

\parit{Proof.}
The required isometry will be denoted by $x\mapsto x'$.

Choose a dense sequence of points $s_1,s_2,\dotsc\in\spc{S}$.
We may assume that $\spc{F}=\{s_1,\dots,s_n\}$, so $s_i'\in \spc{U}$ are defined for $i\le n$.

The sequence $s_i'$ for $i>n$ can be defined recursively using universality of $\spc{U}$.
Namely that $s_1',\dots,s_{i-1}'$ are already defined.
Since $\spc{U}$ is universal, there is a point $s_i'\in \spc{U}$ such that
\[\dist{s_i'}{s_j'}{\spc{U}}=\dist{s_i}{s_j}{\spc{S}}\]
for any $j<i$.

We constructed a distance preserving map $s_i\mapsto s_i'$, it remains to extend it to a continuous map on whole $\spc{S}$.

The first statement follows if $\spc{F}=\emptyset$.\qeds

\begin{thm}{Exercise}\label{ex:geodesics-urysohn}
Show that any two distinct points in an Urysohn space can be jointed by infinite number of geodesics.
\end{thm}

\begin{thm}{Exercise}\label{ex:sc-urysohn}
Show that Urysohn space is simply connected.
\end{thm}

\section{Uniqueness}

\begin{thm}{Theorem}\label{thm:urysohn-unique}
Suppose $\spc{F}\subset \spc{U}$ and $\spc{F}'\subset \spc{U}'$ be finite isometric subspaces in a pair of ($d$-)Urysohn spaces $\spc{U}$ and $\spc{U}'$.
Then any isometry $\spc{F}\to \spc{F}'$ can be extended to an isometry $\spc{U}\to \spc{U}'$.

In particular ($d$-)Urysohn space is unique up to isometry.
\end{thm}

Note that \ref{prop:sep-in-urys} implies that there are distance-preserving maps $\spc{U}\to \spc{U}'$ and $\spc{U}'\to \spc{U}$,
but it does not immideately imply existence of an isometry.
The following construction use the same idea as in the proof of \ref{prop:sep-in-urys}, but we need to apply it \emph{back and forth} to ensure that the constructed distance-preserving map is onto.

\parit{Proof.}
The required isometry $\spc{U}\leftrightarrow \spc{U}'$ will be denoted by $u \leftrightarrow u'$.

Choose a dense sequences $a_1,a_2,\dots\in \spc{U}$ and $b'_1,b'_2,\dots\in \spc{U}$.
Let us define recursively $a_1',b_1, a_2', b_2,\dots$ --- on the odd step we define the images of $a_1,a_2,\dots$ and on the even steps we define invese images of $b'_1,b'_2,\dots$.
The same argument as in the proof of \ref{prop:sep-in-urys} shows that we can construct two sequences $a_1',a_2',\dots\in \spc{U}'$ and $b_1,b_2,\dots\in \spc{U}$ such that
\begin{align*}
\dist{a_i}{a_j}{\spc{U}}&=\dist{a_i'}{a_j'}{\spc{U}'}
\\
\dist{a_i}{b_j}{\spc{U}}&=\dist{a_i'}{b_j'}{\spc{U}'}
\\
\dist{b_i}{b_j}{\spc{U}}&=\dist{b_i'}{b_j'}{\spc{U}'}
\end{align*}
for all $i$ and $j$.

Let us extend the constructed distance preserving bijection defined by $a_i\leftrightarrow a_i'$ and $b_i\leftrightarrow b_i'$ continuousely to whole $\spc{U}$.
Observe that the image of this bijection is dense in $\spc{U}'$ therefore the constructed map $\spc{U}\to \spc{U}'$ is a bijection.
\qeds

Further the Urysohn space will be denoted by $\spc{U}$, and the $d$-Urysohn space will be denoted by $\spc{U}_d$.
Observe that \ref{thm:urysohn-unique} implies that the spaces $\spc{U}$ and $\spc{U}_d$ are finite-set homogeneous; that is,
\begin{itemize}
 \item any distance preserving map from a finite subset to to the whole space can be extended to an isometry.
\end{itemize}
It is unknown if there is a separable universal space that is finite-set homogeneous (this question appeared already in \cite{urysohn} and reappeared in \cite[p. 83]{gromov-2007} with a missing key word). 


\begin{thm}{Exercise}\label{ex:sphere-in-urysohn}
Let $S$ be a sphere of radius $\tfrac d2$ in $\spc{U}_d$;
that is, 
\[S=\set{x\in \spc{U}_d}{\dist{p}{x}{\spc{U}_d}=\tfrac d2}\]
for some point $p\in \spc{U}_d$.
Show that $S$ is isometric to $\spc{U}_d$.

Use it to show that $\spc{U}_d$ is not countable-set homogeneous;
that is there is an distance preserving map from a countable subset of $\spc{U}_d$ to $\spc{U}_d$ that can not be extended to an isometry $\spc{U}_d\to \spc{U}_d$.
\end{thm}


\begin{thm}{Exercise}
Modify the proofs of \ref{prop:completion-univeral} and \ref{thm:urysohn-unique} to prove the following theorem.
\end{thm}

\begin{thm}{Theorem}\label{thm:compact-homogeneous}
Let $K\subset \spc{U}$ be a compact set.
Show that any distance-preserving map $f\:K\to\spc{U}$ can be extended to 
an isometry of~$\spc{U}$.
\end{thm}











\chapter{Injective spaces}

\textit{Injective spaces} (also known as \textit{hyperconvex spaces}) are the metric analog of convex sets in the following sense:

\begin{thm}{Advanced exercise}\label{ex:conv-short}
Show that $A\subset \RR^n$ is a closed convex set if and only if for any  $B\subset \RR^n$ any short map $B\to A$ can be extended to a short map $\RR^n\to A$.
\end{thm}

\section{Definition}

\begin{thm}{Definition}\label{def:injective}
A metric space $\spc{Y}$ is called \index{injective space}\emph{injective} if for any metric space $\spc{X}$ and any of its subspaces $\spc{A}$
any short map $f\:\spc{A}\to \spc{Y}$ can be extended to a short map $F\:\spc{X}\to \spc{Y}$;
that is, $f=F|_{\spc{A}}$.
\end{thm}

\begin{thm}{Exercise}\label{ex:inj=complete-geodesic-contractible}
Show that any injective space is 
\begin{multicols}{3}

\begin{subthm}{ex:inj=complete-geodesic-contractible:complete}
complete,
\end{subthm}

\begin{subthm}{ex:inj=complete-geodesic-contractible:geodesic}
geodesic, and
\end{subthm}

\begin{subthm}{ex:inj=complete-geodesic-contractible:contractible}
contractible.
\end{subthm}

\end{multicols}

\end{thm}

\begin{thm}{Exercise}\label{ex:bicombing}
Let $\spc{Y}$ be an injective space.
Show that one can choose a geodesic path $\gamma_{x,y}\:[0,1]\to \spc{Y}$ from any $x\in \spc{Y}$ to any $y\in \spc{Y}$ such that
$\gamma_{x,y}(t)\equiv\gamma_{y,x}(1-t)$ and
\[\dist{\gamma_{x,y}(t)}{\gamma_{p,q}(t)}{\spc{Y}}\le (1-t)\cdot\dist{p}{x}{\spc{Y}}+t\cdot\dist{q}{y}{\spc{Y}}\]
for any $x,y,p,q\in \spc{Y}$.
\end{thm}

\begin{thm}{Exercise}\label{ex:injective-spaces}
Show that the following spaces are injective:
\begin{subthm}{ex:injective-spaces:R}
the real line;
\end{subthm}


\begin{subthm}{ex:injective-spaces:tree}
complete metric tree;
\end{subthm}

\begin{subthm}{ex:injective-spaces:ell-infty}
coordinate plane with the metric induced by the $\ell^\infty$-norm.
\end{subthm}

%\begin{subthm}{ex:injective-spaces:L-infty}$L^\infty([0,1])$.\end{subthm}%%%???

\end{thm}

\begin{thm}{Exercise}\label{ex:extr-ball}
Let $\spc{Y}$ be an injective space.

\begin{subthm}{ex:extr-ball:one}
Show that any closed ball in $\spc{Y}$ is injective.
\end{subthm}

\begin{subthm}{ex:extr-ball:many}
Show that intersection of an arbitrary collection of closed ball in $\spc{Y}$ is injective.
\end{subthm}

\end{thm}

\begin{thm}{Advanced exercise}\label{ex:extr-fixed}
Let $\spc{Y}$ be a bounded injective space.
Show that any short map $s\:\spc{Y}\to\spc{Y}$ has a fixed point. 
\end{thm}


\section{Admissible and extremal functions}

Let $\spc{X}$ be a metric space.
A function $r\:\spc{X}\to\RR$ is called \label{page:admissible function}\index{admissible function}\emph{admissible} if the following inequality
\[r(x)+r(y)\ge \dist{x}{y}{\spc{X}}\eqlbl{eq:admissible}\]
holds for any $x,y\in \spc{X}$.

\begin{thm}{Observation}\label{obs:admissible}

\begin{subthm}{obs:admissible:nonnegative}
Any admissible function is nonnegative.
\end{subthm}

\begin{subthm}{obs:admissible:balls}
If $\spc{X}$ is a geodesic space, then a function $r\:\spc{X}\to\RR$ is admissible if and only if 
\[\cBall[x,r(x)]\cap\cBall[y,r(y)]\ne \emptyset\]
for any $x,y\in \spc{X}$.
\end{subthm}
 
\end{thm}

\parit{Proof.} For \ref{SHORT.obs:admissible:nonnegative}, take $x=y$ in \ref{eq:admissible}.

Part \ref{SHORT.obs:admissible:balls} follows from the triangle inequality and the existence of a geodesic $[xy]$.
\qeds

A minimal admissible function will be called \label{page:extremal function}\index{extremal function}\emph{extremal}.
More precisely, an admissible function $r\:\spc{X}\to\RR$ is extremal 
if for any admissible function $s\:\spc{X}\to\RR$ we have
\[s\le r\quad\Longrightarrow\quad s=r.\]

Applying Zorn's lemma, we get the following.

\begin{thm}{Observation}\label{obs:extremal:below}
For any admissible function $s$ there is an extremal function $r$ such that $r\le s$.
\end{thm}

\begin{thm}{Lemma}\label{lem:+-c}
Let $r$ be an extremal function and $s$ an admissible function on a metric space $\spc{X}$.
Suppose that $r\ge s-c$ for some constant~$c$.
Then $r\le s+c$; in particular, $c\ge 0$.
\end{thm}

\parit{Proof.}
Note that if $c<0$, then $r>s$.
The latter is impossible since $r$ is extremal and $s$ is admissible.

Observe that the function $\bar r=\min\{\,r,s+c\,\}$ is admissible.
Indeed, choose $x,y\in \spc{X}$.
If $\bar r(x)=r(x)$ and $\bar r(y)=r(y)$, then 
\[\bar r(x)+\bar r(y)=r(x)+ r(y)\ge \dist{x}{y}{}.\]
Further, if $\bar r(x)=s(x)+c$, then 
\begin{align*}
\bar r(x)+\bar r(y)&\ge [s(x)+c]+ [s(y)-c]= 
\\
&=s(x)+s(y) \ge 
\\
&\ge\dist{x}{y}{}.
\end{align*}

Since $r$ is extremal, we have $r=\bar r$;
that is, $r\le s+c$.
\qeds

\begin{thm}{Observations}\label{obs:extremal}
Let $\spc{X}$ be a metric space.

\begin{subthm}{obs:extremal:distfun}
For any point $p\in\spc{X}$
the distance function $r\z=\distfun_p$ is extremal.
\end{subthm}

\begin{subthm}{lem:extremal-lipschitz}
Any extremal function $r$ on $\spc{X}$ is \index{1-Lipschitz function}\emph{1-Lipschitz};
that is,
\[|r(p)-r(q)|\le \dist{p}{q}{}\]
for any $p,q\in\spc{X}$.
In other words, any extremal function is an extension function; see the definition in \ref{sec:Extension property}.
\end{subthm}

\begin{subthm}{lem:opposite}
An admissible function $r$ on $\spc{X}$ is extremal if and only if
for any point $p\in\spc{X}$ and any $\delta>0$, there is a point $q\in \spc{X}$
such that 
\[r(p)+r(q)<\dist{p}{q}{\spc{X}}+\delta.\]
\end{subthm}

\begin{subthm}{lem:opposite-compact}
Suppose $\spc{X}$ is compact.
Then an admissible function $r$ on $\spc{X}$ is extremal if and only if
for any point $p\in\spc{X}$ there is a point $q\in \spc{X}$
such that 
\[r(p)+r(q)=\dist{p}{q}{\spc{X}}.\]
\end{subthm}

\end{thm}

\parit{Proof; \ref{SHORT.obs:extremal:distfun}.}
By the triangle inequality, \ref{eq:admissible} holds;
that is, $r=\distfun_p$ is an admissible function.

Further, if $s\le r$ is another admissible function, then $s(p)=0$ and \ref{eq:admissible} implies that $s(x)\z\ge\dist{p}{x}{}$.
Whence $s=r$.

\parit{\ref{SHORT.lem:extremal-lipschitz}.}
By \ref{SHORT.obs:extremal:distfun}, $\distfun_p$ is admissible.
Since $r$ is admissible, we have that
\[r\ge \distfun_p-r(p).\]
Since $r$ is extremal, \ref{lem:+-c} implies that
\[r\le \distfun_p+r(p),\]
or, equivalently,
\[r(q)-r(p)\le \dist{p}{q}{}\]
for any $p,q\in\spc{X}$.
The same way we can show that
$r(p)-r(q)\le \dist{p}{q}{}$.
Whence the statement follows.

\parit{\ref{SHORT.lem:opposite}.}
Assume $r$ is extremal.
Arguing by contradiction, assume there is $\delta>0$ such that
\[r(q)\ge \distfun_p(q)-r(p)+\delta\]
for any $q$.
By \ref{SHORT.obs:extremal:distfun}, $\distfun_p$ is extremal; in particular, admissible.
Therefore \ref{lem:+-c} implies that
\[r(q)\le \distfun_p(q)+r(p)-\delta\]
for any $q$.
Taking $q=p$, we get $r(p)\le r(p)-\delta$, a contradiction.

Now suppose $r$ is not extremal; that is, there is an admissible function $s\le r$ such that $r(p)-s(p)=\delta>0$ for some $p$.
Then, for any $q$, we have
\[r(p)+r(q)\ge s(p)+s(q)+\delta\ge \dist{p}{q}{\spc{X}}+\delta\]
--- a contradiction.

\parit{\ref{SHORT.lem:opposite-compact}.}
The if part follows from \ref{SHORT.lem:opposite}.

Denote by $q_n$ the point provided by \ref{SHORT.lem:opposite} for $\delta=\tfrac1n$.
Let $q$ be a partial limit of $q_n$. 
Then 
\[r(p)+r(q)\le\dist{p}{q}{\spc{X}}.\]
Since $r$ is admissible, the opposite inequality holds;
whence the only-if part follows.
\qeds

\begin{thm}{Exercise}\label{ex:circle}
Consider the unit circle $\mathbb{S}^1=\set{(x,y)}{x^2+y^2=1}$ in the plane with induced length metric.
Show that $r\:\mathbb{S}^1\to\RR$ is extremal if and only if it is 1-Lipschitz and 
\[r(p)+r(-p)=\pi\] for any $p\in\mathbb{S}^1$.
\end{thm}

\begin{thm}{Exercise}\label{ex:retraction}
Given a real-valued function $s$ on a metric space $\spc{X}$,
consider the function
\[s^*(x)=\sup\set{\dist{z}{y}{\spc{X}}-s(y)}{y\in \spc{X}}\]
Show that if $s$ is admissible then so is $\tfrac12\cdot(s+s^*)$.
\end{thm}

\section{Equivalent conditions}

\begin{thm}{Theorem}\label{thm:injective=hyperconvex}
For any metric space $\spc{Y}$ the following condition are equivalent:

\begin{subthm}{thm:injective=hyperconvex:injective}
$\spc{Y}$ is injective
\end{subthm}


\begin{subthm}{thm:injective=hyperconvex:extremal}
If $r\:\spc{Y}\to\RR$ is an extremal function, then there is a point $p\in \spc{Y}$ such that 
\[\dist{p}{x}{}\le r(x)\]
for any $x\in \spc{Y}$.
\end{subthm}

\begin{subthm}{thm:injective=hyperconvex:balls}
$\spc{Y}$ is \index{hyperconvex space}\emph{hyperconvex};
that is, if $\set{\cBall[x_\alpha,r_\alpha]}{\alpha\in\IndexSet}$ is a family of closed balls in $\spc{Y}$ such that 
 \[r_\alpha+r_\beta\ge \dist{x_\alpha}{x_\beta}{}\]
 for any $\alpha,\beta\in \IndexSet$, then all the balls in the family $\{\cBall[x_\alpha,r_\alpha]\}_{\alpha\in\IndexSet}$ have a common point.
\end{subthm}

\end{thm}

\parit{Proof.} We will prove implications 
\ref{SHORT.thm:injective=hyperconvex:injective}$\Rightarrow$\ref{SHORT.thm:injective=hyperconvex:extremal}$\Rightarrow$\ref{SHORT.thm:injective=hyperconvex:balls}$\Rightarrow$\ref{SHORT.thm:injective=hyperconvex:injective}.

\parit{\ref{SHORT.thm:injective=hyperconvex:injective}$\Rightarrow$\ref{SHORT.thm:injective=hyperconvex:extremal}.}
Let us apply the definition of injective space to a one-point extension of $\spc{Y}$.
It follows that for any extension function $r\:\spc{Y}\to\RR$ there is a point $p\in \spc{Y}$ such that 
\[\dist{p}{x}{}\le r(x)\]
for any $x\in \spc{Y}$.
By \ref{lem:extremal-lipschitz}, any extremal function is an extension function, whence the implication follows.

\parit{\ref{SHORT.thm:injective=hyperconvex:extremal}$\Rightarrow$\ref{SHORT.thm:injective=hyperconvex:balls}.}
By \ref{obs:admissible:balls}, part \ref{SHORT.thm:injective=hyperconvex:balls} is equivalent to the following statement:
\begin{itemize}
 \item If $r\:\spc{Y}\to\RR$ is an admissible function, then there is a point $p\in \spc{Y}$ such that 
\[\dist{p}{x}{}\le r(x)\eqlbl{eq:|p-x|=<r(x)}\]
for any $x\in \spc{Y}$.
\end{itemize}
Indeed, set $r(x)\df\inf\set{r_\alpha}{x_\alpha=x}$.
(If $x_\alpha\ne x$ for any $\alpha$, then $r(x)=\infty$.)
The condition in \ref{SHORT.thm:injective=hyperconvex:balls} implies that $r$ is admissible.
It remains to observe that $p\in \cBall[x_\alpha,r_\alpha]$ for every $\alpha$ if and only if \ref{eq:|p-x|=<r(x)} holds.

By \ref{obs:extremal:below}, for any admissible function $r$ there is an extremal function $\bar r\le r$;
hence \ref{SHORT.thm:injective=hyperconvex:extremal}$\Rightarrow$\ref{SHORT.thm:injective=hyperconvex:balls}.

\parit{\ref{SHORT.thm:injective=hyperconvex:balls}$\Rightarrow$\ref{SHORT.thm:injective=hyperconvex:injective}.}
Arguing by contradiction, suppose $\spc{Y}$ is not injective;
that is, there is a metric space $\spc{X}$ with a subset $\spc{A}$
such that a short map $f\:\spc{A}\to \spc{Y}$ cannot be extended to a short map $F\:\spc{X}\to \spc{Y}$.
By Zorn's lemma, we may assume that $\spc{A}$ is a maximal subset; that is, the domain of $f$ cannot be enlarged by a single point.%
\footnote{In this case, $\spc{A}$ must be closed, but we will not use it.}

Fix a point $p$ in the complement $\spc{X}\setminus \spc{A}$.
To extend $f$ to $p$, we need to choose $f(p)$ in the intersection of the balls 
$\cBall[f(x),r(x)]$, where $r(x)=\dist{p}{x}{}$.
Therefore, this intersection for all $x\in \spc{A}$ has to be empty.

Since $f$ is short, we have that 
\begin{align*}
r(x)+r(y)&\ge \dist{x}{y}{\spc{X}}\ge
\\
&\ge \dist{f(x)}{f(y)}{\spc{Y}}.
\end{align*}
Therefore, by \ref{SHORT.thm:injective=hyperconvex:balls} the balls 
$\cBall[f(x),r(x)]$ have a common point --- a contradiction. 
\qeds

\begin{thm}{Exercise}\label{ex:one-point-gluing}
Suppose a length space $\spc{W}$ has two subspaces $\spc{X}$ and $\spc{Y}$ such that $\spc{X}\cup\spc{Y}=\spc{W}$ and $\spc{X}\cap\spc{Y}$ is a one-point set.
Assume $\spc{X}$ and $\spc{Y}$ are injective.
Show that  $\spc{W}$ is injective
\end{thm}

\begin{thm}{Exercise}\label{ex:Rm-ell-infty}
Show that a $m$-dimensional normed space is injective if and only if it is isometric to $\RR^m$ with the norm
\[|(x_1,\dots,x_m)|=\max_i\{\,|x_i|\,\}.\]
\end{thm}


\begin{thm}{Exercise}\label{ex:urysohn-hyperconvex}
Show that the $d$-Urysohn space is {}\emph{finitely hyperconvex} but not {}\emph{countably hyperconvex};
that is, the condition in \ref{thm:injective=hyperconvex:balls} holds for any finite family of balls, but may not hold for a countable family.
Conclude that the $d$-Urysohn space is not injective.

Try to do the same for the Urysohn space.
\end{thm}

\section{Space of extremal functions}
\label{sec:extremal-functions}

Let $\spc{X}$ be a metric space.
Consider the space $\Inj \spc{X}$ of extremal functions on $\spc{X}$ equipped with sup-norm; \label{page:InjX}
that is,
\[\dist{f}{g}{\Inj \spc{X}}\df\sup\set{|f(x)-g(x)|}{x\in \spc{X}}.\]

Recall that by \ref{obs:extremal:distfun}, any distance function is extremal.
It follows that the map $x\mapsto \distfun_x$ produces a distance-preserving embedding $\spc{X}\hookrightarrow\Inj \spc{X}$.
So we can (and will) treat $\spc{X}$ as a subspace of $\Inj \spc{X}$,
or, equivalently, $\Inj \spc{X}$ as an extension of $\spc{X}$.

Since any extremal function is 1-Lipschitz, for any $f\in \Inj \spc{X}$ and $p\in \spc{X}$, we have that
$f(x)\le f(p)+\distfun_p(x)$.
By \ref{lem:+-c}, we also get $f(x)\ge -f(p)+\distfun_p(x)$.
Therefore
\[
\begin{aligned}
\dist{f}{p}{\Inj \spc{X}}&=\sup\set{|f(x)-\distfun_p(x)|}{x\in \spc{X}}=
\\
&=f(p).
\end{aligned}
\eqlbl{eq:f(p)=|f-p|}
\]
In particular, the statement in \ref{lem:opposite} can be written as 
\[\dist{f}{p}{\Inj\spc{X}}+\dist{f}{q}{\Inj\spc{X}}<\dist{p}{q}{\Inj\spc{X}}+\delta.\]

\begin{thm}{Exercise}\label{ex:Inj(compact)}
Let $\spc{X}$ be a metric space.
Show that $\Inj\spc{X}$ is compact if and only if so is $\spc{X}$.
\end{thm}

\begin{thm}{Exercise}\label{ex:tripod+square}
Describe the set of all extremal functions on a metric space $\spc{X}$ and the metric space $\Inj \spc{X}$ in each of the following cases:

\begin{subthm}{ex:tripod+square:2}
$\spc{X}$ is a metric space with exactly two points $v,w$ on distance 1 from each other.
\end{subthm}


\begin{subthm}{ex:tripod+square:tripod} 
$\spc{X}$ is a metric space with exactly three points $a,b,c$ such that 
\[\dist{a}{b}{\spc{X}}=\dist{b}{c}{\spc{X}}=\dist{c}{a}{\spc{X}}=1.\]
\end{subthm}

\begin{subthm}{ex:tripod+square:square}
$\spc{X}$ is  a metric space with exactly four points $p,q,x,y$ such that 
\[\dist{p}{x}{\spc{X}}=\dist{p}{y}{\spc{X}}=\dist{q}{x}{\spc{X}}=\dist{q}{y}{\spc{X}}=1\]
and
\[\dist{p}{q}{\spc{X}}=\dist{x}{y}{\spc{X}}=2.\]
\end{subthm}

\end{thm}

\begin{thm}{Exercise}\label{ex:kur-inj}
Assume $\spc{X}$ is a compact metric space.
Recall that the map $x\mapsto \distfun_x$ gives an isometric embedding $\spc{X}\hookrightarrow\ell^\infty(\spc{X})$; so we can think that $\spc{X}$ is a subset of $\ell^\infty(\spc{X})$.

Given two points $x,y\in \spc{X}$, denote by $G_{x,y}$ the union of all geodesics from $x$ to $y$ in $\ell^\infty(\spc{X})$.
Show that $\Inj\spc{X}$ is isometric to
\[G=\bigcap_{x\in \spc{X}}\left(\bigcup_{y\in \spc{X}}G_{x,y}\right).\]

\end{thm}


\begin{thm}{Proposition}\label{prop:InjX-is-injective}
For any metric space $\spc{X}$, its extension $\Inj\spc{X}$ is  injective.
\end{thm}

\begin{thm}{Lemma}\label{lem:r|X-extremal}
Let $\spc{X}$ be a metric space.
Suppose $r\in \Inj(\Inj \spc{X})$;
that is, $r$ is an extremal function on $\Inj \spc{X}$.
Then $r|_\spc{X}\in \Inj \spc{X}$;
that is, the restriction of $r$ to $\spc{X}$ is an extremal function.
\end{thm}

\parit{Proof.}
Arguing by contradiction, suppose that there is an admissible function $s\:\spc{X}\to \RR$ such that $s(x)\le r(x)$ for any $x\in\spc{X}$ and $s(p)\z< r(p)$ for some point $p\in\spc{X}$.
Consider another function $\bar r\:\Inj \spc{X}\to\RR$ such that $\bar r(f)\df r(f)$ if $f\ne p$ and $\bar r(p)\df s(p)$.

Let us show that $\bar r$ is admissible; that is, 
\[\dist{f}{g}{\Inj \spc{X}}\le\bar r(f)+\bar r(g)
\eqlbl{r-admissible}\]
for any $f,g\in \Inj \spc{X}$.

Since $r$ is admissible and $\bar r= r$ on $(\Inj \spc{X})\setminus \{p\}$, it is sufficient to prove \ref{r-admissible} if $f\ne g=p$.
By \ref{eq:f(p)=|f-p|}, we have $\dist{f}{p}{\Inj \spc{X}}=f(p)$.
Therefore, \ref{r-admissible} boils down to the following inequality
\[r(f)+s(p)\ge f(p).\eqlbl{eq:r(f)+s(p)>=f(p)}\]
for any $f\in\Inj \spc{X}$.

Fix small $\delta>0$. 
Let $q\in\spc{X}$ be the point provided by \ref{lem:opposite}.
Then
\begin{align*}
r(f)+s(p)&\ge [r(f)-r(q)]+[r(q)+s(p)]\ge
\intertext{since $r$ is 1-Lipschitz, and $r(q)\ge s(q)$, we can continue}
&\ge -\dist{q}{f}{\Inj \spc{X}}+[s(q)+s(p)]\ge
\intertext{by \ref{eq:f(p)=|f-p|} and since $s$ is admissible}
&\ge -f(q)+\dist{p}{q}{}>
\intertext{and by \ref{lem:opposite}}
&> f(p)-\delta.
\end{align*}
Since $\delta>0$ is arbitrary, \ref{eq:r(f)+s(p)>=f(p)} and \ref{r-admissible} follow.

Summarizing: the function $\bar r$ is admissible, $\bar r\le r$ and $\bar r(p)<r(p)$;
that is, $r$ is not extremal --- a contradiction.
\qeds

\parit{Proof of \ref{prop:InjX-is-injective}.}
Choose a function $r\in\Inj(\Inj\spc{X})$.
By \ref{lem:r|X-extremal}, $s\z\df r|_{\spc{X}}\in \Inj\spc{X}$;
that is, $s$ is extremal.
By \ref{thm:injective=hyperconvex:extremal},
it is sufficient to show that  
\[r(f)\ge\dist{s}{f}{\Inj\spc{X}}
\eqlbl{eq:r(f)>=| r-f|}\]
for any $f\in\Inj\spc{X}$.

Since $r$ is $1$-Lipschitz (\ref{lem:extremal-lipschitz}) we have that
\[
s(x)-f(x)=r(x)-\dist{f}{x}{\Inj \spc{X}}\le r(f).
\]
for any $x\in\spc{X}$.
By \ref{lem:+-c},
$
s(x)-f(x)\ge -r(f)
$
for any $x\in\spc{X}$.
Whence \ref{eq:r(f)>=| r-f|} follows.
\qeds

\begin{thm}{Exercise}\label{ex:4-on-a-line}
Let $\spc{X}$ be a compact metric space.
Show that for any two points $f,g\in\Inj \spc{X}$ lie on a geodesic $[pq]$ with $p,q\in \spc{X}$.
\end{thm}

A metric space $\spc{X}$ is called \index{$\delta$-hyperbolic}\emph{$\delta$-hyperbolic} if 
\[\dist{p}{q}{}+\dist{x}{y}{}\le
\max\{\,\dist{p}{x}{}+\dist{q}{y}{},
\,
\dist{p}{y}{}+\dist{q}{x}{}\,\}+2\cdot\delta\]
for any $p,q,x,y\in \spc{X}$.

\begin{thm}{Advanced exercise}\label{ex:delta-hyp}
Show that $\Inj \spc{X}$ is $\delta$-hyperbolic if and only if $\spc{X}$ is.
\end{thm}


\section{Injective envelope}

An extension $\spc{E}$ of a metric space $\spc{X}$ will be called its \index{injective envelope}\emph{injective envelope} if $\spc{E}$ is an injective space, and there is no proper injective subspace of $\spc{E}$ that contains $\spc{X}$.

Two injective envelopes $e\:\spc{X}\hookrightarrow \spc{E}$ and $f\:\spc{X}\hookrightarrow \spc{F}$ are called  equivalent if there is an isometry $\iota\: \spc{E}\to\spc{F}$ such that $f=\iota\circ e$.

\begin{thm}{Theorem}\label{thm:inj-envelope}
For any metric space $\spc{X}$, its extension $\Inj\spc{X}$ is an injective envelope.

Moreover, any other injective envelope of $\spc{X}$ is equivalent to $\Inj\spc{X}$.
\end{thm}

\parit{Proof.} 
Suppose $S\subset \Inj\spc{X}$ is an injective subspace containing $\spc{X}$.
Since $S$ is injective, there is a short map $w\:\Inj\spc{X}\to S$ that fixes all points in $\spc{X}$.

Suppose that $w\:f\mapsto f'$; observe that $f(x)\ge f'(x)$ for any $x\in \spc{X}$.
Since $f$ is extremal, $f=f'$;
that is, $w$ is the identity map, and therefore $S=\Inj\spc{X}$.

Assume we have another injective envelope $e\:\spc{X}\hookrightarrow \spc{E}$.
Then there are short maps $v\:\spc{E}\to \Inj\spc{X}$ and $w\:\Inj\spc{X}\to \spc{E}$ such that $x=v\circ e(x)$ and $e(x)=w(x)$ for any $x\in\spc{X}$.
From above, the composition $v\circ w$ is the identity on $\Inj\spc{X}$.
In particular, $w$ is distance-preserving.

The composition $w\circ v\:\spc{E}\to \spc{E}$ is a short map that fixes points in $e(\spc{X})$.
Since $e\:\spc{X}\hookrightarrow \spc{E}$ is an injective envelope, the composition $w\circ v$ and therefore $w$ are onto.
Whence $w$ is an isometry.
\qeds

\begin{thm}{Exercise}\label{ex:d-p-inclusion}
Suppose $\spc{X}$ is a subspace of a metric space $\spc{U}$.
Show that the inclusion $\spc{X}\hookrightarrow\spc{U}$ can be extended to a distance-preserving inclusion $\Inj\spc{X}\hookrightarrow\Inj\spc{U}$.
\end{thm}


\section{Remarks}

Injective spaces were introduced by Nachman Aronszajn and Prom Panitchpakdi \cite{aronszajn-panitchpakdi}.
The injective envelope was introduced by John Isbell \cite{isbell}.
It was rediscovered a couple of times since then;
as a result, the injective envelope has many other names including \index{tight span}\emph{tight span} and \index{hyperconvex hull}\emph{hyperconvex hull}.

The following two exercise deals with ultrametric spaces which in some sense are dual to the injective spaces. 

Recall that if the following inequality
\[\dist{x}{z}{\spc{X}}
\le
\max\{\,\dist{x}{y}{\spc{X}},\dist{y}{z}{\spc{X}}\,\}\]
holds for any three points $x,y,z$ in a metric space $\spc{X}$,
then $\spc{X}$ is called an \index{ultrametric space}\emph{ultrametric space}.

\begin{thm}{Exercise}\label{ex:ultrametric}
Suppose that a metric space $\spc{X}$ satisfies the following property:
For any subspace $\spc{A}$ in $\spc{X}$ and any other metric space $\spc{Y}$, any short map $f\:\spc{A}\to \spc{Y}$ can be extended to a short map $F\:\spc{X}\to \spc{Y}$.

Show that $\spc{X}$ is an ultrametric space.
\end{thm}

A subspace $\spc{S}$ of a metric space $\spc{X}$ is called its \index{short retract}\emph{short retract} if there is a short map $\spc{X}\to \spc{S}$ that is the identity on $\spc{S}$.

\begin{thm}{Exercise}\label{ex:ultrametric-converse}
Show that any compact subspace $\spc{K}$ of an ultrametric space $\spc{X}$ is its short retract.

Construct an example of a complete ultrametric space $\spc{X}$ with a closed subset $Q$ that is not its short retract.
\end{thm}

The following exercise gives a sufficient condition for existence of a short extension.

\begin{thm}{Exercise}\label{ex:petrunin-stadler}
Let $\spc{X}$ and $\spc{Y}$ be metric spaces, $A\subset \spc{X}$, and $f\:A\z\to \spc{Y}$ be a short map.
Assume $\spc{Y}$ is compact and for any finite set $F\subset \spc{X}$ there is a short map $F\to \spc{Y}$ that agrees with $f$ on $F\cap A$.
Show that there is a short map $\spc{X}\to \spc{Y}$ that agrees with $f$ on $A$.
\end{thm}

\chapter{Space of sets}

\section{Hausdorff distance}

Let $\spc{X}$ be a metric space.
Given a subset $A\subset \spc{X}$,
consider the distance function to $A$
$$\distfun_A: \spc{X} \to [0,\infty)$$
defined as 
$$\distfun_A(x)
\df
\inf_{a\in A}\{\,\dist ax{\spc{X}}\,\}.$$

\begin{thm}{Definition}\label{def:hausdorff-convergence}
Let $A$ and $B$ be two compact subsets of a metric space $\spc{X}$.
Then the \index{Hausdorff distance}\emph{Hausdorff distance} between $A$ and $B$ is defined as 
$$|A-B|_{\Haus\spc{X}}
\df
\sup_{x\in \spc{X}}\{\,|\distfun_A(x)-\distfun_B(x)|\,\}.
$$

\end{thm}

The following observation gives a useful reformulation of the definition:

\begin{thm}{Observation}\label{obs:Haus-nbhds}
Suppose $A$ and $B$ be two compact subsets of a metric space $\spc{X}$.
Then $|A-B|_{\Haus\spc{X}}< R$ if and only if and only if 
$B$ lies in an $R$-neighborhood of $A$, 
and 
$A$ lies in an $R$-neighborhood of~$B$.
\end{thm}



Note that the set of all nonempty compact subsets of a metric space $\spc{X}$ equipped with the Hausdorff metric forms a metric space.
This new metric space will be denoted as $\Haus\spc{X}$.


\begin{thm}{Exercise}\label{ex:diam}
Let $\spc{X}$ be a metric space.
Given a subset $A\subset \spc{X}$ define its \index{diameter}\emph{diameter} as 
$$\diam A\df\sup_{a,b\in A} |a-b|.$$

Show that 
$$\diam\:\Haus\spc{X}\to \RR$$ 
is a \index{Lipschitz function}\emph{$2$-Lipschitz function};
that is,
\[|\diam A-\diam B|\le 2\cdot\dist{A}{B}{\Haus\spc{X}}\]
for any two compact nonempty sets $A,B\subset\spc{X}$.
\end{thm}


\begin{thm}{Exercise}\label{ex:Hausdorff-bry}
Let $A$ and $B$ be two compact subsets in the Euclidean plane $\RR^2$.
Assume $|A-B|_{\Haus\RR^2}<\eps$.

\begin{subthm}{ex:Hausdorff-bry:conv}
Show that $|\Conv A-\Conv B|_{\Haus\RR^2}<\eps$, where $\Conv A$ denoted the convex hull of $A$.
\end{subthm}
\begin{subthm}{ex:Hausdorff-bry:bry}
Is it true that
$|\partial A-\partial B|_{\Haus\RR^2}<\eps$,
where $\partial A$ denotes the boundary of $A$.

Does the converse hold? That is, assume $A$ and $B$ be two compact subsets in $\RR^2$
and $|\partial A-\partial B|_{\Haus\RR^2}<\eps$; 
is it true that $|A-B|_{\Haus\RR^2}\z<\eps$?
\end{subthm}

\end{thm}

Note that part \ref{SHORT.ex:Hausdorff-bry:conv} implies that $A\mapsto \Conv A$ defines a short map $\Haus\RR^2\to \Haus\RR^2$. 

\begin{thm}{Exercise}\label{ex:Haus-func}
Let $A$ and $B$ be two compact subsets in metric space~$\spc{X}$.
Show that 
\[\dist{A}{B}{\Haus\spc{X}}=\sup_f\, \{\,\max_{a\in A}\{f(a)\}-\max_{b\in B}\{f(b)\,\},\]
where the least upper bound is taken for all $1$-Lipschitz functions $f$.

\end{thm}

\begin{thm}{Advanced exercise}\label{ex:H-sections}
\begin{subthm}{ex:H-sections:S}
Construct a family of compact sets $C_t\subset\mathbb{S}^1$, $t\z\in [0,1]$ that is continuous in the Hausdorff topology, 
but does not admit a {}\emph{section}.
That is, there is no path $c\:[0,1]\to \mathbb{S}^1$ such that $c(t)\in C_t$ for all $t$.
\end{subthm}

\begin{subthm}{ex:H-sections:R}
Show that any family of compact sets $C_t\subset\RR^1$, $t\z\in [0,1]$ that is continuous in the Hausdorff topology, 
admits a {}\emph{section}.
That is, there is path $c\:[0,1]\to \RR^1$ such that $c(t)\in C_t$ for all $t$.
\end{subthm}

\end{thm}

\section{Hausdorff convergence}

\begin{thm}{Blaschke selection theorem}\label{thm:compact+Hausdorff}
A metric space $\spc{X}$ is compact if and only if
so is $\Haus\spc{X}$.
\end{thm}

The Hausdorff metric can be used to define convergence.
Namely, suppose $K_1,K_2,\dots$, and $K_\infty$ are compact sets in a metric space $\spc{X}$.
If $|K_\infty-K_n|_{\Haus\spc{X}}\to0$ as $n\to\infty$, then we say that 
the sequence $K_n$ {}\emph{converges} to $K_\infty$ \index{convergence in the sense of Hausdorff}\emph{in the sense of Hausdorff};
or we can say that $K_\infty$ is {}\emph{Hausdorff limit} of the sequence $K_n$.

Note that the theorem implies that from any sequence of compact sets in $\spc{X}$ one can select a subsequence that converges in the sense of Hausdorff; 
for that reason, it is called a \textit{selection} theorem. 

\parit{Proof; if part.}
Consider the map $\iota$ that sends each point $x\in \spc{X}$ to the one-point subset $\{x\}$ of $\spc{X}$.
Note that $\iota\:\spc{X}\to \Haus\spc{X}$ is distance-preserving.

Suppose that $A\subset \spc{X}$.
Note that $\diam A=0$ if and only if $A$ is a one-point set.
By \ref{ex:diam}, $\iota(\spc{X})$ is a closed subset of the compact space $\Haus\spc{X}$.
It follows that $\iota(\spc{X})$, and therefore $\spc{X}$, are compact.
\qeds

Since the map $\iota$ above is distance-preserving, we can and will consider $\spc{X}$ as a subspace of $\Haus\spc{X}$.

\begin{thm}{Exercise}\label{ex:haus-contractible}
Let $\spc{X}$ be a bounded length space with.
Suppose that there is a short retraction $\Haus\spc{X}\to \spc{X}$.
Show that $\spc{X}$ is contractible.
\end{thm}


To prove the only-if part we will need the following two lemmas.

\begin{thm}{Monotone convergence}\label{lem:decreasing-converges}
Let $K_1\supset K_2\supset\dots$ be a nested sequence of nonempty compact sets in a metric space $\spc{X}$.
Then $K_\infty\z=\bigcap_n K_n$ is the Hausdorff limit of $K_n$;
that is, $|K_\infty-K_n|_{\Haus\spc{X}}\to0$ as $n\to\infty$.
\end{thm}

\parit{Proof.}
By finite intersection property, $K_\infty$ is a nonempty compact set.

If the assertion were false, then there is $\eps>0$ such that for each $n$ 
one can choose $x_n\in K_n$
such that $\distfun_{K_\infty}(x_n)\ge\eps$.
Note that $x_n\in K_1$ for each $n$.
Since $K_1$ is compact, 
there is 
a \index{partial limit}\emph{partial limit}%
\footnote{Partial limit is a limit of a subsequence.}
 $x_\infty$ of $x_n$.
Clearly $\distfun_{K_\infty}(x_\infty)\ge \eps$.

On the other hand, since $K_n$ is closed and $x_m\in K_n$ for $m\ge n$,
we get $x_\infty\in K_n$ for each $n$.
It follows that $x_\infty\in K_\infty$ and therefore $\distfun_{K_\infty}(x_\infty)=0$ ---
a contradiction.\qeds


\begin{thm}{Lemma}\label{lem:complete+Hausdorff}
If $\spc{X}$ is a compact metric space, then $\Haus\spc{X}$
is complete.
\end{thm}

\parit{Proof.}
Let $(Q_n)$ be a Cauchy sequence in $\Haus\spc{X}$.
Passing to a subsequence of $Q_n$ we may assume that 
$$|Q_n-Q_{n+1}|_{\Haus\spc{X}}\le \tfrac1{10^n}\eqlbl{eq:eps=1/10}$$
for each $n$.

Denote by $K_n$ the closed $\tfrac1{10^n}$-neighborhood of $Q_n$;
that is,
\begin{align*}
K_n&= \set{x\in \spc{X}}{\distfun_{Q_n}(x)\le \tfrac1{10^n}}
\end{align*}
Since $\spc{X}$ is compact so is each $K_n$.

By \ref{obs:Haus-nbhds}, $|Q_n-K_n|_{\Haus\spc{X}}\le \tfrac1{10^n}$.
From \ref{eq:eps=1/10}, we get
$K_n\supset K_{n+1}$ 
for each $n$.
Set 
$$K_\infty=\bigcap_{n=1}^\infty K_n.$$
By the monotone convergence (\ref{lem:decreasing-converges}),
 $|K_n-K_\infty|_{\Haus\spc{X}}\to 0$ as $n\to\infty$.
Since $|Q_n-K_n|_{\Haus\spc{X}}\le \tfrac1{10^n}$, we get $|Q_n-K_\infty|_{\Haus\spc{X}}\to 0$ as $n\to\infty$ --- hence the lemma.
\qeds

\begin{thm}{Exercise}\label{ex:closure-union}
Let $\spc{X}$ be a complete metric space and $K_1,K_2,\dots$ be a sequence of compact sets 
that converges in the sense of Hausdorff.
Show that the union $K_1\cup K_2\cup\dots$ has compact closure.

Use this statement to show that in Lemma~\ref{lem:complete+Hausdorff} compactness of $\spc{X}$ can be exchanged to completeness.
\end{thm}

\parit{Proof of only-if part in \ref{thm:compact+Hausdorff}.}
According to Lemma~\ref{lem:complete+Hausdorff},
$\Haus\spc{X}$ is complete.
It remains to show that $\Haus\spc{X}$ is totally bounded (\ref{totally-bounded});
that is, given $\eps>0$ there is a finite $\eps$-net in $\Haus\spc{X}$.

Choose a finite $\eps$-net $A$ in $\spc{X}$.
Denote by $B$ the set of all subsets of $A$.
Note that  $B$ is a finite set in $\Haus\spc{X}$.
For each compact set $K\subset \spc{X}$, consider the subset $K'$ of all points $a\in A$
such that $\distfun_K(a)\le \eps$.
Observe that $K' \in B$ and $|K-K'|_{\Haus\spc{X}}\le\eps$.
In other words, $B$ is a finite $\eps$-net in $\Haus\spc{X}$.
\qeds

\begin{thm}{Exercise}\label{ex:Haus-length}
Let $\spc{X}$ be a complete metric space.
Show that $\spc{X}$ is a length space if and only if so is $\Haus\spc{X}$.
\end{thm}

\section{An application}

The following statement is called \index{isoperimetric inequality}\emph{isoperimetric inequality in the plane}.

\begin{thm}{Theorem}\label{thm:isoperimetric}
Among the plane figures bounded by closed curves of length at most $\ell$, the round disk has the maximal area.
\end{thm}

In this section, we will sketch a proof of the isoperimetric inequality that uses the Hausdorff convergence.
It is based on the following exercise.

\begin{thm}{Exercise}\label{ex:Huas-perimeter-area}
Let $\spc{C}$ be a subspace of $\Haus\RR^2$ formed by all compact convex subsets in $\RR^2$.
Show that perimeter\footnote{If the set degenerates to a line segment of length $\ell$, then its perimeter is defined as $2\cdot \ell$.} and area are continuous on~$\spc{C}$.
That is, if a sequence of convex compact plane sets $X_n$ converges to $X_\infty$ in the sense of Hausdorff, then 
\[\perim X_n\to \perim X_\infty\quad\text{and}\quad\area X_n\to\area X_\infty\]
as $n\to\infty$.
\end{thm}

\parit{Semiproof of \ref{thm:isoperimetric}.}
It is sufficient to consider only convex figures of the given perimeter; if a figure is not convex, pass to its convex hull and observe that it has a larger area and smaller perimeter.


Note that the selection theorem (\ref{thm:compact+Hausdorff}) together with the exercise imply the existence of figure $D$ with perimeter $\ell$ and maximal area.

It remains to show that $D$ is a round disk.
This is a problem in elementary geometry.

Let us cut $D$ along a chord $[ab]$ into two lenses, $L_1$ and $L_2$.
Denote by $L_1'$ the reflection of $L_1$ across the perpendicular bisector of $[ab]$.
Note that $D$ and $D'=L_1'\cup L_2$ have the same perimeter and area.
That is, $D'$ has perimeter $\ell$ and maximal possible area;
in particular, $D'$ is convex.

The following exercise will finish the proof.
\qeds

{

\begin{wrapfigure}{o}{57 mm}
\vskip-5mm
\centering
\includegraphics{mppics/pic-405}
\end{wrapfigure}

\begin{thm}{Exercise}\label{ex:round-disc}
Suppose $D$ is a convex figure such that for any chord $[ab]$ of $D$ the above construction produces a convex figure $D'$.
Show that $D$ is a round disk.
\end{thm}


}

Another popular way to prove that $D$ is a round disk is given by the so-called {}\emph{Steiner's 4-joint method} \cite{blaschke}.

\section{Remarks}\label{sec:H-variation}

It seems that Hausdorff convergence was first introduced by Felix Hausdorff~\cite{hausdorff}.
A couple of years later an equivalent definition was given by Wilhelm Blaschke~\cite{blaschke}.

The following refinement was introduced by  Zdeněk Frolík \cite{frolik},
later it was rediscovered by Robert Wijsman~\cite{wijsman}.  
This refinement is also called \index{Hausdorff convergence}\emph{Hausdorff convergence};
in fact, it takes an intermediate place between the original Hausdorff convergence and {}\emph{closed convergence}, also introduced by Hausdorff in \cite{hausdorff}.

\begin{thm}{Definition}\label{def:gen-Haus-conv}
Let $A_1,A_2,\dots$ be a sequence of closed sets in a metric space $\spc{X}$.
We say that the sequence $A_n$ converges to a closed set $A_\infty$ in the sense of Hausdorff if for any $x\in\spc{X}$, we have
$\distfun_{A_n}(x)\z\to \distfun_{A_\infty}(x)$ as $n\to\infty$.
\end{thm}

For example, suppose $\spc{X}$ is the Euclidean plane and $A_n$ is the circle with radius $n$ and center at the point $(n,0)$.
If we use the standard definition (\ref{def:hausdorff-convergence}), then the sequence $(A_n)$ diverges, but it converges to the $y$-axis in the sense of Definition~\ref{def:gen-Haus-conv}.

Further, consider the sequence of one-point sets $B_n=\{(n,0)\}$ in the Euclidean plane.
It diverges in the sense of the standard definition, but, in the sense of \ref{def:gen-Haus-conv}, it converges to the empty set;
indeed, for any point $x$ we have $\distfun_{B_n}(x)\to\infty$ as $n\to \infty$ and $\distfun_{\emptyset}(x)= \infty$ for any~$x$.

The following exercise is analogous to the Blaschke selection theorem (\ref{thm:compact+Hausdorff}) for the modified Hausdorff convergence.

\begin{thm}{Exercise}\label{ex:generalized-selection}
Let $\spc{X}$ be a proper metric space
and $A_1,A_2,\dots$ be a sequence of closed sets in~$\spc{X}$.
Show that the sequence  $A_1,A_2,\dots$ has a convergent subsequence in the sense of Definition~\ref{def:gen-Haus-conv}.
\end{thm}

\chapter{Space of spaces}

\section{Gromov--Hausdorff metric}

The goal of this section is to cook up a metric space out of all compact metric spaces.
More precisely, we want to define the so-called  Gromov--Hausdorff metric on the set of \textit{isometry classes} of compact metric spaces.
(Being isometric is an equivalence relation, 
and an \index{isometry class}\emph{isometry class} is an equivalence class with respect to this relation.)

The obtained metric space will be denoted by $\GH$.
Given two metric spaces $\spc{X}$ and $\spc{Y}$,
denote by $[\spc{X}]$ and $[\spc{Y}]$ their isometry classes;
that is, $\spc{X}'\in [\spc{X}]$ if and only if $\spc{X}'\iso \spc{X}$.
Pedantically, the Gromov--Hausdorff distance from $[\spc{X}]$ 
to $[\spc{Y}]$ should be denoted as $|[\spc{X}]-[\spc{Y}]|_{\GH}$;
but we will write it as $|\spc{X}\z-\spc{Y}|_{\GH}$ and say (not quite correctly) that 
\textit{$|\spc{X}\z-\spc{Y}|_{\GH}$ is the Gromov--Hausdorff distance from  $\spc{X}$ 
to  $\spc{Y}$}.
In other words, from now on the term \textit{metric space} might also stand for its \textit{isometry class}.

The metric on $\GH$ is defined as the maximal metric such that \textit{the distance between subspaces in a metric space is not greater than the Hausdorff distance between them}.
Here is a formal definition:

\begin{thm}{Definition}\label{def:GH}
The \index{Gromov--Hausdorff distance}\emph{Gromov--Hausdorff distance} $|\spc{X}-\spc{Y}|_{\GH}$ between compact metric spaces $\spc{X}$ and $\spc{Y}$
is defined by the following
relation.
 
Given  $r > 0$, we have that $|\spc{X}-\spc{Y}|_{\GH} < r$ if and only if there exists a metric
space $\spc{W}$ and subspaces $\spc{X}'$ and $\spc{Y}'$ in $\spc{W}$ that are isometric to $\spc{X}$ and $\spc{Y}$
respectively such that $|\spc{X}'-\spc{Y}'|_{\Haus\spc{W}} < r$. 
(Here $|\spc{X}'-\spc{Y}'|_{\Haus\spc{W}}$ denotes the Hausdorff distance between sets $\spc{X}'$ and $\spc{Y}'$ in $\spc{W}$.)
\end{thm}

\begin{thm}{Theorem}\label{thm:GH-is-a-metric}
The set of isometry classes of compact metric spaces equipped with Gromov--Hausdorff metric forms a metric space (which is denoted by $\GH$).

In other words, for arbitrary compact metric spaces $\spc{X}$, $\spc{Y}$ and $\spc{Z}$ the following conditions hold:

\begin{subthm}{GH-1} $|\spc{X}-\spc{Y}|_{\GH}\ge 0$;
\end{subthm}

\begin{subthm}{GH-2} $|\spc{X}-\spc{Y}|_{\GH}=0$ if and only if $\spc{X}$ is isometric to $\spc{Y}$;
\end{subthm}

\begin{subthm}{GH-3} $|\spc{X}-\spc{Y}|_{\GH}=|\spc{Y}-\spc{X}|_{\GH}$;
\end{subthm}

\begin{subthm}{GH-4} $|\spc{X}-\spc{Y}|_{\GH}+|\spc{Y}-\spc{Z}|_{\GH}\ge |\spc{X}-\spc{Z}|_{\GH}$.
\end{subthm}
\end{thm}


Note that \ref{SHORT.GH-1}, \ref{SHORT.GH-3},
and the ``if''-part of \ref{SHORT.GH-2} follow directly from Definition \ref{def:GH}.
Part \ref{SHORT.GH-4} will be proved in Section~\ref{sec:GH-approx}.
The ``only-if''-part of \ref{SHORT.GH-2} will be proved in Section~\ref{sec:extfun=GH}.

Recall that $a\cdot\spc{X}$ denotes $\spc{X}$ \index{rescaled space}\emph{rescaled} by factor $a>0$;
that is, $a\cdot\spc{X}$ is a metric space with the underlying set of $\spc{X}$ and the metric defined by
\[\dist{x}{y}{a\cdot\spc{X}}\df a\cdot\dist{x}{y}{\spc{X}}.\]

\begin{thm}{Exercise}\label{ex:d_GH-and-diam}
Let $\spc{X}$ be a compact metric space,
$\spc{O}$ be the one-point metric space.

Prove that 

\begin{subthm}{ex:d_GH-and-diam:point}
$|\spc{X}-\spc{O}|_{\GH}=\tfrac12\cdot \diam \spc{X}.$

\end{subthm}

\begin{subthm}{ex:d_GH-and-diam:scale}
$|a\cdot\spc{X}-b\cdot \spc{X}|_{\GH}=\tfrac12\cdot|a-b|\cdot\diam\spc{X}.$
\end{subthm}

\begin{subthm}{ex:d_GH-and-diam:isometry}
$\iota[\spc{O}]=[\spc{O}]$ for any isometry $\iota\:\GH\to\GH$.
\end{subthm}


\end{thm}




\begin{thm}{Exercise}\label{ex:GH<H}
Find subsets $A,B\subset\RR^2$ such that 
\[|A-B|_{\GH}>|A-\iota(B)|_{\Haus\RR^2}\]
for any isometry $\iota$ of $\RR^2$.
\end{thm}


\begin{thm}{Exercise}\label{ex:rectangle}
Let $\spc{A}_r$ be a rectangle $1$ by $r$ in the Euclidean plane 
and $\spc{B}_r$ be a closed line interval of length $r$.
Show that 
\[|\spc{A}_r-\spc{B}_r|_{\GH}>\tfrac1{10}\]
for all large $r$.
\end{thm}

\begin{thm}{Advanced exercise}\label{ex:GH-inj}
Let $\spc{X}$ and $\spc{Y}$ be compact metric spaces;
denote by $\hat{\spc{X}}$ and $\hat{\spc{Y}}$ their injective envelopes (see \ref{sec:extremal-functions}).
Show that 
\[|\hat{\spc{X}}-\hat{\spc{Y}}|_{\GH}\le 2\cdot|\spc{X}- \spc{Y}|_{\GH}.\] 
In other words $\spc{X}\mapsto \hat{\spc{X}}$ defines a $2$-Lipschitz map $\GH\to\GH$.

\end{thm}



\section{Approximations and almost isometries}\label{sec:GH-approx}

\begin{thm}{Definition}\label{ex:defGHR}
Let $\spc{X}$ and $\spc{Y}$ be two metric spaces.
A relation $\approx$ between points in $\spc{X}$ and $\spc{Y}$ is called \index{$\eps$-approximation}\emph{$\eps$-approximation} if the following conditions hold:
\begin{itemize}
\item For any $x\in  \spc{X}$ there is $y\in \spc{Y}$ such that $x\approx y$.
\item For any $y\in  \spc{Y}$ there is $x\in \spc{X}$ such that $x\approx y$.
\item If $x\approx y$ and $x'\approx y'$ for some $x, x'\in  \spc{X}$ and $y,y'\in \spc{Y}$, then 
\[\bigl|\dist{x}{x'}{\spc{X}}-\dist{y}{y'}{\spc{Y}}\bigr|<2\cdot\eps.\]
\end{itemize}

\end{thm}

\begin{thm}{Exercise}\label{ex:H-R}
Let $\spc{X}$ and $\spc{Y}$ be two compact metric spaces.
Show that
\[\dist{\spc{X}}{\spc{Y}}{\GH}<\eps\]
if and only if there is an $\eps$-approximation between $\spc{X}$ and $\spc{Y}$.

In other words $\dist{\spc{X}}{\spc{Y}}{\GH}$ is the greatest lower bound of values $\eps>0$ such that  there is an $\eps$-approximation between $\spc{X}$ and $\spc{Y}$.
\end{thm}

\parit{Proof of \ref{GH-4}.}
Suppose that 
\begin{itemize}
\item $\approx_1$ is a relation between points in $\spc{X}$ and $\spc{Y}$,
\item $\approx_2$ is a relation between points in $\spc{Y}$ and $\spc{Z}$.
\end{itemize}
Consider the relation $\approx_3$ between points in $\spc{X}$ and $\spc{Z}$ such that
$x\approx_3 z$ if and only if there is $y\in  \spc{Y}$ such that 
$x\approx_1 y$ and $y\approx_2 z$.

It is straightforward to check that if $\approx_1$ is an $\eps_1$-approximation and $\approx_2$ is an $\eps_2$-approximation, then $\approx_3$ is an $(\eps_1+\eps_2)$-approximation.

Applying \ref{ex:H-R}, we get that if 
\[|\spc{X}-\spc{Y}|_{\GH}<\eps_1
\quad\text{and}\quad
|\spc{Y}-\spc{Z}|_{\GH}<\eps_2,
\]
then 
\[|\spc{X}-\spc{Z}|_{\GH}<\eps_1+\eps_2.\]
Hence \ref{GH-4} follows.
\qeds

The following weakened version of isometry is closely related to $\eps$-approximations.

\begin{thm}{Definition} Let $\spc{X}$ and $\spc{Y}$ be metric spaces and $\eps>0$. 
A  map\footnote{possibly noncontinuous} $f\: \spc{X} \z\to \spc{Y}$ is called an \index{almost isometry}\emph{$\eps$-isometry} 
if $f(\spc{X})$ is an $\eps$-net in $\spc{Y}$ and
\[\bigl|\dist{x}{x'}{\spc{X}}-\dist{f(x)}{f(x')}{\spc{Y}}\bigr|<\eps.\]
for any $x,x'\in \spc{X}$.
\end{thm}

\begin{thm}{Exercise}\label{ex:eps-isom}
Let $\spc{X}$ and $\spc{Y}$ be compact metric spaces.

\begin{subthm}{ex:eps-isom:GH>isom}
If $\dist{\spc{X}}{\spc{Y}}{\GH}<\eps$, then there is a $2\cdot\eps$-isometry $f\:\spc{X}\to\spc{Y}$.
\end{subthm}

\begin{subthm}{ex:eps-isom:isom>GH}
If there is an $\eps$-isometry $f\:\spc{X}\to\spc{Y}$, then $\dist{\spc{X}}{\spc{Y}}{\GH}<\eps$.
\end{subthm}

\end{thm}

\section{Optimal realization}\label{sec:extfun=GH}

Note that
\[\dist{\spc{X}'}{\spc{Y}'}{\Haus\spc{W}}\ge \dist{\spc{X}}{\spc{Y}}{\GH},\]
where $\spc{X}$, $\spc{Y}$, $\spc{X}'$, $\spc{Y}'$, and $\spc{W}$ are as in \ref{def:GH}.
The following proposition states that equality holds for some choice of $\spc{X}'$, $\spc{Y}'$, and $\spc{W}$.

\begin{thm}{Proposition}\label{prop:GH=H}
For any two compact metric spaces $\spc{X}$ and $\spc{Y}$ there is a metric space $\spc{W}$
with subsets $\spc{X}'$ and $\spc{Y}'$ such that 
$\spc{X}'\iso\spc{X}$, $\spc{Y}'\iso\spc{Y}$, and 
\[\dist{\spc{X}'}{\spc{Y}'}{\Haus\spc{W}}=\dist{\spc{X}}{\spc{Y}}{\GH}.\]
\end{thm}

Let us introduce the so-called \textit{appropriate functions} and use them in a reinterpretation of Gromov--Hausdorff distance.

Suppose $\spc{X}$, $\spc{Y}$, $\spc{X}'$, $\spc{Y}'$, and $\spc{W}$ are as in \ref{def:GH}.
By passing to the subspace $\spc{X}'\cup\spc{Y}'$ in $\spc{W}$, we can assume that $\spc{W}=\spc{X}'\cup\spc{Y}'$.
Note that in this case the metric on $\spc{W}$ is completely determined by the function 
\[f(x,y)=\dist{x}{y}{\spc{W}};\]
a function $f\:\spc{X}\times \spc{Y}\to\RR$ that can appear this way will be called \index{appropriate function}\emph{appropriate}.

Note that a function $f\:\spc{X}\times\spc{Y}\to\RR$ is appropriate if and only if
$x\mapsto f(x,y)$ and $y\mapsto f(x,y)$ are extension functions;
that is, if
\[
\begin{aligned}
f(x,y)+f(x,y')
&\ge \dist{y}{y'}{\spc{Y}}\ge |f(x,y)-f(x,y')|,
\\
f(x,y)+f(x',y)
&\ge \dist{x}{x'}{\spc{X}}\ge |f(x,y)-f(x',y)|;
\end{aligned}
\eqlbl{eq:appropriate}
\]
for any $x,x',\in\spc{X}$ and  $y,y'\in\spc{X}$;
see \ref{sec:Extension property}.
In other words, the following defines a pseudometric on $\spc{X}\sqcup\spc{Y}$
\[\dist{x}{y}{\spc{X}\sqcup\spc{Y}}=
\begin{cases}
\dist{x}{y}{\spc{X}}&\text{if\ } x,y\in \spc{X},
\\
\dist{x}{y}{\spc{Y}}&\text{if\ } x,y\in \spc{Y},
\\
f(x,y)&\text{if\ } x\in \spc{X}\ \text{and}\ y\in \spc{Y},
\end{cases}
\]
and the corresponding metric space $\spc{W}$ contains isometric copies of $\spc{X}$ and $\spc{Y}$.

Given an appropriate function $f\:\spc{X}\times\spc{Y}\to\RR$, set 
\begin{align*}
a_f&=\max_{x\in \spc{X}}\{\min_{y\in\spc{Y}} \{f(x,y)\}\},
\\
b_f&=\max_{y\in \spc{Y}}\{\min_{x\in\spc{X}} \{f(x,y)\}\}.
\end{align*}

\begin{thm}{Observation}\label{obs:GH=min-appropriate}
If $\spc{X}$, $\spc{Y}$, $\spc{X}'$, $\spc{Y}'$, and $\spc{W}$ as above then
\[\dist{\spc{X}'}{\spc{Y}'}{\Haus\spc{W}}=\inf_f\{a_f,b_f\}.\]

\end{thm}

\parit{Proof of \ref{prop:GH=H}.}
By \ref{eq:appropriate}, any appropriate functions $f\:\spc{X}\times\spc{Y}\to\RR$ is $2$-Lipschitz.
Observe that the functional $f\mapsto \min\{a_f,b_f\}$ is continuous.
Applying the Arzelà--Ascoli theorem, we can get an  appropriate function $f\:\spc{X}\times\spc{Y}\to\RR$ 
with minimal possible value $\min\{a_f,b_f\}$.
It remains to apply \ref{obs:GH=min-appropriate}.
\qeds

\begin{thm}{Exercise}\label{ex:XYZ}
Construct three compact metric spaces $\spc{X}$, $\spc{Y}$, and $\spc{Z}$
such that for any metric space $\spc{W}$
with subsets $\spc{X}'$, $\spc{Y}'$, and $\spc{Z}'$ such that 
$\spc{X}'\iso\spc{X}$, $\spc{Y}'\iso\spc{Y}$, and $\spc{Z}'\iso\spc{Z}$
at least one of the following three inequalities is strict
\begin{align*}
\dist{\spc{X}'}{\spc{Y}'}{\Haus\spc{W}}&\ge \dist{\spc{X}}{\spc{Y}}{\GH},
\\
\dist{\spc{Y}'}{\spc{Z}'}{\Haus\spc{W}}&\ge\dist{\spc{Y}}{\spc{Z}}{\GH},
\\
\dist{\spc{Z}'}{\spc{X}'}{\Haus\spc{W}}&\ge\dist{\spc{Z}}{\spc{X}}{\GH}.
\end{align*}
\end{thm}

\section{Convergence}

The Gromov--Hausdorff metric is used to define \index{Gromov--Hausdorff convergence}\emph{Gromov--Hausdorff convergence}.
Namely, a sequence of compact metric spaces $\spc{X}_n$ converges to compact metric spaces $\spc{X}_\infty$ in the sense of Gromov--Hausdorff if 
\[\dist{\spc{X}_n}{\spc{X}_\infty}{\GH}\to 0\quad\text{as}\quad n\to\infty.\]

This convergence is more important than the metric ---
in all applications, we use only the topology on $\GH$
and we do not care about the particular value of Gromov--Hausdorff distance between spaces.
The following observation follows from \ref{ex:eps-isom}:

\begin{thm}{Observation}\label{obs:GH-e-isom}
A sequence of compact metric spaces $(\spc{X}_n)$ converges to  $\spc{X}_\infty$ in the sense of Gromov--Hausdorff if and only if there is a sequence $\eps_n\to0+$
and an $\eps_n$-isometry $f_n\:\spc{X}_n\to \spc{X}_\infty$ for each $n$.
\end{thm}

In the following exercises, \textit{convergence} is understood in the sense of Gromov--Hausdorff.

\begin{thm}{Exercise}\label{ex:GH-SC}
\begin{subthm}{ex:GH-SC:circle}
Show that a sequence of compact simply-connected length spaces cannot converge to a circle.
\end{subthm}

\begin{subthm}{ex:GH-SC:nonsc-limit}
Construct a sequence of compact simply-connected length spaces that converges to a compact non-simply-connected space.
\end{subthm}
\end{thm}

\begin{thm}{Exercise}\label{ex:sphere-to-ball}
\begin{subthm}{ex:sphere-to-ball:2}
Show that a sequence of length metrics on the 2-sphere cannot converge to the unit disk.
\end{subthm}

\begin{subthm}{ex:sphere-to-ball:3}
Construct a sequence of length metrics on the 3-sphere that converges to a unit 3-ball.
\end{subthm}

\end{thm}

\section{Uniformly totally bonded families}

\begin{thm}{Definition}\label{def:utb}
A family $\spc{Q}$ of (isometry classes) of compact metric spaces is called  \index{uniformly totally bonded family}\emph{uniformly totally bonded} if it meets the following two conditions:

\begin{subthm}{}
spaces in $\spc{Q}$ have uniformly bounded diameters; that is, there is $D\in\RR$ such that
\[\diam\spc{X}\le D\]
for any space $\spc{X}$ in $\spc{Q}$.
\end{subthm}

\begin{subthm}{}
For any $\eps>0$ there is $n\in\NN$ such that any space $\spc{X}$ in $\spc{Q}$ admits an $\eps$-net with at most $n$ points.
\end{subthm}
\end{thm}

\begin{thm}{Exercise}\label{ex:utb+pack}
Let $\spc{Q}$ be a family of compact spaces with uniformly bounded diameters.
Show that $\spc{Q}$ is uniformly totally bonded if for any $\eps>0$ there is $n\in\NN$ such that 
\[\pack_\eps\spc{X}\le n\]
for any space $\spc{X}$ in $\spc{Q}$.
\end{thm}


Fix a real constant $C$.
A Borel measure $\mu$ on a metric space $\spc{X}$ is called \index{doubling space}\emph{$C$-doubling} if
\[\mu[\oBall(p,2\cdot r)]< C\cdot\mu[\oBall(p,r)]\]
for any point $p\in \spc{X}$ and any $r>0$.
A Borel measure is called \index{doubling measure}\emph{doubling} if it is {}\emph{$C$-doubling} for some real constant $C$.

\begin{thm}{Exercise}\label{pr:doubling}
Let $\spc{Q}(C,D)$ be the set of all the compact metric spaces with diameter at most $D$ that admit a $C$-doubling measure.
Show that $\spc{Q}(C,D)$ is totally bounded.
\end{thm}

Given two metric spaces $\spc{X}$ and $\spc{Y}$, we will write $\spc{X}\le \spc{Y}$ if there is a distance-noncontracting map $f\:\spc{X}\to \spc{Y}$;
that is, if 
$$ |x-x'|_{\spc{X}}\le|f(x)-f(x')|_{\spc{Y}}$$
for any $x,x'\in \spc{X}$.

\begin{thm}{Exercise}\label{pr:under}

\begin{subthm}{pr:under:if}
Let $\spc{Y}$ be a compact metric space.
Show that the set of all spaces $\spc{X}$ such that $\spc{X}\le\spc{Y}$
is uniformly totally bounded.
\end{subthm}

\begin{subthm}{pr:under:only-if}
Show that for any uniformly totally bounded set $\spc{Q}\subset\GH$ there is a compact space $\spc{Y}$
such that $\spc{X}\le\spc{Y}$ for any $\spc{X}$ in $\spc{Q}$.
\end{subthm}

\end{thm}

\section{Gromov selection theorem}

The following theorem is analogous to Blaschke selection theorems (\ref{thm:compact+Hausdorff}).

\begin{thm}{Gromov selection theorem}\label{thm:gromov-compactness}
Let $\spc{Q}$ be a closed subset of $\GH$.
Then $\spc{Q}$ is compact if and only if the spaces in $Q$ are uniformly totally bounded.
\end{thm}

\begin{thm}{Lemma}\label{lem:GH-complete}
The space $\GH$ is complete.
\end{thm}


Let us define gluing of metric spaces that will be used in the proof of the lemma.

Suppose 
$\spc{U}$ and $\spc{V}$ are metric spaces 
with isometric closed sets $A\subset\spc{U}$ and $A'\subset\spc{V}$;
let $\iota\:A\to A'$ be an isometry.
Consider the space $\spc{W}$ of all equivalence classes in $\spc{U}\sqcup\spc{V}$ with the equivalence relation given by $a\sim\iota(a)$ for any $a\in A$.

It is straightforward to check that the following defines a metric on~$\spc{W}$:
\begin{align*}
\dist{u}{u'}{\spc{W}}&\df\dist{u}{u'}{\spc{U}}
\\
\dist{v}{v'}{\spc{W}}&\df\dist{v}{v'}{\spc{V}}
\\
\dist{u}{v}{\spc{W}}&\df\min\set{\dist{u}{a}{\spc{U}}+\dist{v}{\iota(a)}{\spc{V}}}{a\in A}
\end{align*}
where $u,u'\in \spc{U}$ and $v,v'\in \spc{V}$.

The  space $\spc{W}$ is called the \index{gluing}\emph{gluing} of $\spc{U}$ and  $\spc{V}$ along~$\iota$; briefly, we can write
$\spc{W}=\spc{U}\sqcup_\iota\spc{V}$.
If one applies this construction to two copies of one space $\spc{U}$ with a set $A\subset \spc{U}$ and the identity map $\iota\:A\to A$, then the obtained space is called the \index{doubling}\emph{doubling} of $\spc{U}$ along~$A$; this space can be denoted by $\sqcup_A^2\spc{U}$.

Note that the inclusions $\spc{U}\hookrightarrow \spc{W}$ and $\spc{V}\hookrightarrow \spc{W}$ are distance preserving.
Therefore we can and will consider $\spc{U}$ and $\spc{V}$ as the subspaces of $\spc{W}$;
this way the subsets $A$ and $A'$ will be identified and denoted further by~$A$.
Note that $A=\spc{U}\cap \spc{V}\subset \spc{W}$.

\parit{Proof.}
Let $\spc{X}_1,\spc{X}_2,\dots$ be a Cauchy sequence in $\GH$.
Passing to a subsequence if necessary, 
we can assume that $|\spc{X}_n-\spc{X}_{n+1}|_{\GH}<\tfrac1{2^n}$ for each~$n$.
In particular, for each $n$ there is a metric space $\spc{V}_n$ with distance preserving inclusions $\spc{X}_n\hookrightarrow \spc{V}_n$ and $\spc{X}_{n+1}\hookrightarrow \spc{V}_n$ such that
\[|\spc{X}_n-\spc{X}_{n+1}|_{\Haus\spc{V}_n}<\tfrac1{2^n}\]
for each $n$.
Moreover, we may assume that $\spc{V}_n=\spc{X}_n\cup\spc{X}_{n+1}$.

Let us glue $\spc{V}_1$ to $\spc{V}_2$ along $\spc{X}_2$;
to the obtained space glue $\spc{V}_3$ along $\spc{X}_3$, and so on.
The obtained metric space $\spc{W}$
has an underlying set formed by the disjoint union of all $\spc{X}_n$ such that each inclusion $\spc{X}_n\z\hookrightarrow\spc{W}$ is distance preserving and
\[|\spc{X}_n-\spc{X}_{n+1}|_{\Haus\spc{W}}<\tfrac1{2^n}\]
for each $n$.
In particular,
\[|\spc{X}_m-\spc{X}_n|_{\Haus\spc{W}}<\tfrac1{2^{n-1}}\eqlbl{eq:|x_m-X_n|}\] 
if $m>n$.

Denote by $\bar{\spc{W}}$ the completion of $\spc{W}$.
Observe that the union $\spc{X}_1\z\cup \spc{X}_2\cup\z\dots\cup \spc{X}_n$ is compact and \ref{eq:|x_m-X_n|} implies that it forms a $\tfrac1{2^{n-1}}$-net in $\bar{\spc{W}}$.
Whence $\bar{\spc{W}}$ is compact; see \ref{totally-bounded} and \ref{ex:compact-net}.

Applying the Blaschke selection theorem (\ref{thm:compact+Hausdorff}),
we can pass to a subsequence of $\spc{X}_n$ that converges in $\Haus\bar{\spc{W}}$; denote its limit by $\spc{X}_\infty$.
It remains to observe that $\spc{X}_\infty$ is the Gromov--Hausdorff limit of $\spc{X}_n$.
\qeds

\parit{Proof of \ref{thm:gromov-compactness}; only-if part.}
Suppose that there is no sequence $\eps_n\to0$ as described in \ref{def:utb}.
Observe that in this case
there is a sequence of spaces $\spc{X}_n\in\spc{Q}$ such that 
$$\pack_\delta \spc{X}_n\to\infty
\quad\text{as}\quad
n\to\infty$$
for some fixed $\delta>0$.

Since $\spc{Q}$ is compact, 
this sequence has a partial limit, say $\spc{X}_\infty\in\spc{Q}$.
Observe that $\pack_{\delta} \spc{X}_\infty=\infty$.
Therefore, $\spc{X}_\infty$ is not compact --- a contradiction.

\parit{If part.}
Given a positive integer $n$ consider the set of all metric spaces $\spc{W}_n$
with the number of points at most $n$ and diameter $\le D$.
Note that $\spc{W}_n$ is a compact set in $\GH$ for each $n$.

Let $D$ and $n=n(\eps)$ be as in the definition of uniformly totally bonded families (\ref{def:utb}).

Note that an $\eps$-net of any $\spc{X}\in\spc{Q}$ belongs to $\spc{W}_{n(\eps)}$.
Therefore, $\spc{W}_{n(\eps)}$ is a compact $\eps$-net of $\spc{Q}$ for any $\eps>0$.
Since $\spc{Q}$ is closed in a complete space $\GH$, it implies that $\spc{Q}$ is compact.
\qeds

\begin{thm}{Exercise}\label{ex-GH-length}
Show that the space $\GH$ is 

\begin{subthm}{ex-GH-length:length}
length,
\end{subthm}

\begin{subthm}{ex-GH-length:geodesic}
geodesic.
\end{subthm}

\end{thm}

\begin{thm}{Exercise}\label{ex:GH-po}
For two metric spaces $\spc{X}$ and $\spc{Y}$,
we write $\spc{X}\le \spc{Y}+\eps$ if
there is a map $f\:\spc{X}\to \spc{Y}$ such that 
\[\dist{x}{x'}{\spc{X}}\le \dist{f(x)}{f(x')}{\spc{Y}}+\eps\]
for any $x,x'\in \spc{X}$.

\begin{subthm}{ex:GH-po:a}
Show that 
$$\dist{\spc{X}}{\spc{Y}}{\GH'}=\inf\set{\eps>0}{\spc{X}\le \spc{Y}+\eps
\quad\text{and}\quad
\spc{Y}\le \spc{X}+\eps}$$
defines a metric on the space of (isometry classes) of compact metric spaces.
\end{subthm}

\begin{subthm}{ex:GH-po:b}
Moreover $\dist{*}{*}{\GH'}$ is equivalent to the Gromov--Hausdorff metric;
that is,
$$|\spc{X}_n-\spc{X}_\infty|_{\GH}\to 0 
\quad\iff\quad 
\dist{\spc{X}_n}{\spc{X}_\infty}{\GH'}\to 0$$ 
as $n\to\infty$.
\end{subthm}
\end{thm}

\section{Universal ambient space}

Recall that a metric space is called universal if it contains an isometric copy of any separable metric space (in particular, any compact metric space).
Examples of universal spaces include $\spc{U}_\infty$ --- the Urysohn space and $\ell^\infty$ --- the space of bounded infinite sequences with the metric defined by $\sup$-norm; see \ref{prop:sep-in-urys} and \ref{ex:frechet}.

The following proposition says that the space $\spc{W}$ in Definition~\ref{def:GH} can be exchanged to a fixed universal space.

\begin{thm}{Proposition}\label{prop:GH-with-fixed-Z}
Let $\spc{U}$ be a universal space.
Then for any compact metric spaces $\spc{X}$ and $\spc{Y}$ we have
$$|\spc{X}-\spc{Y}|_{\GH} = \inf \{|\spc{X}'-\spc{Y}'|_{\Haus\spc{U}}\}$$ 
where the greatest lower bound is taken over all pairs of sets $\spc{X}'$ and $\spc{Y}'$ in $\spc{U}$
which isometric to  $\spc{X}$ and $\spc{Y}$ respectively.  
\end{thm}




\parit{Proof of \ref{prop:GH-with-fixed-Z}.}
By the definition (\ref{def:GH}), we have that 
\[|\spc{X}-\spc{Y}|_{\GH} \le \inf \{|\spc{X}'-\spc{Y}'|_{\Haus\spc{U}}\};\]
it remains to prove the opposite inequality.

Suppse $|\spc{X}-\spc{Y}|_{\GH}<\eps$;
let $\spc{X}'$, $\spc{Y}'$ and $\spc{W}$ be as in \ref{def:GH}.
We can assume that $\spc{W}=\spc{X}'\cup\spc{Y}'$;
otherwise pass to the subspace $\spc{X}'\cup\spc{Y}'$ of~$\spc{W}$.
In this case, $\spc{W}$ is compact;
in particular, it is separable.

Since $\spc{U}$ is universal, there is a distance-preserving embedding of $\spc{W}$ in $\spc{U}$;
let us keep the same notation for $\spc{X}'$, $\spc{Y}'$, and their images.
It follows that 
\[|\spc{X}'-\spc{Y}'|_{\Haus\spc{U}}<\eps,\]
--- hence the result.
\qeds

\begin{thm}{Exercise}\label{ex:GH-urysohn}
Let $\spc{U}_\infty$ be the Urysohn space.
Given two compact sets $A$ and $B$ in $\spc{U}_\infty$ define 
\[\|A-B\|=\inf\{|A-\iota(B)|_{\Haus\spc{U}_\infty}\},\]
where the greatest lower bound is taken for all isometrics $\iota$ of $\spc{U}_\infty$.
Show that $\|{*}\z-{*}\|$ defines a pseudometric%
\footnote{The value $\|A-B\|$ is called Hausdorff distance \index{Hausdorff distance up to isometry}\emph{up to isometry} from $A$ to $B$ in $\spc{U}_\infty$.}
on nonempty compact subsets of $\spc{U}_\infty$ and its corresponding metric space is isometric to $\GH$.
\end{thm}

\section{Remarks}

Suppose $\spc{X}_n\GHto \spc{X}_\infty$, then there is a metric on the disjoint union 
\[\bm{X}=\bigsqcup_{n\in \NN\cup\{\infty\}} \spc{X}_n\] 
that satisfies the following property:

\begin{thm}{Property}\label{propery:GH}
The restriction of metric on each $\spc{X}_n$ and $\spc{X}_\infty$ coincides with its original metric, 
and $\spc{X}_n\Hto \spc{X}_\infty$ as subsets in $\bm{X}$.
\end{thm}

Indeed, since $\spc{X}_n\GHto \spc{X}_\infty$, there is a metric on $\spc{V}_n=\spc{X}_n\sqcup \spc{X}_\infty$ such that the restriction of metric on each $\spc{X}_n$ and $\spc{X}_\infty$ coincides with its original metric, and $\dist{\spc{X}_n}{\spc{X}_\infty}{\Haus\spc{V}_n}<\eps_n$ for some sequence $\eps_n\to 0$.
Gluing all $\spc{V}_n$ along $\spc{X}_\infty$, we obtain the required space $\bm{X}$.

In other words, the metric on $\bm{X}$ \textit{defines} the convergence $\spc{X}_n\z\GHto \spc{X}_\infty$.
This metric makes it possible to talk about limits of sequences $x_n\in \spc{X}_n$ as $n\to\infty$, as well as weak limits of a sequence of Borel measures $\mu_n$ on $\spc{X}_n$ and so on.

For that reason, it is useful to define \index{Hausdorff convergence}\emph{convergence} by specifying the metric on $\bm{X}$ that satisfies the property
for the variation of Hausdorff convergence described in Section~\ref{sec:H-variation}.

This approach is more flexible;
in particular, it can be used to define Gromov--Hausdorff convergence of arbitrary metric spaces (not necessarily compact).
A limit space for this generalized convergence is not uniquely defined.
For example, if each space $\spc{X}_n$ in the sequence is isometric to the half-line, then its limit might be isometric to the half-line or the whole line.
The first convergence is evident and the second could be guessed from the diagram.

\begin{figure}[ht!]
\vskip-0mm
\centering
\includegraphics{mppics/pic-500}
\end{figure}

Often the isometry class of the limit can be fixed by marking a point $p_n$ in each space $\spc{X}_n$, it is called \index{pointed convergence}\emph{pointed Gromov--Hausdorff convergence} --- we say that $(\spc{X}_n,p_n)$ converges to $(\spc{X}_\infty,p_\infty)$ if there is a metric on $\bm{X}$ as in \ref{propery:GH} such that $\spc{X}_n\Hto \spc{X}_\infty$ and $p_n\to p_\infty$.
For example, the sequence $(\spc{X}_n,p_n)=(\RR_+,0)$ converges to $(\RR_+,0)$, while $(\spc{X}_n,p_n)=(\RR_+,n)$ converges to $(\RR,0)$.

The pointed convergence works nicely for proper metric spaces;
the following theorem is an analog of Gromov's selection theorem for this convergence.

\begin{thm}{Theorem}\label{thm:pointed-gromov-compactness}%
Let $\spc{Q}$ be a set of isometry classes of pointed proper metric spaces
$(\spc{X},p)$.
Assume that for any $R>0$, the $R$-balls in the spaces centered at the marked points form a uniformly totally bounded family of spaces.
Then $\spc{Q}$ is precompact with respect to the pointed Gromov--Hausdorff convergence. 
\end{thm}

\chapter{Ultralimits}

Ultralimits provide a very general way to pass to a limit that always works.
It use a set-theoretical construction --- the so called ulrafilter.

In geometry, ultralimits are used only as a canonical way to pass to a convergent subsequence.
It is useful thing in the proofs where one needs to repeat ``pass to convergent subsequence'' too many times.

This lecture is based on the introduction to the paper of Bruce Kleiner and Bernhard Leeb \cite{kleiner-leeb}.

\section{Ultrafilters}

Recall that $\NN$ denotes the set of natural numbers, $\NN=\{1,2,\dots\}$

\begin{thm}{Definition}
A finitely additive measure $\omega$ 
on  $\NN$ 
is called an \index{ultrafilter}\emph{ultrafilter} if it satisfies 
\begin{subthm}{}
$\omega(S)=0$ or $1$ for any subset $S\subset \NN$.
\end{subthm}
An ultrafilter $\omega$ is called 
\emph{nonprinciple}\index{ultrafilter!nonprinciple ultrafilter}\index{nonprinciple ultrafilter} if in addition 
\begin{subthm}{}
$\omega(F)=0$ for any finite subset $F\subset \NN$.
\end{subthm}
\end{thm}

If $\omega(S)=0$ for some subset $S\subset \NN$,
we say that $S$ is \index{$\omega$-small}\emph{$\omega$-small}. 
If $\omega(S)=1$, we say that $S$ contains \index{$\omega$-almost all}\emph{$\omega$-almost all} elements of $\NN$.

\parbf{Classical definition.}
More commonly, a nonprinciple ultrafilter is defined as a collection, say $\mathfrak{F}$, of sets in $\NN$ such that
\begin{enumerate}
\item\label{filter:supset} if $P\in \mathfrak{F}$ and $Q\supset P$, then $Q\in \mathfrak{F}$,
\item\label{filter:cap} if $P, Q\in \mathfrak{F}$, then $P\cap Q\in \mathfrak{F}$,
\item\label{filter:ultra} for any subset $P\subset\NN$, either $P$ or its complement is an element of $\mathfrak{F}$.
\item\label{filter:non-prin} if $F\subset \NN $ is finite, then $F\notin \mathfrak{F}$.
\end{enumerate}
Setting $P\in\mathfrak{F}\Leftrightarrow\omega(P)=1$ makes these two definitions equivalent.

A nonempty collection of sets $\mathfrak{F}$ that does not include the empty set and satisfies only conditions \ref{filter:supset} and \ref{filter:cap} is called a \index{filter}\emph{filter}; 
if in addition $\mathfrak{F}$ satisfies Condition~\ref{filter:ultra} it is called an \index{ultrafilter}\emph{ultrafilter}.
From Zorn's lemma, it follows that every filter contains an ultrafilter.
Thus there is an ultrafilter $\mathfrak{F}$ contained in the filter of all complements of finite sets; clearly this $\mathfrak{F}$ is nonprinciple.


\parbf{Stone--\v{C}ech compactification.}
Given a set $S\subset \NN$, consider subset $\Omega_S$ of all ultrafilters $\omega$ such that $\omega(S)=1$.
It is straightforward to check that the sets $\Omega_S$ for all $S\subset \NN$ form a topology on the set of ultrafilters on $\NN$. 
The obtained space is called \index{Stone--\v{C}ech compactification}\emph{Stone--\v{C}ech compactification} of $\NN$;
it is usually denoted as $\beta\NN$.

Let $\omega_n$ denotes the principle ultrafilter such that $\omega_n(\{n\})=1$; that is, $\omega_n(S)=1$ if and only if $n\in S$.
Note that $n\mapsto\omega_n$ defines a natural embedding $\NN\hookrightarrow\beta\NN$. 
Using the described embedding, we can (and will) consider $\NN$ as a subset of $\beta\NN$.

The space $\beta\NN$ is the maximal compact Hausdorff space that contains $\NN$  as an everywhere dense subset.
More precisely, for any compact Hausdorff space $\spc{X}$ 
and a map $f\:\NN\to \spc{X}$ there is unique continuous map $\bar f\:\beta\NN\to X$ such that the restriction $\bar f|_\NN$ coincides with $f$. 

\section{Ultralimits of points}
\label{ultralimits}\index{ultralimit}

Further we will need the existence of a nonprinciple  ultrafilter $\omega$,
which we fix once and for all.

Assume $(x_n)$ is a sequence of points in a metric space $\spc{X}$. 
Let us define the \index{$\omega$-limit}\emph{$\omega$-limit} of $(x_n)$ as the point $x_\omega$ 
such that for any $\eps>0$, $\omega$-almost all elements of $(x_n)$ lie in $\oBall(x_\omega,\eps)$; 
that is,
\[\omega\set{n\in\NN}{\dist{x_\omega}{x_n}{}<\eps}=1.\]
In this case, we will write 
\[x_\omega=\lim_{n\to\omega} x_n
\ \ \text{or}\ \ 
x_n\to x_\omega\ \text{as}\ n\to\omega.\]

For example if $\omega$ is the principle ultrafilter such that $\omega(\{n\})=1$ for some $n\in\NN$, then
$x_\omega=x_n$.

Alternatively, the sequence $(x_n)$ can be regarded as a map $\NN\to\spc{X}$.
In this case the map $\NN\to\spc{X}$ can be extended to a continuous map $\beta\NN\to\spc{X}$ from the Stone--\v{C}ech compactification of $\NN$.
Then the $\omega$-limit $x_\omega$ can be regarded as the image of $\omega$.

Note that $\omega$-limits of a sequence and its subsequence may differ.
For example, in general
\[\lim_{n\to\omega}x_n
\ne
\lim_{n\to\omega}x_{2\cdot n}.\] 

\begin{thm}{Proposition}\label{prop:ultra/partial}
Let $\omega$ be a nonprinciple ultrafilter.
Assume $(x_n)$ is a sequence of points in a metric space $\spc{X}$
and $x_n\to  x_\omega$ as $n\to\omega$.
Then $x_\omega$ is a partial limit of the sequence $(x_n)$;
that is, there is a subsequence $(x_n)_{n\in S}$ that converges to $x_\omega$ in the usual sense.
\end{thm}

\parit{Proof.}
Given $\eps>0$, 
set $S_\eps=\set{n\in\NN}{\dist{x_n}{x_\omega}{}<\eps}$.

Note that $\omega(S_\eps)=1$ for any $\eps>0$.
Since $\omega$ is nonprinciple, the set $S_\eps$ is infinite.
Therefore we can choose an increasing sequence $(n_k)$
such that $n_k\in S_{\frac1k}$ for each $k\in \NN$.
Clearly $x_{n_k}\to x_\omega$ as $k\to\infty$.
\qeds

The following proposition 
is analogous to the statement that any sequence in a compact metric space 
has a convergent subsequence;
it can be proved the same way.

\begin{thm}{Proposition}\label{prop:ultra/compact}
Let $\spc{X}$ be a compact metric space.
Then
any sequence of points $(x_n)$ in $\spc{X}$ has unique $\omega$-limit $x_\omega$.

In particular, a bounded sequence of real numbers has a unique $\omega$-limit.
\end{thm}

The following lemma is an ultralimit analog of Cauchy convergence test.

\begin{thm}{Lemma}\label{lem:X-X^w}
Let $(x_n)$ be a sequence of points in a complete space $\spc{X}$. 
Assume for each subsequence $(y_n)$ of $(x_n)$, 
the $\omega$-limit 
\[y_\omega=\lim_{n\to\omega}y_{n}\in \spc{X}\]
is defined and does not depend on the choice of subsequence, 
then the sequence $(x_n)$ converges in the usual sense.
\end{thm}

\parit{Proof.} If $(x_n)$ is not a Cauchy sequence,
then for some $\eps>0$, there is a subsequence $(y_n)$ of $(x_n)$ such that $\dist{x_n}{y_n}{}\ge\eps$ for all $n$.

It follows that $\dist{x_\omega}{y_\omega}{}\ge \eps$, a contradiction.\qeds


\section{Ultralimits of spaces}\label{sec:Ultralimit of spaces}

Recall that $\omega$ denotes a nonprinciple ultrafilter on the set of natural numbers.

Let $\spc{X}_n$ be a sequence of metric spaces.
Consider all sequences of points $x_n\in \spc{X}_n$.
On the set of all such sequences,
define a pseudometric by
\[\dist{(x_n)}{(y_n)}{}
=
\lim_{n\to\omega} \dist{x_n}{y_n}{\spc{X}_n}.
\eqlbl{eq:olim-dist}\]
Note that the $\omega$-limit on the right hand side is always defined 
and takes a value in $[0,\infty]$. 

Set $\spc{X}_\omega$ to be the corresponding metric space; 
that is, the underlying set of $\spc{X}_\omega$ is formed by classes of equivalence of sequences of points $x_n\in\spc{X}_n$ 
defined by 
\[(x_n)\sim(y_n)
\ \Leftrightarrow\ 
\lim_{n\to\omega} \dist{x_n}{y_n}{}=0\]
and the distance is defined by \ref{eq:olim-dist}.

The space $\spc{X}_\omega$ is called \index{$\omega$-limit space}\emph{$\omega$-limit} of $\spc{X}_n$.
Typically  $\spc{X}_\omega$ will denote the  
$\omega$-limit of sequence $\spc{X}_n$;
we may also write  
\[\spc{X}_n\to\spc{X}_\omega\ \ \text{as}\ \  n\to\omega\ \ \text{or}\ \ \spc{X}_\omega=\lim_{n\to\omega}\spc{X}_n.\]

Given a sequence $x_n\in \spc{X}_n$,
we will denote by $x_\omega$ its equivalence class which is a point in $\spc{X}_\omega$;
equivalently we will write
\[x_n\to x_\omega \ \ \text{as}\ \  n\to\omega\ \ \text{or}\ \ x_\omega=\lim_{n\to\omega} x_n.\]

\begin{thm}{Observation}\label{obs:ultralimit-is-complete}
The $\omega$-limit of any sequence of metric spaces is complete. 
\end{thm}

\parit{Proof.}
Let $\spc{X}_n$ be a sequence of metric spaces and $\spc{X}_n\to\spc{X}_\omega$ as $n\to\omega$.

Fix a Cauchy sequence $x_{m}\in \spc{X}_\omega$.
Passing to a subsequence we can assume that $\dist{x_m}{x_{m-1}}{\spc{X}_\omega}<\tfrac1{2^m}$ for any $m$.

Let us choose double sequence $x_{n,m}\in \spc{X}_n$ such that for any fixed $m$ we have $x_{n,m}\to x_m$ as $n\to\omega$.
Note that $\dist{x_{n,m}}{x_{n,m-1}}{}<\tfrac1{2^m}$ for $\omega$-almost all $n$.
It follows that we can choose a nested sequence of sets 
\[\NN= S_1\supset S_2\supset\dots\] 
such that 
\begin{itemize}
\item $\omega(S_m)=1$ for each $m$, 
\item $k\ge m$ for any $k\in S_m$, and
\item if $n\in S_m$, then 
\[\dist{x_{n,m}}{x_{n,m-1}}{}<\tfrac1{2^m}\]
\end{itemize}

Consider the sequence $y_n=x_{n,m(n)}$, where $m(n)$ is the largest value such that $m(n)\in S_{m}$.
Denote by $y\in \spc{X}_\omega$ its $\omega$-limit.

Observe that by construction $x_n\to y$ as $n\to \infty$.
Hence the statement follows.
\qeds

\begin{thm}{Observation}\label{obs:ultralimit-is-geodesic}
The $\omega$-limit of any sequence of length spaces is geodesic. 
\end{thm}

\parit{Proof.}
If $\spc{X}_n$ is a sequence length spaces, then for any sequence of pairs $x_n, y_n\in X_n$ there is a sequence of $\tfrac1n$-midpoints $z_n$.

Let $x_n\to x_\omega$, $y_n\to y_\omega$ and $z_n\to z_\omega$ as $n\to \omega$.
Note that $z_\omega$ is a midpoint of $x_\omega$ and $y_\omega$ in $\spc{X}^\omega$.

By Observation~\ref{obs:ultralimit-is-complete}, $\spc{X}^\omega$ is complete.
Applying Lemma~\ref{lem:mid>geod} we get the statement.
\qeds


\begin{thm}{Exercise}\label{ex:lim(tree)}
Show that an ultralimit of metric trees is a metric tree. 
\end{thm}

\section{Ultrapower}

If all the metric spaces in the sequence are identical $\spc{X}_n=\spc{X}$, 
its $\omega$-limit 
$\lim_{n\to\omega}\spc{X}_n$
is denoted by $\spc{X}^\omega$
and called $\omega$-power of $\spc{X}$.



\begin{thm}{Exercise}\label{ex:ultrapower}
For any point $x\in \spc{X}$, consider the constant sequence $x_n=x$
and set $\iota(x)=\lim_{n\to\omega}x_n\in\spc{X}^\omega$.

\begin{subthm}{ex:ultrapower:a}
Show that $\iota\:\spc{X}\to\spc{X}^\omega$ is distance-preserving embedding. (So we can and will consider $\spc{X}$ as a subset of $\spc{X}^\omega$.)
\end{subthm}

\begin{subthm}{ex:ultrapower:compact} 
Show that $\iota$ is onto if and only if $\spc{X}$ compact.
\end{subthm}

\begin{subthm}{ex:ultrapower:proper} 
Show that if $\spc{X}$ is proper, then $\iota(\spc{X})$ forms a metric component of $\spc{X}^\omega$; that is, a subset of $\spc{X}^\omega$ that lie on finite distance from a given point.
\end{subthm}

\end{thm}

\begin{thm}{Observation}\label{obs:ultrapower-is-geodesic}
Let $\spc{X}$ be a complete metric space. 
Then $\spc{X}^\omega$ is geodesic space if and only if $\spc{X}$ is a length space.
\end{thm}

\parit{Proof.}
Assume $\spc{X}^\omega$ is geodesic space.
Then any pair of points $x,y\in \spc{X}$ has a midpoint $z_\omega\in\spc{X}^\omega$.
Fix a sequence of points $z_n\in  \spc{X}$ such that $z_n\to z_\omega$ as $n\to \omega$.

Note that 
$\dist{x}{z_n}{\spc{X}}\to \tfrac12\cdot \dist{x}{y}{\spc{X}}$
and 
$\dist{y}{z_n}{\spc{X}}\to \tfrac12\cdot \dist{x}{y}{\spc{X}}$
as 
$n\to\omega$.
In particular, for any $\eps>0$, the point $z_n$ is an $\eps$-midpoint of $x$ and $y$ for $\omega$-almost all $n$.
It remains to apply \ref{lem:mid>geod}.

The ``if''-part follows from \ref{obs:ultralimit-is-geodesic}.
\qeds

\begin{thm}{Exercise}\label{ex:two-geodesics-in-ultrapower}
Assume $\spc{X}$ is a complete length space 
and $p,q\in\spc{X}$ cannot be joined by a geodesic in $\spc{X}$.  
Then there are at least two distinct geodesics between $p$ and $q$ 
in the ultrapower $\spc{X}^\omega$.
\end{thm}

\begin{thm}{Exercise}
 Construct a proper metric space $\spc{X}$ such that $\spc{X}^\omega$ is not proper; that is, there is a point $p\in \spc{X}^\omega$ and $R<\infty$ such that the closed ball $\cBall[p,R]_{\spc{X}^\omega}$ is not compact.
\end{thm}

\section{Tangent and asymptotic spaces}

Choose a space $\spc{X}$ and a sequence of $\lambda_n>0$.
Consider the sequence of scalings $\spc{X}_n=\lambda_n\cdot\spc{X}=(\spc{X},\lambda_n\cdot\dist{*}{*}{\spc{X}})$.

Choose a point $p\in \spc{X}$ and denote by $p_n$ the corresponding point in $\spc{X}_n$.
Consider the $\omega$-limit $\spc{X}_\omega$ of $\spc{X}_n$ (one may denote it by $\lambda_\omega\cdot \spc{X}$);
set $p_\omega$ to be the $\omega$-limit of $p_n$.

If $\lambda_n\to 0$ as $n\to\omega$, then the metric component of $p_\omega$ in $\spc{X}_\omega$ is called \index{$\omega$-tangent space}\emph{$\omega$-tangent space} at $p$ and denoted by $\T_p^{\lambda_\omega}\spc{X}$ (or $\T_p^{\omega}\spc{X}$ if $\lambda_n=n$).\label{page:ultratangent space}

If $\lambda_n\to \infty$ as $n\to\omega$, then the metric component of $p_\omega$ in called \index{$\omega$-asymptotic space}\emph{$\omega$-asymptotic space}\footnote{Often it is called \emph{asymptotic cone} despite that it is not a cone in general; this name is used since in good cases it has a cone structure.}  and denoted by $\Asym\spc{X}$ or $\Asym^{\lambda_\omega}\spc{X}$.
Note that the space $\Asym\spc{X}$ and its point $p_\omega$ does not depend on the choice of $p\in \spc{X}$.

In general, the tangent and asymptotic spaces depend the sequence $\lambda_n$ and an nonprinciple ultrafiler $\omega$.
For nice spaces different choices may give the same space.

\begin{thm}{Exercise}\label{ex:ultraT}
Construct a metric space $\spc{X}$ with a point $p$ such that the tangent space
$\T_p^{\lambda_\omega}\spc{X}$ depends on the sequence $\lambda_n$ and/or ultrafilter $\omega$.
\end{thm}


\begin{thm}{Exercise}\label{ex:Asym(Lob)}
Let $\spc{L}$ be the Lobachevsky plane; $\spc{T}=\Asym\spc{L}$.

\begin{subthm}{ex:Asym(Lob):metric-tree}
Show that $\spc{T}$ is a complete metric tree.
\end{subthm}

\begin{subthm}{ex:Asym(Lob):continuum}
Show that $\spc{T}$ has {}\emph{continuum degree} at any point;
that is, for any point $t\in \spc{T}$ the set of connected components of the complement $\spc{T}\backslash\{t\}$ has cardinality continuum.
\end{subthm}

\begin{subthm}{ex:Asym(Lob):homogeneous}
Show that $\spc{T}$ is homogeneous; that is given two points $s,t\in \spc{T}$ there is an isometry of $\spc{T}$ that maps $s$ to $t$.
\end{subthm}

\begin{subthm}{ex:Asym(Lob):others}
Prove \ref{SHORT.ex:Asym(Lob):metric-tree}--\ref{SHORT.ex:Asym(Lob):homogeneous} if $\spc{L}$ is Lobachevsky space and/or for the infinite 3-regular%
\footnote{that is, degree of any vertex is 3.}
tree with unit edge. 
\end{subthm}


\end{thm}

As it shown in \cite{dyubina-polterovich}, the properties \ref{SHORT.ex:Asym(Lob):metric-tree} and \ref{SHORT.ex:Asym(Lob):continuum} describe the tree $\spc{T}$ up to isometry.
In particular, the asymptotic space of Lobachevsky plane does not depend on the choice of ultrafilter and the sequence $\lambda_n\to \infty$.


\section{Remarks}

A nonprinciple ultrafilter $\omega$ is called 
\emph{selective}\index{ultrafilter!selective ultrafilter}\index{selective ultrafilter} if for any partition of $\NN$ into sets $\{C_\alpha\}_{\alpha\in\IndexSet}$ such that $\omega(C_\alpha)\z=0$ for each $\alpha$, 
there is a set $S\subset \NN$ such that $\omega(S)=1$ and $S\cap C_\alpha$ is a one-point set for each $\alpha\in\IndexSet$.

The existence of a selective ultrafilter follows from the continuum hypothesis;
it was proved by Walter Rudin \cite{rudin}.

For a selective ultrafilter $\omega$, there is a stronger version of Proposition~\ref{prop:ultra/partial};
namely we can assume that the subsequence $(x_n)_{n\in S}$ can be chosen so that $\omega(S)=1$.
So, if needed, one may assume that the ultrafilter $\omega$ is chosen to be selective and use this stronger version of the proposition.

%\chapter{Metric plus measure}

\section{Borel sets}

Let us remind few definitions assuming knowleage of basic measure theory;
comprehensive treatments can be found in \cite{billingsley} and \cite{bogachev}.

Let $\spc{X}$ be a metric space.
\index{Borel set}\emph{Borel set} is any subset of $\spc{X}$ that can be formed from open sets using the countable union, countable intersection, and complement.
In other words, Borel sets form the minimal sigma-algebra that included open sets.

A measure on metric space will be always assumed to be \index{Borel measure}\emph{Borel};
that is, it is defined on the sigma-algebra of Borel sets.
A Borel measure can be uniquely determined by its values on all open (or closed) sets.

A measure $\mu$ on $\spc{X}$ is called \index{probability measure}\emph{probability measure} if $\mu\spc{X}=1$.

Recall that \index{delta-measure}\emph{delta-measure} is a probability measure with support at one point.
Delta-measure with support in $\{x\}$ will be denoted by~\index{$\delta_{x}$}$\delta_{x}$; so
\[\text{if}\quad x\in A,\quad\text{then}\quad \delta_x(A)=1,\quad\text{otherwise}\quad\delta_x(A)=0.\]

Let $\mu_n$ be a sequence of Borel measures on $\spc{X}$.
A measure $\mu_\infty$ is a \index{weak limit}\emph{weak limit} of $\mu_n$ if 
\[\int_{\spc{X}}f\cdot(\mu_n-\mu_\infty)\to0\gamma
\quad\text{as}\quad
n\to\infty
\]
for any continuous function $f\:\spc{X}\to \RR$.

Suppose $\mu$ is a measure on a metric space $\spc{X}$ and $f\:\spc{X}\to \spc{Y}$ is a measurable map;
that is, for any Borel set $B\subset \spc{Y}$, its inverse image $f^{-1}B$ is a Borel set in $\spc{Y}$.

Consider the unit interval with its Lebesgue mesure.
If $\spc{X}$ is a complete separable metric space with probability measure $\mu$, then there is a measurable map $[0,1]\to \spc{X}$

\section{Metric on measures}

Imagine that we need to transport dirt from one pile of a given shape to make another pile of a needed shape.
Suppose that cost of transporting a unit of dirt equals to the traveled distance.%
\footnote{This is the simplest cost function one can imagine.
One may consider other cost functions; for example, if the cost proportional to the square of the distance, then the problem has more applications.}
We are free to choose a destination point for dirt from a given place.
How to minimize the total cost?

To formalize this question,
suppose that the piles of dirt are described by Borel probability measures $\mu$ and $\nu$ on a metric space~$\spc{X}$.

To describe where each piece of dirt goes, we will use the so called \index{plan}\emph{plan} for $\mu$ and $\nu$.
It is a probability measure $\pi$ on the product $\spc{X}\times\spc{X}$ such that 
for all measurable sets $A \subset \spc{X}$, we have 
\[\mu A= \pi [A \times \spc{X}],
\quad\text{and}\quad
\nu A=\pi [\spc{X}\times A].
\eqlbl{eq:marginals}\]
Equivalently it can be described as a measure that satisfies the following identity
\begin{align*}
\int_{(x,y)\in \spc{X}\times\spc{X}}f(x)\cdot g(y) \cdot \pi
&=
\int_{x\in \spc{X}}f(x)\cdot \mu
\oldcdot \int_{y\in \spc{X}}g(y)\cdot \nu,
\end{align*}
for any continuous functions $f,g\:\spc{X}\to \RR$.

Given a measure $\pi$ on $\spc{X}\times\spc{X}$, the measures $\mu$ and $\nu$ defined by \ref{eq:marginals} are called first and second \index{marginal}\emph{marginals} of $\pi$;
so the statement \textit{$\pi$ is a plan for $\mu$ and $\nu$} is equivalent to \textit{$\mu$ and $\nu$ are the first and second marginals of $\pi$ respectively}.

\begin{thm}{Claim}\label{clm:plan-exists}
There is a plan $\pi$ for any two given Borel probability measures $\mu$ and $\nu$.
\end{thm}

The plan constructed in the proof distributes equally each piece of dirt in the new pile.
As we will see this plan is usually far from optimum.

\parit{Proof.}
Consider the measure $\pi$ that is uniquely defined  defined by the identity
\[\pi(A\times B)=\mu A\cdot \mu B\]
for any Borel subsets $A,B\subset\spc{X}$.
Observe that $\pi$ is a plan for $\mu$ and~$\nu$.
\qeds

Denote by $\Pi(\mu,\nu)$ the set of all plans for $\mu$ and $\nu$;
by \ref{clm:plan-exists}, $\Pi(\mu,\nu)\z\ne\emptyset$.
It is straightforwrd to check that the following formula defines a metric on the space of probability measures on $\spc{X}$.
\[\dist{\mu}{\nu}{\Wass_1\spc{X}}
\df
\inf_{\pi\in\Pi(\mu,\nu)}
\left\{\,\int_{(x,y)\in\spc{X}\times\spc{X}}\dist{x}{y}{\spc{X}}\cdot\pi\,\right\}.\]
This metric is called \index{Wasserstein distance}\emph{Wasserstein distance of order 1} between $\mu$ and $\nu$.

In genereral, the Wasserstein distance $\dist{\mu}{\nu}{}$ might take infinite value, but all measures with compact support lie on finite distance from each other in the obtained $\infty$-metric space.
The metric component of these measures is called \index{Wasserstein space}\emph{Wasserstein space} of order 1 over $\spc{X}$; 
it is denoted by $\Wass_1\spc{X}$.
In other words, $\Wass_1\spc{X}$ is the space of all Borel probability measures $\mu$ such that 
$\int\distfun_p\cdot\mu<\infty$ for some (and therefore any) point $p\in \spc{X}$.

\begin{thm}{Exercise}\label{ex:wasserstein-infty}
Construct two Borel probability measures $\mu$ and $\nu$ on $\RR$ with Wasserstein distance $\dist{\mu}{\nu}{}=\infty$.
\end{thm}


\begin{thm}{Exercise}\label{ex:wasserstein-compact}
Show that $\Wass_1\spc{X}$ is a compact if and only if so is~$\spc{X}$.
\end{thm}

\begin{thm}{Exercise}\label{ex:wasserstein-length}
Show that the Wasserstein space $\Wass_1\spc{X}$ is a geodesic space for any metric space $\spc{X}$.
\end{thm}

\section{Optimal plan}

A plan $\pi$ for given measures $\mu$ and $\nu$ is called \index{optimal plan}\emph{optimal} if 
\[\dist{\mu}{\nu}{\Wass_1\spc{X}}
=\int_{(x,y)\in\spc{X}\times\spc{X}}\dist{x}{y}{\spc{X}}\cdot\pi.\]

\begin{thm}{Theorem} %Vilani:Theorem 1.4
Let $\mu$ and $\nu$ be probability Borel measures on a compact metric space $\spc{X}$.
Then there is an optimal plan $\pi$ for $\mu$ and~$\nu$.
\end{thm}

\parit{Proof.}
By the definition of Wasserstein distance, we can choose a sequence of plans $\pi_n$ for $\mu$ and $\nu$ such that 
\[\int_{(x,y)\in\spc{X}\times\spc{X}}\dist{x}{y}{\spc{X}}\cdot\pi_n\to \dist{\mu}{\nu}{\Wass_1\spc{X}}\]
as $n\to \infty$.

Observe that $\pi_n$ has a weak partial limit, say $\pi$.
Moreover $\pi$ is an optimal plan for $\mu$ and $\nu$.
\qeds

\begin{thm}{Theorem}
Any optimal plan $\pi$ is \index{cyclic monotonicity}\emph{cyclically monotonic}.
That is, suppose $\pi$ is an optimal plan for probability measures $\mu$ and $\nu$ on a metric space $\spc{X}$.
Then any sequence of pairs $(x_1,y_1),\dots,(x_n,y_n)\in\supp\pi\subset\spc{X}\times\spc{X}$ we have
\[\sum_i\dist{x_i}{y_i}{}
\le
\sum_i\dist{x_{i+1}}{y_i}{},\]
here the index $i$ in the sum is taken modulo $n$; in particular $x_{n+1}\z=x_1$.
\end{thm}

\parit{Proof.}
Assume that the cyclic monotonicity does not hold;
that is,
\[R=\sum_i\dist{x_i}{y_i}{}
-
\sum_i\dist{x_{i+1}}{y_i}{}>0,\]
for some $(x_0,y_0),\dots,(x_n,y_n)\in\supp\pi$.
We need to show that $\pi$ is not optimal;
in other words we need to construct another plan $\pi'$ for $\mu$ and $\nu$ such that 
\[\int_{(x,y)\in\spc{X}\times\spc{X}}\dist{x}{y}{\spc{X}}\cdot(\pi'-\pi)<0.\eqlbl{pi'<pi}\]

Assume $\spc{X}$ is finite.
In this case we can choose $\eps>0$ such that 
$\pi\{(x_i,y_i)\}>\eps$ for each $i$.
Let
\[\pi'=\pi-\eps\cdot\sum_i(\sigma_i-\sigma_i')\eqlbl{eq:pi'}\]
where $\sigma_i=\delta_{(x_i,y_i)}$ and $\sigma_i'=\delta_{(x_{i+1},y_i)}$.
It remains to observe that $\pi'$ is a plan for $\mu$ and $\nu$ that satisfies \ref{pi'<pi}.

The general case is similar, we only need to redefine $\eps$, $\sigma_i$, and~$\sigma_i'$.
Note that given $r>0$, we can choose a probability measures $\sigma_i$ with support in $\oBall((x_i,y_i),r)_{\spc{X}\times\spc{X}}$ such that $\eps\cdot \sigma_i<\pi$ for some fixed $\eps>0$ and every $i$.
Further denote by $\zeta_i$ and $\eta_i$ the first and second marginals of $\sigma_i$.
Observe that $\supp\zeta_i\subset\oBall(x_i,r)$ and $\supp\eta_i\subset\oBall(y_i,r)$ for all $i$.
Let $\sigma_i'$ be a plan for $\zeta_{i+1}$ and $\eta_i$.
Evidently 
\begin{align*}
\int_{(x,y)\in\spc{X}\times\spc{X}}\dist{x}{y}{}\cdot \sigma_i
&\lessgtr
\dist{x_i}{y_i}{}\pm 2\cdot r,
\\
\int_{(x,y)\in\spc{X}\times\spc{X}}\dist{x}{y}{}\cdot \sigma_i'
&\lessgtr
\dist{x_{i+1}}{y_i}{}\pm 2\cdot r.
\end{align*}
Taking $r<\tfrac R{10\cdot n}$, we get  \ref{pi'<pi}. 
\qeds




\section{Capitalistic approach}

Imagine that measures $\mu$ and $\nu$ describe the production and consumer of beer in the space.
A transportaition company transports beer from $\mu$ to $\nu$ and want to maximize its profit by adjusting price $f(x)$ of beer the point $x$; they buy beer at price $f(x)$ per unit, move it to an other point $y$ and sale it with (presumably higher) price $f(y)$.
However, the function $f$ is 1-Lipschitz condition;
otherwise the profit goes to second-hand dealers, or maybe it is a government regulation.
In other words we need to maximize the following expression
\[\int_{\spc{X}} f\cdot(\mu-\nu)\]
for all $1$-Lipschitz functions $f$.
The maximal profit defines a metric

\begin{thm}{Theorem}
Let $\mu$ and $\nu$ be probability Borel measures on a compact metric space $\spc{X}$.
Then
\[\dist{\mu}{\nu}{\Wass_1\spc{X}}=\sup\int_{\spc{X}} f\cdot(\mu-\nu),\]
where the least upper bound is taken for all $1$-Lipschitz functions $f\:\spc{X}\z\to\RR$.
\end{thm}

The definition of Wassershtein metric described in the previous section reminds communist's planed economy.
The right-hand side in the above equation reminds capitalistic system.
Indeed, think that measures $\mu$ and $\nu$ describe the production and consumer of beer in the space.
A transportaition company trnasports beer from $\mu$ to $\nu$ and want to maximize its profit by adjusting price $f(x)$ of beer the point $x$.
However, the function $f$ is 1-Lipschitz condition --- this is a government regulation.




\parit{Proof.}
By the definition of Wasserstein metric, we can choose a sequence $\pi_n$ of plans  

Let us choose an optimal plan $\pi$ for $\mu$ and $\nu$; it exists by ???.
We need to find a 1-Lipschitz function $f\:\spc{X}\to\RR$ such that 
\[
\int_{\spc{X}} f\cdot(\mu-\nu)=\int_{(x,y)\in\spc{X}\times\spc{X}}\dist{x}{y}{\spc{X}}\cdot \pi.
\eqlbl{eq:f(mu-nu)}
\]

Choose $x_0\in \supp\mu$.
Note that adding a constant to $f$ does not change the left hand side in \ref{eq:f(mu-nu)}.
Therefore we can assume assume that $f(x_0)=0$ and set
\[f(x)=\sup\{\,|x_0-y_0|+\dots+|x_n-y_n|-(|x_1-y_0|+\dots+|x_n-y_{n-1}|)-|x-y_n|\,\}\]
where the least upper bound is taken for all sequences $(x_0,y_0),\z\dots,(x_n,y_n)\z\in\supp\pi$.

\qeds

\section{Metric-measure space}

A metric measure space is a metric $\spc{X}$ space with a choice of Borel probability measure $\vol$ on it.
In a metric-measure we ignore sets with vanishing volume; in other words, passing from $\spc{X}$ to the support of $\vol$ does not change the metric-measure space.

Alternatively we may start with unit interval $[0,1]$ equipped with Lebesgue measure and equip it with measurable pseudometric $[0,1]\times [0,1]\to \RR$.





\section{Space of measures}


It can be equipped with the \index{Wasserstein metric}\emph{Wasserstein metric}
\[\dist{\mu}{\nu}{}\df\sup\left\{\,\int_{\spc{X}} f\cdot(\mu-\nu)\,\right\},\]
where the least upper bound is taken for all $1$-Lipschitz functions $f\:\spc{X}\to\RR$.

The Wasserstein distance $\dist{\mu}{\nu}{}$ might take infinite value, but all measures with compact support lie on finite distance from each other in the obtained $\infty$-metric space.
The metric component of these measures is called \index{Wasserstein space}\emph{Wasserstein space} of order 1 over $\spc{X}$; 
it is denoted by $\Wass_1\spc{X}$.



\section{Misc}

Suppose $\pi_n$ is a sequence of plans for $\mu$ and $\nu$.
Assume that $\pi_n$ weakly converges to a probability measure $\pi$ on $\spc{X}\times\spc{X}$.

is a weak limit of a sequence of plans $\pi_n$, then $\pi$ is a plan for $\mu$ and $\nu$ if for each $n$ $\pi_n$ is a plane for $\mu$ and $\nu$ 

Suppose that $f\:\spc{X}\to \RR$ is a 1-Lipschitz function,
so $f(x)-f(y)\le\dist{x}{y}{\spc{X}}$ for any $x,y\in \spc{X}$.
It follows that 
\begin{align*}
\int_{\spc{X}} f\cdot(\mu-\nu)&=\int_{x\in\spc{X}}f(x)\cdot\mu-\int_{y\in\spc{X}}f(y)\cdot\nu=
\\
&=\int_{(x,y)\in\spc{X}\times\spc{X}}[f(x)-f(y)]\cdot \pi\le
\\
&\le\int_{(x,y)\in\spc{X}\times\spc{X}}\dist{x}{y}{\spc{X}}\cdot \pi,
\end{align*}
where $\pi$ is a plan for $\mu$ and $\nu$.
By the definition of Wasserstein metric, we get  
\[\dist{\mu}{\nu}{\Wass_1\spc{X}}\le \int_{(x,y)\in\spc{X}\times\spc{X}}\dist{x}{y}{\spc{X}}\cdot\pi\eqlbl{wass=<int.plan}\]
for any plan $\pi$.

Next we want to show that equality holds in \ref{wass=<int.plan} for some plan $\pi$; such plans will be called \index{optimal plan}\emph{optimal}.


\parit{Proof.}
Choose a point $x_0\in \supp\mu$.
Given  $p\in \spc{X}$,
let
\[f(p)=\inf\left\{\sum_{i=0}^n\dist{x_i}{y_i}{}-\sum_{i=0}^n\dist{x_{i+1}}{y_i}{}-\dist{y_n}{p}{}\right\},
\eqlbl{eq:f(p)}\]
where the least upper bound is taken for all sequences of pairs 
\[(x_0,y_0),\z\dots,(x_n,y_n)\in \supp\pi.\eqlbl{eq:sequence}\]

Fix a sequence as in \ref{eq:sequence} and  denote by $F_\sigma(p)$ the expression under infimum in \ref{eq:f(p)}.

Let us show that 
\[F_\sigma(x_0)\ge 0.\]
Indeed, suppose $F_\sigma(x_0)<-\eps<0$.
Since $(x_i,y_i)\in \supp\pi$, we have $x_i\in\supp\mu$ and $y_i\in\supp\nu$ for any $i$.
Therefore we can choose sets $X_i\subset \oBall(x_i,\tfrac{\eps}{10\cdot n})$ and $Y_i\subset \oBall(y_i,\tfrac{\eps}{10\cdot n})$ such that 
$\mu(X_0)=\nu(Y_0)=\dots=\mu(X_n)=\nu(Y_n)$



Let us denote by $F(p)$ the expression under infimum in \ref{eq:f(p)}.
By triangle inequality, 
\[F(q)\le F(p)+\dist{p}{q}{}.\]
Passing to the least upper bound in this inequality, we get
\[f(q)\le f(p)+\dist{p}{q}{}\]
for any $p,q\in\spc{X}$.
Hence $f$ is a 1-Lipschitz function.

Further, let us show that
\[(x,y)\in\supp\pi
\quad\Longrightarrow\quad
f(y)-f(x)=\dist{x}{y}{}\]





Suppose that cyclic monotonicity fails;
that is, there is a sequence of pairs $(x_1,y_1),\dots,(x_n,y_n)\in\spc{X}\times\spc{X}$ such that
\[\dist{x_1}{y_1}{}+\dots+\dist{x_n}{y_n}{}
>
\dist{x_1}{y_2}{}+\dots+\dist{x_{n-1}}{y_n}{}+\dist{x_{n}}{y_1}{}.\]
In this case, it would be more optimal to transport measure from a neighborhood of $x_i$ to a neighborhood of $y_{i+1}$ (
here and further we assume that the indexes are taken modulo $n$, so $n+1=1$).
The latter contradicts optimality of $\pi$.

The following argument makes it precise.
Choose small $\eps>0$.
For each $n$,
choose disjoint sets $X_i$ and $Y_i$ in $\eps$-neighborhood of $x_i$ and $y_i$
such that for some $\delta>0$ we have 
\[\pi [X_i\times Y_i]=\delta\]
for each $i$.

Let us modify the plan $\pi$ in the union $X_1\times Y_1 \cup\dots\cup X_n\times Y_n$ and such that 
$\pi'(X_i\times Y_{i+1})=\delta$ for each $i$;


Observe that
\[\int_{(x,y)\in\spc{X}\times\spc{X}}\dist{x}{y}{\spc{X}}\cdot(\pi'-\pi)>\]
\qeds


\appendix
\chapter{Semisolutions}
\parbf{\ref{ex:almost-min}.}
Assume the statement is wrong. 
Then for any point $x\in \spc{X}$, there is a point $x'\in \spc{X}$ such that 
\[\dist{x}{x'}{}< \rho(x)
\quad\text{and}\quad
\rho(x')\le\frac{\rho(x)}{1+\eps}.\]
Consider a sequence of points $(x_n)$ such that $x_{n+1}\z=x_n'$.
Clearly 
\[\dist{x_{n+1}}{x_n}{}
\le
\frac{\rho(x_0)}{\eps\cdot(1+\eps)^n}
\quad\hbox{and}\quad
\rho(x_n)\le \frac{\rho(x_0)}{(1+\eps)^n}.\] 
Therefore $(x_n)$ is a Cauchy sequence.
Since $\spc{X}$ is complete, the sequence $(x_n)$ converges;
denote its limit by $x_\infty$.
Since $\rho$ is a continuous function we get
\begin{align*}\rho(x_\infty)&=\lim_{n\to\infty}\rho(x_n)=
\\&=0.
\end{align*}

The latter contradicts that $\rho>0$.


\parbf{\ref{ex:non-contracting-map}.}
Given any pair of point $x_0,y_0\in \spc{K}$, 
consider two sequences $x_0,x_1,\dots$ and $y_0,y_1,\dots$
such that $x_{n+1}=f(x_n)$ and $y_{n+1}\z=f(y_n)$ for each $n$.

Since $\spc{K}$ is compact, 
we can choose an increasing sequence of integers $n_k$
such that both sequences $(x_{n_i})_{i=1}^\infty$ and $(y_{n_i})_{i=1}^\infty$
converge.
In particular, both are Cauchy;
that is,
\[
|x_{n_i}-x_{n_j}|_{\spc{K}}, |y_{n_i}-y_{n_j}|_{\spc{K}}\to 0
\quad
\text{as}
\quad\min\{i,j\}\to\infty.
\]


Since $f$ is non-contracting, we get
\[
|x_0-x_{|n_i-n_j|}|
\le 
|x_{n_i}-x_{n_j}|.
\]

It follows that  
there is a sequence $m_i\to\infty$ such that
\[
x_{m_i}\to x_0\quad\text{and}\quad y_{m_i}\to y_0\quad\text{as}\quad i\to\infty.
\leqno({*})\]

Set \[\ell_n=|x_n-y_n|_{\spc{K}}.\]
Since $f$ is non-contracting, the sequence $(\ell_n)$ is nondecreasing.

By $({*})$,  $\ell_{m_i}\to\ell_0$ as $m_i\to\infty$.
It follows that $(\ell_n)$ is a constant sequence.

In particular 
\[|x_0-y_0|_{\spc{K}}=\ell_0=\ell_1=|f(x_0)-f(y_0)|_{\spc{K}}\]
for any pair of points $(x_0,y_0)$ in $\spc{K}$.
That is, $f$ is distance-preserving, in particular injective.

From $({*})$, we also get that $f(\spc{K})$ is everywhere dense.
Since $\spc{K}$ is compact $f\:\spc{K}\to \spc{K}$ is surjective. Hence the result follows.

\parit{Remarks.}
This is a basic lemma in the introduction to Gromov--Hausdorff distance \cite[see 7.3.30 in][]{burago-burago-ivanov}.
This proof is not quite standard,
I learned this proof from Travis Morrison, 
a student in my MASS class at Penn State, Fall 2011.

Note that as an easy corollary one can see that any surjective non-expanding map from a compact metric space to itself is an isometry. 

\parbf{\ref{ex:pogorelov}.}
The conditions \ref{SHORT.metric>=0}--\ref{SHORT.metric:sym} in Definition \ref{def:metric} are evident.

The triangle inequality \ref{SHORT.metric:triangle} follows since
\[[B(x,\tfrac \pi2)\backslash B(y,\tfrac\pi2)]
\cup 
[B(y,\tfrac\pi2)\backslash B(z,\tfrac\pi2)]
\supseteq
B(x,\tfrac \pi2) \backslash B(z,\tfrac\pi2).
\leqno(*)\]

\begin{wrapfigure}[8]{o}{31 mm}
\vskip-2mm
\centering
\includegraphics{mppics/pic-29}
\end{wrapfigure}

Observe that
$B(x,\tfrac \pi2)\backslash B(y,\tfrac\pi2)$
does not overlap
$B(y,\tfrac\pi2)\backslash B(z,\tfrac\pi2)$ and  we get equality in $(*)$ if and only if $y$ lies on the great circle arc from $x$ to $z$.
Therefore the second statement follows.


\parit{Remarks.}
This construction was given by 
Aleksei Pogorelov \cite{pogorelov}.
It is closely related to the construction given 
by David Hilbert in \cite{hilbert}
which was the motivating example for his 4-th problem.


\parbf{\ref{ex:no-geod}.}
We assume that the space is nontrivial, otherwise a one-point space is an example.

Consider the unit ball $(B,\rho_0)$
in the space $c_0$ of all sequences converging to zero equipped with the sup-norm.

Consider another metric $\rho_1$ which is different from $\rho_0$ by the conformal factor
\[\phi(\bm{x})=2+\tfrac{1}2\cdot x_1+\tfrac{1}4\cdot x_2+\tfrac{1}8\cdot x_3+\dots,\]
where $\bm{x}=(x_1,x_2\,\dots)\in B$.
That is, if $\bm{x}(t)$, $t\in[0,\ell]$, is a curve parametrized by $\rho_0$-length 
then its $\rho_1$-length is defined by
\[\length_{\rho_1}\bm{x}\df\int\limits_0^\ell\phi\circ\bm{x}(t)\cdot dt.\]
Note that the metric $\rho_1$ is bi-Lipschitz to~$\rho_0$.

Assume $\bm{x}(t)$ and $\bm{x}'(t)$ are two curves parametrized by $\rho_0$-length that differ only in the $m$-th coordinate, denoted by $x_m(t)$ and $x_m'(t)$ respectively.
Note that if $x'_m(t)\le x_m(t)$ for any $t$ and 
the function $x'_m(t)$ is locally $1$-Lipschitz at all $t$ such that $x'_m(t)< x_m(t)$, then 
\[\length_{\rho_1}\bm{x}'\le \length_{\rho_1}\bm{x}.\]
Moreover this inequality is strict if $x'_m(t)< x_m(t)$ for some~$t$.

Fix a curve $\bm{x}(t)$, $t\in[0,\ell]$, parametrized by  $\rho_0$-length.
We can choose $m$ large, so that $x_m(t)$ is sufficiently close to $0$ for any~$t$.
In particular, for some values $t$, we have $y_m(t)<x_m(t)$, where
\[y_m(t)=(1-\tfrac t\ell)\cdot x_m(0)
+\tfrac t\ell\cdot x_m(\ell)
-\tfrac 1{100}\cdot \min\{t,\ell-t\}.\]
Consider the curve $\bm{x}'(t)$ as above with
\[x'_m(t)=\min\{x_m(t),y_m(t)\}.\]
Note that $\bm{x}'(t)$ and $\bm{x}(t)$ have the same end points, and by the above
\[\length_{\rho_1}\bm{x}'<\length_{\rho_1}\bm{x}.\]
That is, for any curve $\bm{x}(t)$ in $(B,\rho_1)$, we can find a shorter curve $\bm{x}'(t)$ with the same end points.
In particular, $(B,\rho_1)$ has no geodesics.

\parit{Remarks.}
This solution was suggested by Fedor Nazarov~\cite{nazarov}.

\parbf{\ref{ex:compact+connceted}.}
Choose a sequence $\varepsilon_n\to 0$ and a $\varepsilon_n$-net $N_n$ of $K$ for each $n$.
Assume $N_0$ is a one-point set, so $\eps_0>\diam K$.
Connect each point $x\in N_{k+1}$ to a point $y\in N_{k}$ by a curve of length at most $\eps_k$.

Consider the union $K'$ of all these curves with $K$; observe that $K'$ is compact and path connected.

\parit{Source:} This problem was suggested by Eugene Bilokopytov \cite{bilokopytov}.

\parbf{\ref{ex:compact=>complete}.}
Choose a Cauchy sequence $(x_n)$ in $(\spc{X},\|*-*\|)$; it sufficient to show that a subsequence of $(x_n)$ converges.

Note that the sequence $(x_n)$ is Cauchy in $(\spc{X},|*-*|)$;
denote its limit by $x_\infty$.

After passing to a subsequence, we can assume that $\|x_n-x_{n+1}\|\z<\tfrac1{2^n}$.
It follows that there is a 1-Lipschitz path $\gamma$ in $(\spc{X},\|*-*\|)$ such that $x_n=\gamma(\tfrac1{2^n})$ for each $n$ and $x_\infty=\gamma(0)$.

It follows that
\begin{align*}
\|x_\infty-x_n\|&\le \length\gamma|_{[0,\frac1{2^n}]}\le
\\
&\le \tfrac1{2^n}.
\end{align*}
In particular $x_n$ converges.

\parit{Source:} \cite[Lemma 2.3]{petrunin-stadler}.


\begin{wrapfigure}{r}{20 mm}
\vskip-0mm
\centering
\includegraphics{mppics/pic-1}
\end{wrapfigure}

\parbf{\ref{exercise from BH}.}
Consider the following subset of $\RR^2$ equipped with the induced length metric
\[
\spc{X}
=
\bigl((0,1]\times\{0,1\}\bigr)
\cup
\bigl(\{1,\tfrac12,\tfrac13,\dots\}\times[0,1]\bigr)
\]
Note that $\spc{X}$ is locally compact and geodesic.

Its completion $\bar{\spc{X}}$ is isometric to the closure of $\spc{X}$ equipped with the induced length metric.
Note that $\bar{\spc{X}}$ is obtained from $\spc{X}$ by adding two points $p=(0,0)$ and $q=(0,1)$.

Observe that the point $p$ admits no compact neighborhood in $\bar{\spc{X}}$ 
and there is no geodesic connecting $p$ to $q$ in~$\bar{\spc{X}}$. 

\parit{Source:} \cite[I.3.6(4)]{bridson-haefliger}.

%%%%%%%%%%%%%%%%%%%%%%%%%%%%%%



\parbf{\ref{ex:gross}.}
If such a number does not exist, then the ranges of average distance functions have empty intersection.
Since $\spc{X}$ is a compact length-metric space, the range of any continuous function on $\spc{X}$ is a closed interval.
By 1-dimensional Helly's theorem, there is a pair of such range intervals that do not intersect.
That is, for two point-arrays $(x_1,\dots,x_n)$ and $(y_1,\dots,y_m)$
and their average distance functions 
\[f(z)=\tfrac1n\cdot\sum_i|x_i-z|_{\spc{X}}\quad\text{and}\quad h(z)=\tfrac1m\cdot\sum_j|y_j-z|_{\spc{X}},\] we have 
$$\min\set{f(z)}{z\in \spc{X}}>\max\set{h(z)}{z\in \spc{X}}.\leqno({*})$$

Note that 
$$\tfrac1m\cdot\sum_j f(y_j)=\tfrac1{m\cdot n}\cdot\sum_{i,j}|x_i-y_j|_{\spc{X}}=\tfrac1n\cdot\sum_i h(x_i);$$
that is, the average value of $f(y_j)$ coincides with the average value of $h(x_i)$, 
which contradicts $({*})$.

\parit{Remarks.}
In fact the value $\ell$ is uniquely defined;
it is called the \index{rendezvous value}\emph{rendezvous value} of ${\spc{X}}$.
This is a result of Oliver Gross \cite{gross}.

\parbf{\ref{ex:wasserstein}.}
Choose a finite $\eps$-net $F\subset\spc{X}$.
Show that the space $P_F$ of probability measures with support in $F$ is a compact net in $\Wass_1\spc{X}$.
Observe that $\Wass_1\spc{X}$ is complete; 
by \ref{ex:compact-net}, it follows that $\Wass_1\spc{X}$ is compact.

Show that 

Choose an integer $n$.
Consider the set of probability measures $P_n$ of the form 
\[\tfrac1n\cdot\sum_{i=1}^n\delta_{x_i},\]
where $\delta_{x_i}$ denotes the delta-measure supported at $x_i\in\spc{X}$. 

Show that $P_n$ is a compact subset of $\Wass_1\spc{X}$.
Moreover, for any $\eps>0$ there is $n$ such that $P_n$ is 




%%%%%%%%%%%%%%%%%%%%%%%%%

\parbf{\ref{ex:compact-length}.} By Fréchet lemma (\ref{lem:frechet}) we can identify $\spc{K}$ with a compact subset of $\ell^\infty$.

Denote by $\spc{L}=\Conv\spc{K}$ --- it is defined as the minimal convex closed set in $\ell^\infty$ that contains $\spc{K}$.
(In other words, $\spc{L}$ is the intersection of all convex closed sets that contain $\spc{K}$.)

Observe that $\spc{L}$ is a length space.
It remains to show that $\spc{L}$ is compact.

By construction $\spc{L}$ is a closed subset of $\ell^\infty$, in particular it is a complete space.
By \ref{totally-bounded}, it remains to show that $\spc{L}$ is totally bounded.

Recall that Minkowski sum $A + B$ of two sets $A$ and $B$ in a vector space is defined by
\[A + B = \set{a+b}{a\in A,\ b\in B}.\]
Observe that the Minkowski sum of two convex sets is convex.

Denote by $\bar B_\eps$ the closed $\eps$-ball in $\ell^\infty$ centered at the origin.
Choose a finite $\eps$-net $N$ in $\spc{K}$ for some $\eps>0$.
Note that $P=\Conv N$ is a convex polyhedron, in particular $\Conv N$ is compact.

Observe that $N+\bar B_\eps$ is closed $\eps$-neighborhood of $N$.
It follows that $N+\bar B_\eps\supset K$ and therefore $P+\bar B_\eps\supset \spc{L}$.
In particular $P$ is a $2\cdot\eps$-net in $\spc{L}$;
since $P$ is compact and $\eps>0$ is arbitrary, $\spc{L}$ is totally bounded (see \ref{ex:compact-net}).

\parit{Remark.}
Another solution follows since the injective envelope of a compact space is compact; see \ref{ex:inj=complete-geodesic-contractible:geodesic}, \ref{ex:Inj(compact)}, and \ref{prop:InjX-is-injective}.

\parbf{\ref{ex:geodesics-urysohn}.}
Choose a separable space $\spc{X}$ that has an infinite number of geodesics between a pair of points, say a square will $\ell^\infty$-metric in $\RR^2$.
Apply to $\spc{X}$ universality of Urysohn space (\ref{prop:sep-in-urys}).

\parbf{\ref{ex:compact-extension}.} 
First let us prove the following claim:

\begin{itemize}
\item 
Suppose $f\: K\to\RR$ is an extension function defined on a compact subset $K$ of the Urysohn space $\spc{U}$.
Then there is a point $p\in \spc{U}$ such that 
$\dist{p}{x}{}=f(x)$ for any $x\in K$.
\end{itemize}

Without loss of generality we may assume that $f(x)>0$ for any $x\in K$.
Since $K$ is compact, we may fix $\eps>0$ such that $f(x)>\eps$.

Consider the sequence $\eps_n=\tfrac\eps{100\cdot 2^n}$.
Choose a sequence of $\eps_n$-nets $N_n\subset K$.
Applying universality of $\spc{U}$ recursively, we may choose a point $p_n$ such that $\dist{p_n}{x}{}=f(x)$ for any $x\in N_n$ and $\dist{p_n}{p_{n-1}}{}\z=10\cdot\eps_{n-1}$.
Observe that the sequence $(p_n)$ is Cauchy and its limit $p$ meets 
$\dist{p}{x}{}=f(x)$ for any $x\in K$.

Now, choose a sequence of points $(x_n)$ in $\spc{S}$.
Applying the claim, we may extend the map from $K$ to $K\cup\{x_1\}$, and further to $K\cup\{x_1,x_2\}$, and so on.
As a result we extend the distance-preserving map $f$ to the whole sequence $(x_n)$.
It remains to extend it continuously to the whole space~$\spc{S}$.

\parbf{\ref{ex:sc-urysohn}.}
It is sufficient to show that any compact subspace $\spc{K}$ of Urysohn space can be contracted to a point.

Note that any compact space $\spc{K}$ can be extended to a contractible compact space $\spc{K}'$; for example we may embed $\spc{K}$ into $\ell^\infty$ and pass to its convex hull, as it was done in \ref{ex:compact-length}.

By \ref{thm:compact-homogeneous}, there is an isometric embedding of $\spc{K}'$ that agrees with inclusion of $\spc{K}$.
Since $\spc{K}$ is contractible in $\spc{K}'$, it is contractible in $\spc{U}$.

\parit{A better way.}
One can contract the whole Urysohn space using the following construction.

Note that points in the space $\spc{X}_\infty$ constructed in the proof of \ref{prop:univeral-separable} can be multiplied number $t\in [0,1]$ --- simply multiply each function by $t$.
That defines a map 
\[\lambda_t\:\spc{X}_\infty\to \spc{X}_\infty\]
that scales all distances by factor $t$.
The map $\lambda_t$ can be extended to the completion of $\spc{X}_\infty$, which is isometric to $\spc{U}_d$ (or $\spc{U}$).

Observe that 
the map $\lambda_1$ is the identity  and $\lambda_0$ maps whole space to a single point, say $x_0$ --- this is the only point of $\spc{X}_0$.
Further note that the map $(t,p)\mapsto \lambda_t(p)$ is continuous ---  in particular $\spc{U}_d$ and $\spc{U}$ are contractible.

As a bonus, observe that for any point $p\in \spc{U}_d$ the curve $t\mapsto \lambda_t(p)$ is a geodesic path from $p$ to $x_0$.

\parit{Source:} \cite[(d) on page 82]{gromov-2007}.

\parbf{\ref{ex:sphere-in-urysohn}}; \ref{SHORT.ex:sphere-in-urysohn:sphere} and \ref{SHORT.ex:sphere-in-urysohn:midpoint}.
Observe that $L$ and $M$ satisfy the definition of $d$-Urysohn space and apply the uniqueness (\ref{thm:urysohn-unique}).

\parit{\ref{SHORT.ex:sphere-in-urysohn:homogeneous}.} 
Use \ref{SHORT.ex:sphere-in-urysohn:sphere}, maybe twice.

\parbf{\ref{ex:homogeneous}}; \ref{SHORT.ex:homogeneous:euclidean}.
The euclidean plane is homogeneous in every sense.

\parit{\ref{SHORT.ex:homogeneous:hilbert}.} The ilbert space $\ell^2$ is finite set homogeneous, but not compact set homogeneous, nor countable homogeneous.

\parit{\ref{SHORT.ex:homogeneous:ell-infty}.} The space $\ell^\infty$ is 1-point homogeneous, but not 2-point homogeneous.
Try to show that there is no isometry of $\ell^\infty$ such that
\begin{align*}
(0,0,0,\dots)&\mapsto (0,0,0,\dots),
\\
(1,1,1,\dots)&\mapsto (1,0,0,\dots).
\end{align*}

\parit{\ref{SHORT.ex:homogeneous:ell-1}.}
The space $\ell^1$ is 1-point homogeneous, but not 2-point homogeneous.
Try to show that there is no isometry of $\ell^\infty$ such that
\begin{align*}
(0,0,0,\dots)&\mapsto (0,0,0,\dots),
\\
(2,0,0\dots)&\mapsto (1,1,0,\dots).
\end{align*}

\parbf{\ref{ex:+-c}.}
Note that if $c<0$, then $r>s$.
The latter is impossible since $r$ is extremal and $s$ is admissible.

Observe that the function $\bar r=\min\{\,r,s+c\}$ is admissible.
Indeed if $\bar r(x)=r(x)$ and $\bar r(y)=r(y)$ then 
\[\bar r(x)+\bar r(y)=r(x)+ r(y)\ge \dist{x}{y}{}.\]
Further if $\bar r(x)=s(x)+c$ then 
\begin{align*}
\bar r(x)+\bar r(y)&\ge [s(x)+c]+ [s(y)-c]= 
\\
&=s(x)+s(y) \ge 
\\
&\ge\dist{x}{y}{}.
\end{align*}

Since $r$ is extremal, we have $r=\bar r$;
that is, $r\le s+c$.

\parbf{\ref{ex:inj=complete-geodesic-contractible}.}
Choose an injective space $\spc{Y}$.

\textit{\ref{SHORT.ex:inj=complete-geodesic-contractible:complete}.}
Fix a Cauchy sequence $(x_n)$ in $\spc{Y}$;
we need to show that it has a limit $x_\infty\in \spc{Y}$.
Consider metric on $\spc{X}=\NN\cup\{\infty\}$ defined by 
\begin{align*}
\dist{m}{n}{\spc{X}}&=\dist{x_m}{x_n}{\spc{Y}},
\\
\dist{m}{\infty}{\spc{X}}&=\lim_{n\to\infty}\dist{x_m}{x_n}{\spc{Y}}.
\end{align*}
Since the sequence is Cauchy, so is the sequence $\ell_n=\dist{p}{x_n}{\spc{Y}}$.
Therefore the last limit is defined.

By construction the map $n\mapsto x_n$ is distance-preserving on $\NN\subset \spc{X}$.
Since $\spc{Y}$ is injective, this map can be extended to $\infty$ as a short map; set $\infty\mapsto x_\infty$.
Since $\dist{x_n}{x_\infty}{\spc{Y}}\le \dist{n}{\infty}{\spc{X}}$ 
and $\dist{n}{\infty}{\spc{X}}\to 0$, we get that
$x_n\to x_\infty$ as $n\to\infty$.

\textit{\ref{SHORT.ex:inj=complete-geodesic-contractible:geodesic}.}
Applying the definition of injective space, we get a midpoint for any pair of points in $\spc{Y}$.
By \ref{SHORT.ex:inj=complete-geodesic-contractible:complete},
$\spc{Y}$ is a complete space.
It remains to apply \ref{lem:mid>geod:geod}.

\textit{\ref{SHORT.ex:inj=complete-geodesic-contractible:contractible}.}
Let $k\:\spc{Y}\hookrightarrow \ell^\infty(\spc{Y})$ be the Kuratowski embedding (\ref{lem:kuratowski}).
Observe that $\ell^\infty(\spc{Y})$ is contractible;
in particular, there is a homotopy $k_t\:\spc{Y}\hookrightarrow \ell^\infty(\spc{Y})$ such that $k_0=k$ and $k_1$ is a constant map.
(In fact one can take $k_t=(1-t)\cdot k$.)

Since $k$ is distance-preserving and $\spc{Y}$ is injective,
there is a short map $f\:\ell^\infty(\spc{Y})\to \spc{Y}$ such that the composition $f\circ k$ is the identity map on $\spc{Y}$.
The composition $f\circ k_t\:\spc{Y}\hookrightarrow \spc{Y}$ is a needed homotopy. 

\parbf{\ref{ex:injective-spaces}.}
Suppose that a short map $f\:A\to\spc{Y}$ is defined on a subset $A$ of a metric space $\spc{X}$.
We need to construct a short extension $F$ of $f$.

\parit{\ref{SHORT.ex:injective-spaces:R}.}
Suppose $\spc{Y}=\RR$.
Without loss of generality, we may assume that $A\ne\emptyset$, otherwise map whole $\spc{X}$ to a single point.
Set 
\[F(x)=\inf\set{f(a)-\dist{a}{x}{}}{a\in A}.\] 
Observe that $F$ is short and $F(a)=f(a)$ for any $a\in A$.

\parit{\ref{SHORT.ex:injective-spaces:tree}.}
Suppose  $\spc{Y}$ is a complete metric tree.
Fix points $p\in \spc{X}$ and $q\in\spc{Y}$.
Given a point $a\in A$,
let $x_a\in\cBall[f(a),\dist{a}{p}{}]$ be the point closest to $f(x)$.
Note that $x_a\in[q\,f(a)]$ and either $x_a=q$ or $x_a$ lies on distance $\dist{a}{p}{}$ from $f(a)$.

Note that the geodesics $[q\,x_a]$ are nested;
that is, for any $a,b\in A$ we have either $[q\,x_a]\subset [q\,x_b]$ or $[q\,x_b]\subset [q\,x_a]$.
Moreover, in the first case we have $\dist{x_b}{f(a)}{}\le \dist{p}{a}{}$ and in the second $\dist{x_a}{f(b)}{}\le \dist{p}{b}{}$.

It follows that the closure of the union of all geodesics $[q\,x_a]$ for $a\in\spc{A}$ is a geodesic.
Denote by $x$ its endpoint; it exists since $\spc{Y}$ is complete.
It remains to observe that $\dist{x}{f(a)}{}\le \dist{p}{a}{}$ for any $a\in\spc{A}$;
that is, one can take $f(p)=x$.

\parbf{\ref{ex:ultrametric}.}
Choose three points $x,y,z\in\spc{X}$ and set $\spc{A}=\{x,z\}$.
Let $f\:\spc{A}\to \spc{Y}$ be an isometry.
Then $F(y)=f(x)$ or $F(y)=f(z)$.
If  $f(y)=f(x)$, then
\begin{align*}
\dist{y}{z}{\spc{X}}&\ge  \dist{F(y)}{f(z)}{\spc{Y}}=
\\
 &=\dist{x}{z}{\spc{X}}.
\end{align*}
Analogously if $f(y)=f(z)$, then $\dist{x}{y}{\spc{X}}\ge\dist{x}{z}{\spc{X}}$.

It remains to observe that the strong triangle inequality holds in both cases.

\parit{\ref{SHORT.ex:injective-spaces:ell-infty}.}
In this case $\spc{Y}=(\RR^2,\ell^\infty)$.
Note that the map $\spc{X}\to (\RR^2,\ell^\infty)$ is short if and only if both of its coordinate projections are short.
It remains to apply \ref{SHORT.ex:injective-spaces:R}.

\parbf{\ref{ex:tripod+square}}; \ref{SHORT.ex:tripod+square:tripod}.
Let $f$ be an extremal function.
Observe that at least two of the numbers $f(a)+f(b)$, $f(b)+f(c)$, and $f(c)+f(a)$ are $1$.
It follows that for some $x\in[0,\tfrac12]$, we have 
\begin{align*}
f(a)&=1\pm x,&
f(b)&=1\pm x,&
f(c)&=1\pm x,
\end{align*}
where we have one ``minus'' and two ``pluses'' in these three formulas.

Suppose that
\begin{align*}
g(a)&=1\pm y,& g(b)&=1\pm y,& g(c)&=1\pm y
\end{align*}
is another extremal function.
Then $|f-g|\z=|x-y|$ if $g$ has ``minus'' at the same place as $f$ and $|f-g|=|x+y|$ otherwise.

\begin{wrapfigure}{o}{30 mm}
\vskip-0mm
\centering
\includegraphics{mppics/pic-3}
\bigskip
\includegraphics{mppics/pic-4}
\end{wrapfigure}

It follows that $\Inj\spc{X}$ is isometric to a tripod;
that is, $\Inj\spc{X}$ is formed by three segments of length $\tfrac12$ glued at one end.

\parit{\ref{SHORT.ex:tripod+square:square}.}
Assume $f$ is an extremal function.
Observe that 
$f(x)+f(y)=f(p)+f(q)=2$;
in particular, two values $a=f(x)-1$ and $b=f(p)-1$ completely describe the function $f$.
Since $f$ is extremal, we also have that 
\[(1\pm a)+(1\pm b)\ge 1\]
for all 4 choices of signs;
that is, $|a|+|b|\le 1$.

It follows that $\Inj\spc{X}$ is isometric to the rhombus $|a|+|b|\le 1$ in the $(a,b)$-plane with the metric induced by the $\ell^\infty$-norm.





\parbf{\ref{ex:4-on-a-line}.}
Recall that 
\[\dist{f}{g}{\Inj\spc{X}}=\sup\set{|f(x)-g(x)|}{x\in\spc{X}}\]
and 
\[\dist{f}{p}{\Inj\spc{X}}=f(p)\]
for any $f,g\in \Inj\spc{X}$ and $p\in \spc{X}$.

Since $\spc{X}$ is compact we can find a point $p\in\spc{X}$ such that 
\[\dist{f}{g}{\Inj\spc{X}}=|f(p)-g(p)|=\left|\dist{f}{p}{\Inj\spc{X}}-\dist{g}{p}{\Inj\spc{X}}\right|.\]
Without loss of generality we may assume that 
\[\dist{f}{p}{\Inj\spc{X}}
=
\dist{g}{p}{\Inj\spc{X}}
+
\dist{f}{g}{\Inj\spc{X}}.\]
Applying \ref{lem:opposite}, we can find a point $q\in\spc{X}$ such that 
\[\dist{q}{p}{\Inj\spc{X}}
=
\dist{f}{p}{\Inj\spc{X}}
+
\dist{f}{q}{\Inj\spc{X}},\]
whence the result.

Since $\Inj\spc{X}$ is injective (\ref{prop:InjX-is-injective}), by \ref{ex:inj=complete-geodesic-contractible:geodesic} it has to be geodesic. It remains to note that the concatenation of geodesics $[pq]$, $[gf]$, and $[fq]$ forms a required geodesic $[pq]$.



\parbf{\ref{ex:Hausdorff-bry}}; \ref{SHORT.ex:Hausdorff-bry:conv}.
Denote by $X_r$ the $r$ neighborhood of a set $X\z\subset \RR^2$.
Observe  that 
\[(\Conv X)_r=\Conv(X_r),\]
and try to use it.

\parit{\ref{SHORT.ex:Hausdorff-bry:bry}.}
The answer is ``no'' in both parts.

For the first part let $X$ be a unit disc and $Y$ a finite $\eps$-net in $X$.
Evidently $|X-Y|_{\Haus\RR^2}<\eps$, 
but
$|\partial X-\partial Y|_{\Haus\RR^2}\approx 1$.

For the second part take $X$ to be a unit disc and $Y=\partial X$ to be its boundary circle.
Note that $\partial X=\partial Y$ in particular $\dist{\partial X}{\partial Y}{\Haus\RR^2}=0$ while $\dist{ X}{ Y}{\Haus\RR^2}=1$.

\parit{Remark.}
A more interesting example for the second part can be build on {}\emph{lakes of Wada} --- and example of three open bounded topological disks in the plane that have identical boundary.

\parbf{\ref{ex:Huas-perimeter-area}.}
Let $A$ be a compact convex set in the plane.
Denote by $A^r$ the closed $r$-neighborhood of $A$.
Recall that by Steiner's formula we have
\[\area A^r=\area A+r\cdot\perim A+\pi\cdot r^2.\]
Taking derivative and applying coarea formula, we get 
\[\perim A^r=\perim A+2\cdot\pi\cdot r.\]

Observe that if $A$ lies in a compact set $B$ bounded by a closed curve, then 
\[\perim A\le \perim B.\]
Indeed the closest-point projection $\RR^2\to A$ is short and it maps $\partial B$ onto $\partial A$.

It remains to observe that if $A_n\to A_\infty$, then for any $r>0$ we have that
\[A_\infty^r\supset A_n
\quad\text{and}\quad
A_\infty\subset A_n^r\]
for all large $n$.

%%%%%%%%%%%%%%%%%%%%%%%%%%%%%%

\parbf{\ref{ex:GH-inj}.}
Let $\spc{U}$ be as in \ref{prop:GH=X+Y}.
Suppose that $|\spc{X}-\spc{Y}|_{\spc{U}}<\eps$;
we need to show that 
\[|\hat{\spc{X}}-\hat{\spc{Y}}|_{\GH}<2\cdot \eps.\]

Denote by $\hat{\spc{U}}$ the injective envelop of $\spc{U}$.
Recall that $\spc{U}$, $\spc{X}$, and $\spc{Y}$ can be considered as subspaces of $\hat{\spc{U}}$, $\hat{\spc{X}}$, and $\hat{\spc{Y}}$ respectively.

According to \ref{ex:d-p-inclusion}, the inclusions $\spc{X}\hookrightarrow\spc{U}$ and $\spc{Y}\hookrightarrow\spc{U}$ can be extended to distance preserving inclusions $\hat{\spc{X}}\hookrightarrow\hat{\spc{U}}$ and $\hat{\spc{Y}}\hookrightarrow\hat{\spc{U}}$.
Therefore we can and will consider  $\hat{\spc{X}}$ and $\hat{\spc{Y}}$ as subspaces of $\hat{\spc{U}}$.


Given $f\in \hat{\spc{U}}$,
let us find $g\in \hat{\spc{X}}$ such that 
\[|f(u)-g(u)|<2\cdot\eps\eqlbl{|g-f|}\]
for any $u\in\spc{U}$.
Note that the restriction $f|_{\spc{X}}$ is admissible on ${\spc{X}}$.
By \ref{obs:extremal:below}, there is $g\in \hat{\spc{X}}$ such that 
\[g(x)\le f(x)\eqlbl{g(x)=<f(x)}\]
for any $x\in\spc{X}$.


Recall that any extremal function is $1$-Lipschitz;
in particular $f$ and $g$ are $1$-Lipschitz on $\spc{U}$.
Therefore \ref{g(x)=<f(x)} and $|\spc{X}-\spc{Y}|_{\spc{U}}<\eps$ imply that
\[g(u)< f(u)+2\cdot \eps\]
for any $u\in\spc{U}$.
By \ref{ex:+-c}, we also have 
\[g(u)> f(u)-2\cdot \eps\]
for any $u\in\spc{U}$.
Whence \ref{|g-f|} follows.

It follows that $\hat{\spc{Y}}$ lies in a $2\cdot\eps$-neighborhood of $\hat{\spc{X}}$ in $\hat{\spc{U}}$.
The same way we show that $\hat{\spc{X}}$ lies in a $2\cdot\eps$-neighborhood of $\hat{\spc{Y}}$ in $\hat{\spc{U}}$.
The later means that
$|\hat{\spc{X}}-\hat{\spc{Y}}|_{\Haus\spc{U}}<2\cdot\eps$,
and therefore
$|\hat{\spc{X}}-\hat{\spc{Y}}|_{\GH}<2\cdot\eps$.

\parit{Comment.} 
This problem was discussed by Urs Lang, Maël Pavón, and Roger Züst \cite[3.1]{lang-pavon-zust}.
\begin{figure}[h!]
\vskip-0mm
\centering
\includegraphics{mppics/pic-505}
\end{figure}
They also show that the constant 2 is optimal.
To see this look at the injective envelops of two 4-point metric spaces shown on the diagram and observe that the Gromov--Hausdorff distance between the 4-point metric spaces is 1, while the distance between their injective envelops approaches 2 as $s\to\infty$. 




\parbf{\ref{ex:GH-po:a}.}
In order to check that $\dist{*}{*}{\GH'}$ is a metric, it is sufficient to show that
\[\dist{\spc{X}}{\spc{Y}}{\GH'}=0 
\quad\Longrightarrow\quad
\spc{X}\iso\spc{Y};\]
the remaining conditions are trivial.

If $\dist{\spc{X}}{\spc{Y}}{\GH'}=0$, then there is a sequence of maps $f_n\:\spc{X}\to \spc{Y}$ such that 
\[\dist{f_n(x)}{f_n(x')}{\spc{Y}}\ge \dist{x}{x'}{\spc{X}}-\tfrac1n.\]

Choose a countable dense set $S$ in $\spc{X}$.
Passing to a subsequence of $f_n$ we can assume that $f_n(x)$ converges for any $x\in S$ as $n\to\infty$;
denote its limit by $f_\infty(x)$.

For each point $x\in\spc{X}$ choose a sequence $x_m\in S$ converging to $x$.
Since $\spc{Y}$ is compact, we can assume in addition that $y_m=f_\infty(x_m)$ converges in $\spc{Y}$.
Set $f_\infty(x)=y$.
Note that the map $f_\infty\:\spc{X}\to \spc{Y}$ is  distance-nondecreasing.

The same way we can construct a distance-nondecreasing map 
$g_\infty\:\spc{Y}\to \spc{X}$.

By \ref{ex:non-contracting-map}, the compositions $f_\infty\circ g_\infty\:\spc{Y}\to \spc{Y}$ and $g_\infty\z\circ f_\infty\:\spc{X}\to \spc{X}$ are isometries.
Therefore $f_\infty$ and $g_\infty$ are isometries as well.

%Observe that 
%$$|\spc{X}_n-\spc{X}_\infty|_{\mathcal{M}}\to 0 
%\quad\Longrightarrow\quad 
%\dist{\spc{X}_n}{\spc{X}_\infty}{\GH'}\to 0$$
%follows from Proposition~\ref{prop:GH-e-isom} and Exercise~\ref{ex:alm-isom:inverse}.
%To prove that 
%$$|\spc{X}_n-\spc{X}_\infty|_{\mathcal{M}}\to 0 
%\quad\Longleftarrow\quad 
%\dist{\spc{X}_n}{\spc{X}_\infty}{\GH'}\to 0,$$
%Suppose that $f_n\:\spc{X}_n\to\spc{X}_\infty$ and $g_n\:\spc{X}_\infty\to\spc{X}_n$ are $\eps_n$-almost distance-nondecreasing maps for $\eps_n\to 0$.
%Arguing as above, pass to a partial limit $h$ of the sequence $f_n\circ g_n\:\spc{X}_\infty\to\spc{X}_\infty$.
%Note that $h$ is a distance non-deceasing map from a compact space to an itself.
%By Exercise~\ref{ex:non-contracting-map}, $h$ is an isometry.


\parbf{\ref{pr:doubling}.}
Choose a space $\spc{X}$ in $\spc{Q}(C,D)$, denote a $C$-doubling measure by $\mu$.
Without loss of generality we may assume that $\mu(\spc{X})\z=1$.

The doubling condition implies that 
\[\mu[\oBall(p,\tfrac D{2^n})]\ge\tfrac 1{C^n}\]
for any point $x\in \spc{X}$.
It follows that 
\[\pack_{\frac D{2^n}}\spc{X}\le C^n.\]

By \ref{ex:pack-net}, for any $\eps\ge\frac D{2^{n-1}}$, the space $\spc{X}$ admits an $\eps$-net with at most $C^n$ points.
Whence $\spc{Q}(C,D)$ is uniformly totally bounded.

\parbf{\ref{pr:under}.} 
Since $\spc{Y}$ is compact, it has a finite $\eps$-net for any $\eps>0$.
For each $\eps>0$ choose a finite $\eps$-net $\{y_1,\dots,y_{n_\eps}\}$ in $\spc{Y}$.

Suppose $f\:\spc{X}\to \spc{Y}$ be a distance-nondecreasing map.
Choose one point $x_i$ in each nonempty subset $B_i=f^{-1}[\oBall(y_i,\eps)]$.
Note that the subset $B_i$ has diameter at most $2\cdot \eps$ and 
\[\spc{X}=\bigcup_iB_i.\]
Therefore the set of points $\{x_i\}$ forms a $2\cdot\eps$ net in $\spc{X}$.
Whence \ref{SHORT.pr:under:if} follows.

\parit{\ref{SHORT.pr:under:only-if}.} Let $\spc{Q}$ be a uniformly totally bounded family of spaces. 
Suppose that each space in $\spc{Q}$ has an $\tfrac1{2^n}$-net with at most $M_n$ points; we may assume that $M_0=1$.

Consider the space $\spc{Y}$ of all infinite integer sequences $m_0,m_1,\dots$ such that $1\le m_n\le M_n$ for any $n$.
Given two sequences $(\ell_n)$, and $(m_n)$ of points in $\spc{Y}$, set 
\[\dist{(\ell_n)}{(m_n)}{\spc{Y}}=\tfrac C{2^{n}},\]
where $n$ is minimal index such that $\ell_n\ne m_n$ and $C$ is a positive constant.

Observe that $\spc{Y}$ is compact.
Indeed it is complete and the sequences constant starting from index $n$ form a finite $\tfrac C{2^{n}}$-net in $\spc{Y}$.

Given a space $\spc{X}$ in $\spc{Q}$,
choose a sequence of $\tfrac1{2^n}$ nets 
$N_n\subset\spc{X}$ for each natural $n$.
We can assume that $|N_n|\le M_n$; let us enumerate the points in $N_n$ by $\{1,\dots,M_n\}$.
Consider the map $f:\spc{X}\to\spc{Y}$ defined by $f:x\to (m_1(x),m_2(x),\dots)$ where $m_n(x)$ is a number of the point in $N_n$ that lies on the distance $<\tfrac1{2^n}$ from $x$.

If $\tfrac1{2^{n-2}}\ge \dist{x}{x'}{\spc{X}}>\tfrac1{2^{n-1}}$, then $m_n(x)\ne m_n(x')$.
It follows that $\dist{f(x)}{f(x')}{\spc{Y}}\ge \tfrac C{2^{n}}$.
In particular, if $C>10$, then 
\[\dist{f(x)}{f(x')}{\spc{Y}}\ge \dist{x}{x'}{\spc{X}}\]
for any $x,x'\in \spc{X}$.
That is, $f$ is a distance-nondecreasing map $\spc{X}\to \spc{Y}$.

\parbf{\ref{ex:GH-SC},} \ref{SHORT.ex:GH-SC:circle}.
Suppose $\spc{X}_n\GHto \spc{X}$ and $\spc{X}_n$ are simply connected length metric space.
It is sufficient to show that any nontrivial covering map $f\:\tilde{\spc{X}}\to \spc{X}$ corresponds to a nontrivial covering map $f_n\:\tilde{\spc{X}}_n\to \spc{X}_n$ for large $n$.

The latter can be constructed by covering $\spc{X}_n$ by small balls that lie close to sets in $\spc{X}$ evenly covered by $f$, prepare few copies of these sets and glue them the same way as the inverse images of the evenly covered sets in $\spc{X}$ glued to obtain $\tilde{\spc{X}}$.

\begin{wrapfigure}{r}{40 mm}
\vskip-0mm
\centering
\includegraphics{mppics/pic-2}
\end{wrapfigure}

\parit{\ref{SHORT.ex:GH-SC:nonsc-limit}.}
Let $\spc{V}$ be a cone over Hawaiian earring.
Consider the {}\emph{doubled cone} $\spc{W}$ --- two copies of $\spc{V}$ with glued base points earrings (see the diagram).

The space $\spc{W}$ can be equipped with length metric for example the induced length metric from the shown embedding.

Note that $\spc{V}$ is simply connected, but $\spc{W}$ is not --- it is a good exercise in topology.

If we delete from the earrings all small circles, then the obtained double cone becomes simply connected and it remains to be close to $\spc{W}$ in the sense of Gromov--Hausdorff.

\parit{Remark.}
Note that from part \ref{SHORT.ex:GH-SC:nonsc-limit}, the limit does not admit a nontrivial covering.
So if we define fundamental group right --- as the inverse image of groups of deck transformations for all its coverings, then one may say that Gromov--Hausdorff limit of simply connected length spaces is simply connected.

\parbf{\ref{ex:sphere-to-ball},}
\textit{\ref{SHORT.ex:sphere-to-ball:2}.}
Suppose that a metric on $\mathbb{S}^2$ is close to the disc $\DD^2$.
Note that $\mathbb{S}^2$ contains a circle $\gamma$ that is close to the boundary curve of $\DD^2$.
By Jordan curve theorem, $\gamma$ divides $\mathbb{S}^2$ into two discs, say $D_1$ and $D_2$.

By \ref{ex:GH-SC:nonsc-limit}, the Gromov--Hausdorff limit of $D_1$ and $D_2$ have to contain whole $\DD^2$, otherwise the limit would admit a nontrivial covering.
Consider points $p_1\in D_1$ and $p_2\in D_2$ that a close to the center of $\DD^2$.
On one hand the distance $\dist{p_1}{p_2}{n}$ have to be very small.
On the other hand, any curve from $p_1$ to $p_2$ must cross $\gamma$, so it has length about 2 at least --- a contradiction.



\parit{\ref{SHORT.ex:sphere-to-ball:3}.}
Make fine burrows in the standard 3-ball without changing its topology,
but at the same time come sufficiently close to any point in the ball.

Consider the \index{doubling}\emph{doubling} of the obtained ball along its boundary;
that is, two copies of the ball with identified corresponding points on their boundaries.
The obtained space is homeomorphic to $\mathbb{S}^3$.
Note that the burrows can be made 
so that the obtained space is sufficiently close to the original ball 
in the Gromov--Hausdorff metric.

\parit{Source:} \cite[Exercises 7.5.13 and 7.5.17]{burago-burago-ivanov}. 

%%%%%%%%%%%%%%%%%%%%%%%%%%%%%%%%

%%%%%%%%%%%%%%%%%%%%%%%%%%%%%%

\parbf{\ref{ex:prop:eps-isometry=isometry}.}
Suppose that  $f_n\:\spc{X}\to \spc{Y}$ is a $\tfrac1n$-isometry between compact spaces for each $n\in\NN$.
Consider the $\omega$-limit $f_\omega$ of~$f_n$,
\[f_\omega(x)=\lim_{n\to\omega}f_n(x);\]
according to \ref{prop:ultra/compact}, $f_\omega$ is defined.
Since 
\[|f_n(x)-f_n(x')|\lege |x-x'|\pm\tfrac1n\]

we get that 
\[|f_\omega(x)-f_\omega(x')|= |x-x'|\]
for any $x,x'\in \spc{X}$;
that is, $f_\omega$ is distance-preserving.
Further, since $f_n$ is a $\tfrac1n$-isometry,
for any $y\in \spc{Y}$ there is $x_m$ such that $|f_n(x_n)-y|\le \tfrac1n$.
Therefore,
\[f_\omega(x_\omega)=y,\]
where $x_\omega$ is the $\omega$-limit of $x_n$;
that is, $f_\omega$ is onto.
It follows that $f_\omega\:\spc{X}\to\spc{Y}$ is an isometry.

\parbf{\ref{ex:linear}.}
Choose a nonprincipal ultrafilter $\omega$ and set $L(\bm{s})=s_\omega$.
It remains to observe that $L$ is linear.

\parbf{\ref{ex:lim(tree)}.}
Let $\gamma$ be a path from $p$ to $q$ in a metric tree $\spc{T}$.
Assume that $\gamma$ passes thru a point $x$ on distance $\ell$ from $[pq]$.
Then 
\[\length\gamma\ge \dist{p}{q}{}+2\cdot \ell.
\eqlbl{eq:+ell}\]

Suppose that $\spc{T}_n$ is a sequence of metric trees that $\omega$-converges to $\spc{T}_\omega$.
By \ref{obs:ultralimit-is-geodesic}, the space $\spc{T}_\omega$ is geodesic.

The uniqueness of geodesics follows from \ref{eq:+ell}.
Indeed, if for a geodesic $[p_\omega q_\omega]$ there is another geodesic $\gamma_\omega$ connecting its ends, then it has to pass thru a point $x_\omega\notin [p_\omega q_\omega]$.
Choose sequences $p_n,q_n,x_n\in\spc{T}_n$ such that $p_n\to p_\omega$, $q_n\to q_\omega$, and $x_n\to x_\omega$ as $n\to\omega$.
Then 
\begin{align*}
\dist{p_\omega}{q_\omega}{}&=\length\gamma\ge \lim_{n\to\omega}(\dist{p_n}{x_n}{}+\dist{q_n}{x_n}{})\ge
\\
&\ge \lim_{n\to\omega}(\dist{p_n}{q_n}{}+2\cdot\ell_n)=
\\
&=\dist{p_\omega}{q_\omega}{}+2\cdot\ell_\omega.
\end{align*}
Since $x_\omega\notin [p_\omega q_\omega]$, we have that $\ell_\omega>0$ --- a contradiction.

It remains to show that any geodesic triangle $\spc{T}_\omega$ is a tripod.
Consider the sequence of centers of tripods $m_n$ for a given sequences of points $x_n,y_n,z_n\in \spc{T}_n$.
Observe that its ultralimit $m_\omega$ is the center of the tripod with ends at $x_\omega,y_\omega,z_\omega\in \spc{T}_\omega$.

\parbf{\ref{ex:ultrapower}.}
Further, we consider $\spc{X}$ as a subset of $\spc{X}^\omega$.

\parit{\ref{SHORT.ex:ultrapower:a}.} Follows directly from the definitions.

\parit{\ref{SHORT.ex:ultrapower:compact}.}
Suppose $\spc{X}$ compact.
Given a sequence $x_n$ in $\spc{X}$, denote its $\omega$-limit in $\spc{X}^\omega$ by $x^\omega$ and its $\omega$-limit in $\spc{X}$ by $x_\omega$.

Observe that $x^\omega=\iota(x_\omega)$.
Therefore, $\iota$ is onto.

If $\spc{X}$ is not compact, we can choose a sequence $x_n$ such that $\dist{x_m}{x_n}{}>\eps$ for fixed $\eps>0$ and all $m\ne n$.
Observe that
\[\lim_{n\to\omega}\dist{x_n}{y}{\spc{X}}\ge \tfrac\eps2\]
for any $y\in\spc{X}$.
It follows that $x_\omega$ lies on the distance at least $\tfrac\eps2$ from~$\spc{X}$.

\parit{\ref{SHORT.ex:ultrapower:proper}.}
A sequence of points $x_n$ in $\spc{X}$ will be called $\omega$-bounded if there is a real constant $C$ such that
\[\dist{p}{x_n}{\spc{X}}\le C\] 
for $\omega$-almost all $n$.

The same argument as in \ref{SHORT.ex:ultrapower:compact} shows that any $\omega$-bounded sequence has its $\omega$-limit in $\spc{X}$.
Further, if $(x_n)$ is not  $\omega$-bounded, then 
\[\lim_{n\to\omega}\dist{p}{x_n}{\spc{X}}=\infty;\]
that is, $x_\omega$ does not lie in the metric component of $p$ in $\spc{X}^\omega$.

\parbf{\ref{ex:isom-ultrapower}.} Show and use that the spaces $\spc{X}^\omega$ and $(\spc{X}^\omega)^\omega$ have discrete metric and both have cardinality continuum.

\parbf{\ref{ex:two-geodesics-in-ultrapower}.}
Apply \ref{lem:X-X^w} and \ref{obs:ultrapower-is-geodesic}.

\parbf{\ref{ex:notproper-limit}.} Consider the infinite metric $\spc{T}$ tree with unit edges shown
\begin{figure}[h!]
\vskip-0mm
\centering
\includegraphics{mppics/pic-605}
\end{figure}
on the diagram. Observe that $\spc{T}$ is proper.

Consider the vertex $v_\omega=\lim_{n\to\omega}v_n$ in the ultrapower $\spc{T}^\omega$.
Observe that $\omega$ has an infinite degree.
Conclude that $\spc{T}^\omega$ is not locally compact.

\parbf{\ref{ex:ultraT}.} Consider a product space $[0,1]\times[0,\tfrac12]\times[0,\tfrac14]\times\dots$.

\parbf{\ref{ex:Asym(Lob)}}; \ref{SHORT.ex:Asym(Lob):metric-tree}.
Show that there is $\delta>0$ such that sides of any geodesic triangle intersect a disk of radius $\delta$.
Conclude that any geodesic triangle in $\Asym\spc{L}$ is a tripod.
Make a conclusion.

\parit{\ref{SHORT.ex:Asym(Lob):homogeneous}.} Observe that $L$ is one-point homogeneous and use it.

\parit{\ref{SHORT.ex:Asym(Lob):continuum}.} 
By \ref{SHORT.ex:Asym(Lob):homogeneous}, it is sufficient to show that $p_\omega$ has a continuum degree.

Choose distinct geodesics $\gamma_1,\gamma_2\:[0,\infty)\to L$ that start at a point $p$.
Show that the limits of $\gamma_1$ and $\gamma_2$ run in the different connected components of $(\Asym\spc{L})\setminus \{p_\omega\}$.
Since there is a continuum of distinct geodesics starting at $p$,
we get that the degree of $p_\omega$ is at least continuum.

On the other hand, the set of sequences of points in $L$  has cardinality continuum.
In particular, the set of points in $\Asym\spc{L}$ has cardinality at most continuum.
It follows that the degree of any vertex is at most continuum.

\parit{\ref{SHORT.ex:Asym(Lob):others}.}
The proof for the Lobachevsky space goes along the same lines.

For the infinite 3-regular tree, part \ref{SHORT.ex:Asym(Lob):metric-tree} follows from \ref{ex:lim(tree)}.
The 3-regular tree is not one-point homogeneous, but it is vertex homogeneous; the latter is sufficient to prove \ref{SHORT.ex:Asym(Lob):homogeneous}.
No changes are needed in \ref{SHORT.ex:Asym(Lob):continuum}.

\parit{Remark.}
Anna Dyubina and Iosif Polterovich \cite{dyubina-polterovich} proved that the properties \ref{SHORT.ex:Asym(Lob):homogeneous} and \ref{SHORT.ex:Asym(Lob):continuum} describe the tree $\spc{T}$ up to isometry.
In particular, the asymptotic space of the Lobachevsky plane does not depend on the choice of ultrafilter and the sequence $\lambda_n\to \infty$.


%%%%%%%%%%%%%%%%%%%%%%%%%%%%
{\small\sloppy
\documentclass[twoside]{book}

\usepackage{lectures}
\usepackage[colorlinks=true,
citecolor=black,
linkcolor=black,
anchorcolor=black,
filecolor=black,
menucolor=black,
urlcolor=black,
pdftitle={Pure metric geometry: introductory lectures},
pdfsubject={Geometry},
pdfauthor={Anton Petrunin}
]{hyperref}
\makeindex

\begin{document}
%\pagestyle{empty}\renewcommand\includegraphics[2][{}]{}\def\emph{\textit}
%\overfullrule=100mm

 
\title{Pure metric geometry:\\
introductory lectures}
\author{Anton Petrunin}
\date{}
\maketitle

We discuss only domestic affairs of metric spaces;
applications are given only as illustrations.

These notes are based on couses at PSU (Spring 2020) and SPbSU (Fall 2022).
An extended version can be found on the author's website;
it includes an introduction to Alexandrov geometry based on \cite{alexander-kapovitch-petrunin-2019} and metric geometry on manifolds \cite{petrunin2020mnfld} based on a simplified proof of Gromov's systolic inequality given by Alexander Nabutovsky~\cite{nabutovsky}.

A part of the text is a compilation from \cite{alexander-kapovitch-petrunin-2019, alexander-kapovitch-petrunin-2025, petrunin-yashinski, petrunin-2022-PIGTIKAL, petrunin-zamorabarrera} and its drafts.

I want to thank
Alexander Lytchak,
Julien Melleray,
and Sergio Zamora Barrera for help.
The present work is partially supported by NSF grant DMS-2005279
and by the Simons Foundation under grant \#584781.

\thispagestyle{empty}
\tableofcontents
\thispagestyle{empty}

\chapter{Definitions}

\section{Metric spaces}
\label{sec:metric spaces}


The distance between two points $x$ and $y$ in a metric space $\spc{X}$ will be denoted by $\dist{x}{y}{}$ or $\dist{x}{y}{\spc{X}}$.
The latter notation is used if we need to emphasize 
that the distance is taken in the space~${\spc{X}}$.

The function 
\[\distfun_x\:y\mapsto \dist{x}{y}{}\]
is called the \index{distance function}\emph{distance function} from~$x$. 

Given $R\in[0,\infty]$ and $x\in \spc{X}$, the sets
\begin{align*}
\oBall(x,R)&=\{y\in \spc{X}\mid \dist{x}{y}{}<R\},
\\
\cBall[x,R]&=\{y\in \spc{X}\mid \dist{x}{y}{}\le R\}
\end{align*}
are called, respectively, the  \index{open ball}\emph{open} and  the \index{closed ball}\emph{closed  balls}   of radius $R$ with center~$x$.
Again, if we need to emphasize that these balls are taken in the metric space $\spc{X}$,
we write 
\[\oBall(x,R)_{\spc{X}}\quad\text{and}\quad\cBall[x,R]_{\spc{X}}.\]


\section{Variations of definition}

Recall that a metric is a real-valued function $(x,y)\mapsto\dist{x}{y}{\spc{X}}$ that satisfies the following conditions for any three points $x,y,z\in \spc{X}$:
\begin{enumerate}[(i)]
\item $\dist{x}{y}{\spc{X}}\ge 0$,
\item\label{metric=0} $\dist{x}{y}{\spc{X}}= 0$ $\iff$ $x=y$,
\item $\dist{x}{y}{\spc{X}}=\dist{y}{x}{\spc{X}}$,
\item $\dist{x}{y}{\spc{X}}+\dist{y}{z}{\spc{X}}\ge\dist{x}{z}{\spc{X}}$,
\end{enumerate}

\parbf{Pseudometrics.}
A generalization of a metric in which the distance between two distinct points can be zero is called \emph{pseudometric}.
In other words, to define pseudometric, we need to remove condition (\ref{metric=0}) from the list.

The following two observations show that
nearly any question about pseudometric spaces can be reduced to a question about genuine metric spaces.

Assume $\spc{X}$ is a pseudometric space.
Set
$x\sim y$ if $\dist{x}{y}{}=0$. 
Note that if $x\sim x'$, then $\dist{y}{x}{}=\dist{y}{x'}{}$ for any $y\in\spc{X}$.
Thus, $\dist{*}{*}{}$ defines a metric on the
quotient set $\spc{X}/{\sim}$.
In this way we obtain a metric space $\spc{X}'$.
The space $\spc{X}'$ is called the 
\emph{corresponding metric space} for the pseudometric space $\spc{X}$.
Often we do not distinguish between $\spc{X}'$ and~$\spc{X}$. 

\parbf{$\bm{\infty}$-metrics.}
One may also consider metrics with values in $\RR\cup\{\infty\}$;
we might call them $\infty$-metrics or simply metrics.

Again nearly any question about $\infty$-metric spaces can be reduced to a question about genuine metric spaces. 

Indeed, set $x\approx y$ if and only if $\dist{x}{y}{}<\infty$;
this is an other equivalence relation on $\spc{X}$.
The equivalence class of a point $x\in\spc{X}$ will be called the \emph{metric component}\index{metric component} 
 of $x$; it will be denoted as $\spc{X}_x$.
One could think of $\spc{X}_x$ as  $\oBall(x,\infty)_{\spc{X}}$ --- the open ball centered at $x$ and radius $\infty$ in $\spc{X}$.

It follows that any $\infty$-metric space is a \emph{disjoint union} of genuine metric spaces --- the metric components of the original $\infty$-metric space.

\begin{thm}{Exercise}
Given two sets $A$ and $B$ on the plane, set 
\[\dist{A}{B}{}=\mu(A\backslash B)+\mu(B\backslash A),\]
where $\mu$ denotes the Lebesgue measure.
\begin{subthm}{}
Show that $\dist{*}{*}{}$ is a pseudometric on the set of bounded measurable sets of the plane.
\end{subthm}

\begin{subthm}{}
Show that $\dist{*}{*}{}$ is an $\infty$-metric on the set of all open sets of the plane.
\end{subthm}
\end{thm}

\section{Completeness}

Recall that a metric space $\spc{X}$ is called \emph{complete} if every Cauchy sequence of points in $\spc{X}$ converges in $\spc{X}$.

\begin{thm}{Exercise}\label{ex:almost-min}
Suppose that $\rho$ is a positive continuous function on a complete metric space $\spc{X}$.
Show that for any $\eps>0$ there is a point $x\in \spc{X}$ such that 
\[\rho(x)<(1+\eps)\cdot\rho(y)\]
for any point $y\in \oBall(x,\rho(x))$.
\end{thm}

Most of the time we will assume that a metric space is complete.
The following construction produces a complete metric space $\bar{\spc{X}}$ for any given metric space $\spc{X}$.
The space $\bar{\spc{X}}$ is called \emph{completion} of $\spc{X}$;
the original space $\spc{X}$ forms a dense subset in $\bar{\spc{X}}$.

\parbf{Completion.}
Given metric space $\spc{X}$, 
consider the set of all Cauchy sequences in $\spc{X}$.
Note that for any two Cauchy sequences $(x_n)$ and $(y_n)$ the right hand side in \ref{eq:cauchy-dist} is defined; moreover it defines a pseudometric on the set $\spc{C}$ of all Cauchy sequences
\[\dist{(x_n)}{(y_n)}{\spc{C}}\df\lim_{n\to\infty}\dist{x_n}{y_n}{\spc{X}}.\eqlbl{eq:cauchy-dist}\]
The corresponding metric space is called a completion of $\spc{X}$.

It is left as an exercise that completion of $\spc{X}$ is complete.

Note that for each point $x\in\spc{X}$ one can consider a constant sequence $x_n=x$ which is Cauchy.
It defines a natural map $\spc{X}\to \bar{\spc{X}}$.
It is easy to check that this map is distance preserving.
In partucular we can (and will) consider $\spc{X}$ as a subset of $\bar{\spc{X}}$.

\section{Compactness}

Let us recall few equivalent definitions of compact metric spaces.

\begin{thm}{Definition}\label{def:compact}
A metric space $\spc{K}$ is compact if and only if one of the following equivalent condition holds:

\begin{subthm}{}
 Every open cover of $\spc{K}$ has a finite subcover.
\end{subthm}

\begin{subthm}{}
 For any open cover of $\spc{K}$ there is $\eps>0$ such that any $\eps$-ball in $\spc{K}$ lie in one element of the cover. (The value $\eps$ is called Lebesgue number of the covering.)
\end{subthm}

\begin{subthm}{}
 Every sequence in $\spc{K}$ has a convergent subsequence.
\end{subthm}

\begin{subthm}{totally-bounded}
The space $\spc{K}$ is complete and \emph{totally bounded}; that is, for any $\eps>0$, the space $\spc{K}$ admits a finite cover by open $\eps$-balls.\footnote{Equivalently, for any $\eps>0$ there is a finite \emph{$\eps$-net}; that is a finite set of points $x_1,\dots,x_n\in \spc{K}$ such that any other point $x$ lies on the distance less than $\eps$ from one of $x_i$.}
\end{subthm}

\end{thm}

Let $\pack_\eps\spc{X}$ be exact upper bound on the number of points $x_1,\z\dots,x_n\in \spc{X}$ such that $\dist{x_i}{x_j}{}\ge\eps$ for any $i\ne j$.

If $n=\pack_\eps\spc{X}<\infty$, then
the collection of points $x_1,\dots,x_n$ is called a \emph{maximal $\eps$-packing}.
Note that $n$ is the maximal number of open disjoint $\tfrac\eps2$-balls in $\spc{X}$.

\begin{thm}{Exercise}\label{ex:pack-net}
Show that a complete space $\spc{X}$ is compact if and only of $\pack_\eps\spc{X}\z<\infty$ for any $\eps>0$.

Show that any maximal $\eps$-packing is an $\eps$-net.
\end{thm}


\begin{thm}{Exercise}\label{ex:non-contracting-map}
Let $\spc{K}$  be a compact metric space and
\[f\:\spc{K}\z\to \spc{K}\] 
be a distance non-decreasing map.
Prove that $f$ is an isometry.
\end{thm}


A metric space $\spc{X}$ is called \index{proper space}\emph{proper} if all closed bounded sets in $\spc{X}$ are compact. 
This condition is equivalent to each of the following statements:
\begin{itemize}
\item For some (and therefore any) point $p\in \spc{X}$ and any $R<\infty$, 
the closed ball $\cBall[p,R]_{\spc{X}}$ is compact. 
\item The function $\distfun_p\:\spc{X}\to\RR$ is proper for some (and therefore any) point $p\in \spc{X}$;
that is, for any compact set $K\subset \RR$, its inverse image 
\[\distfun_p^{-1}(K)=\set{x\in \spc{X}}{\dist{p}{x}{\spc{X}}\in K}\]
is compact.
\end{itemize}

A metric space $\spc{X}$ is called \emph{locally compact} if any point in $\spc{X}$ admits a compact neighborhood;
in other words, for any point $x\in\spc{X}$ a closed ball $\cBall[x,r]$ is compact for some $r>0$.

\section{Geodesics}
\label{sec:geods}

Let $\spc{X}$ be a metric space 
and $\II$\index{$\II$} a real interval. 
A~globally isometric map $\gamma\:\II\to \spc{X}$ is called a \index{geodesic}\emph{geodesic}%
\footnote{Various authors call it differently: {}\emph{shortest path}, {}\emph{minimizing geodesic}.}; 
in other words, $\gamma\:\II\to \spc{X}$ is a geodesic if 
\[\dist{\gamma(s)}{\gamma(t)}{\spc{X}}=|s-t|\]
for any pair $s,t\in \II$.

We say that  $\gamma\:\II\to \spc{X}$ is a geodesic from point $p$ to point $q$ if 
$\II=[a,b]$ and $p=\gamma(a)$, $q=\gamma(b)$.
In this case the image of $\gamma$ is denoted by $[p q]$\index{$[{*}{*}]$} and with an abuse of notations  we also call it a \index{geodesic}\emph{geodesic}.
Given a geodesic $[pq]$, we can parametrize it by distance to $p$;
this parametrization will be denoted by $\geod_{[p q]}(t)$.


We may write $[p q]_{\spc{X}}$ 
to emphasize that the geodesic $[p q]$ is in the space  ${\spc{X}}$.
We also use the following shortcut notation:
\begin{align*}
\left] p q \right[&=[pq]\backslash\{p,q\},
&
\left] p q \right]&=[pq]\backslash\{p\},
&
\left[ p q \right[&=[pq]\backslash\{q\}.
\end{align*}

In general, a geodesic from $p$ to $q$ need not exist and if it exists, it need not  be unique.  
However, once we write $[p q]$ we assume mean that we have made a choice of geodesic.

A metric space is called \index{geodesic}\emph{geodesic} if any pair of its points can be joined by a geodesic. 


A \index{geodesic path}\emph{geodesic path} is a geodesic with constant-speed parametrization by $[0,1]$.
Given a geodesic $[p q]$,
we denote by $\geodpath_{[pq]}$ the corresponding geodesic path;
that is,
$$\geodpath_{[pq]}(t)\z\df\geod_{[pq]}(t\cdot\dist[{{}}]{p}{q}{}).$$

A curve $\gamma\:\II\to \spc{X}$  is called a \index{geodesic!local geodesic}\emph{local geodesic} if for any $t\in\II$ there is a neighborhood $U$ of $t$ in $\II$ such that the restriction $\gamma|_U$ is a  geodesic.
A constant-speed parametrization of a local geodesic by the unit interval $[0,1]$ is called a \index{geodesic!local geodesic}\emph{local geodesic path}. 



\section{Length}

A \emph{curve} is defined as a continuous map from a real interval to a space.
If the real interval is $[0,1]$, then the curve is called a \emph{path}.

\begin{thm}{Definition}
Let $\spc{X}$ be a metric space and
$\alpha\: \II\to \spc{X}$ be a curve.
We define the \index{length}\emph{length} of $\alpha$ as 
\[
\length \alpha \df \sup_{t_0\le t_1\le\ldots\le t_n}\sum_i \dist{\alpha(t_i)}{\alpha(t_{i-1})}{}.
\]

A curve $\alpha$ is called \emph{rectifiable} if $\length \alpha<\infty$.
\end{thm}



\begin{thm}{Theorem}\label{thm:length-semicont}
Length is a lower semi-continuous with respect to pointwise convergence of curves. 

More precisely, assume that a sequence
of curves $\gamma_n\:\II\to \spc{X}$ in a metric space $\spc{X}$ converges pointwise 
to a curve $\gamma_\infty\:\II\to \spc{X}$;
that is, for any fixed $t \in \II$, $\gamma_n(t)\z\to\gamma_\infty(t)$ as $n\to\infty$. 
Then 
$$\liminf_{n\to\infty} \length\gamma_n \ge \length\gamma_\infty.\eqlbl{eq:semicont-length}$$
\end{thm}


\begin{wrapfigure}{o}{20 mm}
\vskip-0mm
\centering
\includegraphics{mppics/pic-10}
\end{wrapfigure}


Note that the inequality \ref{eq:semicont-length} might be strict.
For example the diagonal $\gamma_\infty$ of the unit square 
can be  approximated by a stairs-like
polygonal curves $\gamma_n$
with sides parallel to the sides of the square ($\gamma_6$ is on the picture).
In this case
\[\length\gamma_\infty=\sqrt{2}\quad
\text{and}\quad \length\gamma_n=2\]
for any $n$.

\parit{Proof.}
Fix a sequence $t_0<t_1<\dots<t_k$ in $\II$.
Set 
\begin{align*}\Sigma_n
&\df
|\gamma_n(t_0)-\gamma_n(t_1)|+\dots+|\gamma_n(t_{k-1})-\gamma_n(t_k)|.
\\
\Sigma_\infty
&\df
|\gamma_\infty(t_0)-\gamma_\infty(t_1)|+\dots+|\gamma_\infty(t_{k-1})-\gamma_\infty(t_k)|.
\end{align*}

Note that for each $i$ we have 
\[|\gamma_n(t_{i-1})-\gamma_n(t_i)|\to|\gamma_\infty(t_{i-1})-\gamma_\infty(t_i)|\]
and therefore
\[\Sigma_n\to \Sigma_\infty\] 
as $n\to\infty$.
Note that 
\[\Sigma_n\le\length\gamma_n\]
for each $n$.
Hence
$$\liminf_{n\to\infty} \length\gamma_n \ge \Sigma_\infty.\eqlbl{>=Sigma-infty}$$

If $\gamma_\infty$ is rectifiable, we can assume that 
\begin{align*}
\length\gamma_\infty<\Sigma_\infty+\eps.
\end{align*}
for any given $\eps>0$.
By \ref{>=Sigma-infty} it follows that 
$$\liminf_{n\to\infty} \length\gamma_n > \length\gamma_\infty-\eps$$
for any $\eps>0$; whence \ref{eq:semicont-length} follows.

It remains to consider the case when $\gamma_\infty$ is not rectifiable; 
that is, $\length\gamma_\infty=\infty$.
In this case we can choose a partition so that $\Sigma_\infty>L$ for any real number $L$.
By \ref{>=Sigma-infty} it follows that 
$$\liminf_{n\to\infty} \length\gamma_n > L$$
for any given $L$; whence 
\[\liminf_{n\to\infty}\length\gamma_n=\infty\]
and \ref{eq:semicont-length} follows.
\qeds

\section{Length spaces}\label{sec:intrinsic}

If for any $\eps>0$ and any pair of points $x$ and $y$ in a metric space $\spc{X}$, there is a path $\alpha$ connecting $x$ to $y$ such that
\[\length\alpha< \dist{x}{y}{}+\eps,\]
then $\spc{X}$ is called a \index{length space}\emph{length space} and the metric on $\spc{X}$ is called a \index{length metric}\emph{length metric}.\label{page:length metric}

Note that any geodesic space is a length space.
As can be seen from the following example, the converse does not hold.


\begin{thm}{Example}
Let $\spc{X}$ be obtained by gluing a countable collection of disjoint intervals $\{\II_n\}$ of length $1+\tfrac1n$, where for each $\II_n$ the left end is glued to $p$ and the right end to~$q$.

Observe that the space $\spc{X}$ carries a natural complete length metric with respect to which $\dist{p}{q}{}=1$ but there is no geodesic connecting $p$ to~$q$.
\end{thm}



\begin{thm}{Exercise}\label{ex:no-geod}
Give an example of a complete length space for which no pair of distinct points can be joined by a geodesic.
\end{thm}

Directly from the definition, it follows that if a path $\alpha\:[0,1]\to\spc{X}$ connects two points $x$ and $y$ 
(that is, if $\alpha(0)=x$ and $\alpha(1)=y$), then 
\[\length\alpha\ge \dist{x}{y}{}.\]
Set 
\[\yetdist{x}{y}{}=\inf\{\length\alpha\}\]
where the greatest lower bound is taken for all paths connecing $x$ and $y$.
It is straightforward to check that $(x,y)\mapsto \yetdist{x}{y}{}$ is an $\infty$-metric; moreover $(\spc{X},\yetdist{*}{*}{})$ is a length space.
The metric $\yetdist{*}{*}{}$ is called \emph{induced length metric}.

\begin{thm}{Exercise}\label{ex:compact=>complete}
Suppose $(\spc{X},\dist{*}{*}{})$ is a complete metric space.
Show that $(\spc{X},\yetdist{*}{*}{})$ is complete.
\end{thm}


Let $A$ be a subset of a metric space $\spc{X}$.
Given two points $x,y\in A$,
consider the value
\[\dist{x}{y}{A}=\inf_{\alpha}\{\length\alpha\},\]
where the greatest lower bound is taken for all paths $\alpha$ from $x$ to $y$ in $A$.%
\footnote{This notation slightly conflicts with the previously defined notation for distance $\dist{x}{y}{\spc{X}}$ in a metric space $\spc{X}$. However, most of the time we will work with ambient length spaces where the meaning will be unambiguous.}

Let $\spc{X}$ be a metric space and $x,y\in\spc{X}$.

\begin{enumerate}[(i)]
\item A point $z\in \spc{X}$ is called a \index{midpoint}\emph{midpoint} between $x$ and $y$
if 
\[\dist{x}{z}{}=\dist{y}{z}{}=\tfrac12\cdot\dist[{{}}]{x}{y}{}.\]
\item Assume $\eps\ge 0$.
A point $z\in \spc{X}$ is called an \index{$\eps$-midpoint}\emph{$\eps$-midpoint} between $x$ and $y$
if 
\[\dist{x}{z}{},\quad\dist{y}{z}{}\le\tfrac12\cdot\dist[{{}}]{x}{y}{}+\eps.\]
\end{enumerate}


Note that a $0$-midpoint is the same as a midpoint.


\begin{thm}{Lemma}\label{lem:mid>geod}
Let $\spc{X}$ be a complete metric space.
\begin{subthm}{lem:mid>length}
Assume that for any pair of points $x,y\in \spc{X}$  
 and any $\eps>0$
there is an $\eps$-midpoint~$z$.
Then $\spc{X}$ is a length space.
\end{subthm}

\begin{subthm}{lem:mid>geod:geod}
Assume that for any pair of points $x,y\in \spc{X}$, 
there is a midpoint~$z$.
Then $\spc{X}$ is a geodesic space.
\end{subthm}
\end{thm}

\parit{Proof.}
We first prove (\ref{SHORT.lem:mid>length}).
Let $x,y\in \spc{X}$ be a pair of points.

Set $\eps_n=\frac\eps{4^n}$, $\alpha(0)=x$ and $\alpha(1)=y$.

Let $\alpha(\tfrac12)$ be an $\eps_1$-midpoint between $\alpha(0)$ and $\alpha(1)$.
Further, let $\alpha(\frac14)$ 
and $\alpha(\frac34)$ be $\eps_2$-midpoints between the pairs $(\alpha(0),\alpha(\tfrac12))$ 
and $(\alpha(\tfrac12),\alpha(1))$ respectively.
Applying the above procedure recursively,
on the $n$-th step we define $\alpha(\tfrac{k}{2^n})$,
for every odd integer $k$ such that $0<\tfrac k{2^n}<1$, 
as an $\eps_{n}$-midpoint between the already defined
$\alpha(\tfrac{k-1}{2^n})$ and $\alpha(\tfrac{k+1}{2^n})$.


In this way we define $\alpha(t)$ for $t\in W$,
where $W$ denotes the set of dyadic rationals in $[0,1]$.
Since $\spc{X}$ is complete, the map $\alpha$ can be extended continuously to $[0,1]$.
Moreover,
\[\begin{aligned}
\length\alpha&\le \dist{x}{y}{}+\sum_{n=1}^\infty 2^{n-1}\cdot\eps_n\le
\\
&\le \dist{x}{y}{}+\tfrac\eps2.
\end{aligned}
\eqlbl{eq:eps-midpoint}
\]
Since $\eps>0$ is arbitrary, we get (\ref{SHORT.lem:mid>length}).

To prove (\ref{SHORT.lem:mid>geod:geod}), 
one should repeat the same argument 
taking midpoints instead of $\eps_n$-midpoints.
In this case \ref{eq:eps-midpoint} holds for $\eps_n=\eps=0$.
\qeds

Since in a compact space a sequence of $\tfrac1n$-midpoints $z_n$ contains a convergent subsequence, Lemma~\ref{lem:mid>geod} immediately implies

\begin{thm}{Proposition}\label{prop:length+proper=>geodesic}
A proper length space is geodesic.
\end{thm}

\begin{thm}{Hopf--Rinow theorem}\label{thm:Hopf-Rinow}
Any complete, locally compact length space is proper.
\end{thm}

It is instructive to solve the following exercise before reading the proof.

\begin{thm}{Exercise}
Give an example of space which is locally compact but not proper.
\end{thm}

\parit{Proof.}
Let $\spc{X}$ be a locally compact length space.
Given $x\in \spc{X}$, denote by $\rho(x)$ the supremum of all $R>0$ such that
the closed ball $\cBall[x,R]$ is compact.
Since $\spc{X}$ is locally compact, 
$$\rho(x)>0
\quad\text{for any}\quad
x\in \spc{X}.\eqlbl{eq:rho>0}$$
It is sufficient to show that $\rho(x)=\infty$ for some (and therefore any) point $x\in \spc{X}$.

Assume the contrary; that is, $\rho(x)<\infty$. We claim that

\begin{clm}{} $B=\cBall[x,\rho(x)]$ is compact for any~$x$.
\end{clm}

Indeed, $\spc{X}$ is a length space;
therefore for any $\eps>0$, 
the set $\cBall[x,\rho(x)-\eps]$ is a compact $\eps$-net in~$B$.
Since $B$ is closed and hence complete, it must be compact.
\claimqeds
Next we claim that
\begin{clm}{} $|\rho(x)-\rho(y)|\le \dist{x}{y}{\spc{X}}$ for any $x,y\in \spc{X}$;
in particular $\rho\:\spc{X}\to\RR$ is a continuous function.
\end{clm}

Indeed, 
assume the contrary; that is, $\rho(x)+|x-y|<\rho(y)$ for some $x,y\in \spc{X}$. 
Then 
$\cBall[x,\rho(x)+\eps]$ is a closed subset of $\cBall[y,\rho(y)]$ for some $\eps>0$.
Then  compactness of $\cBall[y,\rho(y)]$ implies compactness of $\cBall[x,\rho(x)+\eps]$, a contradiction.\claimqeds

Set $\eps=\min\set{\rho(y)}{y\in B}$; the minimum is defined since $B$ is compact.
From \ref{eq:rho>0}, we have $\eps>0$.

Choose a finite $\tfrac\eps{10}$-net $\{a_1,a_2,\dots,a_n\}$ in $B$.
The union $W$ of the closed balls $\cBall[a_i,\eps]$ is compact.
Clearly 
$\cBall[x,\rho(x)+\frac\eps{10}]\subset W$.
Therefore $\cBall[x,\rho(x)+\frac\eps{10}]$ is compact,
a contradiction.
\qeds

\begin{thm}{Exercise}\label{exercise from BH}
Construct a geodesic space that is locally compact,
but whose completion is neither geodesic nor locally compact.
\end{thm}

\section{Subsets in normed spaces}

Recall that a function $v\mapsto |v|$ on a vector space $\spc{V}$ is called \emph{norm} if it satisfies the following condition for any two vectors $v,w\in \spc{V}$ and a scalar $\alpha$:
\begin{itemize}
\item $|v|\ge 0$;
\item $|\alpha\cdot v|=|\alpha|\cdot |v|$;
\item $|v|+|w|\ge|v+w|$.
\end{itemize}

It is straightforward to check that for any normed space the function $(v,w)\mapsto |v-w|$ defines a metric on it.
Therefore any normed space is an example of metric space (which is in fact geodesic).
The following lemma says in particular that any metric space is isometric to a subset of a normed space.

\begin{thm}{Lemma}\label{lem:frechet}
Suppose $\spc{X}$ is a bounded separable space;
that is, $\diam\spc{X}$ is finite and $\spc{X}$ contains a countable, dense set $\{w_n\}$.
Given $x\in \spc{X}$, set $a_n(x)=\dist{w_n}{x}{\spc{X}}$.
Then 
\[\iota\:x\mapsto (a_1(x), a_2(x),\dots)\]
defines a distance preserving embedding $\iota\:\spc{X}\hookrightarrow \ell^\infty$.
\end{thm}

\parit{Proof.}
By the triangle inequality 
\[|a_n(x)-a_n(y)|\le \dist{x}{y}{\spc{X}}.\]
Therefore $\iota$ is short.

Again by triangle inequality we have 
\[|a_n(x)-a_n(y)|\ge \dist{x}{y}{\spc{X}}-2\cdot\dist{w_n}{x}{\spc{X}}.\]
Since the set $\{w_n\}$ is dense, we can choose $w_n$ arbitrary close to $x$.
Whence the value $|a_n(x)-a_n(y)|$ can be chosen arbitrary close to $\dist{x}{y}{\spc{X}}$.
In other words 
\[\sup_n\{\,|\dist{w_n}{x}{\spc{X}}-\dist{w_n}{y}{\spc{X}}|\,\}\ge \dist{x}{y}{\spc{X}};\]
hence $\iota$ is distance non-decreasing.
\qeds

The following exercise generalizes the lemma to arbitrary separable spaces.

\begin{thm}{Exercise}
Suppose $\{w_n\}$ is a countable, dense set in a metric space $\spc{X}$.
Choose $x_0\in \spc{X}$;
given $x\in \spc{X}$, set 
\[a_n(x)=\dist{w_n}{x}{\spc{X}}-\dist{w_n}{x_0}{\spc{X}}.\]
Show that $\iota\:x\mapsto (a_1(x), a_2(x),\dots)$ defines a distance preserving embedding $\iota\:\spc{X}\hookrightarrow \ell^\infty$.
\end{thm}


\begin{thm}{Exercise}\label{ex:compact-length}
Show that any compact metric space is isometric to a subspace of a compact geodesic space. 
\end{thm}

The lemma above was proved by Maurice René Fréchet in the paper where he defined metric space \cite{frechet}.
Nearly identical construction was rediscovered later by Kazimierz Kuratowski~\cite{kuratowski}.
Namely he made the following claim:

\begin{thm}{Lemma}\label{lem:kuratowski}
Let $\spc{X}$ be arbitrary metric space.
Denote by $\ell^\infty(\spc{X})$ the space of all bounded functions of $\spc{X}$ equipped with sup-norm.

Then for any point $x_0\in \spc{X}$, the map $\iota\:\spc{X}\to \ell^\infty(\spc{X})$ defied by 
\[\iota\:x\mapsto (\distfun_x-\distfun_{x_0})\]
is distance preserving.
\end{thm}

Note that this claim implies that \emph{any metric space is isometric to a subset of a normed vector space}.








\chapter{Urysohn space}

We discuss a construction introduced by Pavel Urysohn~\cite{urysohn}.
Our presentation is very close to the one in \cite{gromov-2007}.

This subject is closely related to the so called \emph{Rado graph},
also known as \emph{Erd\H{o}s–R\'enyi graph} or \emph{random graph}; a good survey this subject is written by Peter Cameron~\cite{cameron}.

\section{Existance}
Suppose a metric space $\spc{X}$ is a subspace of a pseudometric space $\spc{X}'$.
In this case we may say that $\spc{X}'$ is an \emph{extension} of $\spc{X}$.
If $\diam\spc{X}'\le d$, then we say that $\spc{X}'$ is a \emph{$d$-extension}.

If the complement $\spc{X}'\backslash \spc{X}$ contains a single point, say $p$, we say that $\spc{X}'$ is a \emph{one-point extension} of $\spc{X}$.
In this case, to define metric on $\spc{X}'$, it is sufficient to specify the distance function from $p$; that is, a function $f\:\spc{X}\to\RR$ defined by 
\[f(x)=\dist{p}{x}{\spc{X}'}.\]

The function $f$ can not be taken arbitrary --- the triangle inequality implies that 
\[f(x)+f(y)\ge \dist{x}{y}{\spc{X}}\ge |f(x)-f(y)|\]
for any $x,y\in \spc{X}$.
In particular $f$ is a non-negative 1-Lipschitz function on $\spc{X}$.
For a $d$-extension we need to assume in addition that $\diam\spc{X}\le d$ and $f(x)\le d$ for any $x\in \spc{X}$.

Any function $f$ of that type will be called \emph{extension function} or \emph{$d$-extension function} correspondingly.

\begin{thm}{Definition}\label{def:universal}
A metric space $\spc{U}$ is called \emph{universal}  if for any finite subspace $\spc{F}\subset\spc{U}$ and any extension function $f\:\spc{F}\to\RR$ there is a point $p\in \spc{U}$ such that $\dist{p}{x}{}=f(x)$ for any $x\in \spc{F}$.

If instead of extension functions we consider only $d$-extension functions and assume in addition that $\diam \spc{U}\le d$, then we arrive to a definition of \emph{$d$-universal space}.

If in addition $\spc{U}$ is separable and complete, then it is called \emph{Urysohn space} or \emph{$d$-Urysohn space}.
\end{thm}


\begin{thm}{Proposition}\label{prop:univeral-separable}
Given a positive $d$, there is a separable $d$-universal metric space.

Moreover, a separable universal space metric exists.
\end{thm}

\parit{Proof.}
Let $\spc{X}$ be a compact metric space such that $\diam\spc{X}\le d$.
Denote by $\spc{X}^d$ the space of all $d$-extension functions on $\spc{X}$ equipped with the metric defined by sup-norm.
Note that the map $\spc{X} \to \spc{X}^d$ defined by $x\mapsto\distfun_x$ is a distance preserving embedding,
so we can (and will) treat $\spc{X}$ as a subspace of $\spc{X}^d$, or, equivalently, $\spc{X}^d$ is an extension of $\spc{X}$.

Let us iterate this construction.
Start with a one-point space $\spc{X}_0$ and consider a sequence of spaces $(\spc{X}_n)$ defined by $\spc{X}_{n+1}\z=\spc{X}_n^d$.
Note that the sequence is nested, that is $\spc{X}_0\subset \spc{X}_1\subset\dots$
and the union
\[\spc{X}_\infty=\bigcup_n\spc{X}_n;\]
comes with metric such that
$\dist{x}{y}{\spc{X}_\infty} = \dist{x}{y}{\spc{X}_n}$
if $x,y\in\spc{X}_n$.

Note that if $\spc{X}$ is compact, then so is $\spc{X}^d$.
It follows that each space $\spc{X}_n$ is compact.
Since $\spc{X}_\infty$ is a countable union of compact spaces, it is separable.

Any finite subspace $\spc{F}$ of $\spc{X}_\infty$ lies in some $\spc{X}_n$ for $n<\infty$.
By construction, there is a point $p\in \spc{X}_{n+1}$ that meets the condition in Definiton~\ref{def:universal}.
That is, $\spc{X}_\infty$ is $d$-universal.

A construction of a universal separable metric space is done along the same lines, but one has the sequence should be defined by $\spc{X}_{n+1}\z=\spc{X}_n^{d_n}$ for some sequence $d_n\to\infty$.
\qeds

\begin{thm}{Proposition}\label{prop:completion-univeral}
A completion of $d$-universlal space is $d$-universal.

A completion of universal space universal.
\end{thm}

Note that \ref{prop:univeral-separable} and \ref{prop:completion-univeral} imply the following:

\begin{thm}{Theorem}\label{thm:urysohn-exists}
Urysohn space, and $d$-Urysohn space for any $d>0$, exist.
\end{thm}


\parit{Proof.} Suppose $\spc{V}$ be a $d$-universal space;
denote by $\spc{U}$ its completion; so $\spc{V}$ is a dense subset in a complete space $\spc{U}$.

Observe that $\spc{U}$ is \emph{approximately $d$-universal};
that is, if $\spc{F}\subset\spc{U}$ is a finite set, and $f\:\spc{F}\to \RR$ is a $d$-extension function, then
there exists $p\in \spc{U}$ such that
\[\dist{p}{x}{}\lg f(x)\pm\eps.\]
for any $x\in\spc{F}$.

Therefore there is a sequence of points $p_n\in \spc{U}$ such that for any $x\in \spc{F}$, 
\[\dist{p_n}{x}{}\lg f(x)\pm\tfrac1{2^n}.\]

Moreover, we can assume that 
\[\dist{p_n}{p_{n+1}}{} < \tfrac1{2^n}\eqlbl{eq:|pn-pn|}\]
for all large $n$.
Indeed, consider the sets $\spc{F}_n=\spc{F}\cup\{p_n\}$ and the functions $f_n$ defined by $f_n(x)=f(x)$ for any $x\in \spc{F}$, and
\[f_n(p_n)=\max\set{\bigl|\dist{p_n}{x}{}- f(x)\bigr|}{x\in \spc{F}}.\]
Observe that $f_n$ is a an $d$-extension function for large $n$ and
$f_n(p_n)\z<\tfrac1{2^n}$.
By applying approximate universal property recursively we get~\ref{eq:|pn-pn|}.

By \ref{eq:|pn-pn|}, $(p_n)$ is a Cauchy sequence and its limit meets the condition in the definition of universal space (\ref{def:universal}).
\qeds

\section{Universality}

\begin{thm}{Proposition}\label{prop:sep-in-urys}
Let $\spc{U}$ a Urysohn space.
Then any separable metric space $\spc{S}$ admits a distance preserving embedding $\spc{S}\hookrightarrow\spc{U}$.

Moreover, for any finite subspace $\spc{F}\subset \spc{S}$,
any distance preserving embedding $\spc{F}\hookrightarrow \spc{U}$ can be extended to an distance preserving embedding $\spc{S}\hookrightarrow\spc{U}$.

If $\spc{U}$ is $d$-Urysohn,
then the statements hold provided $\diam\spc{S}\le d$.  
\end{thm}

\parit{Proof.}
The required isometry will be denoted by $x\mapsto x'$.

Choose a dense sequence of points $s_1,s_2,\dotsc\in\spc{S}$.
We may assume that $\spc{F}=\{s_1,\dots,s_n\}$, so $s_i'\in \spc{U}$ are defined for $i\le n$.

The sequence $s_i'$ for $i>n$ can be defined recursively using universality of $\spc{U}$.
Namely that $s_1',\dots,s_{i-1}'$ are already defined.
Since $\spc{U}$ is universal, there is a point $s_i'\in \spc{U}$ such that
\[\dist{s_i'}{s_j'}{\spc{U}}=\dist{s_i}{s_j}{\spc{S}}\]
for any $j<i$.

We constructed a distance preserving map $s_i\mapsto s_i'$, it remains to extend it to a continuous map on whole $\spc{S}$.

The first statement follows if $\spc{F}=\emptyset$.\qeds

\begin{thm}{Exercise}\label{ex:geodesics-urysohn}
Show that any two distinct points in an Urysohn space can be jointed by infinite number of geodesics.
\end{thm}

\begin{thm}{Exercise}\label{ex:sc-urysohn}
Show that Urysohn space is simply connected.
\end{thm}

\section{Uniqueness}

\begin{thm}{Theorem}\label{thm:urysohn-unique}
Suppose $\spc{F}\subset \spc{U}$ and $\spc{F}'\subset \spc{U}'$ be finite isometric subspaces in a pair of ($d$-)Urysohn spaces $\spc{U}$ and $\spc{U}'$.
Then any isometry $\spc{F}\to \spc{F}'$ can be extended to an isometry $\spc{U}\to \spc{U}'$.

In particular ($d$-)Urysohn space is unique up to isometry.
\end{thm}

Note that \ref{prop:sep-in-urys} implies that there are distance-preserving maps $\spc{U}\to \spc{U}'$ and $\spc{U}'\to \spc{U}$,
but it does not immideately imply existence of an isometry.
The following construction use the same idea as in the proof of \ref{prop:sep-in-urys}, but we need to apply it \emph{back and forth} to ensure that the constructed distance-preserving map is onto.

\parit{Proof.}
The required isometry $\spc{U}\leftrightarrow \spc{U}'$ will be denoted by $u \leftrightarrow u'$.

Choose a dense sequences $a_1,a_2,\dots\in \spc{U}$ and $b'_1,b'_2,\dots\in \spc{U}$.
Let us define recursively $a_1',b_1, a_2', b_2,\dots$ --- on the odd step we define the images of $a_1,a_2,\dots$ and on the even steps we define invese images of $b'_1,b'_2,\dots$.
The same argument as in the proof of \ref{prop:sep-in-urys} shows that we can construct two sequences $a_1',a_2',\dots\in \spc{U}'$ and $b_1,b_2,\dots\in \spc{U}$ such that
\begin{align*}
\dist{a_i}{a_j}{\spc{U}}&=\dist{a_i'}{a_j'}{\spc{U}'}
\\
\dist{a_i}{b_j}{\spc{U}}&=\dist{a_i'}{b_j'}{\spc{U}'}
\\
\dist{b_i}{b_j}{\spc{U}}&=\dist{b_i'}{b_j'}{\spc{U}'}
\end{align*}
for all $i$ and $j$.

Let us extend the constructed distance preserving bijection defined by $a_i\leftrightarrow a_i'$ and $b_i\leftrightarrow b_i'$ continuousely to whole $\spc{U}$.
Observe that the image of this bijection is dense in $\spc{U}'$ therefore the constructed map $\spc{U}\to \spc{U}'$ is a bijection.
\qeds

Further the Urysohn space will be denoted by $\spc{U}$, and the $d$-Urysohn space will be denoted by $\spc{U}_d$.
Observe that \ref{thm:urysohn-unique} implies that the spaces $\spc{U}$ and $\spc{U}_d$ are finite-set homogeneous; that is,
\begin{itemize}
 \item any distance preserving map from a finite subset to to the whole space can be extended to an isometry.
\end{itemize}
It is unknown if there is a separable universal space that is finite-set homogeneous (this question appeared already in \cite{urysohn} and reappeared in \cite[p. 83]{gromov-2007} with a missing key word). 


\begin{thm}{Exercise}\label{ex:sphere-in-urysohn}
Let $S$ be a sphere of radius $\tfrac d2$ in $\spc{U}_d$;
that is, 
\[S=\set{x\in \spc{U}_d}{\dist{p}{x}{\spc{U}_d}=\tfrac d2}\]
for some point $p\in \spc{U}_d$.
Show that $S$ is isometric to $\spc{U}_d$.

Use it to show that $\spc{U}_d$ is not countable-set homogeneous;
that is there is an distance preserving map from a countable subset of $\spc{U}_d$ to $\spc{U}_d$ that can not be extended to an isometry $\spc{U}_d\to \spc{U}_d$.
\end{thm}


\begin{thm}{Exercise}
Modify the proofs of \ref{prop:completion-univeral} and \ref{thm:urysohn-unique} to prove the following theorem.
\end{thm}

\begin{thm}{Theorem}\label{thm:compact-homogeneous}
Let $K\subset \spc{U}$ be a compact set.
Show that any distance-preserving map $f\:K\to\spc{U}$ can be extended to 
an isometry of~$\spc{U}$.
\end{thm}











\chapter{Injective spaces}

\textit{Injective spaces} (also known as \textit{hyperconvex spaces}) are the metric analog of convex sets in the following sense:

\begin{thm}{Advanced exercise}\label{ex:conv-short}
Show that $A\subset \RR^n$ is a closed convex set if and only if for any  $B\subset \RR^n$ any short map $B\to A$ can be extended to a short map $\RR^n\to A$.
\end{thm}

\section{Definition}

\begin{thm}{Definition}\label{def:injective}
A metric space $\spc{Y}$ is called \index{injective space}\emph{injective} if for any metric space $\spc{X}$ and any of its subspaces $\spc{A}$
any short map $f\:\spc{A}\to \spc{Y}$ can be extended to a short map $F\:\spc{X}\to \spc{Y}$;
that is, $f=F|_{\spc{A}}$.
\end{thm}

\begin{thm}{Exercise}\label{ex:inj=complete-geodesic-contractible}
Show that any injective space is 
\begin{multicols}{3}

\begin{subthm}{ex:inj=complete-geodesic-contractible:complete}
complete,
\end{subthm}

\begin{subthm}{ex:inj=complete-geodesic-contractible:geodesic}
geodesic, and
\end{subthm}

\begin{subthm}{ex:inj=complete-geodesic-contractible:contractible}
contractible.
\end{subthm}

\end{multicols}

\end{thm}

\begin{thm}{Exercise}\label{ex:bicombing}
Let $\spc{Y}$ be an injective space.
Show that one can choose a geodesic path $\gamma_{x,y}\:[0,1]\to \spc{Y}$ from any $x\in \spc{Y}$ to any $y\in \spc{Y}$ such that
$\gamma_{x,y}(t)\equiv\gamma_{y,x}(1-t)$ and
\[\dist{\gamma_{x,y}(t)}{\gamma_{p,q}(t)}{\spc{Y}}\le (1-t)\cdot\dist{p}{x}{\spc{Y}}+t\cdot\dist{q}{y}{\spc{Y}}\]
for any $x,y,p,q\in \spc{Y}$.
\end{thm}

\begin{thm}{Exercise}\label{ex:injective-spaces}
Show that the following spaces are injective:
\begin{subthm}{ex:injective-spaces:R}
the real line;
\end{subthm}


\begin{subthm}{ex:injective-spaces:tree}
complete metric tree;
\end{subthm}

\begin{subthm}{ex:injective-spaces:ell-infty}
coordinate plane with the metric induced by the $\ell^\infty$-norm.
\end{subthm}

%\begin{subthm}{ex:injective-spaces:L-infty}$L^\infty([0,1])$.\end{subthm}%%%???

\end{thm}

\begin{thm}{Exercise}\label{ex:extr-ball}
Let $\spc{Y}$ be an injective space.

\begin{subthm}{ex:extr-ball:one}
Show that any closed ball in $\spc{Y}$ is injective.
\end{subthm}

\begin{subthm}{ex:extr-ball:many}
Show that intersection of an arbitrary collection of closed ball in $\spc{Y}$ is injective.
\end{subthm}

\end{thm}

\begin{thm}{Advanced exercise}\label{ex:extr-fixed}
Let $\spc{Y}$ be a bounded injective space.
Show that any short map $s\:\spc{Y}\to\spc{Y}$ has a fixed point. 
\end{thm}


\section{Admissible and extremal functions}

Let $\spc{X}$ be a metric space.
A function $r\:\spc{X}\to\RR$ is called \label{page:admissible function}\index{admissible function}\emph{admissible} if the following inequality
\[r(x)+r(y)\ge \dist{x}{y}{\spc{X}}\eqlbl{eq:admissible}\]
holds for any $x,y\in \spc{X}$.

\begin{thm}{Observation}\label{obs:admissible}

\begin{subthm}{obs:admissible:nonnegative}
Any admissible function is nonnegative.
\end{subthm}

\begin{subthm}{obs:admissible:balls}
If $\spc{X}$ is a geodesic space, then a function $r\:\spc{X}\to\RR$ is admissible if and only if 
\[\cBall[x,r(x)]\cap\cBall[y,r(y)]\ne \emptyset\]
for any $x,y\in \spc{X}$.
\end{subthm}
 
\end{thm}

\parit{Proof.} For \ref{SHORT.obs:admissible:nonnegative}, take $x=y$ in \ref{eq:admissible}.

Part \ref{SHORT.obs:admissible:balls} follows from the triangle inequality and the existence of a geodesic $[xy]$.
\qeds

A minimal admissible function will be called \label{page:extremal function}\index{extremal function}\emph{extremal}.
More precisely, an admissible function $r\:\spc{X}\to\RR$ is extremal 
if for any admissible function $s\:\spc{X}\to\RR$ we have
\[s\le r\quad\Longrightarrow\quad s=r.\]

Applying Zorn's lemma, we get the following.

\begin{thm}{Observation}\label{obs:extremal:below}
For any admissible function $s$ there is an extremal function $r$ such that $r\le s$.
\end{thm}

\begin{thm}{Lemma}\label{lem:+-c}
Let $r$ be an extremal function and $s$ an admissible function on a metric space $\spc{X}$.
Suppose that $r\ge s-c$ for some constant~$c$.
Then $r\le s+c$; in particular, $c\ge 0$.
\end{thm}

\parit{Proof.}
Note that if $c<0$, then $r>s$.
The latter is impossible since $r$ is extremal and $s$ is admissible.

Observe that the function $\bar r=\min\{\,r,s+c\,\}$ is admissible.
Indeed, choose $x,y\in \spc{X}$.
If $\bar r(x)=r(x)$ and $\bar r(y)=r(y)$, then 
\[\bar r(x)+\bar r(y)=r(x)+ r(y)\ge \dist{x}{y}{}.\]
Further, if $\bar r(x)=s(x)+c$, then 
\begin{align*}
\bar r(x)+\bar r(y)&\ge [s(x)+c]+ [s(y)-c]= 
\\
&=s(x)+s(y) \ge 
\\
&\ge\dist{x}{y}{}.
\end{align*}

Since $r$ is extremal, we have $r=\bar r$;
that is, $r\le s+c$.
\qeds

\begin{thm}{Observations}\label{obs:extremal}
Let $\spc{X}$ be a metric space.

\begin{subthm}{obs:extremal:distfun}
For any point $p\in\spc{X}$
the distance function $r\z=\distfun_p$ is extremal.
\end{subthm}

\begin{subthm}{lem:extremal-lipschitz}
Any extremal function $r$ on $\spc{X}$ is \index{1-Lipschitz function}\emph{1-Lipschitz};
that is,
\[|r(p)-r(q)|\le \dist{p}{q}{}\]
for any $p,q\in\spc{X}$.
In other words, any extremal function is an extension function; see the definition in \ref{sec:Extension property}.
\end{subthm}

\begin{subthm}{lem:opposite}
An admissible function $r$ on $\spc{X}$ is extremal if and only if
for any point $p\in\spc{X}$ and any $\delta>0$, there is a point $q\in \spc{X}$
such that 
\[r(p)+r(q)<\dist{p}{q}{\spc{X}}+\delta.\]
\end{subthm}

\begin{subthm}{lem:opposite-compact}
Suppose $\spc{X}$ is compact.
Then an admissible function $r$ on $\spc{X}$ is extremal if and only if
for any point $p\in\spc{X}$ there is a point $q\in \spc{X}$
such that 
\[r(p)+r(q)=\dist{p}{q}{\spc{X}}.\]
\end{subthm}

\end{thm}

\parit{Proof; \ref{SHORT.obs:extremal:distfun}.}
By the triangle inequality, \ref{eq:admissible} holds;
that is, $r=\distfun_p$ is an admissible function.

Further, if $s\le r$ is another admissible function, then $s(p)=0$ and \ref{eq:admissible} implies that $s(x)\z\ge\dist{p}{x}{}$.
Whence $s=r$.

\parit{\ref{SHORT.lem:extremal-lipschitz}.}
By \ref{SHORT.obs:extremal:distfun}, $\distfun_p$ is admissible.
Since $r$ is admissible, we have that
\[r\ge \distfun_p-r(p).\]
Since $r$ is extremal, \ref{lem:+-c} implies that
\[r\le \distfun_p+r(p),\]
or, equivalently,
\[r(q)-r(p)\le \dist{p}{q}{}\]
for any $p,q\in\spc{X}$.
The same way we can show that
$r(p)-r(q)\le \dist{p}{q}{}$.
Whence the statement follows.

\parit{\ref{SHORT.lem:opposite}.}
Assume $r$ is extremal.
Arguing by contradiction, assume there is $\delta>0$ such that
\[r(q)\ge \distfun_p(q)-r(p)+\delta\]
for any $q$.
By \ref{SHORT.obs:extremal:distfun}, $\distfun_p$ is extremal; in particular, admissible.
Therefore \ref{lem:+-c} implies that
\[r(q)\le \distfun_p(q)+r(p)-\delta\]
for any $q$.
Taking $q=p$, we get $r(p)\le r(p)-\delta$, a contradiction.

Now suppose $r$ is not extremal; that is, there is an admissible function $s\le r$ such that $r(p)-s(p)=\delta>0$ for some $p$.
Then, for any $q$, we have
\[r(p)+r(q)\ge s(p)+s(q)+\delta\ge \dist{p}{q}{\spc{X}}+\delta\]
--- a contradiction.

\parit{\ref{SHORT.lem:opposite-compact}.}
The if part follows from \ref{SHORT.lem:opposite}.

Denote by $q_n$ the point provided by \ref{SHORT.lem:opposite} for $\delta=\tfrac1n$.
Let $q$ be a partial limit of $q_n$. 
Then 
\[r(p)+r(q)\le\dist{p}{q}{\spc{X}}.\]
Since $r$ is admissible, the opposite inequality holds;
whence the only-if part follows.
\qeds

\begin{thm}{Exercise}\label{ex:circle}
Consider the unit circle $\mathbb{S}^1=\set{(x,y)}{x^2+y^2=1}$ in the plane with induced length metric.
Show that $r\:\mathbb{S}^1\to\RR$ is extremal if and only if it is 1-Lipschitz and 
\[r(p)+r(-p)=\pi\] for any $p\in\mathbb{S}^1$.
\end{thm}

\begin{thm}{Exercise}\label{ex:retraction}
Given a real-valued function $s$ on a metric space $\spc{X}$,
consider the function
\[s^*(x)=\sup\set{\dist{z}{y}{\spc{X}}-s(y)}{y\in \spc{X}}\]
Show that if $s$ is admissible then so is $\tfrac12\cdot(s+s^*)$.
\end{thm}

\section{Equivalent conditions}

\begin{thm}{Theorem}\label{thm:injective=hyperconvex}
For any metric space $\spc{Y}$ the following condition are equivalent:

\begin{subthm}{thm:injective=hyperconvex:injective}
$\spc{Y}$ is injective
\end{subthm}


\begin{subthm}{thm:injective=hyperconvex:extremal}
If $r\:\spc{Y}\to\RR$ is an extremal function, then there is a point $p\in \spc{Y}$ such that 
\[\dist{p}{x}{}\le r(x)\]
for any $x\in \spc{Y}$.
\end{subthm}

\begin{subthm}{thm:injective=hyperconvex:balls}
$\spc{Y}$ is \index{hyperconvex space}\emph{hyperconvex};
that is, if $\set{\cBall[x_\alpha,r_\alpha]}{\alpha\in\IndexSet}$ is a family of closed balls in $\spc{Y}$ such that 
 \[r_\alpha+r_\beta\ge \dist{x_\alpha}{x_\beta}{}\]
 for any $\alpha,\beta\in \IndexSet$, then all the balls in the family $\{\cBall[x_\alpha,r_\alpha]\}_{\alpha\in\IndexSet}$ have a common point.
\end{subthm}

\end{thm}

\parit{Proof.} We will prove implications 
\ref{SHORT.thm:injective=hyperconvex:injective}$\Rightarrow$\ref{SHORT.thm:injective=hyperconvex:extremal}$\Rightarrow$\ref{SHORT.thm:injective=hyperconvex:balls}$\Rightarrow$\ref{SHORT.thm:injective=hyperconvex:injective}.

\parit{\ref{SHORT.thm:injective=hyperconvex:injective}$\Rightarrow$\ref{SHORT.thm:injective=hyperconvex:extremal}.}
Let us apply the definition of injective space to a one-point extension of $\spc{Y}$.
It follows that for any extension function $r\:\spc{Y}\to\RR$ there is a point $p\in \spc{Y}$ such that 
\[\dist{p}{x}{}\le r(x)\]
for any $x\in \spc{Y}$.
By \ref{lem:extremal-lipschitz}, any extremal function is an extension function, whence the implication follows.

\parit{\ref{SHORT.thm:injective=hyperconvex:extremal}$\Rightarrow$\ref{SHORT.thm:injective=hyperconvex:balls}.}
By \ref{obs:admissible:balls}, part \ref{SHORT.thm:injective=hyperconvex:balls} is equivalent to the following statement:
\begin{itemize}
 \item If $r\:\spc{Y}\to\RR$ is an admissible function, then there is a point $p\in \spc{Y}$ such that 
\[\dist{p}{x}{}\le r(x)\eqlbl{eq:|p-x|=<r(x)}\]
for any $x\in \spc{Y}$.
\end{itemize}
Indeed, set $r(x)\df\inf\set{r_\alpha}{x_\alpha=x}$.
(If $x_\alpha\ne x$ for any $\alpha$, then $r(x)=\infty$.)
The condition in \ref{SHORT.thm:injective=hyperconvex:balls} implies that $r$ is admissible.
It remains to observe that $p\in \cBall[x_\alpha,r_\alpha]$ for every $\alpha$ if and only if \ref{eq:|p-x|=<r(x)} holds.

By \ref{obs:extremal:below}, for any admissible function $r$ there is an extremal function $\bar r\le r$;
hence \ref{SHORT.thm:injective=hyperconvex:extremal}$\Rightarrow$\ref{SHORT.thm:injective=hyperconvex:balls}.

\parit{\ref{SHORT.thm:injective=hyperconvex:balls}$\Rightarrow$\ref{SHORT.thm:injective=hyperconvex:injective}.}
Arguing by contradiction, suppose $\spc{Y}$ is not injective;
that is, there is a metric space $\spc{X}$ with a subset $\spc{A}$
such that a short map $f\:\spc{A}\to \spc{Y}$ cannot be extended to a short map $F\:\spc{X}\to \spc{Y}$.
By Zorn's lemma, we may assume that $\spc{A}$ is a maximal subset; that is, the domain of $f$ cannot be enlarged by a single point.%
\footnote{In this case, $\spc{A}$ must be closed, but we will not use it.}

Fix a point $p$ in the complement $\spc{X}\setminus \spc{A}$.
To extend $f$ to $p$, we need to choose $f(p)$ in the intersection of the balls 
$\cBall[f(x),r(x)]$, where $r(x)=\dist{p}{x}{}$.
Therefore, this intersection for all $x\in \spc{A}$ has to be empty.

Since $f$ is short, we have that 
\begin{align*}
r(x)+r(y)&\ge \dist{x}{y}{\spc{X}}\ge
\\
&\ge \dist{f(x)}{f(y)}{\spc{Y}}.
\end{align*}
Therefore, by \ref{SHORT.thm:injective=hyperconvex:balls} the balls 
$\cBall[f(x),r(x)]$ have a common point --- a contradiction. 
\qeds

\begin{thm}{Exercise}\label{ex:one-point-gluing}
Suppose a length space $\spc{W}$ has two subspaces $\spc{X}$ and $\spc{Y}$ such that $\spc{X}\cup\spc{Y}=\spc{W}$ and $\spc{X}\cap\spc{Y}$ is a one-point set.
Assume $\spc{X}$ and $\spc{Y}$ are injective.
Show that  $\spc{W}$ is injective
\end{thm}

\begin{thm}{Exercise}\label{ex:Rm-ell-infty}
Show that a $m$-dimensional normed space is injective if and only if it is isometric to $\RR^m$ with the norm
\[|(x_1,\dots,x_m)|=\max_i\{\,|x_i|\,\}.\]
\end{thm}


\begin{thm}{Exercise}\label{ex:urysohn-hyperconvex}
Show that the $d$-Urysohn space is {}\emph{finitely hyperconvex} but not {}\emph{countably hyperconvex};
that is, the condition in \ref{thm:injective=hyperconvex:balls} holds for any finite family of balls, but may not hold for a countable family.
Conclude that the $d$-Urysohn space is not injective.

Try to do the same for the Urysohn space.
\end{thm}

\section{Space of extremal functions}
\label{sec:extremal-functions}

Let $\spc{X}$ be a metric space.
Consider the space $\Inj \spc{X}$ of extremal functions on $\spc{X}$ equipped with sup-norm; \label{page:InjX}
that is,
\[\dist{f}{g}{\Inj \spc{X}}\df\sup\set{|f(x)-g(x)|}{x\in \spc{X}}.\]

Recall that by \ref{obs:extremal:distfun}, any distance function is extremal.
It follows that the map $x\mapsto \distfun_x$ produces a distance-preserving embedding $\spc{X}\hookrightarrow\Inj \spc{X}$.
So we can (and will) treat $\spc{X}$ as a subspace of $\Inj \spc{X}$,
or, equivalently, $\Inj \spc{X}$ as an extension of $\spc{X}$.

Since any extremal function is 1-Lipschitz, for any $f\in \Inj \spc{X}$ and $p\in \spc{X}$, we have that
$f(x)\le f(p)+\distfun_p(x)$.
By \ref{lem:+-c}, we also get $f(x)\ge -f(p)+\distfun_p(x)$.
Therefore
\[
\begin{aligned}
\dist{f}{p}{\Inj \spc{X}}&=\sup\set{|f(x)-\distfun_p(x)|}{x\in \spc{X}}=
\\
&=f(p).
\end{aligned}
\eqlbl{eq:f(p)=|f-p|}
\]
In particular, the statement in \ref{lem:opposite} can be written as 
\[\dist{f}{p}{\Inj\spc{X}}+\dist{f}{q}{\Inj\spc{X}}<\dist{p}{q}{\Inj\spc{X}}+\delta.\]

\begin{thm}{Exercise}\label{ex:Inj(compact)}
Let $\spc{X}$ be a metric space.
Show that $\Inj\spc{X}$ is compact if and only if so is $\spc{X}$.
\end{thm}

\begin{thm}{Exercise}\label{ex:tripod+square}
Describe the set of all extremal functions on a metric space $\spc{X}$ and the metric space $\Inj \spc{X}$ in each of the following cases:

\begin{subthm}{ex:tripod+square:2}
$\spc{X}$ is a metric space with exactly two points $v,w$ on distance 1 from each other.
\end{subthm}


\begin{subthm}{ex:tripod+square:tripod} 
$\spc{X}$ is a metric space with exactly three points $a,b,c$ such that 
\[\dist{a}{b}{\spc{X}}=\dist{b}{c}{\spc{X}}=\dist{c}{a}{\spc{X}}=1.\]
\end{subthm}

\begin{subthm}{ex:tripod+square:square}
$\spc{X}$ is  a metric space with exactly four points $p,q,x,y$ such that 
\[\dist{p}{x}{\spc{X}}=\dist{p}{y}{\spc{X}}=\dist{q}{x}{\spc{X}}=\dist{q}{y}{\spc{X}}=1\]
and
\[\dist{p}{q}{\spc{X}}=\dist{x}{y}{\spc{X}}=2.\]
\end{subthm}

\end{thm}

\begin{thm}{Exercise}\label{ex:kur-inj}
Assume $\spc{X}$ is a compact metric space.
Recall that the map $x\mapsto \distfun_x$ gives an isometric embedding $\spc{X}\hookrightarrow\ell^\infty(\spc{X})$; so we can think that $\spc{X}$ is a subset of $\ell^\infty(\spc{X})$.

Given two points $x,y\in \spc{X}$, denote by $G_{x,y}$ the union of all geodesics from $x$ to $y$ in $\ell^\infty(\spc{X})$.
Show that $\Inj\spc{X}$ is isometric to
\[G=\bigcap_{x\in \spc{X}}\left(\bigcup_{y\in \spc{X}}G_{x,y}\right).\]

\end{thm}


\begin{thm}{Proposition}\label{prop:InjX-is-injective}
For any metric space $\spc{X}$, its extension $\Inj\spc{X}$ is  injective.
\end{thm}

\begin{thm}{Lemma}\label{lem:r|X-extremal}
Let $\spc{X}$ be a metric space.
Suppose $r\in \Inj(\Inj \spc{X})$;
that is, $r$ is an extremal function on $\Inj \spc{X}$.
Then $r|_\spc{X}\in \Inj \spc{X}$;
that is, the restriction of $r$ to $\spc{X}$ is an extremal function.
\end{thm}

\parit{Proof.}
Arguing by contradiction, suppose that there is an admissible function $s\:\spc{X}\to \RR$ such that $s(x)\le r(x)$ for any $x\in\spc{X}$ and $s(p)\z< r(p)$ for some point $p\in\spc{X}$.
Consider another function $\bar r\:\Inj \spc{X}\to\RR$ such that $\bar r(f)\df r(f)$ if $f\ne p$ and $\bar r(p)\df s(p)$.

Let us show that $\bar r$ is admissible; that is, 
\[\dist{f}{g}{\Inj \spc{X}}\le\bar r(f)+\bar r(g)
\eqlbl{r-admissible}\]
for any $f,g\in \Inj \spc{X}$.

Since $r$ is admissible and $\bar r= r$ on $(\Inj \spc{X})\setminus \{p\}$, it is sufficient to prove \ref{r-admissible} if $f\ne g=p$.
By \ref{eq:f(p)=|f-p|}, we have $\dist{f}{p}{\Inj \spc{X}}=f(p)$.
Therefore, \ref{r-admissible} boils down to the following inequality
\[r(f)+s(p)\ge f(p).\eqlbl{eq:r(f)+s(p)>=f(p)}\]
for any $f\in\Inj \spc{X}$.

Fix small $\delta>0$. 
Let $q\in\spc{X}$ be the point provided by \ref{lem:opposite}.
Then
\begin{align*}
r(f)+s(p)&\ge [r(f)-r(q)]+[r(q)+s(p)]\ge
\intertext{since $r$ is 1-Lipschitz, and $r(q)\ge s(q)$, we can continue}
&\ge -\dist{q}{f}{\Inj \spc{X}}+[s(q)+s(p)]\ge
\intertext{by \ref{eq:f(p)=|f-p|} and since $s$ is admissible}
&\ge -f(q)+\dist{p}{q}{}>
\intertext{and by \ref{lem:opposite}}
&> f(p)-\delta.
\end{align*}
Since $\delta>0$ is arbitrary, \ref{eq:r(f)+s(p)>=f(p)} and \ref{r-admissible} follow.

Summarizing: the function $\bar r$ is admissible, $\bar r\le r$ and $\bar r(p)<r(p)$;
that is, $r$ is not extremal --- a contradiction.
\qeds

\parit{Proof of \ref{prop:InjX-is-injective}.}
Choose a function $r\in\Inj(\Inj\spc{X})$.
By \ref{lem:r|X-extremal}, $s\z\df r|_{\spc{X}}\in \Inj\spc{X}$;
that is, $s$ is extremal.
By \ref{thm:injective=hyperconvex:extremal},
it is sufficient to show that  
\[r(f)\ge\dist{s}{f}{\Inj\spc{X}}
\eqlbl{eq:r(f)>=| r-f|}\]
for any $f\in\Inj\spc{X}$.

Since $r$ is $1$-Lipschitz (\ref{lem:extremal-lipschitz}) we have that
\[
s(x)-f(x)=r(x)-\dist{f}{x}{\Inj \spc{X}}\le r(f).
\]
for any $x\in\spc{X}$.
By \ref{lem:+-c},
$
s(x)-f(x)\ge -r(f)
$
for any $x\in\spc{X}$.
Whence \ref{eq:r(f)>=| r-f|} follows.
\qeds

\begin{thm}{Exercise}\label{ex:4-on-a-line}
Let $\spc{X}$ be a compact metric space.
Show that for any two points $f,g\in\Inj \spc{X}$ lie on a geodesic $[pq]$ with $p,q\in \spc{X}$.
\end{thm}

A metric space $\spc{X}$ is called \index{$\delta$-hyperbolic}\emph{$\delta$-hyperbolic} if 
\[\dist{p}{q}{}+\dist{x}{y}{}\le
\max\{\,\dist{p}{x}{}+\dist{q}{y}{},
\,
\dist{p}{y}{}+\dist{q}{x}{}\,\}+2\cdot\delta\]
for any $p,q,x,y\in \spc{X}$.

\begin{thm}{Advanced exercise}\label{ex:delta-hyp}
Show that $\Inj \spc{X}$ is $\delta$-hyperbolic if and only if $\spc{X}$ is.
\end{thm}


\section{Injective envelope}

An extension $\spc{E}$ of a metric space $\spc{X}$ will be called its \index{injective envelope}\emph{injective envelope} if $\spc{E}$ is an injective space, and there is no proper injective subspace of $\spc{E}$ that contains $\spc{X}$.

Two injective envelopes $e\:\spc{X}\hookrightarrow \spc{E}$ and $f\:\spc{X}\hookrightarrow \spc{F}$ are called  equivalent if there is an isometry $\iota\: \spc{E}\to\spc{F}$ such that $f=\iota\circ e$.

\begin{thm}{Theorem}\label{thm:inj-envelope}
For any metric space $\spc{X}$, its extension $\Inj\spc{X}$ is an injective envelope.

Moreover, any other injective envelope of $\spc{X}$ is equivalent to $\Inj\spc{X}$.
\end{thm}

\parit{Proof.} 
Suppose $S\subset \Inj\spc{X}$ is an injective subspace containing $\spc{X}$.
Since $S$ is injective, there is a short map $w\:\Inj\spc{X}\to S$ that fixes all points in $\spc{X}$.

Suppose that $w\:f\mapsto f'$; observe that $f(x)\ge f'(x)$ for any $x\in \spc{X}$.
Since $f$ is extremal, $f=f'$;
that is, $w$ is the identity map, and therefore $S=\Inj\spc{X}$.

Assume we have another injective envelope $e\:\spc{X}\hookrightarrow \spc{E}$.
Then there are short maps $v\:\spc{E}\to \Inj\spc{X}$ and $w\:\Inj\spc{X}\to \spc{E}$ such that $x=v\circ e(x)$ and $e(x)=w(x)$ for any $x\in\spc{X}$.
From above, the composition $v\circ w$ is the identity on $\Inj\spc{X}$.
In particular, $w$ is distance-preserving.

The composition $w\circ v\:\spc{E}\to \spc{E}$ is a short map that fixes points in $e(\spc{X})$.
Since $e\:\spc{X}\hookrightarrow \spc{E}$ is an injective envelope, the composition $w\circ v$ and therefore $w$ are onto.
Whence $w$ is an isometry.
\qeds

\begin{thm}{Exercise}\label{ex:d-p-inclusion}
Suppose $\spc{X}$ is a subspace of a metric space $\spc{U}$.
Show that the inclusion $\spc{X}\hookrightarrow\spc{U}$ can be extended to a distance-preserving inclusion $\Inj\spc{X}\hookrightarrow\Inj\spc{U}$.
\end{thm}


\section{Remarks}

Injective spaces were introduced by Nachman Aronszajn and Prom Panitchpakdi \cite{aronszajn-panitchpakdi}.
The injective envelope was introduced by John Isbell \cite{isbell}.
It was rediscovered a couple of times since then;
as a result, the injective envelope has many other names including \index{tight span}\emph{tight span} and \index{hyperconvex hull}\emph{hyperconvex hull}.

The following two exercise deals with ultrametric spaces which in some sense are dual to the injective spaces. 

Recall that if the following inequality
\[\dist{x}{z}{\spc{X}}
\le
\max\{\,\dist{x}{y}{\spc{X}},\dist{y}{z}{\spc{X}}\,\}\]
holds for any three points $x,y,z$ in a metric space $\spc{X}$,
then $\spc{X}$ is called an \index{ultrametric space}\emph{ultrametric space}.

\begin{thm}{Exercise}\label{ex:ultrametric}
Suppose that a metric space $\spc{X}$ satisfies the following property:
For any subspace $\spc{A}$ in $\spc{X}$ and any other metric space $\spc{Y}$, any short map $f\:\spc{A}\to \spc{Y}$ can be extended to a short map $F\:\spc{X}\to \spc{Y}$.

Show that $\spc{X}$ is an ultrametric space.
\end{thm}

A subspace $\spc{S}$ of a metric space $\spc{X}$ is called its \index{short retract}\emph{short retract} if there is a short map $\spc{X}\to \spc{S}$ that is the identity on $\spc{S}$.

\begin{thm}{Exercise}\label{ex:ultrametric-converse}
Show that any compact subspace $\spc{K}$ of an ultrametric space $\spc{X}$ is its short retract.

Construct an example of a complete ultrametric space $\spc{X}$ with a closed subset $Q$ that is not its short retract.
\end{thm}

The following exercise gives a sufficient condition for existence of a short extension.

\begin{thm}{Exercise}\label{ex:petrunin-stadler}
Let $\spc{X}$ and $\spc{Y}$ be metric spaces, $A\subset \spc{X}$, and $f\:A\z\to \spc{Y}$ be a short map.
Assume $\spc{Y}$ is compact and for any finite set $F\subset \spc{X}$ there is a short map $F\to \spc{Y}$ that agrees with $f$ on $F\cap A$.
Show that there is a short map $\spc{X}\to \spc{Y}$ that agrees with $f$ on $A$.
\end{thm}

\chapter{Space of sets}

\section{Hausdorff distance}

Let $\spc{X}$ be a metric space.
Given a subset $A\subset \spc{X}$,
consider the distance function to $A$
$$\distfun_A: \spc{X} \to [0,\infty)$$
defined as 
$$\distfun_A(x)
\df
\inf_{a\in A}\{\,\dist ax{\spc{X}}\,\}.$$

\begin{thm}{Definition}\label{def:hausdorff-convergence}
Let $A$ and $B$ be two compact subsets of a metric space $\spc{X}$.
Then the \index{Hausdorff distance}\emph{Hausdorff distance} between $A$ and $B$ is defined as 
$$|A-B|_{\Haus\spc{X}}
\df
\sup_{x\in \spc{X}}\{\,|\distfun_A(x)-\distfun_B(x)|\,\}.
$$

\end{thm}

The following observation gives a useful reformulation of the definition:

\begin{thm}{Observation}\label{obs:Haus-nbhds}
Suppose $A$ and $B$ be two compact subsets of a metric space $\spc{X}$.
Then $|A-B|_{\Haus\spc{X}}< R$ if and only if and only if 
$B$ lies in an $R$-neighborhood of $A$, 
and 
$A$ lies in an $R$-neighborhood of~$B$.
\end{thm}



Note that the set of all nonempty compact subsets of a metric space $\spc{X}$ equipped with the Hausdorff metric forms a metric space.
This new metric space will be denoted as $\Haus\spc{X}$.


\begin{thm}{Exercise}\label{ex:diam}
Let $\spc{X}$ be a metric space.
Given a subset $A\subset \spc{X}$ define its \index{diameter}\emph{diameter} as 
$$\diam A\df\sup_{a,b\in A} |a-b|.$$

Show that 
$$\diam\:\Haus\spc{X}\to \RR$$ 
is a \index{Lipschitz function}\emph{$2$-Lipschitz function};
that is,
\[|\diam A-\diam B|\le 2\cdot\dist{A}{B}{\Haus\spc{X}}\]
for any two compact nonempty sets $A,B\subset\spc{X}$.
\end{thm}


\begin{thm}{Exercise}\label{ex:Hausdorff-bry}
Let $A$ and $B$ be two compact subsets in the Euclidean plane $\RR^2$.
Assume $|A-B|_{\Haus\RR^2}<\eps$.

\begin{subthm}{ex:Hausdorff-bry:conv}
Show that $|\Conv A-\Conv B|_{\Haus\RR^2}<\eps$, where $\Conv A$ denoted the convex hull of $A$.
\end{subthm}
\begin{subthm}{ex:Hausdorff-bry:bry}
Is it true that
$|\partial A-\partial B|_{\Haus\RR^2}<\eps$,
where $\partial A$ denotes the boundary of $A$.

Does the converse hold? That is, assume $A$ and $B$ be two compact subsets in $\RR^2$
and $|\partial A-\partial B|_{\Haus\RR^2}<\eps$; 
is it true that $|A-B|_{\Haus\RR^2}\z<\eps$?
\end{subthm}

\end{thm}

Note that part \ref{SHORT.ex:Hausdorff-bry:conv} implies that $A\mapsto \Conv A$ defines a short map $\Haus\RR^2\to \Haus\RR^2$. 

\begin{thm}{Exercise}\label{ex:Haus-func}
Let $A$ and $B$ be two compact subsets in metric space~$\spc{X}$.
Show that 
\[\dist{A}{B}{\Haus\spc{X}}=\sup_f\, \{\,\max_{a\in A}\{f(a)\}-\max_{b\in B}\{f(b)\,\},\]
where the least upper bound is taken for all $1$-Lipschitz functions $f$.

\end{thm}

\begin{thm}{Advanced exercise}\label{ex:H-sections}
\begin{subthm}{ex:H-sections:S}
Construct a family of compact sets $C_t\subset\mathbb{S}^1$, $t\z\in [0,1]$ that is continuous in the Hausdorff topology, 
but does not admit a {}\emph{section}.
That is, there is no path $c\:[0,1]\to \mathbb{S}^1$ such that $c(t)\in C_t$ for all $t$.
\end{subthm}

\begin{subthm}{ex:H-sections:R}
Show that any family of compact sets $C_t\subset\RR^1$, $t\z\in [0,1]$ that is continuous in the Hausdorff topology, 
admits a {}\emph{section}.
That is, there is path $c\:[0,1]\to \RR^1$ such that $c(t)\in C_t$ for all $t$.
\end{subthm}

\end{thm}

\section{Hausdorff convergence}

\begin{thm}{Blaschke selection theorem}\label{thm:compact+Hausdorff}
A metric space $\spc{X}$ is compact if and only if
so is $\Haus\spc{X}$.
\end{thm}

The Hausdorff metric can be used to define convergence.
Namely, suppose $K_1,K_2,\dots$, and $K_\infty$ are compact sets in a metric space $\spc{X}$.
If $|K_\infty-K_n|_{\Haus\spc{X}}\to0$ as $n\to\infty$, then we say that 
the sequence $K_n$ {}\emph{converges} to $K_\infty$ \index{convergence in the sense of Hausdorff}\emph{in the sense of Hausdorff};
or we can say that $K_\infty$ is {}\emph{Hausdorff limit} of the sequence $K_n$.

Note that the theorem implies that from any sequence of compact sets in $\spc{X}$ one can select a subsequence that converges in the sense of Hausdorff; 
for that reason, it is called a \textit{selection} theorem. 

\parit{Proof; if part.}
Consider the map $\iota$ that sends each point $x\in \spc{X}$ to the one-point subset $\{x\}$ of $\spc{X}$.
Note that $\iota\:\spc{X}\to \Haus\spc{X}$ is distance-preserving.

Suppose that $A\subset \spc{X}$.
Note that $\diam A=0$ if and only if $A$ is a one-point set.
By \ref{ex:diam}, $\iota(\spc{X})$ is a closed subset of the compact space $\Haus\spc{X}$.
It follows that $\iota(\spc{X})$, and therefore $\spc{X}$, are compact.
\qeds

Since the map $\iota$ above is distance-preserving, we can and will consider $\spc{X}$ as a subspace of $\Haus\spc{X}$.

\begin{thm}{Exercise}\label{ex:haus-contractible}
Let $\spc{X}$ be a bounded length space with.
Suppose that there is a short retraction $\Haus\spc{X}\to \spc{X}$.
Show that $\spc{X}$ is contractible.
\end{thm}


To prove the only-if part we will need the following two lemmas.

\begin{thm}{Monotone convergence}\label{lem:decreasing-converges}
Let $K_1\supset K_2\supset\dots$ be a nested sequence of nonempty compact sets in a metric space $\spc{X}$.
Then $K_\infty\z=\bigcap_n K_n$ is the Hausdorff limit of $K_n$;
that is, $|K_\infty-K_n|_{\Haus\spc{X}}\to0$ as $n\to\infty$.
\end{thm}

\parit{Proof.}
By finite intersection property, $K_\infty$ is a nonempty compact set.

If the assertion were false, then there is $\eps>0$ such that for each $n$ 
one can choose $x_n\in K_n$
such that $\distfun_{K_\infty}(x_n)\ge\eps$.
Note that $x_n\in K_1$ for each $n$.
Since $K_1$ is compact, 
there is 
a \index{partial limit}\emph{partial limit}%
\footnote{Partial limit is a limit of a subsequence.}
 $x_\infty$ of $x_n$.
Clearly $\distfun_{K_\infty}(x_\infty)\ge \eps$.

On the other hand, since $K_n$ is closed and $x_m\in K_n$ for $m\ge n$,
we get $x_\infty\in K_n$ for each $n$.
It follows that $x_\infty\in K_\infty$ and therefore $\distfun_{K_\infty}(x_\infty)=0$ ---
a contradiction.\qeds


\begin{thm}{Lemma}\label{lem:complete+Hausdorff}
If $\spc{X}$ is a compact metric space, then $\Haus\spc{X}$
is complete.
\end{thm}

\parit{Proof.}
Let $(Q_n)$ be a Cauchy sequence in $\Haus\spc{X}$.
Passing to a subsequence of $Q_n$ we may assume that 
$$|Q_n-Q_{n+1}|_{\Haus\spc{X}}\le \tfrac1{10^n}\eqlbl{eq:eps=1/10}$$
for each $n$.

Denote by $K_n$ the closed $\tfrac1{10^n}$-neighborhood of $Q_n$;
that is,
\begin{align*}
K_n&= \set{x\in \spc{X}}{\distfun_{Q_n}(x)\le \tfrac1{10^n}}
\end{align*}
Since $\spc{X}$ is compact so is each $K_n$.

By \ref{obs:Haus-nbhds}, $|Q_n-K_n|_{\Haus\spc{X}}\le \tfrac1{10^n}$.
From \ref{eq:eps=1/10}, we get
$K_n\supset K_{n+1}$ 
for each $n$.
Set 
$$K_\infty=\bigcap_{n=1}^\infty K_n.$$
By the monotone convergence (\ref{lem:decreasing-converges}),
 $|K_n-K_\infty|_{\Haus\spc{X}}\to 0$ as $n\to\infty$.
Since $|Q_n-K_n|_{\Haus\spc{X}}\le \tfrac1{10^n}$, we get $|Q_n-K_\infty|_{\Haus\spc{X}}\to 0$ as $n\to\infty$ --- hence the lemma.
\qeds

\begin{thm}{Exercise}\label{ex:closure-union}
Let $\spc{X}$ be a complete metric space and $K_1,K_2,\dots$ be a sequence of compact sets 
that converges in the sense of Hausdorff.
Show that the union $K_1\cup K_2\cup\dots$ has compact closure.

Use this statement to show that in Lemma~\ref{lem:complete+Hausdorff} compactness of $\spc{X}$ can be exchanged to completeness.
\end{thm}

\parit{Proof of only-if part in \ref{thm:compact+Hausdorff}.}
According to Lemma~\ref{lem:complete+Hausdorff},
$\Haus\spc{X}$ is complete.
It remains to show that $\Haus\spc{X}$ is totally bounded (\ref{totally-bounded});
that is, given $\eps>0$ there is a finite $\eps$-net in $\Haus\spc{X}$.

Choose a finite $\eps$-net $A$ in $\spc{X}$.
Denote by $B$ the set of all subsets of $A$.
Note that  $B$ is a finite set in $\Haus\spc{X}$.
For each compact set $K\subset \spc{X}$, consider the subset $K'$ of all points $a\in A$
such that $\distfun_K(a)\le \eps$.
Observe that $K' \in B$ and $|K-K'|_{\Haus\spc{X}}\le\eps$.
In other words, $B$ is a finite $\eps$-net in $\Haus\spc{X}$.
\qeds

\begin{thm}{Exercise}\label{ex:Haus-length}
Let $\spc{X}$ be a complete metric space.
Show that $\spc{X}$ is a length space if and only if so is $\Haus\spc{X}$.
\end{thm}

\section{An application}

The following statement is called \index{isoperimetric inequality}\emph{isoperimetric inequality in the plane}.

\begin{thm}{Theorem}\label{thm:isoperimetric}
Among the plane figures bounded by closed curves of length at most $\ell$, the round disk has the maximal area.
\end{thm}

In this section, we will sketch a proof of the isoperimetric inequality that uses the Hausdorff convergence.
It is based on the following exercise.

\begin{thm}{Exercise}\label{ex:Huas-perimeter-area}
Let $\spc{C}$ be a subspace of $\Haus\RR^2$ formed by all compact convex subsets in $\RR^2$.
Show that perimeter\footnote{If the set degenerates to a line segment of length $\ell$, then its perimeter is defined as $2\cdot \ell$.} and area are continuous on~$\spc{C}$.
That is, if a sequence of convex compact plane sets $X_n$ converges to $X_\infty$ in the sense of Hausdorff, then 
\[\perim X_n\to \perim X_\infty\quad\text{and}\quad\area X_n\to\area X_\infty\]
as $n\to\infty$.
\end{thm}

\parit{Semiproof of \ref{thm:isoperimetric}.}
It is sufficient to consider only convex figures of the given perimeter; if a figure is not convex, pass to its convex hull and observe that it has a larger area and smaller perimeter.


Note that the selection theorem (\ref{thm:compact+Hausdorff}) together with the exercise imply the existence of figure $D$ with perimeter $\ell$ and maximal area.

It remains to show that $D$ is a round disk.
This is a problem in elementary geometry.

Let us cut $D$ along a chord $[ab]$ into two lenses, $L_1$ and $L_2$.
Denote by $L_1'$ the reflection of $L_1$ across the perpendicular bisector of $[ab]$.
Note that $D$ and $D'=L_1'\cup L_2$ have the same perimeter and area.
That is, $D'$ has perimeter $\ell$ and maximal possible area;
in particular, $D'$ is convex.

The following exercise will finish the proof.
\qeds

{

\begin{wrapfigure}{o}{57 mm}
\vskip-5mm
\centering
\includegraphics{mppics/pic-405}
\end{wrapfigure}

\begin{thm}{Exercise}\label{ex:round-disc}
Suppose $D$ is a convex figure such that for any chord $[ab]$ of $D$ the above construction produces a convex figure $D'$.
Show that $D$ is a round disk.
\end{thm}


}

Another popular way to prove that $D$ is a round disk is given by the so-called {}\emph{Steiner's 4-joint method} \cite{blaschke}.

\section{Remarks}\label{sec:H-variation}

It seems that Hausdorff convergence was first introduced by Felix Hausdorff~\cite{hausdorff}.
A couple of years later an equivalent definition was given by Wilhelm Blaschke~\cite{blaschke}.

The following refinement was introduced by  Zdeněk Frolík \cite{frolik},
later it was rediscovered by Robert Wijsman~\cite{wijsman}.  
This refinement is also called \index{Hausdorff convergence}\emph{Hausdorff convergence};
in fact, it takes an intermediate place between the original Hausdorff convergence and {}\emph{closed convergence}, also introduced by Hausdorff in \cite{hausdorff}.

\begin{thm}{Definition}\label{def:gen-Haus-conv}
Let $A_1,A_2,\dots$ be a sequence of closed sets in a metric space $\spc{X}$.
We say that the sequence $A_n$ converges to a closed set $A_\infty$ in the sense of Hausdorff if for any $x\in\spc{X}$, we have
$\distfun_{A_n}(x)\z\to \distfun_{A_\infty}(x)$ as $n\to\infty$.
\end{thm}

For example, suppose $\spc{X}$ is the Euclidean plane and $A_n$ is the circle with radius $n$ and center at the point $(n,0)$.
If we use the standard definition (\ref{def:hausdorff-convergence}), then the sequence $(A_n)$ diverges, but it converges to the $y$-axis in the sense of Definition~\ref{def:gen-Haus-conv}.

Further, consider the sequence of one-point sets $B_n=\{(n,0)\}$ in the Euclidean plane.
It diverges in the sense of the standard definition, but, in the sense of \ref{def:gen-Haus-conv}, it converges to the empty set;
indeed, for any point $x$ we have $\distfun_{B_n}(x)\to\infty$ as $n\to \infty$ and $\distfun_{\emptyset}(x)= \infty$ for any~$x$.

The following exercise is analogous to the Blaschke selection theorem (\ref{thm:compact+Hausdorff}) for the modified Hausdorff convergence.

\begin{thm}{Exercise}\label{ex:generalized-selection}
Let $\spc{X}$ be a proper metric space
and $A_1,A_2,\dots$ be a sequence of closed sets in~$\spc{X}$.
Show that the sequence  $A_1,A_2,\dots$ has a convergent subsequence in the sense of Definition~\ref{def:gen-Haus-conv}.
\end{thm}

\chapter{Space of spaces}

\section{Gromov--Hausdorff metric}

The goal of this section is to cook up a metric space out of all compact metric spaces.
More precisely, we want to define the so-called  Gromov--Hausdorff metric on the set of \textit{isometry classes} of compact metric spaces.
(Being isometric is an equivalence relation, 
and an \index{isometry class}\emph{isometry class} is an equivalence class with respect to this relation.)

The obtained metric space will be denoted by $\GH$.
Given two metric spaces $\spc{X}$ and $\spc{Y}$,
denote by $[\spc{X}]$ and $[\spc{Y}]$ their isometry classes;
that is, $\spc{X}'\in [\spc{X}]$ if and only if $\spc{X}'\iso \spc{X}$.
Pedantically, the Gromov--Hausdorff distance from $[\spc{X}]$ 
to $[\spc{Y}]$ should be denoted as $|[\spc{X}]-[\spc{Y}]|_{\GH}$;
but we will write it as $|\spc{X}\z-\spc{Y}|_{\GH}$ and say (not quite correctly) that 
\textit{$|\spc{X}\z-\spc{Y}|_{\GH}$ is the Gromov--Hausdorff distance from  $\spc{X}$ 
to  $\spc{Y}$}.
In other words, from now on the term \textit{metric space} might also stand for its \textit{isometry class}.

The metric on $\GH$ is defined as the maximal metric such that \textit{the distance between subspaces in a metric space is not greater than the Hausdorff distance between them}.
Here is a formal definition:

\begin{thm}{Definition}\label{def:GH}
The \index{Gromov--Hausdorff distance}\emph{Gromov--Hausdorff distance} $|\spc{X}-\spc{Y}|_{\GH}$ between compact metric spaces $\spc{X}$ and $\spc{Y}$
is defined by the following
relation.
 
Given  $r > 0$, we have that $|\spc{X}-\spc{Y}|_{\GH} < r$ if and only if there exists a metric
space $\spc{W}$ and subspaces $\spc{X}'$ and $\spc{Y}'$ in $\spc{W}$ that are isometric to $\spc{X}$ and $\spc{Y}$
respectively such that $|\spc{X}'-\spc{Y}'|_{\Haus\spc{W}} < r$. 
(Here $|\spc{X}'-\spc{Y}'|_{\Haus\spc{W}}$ denotes the Hausdorff distance between sets $\spc{X}'$ and $\spc{Y}'$ in $\spc{W}$.)
\end{thm}

\begin{thm}{Theorem}\label{thm:GH-is-a-metric}
The set of isometry classes of compact metric spaces equipped with Gromov--Hausdorff metric forms a metric space (which is denoted by $\GH$).

In other words, for arbitrary compact metric spaces $\spc{X}$, $\spc{Y}$ and $\spc{Z}$ the following conditions hold:

\begin{subthm}{GH-1} $|\spc{X}-\spc{Y}|_{\GH}\ge 0$;
\end{subthm}

\begin{subthm}{GH-2} $|\spc{X}-\spc{Y}|_{\GH}=0$ if and only if $\spc{X}$ is isometric to $\spc{Y}$;
\end{subthm}

\begin{subthm}{GH-3} $|\spc{X}-\spc{Y}|_{\GH}=|\spc{Y}-\spc{X}|_{\GH}$;
\end{subthm}

\begin{subthm}{GH-4} $|\spc{X}-\spc{Y}|_{\GH}+|\spc{Y}-\spc{Z}|_{\GH}\ge |\spc{X}-\spc{Z}|_{\GH}$.
\end{subthm}
\end{thm}


Note that \ref{SHORT.GH-1}, \ref{SHORT.GH-3},
and the ``if''-part of \ref{SHORT.GH-2} follow directly from Definition \ref{def:GH}.
Part \ref{SHORT.GH-4} will be proved in Section~\ref{sec:GH-approx}.
The ``only-if''-part of \ref{SHORT.GH-2} will be proved in Section~\ref{sec:extfun=GH}.

Recall that $a\cdot\spc{X}$ denotes $\spc{X}$ \index{rescaled space}\emph{rescaled} by factor $a>0$;
that is, $a\cdot\spc{X}$ is a metric space with the underlying set of $\spc{X}$ and the metric defined by
\[\dist{x}{y}{a\cdot\spc{X}}\df a\cdot\dist{x}{y}{\spc{X}}.\]

\begin{thm}{Exercise}\label{ex:d_GH-and-diam}
Let $\spc{X}$ be a compact metric space,
$\spc{O}$ be the one-point metric space.

Prove that 

\begin{subthm}{ex:d_GH-and-diam:point}
$|\spc{X}-\spc{O}|_{\GH}=\tfrac12\cdot \diam \spc{X}.$

\end{subthm}

\begin{subthm}{ex:d_GH-and-diam:scale}
$|a\cdot\spc{X}-b\cdot \spc{X}|_{\GH}=\tfrac12\cdot|a-b|\cdot\diam\spc{X}.$
\end{subthm}

\begin{subthm}{ex:d_GH-and-diam:isometry}
$\iota[\spc{O}]=[\spc{O}]$ for any isometry $\iota\:\GH\to\GH$.
\end{subthm}


\end{thm}




\begin{thm}{Exercise}\label{ex:GH<H}
Find subsets $A,B\subset\RR^2$ such that 
\[|A-B|_{\GH}>|A-\iota(B)|_{\Haus\RR^2}\]
for any isometry $\iota$ of $\RR^2$.
\end{thm}


\begin{thm}{Exercise}\label{ex:rectangle}
Let $\spc{A}_r$ be a rectangle $1$ by $r$ in the Euclidean plane 
and $\spc{B}_r$ be a closed line interval of length $r$.
Show that 
\[|\spc{A}_r-\spc{B}_r|_{\GH}>\tfrac1{10}\]
for all large $r$.
\end{thm}

\begin{thm}{Advanced exercise}\label{ex:GH-inj}
Let $\spc{X}$ and $\spc{Y}$ be compact metric spaces;
denote by $\hat{\spc{X}}$ and $\hat{\spc{Y}}$ their injective envelopes (see \ref{sec:extremal-functions}).
Show that 
\[|\hat{\spc{X}}-\hat{\spc{Y}}|_{\GH}\le 2\cdot|\spc{X}- \spc{Y}|_{\GH}.\] 
In other words $\spc{X}\mapsto \hat{\spc{X}}$ defines a $2$-Lipschitz map $\GH\to\GH$.

\end{thm}



\section{Approximations and almost isometries}\label{sec:GH-approx}

\begin{thm}{Definition}\label{ex:defGHR}
Let $\spc{X}$ and $\spc{Y}$ be two metric spaces.
A relation $\approx$ between points in $\spc{X}$ and $\spc{Y}$ is called \index{$\eps$-approximation}\emph{$\eps$-approximation} if the following conditions hold:
\begin{itemize}
\item For any $x\in  \spc{X}$ there is $y\in \spc{Y}$ such that $x\approx y$.
\item For any $y\in  \spc{Y}$ there is $x\in \spc{X}$ such that $x\approx y$.
\item If $x\approx y$ and $x'\approx y'$ for some $x, x'\in  \spc{X}$ and $y,y'\in \spc{Y}$, then 
\[\bigl|\dist{x}{x'}{\spc{X}}-\dist{y}{y'}{\spc{Y}}\bigr|<2\cdot\eps.\]
\end{itemize}

\end{thm}

\begin{thm}{Exercise}\label{ex:H-R}
Let $\spc{X}$ and $\spc{Y}$ be two compact metric spaces.
Show that
\[\dist{\spc{X}}{\spc{Y}}{\GH}<\eps\]
if and only if there is an $\eps$-approximation between $\spc{X}$ and $\spc{Y}$.

In other words $\dist{\spc{X}}{\spc{Y}}{\GH}$ is the greatest lower bound of values $\eps>0$ such that  there is an $\eps$-approximation between $\spc{X}$ and $\spc{Y}$.
\end{thm}

\parit{Proof of \ref{GH-4}.}
Suppose that 
\begin{itemize}
\item $\approx_1$ is a relation between points in $\spc{X}$ and $\spc{Y}$,
\item $\approx_2$ is a relation between points in $\spc{Y}$ and $\spc{Z}$.
\end{itemize}
Consider the relation $\approx_3$ between points in $\spc{X}$ and $\spc{Z}$ such that
$x\approx_3 z$ if and only if there is $y\in  \spc{Y}$ such that 
$x\approx_1 y$ and $y\approx_2 z$.

It is straightforward to check that if $\approx_1$ is an $\eps_1$-approximation and $\approx_2$ is an $\eps_2$-approximation, then $\approx_3$ is an $(\eps_1+\eps_2)$-approximation.

Applying \ref{ex:H-R}, we get that if 
\[|\spc{X}-\spc{Y}|_{\GH}<\eps_1
\quad\text{and}\quad
|\spc{Y}-\spc{Z}|_{\GH}<\eps_2,
\]
then 
\[|\spc{X}-\spc{Z}|_{\GH}<\eps_1+\eps_2.\]
Hence \ref{GH-4} follows.
\qeds

The following weakened version of isometry is closely related to $\eps$-approximations.

\begin{thm}{Definition} Let $\spc{X}$ and $\spc{Y}$ be metric spaces and $\eps>0$. 
A  map\footnote{possibly noncontinuous} $f\: \spc{X} \z\to \spc{Y}$ is called an \index{almost isometry}\emph{$\eps$-isometry} 
if $f(\spc{X})$ is an $\eps$-net in $\spc{Y}$ and
\[\bigl|\dist{x}{x'}{\spc{X}}-\dist{f(x)}{f(x')}{\spc{Y}}\bigr|<\eps.\]
for any $x,x'\in \spc{X}$.
\end{thm}

\begin{thm}{Exercise}\label{ex:eps-isom}
Let $\spc{X}$ and $\spc{Y}$ be compact metric spaces.

\begin{subthm}{ex:eps-isom:GH>isom}
If $\dist{\spc{X}}{\spc{Y}}{\GH}<\eps$, then there is a $2\cdot\eps$-isometry $f\:\spc{X}\to\spc{Y}$.
\end{subthm}

\begin{subthm}{ex:eps-isom:isom>GH}
If there is an $\eps$-isometry $f\:\spc{X}\to\spc{Y}$, then $\dist{\spc{X}}{\spc{Y}}{\GH}<\eps$.
\end{subthm}

\end{thm}

\section{Optimal realization}\label{sec:extfun=GH}

Note that
\[\dist{\spc{X}'}{\spc{Y}'}{\Haus\spc{W}}\ge \dist{\spc{X}}{\spc{Y}}{\GH},\]
where $\spc{X}$, $\spc{Y}$, $\spc{X}'$, $\spc{Y}'$, and $\spc{W}$ are as in \ref{def:GH}.
The following proposition states that equality holds for some choice of $\spc{X}'$, $\spc{Y}'$, and $\spc{W}$.

\begin{thm}{Proposition}\label{prop:GH=H}
For any two compact metric spaces $\spc{X}$ and $\spc{Y}$ there is a metric space $\spc{W}$
with subsets $\spc{X}'$ and $\spc{Y}'$ such that 
$\spc{X}'\iso\spc{X}$, $\spc{Y}'\iso\spc{Y}$, and 
\[\dist{\spc{X}'}{\spc{Y}'}{\Haus\spc{W}}=\dist{\spc{X}}{\spc{Y}}{\GH}.\]
\end{thm}

Let us introduce the so-called \textit{appropriate functions} and use them in a reinterpretation of Gromov--Hausdorff distance.

Suppose $\spc{X}$, $\spc{Y}$, $\spc{X}'$, $\spc{Y}'$, and $\spc{W}$ are as in \ref{def:GH}.
By passing to the subspace $\spc{X}'\cup\spc{Y}'$ in $\spc{W}$, we can assume that $\spc{W}=\spc{X}'\cup\spc{Y}'$.
Note that in this case the metric on $\spc{W}$ is completely determined by the function 
\[f(x,y)=\dist{x}{y}{\spc{W}};\]
a function $f\:\spc{X}\times \spc{Y}\to\RR$ that can appear this way will be called \index{appropriate function}\emph{appropriate}.

Note that a function $f\:\spc{X}\times\spc{Y}\to\RR$ is appropriate if and only if
$x\mapsto f(x,y)$ and $y\mapsto f(x,y)$ are extension functions;
that is, if
\[
\begin{aligned}
f(x,y)+f(x,y')
&\ge \dist{y}{y'}{\spc{Y}}\ge |f(x,y)-f(x,y')|,
\\
f(x,y)+f(x',y)
&\ge \dist{x}{x'}{\spc{X}}\ge |f(x,y)-f(x',y)|;
\end{aligned}
\eqlbl{eq:appropriate}
\]
for any $x,x',\in\spc{X}$ and  $y,y'\in\spc{X}$;
see \ref{sec:Extension property}.
In other words, the following defines a pseudometric on $\spc{X}\sqcup\spc{Y}$
\[\dist{x}{y}{\spc{X}\sqcup\spc{Y}}=
\begin{cases}
\dist{x}{y}{\spc{X}}&\text{if\ } x,y\in \spc{X},
\\
\dist{x}{y}{\spc{Y}}&\text{if\ } x,y\in \spc{Y},
\\
f(x,y)&\text{if\ } x\in \spc{X}\ \text{and}\ y\in \spc{Y},
\end{cases}
\]
and the corresponding metric space $\spc{W}$ contains isometric copies of $\spc{X}$ and $\spc{Y}$.

Given an appropriate function $f\:\spc{X}\times\spc{Y}\to\RR$, set 
\begin{align*}
a_f&=\max_{x\in \spc{X}}\{\min_{y\in\spc{Y}} \{f(x,y)\}\},
\\
b_f&=\max_{y\in \spc{Y}}\{\min_{x\in\spc{X}} \{f(x,y)\}\}.
\end{align*}

\begin{thm}{Observation}\label{obs:GH=min-appropriate}
If $\spc{X}$, $\spc{Y}$, $\spc{X}'$, $\spc{Y}'$, and $\spc{W}$ as above then
\[\dist{\spc{X}'}{\spc{Y}'}{\Haus\spc{W}}=\inf_f\{a_f,b_f\}.\]

\end{thm}

\parit{Proof of \ref{prop:GH=H}.}
By \ref{eq:appropriate}, any appropriate functions $f\:\spc{X}\times\spc{Y}\to\RR$ is $2$-Lipschitz.
Observe that the functional $f\mapsto \min\{a_f,b_f\}$ is continuous.
Applying the Arzelà--Ascoli theorem, we can get an  appropriate function $f\:\spc{X}\times\spc{Y}\to\RR$ 
with minimal possible value $\min\{a_f,b_f\}$.
It remains to apply \ref{obs:GH=min-appropriate}.
\qeds

\begin{thm}{Exercise}\label{ex:XYZ}
Construct three compact metric spaces $\spc{X}$, $\spc{Y}$, and $\spc{Z}$
such that for any metric space $\spc{W}$
with subsets $\spc{X}'$, $\spc{Y}'$, and $\spc{Z}'$ such that 
$\spc{X}'\iso\spc{X}$, $\spc{Y}'\iso\spc{Y}$, and $\spc{Z}'\iso\spc{Z}$
at least one of the following three inequalities is strict
\begin{align*}
\dist{\spc{X}'}{\spc{Y}'}{\Haus\spc{W}}&\ge \dist{\spc{X}}{\spc{Y}}{\GH},
\\
\dist{\spc{Y}'}{\spc{Z}'}{\Haus\spc{W}}&\ge\dist{\spc{Y}}{\spc{Z}}{\GH},
\\
\dist{\spc{Z}'}{\spc{X}'}{\Haus\spc{W}}&\ge\dist{\spc{Z}}{\spc{X}}{\GH}.
\end{align*}
\end{thm}

\section{Convergence}

The Gromov--Hausdorff metric is used to define \index{Gromov--Hausdorff convergence}\emph{Gromov--Hausdorff convergence}.
Namely, a sequence of compact metric spaces $\spc{X}_n$ converges to compact metric spaces $\spc{X}_\infty$ in the sense of Gromov--Hausdorff if 
\[\dist{\spc{X}_n}{\spc{X}_\infty}{\GH}\to 0\quad\text{as}\quad n\to\infty.\]

This convergence is more important than the metric ---
in all applications, we use only the topology on $\GH$
and we do not care about the particular value of Gromov--Hausdorff distance between spaces.
The following observation follows from \ref{ex:eps-isom}:

\begin{thm}{Observation}\label{obs:GH-e-isom}
A sequence of compact metric spaces $(\spc{X}_n)$ converges to  $\spc{X}_\infty$ in the sense of Gromov--Hausdorff if and only if there is a sequence $\eps_n\to0+$
and an $\eps_n$-isometry $f_n\:\spc{X}_n\to \spc{X}_\infty$ for each $n$.
\end{thm}

In the following exercises, \textit{convergence} is understood in the sense of Gromov--Hausdorff.

\begin{thm}{Exercise}\label{ex:GH-SC}
\begin{subthm}{ex:GH-SC:circle}
Show that a sequence of compact simply-connected length spaces cannot converge to a circle.
\end{subthm}

\begin{subthm}{ex:GH-SC:nonsc-limit}
Construct a sequence of compact simply-connected length spaces that converges to a compact non-simply-connected space.
\end{subthm}
\end{thm}

\begin{thm}{Exercise}\label{ex:sphere-to-ball}
\begin{subthm}{ex:sphere-to-ball:2}
Show that a sequence of length metrics on the 2-sphere cannot converge to the unit disk.
\end{subthm}

\begin{subthm}{ex:sphere-to-ball:3}
Construct a sequence of length metrics on the 3-sphere that converges to a unit 3-ball.
\end{subthm}

\end{thm}

\section{Uniformly totally bonded families}

\begin{thm}{Definition}\label{def:utb}
A family $\spc{Q}$ of (isometry classes) of compact metric spaces is called  \index{uniformly totally bonded family}\emph{uniformly totally bonded} if it meets the following two conditions:

\begin{subthm}{}
spaces in $\spc{Q}$ have uniformly bounded diameters; that is, there is $D\in\RR$ such that
\[\diam\spc{X}\le D\]
for any space $\spc{X}$ in $\spc{Q}$.
\end{subthm}

\begin{subthm}{}
For any $\eps>0$ there is $n\in\NN$ such that any space $\spc{X}$ in $\spc{Q}$ admits an $\eps$-net with at most $n$ points.
\end{subthm}
\end{thm}

\begin{thm}{Exercise}\label{ex:utb+pack}
Let $\spc{Q}$ be a family of compact spaces with uniformly bounded diameters.
Show that $\spc{Q}$ is uniformly totally bonded if for any $\eps>0$ there is $n\in\NN$ such that 
\[\pack_\eps\spc{X}\le n\]
for any space $\spc{X}$ in $\spc{Q}$.
\end{thm}


Fix a real constant $C$.
A Borel measure $\mu$ on a metric space $\spc{X}$ is called \index{doubling space}\emph{$C$-doubling} if
\[\mu[\oBall(p,2\cdot r)]< C\cdot\mu[\oBall(p,r)]\]
for any point $p\in \spc{X}$ and any $r>0$.
A Borel measure is called \index{doubling measure}\emph{doubling} if it is {}\emph{$C$-doubling} for some real constant $C$.

\begin{thm}{Exercise}\label{pr:doubling}
Let $\spc{Q}(C,D)$ be the set of all the compact metric spaces with diameter at most $D$ that admit a $C$-doubling measure.
Show that $\spc{Q}(C,D)$ is totally bounded.
\end{thm}

Given two metric spaces $\spc{X}$ and $\spc{Y}$, we will write $\spc{X}\le \spc{Y}$ if there is a distance-noncontracting map $f\:\spc{X}\to \spc{Y}$;
that is, if 
$$ |x-x'|_{\spc{X}}\le|f(x)-f(x')|_{\spc{Y}}$$
for any $x,x'\in \spc{X}$.

\begin{thm}{Exercise}\label{pr:under}

\begin{subthm}{pr:under:if}
Let $\spc{Y}$ be a compact metric space.
Show that the set of all spaces $\spc{X}$ such that $\spc{X}\le\spc{Y}$
is uniformly totally bounded.
\end{subthm}

\begin{subthm}{pr:under:only-if}
Show that for any uniformly totally bounded set $\spc{Q}\subset\GH$ there is a compact space $\spc{Y}$
such that $\spc{X}\le\spc{Y}$ for any $\spc{X}$ in $\spc{Q}$.
\end{subthm}

\end{thm}

\section{Gromov selection theorem}

The following theorem is analogous to Blaschke selection theorems (\ref{thm:compact+Hausdorff}).

\begin{thm}{Gromov selection theorem}\label{thm:gromov-compactness}
Let $\spc{Q}$ be a closed subset of $\GH$.
Then $\spc{Q}$ is compact if and only if the spaces in $Q$ are uniformly totally bounded.
\end{thm}

\begin{thm}{Lemma}\label{lem:GH-complete}
The space $\GH$ is complete.
\end{thm}


Let us define gluing of metric spaces that will be used in the proof of the lemma.

Suppose 
$\spc{U}$ and $\spc{V}$ are metric spaces 
with isometric closed sets $A\subset\spc{U}$ and $A'\subset\spc{V}$;
let $\iota\:A\to A'$ be an isometry.
Consider the space $\spc{W}$ of all equivalence classes in $\spc{U}\sqcup\spc{V}$ with the equivalence relation given by $a\sim\iota(a)$ for any $a\in A$.

It is straightforward to check that the following defines a metric on~$\spc{W}$:
\begin{align*}
\dist{u}{u'}{\spc{W}}&\df\dist{u}{u'}{\spc{U}}
\\
\dist{v}{v'}{\spc{W}}&\df\dist{v}{v'}{\spc{V}}
\\
\dist{u}{v}{\spc{W}}&\df\min\set{\dist{u}{a}{\spc{U}}+\dist{v}{\iota(a)}{\spc{V}}}{a\in A}
\end{align*}
where $u,u'\in \spc{U}$ and $v,v'\in \spc{V}$.

The  space $\spc{W}$ is called the \index{gluing}\emph{gluing} of $\spc{U}$ and  $\spc{V}$ along~$\iota$; briefly, we can write
$\spc{W}=\spc{U}\sqcup_\iota\spc{V}$.
If one applies this construction to two copies of one space $\spc{U}$ with a set $A\subset \spc{U}$ and the identity map $\iota\:A\to A$, then the obtained space is called the \index{doubling}\emph{doubling} of $\spc{U}$ along~$A$; this space can be denoted by $\sqcup_A^2\spc{U}$.

Note that the inclusions $\spc{U}\hookrightarrow \spc{W}$ and $\spc{V}\hookrightarrow \spc{W}$ are distance preserving.
Therefore we can and will consider $\spc{U}$ and $\spc{V}$ as the subspaces of $\spc{W}$;
this way the subsets $A$ and $A'$ will be identified and denoted further by~$A$.
Note that $A=\spc{U}\cap \spc{V}\subset \spc{W}$.

\parit{Proof.}
Let $\spc{X}_1,\spc{X}_2,\dots$ be a Cauchy sequence in $\GH$.
Passing to a subsequence if necessary, 
we can assume that $|\spc{X}_n-\spc{X}_{n+1}|_{\GH}<\tfrac1{2^n}$ for each~$n$.
In particular, for each $n$ there is a metric space $\spc{V}_n$ with distance preserving inclusions $\spc{X}_n\hookrightarrow \spc{V}_n$ and $\spc{X}_{n+1}\hookrightarrow \spc{V}_n$ such that
\[|\spc{X}_n-\spc{X}_{n+1}|_{\Haus\spc{V}_n}<\tfrac1{2^n}\]
for each $n$.
Moreover, we may assume that $\spc{V}_n=\spc{X}_n\cup\spc{X}_{n+1}$.

Let us glue $\spc{V}_1$ to $\spc{V}_2$ along $\spc{X}_2$;
to the obtained space glue $\spc{V}_3$ along $\spc{X}_3$, and so on.
The obtained metric space $\spc{W}$
has an underlying set formed by the disjoint union of all $\spc{X}_n$ such that each inclusion $\spc{X}_n\z\hookrightarrow\spc{W}$ is distance preserving and
\[|\spc{X}_n-\spc{X}_{n+1}|_{\Haus\spc{W}}<\tfrac1{2^n}\]
for each $n$.
In particular,
\[|\spc{X}_m-\spc{X}_n|_{\Haus\spc{W}}<\tfrac1{2^{n-1}}\eqlbl{eq:|x_m-X_n|}\] 
if $m>n$.

Denote by $\bar{\spc{W}}$ the completion of $\spc{W}$.
Observe that the union $\spc{X}_1\z\cup \spc{X}_2\cup\z\dots\cup \spc{X}_n$ is compact and \ref{eq:|x_m-X_n|} implies that it forms a $\tfrac1{2^{n-1}}$-net in $\bar{\spc{W}}$.
Whence $\bar{\spc{W}}$ is compact; see \ref{totally-bounded} and \ref{ex:compact-net}.

Applying the Blaschke selection theorem (\ref{thm:compact+Hausdorff}),
we can pass to a subsequence of $\spc{X}_n$ that converges in $\Haus\bar{\spc{W}}$; denote its limit by $\spc{X}_\infty$.
It remains to observe that $\spc{X}_\infty$ is the Gromov--Hausdorff limit of $\spc{X}_n$.
\qeds

\parit{Proof of \ref{thm:gromov-compactness}; only-if part.}
Suppose that there is no sequence $\eps_n\to0$ as described in \ref{def:utb}.
Observe that in this case
there is a sequence of spaces $\spc{X}_n\in\spc{Q}$ such that 
$$\pack_\delta \spc{X}_n\to\infty
\quad\text{as}\quad
n\to\infty$$
for some fixed $\delta>0$.

Since $\spc{Q}$ is compact, 
this sequence has a partial limit, say $\spc{X}_\infty\in\spc{Q}$.
Observe that $\pack_{\delta} \spc{X}_\infty=\infty$.
Therefore, $\spc{X}_\infty$ is not compact --- a contradiction.

\parit{If part.}
Given a positive integer $n$ consider the set of all metric spaces $\spc{W}_n$
with the number of points at most $n$ and diameter $\le D$.
Note that $\spc{W}_n$ is a compact set in $\GH$ for each $n$.

Let $D$ and $n=n(\eps)$ be as in the definition of uniformly totally bonded families (\ref{def:utb}).

Note that an $\eps$-net of any $\spc{X}\in\spc{Q}$ belongs to $\spc{W}_{n(\eps)}$.
Therefore, $\spc{W}_{n(\eps)}$ is a compact $\eps$-net of $\spc{Q}$ for any $\eps>0$.
Since $\spc{Q}$ is closed in a complete space $\GH$, it implies that $\spc{Q}$ is compact.
\qeds

\begin{thm}{Exercise}\label{ex-GH-length}
Show that the space $\GH$ is 

\begin{subthm}{ex-GH-length:length}
length,
\end{subthm}

\begin{subthm}{ex-GH-length:geodesic}
geodesic.
\end{subthm}

\end{thm}

\begin{thm}{Exercise}\label{ex:GH-po}
For two metric spaces $\spc{X}$ and $\spc{Y}$,
we write $\spc{X}\le \spc{Y}+\eps$ if
there is a map $f\:\spc{X}\to \spc{Y}$ such that 
\[\dist{x}{x'}{\spc{X}}\le \dist{f(x)}{f(x')}{\spc{Y}}+\eps\]
for any $x,x'\in \spc{X}$.

\begin{subthm}{ex:GH-po:a}
Show that 
$$\dist{\spc{X}}{\spc{Y}}{\GH'}=\inf\set{\eps>0}{\spc{X}\le \spc{Y}+\eps
\quad\text{and}\quad
\spc{Y}\le \spc{X}+\eps}$$
defines a metric on the space of (isometry classes) of compact metric spaces.
\end{subthm}

\begin{subthm}{ex:GH-po:b}
Moreover $\dist{*}{*}{\GH'}$ is equivalent to the Gromov--Hausdorff metric;
that is,
$$|\spc{X}_n-\spc{X}_\infty|_{\GH}\to 0 
\quad\iff\quad 
\dist{\spc{X}_n}{\spc{X}_\infty}{\GH'}\to 0$$ 
as $n\to\infty$.
\end{subthm}
\end{thm}

\section{Universal ambient space}

Recall that a metric space is called universal if it contains an isometric copy of any separable metric space (in particular, any compact metric space).
Examples of universal spaces include $\spc{U}_\infty$ --- the Urysohn space and $\ell^\infty$ --- the space of bounded infinite sequences with the metric defined by $\sup$-norm; see \ref{prop:sep-in-urys} and \ref{ex:frechet}.

The following proposition says that the space $\spc{W}$ in Definition~\ref{def:GH} can be exchanged to a fixed universal space.

\begin{thm}{Proposition}\label{prop:GH-with-fixed-Z}
Let $\spc{U}$ be a universal space.
Then for any compact metric spaces $\spc{X}$ and $\spc{Y}$ we have
$$|\spc{X}-\spc{Y}|_{\GH} = \inf \{|\spc{X}'-\spc{Y}'|_{\Haus\spc{U}}\}$$ 
where the greatest lower bound is taken over all pairs of sets $\spc{X}'$ and $\spc{Y}'$ in $\spc{U}$
which isometric to  $\spc{X}$ and $\spc{Y}$ respectively.  
\end{thm}




\parit{Proof of \ref{prop:GH-with-fixed-Z}.}
By the definition (\ref{def:GH}), we have that 
\[|\spc{X}-\spc{Y}|_{\GH} \le \inf \{|\spc{X}'-\spc{Y}'|_{\Haus\spc{U}}\};\]
it remains to prove the opposite inequality.

Suppse $|\spc{X}-\spc{Y}|_{\GH}<\eps$;
let $\spc{X}'$, $\spc{Y}'$ and $\spc{W}$ be as in \ref{def:GH}.
We can assume that $\spc{W}=\spc{X}'\cup\spc{Y}'$;
otherwise pass to the subspace $\spc{X}'\cup\spc{Y}'$ of~$\spc{W}$.
In this case, $\spc{W}$ is compact;
in particular, it is separable.

Since $\spc{U}$ is universal, there is a distance-preserving embedding of $\spc{W}$ in $\spc{U}$;
let us keep the same notation for $\spc{X}'$, $\spc{Y}'$, and their images.
It follows that 
\[|\spc{X}'-\spc{Y}'|_{\Haus\spc{U}}<\eps,\]
--- hence the result.
\qeds

\begin{thm}{Exercise}\label{ex:GH-urysohn}
Let $\spc{U}_\infty$ be the Urysohn space.
Given two compact sets $A$ and $B$ in $\spc{U}_\infty$ define 
\[\|A-B\|=\inf\{|A-\iota(B)|_{\Haus\spc{U}_\infty}\},\]
where the greatest lower bound is taken for all isometrics $\iota$ of $\spc{U}_\infty$.
Show that $\|{*}\z-{*}\|$ defines a pseudometric%
\footnote{The value $\|A-B\|$ is called Hausdorff distance \index{Hausdorff distance up to isometry}\emph{up to isometry} from $A$ to $B$ in $\spc{U}_\infty$.}
on nonempty compact subsets of $\spc{U}_\infty$ and its corresponding metric space is isometric to $\GH$.
\end{thm}

\section{Remarks}

Suppose $\spc{X}_n\GHto \spc{X}_\infty$, then there is a metric on the disjoint union 
\[\bm{X}=\bigsqcup_{n\in \NN\cup\{\infty\}} \spc{X}_n\] 
that satisfies the following property:

\begin{thm}{Property}\label{propery:GH}
The restriction of metric on each $\spc{X}_n$ and $\spc{X}_\infty$ coincides with its original metric, 
and $\spc{X}_n\Hto \spc{X}_\infty$ as subsets in $\bm{X}$.
\end{thm}

Indeed, since $\spc{X}_n\GHto \spc{X}_\infty$, there is a metric on $\spc{V}_n=\spc{X}_n\sqcup \spc{X}_\infty$ such that the restriction of metric on each $\spc{X}_n$ and $\spc{X}_\infty$ coincides with its original metric, and $\dist{\spc{X}_n}{\spc{X}_\infty}{\Haus\spc{V}_n}<\eps_n$ for some sequence $\eps_n\to 0$.
Gluing all $\spc{V}_n$ along $\spc{X}_\infty$, we obtain the required space $\bm{X}$.

In other words, the metric on $\bm{X}$ \textit{defines} the convergence $\spc{X}_n\z\GHto \spc{X}_\infty$.
This metric makes it possible to talk about limits of sequences $x_n\in \spc{X}_n$ as $n\to\infty$, as well as weak limits of a sequence of Borel measures $\mu_n$ on $\spc{X}_n$ and so on.

For that reason, it is useful to define \index{Hausdorff convergence}\emph{convergence} by specifying the metric on $\bm{X}$ that satisfies the property
for the variation of Hausdorff convergence described in Section~\ref{sec:H-variation}.

This approach is more flexible;
in particular, it can be used to define Gromov--Hausdorff convergence of arbitrary metric spaces (not necessarily compact).
A limit space for this generalized convergence is not uniquely defined.
For example, if each space $\spc{X}_n$ in the sequence is isometric to the half-line, then its limit might be isometric to the half-line or the whole line.
The first convergence is evident and the second could be guessed from the diagram.

\begin{figure}[ht!]
\vskip-0mm
\centering
\includegraphics{mppics/pic-500}
\end{figure}

Often the isometry class of the limit can be fixed by marking a point $p_n$ in each space $\spc{X}_n$, it is called \index{pointed convergence}\emph{pointed Gromov--Hausdorff convergence} --- we say that $(\spc{X}_n,p_n)$ converges to $(\spc{X}_\infty,p_\infty)$ if there is a metric on $\bm{X}$ as in \ref{propery:GH} such that $\spc{X}_n\Hto \spc{X}_\infty$ and $p_n\to p_\infty$.
For example, the sequence $(\spc{X}_n,p_n)=(\RR_+,0)$ converges to $(\RR_+,0)$, while $(\spc{X}_n,p_n)=(\RR_+,n)$ converges to $(\RR,0)$.

The pointed convergence works nicely for proper metric spaces;
the following theorem is an analog of Gromov's selection theorem for this convergence.

\begin{thm}{Theorem}\label{thm:pointed-gromov-compactness}%
Let $\spc{Q}$ be a set of isometry classes of pointed proper metric spaces
$(\spc{X},p)$.
Assume that for any $R>0$, the $R$-balls in the spaces centered at the marked points form a uniformly totally bounded family of spaces.
Then $\spc{Q}$ is precompact with respect to the pointed Gromov--Hausdorff convergence. 
\end{thm}

\chapter{Ultralimits}

Ultralimits provide a very general way to pass to a limit that always works.
It use a set-theoretical construction --- the so called ulrafilter.

In geometry, ultralimits are used only as a canonical way to pass to a convergent subsequence.
It is useful thing in the proofs where one needs to repeat ``pass to convergent subsequence'' too many times.

This lecture is based on the introduction to the paper of Bruce Kleiner and Bernhard Leeb \cite{kleiner-leeb}.

\section{Ultrafilters}

Recall that $\NN$ denotes the set of natural numbers, $\NN=\{1,2,\dots\}$

\begin{thm}{Definition}
A finitely additive measure $\omega$ 
on  $\NN$ 
is called an \index{ultrafilter}\emph{ultrafilter} if it satisfies 
\begin{subthm}{}
$\omega(S)=0$ or $1$ for any subset $S\subset \NN$.
\end{subthm}
An ultrafilter $\omega$ is called 
\emph{nonprinciple}\index{ultrafilter!nonprinciple ultrafilter}\index{nonprinciple ultrafilter} if in addition 
\begin{subthm}{}
$\omega(F)=0$ for any finite subset $F\subset \NN$.
\end{subthm}
\end{thm}

If $\omega(S)=0$ for some subset $S\subset \NN$,
we say that $S$ is \index{$\omega$-small}\emph{$\omega$-small}. 
If $\omega(S)=1$, we say that $S$ contains \index{$\omega$-almost all}\emph{$\omega$-almost all} elements of $\NN$.

\parbf{Classical definition.}
More commonly, a nonprinciple ultrafilter is defined as a collection, say $\mathfrak{F}$, of sets in $\NN$ such that
\begin{enumerate}
\item\label{filter:supset} if $P\in \mathfrak{F}$ and $Q\supset P$, then $Q\in \mathfrak{F}$,
\item\label{filter:cap} if $P, Q\in \mathfrak{F}$, then $P\cap Q\in \mathfrak{F}$,
\item\label{filter:ultra} for any subset $P\subset\NN$, either $P$ or its complement is an element of $\mathfrak{F}$.
\item\label{filter:non-prin} if $F\subset \NN $ is finite, then $F\notin \mathfrak{F}$.
\end{enumerate}
Setting $P\in\mathfrak{F}\Leftrightarrow\omega(P)=1$ makes these two definitions equivalent.

A nonempty collection of sets $\mathfrak{F}$ that does not include the empty set and satisfies only conditions \ref{filter:supset} and \ref{filter:cap} is called a \index{filter}\emph{filter}; 
if in addition $\mathfrak{F}$ satisfies Condition~\ref{filter:ultra} it is called an \index{ultrafilter}\emph{ultrafilter}.
From Zorn's lemma, it follows that every filter contains an ultrafilter.
Thus there is an ultrafilter $\mathfrak{F}$ contained in the filter of all complements of finite sets; clearly this $\mathfrak{F}$ is nonprinciple.


\parbf{Stone--\v{C}ech compactification.}
Given a set $S\subset \NN$, consider subset $\Omega_S$ of all ultrafilters $\omega$ such that $\omega(S)=1$.
It is straightforward to check that the sets $\Omega_S$ for all $S\subset \NN$ form a topology on the set of ultrafilters on $\NN$. 
The obtained space is called \index{Stone--\v{C}ech compactification}\emph{Stone--\v{C}ech compactification} of $\NN$;
it is usually denoted as $\beta\NN$.

Let $\omega_n$ denotes the principle ultrafilter such that $\omega_n(\{n\})=1$; that is, $\omega_n(S)=1$ if and only if $n\in S$.
Note that $n\mapsto\omega_n$ defines a natural embedding $\NN\hookrightarrow\beta\NN$. 
Using the described embedding, we can (and will) consider $\NN$ as a subset of $\beta\NN$.

The space $\beta\NN$ is the maximal compact Hausdorff space that contains $\NN$  as an everywhere dense subset.
More precisely, for any compact Hausdorff space $\spc{X}$ 
and a map $f\:\NN\to \spc{X}$ there is unique continuous map $\bar f\:\beta\NN\to X$ such that the restriction $\bar f|_\NN$ coincides with $f$. 

\section{Ultralimits of points}
\label{ultralimits}\index{ultralimit}

Further we will need the existence of a nonprinciple  ultrafilter $\omega$,
which we fix once and for all.

Assume $(x_n)$ is a sequence of points in a metric space $\spc{X}$. 
Let us define the \index{$\omega$-limit}\emph{$\omega$-limit} of $(x_n)$ as the point $x_\omega$ 
such that for any $\eps>0$, $\omega$-almost all elements of $(x_n)$ lie in $\oBall(x_\omega,\eps)$; 
that is,
\[\omega\set{n\in\NN}{\dist{x_\omega}{x_n}{}<\eps}=1.\]
In this case, we will write 
\[x_\omega=\lim_{n\to\omega} x_n
\ \ \text{or}\ \ 
x_n\to x_\omega\ \text{as}\ n\to\omega.\]

For example if $\omega$ is the principle ultrafilter such that $\omega(\{n\})=1$ for some $n\in\NN$, then
$x_\omega=x_n$.

Alternatively, the sequence $(x_n)$ can be regarded as a map $\NN\to\spc{X}$.
In this case the map $\NN\to\spc{X}$ can be extended to a continuous map $\beta\NN\to\spc{X}$ from the Stone--\v{C}ech compactification of $\NN$.
Then the $\omega$-limit $x_\omega$ can be regarded as the image of $\omega$.

Note that $\omega$-limits of a sequence and its subsequence may differ.
For example, in general
\[\lim_{n\to\omega}x_n
\ne
\lim_{n\to\omega}x_{2\cdot n}.\] 

\begin{thm}{Proposition}\label{prop:ultra/partial}
Let $\omega$ be a nonprinciple ultrafilter.
Assume $(x_n)$ is a sequence of points in a metric space $\spc{X}$
and $x_n\to  x_\omega$ as $n\to\omega$.
Then $x_\omega$ is a partial limit of the sequence $(x_n)$;
that is, there is a subsequence $(x_n)_{n\in S}$ that converges to $x_\omega$ in the usual sense.
\end{thm}

\parit{Proof.}
Given $\eps>0$, 
set $S_\eps=\set{n\in\NN}{\dist{x_n}{x_\omega}{}<\eps}$.

Note that $\omega(S_\eps)=1$ for any $\eps>0$.
Since $\omega$ is nonprinciple, the set $S_\eps$ is infinite.
Therefore we can choose an increasing sequence $(n_k)$
such that $n_k\in S_{\frac1k}$ for each $k\in \NN$.
Clearly $x_{n_k}\to x_\omega$ as $k\to\infty$.
\qeds

The following proposition 
is analogous to the statement that any sequence in a compact metric space 
has a convergent subsequence;
it can be proved the same way.

\begin{thm}{Proposition}\label{prop:ultra/compact}
Let $\spc{X}$ be a compact metric space.
Then
any sequence of points $(x_n)$ in $\spc{X}$ has unique $\omega$-limit $x_\omega$.

In particular, a bounded sequence of real numbers has a unique $\omega$-limit.
\end{thm}

The following lemma is an ultralimit analog of Cauchy convergence test.

\begin{thm}{Lemma}\label{lem:X-X^w}
Let $(x_n)$ be a sequence of points in a complete space $\spc{X}$. 
Assume for each subsequence $(y_n)$ of $(x_n)$, 
the $\omega$-limit 
\[y_\omega=\lim_{n\to\omega}y_{n}\in \spc{X}\]
is defined and does not depend on the choice of subsequence, 
then the sequence $(x_n)$ converges in the usual sense.
\end{thm}

\parit{Proof.} If $(x_n)$ is not a Cauchy sequence,
then for some $\eps>0$, there is a subsequence $(y_n)$ of $(x_n)$ such that $\dist{x_n}{y_n}{}\ge\eps$ for all $n$.

It follows that $\dist{x_\omega}{y_\omega}{}\ge \eps$, a contradiction.\qeds


\section{Ultralimits of spaces}\label{sec:Ultralimit of spaces}

Recall that $\omega$ denotes a nonprinciple ultrafilter on the set of natural numbers.

Let $\spc{X}_n$ be a sequence of metric spaces.
Consider all sequences of points $x_n\in \spc{X}_n$.
On the set of all such sequences,
define a pseudometric by
\[\dist{(x_n)}{(y_n)}{}
=
\lim_{n\to\omega} \dist{x_n}{y_n}{\spc{X}_n}.
\eqlbl{eq:olim-dist}\]
Note that the $\omega$-limit on the right hand side is always defined 
and takes a value in $[0,\infty]$. 

Set $\spc{X}_\omega$ to be the corresponding metric space; 
that is, the underlying set of $\spc{X}_\omega$ is formed by classes of equivalence of sequences of points $x_n\in\spc{X}_n$ 
defined by 
\[(x_n)\sim(y_n)
\ \Leftrightarrow\ 
\lim_{n\to\omega} \dist{x_n}{y_n}{}=0\]
and the distance is defined by \ref{eq:olim-dist}.

The space $\spc{X}_\omega$ is called \index{$\omega$-limit space}\emph{$\omega$-limit} of $\spc{X}_n$.
Typically  $\spc{X}_\omega$ will denote the  
$\omega$-limit of sequence $\spc{X}_n$;
we may also write  
\[\spc{X}_n\to\spc{X}_\omega\ \ \text{as}\ \  n\to\omega\ \ \text{or}\ \ \spc{X}_\omega=\lim_{n\to\omega}\spc{X}_n.\]

Given a sequence $x_n\in \spc{X}_n$,
we will denote by $x_\omega$ its equivalence class which is a point in $\spc{X}_\omega$;
equivalently we will write
\[x_n\to x_\omega \ \ \text{as}\ \  n\to\omega\ \ \text{or}\ \ x_\omega=\lim_{n\to\omega} x_n.\]

\begin{thm}{Observation}\label{obs:ultralimit-is-complete}
The $\omega$-limit of any sequence of metric spaces is complete. 
\end{thm}

\parit{Proof.}
Let $\spc{X}_n$ be a sequence of metric spaces and $\spc{X}_n\to\spc{X}_\omega$ as $n\to\omega$.

Fix a Cauchy sequence $x_{m}\in \spc{X}_\omega$.
Passing to a subsequence we can assume that $\dist{x_m}{x_{m-1}}{\spc{X}_\omega}<\tfrac1{2^m}$ for any $m$.

Let us choose double sequence $x_{n,m}\in \spc{X}_n$ such that for any fixed $m$ we have $x_{n,m}\to x_m$ as $n\to\omega$.
Note that $\dist{x_{n,m}}{x_{n,m-1}}{}<\tfrac1{2^m}$ for $\omega$-almost all $n$.
It follows that we can choose a nested sequence of sets 
\[\NN= S_1\supset S_2\supset\dots\] 
such that 
\begin{itemize}
\item $\omega(S_m)=1$ for each $m$, 
\item $k\ge m$ for any $k\in S_m$, and
\item if $n\in S_m$, then 
\[\dist{x_{n,m}}{x_{n,m-1}}{}<\tfrac1{2^m}\]
\end{itemize}

Consider the sequence $y_n=x_{n,m(n)}$, where $m(n)$ is the largest value such that $m(n)\in S_{m}$.
Denote by $y\in \spc{X}_\omega$ its $\omega$-limit.

Observe that by construction $x_n\to y$ as $n\to \infty$.
Hence the statement follows.
\qeds

\begin{thm}{Observation}\label{obs:ultralimit-is-geodesic}
The $\omega$-limit of any sequence of length spaces is geodesic. 
\end{thm}

\parit{Proof.}
If $\spc{X}_n$ is a sequence length spaces, then for any sequence of pairs $x_n, y_n\in X_n$ there is a sequence of $\tfrac1n$-midpoints $z_n$.

Let $x_n\to x_\omega$, $y_n\to y_\omega$ and $z_n\to z_\omega$ as $n\to \omega$.
Note that $z_\omega$ is a midpoint of $x_\omega$ and $y_\omega$ in $\spc{X}^\omega$.

By Observation~\ref{obs:ultralimit-is-complete}, $\spc{X}^\omega$ is complete.
Applying Lemma~\ref{lem:mid>geod} we get the statement.
\qeds


\begin{thm}{Exercise}\label{ex:lim(tree)}
Show that an ultralimit of metric trees is a metric tree. 
\end{thm}

\section{Ultrapower}

If all the metric spaces in the sequence are identical $\spc{X}_n=\spc{X}$, 
its $\omega$-limit 
$\lim_{n\to\omega}\spc{X}_n$
is denoted by $\spc{X}^\omega$
and called $\omega$-power of $\spc{X}$.



\begin{thm}{Exercise}\label{ex:ultrapower}
For any point $x\in \spc{X}$, consider the constant sequence $x_n=x$
and set $\iota(x)=\lim_{n\to\omega}x_n\in\spc{X}^\omega$.

\begin{subthm}{ex:ultrapower:a}
Show that $\iota\:\spc{X}\to\spc{X}^\omega$ is distance-preserving embedding. (So we can and will consider $\spc{X}$ as a subset of $\spc{X}^\omega$.)
\end{subthm}

\begin{subthm}{ex:ultrapower:compact} 
Show that $\iota$ is onto if and only if $\spc{X}$ compact.
\end{subthm}

\begin{subthm}{ex:ultrapower:proper} 
Show that if $\spc{X}$ is proper, then $\iota(\spc{X})$ forms a metric component of $\spc{X}^\omega$; that is, a subset of $\spc{X}^\omega$ that lie on finite distance from a given point.
\end{subthm}

\end{thm}

\begin{thm}{Observation}\label{obs:ultrapower-is-geodesic}
Let $\spc{X}$ be a complete metric space. 
Then $\spc{X}^\omega$ is geodesic space if and only if $\spc{X}$ is a length space.
\end{thm}

\parit{Proof.}
Assume $\spc{X}^\omega$ is geodesic space.
Then any pair of points $x,y\in \spc{X}$ has a midpoint $z_\omega\in\spc{X}^\omega$.
Fix a sequence of points $z_n\in  \spc{X}$ such that $z_n\to z_\omega$ as $n\to \omega$.

Note that 
$\dist{x}{z_n}{\spc{X}}\to \tfrac12\cdot \dist{x}{y}{\spc{X}}$
and 
$\dist{y}{z_n}{\spc{X}}\to \tfrac12\cdot \dist{x}{y}{\spc{X}}$
as 
$n\to\omega$.
In particular, for any $\eps>0$, the point $z_n$ is an $\eps$-midpoint of $x$ and $y$ for $\omega$-almost all $n$.
It remains to apply \ref{lem:mid>geod}.

The ``if''-part follows from \ref{obs:ultralimit-is-geodesic}.
\qeds

\begin{thm}{Exercise}\label{ex:two-geodesics-in-ultrapower}
Assume $\spc{X}$ is a complete length space 
and $p,q\in\spc{X}$ cannot be joined by a geodesic in $\spc{X}$.  
Then there are at least two distinct geodesics between $p$ and $q$ 
in the ultrapower $\spc{X}^\omega$.
\end{thm}

\begin{thm}{Exercise}
 Construct a proper metric space $\spc{X}$ such that $\spc{X}^\omega$ is not proper; that is, there is a point $p\in \spc{X}^\omega$ and $R<\infty$ such that the closed ball $\cBall[p,R]_{\spc{X}^\omega}$ is not compact.
\end{thm}

\section{Tangent and asymptotic spaces}

Choose a space $\spc{X}$ and a sequence of $\lambda_n>0$.
Consider the sequence of scalings $\spc{X}_n=\lambda_n\cdot\spc{X}=(\spc{X},\lambda_n\cdot\dist{*}{*}{\spc{X}})$.

Choose a point $p\in \spc{X}$ and denote by $p_n$ the corresponding point in $\spc{X}_n$.
Consider the $\omega$-limit $\spc{X}_\omega$ of $\spc{X}_n$ (one may denote it by $\lambda_\omega\cdot \spc{X}$);
set $p_\omega$ to be the $\omega$-limit of $p_n$.

If $\lambda_n\to 0$ as $n\to\omega$, then the metric component of $p_\omega$ in $\spc{X}_\omega$ is called \index{$\omega$-tangent space}\emph{$\omega$-tangent space} at $p$ and denoted by $\T_p^{\lambda_\omega}\spc{X}$ (or $\T_p^{\omega}\spc{X}$ if $\lambda_n=n$).\label{page:ultratangent space}

If $\lambda_n\to \infty$ as $n\to\omega$, then the metric component of $p_\omega$ in called \index{$\omega$-asymptotic space}\emph{$\omega$-asymptotic space}\footnote{Often it is called \emph{asymptotic cone} despite that it is not a cone in general; this name is used since in good cases it has a cone structure.}  and denoted by $\Asym\spc{X}$ or $\Asym^{\lambda_\omega}\spc{X}$.
Note that the space $\Asym\spc{X}$ and its point $p_\omega$ does not depend on the choice of $p\in \spc{X}$.

In general, the tangent and asymptotic spaces depend the sequence $\lambda_n$ and an nonprinciple ultrafiler $\omega$.
For nice spaces different choices may give the same space.

\begin{thm}{Exercise}\label{ex:ultraT}
Construct a metric space $\spc{X}$ with a point $p$ such that the tangent space
$\T_p^{\lambda_\omega}\spc{X}$ depends on the sequence $\lambda_n$ and/or ultrafilter $\omega$.
\end{thm}


\begin{thm}{Exercise}\label{ex:Asym(Lob)}
Let $\spc{L}$ be the Lobachevsky plane; $\spc{T}=\Asym\spc{L}$.

\begin{subthm}{ex:Asym(Lob):metric-tree}
Show that $\spc{T}$ is a complete metric tree.
\end{subthm}

\begin{subthm}{ex:Asym(Lob):continuum}
Show that $\spc{T}$ has {}\emph{continuum degree} at any point;
that is, for any point $t\in \spc{T}$ the set of connected components of the complement $\spc{T}\backslash\{t\}$ has cardinality continuum.
\end{subthm}

\begin{subthm}{ex:Asym(Lob):homogeneous}
Show that $\spc{T}$ is homogeneous; that is given two points $s,t\in \spc{T}$ there is an isometry of $\spc{T}$ that maps $s$ to $t$.
\end{subthm}

\begin{subthm}{ex:Asym(Lob):others}
Prove \ref{SHORT.ex:Asym(Lob):metric-tree}--\ref{SHORT.ex:Asym(Lob):homogeneous} if $\spc{L}$ is Lobachevsky space and/or for the infinite 3-regular%
\footnote{that is, degree of any vertex is 3.}
tree with unit edge. 
\end{subthm}


\end{thm}

As it shown in \cite{dyubina-polterovich}, the properties \ref{SHORT.ex:Asym(Lob):metric-tree} and \ref{SHORT.ex:Asym(Lob):continuum} describe the tree $\spc{T}$ up to isometry.
In particular, the asymptotic space of Lobachevsky plane does not depend on the choice of ultrafilter and the sequence $\lambda_n\to \infty$.


\section{Remarks}

A nonprinciple ultrafilter $\omega$ is called 
\emph{selective}\index{ultrafilter!selective ultrafilter}\index{selective ultrafilter} if for any partition of $\NN$ into sets $\{C_\alpha\}_{\alpha\in\IndexSet}$ such that $\omega(C_\alpha)\z=0$ for each $\alpha$, 
there is a set $S\subset \NN$ such that $\omega(S)=1$ and $S\cap C_\alpha$ is a one-point set for each $\alpha\in\IndexSet$.

The existence of a selective ultrafilter follows from the continuum hypothesis;
it was proved by Walter Rudin \cite{rudin}.

For a selective ultrafilter $\omega$, there is a stronger version of Proposition~\ref{prop:ultra/partial};
namely we can assume that the subsequence $(x_n)_{n\in S}$ can be chosen so that $\omega(S)=1$.
So, if needed, one may assume that the ultrafilter $\omega$ is chosen to be selective and use this stronger version of the proposition.

%\chapter{Metric plus measure}

\section{Borel sets}

Let us remind few definitions assuming knowleage of basic measure theory;
comprehensive treatments can be found in \cite{billingsley} and \cite{bogachev}.

Let $\spc{X}$ be a metric space.
\index{Borel set}\emph{Borel set} is any subset of $\spc{X}$ that can be formed from open sets using the countable union, countable intersection, and complement.
In other words, Borel sets form the minimal sigma-algebra that included open sets.

A measure on metric space will be always assumed to be \index{Borel measure}\emph{Borel};
that is, it is defined on the sigma-algebra of Borel sets.
A Borel measure can be uniquely determined by its values on all open (or closed) sets.

A measure $\mu$ on $\spc{X}$ is called \index{probability measure}\emph{probability measure} if $\mu\spc{X}=1$.

Recall that \index{delta-measure}\emph{delta-measure} is a probability measure with support at one point.
Delta-measure with support in $\{x\}$ will be denoted by~\index{$\delta_{x}$}$\delta_{x}$; so
\[\text{if}\quad x\in A,\quad\text{then}\quad \delta_x(A)=1,\quad\text{otherwise}\quad\delta_x(A)=0.\]

Let $\mu_n$ be a sequence of Borel measures on $\spc{X}$.
A measure $\mu_\infty$ is a \index{weak limit}\emph{weak limit} of $\mu_n$ if 
\[\int_{\spc{X}}f\cdot(\mu_n-\mu_\infty)\to0\gamma
\quad\text{as}\quad
n\to\infty
\]
for any continuous function $f\:\spc{X}\to \RR$.

Suppose $\mu$ is a measure on a metric space $\spc{X}$ and $f\:\spc{X}\to \spc{Y}$ is a measurable map;
that is, for any Borel set $B\subset \spc{Y}$, its inverse image $f^{-1}B$ is a Borel set in $\spc{Y}$.

Consider the unit interval with its Lebesgue mesure.
If $\spc{X}$ is a complete separable metric space with probability measure $\mu$, then there is a measurable map $[0,1]\to \spc{X}$

\section{Metric on measures}

Imagine that we need to transport dirt from one pile of a given shape to make another pile of a needed shape.
Suppose that cost of transporting a unit of dirt equals to the traveled distance.%
\footnote{This is the simplest cost function one can imagine.
One may consider other cost functions; for example, if the cost proportional to the square of the distance, then the problem has more applications.}
We are free to choose a destination point for dirt from a given place.
How to minimize the total cost?

To formalize this question,
suppose that the piles of dirt are described by Borel probability measures $\mu$ and $\nu$ on a metric space~$\spc{X}$.

To describe where each piece of dirt goes, we will use the so called \index{plan}\emph{plan} for $\mu$ and $\nu$.
It is a probability measure $\pi$ on the product $\spc{X}\times\spc{X}$ such that 
for all measurable sets $A \subset \spc{X}$, we have 
\[\mu A= \pi [A \times \spc{X}],
\quad\text{and}\quad
\nu A=\pi [\spc{X}\times A].
\eqlbl{eq:marginals}\]
Equivalently it can be described as a measure that satisfies the following identity
\begin{align*}
\int_{(x,y)\in \spc{X}\times\spc{X}}f(x)\cdot g(y) \cdot \pi
&=
\int_{x\in \spc{X}}f(x)\cdot \mu
\oldcdot \int_{y\in \spc{X}}g(y)\cdot \nu,
\end{align*}
for any continuous functions $f,g\:\spc{X}\to \RR$.

Given a measure $\pi$ on $\spc{X}\times\spc{X}$, the measures $\mu$ and $\nu$ defined by \ref{eq:marginals} are called first and second \index{marginal}\emph{marginals} of $\pi$;
so the statement \textit{$\pi$ is a plan for $\mu$ and $\nu$} is equivalent to \textit{$\mu$ and $\nu$ are the first and second marginals of $\pi$ respectively}.

\begin{thm}{Claim}\label{clm:plan-exists}
There is a plan $\pi$ for any two given Borel probability measures $\mu$ and $\nu$.
\end{thm}

The plan constructed in the proof distributes equally each piece of dirt in the new pile.
As we will see this plan is usually far from optimum.

\parit{Proof.}
Consider the measure $\pi$ that is uniquely defined  defined by the identity
\[\pi(A\times B)=\mu A\cdot \mu B\]
for any Borel subsets $A,B\subset\spc{X}$.
Observe that $\pi$ is a plan for $\mu$ and~$\nu$.
\qeds

Denote by $\Pi(\mu,\nu)$ the set of all plans for $\mu$ and $\nu$;
by \ref{clm:plan-exists}, $\Pi(\mu,\nu)\z\ne\emptyset$.
It is straightforwrd to check that the following formula defines a metric on the space of probability measures on $\spc{X}$.
\[\dist{\mu}{\nu}{\Wass_1\spc{X}}
\df
\inf_{\pi\in\Pi(\mu,\nu)}
\left\{\,\int_{(x,y)\in\spc{X}\times\spc{X}}\dist{x}{y}{\spc{X}}\cdot\pi\,\right\}.\]
This metric is called \index{Wasserstein distance}\emph{Wasserstein distance of order 1} between $\mu$ and $\nu$.

In genereral, the Wasserstein distance $\dist{\mu}{\nu}{}$ might take infinite value, but all measures with compact support lie on finite distance from each other in the obtained $\infty$-metric space.
The metric component of these measures is called \index{Wasserstein space}\emph{Wasserstein space} of order 1 over $\spc{X}$; 
it is denoted by $\Wass_1\spc{X}$.
In other words, $\Wass_1\spc{X}$ is the space of all Borel probability measures $\mu$ such that 
$\int\distfun_p\cdot\mu<\infty$ for some (and therefore any) point $p\in \spc{X}$.

\begin{thm}{Exercise}\label{ex:wasserstein-infty}
Construct two Borel probability measures $\mu$ and $\nu$ on $\RR$ with Wasserstein distance $\dist{\mu}{\nu}{}=\infty$.
\end{thm}


\begin{thm}{Exercise}\label{ex:wasserstein-compact}
Show that $\Wass_1\spc{X}$ is a compact if and only if so is~$\spc{X}$.
\end{thm}

\begin{thm}{Exercise}\label{ex:wasserstein-length}
Show that the Wasserstein space $\Wass_1\spc{X}$ is a geodesic space for any metric space $\spc{X}$.
\end{thm}

\section{Optimal plan}

A plan $\pi$ for given measures $\mu$ and $\nu$ is called \index{optimal plan}\emph{optimal} if 
\[\dist{\mu}{\nu}{\Wass_1\spc{X}}
=\int_{(x,y)\in\spc{X}\times\spc{X}}\dist{x}{y}{\spc{X}}\cdot\pi.\]

\begin{thm}{Theorem} %Vilani:Theorem 1.4
Let $\mu$ and $\nu$ be probability Borel measures on a compact metric space $\spc{X}$.
Then there is an optimal plan $\pi$ for $\mu$ and~$\nu$.
\end{thm}

\parit{Proof.}
By the definition of Wasserstein distance, we can choose a sequence of plans $\pi_n$ for $\mu$ and $\nu$ such that 
\[\int_{(x,y)\in\spc{X}\times\spc{X}}\dist{x}{y}{\spc{X}}\cdot\pi_n\to \dist{\mu}{\nu}{\Wass_1\spc{X}}\]
as $n\to \infty$.

Observe that $\pi_n$ has a weak partial limit, say $\pi$.
Moreover $\pi$ is an optimal plan for $\mu$ and $\nu$.
\qeds

\begin{thm}{Theorem}
Any optimal plan $\pi$ is \index{cyclic monotonicity}\emph{cyclically monotonic}.
That is, suppose $\pi$ is an optimal plan for probability measures $\mu$ and $\nu$ on a metric space $\spc{X}$.
Then any sequence of pairs $(x_1,y_1),\dots,(x_n,y_n)\in\supp\pi\subset\spc{X}\times\spc{X}$ we have
\[\sum_i\dist{x_i}{y_i}{}
\le
\sum_i\dist{x_{i+1}}{y_i}{},\]
here the index $i$ in the sum is taken modulo $n$; in particular $x_{n+1}\z=x_1$.
\end{thm}

\parit{Proof.}
Assume that the cyclic monotonicity does not hold;
that is,
\[R=\sum_i\dist{x_i}{y_i}{}
-
\sum_i\dist{x_{i+1}}{y_i}{}>0,\]
for some $(x_0,y_0),\dots,(x_n,y_n)\in\supp\pi$.
We need to show that $\pi$ is not optimal;
in other words we need to construct another plan $\pi'$ for $\mu$ and $\nu$ such that 
\[\int_{(x,y)\in\spc{X}\times\spc{X}}\dist{x}{y}{\spc{X}}\cdot(\pi'-\pi)<0.\eqlbl{pi'<pi}\]

Assume $\spc{X}$ is finite.
In this case we can choose $\eps>0$ such that 
$\pi\{(x_i,y_i)\}>\eps$ for each $i$.
Let
\[\pi'=\pi-\eps\cdot\sum_i(\sigma_i-\sigma_i')\eqlbl{eq:pi'}\]
where $\sigma_i=\delta_{(x_i,y_i)}$ and $\sigma_i'=\delta_{(x_{i+1},y_i)}$.
It remains to observe that $\pi'$ is a plan for $\mu$ and $\nu$ that satisfies \ref{pi'<pi}.

The general case is similar, we only need to redefine $\eps$, $\sigma_i$, and~$\sigma_i'$.
Note that given $r>0$, we can choose a probability measures $\sigma_i$ with support in $\oBall((x_i,y_i),r)_{\spc{X}\times\spc{X}}$ such that $\eps\cdot \sigma_i<\pi$ for some fixed $\eps>0$ and every $i$.
Further denote by $\zeta_i$ and $\eta_i$ the first and second marginals of $\sigma_i$.
Observe that $\supp\zeta_i\subset\oBall(x_i,r)$ and $\supp\eta_i\subset\oBall(y_i,r)$ for all $i$.
Let $\sigma_i'$ be a plan for $\zeta_{i+1}$ and $\eta_i$.
Evidently 
\begin{align*}
\int_{(x,y)\in\spc{X}\times\spc{X}}\dist{x}{y}{}\cdot \sigma_i
&\lessgtr
\dist{x_i}{y_i}{}\pm 2\cdot r,
\\
\int_{(x,y)\in\spc{X}\times\spc{X}}\dist{x}{y}{}\cdot \sigma_i'
&\lessgtr
\dist{x_{i+1}}{y_i}{}\pm 2\cdot r.
\end{align*}
Taking $r<\tfrac R{10\cdot n}$, we get  \ref{pi'<pi}. 
\qeds




\section{Capitalistic approach}

Imagine that measures $\mu$ and $\nu$ describe the production and consumer of beer in the space.
A transportaition company transports beer from $\mu$ to $\nu$ and want to maximize its profit by adjusting price $f(x)$ of beer the point $x$; they buy beer at price $f(x)$ per unit, move it to an other point $y$ and sale it with (presumably higher) price $f(y)$.
However, the function $f$ is 1-Lipschitz condition;
otherwise the profit goes to second-hand dealers, or maybe it is a government regulation.
In other words we need to maximize the following expression
\[\int_{\spc{X}} f\cdot(\mu-\nu)\]
for all $1$-Lipschitz functions $f$.
The maximal profit defines a metric

\begin{thm}{Theorem}
Let $\mu$ and $\nu$ be probability Borel measures on a compact metric space $\spc{X}$.
Then
\[\dist{\mu}{\nu}{\Wass_1\spc{X}}=\sup\int_{\spc{X}} f\cdot(\mu-\nu),\]
where the least upper bound is taken for all $1$-Lipschitz functions $f\:\spc{X}\z\to\RR$.
\end{thm}

The definition of Wassershtein metric described in the previous section reminds communist's planed economy.
The right-hand side in the above equation reminds capitalistic system.
Indeed, think that measures $\mu$ and $\nu$ describe the production and consumer of beer in the space.
A transportaition company trnasports beer from $\mu$ to $\nu$ and want to maximize its profit by adjusting price $f(x)$ of beer the point $x$.
However, the function $f$ is 1-Lipschitz condition --- this is a government regulation.




\parit{Proof.}
By the definition of Wasserstein metric, we can choose a sequence $\pi_n$ of plans  

Let us choose an optimal plan $\pi$ for $\mu$ and $\nu$; it exists by ???.
We need to find a 1-Lipschitz function $f\:\spc{X}\to\RR$ such that 
\[
\int_{\spc{X}} f\cdot(\mu-\nu)=\int_{(x,y)\in\spc{X}\times\spc{X}}\dist{x}{y}{\spc{X}}\cdot \pi.
\eqlbl{eq:f(mu-nu)}
\]

Choose $x_0\in \supp\mu$.
Note that adding a constant to $f$ does not change the left hand side in \ref{eq:f(mu-nu)}.
Therefore we can assume assume that $f(x_0)=0$ and set
\[f(x)=\sup\{\,|x_0-y_0|+\dots+|x_n-y_n|-(|x_1-y_0|+\dots+|x_n-y_{n-1}|)-|x-y_n|\,\}\]
where the least upper bound is taken for all sequences $(x_0,y_0),\z\dots,(x_n,y_n)\z\in\supp\pi$.

\qeds

\section{Metric-measure space}

A metric measure space is a metric $\spc{X}$ space with a choice of Borel probability measure $\vol$ on it.
In a metric-measure we ignore sets with vanishing volume; in other words, passing from $\spc{X}$ to the support of $\vol$ does not change the metric-measure space.

Alternatively we may start with unit interval $[0,1]$ equipped with Lebesgue measure and equip it with measurable pseudometric $[0,1]\times [0,1]\to \RR$.





\section{Space of measures}


It can be equipped with the \index{Wasserstein metric}\emph{Wasserstein metric}
\[\dist{\mu}{\nu}{}\df\sup\left\{\,\int_{\spc{X}} f\cdot(\mu-\nu)\,\right\},\]
where the least upper bound is taken for all $1$-Lipschitz functions $f\:\spc{X}\to\RR$.

The Wasserstein distance $\dist{\mu}{\nu}{}$ might take infinite value, but all measures with compact support lie on finite distance from each other in the obtained $\infty$-metric space.
The metric component of these measures is called \index{Wasserstein space}\emph{Wasserstein space} of order 1 over $\spc{X}$; 
it is denoted by $\Wass_1\spc{X}$.



\section{Misc}

Suppose $\pi_n$ is a sequence of plans for $\mu$ and $\nu$.
Assume that $\pi_n$ weakly converges to a probability measure $\pi$ on $\spc{X}\times\spc{X}$.

is a weak limit of a sequence of plans $\pi_n$, then $\pi$ is a plan for $\mu$ and $\nu$ if for each $n$ $\pi_n$ is a plane for $\mu$ and $\nu$ 

Suppose that $f\:\spc{X}\to \RR$ is a 1-Lipschitz function,
so $f(x)-f(y)\le\dist{x}{y}{\spc{X}}$ for any $x,y\in \spc{X}$.
It follows that 
\begin{align*}
\int_{\spc{X}} f\cdot(\mu-\nu)&=\int_{x\in\spc{X}}f(x)\cdot\mu-\int_{y\in\spc{X}}f(y)\cdot\nu=
\\
&=\int_{(x,y)\in\spc{X}\times\spc{X}}[f(x)-f(y)]\cdot \pi\le
\\
&\le\int_{(x,y)\in\spc{X}\times\spc{X}}\dist{x}{y}{\spc{X}}\cdot \pi,
\end{align*}
where $\pi$ is a plan for $\mu$ and $\nu$.
By the definition of Wasserstein metric, we get  
\[\dist{\mu}{\nu}{\Wass_1\spc{X}}\le \int_{(x,y)\in\spc{X}\times\spc{X}}\dist{x}{y}{\spc{X}}\cdot\pi\eqlbl{wass=<int.plan}\]
for any plan $\pi$.

Next we want to show that equality holds in \ref{wass=<int.plan} for some plan $\pi$; such plans will be called \index{optimal plan}\emph{optimal}.


\parit{Proof.}
Choose a point $x_0\in \supp\mu$.
Given  $p\in \spc{X}$,
let
\[f(p)=\inf\left\{\sum_{i=0}^n\dist{x_i}{y_i}{}-\sum_{i=0}^n\dist{x_{i+1}}{y_i}{}-\dist{y_n}{p}{}\right\},
\eqlbl{eq:f(p)}\]
where the least upper bound is taken for all sequences of pairs 
\[(x_0,y_0),\z\dots,(x_n,y_n)\in \supp\pi.\eqlbl{eq:sequence}\]

Fix a sequence as in \ref{eq:sequence} and  denote by $F_\sigma(p)$ the expression under infimum in \ref{eq:f(p)}.

Let us show that 
\[F_\sigma(x_0)\ge 0.\]
Indeed, suppose $F_\sigma(x_0)<-\eps<0$.
Since $(x_i,y_i)\in \supp\pi$, we have $x_i\in\supp\mu$ and $y_i\in\supp\nu$ for any $i$.
Therefore we can choose sets $X_i\subset \oBall(x_i,\tfrac{\eps}{10\cdot n})$ and $Y_i\subset \oBall(y_i,\tfrac{\eps}{10\cdot n})$ such that 
$\mu(X_0)=\nu(Y_0)=\dots=\mu(X_n)=\nu(Y_n)$



Let us denote by $F(p)$ the expression under infimum in \ref{eq:f(p)}.
By triangle inequality, 
\[F(q)\le F(p)+\dist{p}{q}{}.\]
Passing to the least upper bound in this inequality, we get
\[f(q)\le f(p)+\dist{p}{q}{}\]
for any $p,q\in\spc{X}$.
Hence $f$ is a 1-Lipschitz function.

Further, let us show that
\[(x,y)\in\supp\pi
\quad\Longrightarrow\quad
f(y)-f(x)=\dist{x}{y}{}\]





Suppose that cyclic monotonicity fails;
that is, there is a sequence of pairs $(x_1,y_1),\dots,(x_n,y_n)\in\spc{X}\times\spc{X}$ such that
\[\dist{x_1}{y_1}{}+\dots+\dist{x_n}{y_n}{}
>
\dist{x_1}{y_2}{}+\dots+\dist{x_{n-1}}{y_n}{}+\dist{x_{n}}{y_1}{}.\]
In this case, it would be more optimal to transport measure from a neighborhood of $x_i$ to a neighborhood of $y_{i+1}$ (
here and further we assume that the indexes are taken modulo $n$, so $n+1=1$).
The latter contradicts optimality of $\pi$.

The following argument makes it precise.
Choose small $\eps>0$.
For each $n$,
choose disjoint sets $X_i$ and $Y_i$ in $\eps$-neighborhood of $x_i$ and $y_i$
such that for some $\delta>0$ we have 
\[\pi [X_i\times Y_i]=\delta\]
for each $i$.

Let us modify the plan $\pi$ in the union $X_1\times Y_1 \cup\dots\cup X_n\times Y_n$ and such that 
$\pi'(X_i\times Y_{i+1})=\delta$ for each $i$;


Observe that
\[\int_{(x,y)\in\spc{X}\times\spc{X}}\dist{x}{y}{\spc{X}}\cdot(\pi'-\pi)>\]
\qeds


\appendix
\chapter{Semisolutions}
\parbf{\ref{ex:almost-min}.}
Assume the statement is wrong. 
Then for any point $x\in \spc{X}$, there is a point $x'\in \spc{X}$ such that 
\[\dist{x}{x'}{}< \rho(x)
\quad\text{and}\quad
\rho(x')\le\frac{\rho(x)}{1+\eps}.\]
Consider a sequence of points $(x_n)$ such that $x_{n+1}\z=x_n'$.
Clearly 
\[\dist{x_{n+1}}{x_n}{}
\le
\frac{\rho(x_0)}{\eps\cdot(1+\eps)^n}
\quad\hbox{and}\quad
\rho(x_n)\le \frac{\rho(x_0)}{(1+\eps)^n}.\] 
Therefore $(x_n)$ is a Cauchy sequence.
Since $\spc{X}$ is complete, the sequence $(x_n)$ converges;
denote its limit by $x_\infty$.
Since $\rho$ is a continuous function we get
\begin{align*}\rho(x_\infty)&=\lim_{n\to\infty}\rho(x_n)=
\\&=0.
\end{align*}

The latter contradicts that $\rho>0$.


\parbf{\ref{ex:non-contracting-map}.}
Given any pair of point $x_0,y_0\in \spc{K}$, 
consider two sequences $x_0,x_1,\dots$ and $y_0,y_1,\dots$
such that $x_{n+1}=f(x_n)$ and $y_{n+1}\z=f(y_n)$ for each $n$.

Since $\spc{K}$ is compact, 
we can choose an increasing sequence of integers $n_k$
such that both sequences $(x_{n_i})_{i=1}^\infty$ and $(y_{n_i})_{i=1}^\infty$
converge.
In particular, both are Cauchy;
that is,
\[
|x_{n_i}-x_{n_j}|_{\spc{K}}, |y_{n_i}-y_{n_j}|_{\spc{K}}\to 0
\quad
\text{as}
\quad\min\{i,j\}\to\infty.
\]


Since $f$ is non-contracting, we get
\[
|x_0-x_{|n_i-n_j|}|
\le 
|x_{n_i}-x_{n_j}|.
\]

It follows that  
there is a sequence $m_i\to\infty$ such that
\[
x_{m_i}\to x_0\quad\text{and}\quad y_{m_i}\to y_0\quad\text{as}\quad i\to\infty.
\leqno({*})\]

Set \[\ell_n=|x_n-y_n|_{\spc{K}}.\]
Since $f$ is non-contracting, the sequence $(\ell_n)$ is nondecreasing.

By $({*})$,  $\ell_{m_i}\to\ell_0$ as $m_i\to\infty$.
It follows that $(\ell_n)$ is a constant sequence.

In particular 
\[|x_0-y_0|_{\spc{K}}=\ell_0=\ell_1=|f(x_0)-f(y_0)|_{\spc{K}}\]
for any pair of points $(x_0,y_0)$ in $\spc{K}$.
That is, $f$ is distance-preserving, in particular injective.

From $({*})$, we also get that $f(\spc{K})$ is everywhere dense.
Since $\spc{K}$ is compact $f\:\spc{K}\to \spc{K}$ is surjective. Hence the result follows.

\parit{Remarks.}
This is a basic lemma in the introduction to Gromov--Hausdorff distance \cite[see 7.3.30 in][]{burago-burago-ivanov}.
This proof is not quite standard,
I learned this proof from Travis Morrison, 
a student in my MASS class at Penn State, Fall 2011.

Note that as an easy corollary one can see that any surjective non-expanding map from a compact metric space to itself is an isometry. 

\parbf{\ref{ex:pogorelov}.}
The conditions \ref{SHORT.metric>=0}--\ref{SHORT.metric:sym} in Definition \ref{def:metric} are evident.

The triangle inequality \ref{SHORT.metric:triangle} follows since
\[[B(x,\tfrac \pi2)\backslash B(y,\tfrac\pi2)]
\cup 
[B(y,\tfrac\pi2)\backslash B(z,\tfrac\pi2)]
\supseteq
B(x,\tfrac \pi2) \backslash B(z,\tfrac\pi2).
\leqno(*)\]

\begin{wrapfigure}[8]{o}{31 mm}
\vskip-2mm
\centering
\includegraphics{mppics/pic-29}
\end{wrapfigure}

Observe that
$B(x,\tfrac \pi2)\backslash B(y,\tfrac\pi2)$
does not overlap
$B(y,\tfrac\pi2)\backslash B(z,\tfrac\pi2)$ and  we get equality in $(*)$ if and only if $y$ lies on the great circle arc from $x$ to $z$.
Therefore the second statement follows.


\parit{Remarks.}
This construction was given by 
Aleksei Pogorelov \cite{pogorelov}.
It is closely related to the construction given 
by David Hilbert in \cite{hilbert}
which was the motivating example for his 4-th problem.


\parbf{\ref{ex:no-geod}.}
We assume that the space is nontrivial, otherwise a one-point space is an example.

Consider the unit ball $(B,\rho_0)$
in the space $c_0$ of all sequences converging to zero equipped with the sup-norm.

Consider another metric $\rho_1$ which is different from $\rho_0$ by the conformal factor
\[\phi(\bm{x})=2+\tfrac{1}2\cdot x_1+\tfrac{1}4\cdot x_2+\tfrac{1}8\cdot x_3+\dots,\]
where $\bm{x}=(x_1,x_2\,\dots)\in B$.
That is, if $\bm{x}(t)$, $t\in[0,\ell]$, is a curve parametrized by $\rho_0$-length 
then its $\rho_1$-length is defined by
\[\length_{\rho_1}\bm{x}\df\int\limits_0^\ell\phi\circ\bm{x}(t)\cdot dt.\]
Note that the metric $\rho_1$ is bi-Lipschitz to~$\rho_0$.

Assume $\bm{x}(t)$ and $\bm{x}'(t)$ are two curves parametrized by $\rho_0$-length that differ only in the $m$-th coordinate, denoted by $x_m(t)$ and $x_m'(t)$ respectively.
Note that if $x'_m(t)\le x_m(t)$ for any $t$ and 
the function $x'_m(t)$ is locally $1$-Lipschitz at all $t$ such that $x'_m(t)< x_m(t)$, then 
\[\length_{\rho_1}\bm{x}'\le \length_{\rho_1}\bm{x}.\]
Moreover this inequality is strict if $x'_m(t)< x_m(t)$ for some~$t$.

Fix a curve $\bm{x}(t)$, $t\in[0,\ell]$, parametrized by  $\rho_0$-length.
We can choose $m$ large, so that $x_m(t)$ is sufficiently close to $0$ for any~$t$.
In particular, for some values $t$, we have $y_m(t)<x_m(t)$, where
\[y_m(t)=(1-\tfrac t\ell)\cdot x_m(0)
+\tfrac t\ell\cdot x_m(\ell)
-\tfrac 1{100}\cdot \min\{t,\ell-t\}.\]
Consider the curve $\bm{x}'(t)$ as above with
\[x'_m(t)=\min\{x_m(t),y_m(t)\}.\]
Note that $\bm{x}'(t)$ and $\bm{x}(t)$ have the same end points, and by the above
\[\length_{\rho_1}\bm{x}'<\length_{\rho_1}\bm{x}.\]
That is, for any curve $\bm{x}(t)$ in $(B,\rho_1)$, we can find a shorter curve $\bm{x}'(t)$ with the same end points.
In particular, $(B,\rho_1)$ has no geodesics.

\parit{Remarks.}
This solution was suggested by Fedor Nazarov~\cite{nazarov}.

\parbf{\ref{ex:compact+connceted}.}
Choose a sequence $\varepsilon_n\to 0$ and a $\varepsilon_n$-net $N_n$ of $K$ for each $n$.
Assume $N_0$ is a one-point set, so $\eps_0>\diam K$.
Connect each point $x\in N_{k+1}$ to a point $y\in N_{k}$ by a curve of length at most $\eps_k$.

Consider the union $K'$ of all these curves with $K$; observe that $K'$ is compact and path connected.

\parit{Source:} This problem was suggested by Eugene Bilokopytov \cite{bilokopytov}.

\parbf{\ref{ex:compact=>complete}.}
Choose a Cauchy sequence $(x_n)$ in $(\spc{X},\|*-*\|)$; it sufficient to show that a subsequence of $(x_n)$ converges.

Note that the sequence $(x_n)$ is Cauchy in $(\spc{X},|*-*|)$;
denote its limit by $x_\infty$.

After passing to a subsequence, we can assume that $\|x_n-x_{n+1}\|\z<\tfrac1{2^n}$.
It follows that there is a 1-Lipschitz path $\gamma$ in $(\spc{X},\|*-*\|)$ such that $x_n=\gamma(\tfrac1{2^n})$ for each $n$ and $x_\infty=\gamma(0)$.

It follows that
\begin{align*}
\|x_\infty-x_n\|&\le \length\gamma|_{[0,\frac1{2^n}]}\le
\\
&\le \tfrac1{2^n}.
\end{align*}
In particular $x_n$ converges.

\parit{Source:} \cite[Lemma 2.3]{petrunin-stadler}.


\begin{wrapfigure}{r}{20 mm}
\vskip-0mm
\centering
\includegraphics{mppics/pic-1}
\end{wrapfigure}

\parbf{\ref{exercise from BH}.}
Consider the following subset of $\RR^2$ equipped with the induced length metric
\[
\spc{X}
=
\bigl((0,1]\times\{0,1\}\bigr)
\cup
\bigl(\{1,\tfrac12,\tfrac13,\dots\}\times[0,1]\bigr)
\]
Note that $\spc{X}$ is locally compact and geodesic.

Its completion $\bar{\spc{X}}$ is isometric to the closure of $\spc{X}$ equipped with the induced length metric.
Note that $\bar{\spc{X}}$ is obtained from $\spc{X}$ by adding two points $p=(0,0)$ and $q=(0,1)$.

Observe that the point $p$ admits no compact neighborhood in $\bar{\spc{X}}$ 
and there is no geodesic connecting $p$ to $q$ in~$\bar{\spc{X}}$. 

\parit{Source:} \cite[I.3.6(4)]{bridson-haefliger}.

%%%%%%%%%%%%%%%%%%%%%%%%%%%%%%



\parbf{\ref{ex:gross}.}
If such a number does not exist, then the ranges of average distance functions have empty intersection.
Since $\spc{X}$ is a compact length-metric space, the range of any continuous function on $\spc{X}$ is a closed interval.
By 1-dimensional Helly's theorem, there is a pair of such range intervals that do not intersect.
That is, for two point-arrays $(x_1,\dots,x_n)$ and $(y_1,\dots,y_m)$
and their average distance functions 
\[f(z)=\tfrac1n\cdot\sum_i|x_i-z|_{\spc{X}}\quad\text{and}\quad h(z)=\tfrac1m\cdot\sum_j|y_j-z|_{\spc{X}},\] we have 
$$\min\set{f(z)}{z\in \spc{X}}>\max\set{h(z)}{z\in \spc{X}}.\leqno({*})$$

Note that 
$$\tfrac1m\cdot\sum_j f(y_j)=\tfrac1{m\cdot n}\cdot\sum_{i,j}|x_i-y_j|_{\spc{X}}=\tfrac1n\cdot\sum_i h(x_i);$$
that is, the average value of $f(y_j)$ coincides with the average value of $h(x_i)$, 
which contradicts $({*})$.

\parit{Remarks.}
In fact the value $\ell$ is uniquely defined;
it is called the \index{rendezvous value}\emph{rendezvous value} of ${\spc{X}}$.
This is a result of Oliver Gross \cite{gross}.

\parbf{\ref{ex:wasserstein}.}
Choose a finite $\eps$-net $F\subset\spc{X}$.
Show that the space $P_F$ of probability measures with support in $F$ is a compact net in $\Wass_1\spc{X}$.
Observe that $\Wass_1\spc{X}$ is complete; 
by \ref{ex:compact-net}, it follows that $\Wass_1\spc{X}$ is compact.

Show that 

Choose an integer $n$.
Consider the set of probability measures $P_n$ of the form 
\[\tfrac1n\cdot\sum_{i=1}^n\delta_{x_i},\]
where $\delta_{x_i}$ denotes the delta-measure supported at $x_i\in\spc{X}$. 

Show that $P_n$ is a compact subset of $\Wass_1\spc{X}$.
Moreover, for any $\eps>0$ there is $n$ such that $P_n$ is 




%%%%%%%%%%%%%%%%%%%%%%%%%

\parbf{\ref{ex:compact-length}.} By Fréchet lemma (\ref{lem:frechet}) we can identify $\spc{K}$ with a compact subset of $\ell^\infty$.

Denote by $\spc{L}=\Conv\spc{K}$ --- it is defined as the minimal convex closed set in $\ell^\infty$ that contains $\spc{K}$.
(In other words, $\spc{L}$ is the intersection of all convex closed sets that contain $\spc{K}$.)

Observe that $\spc{L}$ is a length space.
It remains to show that $\spc{L}$ is compact.

By construction $\spc{L}$ is a closed subset of $\ell^\infty$, in particular it is a complete space.
By \ref{totally-bounded}, it remains to show that $\spc{L}$ is totally bounded.

Recall that Minkowski sum $A + B$ of two sets $A$ and $B$ in a vector space is defined by
\[A + B = \set{a+b}{a\in A,\ b\in B}.\]
Observe that the Minkowski sum of two convex sets is convex.

Denote by $\bar B_\eps$ the closed $\eps$-ball in $\ell^\infty$ centered at the origin.
Choose a finite $\eps$-net $N$ in $\spc{K}$ for some $\eps>0$.
Note that $P=\Conv N$ is a convex polyhedron, in particular $\Conv N$ is compact.

Observe that $N+\bar B_\eps$ is closed $\eps$-neighborhood of $N$.
It follows that $N+\bar B_\eps\supset K$ and therefore $P+\bar B_\eps\supset \spc{L}$.
In particular $P$ is a $2\cdot\eps$-net in $\spc{L}$;
since $P$ is compact and $\eps>0$ is arbitrary, $\spc{L}$ is totally bounded (see \ref{ex:compact-net}).

\parit{Remark.}
Another solution follows since the injective envelope of a compact space is compact; see \ref{ex:inj=complete-geodesic-contractible:geodesic}, \ref{ex:Inj(compact)}, and \ref{prop:InjX-is-injective}.

\parbf{\ref{ex:geodesics-urysohn}.}
Choose a separable space $\spc{X}$ that has an infinite number of geodesics between a pair of points, say a square will $\ell^\infty$-metric in $\RR^2$.
Apply to $\spc{X}$ universality of Urysohn space (\ref{prop:sep-in-urys}).

\parbf{\ref{ex:compact-extension}.} 
First let us prove the following claim:

\begin{itemize}
\item 
Suppose $f\: K\to\RR$ is an extension function defined on a compact subset $K$ of the Urysohn space $\spc{U}$.
Then there is a point $p\in \spc{U}$ such that 
$\dist{p}{x}{}=f(x)$ for any $x\in K$.
\end{itemize}

Without loss of generality we may assume that $f(x)>0$ for any $x\in K$.
Since $K$ is compact, we may fix $\eps>0$ such that $f(x)>\eps$.

Consider the sequence $\eps_n=\tfrac\eps{100\cdot 2^n}$.
Choose a sequence of $\eps_n$-nets $N_n\subset K$.
Applying universality of $\spc{U}$ recursively, we may choose a point $p_n$ such that $\dist{p_n}{x}{}=f(x)$ for any $x\in N_n$ and $\dist{p_n}{p_{n-1}}{}\z=10\cdot\eps_{n-1}$.
Observe that the sequence $(p_n)$ is Cauchy and its limit $p$ meets 
$\dist{p}{x}{}=f(x)$ for any $x\in K$.

Now, choose a sequence of points $(x_n)$ in $\spc{S}$.
Applying the claim, we may extend the map from $K$ to $K\cup\{x_1\}$, and further to $K\cup\{x_1,x_2\}$, and so on.
As a result we extend the distance-preserving map $f$ to the whole sequence $(x_n)$.
It remains to extend it continuously to the whole space~$\spc{S}$.

\parbf{\ref{ex:sc-urysohn}.}
It is sufficient to show that any compact subspace $\spc{K}$ of Urysohn space can be contracted to a point.

Note that any compact space $\spc{K}$ can be extended to a contractible compact space $\spc{K}'$; for example we may embed $\spc{K}$ into $\ell^\infty$ and pass to its convex hull, as it was done in \ref{ex:compact-length}.

By \ref{thm:compact-homogeneous}, there is an isometric embedding of $\spc{K}'$ that agrees with inclusion of $\spc{K}$.
Since $\spc{K}$ is contractible in $\spc{K}'$, it is contractible in $\spc{U}$.

\parit{A better way.}
One can contract the whole Urysohn space using the following construction.

Note that points in the space $\spc{X}_\infty$ constructed in the proof of \ref{prop:univeral-separable} can be multiplied number $t\in [0,1]$ --- simply multiply each function by $t$.
That defines a map 
\[\lambda_t\:\spc{X}_\infty\to \spc{X}_\infty\]
that scales all distances by factor $t$.
The map $\lambda_t$ can be extended to the completion of $\spc{X}_\infty$, which is isometric to $\spc{U}_d$ (or $\spc{U}$).

Observe that 
the map $\lambda_1$ is the identity  and $\lambda_0$ maps whole space to a single point, say $x_0$ --- this is the only point of $\spc{X}_0$.
Further note that the map $(t,p)\mapsto \lambda_t(p)$ is continuous ---  in particular $\spc{U}_d$ and $\spc{U}$ are contractible.

As a bonus, observe that for any point $p\in \spc{U}_d$ the curve $t\mapsto \lambda_t(p)$ is a geodesic path from $p$ to $x_0$.

\parit{Source:} \cite[(d) on page 82]{gromov-2007}.

\parbf{\ref{ex:sphere-in-urysohn}}; \ref{SHORT.ex:sphere-in-urysohn:sphere} and \ref{SHORT.ex:sphere-in-urysohn:midpoint}.
Observe that $L$ and $M$ satisfy the definition of $d$-Urysohn space and apply the uniqueness (\ref{thm:urysohn-unique}).

\parit{\ref{SHORT.ex:sphere-in-urysohn:homogeneous}.} 
Use \ref{SHORT.ex:sphere-in-urysohn:sphere}, maybe twice.

\parbf{\ref{ex:homogeneous}}; \ref{SHORT.ex:homogeneous:euclidean}.
The euclidean plane is homogeneous in every sense.

\parit{\ref{SHORT.ex:homogeneous:hilbert}.} The ilbert space $\ell^2$ is finite set homogeneous, but not compact set homogeneous, nor countable homogeneous.

\parit{\ref{SHORT.ex:homogeneous:ell-infty}.} The space $\ell^\infty$ is 1-point homogeneous, but not 2-point homogeneous.
Try to show that there is no isometry of $\ell^\infty$ such that
\begin{align*}
(0,0,0,\dots)&\mapsto (0,0,0,\dots),
\\
(1,1,1,\dots)&\mapsto (1,0,0,\dots).
\end{align*}

\parit{\ref{SHORT.ex:homogeneous:ell-1}.}
The space $\ell^1$ is 1-point homogeneous, but not 2-point homogeneous.
Try to show that there is no isometry of $\ell^\infty$ such that
\begin{align*}
(0,0,0,\dots)&\mapsto (0,0,0,\dots),
\\
(2,0,0\dots)&\mapsto (1,1,0,\dots).
\end{align*}

\parbf{\ref{ex:+-c}.}
Note that if $c<0$, then $r>s$.
The latter is impossible since $r$ is extremal and $s$ is admissible.

Observe that the function $\bar r=\min\{\,r,s+c\}$ is admissible.
Indeed if $\bar r(x)=r(x)$ and $\bar r(y)=r(y)$ then 
\[\bar r(x)+\bar r(y)=r(x)+ r(y)\ge \dist{x}{y}{}.\]
Further if $\bar r(x)=s(x)+c$ then 
\begin{align*}
\bar r(x)+\bar r(y)&\ge [s(x)+c]+ [s(y)-c]= 
\\
&=s(x)+s(y) \ge 
\\
&\ge\dist{x}{y}{}.
\end{align*}

Since $r$ is extremal, we have $r=\bar r$;
that is, $r\le s+c$.

\parbf{\ref{ex:inj=complete-geodesic-contractible}.}
Choose an injective space $\spc{Y}$.

\textit{\ref{SHORT.ex:inj=complete-geodesic-contractible:complete}.}
Fix a Cauchy sequence $(x_n)$ in $\spc{Y}$;
we need to show that it has a limit $x_\infty\in \spc{Y}$.
Consider metric on $\spc{X}=\NN\cup\{\infty\}$ defined by 
\begin{align*}
\dist{m}{n}{\spc{X}}&=\dist{x_m}{x_n}{\spc{Y}},
\\
\dist{m}{\infty}{\spc{X}}&=\lim_{n\to\infty}\dist{x_m}{x_n}{\spc{Y}}.
\end{align*}
Since the sequence is Cauchy, so is the sequence $\ell_n=\dist{p}{x_n}{\spc{Y}}$.
Therefore the last limit is defined.

By construction the map $n\mapsto x_n$ is distance-preserving on $\NN\subset \spc{X}$.
Since $\spc{Y}$ is injective, this map can be extended to $\infty$ as a short map; set $\infty\mapsto x_\infty$.
Since $\dist{x_n}{x_\infty}{\spc{Y}}\le \dist{n}{\infty}{\spc{X}}$ 
and $\dist{n}{\infty}{\spc{X}}\to 0$, we get that
$x_n\to x_\infty$ as $n\to\infty$.

\textit{\ref{SHORT.ex:inj=complete-geodesic-contractible:geodesic}.}
Applying the definition of injective space, we get a midpoint for any pair of points in $\spc{Y}$.
By \ref{SHORT.ex:inj=complete-geodesic-contractible:complete},
$\spc{Y}$ is a complete space.
It remains to apply \ref{lem:mid>geod:geod}.

\textit{\ref{SHORT.ex:inj=complete-geodesic-contractible:contractible}.}
Let $k\:\spc{Y}\hookrightarrow \ell^\infty(\spc{Y})$ be the Kuratowski embedding (\ref{lem:kuratowski}).
Observe that $\ell^\infty(\spc{Y})$ is contractible;
in particular, there is a homotopy $k_t\:\spc{Y}\hookrightarrow \ell^\infty(\spc{Y})$ such that $k_0=k$ and $k_1$ is a constant map.
(In fact one can take $k_t=(1-t)\cdot k$.)

Since $k$ is distance-preserving and $\spc{Y}$ is injective,
there is a short map $f\:\ell^\infty(\spc{Y})\to \spc{Y}$ such that the composition $f\circ k$ is the identity map on $\spc{Y}$.
The composition $f\circ k_t\:\spc{Y}\hookrightarrow \spc{Y}$ is a needed homotopy. 

\parbf{\ref{ex:injective-spaces}.}
Suppose that a short map $f\:A\to\spc{Y}$ is defined on a subset $A$ of a metric space $\spc{X}$.
We need to construct a short extension $F$ of $f$.

\parit{\ref{SHORT.ex:injective-spaces:R}.}
Suppose $\spc{Y}=\RR$.
Without loss of generality, we may assume that $A\ne\emptyset$, otherwise map whole $\spc{X}$ to a single point.
Set 
\[F(x)=\inf\set{f(a)-\dist{a}{x}{}}{a\in A}.\] 
Observe that $F$ is short and $F(a)=f(a)$ for any $a\in A$.

\parit{\ref{SHORT.ex:injective-spaces:tree}.}
Suppose  $\spc{Y}$ is a complete metric tree.
Fix points $p\in \spc{X}$ and $q\in\spc{Y}$.
Given a point $a\in A$,
let $x_a\in\cBall[f(a),\dist{a}{p}{}]$ be the point closest to $f(x)$.
Note that $x_a\in[q\,f(a)]$ and either $x_a=q$ or $x_a$ lies on distance $\dist{a}{p}{}$ from $f(a)$.

Note that the geodesics $[q\,x_a]$ are nested;
that is, for any $a,b\in A$ we have either $[q\,x_a]\subset [q\,x_b]$ or $[q\,x_b]\subset [q\,x_a]$.
Moreover, in the first case we have $\dist{x_b}{f(a)}{}\le \dist{p}{a}{}$ and in the second $\dist{x_a}{f(b)}{}\le \dist{p}{b}{}$.

It follows that the closure of the union of all geodesics $[q\,x_a]$ for $a\in\spc{A}$ is a geodesic.
Denote by $x$ its endpoint; it exists since $\spc{Y}$ is complete.
It remains to observe that $\dist{x}{f(a)}{}\le \dist{p}{a}{}$ for any $a\in\spc{A}$;
that is, one can take $f(p)=x$.

\parbf{\ref{ex:ultrametric}.}
Choose three points $x,y,z\in\spc{X}$ and set $\spc{A}=\{x,z\}$.
Let $f\:\spc{A}\to \spc{Y}$ be an isometry.
Then $F(y)=f(x)$ or $F(y)=f(z)$.
If  $f(y)=f(x)$, then
\begin{align*}
\dist{y}{z}{\spc{X}}&\ge  \dist{F(y)}{f(z)}{\spc{Y}}=
\\
 &=\dist{x}{z}{\spc{X}}.
\end{align*}
Analogously if $f(y)=f(z)$, then $\dist{x}{y}{\spc{X}}\ge\dist{x}{z}{\spc{X}}$.

It remains to observe that the strong triangle inequality holds in both cases.

\parit{\ref{SHORT.ex:injective-spaces:ell-infty}.}
In this case $\spc{Y}=(\RR^2,\ell^\infty)$.
Note that the map $\spc{X}\to (\RR^2,\ell^\infty)$ is short if and only if both of its coordinate projections are short.
It remains to apply \ref{SHORT.ex:injective-spaces:R}.

\parbf{\ref{ex:tripod+square}}; \ref{SHORT.ex:tripod+square:tripod}.
Let $f$ be an extremal function.
Observe that at least two of the numbers $f(a)+f(b)$, $f(b)+f(c)$, and $f(c)+f(a)$ are $1$.
It follows that for some $x\in[0,\tfrac12]$, we have 
\begin{align*}
f(a)&=1\pm x,&
f(b)&=1\pm x,&
f(c)&=1\pm x,
\end{align*}
where we have one ``minus'' and two ``pluses'' in these three formulas.

Suppose that
\begin{align*}
g(a)&=1\pm y,& g(b)&=1\pm y,& g(c)&=1\pm y
\end{align*}
is another extremal function.
Then $|f-g|\z=|x-y|$ if $g$ has ``minus'' at the same place as $f$ and $|f-g|=|x+y|$ otherwise.

\begin{wrapfigure}{o}{30 mm}
\vskip-0mm
\centering
\includegraphics{mppics/pic-3}
\bigskip
\includegraphics{mppics/pic-4}
\end{wrapfigure}

It follows that $\Inj\spc{X}$ is isometric to a tripod;
that is, $\Inj\spc{X}$ is formed by three segments of length $\tfrac12$ glued at one end.

\parit{\ref{SHORT.ex:tripod+square:square}.}
Assume $f$ is an extremal function.
Observe that 
$f(x)+f(y)=f(p)+f(q)=2$;
in particular, two values $a=f(x)-1$ and $b=f(p)-1$ completely describe the function $f$.
Since $f$ is extremal, we also have that 
\[(1\pm a)+(1\pm b)\ge 1\]
for all 4 choices of signs;
that is, $|a|+|b|\le 1$.

It follows that $\Inj\spc{X}$ is isometric to the rhombus $|a|+|b|\le 1$ in the $(a,b)$-plane with the metric induced by the $\ell^\infty$-norm.





\parbf{\ref{ex:4-on-a-line}.}
Recall that 
\[\dist{f}{g}{\Inj\spc{X}}=\sup\set{|f(x)-g(x)|}{x\in\spc{X}}\]
and 
\[\dist{f}{p}{\Inj\spc{X}}=f(p)\]
for any $f,g\in \Inj\spc{X}$ and $p\in \spc{X}$.

Since $\spc{X}$ is compact we can find a point $p\in\spc{X}$ such that 
\[\dist{f}{g}{\Inj\spc{X}}=|f(p)-g(p)|=\left|\dist{f}{p}{\Inj\spc{X}}-\dist{g}{p}{\Inj\spc{X}}\right|.\]
Without loss of generality we may assume that 
\[\dist{f}{p}{\Inj\spc{X}}
=
\dist{g}{p}{\Inj\spc{X}}
+
\dist{f}{g}{\Inj\spc{X}}.\]
Applying \ref{lem:opposite}, we can find a point $q\in\spc{X}$ such that 
\[\dist{q}{p}{\Inj\spc{X}}
=
\dist{f}{p}{\Inj\spc{X}}
+
\dist{f}{q}{\Inj\spc{X}},\]
whence the result.

Since $\Inj\spc{X}$ is injective (\ref{prop:InjX-is-injective}), by \ref{ex:inj=complete-geodesic-contractible:geodesic} it has to be geodesic. It remains to note that the concatenation of geodesics $[pq]$, $[gf]$, and $[fq]$ forms a required geodesic $[pq]$.



\parbf{\ref{ex:Hausdorff-bry}}; \ref{SHORT.ex:Hausdorff-bry:conv}.
Denote by $X_r$ the $r$ neighborhood of a set $X\z\subset \RR^2$.
Observe  that 
\[(\Conv X)_r=\Conv(X_r),\]
and try to use it.

\parit{\ref{SHORT.ex:Hausdorff-bry:bry}.}
The answer is ``no'' in both parts.

For the first part let $X$ be a unit disc and $Y$ a finite $\eps$-net in $X$.
Evidently $|X-Y|_{\Haus\RR^2}<\eps$, 
but
$|\partial X-\partial Y|_{\Haus\RR^2}\approx 1$.

For the second part take $X$ to be a unit disc and $Y=\partial X$ to be its boundary circle.
Note that $\partial X=\partial Y$ in particular $\dist{\partial X}{\partial Y}{\Haus\RR^2}=0$ while $\dist{ X}{ Y}{\Haus\RR^2}=1$.

\parit{Remark.}
A more interesting example for the second part can be build on {}\emph{lakes of Wada} --- and example of three open bounded topological disks in the plane that have identical boundary.

\parbf{\ref{ex:Huas-perimeter-area}.}
Let $A$ be a compact convex set in the plane.
Denote by $A^r$ the closed $r$-neighborhood of $A$.
Recall that by Steiner's formula we have
\[\area A^r=\area A+r\cdot\perim A+\pi\cdot r^2.\]
Taking derivative and applying coarea formula, we get 
\[\perim A^r=\perim A+2\cdot\pi\cdot r.\]

Observe that if $A$ lies in a compact set $B$ bounded by a closed curve, then 
\[\perim A\le \perim B.\]
Indeed the closest-point projection $\RR^2\to A$ is short and it maps $\partial B$ onto $\partial A$.

It remains to observe that if $A_n\to A_\infty$, then for any $r>0$ we have that
\[A_\infty^r\supset A_n
\quad\text{and}\quad
A_\infty\subset A_n^r\]
for all large $n$.

%%%%%%%%%%%%%%%%%%%%%%%%%%%%%%

\parbf{\ref{ex:GH-inj}.}
Let $\spc{U}$ be as in \ref{prop:GH=X+Y}.
Suppose that $|\spc{X}-\spc{Y}|_{\spc{U}}<\eps$;
we need to show that 
\[|\hat{\spc{X}}-\hat{\spc{Y}}|_{\GH}<2\cdot \eps.\]

Denote by $\hat{\spc{U}}$ the injective envelop of $\spc{U}$.
Recall that $\spc{U}$, $\spc{X}$, and $\spc{Y}$ can be considered as subspaces of $\hat{\spc{U}}$, $\hat{\spc{X}}$, and $\hat{\spc{Y}}$ respectively.

According to \ref{ex:d-p-inclusion}, the inclusions $\spc{X}\hookrightarrow\spc{U}$ and $\spc{Y}\hookrightarrow\spc{U}$ can be extended to distance preserving inclusions $\hat{\spc{X}}\hookrightarrow\hat{\spc{U}}$ and $\hat{\spc{Y}}\hookrightarrow\hat{\spc{U}}$.
Therefore we can and will consider  $\hat{\spc{X}}$ and $\hat{\spc{Y}}$ as subspaces of $\hat{\spc{U}}$.


Given $f\in \hat{\spc{U}}$,
let us find $g\in \hat{\spc{X}}$ such that 
\[|f(u)-g(u)|<2\cdot\eps\eqlbl{|g-f|}\]
for any $u\in\spc{U}$.
Note that the restriction $f|_{\spc{X}}$ is admissible on ${\spc{X}}$.
By \ref{obs:extremal:below}, there is $g\in \hat{\spc{X}}$ such that 
\[g(x)\le f(x)\eqlbl{g(x)=<f(x)}\]
for any $x\in\spc{X}$.


Recall that any extremal function is $1$-Lipschitz;
in particular $f$ and $g$ are $1$-Lipschitz on $\spc{U}$.
Therefore \ref{g(x)=<f(x)} and $|\spc{X}-\spc{Y}|_{\spc{U}}<\eps$ imply that
\[g(u)< f(u)+2\cdot \eps\]
for any $u\in\spc{U}$.
By \ref{ex:+-c}, we also have 
\[g(u)> f(u)-2\cdot \eps\]
for any $u\in\spc{U}$.
Whence \ref{|g-f|} follows.

It follows that $\hat{\spc{Y}}$ lies in a $2\cdot\eps$-neighborhood of $\hat{\spc{X}}$ in $\hat{\spc{U}}$.
The same way we show that $\hat{\spc{X}}$ lies in a $2\cdot\eps$-neighborhood of $\hat{\spc{Y}}$ in $\hat{\spc{U}}$.
The later means that
$|\hat{\spc{X}}-\hat{\spc{Y}}|_{\Haus\spc{U}}<2\cdot\eps$,
and therefore
$|\hat{\spc{X}}-\hat{\spc{Y}}|_{\GH}<2\cdot\eps$.

\parit{Comment.} 
This problem was discussed by Urs Lang, Maël Pavón, and Roger Züst \cite[3.1]{lang-pavon-zust}.
\begin{figure}[h!]
\vskip-0mm
\centering
\includegraphics{mppics/pic-505}
\end{figure}
They also show that the constant 2 is optimal.
To see this look at the injective envelops of two 4-point metric spaces shown on the diagram and observe that the Gromov--Hausdorff distance between the 4-point metric spaces is 1, while the distance between their injective envelops approaches 2 as $s\to\infty$. 




\parbf{\ref{ex:GH-po:a}.}
In order to check that $\dist{*}{*}{\GH'}$ is a metric, it is sufficient to show that
\[\dist{\spc{X}}{\spc{Y}}{\GH'}=0 
\quad\Longrightarrow\quad
\spc{X}\iso\spc{Y};\]
the remaining conditions are trivial.

If $\dist{\spc{X}}{\spc{Y}}{\GH'}=0$, then there is a sequence of maps $f_n\:\spc{X}\to \spc{Y}$ such that 
\[\dist{f_n(x)}{f_n(x')}{\spc{Y}}\ge \dist{x}{x'}{\spc{X}}-\tfrac1n.\]

Choose a countable dense set $S$ in $\spc{X}$.
Passing to a subsequence of $f_n$ we can assume that $f_n(x)$ converges for any $x\in S$ as $n\to\infty$;
denote its limit by $f_\infty(x)$.

For each point $x\in\spc{X}$ choose a sequence $x_m\in S$ converging to $x$.
Since $\spc{Y}$ is compact, we can assume in addition that $y_m=f_\infty(x_m)$ converges in $\spc{Y}$.
Set $f_\infty(x)=y$.
Note that the map $f_\infty\:\spc{X}\to \spc{Y}$ is  distance-nondecreasing.

The same way we can construct a distance-nondecreasing map 
$g_\infty\:\spc{Y}\to \spc{X}$.

By \ref{ex:non-contracting-map}, the compositions $f_\infty\circ g_\infty\:\spc{Y}\to \spc{Y}$ and $g_\infty\z\circ f_\infty\:\spc{X}\to \spc{X}$ are isometries.
Therefore $f_\infty$ and $g_\infty$ are isometries as well.

%Observe that 
%$$|\spc{X}_n-\spc{X}_\infty|_{\mathcal{M}}\to 0 
%\quad\Longrightarrow\quad 
%\dist{\spc{X}_n}{\spc{X}_\infty}{\GH'}\to 0$$
%follows from Proposition~\ref{prop:GH-e-isom} and Exercise~\ref{ex:alm-isom:inverse}.
%To prove that 
%$$|\spc{X}_n-\spc{X}_\infty|_{\mathcal{M}}\to 0 
%\quad\Longleftarrow\quad 
%\dist{\spc{X}_n}{\spc{X}_\infty}{\GH'}\to 0,$$
%Suppose that $f_n\:\spc{X}_n\to\spc{X}_\infty$ and $g_n\:\spc{X}_\infty\to\spc{X}_n$ are $\eps_n$-almost distance-nondecreasing maps for $\eps_n\to 0$.
%Arguing as above, pass to a partial limit $h$ of the sequence $f_n\circ g_n\:\spc{X}_\infty\to\spc{X}_\infty$.
%Note that $h$ is a distance non-deceasing map from a compact space to an itself.
%By Exercise~\ref{ex:non-contracting-map}, $h$ is an isometry.


\parbf{\ref{pr:doubling}.}
Choose a space $\spc{X}$ in $\spc{Q}(C,D)$, denote a $C$-doubling measure by $\mu$.
Without loss of generality we may assume that $\mu(\spc{X})\z=1$.

The doubling condition implies that 
\[\mu[\oBall(p,\tfrac D{2^n})]\ge\tfrac 1{C^n}\]
for any point $x\in \spc{X}$.
It follows that 
\[\pack_{\frac D{2^n}}\spc{X}\le C^n.\]

By \ref{ex:pack-net}, for any $\eps\ge\frac D{2^{n-1}}$, the space $\spc{X}$ admits an $\eps$-net with at most $C^n$ points.
Whence $\spc{Q}(C,D)$ is uniformly totally bounded.

\parbf{\ref{pr:under}.} 
Since $\spc{Y}$ is compact, it has a finite $\eps$-net for any $\eps>0$.
For each $\eps>0$ choose a finite $\eps$-net $\{y_1,\dots,y_{n_\eps}\}$ in $\spc{Y}$.

Suppose $f\:\spc{X}\to \spc{Y}$ be a distance-nondecreasing map.
Choose one point $x_i$ in each nonempty subset $B_i=f^{-1}[\oBall(y_i,\eps)]$.
Note that the subset $B_i$ has diameter at most $2\cdot \eps$ and 
\[\spc{X}=\bigcup_iB_i.\]
Therefore the set of points $\{x_i\}$ forms a $2\cdot\eps$ net in $\spc{X}$.
Whence \ref{SHORT.pr:under:if} follows.

\parit{\ref{SHORT.pr:under:only-if}.} Let $\spc{Q}$ be a uniformly totally bounded family of spaces. 
Suppose that each space in $\spc{Q}$ has an $\tfrac1{2^n}$-net with at most $M_n$ points; we may assume that $M_0=1$.

Consider the space $\spc{Y}$ of all infinite integer sequences $m_0,m_1,\dots$ such that $1\le m_n\le M_n$ for any $n$.
Given two sequences $(\ell_n)$, and $(m_n)$ of points in $\spc{Y}$, set 
\[\dist{(\ell_n)}{(m_n)}{\spc{Y}}=\tfrac C{2^{n}},\]
where $n$ is minimal index such that $\ell_n\ne m_n$ and $C$ is a positive constant.

Observe that $\spc{Y}$ is compact.
Indeed it is complete and the sequences constant starting from index $n$ form a finite $\tfrac C{2^{n}}$-net in $\spc{Y}$.

Given a space $\spc{X}$ in $\spc{Q}$,
choose a sequence of $\tfrac1{2^n}$ nets 
$N_n\subset\spc{X}$ for each natural $n$.
We can assume that $|N_n|\le M_n$; let us enumerate the points in $N_n$ by $\{1,\dots,M_n\}$.
Consider the map $f:\spc{X}\to\spc{Y}$ defined by $f:x\to (m_1(x),m_2(x),\dots)$ where $m_n(x)$ is a number of the point in $N_n$ that lies on the distance $<\tfrac1{2^n}$ from $x$.

If $\tfrac1{2^{n-2}}\ge \dist{x}{x'}{\spc{X}}>\tfrac1{2^{n-1}}$, then $m_n(x)\ne m_n(x')$.
It follows that $\dist{f(x)}{f(x')}{\spc{Y}}\ge \tfrac C{2^{n}}$.
In particular, if $C>10$, then 
\[\dist{f(x)}{f(x')}{\spc{Y}}\ge \dist{x}{x'}{\spc{X}}\]
for any $x,x'\in \spc{X}$.
That is, $f$ is a distance-nondecreasing map $\spc{X}\to \spc{Y}$.

\parbf{\ref{ex:GH-SC},} \ref{SHORT.ex:GH-SC:circle}.
Suppose $\spc{X}_n\GHto \spc{X}$ and $\spc{X}_n$ are simply connected length metric space.
It is sufficient to show that any nontrivial covering map $f\:\tilde{\spc{X}}\to \spc{X}$ corresponds to a nontrivial covering map $f_n\:\tilde{\spc{X}}_n\to \spc{X}_n$ for large $n$.

The latter can be constructed by covering $\spc{X}_n$ by small balls that lie close to sets in $\spc{X}$ evenly covered by $f$, prepare few copies of these sets and glue them the same way as the inverse images of the evenly covered sets in $\spc{X}$ glued to obtain $\tilde{\spc{X}}$.

\begin{wrapfigure}{r}{40 mm}
\vskip-0mm
\centering
\includegraphics{mppics/pic-2}
\end{wrapfigure}

\parit{\ref{SHORT.ex:GH-SC:nonsc-limit}.}
Let $\spc{V}$ be a cone over Hawaiian earring.
Consider the {}\emph{doubled cone} $\spc{W}$ --- two copies of $\spc{V}$ with glued base points earrings (see the diagram).

The space $\spc{W}$ can be equipped with length metric for example the induced length metric from the shown embedding.

Note that $\spc{V}$ is simply connected, but $\spc{W}$ is not --- it is a good exercise in topology.

If we delete from the earrings all small circles, then the obtained double cone becomes simply connected and it remains to be close to $\spc{W}$ in the sense of Gromov--Hausdorff.

\parit{Remark.}
Note that from part \ref{SHORT.ex:GH-SC:nonsc-limit}, the limit does not admit a nontrivial covering.
So if we define fundamental group right --- as the inverse image of groups of deck transformations for all its coverings, then one may say that Gromov--Hausdorff limit of simply connected length spaces is simply connected.

\parbf{\ref{ex:sphere-to-ball},}
\textit{\ref{SHORT.ex:sphere-to-ball:2}.}
Suppose that a metric on $\mathbb{S}^2$ is close to the disc $\DD^2$.
Note that $\mathbb{S}^2$ contains a circle $\gamma$ that is close to the boundary curve of $\DD^2$.
By Jordan curve theorem, $\gamma$ divides $\mathbb{S}^2$ into two discs, say $D_1$ and $D_2$.

By \ref{ex:GH-SC:nonsc-limit}, the Gromov--Hausdorff limit of $D_1$ and $D_2$ have to contain whole $\DD^2$, otherwise the limit would admit a nontrivial covering.
Consider points $p_1\in D_1$ and $p_2\in D_2$ that a close to the center of $\DD^2$.
On one hand the distance $\dist{p_1}{p_2}{n}$ have to be very small.
On the other hand, any curve from $p_1$ to $p_2$ must cross $\gamma$, so it has length about 2 at least --- a contradiction.



\parit{\ref{SHORT.ex:sphere-to-ball:3}.}
Make fine burrows in the standard 3-ball without changing its topology,
but at the same time come sufficiently close to any point in the ball.

Consider the \index{doubling}\emph{doubling} of the obtained ball along its boundary;
that is, two copies of the ball with identified corresponding points on their boundaries.
The obtained space is homeomorphic to $\mathbb{S}^3$.
Note that the burrows can be made 
so that the obtained space is sufficiently close to the original ball 
in the Gromov--Hausdorff metric.

\parit{Source:} \cite[Exercises 7.5.13 and 7.5.17]{burago-burago-ivanov}. 

%%%%%%%%%%%%%%%%%%%%%%%%%%%%%%%%

%%%%%%%%%%%%%%%%%%%%%%%%%%%%%%

\parbf{\ref{ex:prop:eps-isometry=isometry}.}
Suppose that  $f_n\:\spc{X}\to \spc{Y}$ is a $\tfrac1n$-isometry between compact spaces for each $n\in\NN$.
Consider the $\omega$-limit $f_\omega$ of~$f_n$,
\[f_\omega(x)=\lim_{n\to\omega}f_n(x);\]
according to \ref{prop:ultra/compact}, $f_\omega$ is defined.
Since 
\[|f_n(x)-f_n(x')|\lege |x-x'|\pm\tfrac1n\]

we get that 
\[|f_\omega(x)-f_\omega(x')|= |x-x'|\]
for any $x,x'\in \spc{X}$;
that is, $f_\omega$ is distance-preserving.
Further, since $f_n$ is a $\tfrac1n$-isometry,
for any $y\in \spc{Y}$ there is $x_m$ such that $|f_n(x_n)-y|\le \tfrac1n$.
Therefore,
\[f_\omega(x_\omega)=y,\]
where $x_\omega$ is the $\omega$-limit of $x_n$;
that is, $f_\omega$ is onto.
It follows that $f_\omega\:\spc{X}\to\spc{Y}$ is an isometry.

\parbf{\ref{ex:linear}.}
Choose a nonprincipal ultrafilter $\omega$ and set $L(\bm{s})=s_\omega$.
It remains to observe that $L$ is linear.

\parbf{\ref{ex:lim(tree)}.}
Let $\gamma$ be a path from $p$ to $q$ in a metric tree $\spc{T}$.
Assume that $\gamma$ passes thru a point $x$ on distance $\ell$ from $[pq]$.
Then 
\[\length\gamma\ge \dist{p}{q}{}+2\cdot \ell.
\eqlbl{eq:+ell}\]

Suppose that $\spc{T}_n$ is a sequence of metric trees that $\omega$-converges to $\spc{T}_\omega$.
By \ref{obs:ultralimit-is-geodesic}, the space $\spc{T}_\omega$ is geodesic.

The uniqueness of geodesics follows from \ref{eq:+ell}.
Indeed, if for a geodesic $[p_\omega q_\omega]$ there is another geodesic $\gamma_\omega$ connecting its ends, then it has to pass thru a point $x_\omega\notin [p_\omega q_\omega]$.
Choose sequences $p_n,q_n,x_n\in\spc{T}_n$ such that $p_n\to p_\omega$, $q_n\to q_\omega$, and $x_n\to x_\omega$ as $n\to\omega$.
Then 
\begin{align*}
\dist{p_\omega}{q_\omega}{}&=\length\gamma\ge \lim_{n\to\omega}(\dist{p_n}{x_n}{}+\dist{q_n}{x_n}{})\ge
\\
&\ge \lim_{n\to\omega}(\dist{p_n}{q_n}{}+2\cdot\ell_n)=
\\
&=\dist{p_\omega}{q_\omega}{}+2\cdot\ell_\omega.
\end{align*}
Since $x_\omega\notin [p_\omega q_\omega]$, we have that $\ell_\omega>0$ --- a contradiction.

It remains to show that any geodesic triangle $\spc{T}_\omega$ is a tripod.
Consider the sequence of centers of tripods $m_n$ for a given sequences of points $x_n,y_n,z_n\in \spc{T}_n$.
Observe that its ultralimit $m_\omega$ is the center of the tripod with ends at $x_\omega,y_\omega,z_\omega\in \spc{T}_\omega$.

\parbf{\ref{ex:ultrapower}.}
Further, we consider $\spc{X}$ as a subset of $\spc{X}^\omega$.

\parit{\ref{SHORT.ex:ultrapower:a}.} Follows directly from the definitions.

\parit{\ref{SHORT.ex:ultrapower:compact}.}
Suppose $\spc{X}$ compact.
Given a sequence $x_n$ in $\spc{X}$, denote its $\omega$-limit in $\spc{X}^\omega$ by $x^\omega$ and its $\omega$-limit in $\spc{X}$ by $x_\omega$.

Observe that $x^\omega=\iota(x_\omega)$.
Therefore, $\iota$ is onto.

If $\spc{X}$ is not compact, we can choose a sequence $x_n$ such that $\dist{x_m}{x_n}{}>\eps$ for fixed $\eps>0$ and all $m\ne n$.
Observe that
\[\lim_{n\to\omega}\dist{x_n}{y}{\spc{X}}\ge \tfrac\eps2\]
for any $y\in\spc{X}$.
It follows that $x_\omega$ lies on the distance at least $\tfrac\eps2$ from~$\spc{X}$.

\parit{\ref{SHORT.ex:ultrapower:proper}.}
A sequence of points $x_n$ in $\spc{X}$ will be called $\omega$-bounded if there is a real constant $C$ such that
\[\dist{p}{x_n}{\spc{X}}\le C\] 
for $\omega$-almost all $n$.

The same argument as in \ref{SHORT.ex:ultrapower:compact} shows that any $\omega$-bounded sequence has its $\omega$-limit in $\spc{X}$.
Further, if $(x_n)$ is not  $\omega$-bounded, then 
\[\lim_{n\to\omega}\dist{p}{x_n}{\spc{X}}=\infty;\]
that is, $x_\omega$ does not lie in the metric component of $p$ in $\spc{X}^\omega$.

\parbf{\ref{ex:isom-ultrapower}.} Show and use that the spaces $\spc{X}^\omega$ and $(\spc{X}^\omega)^\omega$ have discrete metric and both have cardinality continuum.

\parbf{\ref{ex:two-geodesics-in-ultrapower}.}
Apply \ref{lem:X-X^w} and \ref{obs:ultrapower-is-geodesic}.

\parbf{\ref{ex:notproper-limit}.} Consider the infinite metric $\spc{T}$ tree with unit edges shown
\begin{figure}[h!]
\vskip-0mm
\centering
\includegraphics{mppics/pic-605}
\end{figure}
on the diagram. Observe that $\spc{T}$ is proper.

Consider the vertex $v_\omega=\lim_{n\to\omega}v_n$ in the ultrapower $\spc{T}^\omega$.
Observe that $\omega$ has an infinite degree.
Conclude that $\spc{T}^\omega$ is not locally compact.

\parbf{\ref{ex:ultraT}.} Consider a product space $[0,1]\times[0,\tfrac12]\times[0,\tfrac14]\times\dots$.

\parbf{\ref{ex:Asym(Lob)}}; \ref{SHORT.ex:Asym(Lob):metric-tree}.
Show that there is $\delta>0$ such that sides of any geodesic triangle intersect a disk of radius $\delta$.
Conclude that any geodesic triangle in $\Asym\spc{L}$ is a tripod.
Make a conclusion.

\parit{\ref{SHORT.ex:Asym(Lob):homogeneous}.} Observe that $L$ is one-point homogeneous and use it.

\parit{\ref{SHORT.ex:Asym(Lob):continuum}.} 
By \ref{SHORT.ex:Asym(Lob):homogeneous}, it is sufficient to show that $p_\omega$ has a continuum degree.

Choose distinct geodesics $\gamma_1,\gamma_2\:[0,\infty)\to L$ that start at a point $p$.
Show that the limits of $\gamma_1$ and $\gamma_2$ run in the different connected components of $(\Asym\spc{L})\setminus \{p_\omega\}$.
Since there is a continuum of distinct geodesics starting at $p$,
we get that the degree of $p_\omega$ is at least continuum.

On the other hand, the set of sequences of points in $L$  has cardinality continuum.
In particular, the set of points in $\Asym\spc{L}$ has cardinality at most continuum.
It follows that the degree of any vertex is at most continuum.

\parit{\ref{SHORT.ex:Asym(Lob):others}.}
The proof for the Lobachevsky space goes along the same lines.

For the infinite 3-regular tree, part \ref{SHORT.ex:Asym(Lob):metric-tree} follows from \ref{ex:lim(tree)}.
The 3-regular tree is not one-point homogeneous, but it is vertex homogeneous; the latter is sufficient to prove \ref{SHORT.ex:Asym(Lob):homogeneous}.
No changes are needed in \ref{SHORT.ex:Asym(Lob):continuum}.

\parit{Remark.}
Anna Dyubina and Iosif Polterovich \cite{dyubina-polterovich} proved that the properties \ref{SHORT.ex:Asym(Lob):homogeneous} and \ref{SHORT.ex:Asym(Lob):continuum} describe the tree $\spc{T}$ up to isometry.
In particular, the asymptotic space of the Lobachevsky plane does not depend on the choice of ultrafilter and the sequence $\lambda_n\to \infty$.


%%%%%%%%%%%%%%%%%%%%%%%%%%%%
{\small\sloppy
\documentclass[twoside]{book}

\usepackage{lectures}
\usepackage[colorlinks=true,
citecolor=black,
linkcolor=black,
anchorcolor=black,
filecolor=black,
menucolor=black,
urlcolor=black,
pdftitle={Pure metric geometry: introductory lectures},
pdfsubject={Geometry},
pdfauthor={Anton Petrunin}
]{hyperref}
\makeindex

\begin{document}
%\pagestyle{empty}\renewcommand\includegraphics[2][{}]{}\def\emph{\textit}
%\overfullrule=100mm

 
\title{Pure metric geometry:\\
introductory lectures}
\author{Anton Petrunin}
\date{}
\maketitle

We discuss only domestic affairs of metric spaces;
applications are given only as illustrations.

These notes are based on couses at PSU (Spring 2020) and SPbSU (Fall 2022).
An extended version can be found on the author's website;
it includes an introduction to Alexandrov geometry based on \cite{alexander-kapovitch-petrunin-2019} and metric geometry on manifolds \cite{petrunin2020mnfld} based on a simplified proof of Gromov's systolic inequality given by Alexander Nabutovsky~\cite{nabutovsky}.

A part of the text is a compilation from \cite{alexander-kapovitch-petrunin-2019, alexander-kapovitch-petrunin-2025, petrunin-yashinski, petrunin-2022-PIGTIKAL, petrunin-zamorabarrera} and its drafts.

I want to thank
Alexander Lytchak,
Julien Melleray,
and Sergio Zamora Barrera for help.
The present work is partially supported by NSF grant DMS-2005279
and by the Simons Foundation under grant \#584781.

\thispagestyle{empty}
\tableofcontents
\thispagestyle{empty}

\include{metric}
\include{uryson}
\include{injective}
\include{converge}
\include{ultralimit}
%\include{wasserstein}

\appendix
\chapter{Semisolutions}
\input{metric-sol}
\input{uryson-sol}
\input{injective-sol}
\input{converge-sol}
\input{ultralimit-sol}

%%%%%%%%%%%%%%%%%%%%%%%%%%%%
{\small\sloppy
\input{pure-metric.ind}

\def\emph{\textit}

\printbibliography[heading=bibintoc]
\fussy
}


\end{document}


\def\emph{\textit}

\printbibliography[heading=bibintoc]
\fussy
}


\end{document}


\def\emph{\textit}

\printbibliography[heading=bibintoc]
\fussy
}


\end{document}


\def\emph{\textit}

\printbibliography[heading=bibintoc]
\fussy
}


\end{document}
