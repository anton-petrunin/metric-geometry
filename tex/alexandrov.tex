\chapter{Introduction}

\section{Manifesto}

Alexandrov geometry can use ``back to Euclid'' as a slogan.
Alexandrov spaces are defined via axioms similar to those given by Euclid,
but certain  equalities are changed to inequalities. 
Depending on the sign of the inequalities, we get Alexandrov spaces with {}\emph{curvature bounded above} or {}\emph{curvature bounded below}.
The definitions of the two classes of spaces are similar, but their properties and known applications are quite different.


Consider the space $\mathcal{M}_4$ of all isometry classes of 4-point metric spaces.
Each element in $\mathcal{M}_4$ can be described by 6 numbers 
 --- the distances between all 6 pairs of its points, say $\ell_{i,j}$ for $1\le i< j\le 4$ modulo permutations of the index set $(1,2,3,4)$.
These 6 numbers are subject to 12 triangle inequalities; that is,
\[\ell_{i,j}+\ell_{j,k}\ge \ell_{i,k}\]
holds for all $i$, $j$ and $k$, where we assume that $\ell_{j,i}=\ell_{i,j}$ and $\ell_{i,i}=0$.

The space $\mathcal{M}_4$ comes with topology.
It can defined as a quotient of the cone in $\RR^6$ by permutations of the 4 points of the space.
And, the same topology is induced on $\mathcal{M}_4$ by the Gromov--Hausdorff metric.

\begin{wrapfigure}[7]{o}{33mm}
\vskip-0mm
\centering
\includegraphics{mppics/pic-700}
\end{wrapfigure}

Consider the subset $\mathcal{E}_4\subset \mathcal{M}_4$ of all isometry classes of 4-point metric spaces that admit isometric embeddings into Euclidean space.

\begin{thm}{Claim}\label{clm:two-components-of-M4}
The complement $\mathcal{M}_4\setminus \mathcal{E}_4$ has two connected components.
\end{thm}

A proof of the claim can be extracted from~\ref{ex:CAT+CBB}.

The definition of Alexandrov spaces is based on this claim.
Let us denote one of the components by $\mathcal{P}_4$ and the other by~$\mathcal{N}_4$.
Here $\mathcal{P}$ and $\mathcal{N}$ stand for {}\emph{positive} 
and {}\emph{negative curvature} because spheres have no quadruples of type $\mathcal{N}_4$ and 
hyperbolic space
has no quadruples of type~$\mathcal{P}_4$.

A metric space, with length metric, 
that has no quadruples of points of type $\mathcal{P}_4$ or $\mathcal{N}_4$
respectively 
is called an Alexandrov space with non-positive ($\CAT(0)$) or non-negative curvature ($\CBB(0)$).

\begin{wrapfigure}{r}{33mm}
\vskip-4mm
\centering
\includegraphics{mppics/pic-710}
\end{wrapfigure}

Let us describe the subdivision into  $\mathcal{P}_4$, $\mathcal{E}_4$, and $\mathcal{N}_4$ intuitively.
Imagine that you move out of $\mathcal{E}_4$ --- your path is a one-parameter family of 4-point metric spaces.
The last thing you see in $\mathcal{E}_4$ is one of the two plane configurations shown on the diagram.
If you see the left configuration then you move into $\mathcal{N}_4$;
if it is the one on the right, then you move into $\mathcal{P}_4$.
More degenerate pictures can be avoided; for example, a triangle with a point on a side.
From such a configuration one may move in $\mathcal{N}_4$ and $\mathcal{P}_4$ (as well as come back to $\mathcal{E}_4$).

Here is an exercise, solving which would force you to rebuild a considerable part of Alexandrov geometry.
It might be helpful to spend some time thinking about this exercise before proceeding.

\begin{thm}{Advanced exercise}\label{ex:convex-set}
Assume $\spc{X}$ is a complete metric space with length metric, 
containing only quadruples of type~$\mathcal{E}_4$.
Show that $\spc{X}$ is isometric to a convex set in a Hilbert space.
\end{thm}

In the definition above, 
instead of  Euclidean space 
one can take  
hyperbolic space of curvature~$-1$.
In this case,
one obtains the definition of spaces with curvature bounded above or below by~$-1$ ($\CAT(-1)$ or $\CBB(-1)$).

To define spaces with curvature bounded above or below by $1$ ($\CAT(1)$ or $\CBB(1)$),
one has to take the unit 3-sphere 
and specify that only the quadruples of points such that each of the four triangles has perimeter 
less than $2\cdot\pi$ are checked.
The latter condition could be considered as a part of the {}\emph{spherical triangle inequality}.

\section{Triangles, hinges and angles}

Let $\spc{X}$ be a metric space.

\parbf{Triangles.}
For a triple of points $p,q,r\in \spc{X}$, a choice of a triple of geodesics $([q r], [r p], [p q])$ will be called a \index{triangle}\emph{triangle}; we will use the short notation 
$\trig p q r=\trig p q r_{\spc{X}}=([q r], [r p], [p q])$\index{$\trig {{*}}{{*}}{{*}}$}.

Given a triple $p,q,r\in \spc{X}$ there may be no triangle 
$\trig p q r$ simply because one of the pairs of these points cannot be joined by a geodesic.
Also, many different triangles with these vertices may exist, any of which can be denoted by $\trig p q r$.
However, if we write $\trig p q r$, it means that we have made a choice of such a triangle; 
that is, we have  fixed a choice of the geodesics $[q r]$, $[r p]$, and $[p q]$.

The value 
\[\dist{p}{q}{}+\dist{q}{r}{}+\dist{r}{p}{}\]
will be called the {}\emph{perimeter of the triangle} $\trig p q r$.

\parbf{Model triangles.}Given three points $p,q,r$ in a metric space $\spc{X}$,
let us define the \index{model triangle}\emph{model triangle} $\trig{\tilde p}{\tilde q}{\tilde r}$ 
(briefly, 
$\trig{\tilde p}{\tilde q}{\tilde r}=\modtrig(p q r)_{\EE^2}$%
\index{$\modtrig$!$\modtrig({*}{*}{*})_{\EE^2}$}) to be a triangle in the Euclidean plane $\EE^2$ such that
\begin{align*}\dist{\tilde p}{\tilde q}{\EE^2}&=\dist{p}{q}{\spc{X}},
\\
\quad\dist{\tilde q}{\tilde r}{\EE^2}&=\dist{q}{r}{\spc{X}},
\\
\quad\dist{\tilde r}{\tilde p}{\EE^2}&=\dist{r}{p}{\spc{X}}.
\end{align*}

In the same way we can define the \index{hyperbolic model triangle}\emph{hyperbolic} and the \index{spherical model triangles}\emph{spherical model triangles} $\modtrig(p q r)_{\HH^2}$, $\modtrig(p q r)_{\SSS^2}$
in the hyperbolic plane $\HH^2$ and the unit sphere~$\SSS^2$.
In the latter case the model triangle is said to be defined if in addition
\[\dist{p}{q}{}+\dist{q}{r}{}+\dist{r}{p}{}< 2\cdot\pi.\]
In this case the model triangle again exists and is unique up to an isometry of~$\SSS^2$.

\parbf{Model angles.}
If 
$\trig{\tilde p}{\tilde q}{\tilde r}=\modtrig(p q r)_{\EE^2}$ 
and $\dist{p}{q}{},\dist{p}{r}{}>0$, 
the angle measure of 
$\trig{\tilde p}{\tilde q}{\tilde r}$ at $\tilde p$ 
will be called the \index{model angle}\emph{model angle} of the triple $p$, $q$, $r$ and will be denoted by
$\angk p q r_{\EE^2}$%
\index{$\tilde\measuredangle$!$\angk{{*}}{{*}}{{*}}$}.
In the same way we define $\angk p q r_{\HH^2}$ and $\angk p q r_{\SSS^2}$;
in the latter case  we assume in addition that the model triangle $\modtrig(p q r)_{\SSS^2}$ is defined.

We may use the notation $\angk p q r$ if it is evident which of the model spaces $\HH^2$, $\EE^2$ or $\SSS^2$ is meant.

\parbf{Hinges.} Let $p,x,y\in \spc{X}$ be a triple of points such that $p$ is distinct from $x$ and~$y$.
A pair of geodesics $([p x],[p y])$ will be called  a \index{hinge}\emph{hinge} and will be denoted by 
$\hinge p x y=([p x],[p y])$\index{$\hinge{{*}}{{*}}{{*}}$}.

\begin{wrapfigure}{r}{25mm}
\vskip-0mm
\centering
\includegraphics{mppics/pic-750}
\end{wrapfigure}

\parbf{Angles.}
Given a hinge $\hinge p x y$, we define its \index{angle}\emph{angle} as 
the limit\index{$\mangle$!$\mangle\hinge{{*}}{{*}}{{*}}$}
\[\mangle\hinge p x y
\df
\lim_{\bar x,\bar y\to p} \angk p{\bar x}{\bar y},\eqlbl{eq:def-angle}\]
where $\bar x\in\left]p x\right]$ and $\bar y\in\left]p y\right]$.
The angle $\mangle\hinge p x y$ is defined if the limit exists.

It is straightforward to check that in \ref{eq:def-angle}, one can use $\angk p{\bar x}{\bar y}_{\SSS^2}$ or  $\angk p{\bar x}{\bar y}_{\HH^2}$ or $\angk p{\bar x}{\bar y}_{\EE^2}$, the result will be the same.

\begin{thm}{Exercise} Give an example of a hinge $\hinge p x y$ in a metric space with undefined angle $\mangle\hinge p x y$.
\end{thm}

\begin{thm}{Exercise}\label{ex:tringle-inq-angles} Suppose that for three geodesics $[px]$, $[py]$, and $[pz]$ in a metric space, the angles 
$\alpha=\mangle\hinge pxy$,
$\beta=\mangle\hinge pyz$,
and 
$\gamma=\mangle\hinge pzx$ are defined.
Show that $\alpha$, $\beta$ and $\gamma$ satisfy all triangle inequalities:
\begin{align*}
\alpha&\le \beta+\gamma,
&
\beta&\le\gamma+\alpha,
&
\gamma&\le \alpha+\beta,
\end{align*}

\end{thm}


\section{Definitions}
\label{sec:def-CAT}

\parbf{Curvature bounded above.}
Given a quadruple of points $p,q,x,y$ in a metric space $\spc{X}$,
consider two model triangles 
$\trig{\tilde p}{\tilde x}{\tilde y}=\modtrig{}(pxy)_{\EE^2}$ 
and 
$\trig{\tilde q}{\tilde x}{\tilde y}\z=\modtrig{}(qxy)_{\EE^2}$ with common side $[\tilde x\tilde y]$.

\begin{wrapfigure}{r}{25mm}
\vskip-4mm
\centering
\includegraphics{mppics/pic-720}
\end{wrapfigure}

If the inequality
\[\dist{p}{q}{\spc{X}}\le \dist{\tilde p}{\tilde z}{\EE^2}+\dist{\tilde z}{\tilde q}{\EE^2}\]
holds for any point $\tilde z\in [\tilde x\tilde y]$, then we say that 
the quadruple $p,q,x,y$ satisfies \index{$\CAT(0)$ comparison}\emph{$\CAT(0)$ comparison}.
\label{page:CAT-comparison}


If we do the same for spherical model triangles  
$\trig{\tilde p}{\tilde x}{\tilde y}=\modtrig{}(pxy)_{\SSS^2}$ 
and 
$\trig{\tilde q}{\tilde x}{\tilde y}=\modtrig{}(qxy)_{\SSS^2}$,
then we arrive at the definition of $\CAT(1)$ comparison.
If one of the spherical model triangles is undefined,\footnote{That is, if 
\[\dist{p}{x}{}+\dist{p}{y}{}+\dist{x}{y}{}\ge 2\cdot\pi
\quad
\text{or}
\quad
\dist{q}{x}{}+\dist{q}{y}{}+\dist{x}{y}{}\ge 2\cdot\pi.\]}
then it is assumed that $\CAT(1)$ comparison automatically holds for this quadruple.

We can do the same for the \index{model plane}\emph{model plane} of curvature $\kappa$;
that is, a sphere if $\kappa>0$, 
Euclidean plane if $\kappa=0$ 
and Lobachevsky plane if $\kappa<0$.
In this case we arrive at the definition of $\CAT(\kappa)$ comparison.
However we will mostly consider $\CAT(0)$ comparison and occasionally $\CAT(1)$ comparison;
so, if you see $\CAT(\kappa)$, you can assume that $\kappa$ is $0$ or~$1$.

If all quadruples in a metric space $\spc{X}$ satisfy $\CAT(\kappa)$ comparison, then we say that the space $\spc{X}$ is $\CAT(\kappa)$
(we use $\CAT(\kappa)$ as an adjective).

Here $\CAT$ is an acronym for Cartan, Alexandrov, and Toponogov,
but usually pronounced as ``cat'' in the sense of ``miauw''.
The term was coined by Mikhael Gromov in 1987.
Originally, Alexandrov called these spaces \emph{$\mathfrak{R}_\kappa$ domain};
this term is still in use.

\begin{thm}{Exercise}\label{ex:sba-2+2-short}
Show that a metric space $\spc{X}$ is $\CAT(0)$
if and only if for any quadruple of points 
$p,q,x,y$ in $\spc{X}$ 
there is a quadruple $\tilde p,\tilde q,\tilde x,\tilde y$ in $\EE^2$
such that 
\begin{align*}
\dist{\tilde p}{\tilde q}{}&=\dist{p}{q}{},
&
\dist{\tilde x}{\tilde y}{}&=\dist{x}{y}{},
\\
\dist{\tilde p}{\tilde x}{}&\le \dist{p}{x}{},
&
\dist{\tilde p}{\tilde y}{}&\le \dist{p}{y}{},
\\
\dist{\tilde q}{\tilde x}{}&\le \dist{q}{x}{},
&
\dist{\tilde q}{\tilde y}{}&\le \dist{q}{y}{}.
\end{align*}

\end{thm}

\parbf{Curvature bounded below.}
If the inequality
\[\angk pxy_{\EE^2}+\angk pyz_{\EE^2}+\angk pzx_{\EE^2}\le2\cdot\pi\]
holds for points $p,x,y,z$ in a metric space $\spc{X}$,
then we say that 
the quadruple $p,x,y,z$ \index{$\CBB(0)$ comparison}\emph{satisfies $\CBB(0)$ comparison}.\label{page:CBB-comparison}

If we do the same for spherical or hyperbolic model angles,
then we arrive at the definition of $\CBB(1)$ or $\CBB(-1)$ comparison.
Here $\CBB(\kappa)$ is an abbreviation of {}\emph{curvature bounded below by $\kappa$}.
If one of one of the model angles is undefined,
then we assume that $\CBB(1)$ comparison automatically holds for this quadruple.

We can do the same for the model plane of curvature $\kappa$.
In this case we arrive at the definition of $\CAT(\kappa)$ comparison.
But we will mostly consider  $\CBB(0)$ comparison and occasionally $\CBB(1)$ comparison;
so, if you see $\CBB(\kappa)$, you can assume that $\kappa$ is $0$ or~$1$.

If all quadruples in a metric space $\spc{X}$ satisfy $\CBB(\kappa)$ comparison, then we say that the space $\spc{X}$ is $\CBB(\kappa)$.
(Again --- $\CBB(\kappa)$ is an adjective.)

\begin{thm}{Exercise}\label{ex:(3+1)-expanding}
Show that a metric space $\spc{X}$ is $\Alex0$
if and only if for any quadruple of points $p,x,y,z\in \spc{X}$,
there is a quadruple of points $\tilde p,\tilde x,\tilde y,\tilde z\in\EE^2$
such that 
\begin{align*}
\dist{p}{x}{\spc{X}}&=\dist{\tilde p}{\tilde x}{\EE^2},
&
\dist{p}{y}{\spc{X}}&=\dist{\tilde p}{\tilde y}{\EE^2},
&
\dist{p}{z}{\spc{X}}&=\dist{\tilde p}{\tilde z}{\EE^2},
\\
\dist{x}{y}{\spc{X}}&\le\dist{\tilde x}{\tilde y}{\EE^2},
&
\dist{y}{z}{\spc{X}}&\le\dist{\tilde y}{\tilde z}{\EE^2},
&
\dist{z}{x}{\spc{X}}&\le\dist{\tilde z}{\tilde x}{\EE^2}
\end{align*}
for all $i$ and $j$.
\end{thm}

\begin{thm}{Exercise}\label{ex:CAT+CBB}
Suppose that a quadruple of points satisfies $\CAT(0)$ and $\CBB(0)$ for all labeling.
Show that the quadruple is isometric to a subset of Euclidean space.
\end{thm}

Observe that in order to check $\CAT(\kappa)$ or $\CBB(\kappa)$ comparison, it is sufficient to know the 6 distances between all pairs of points in the quadruple.
This observation implies the following.

\begin{thm}{Proposition}\label{prop:cat-limit}
Any Gromov--Hausdorff limit (as well as ultra limit) of a sequence of $\CAT(\kappa)$ or $\CBB(\kappa)$ spaces is $\CAT(\kappa)$ or $\CBB(\kappa)$ respectively. 
\end{thm}

In the proposition above, 
it does not matter which definition of convergence for metric spaces you use, 
as long as any quadruple of points in the limit space can be arbitrarily well approximated by  quadruples in the sequence of metric spaces. 

\section{Products and cones}
\label{sec:Products and cones}

Given two metric spaces $\spc{U}$ and $\spc{V}$, the \index{product space}\emph{product space} 
$\spc{U}\times\spc{V}$ is defined as the set of all pairs $(u,v)$ where $u\in\spc{U}$ and $v\in \spc{V}$ 
with the metric defined by formula
\[\dist{(u_1,v_1)}{(u_2,v_2)}{\spc{U}\times\spc{V}}=\sqrt{\dist[2]{u_1}{u_2}{\spc{U}}+\dist[2]{v_1}{v_2}{\spc{V}}}.\]

\begin{thm}{Proposition}\label{ex:product-CAT}
Let $\spc{U}$ and $\spc{V}$ be $\CAT(0)$ spaces.
Then the product space $\spc{U}\times\spc{V}$ is $\CAT(0)$.
\end{thm}

\parit{Proof.}
Fix a quadruple in $\spc{U}\times \spc{V}$:
\begin{align*}
p&=(p_1,p_2),
&
q&=(q_1,q_2), 
&
x&=(x_1,x_2),
&
y&=(y_1,y_2).
\end{align*}
For the quadruple $p_1,q_1,x_1,y_1$ in $\spc{U}$,
construct two model triangles $\trig{\tilde p_1}{\tilde x_1}{\tilde y_1}=\modtrig(p_1x_1y_1)_{\EE^2}$ 
and $\trig{\tilde q_1}{\tilde x_1}{\tilde y_1}=\modtrig(q_1x_1y_1)_{\EE^2}$.  
Similarly, for the quadruple $p_2,q_2,x_2,y_2$ in $\spc{V}$
construct two model triangles $\trig{\tilde p_2}{\tilde x_2}{\tilde y_2}$ and $\trig{\tilde q_2}{\tilde x_2}{\tilde y_2}$.

Consider four points in $\EE^4=\EE^2\times\EE^2$ 
\begin{align*}
\tilde p&=(\tilde p_1,\tilde p_2),
&
\tilde q&=(\tilde q_1,\tilde q_2),
&
\tilde x&=(\tilde x_1,\tilde x_2),
&
\tilde y&=(\tilde y_1,\tilde y_2).
\end{align*}
Note that the triangles $\trig{\tilde p}{\tilde x}{\tilde y}$ and $\trig{\tilde q}{\tilde x}{\tilde y}$ in $\EE^4$ are isometric to the model triangles 
$\modtrig(pxy)_{\EE^2}$ and $\modtrig(qxy)_{\EE^2}$.

If $\tilde z=(\tilde z_1,\tilde z_2)\in [\tilde x\tilde y]$, then $\tilde z_1\in [\tilde x_1\tilde y_1]$ and $\tilde z_2\in [\tilde x_2\tilde y_2]$ and
\begin{align*}
\dist[2]{\tilde z}{\tilde p}{\EE^4}&=\dist[2]{\tilde z_1}{\tilde p_1}{\EE^2}+\dist[2]{\tilde z_2}{\tilde p_2}{\EE^2},
\\
\dist[2]{\tilde z}{\tilde q}{\EE^4}&=\dist[2]{\tilde z_1}{\tilde q_1}{\EE^2}+\dist[2]{\tilde z_2}{\tilde q_2}{\EE^2},
\\
\dist[2]{p}{q}{\spc{U}\times\spc{V}}&=\dist[2]{p_1}{q_1}{\spc{U}}+\dist[2]{p_2}{q_2}{\spc{V}}.
\end{align*}
Therefore $\CAT(0)$ comparison for the quadruples $p_1,q_1,x_1,y_1$ in $\spc{U}$
and 
$p_2,q_2,x_2,y_2$ in $\spc{V}$ implies 
$\CAT(0)$ comparison for the quadruples $p,q,x,y$ in $\spc{U}\times \spc{V}$.
\qeds

\begin{thm}{Exercise}\label{ex:product-CBB}
Assume $\spc{U}$ and $\spc{V}$ are $\CBB(0)$ spaces.
Show that the product space $\spc{U}\times\spc{V}$ is $\CBB(0)$.
\end{thm}

The \label{page:cone}\index{cone}\emph{cone} $\spc{V}=\Cone\spc{U}$ over a metric space $\spc{U}$
is defined as the metric space whose underlying set consists of
equivalence classes in
$[0,\infty)\times \spc{U}$ with the equivalence relation ``$\sim$'' given by $(0,p)\sim (0,q)$ for any points $p,q\in\spc{U}$,
and whose metric is given by the cosine rule
\[
\dist{(p,s)}{(q,t)}{\spc{V}} 
=
\sqrt{s^2+t^2-2\cdot s\cdot t\cdot \cos\alpha},
\]
where $\alpha= \min\{\pi, \dist{p}{q}{\spc{U}}\}$.

The point in the cone $\spc{V}$ formed by the equivalence class of $0\times\spc{U}$ is called the \index{tip of the cone}\emph{tip of the cone} and is denoted by $0$ or $0_{\spc{V}}$.
The distance $\dist{0}{v}{\spc{V}}$ is called the norm of $v$ and is denoted by $|v|$ or $|v|_{\spc{V}}$.
The space $\spc{U}$ can be identified with the subset $x\in \spc{V}$ such that $|x|=1$.

The points in the cone $\spc{V}$ can be multiplied by a real number $\lambda\ge 0$;
namely, if $x=(x',r)$, then $\lambda\cdot x\df(x',\lambda\cdot r)$.



\begin{thm}{Proposition}\label{ex:cone+susp}
Let $\spc{U}$ be a metric space.
Then $\Cone\spc{U}$ is  $\CAT(0)$ if and only if $\spc{U}$ is $\CAT(1)$.
\end{thm}

\parit{Proof; if part.}
Given a point $x\in \Cone\spc{U}$, denote by $x'$ its projection to $\spc{U}$
and by $|x|$ the distance from $x$ to the tip of the cone;
if $x$ is the tip, then $|x|=0$ and we can take any point of $\spc{U}$ as~$x'$.

Let $p$, $q$, $x$, $y$
be a quadruple in $\Cone\spc{U}$.
Assume that the spherical model triangles $\trig{\tilde p'}{\tilde x'}{\tilde y'}_{\SSS^2}=\modtrig(p'x'y')_{\SSS^2}$ and $\trig{\tilde q'}{\tilde x'}{\tilde y'}_{\SSS^2}=\modtrig(q'x'y')_{\SSS^2}$ are defined.
Consider the following points in $\EE^3=\Cone\SSS^2$: 
\begin{align*}
\tilde p&=|p|\cdot\tilde p',
&
\tilde q&=|q|\cdot\tilde q',
&
\tilde x&=|x|\cdot\tilde x',
&
\tilde y&=|y|\cdot\tilde y'.
\end{align*}

Note that
$\trig{\tilde p}{\tilde x}{\tilde y}_{\EE^3}\iso\modtrig(pxy)_{\EE^2}$
and
$\trig{\tilde q}{\tilde x}{\tilde y}_{\EE^3}\iso\modtrig(qxy)_{\EE^2}$.
Further note that if $\tilde z\in [\tilde x\tilde y]_{\EE^3}$, then
$\tilde z'=\tilde z/|\tilde z|$ lies on the geodesic $[\tilde x'\tilde y']_{\SSS^2}$.
Therefore the $\CAT(1)$ comparison for $\dist{p'}{q'}{}$ with $\tilde z'\in[\tilde x'\tilde y']_{\SSS^2}$ implies the 
$\CAT(0)$ comparison for $\dist{p}{q}{}$ with $\tilde z\in[\tilde x\tilde y]_{\EE^3}$.

If at least one of the model triangles $\modtrig(p'x'y')_{\SSS^2}$ and $\modtrig(q'x'y')_{\SSS^2}$ is undefined,
then the statement follows from the triangle inequalities 
\begin{align*}
|p'-x'|_{\spc{U}}+|q'-x'|_{\spc{U}}
&\ge |p'-q'|_{\spc{U}}
\\
|p'-y'|_{\spc{U}}+|q'-y'|_{\spc{U}}
&\ge |p'-q'|_{\spc{U}}
\end{align*}
This case is left as an exercise. %???

\parit{Only-if part.}
Suppose that $\tilde p'$, $\tilde q'$, $\tilde x'$, $\tilde y'$ are defined as above.
Assume all these points lie in a half-space of $\EE^3=\Cone\SSS^2$ with origin at its boundary. 
Then we can choose positive values $a$, $b$, $c$, and $d$ such that the points $a\cdot \tilde p'$, $b\cdot \tilde q'$, $c\cdot \tilde x'$, $d\cdot \tilde y'$ lie in one plane.
Consider the corresponding points $a\cdot  p'$, $b\cdot  q'$, $c\cdot  x'$, $d\cdot y'$ in $\Cone\spc{U}$.
Applying the $\CAT(0)$ comparison for the these points leads to $\CAT(1)$ comparison for the quadruple $ p'$, $q'$, $ x'$, $ y'$ in $\spc{U}$.

It remains to consider the case when $\tilde p'$, $\tilde q'$, $\tilde x'$, $\tilde y'$ do not in a half-space.
Fix $\tilde z'\in [\tilde x' \tilde y']_{\SSS^2}$.
Observe that 
\begin{align*}\dist{\tilde p'}{\tilde x'}{\SSS^2}+\dist{\tilde q'}{\tilde x'}{\SSS^2}
 &
\le \dist{\tilde p'}{\tilde z'}{\SSS^2}+\dist{\tilde q'}{\tilde z'}{\SSS^2}
\intertext{or} 
\dist{\tilde p'}{\tilde y'}{\SSS^2}+\dist{\tilde q'}{\tilde y'}{\SSS^2}
&\le
\dist{\tilde p'}{\tilde z'}{\SSS^2}+\dist{\tilde q'}{\tilde z'}{\SSS^2}.\end{align*}
That is, in this case, the $\CAT(1)$ comparison follow from the triangle inequality.
\qeds


\section{Geodesics}


\begin{thm}{Proposition}\label{ex:CAT-geodesic}
Let $\spc{X}$ be a complete length $\CAT(0)$ space.
Then any two points in $\spc{X}$ are joint by a unique geodesic.
\end{thm}

\parit{Proof.} 
Fix two points $x,y\in \spc{X}$.
Choose a sequence of approximate midpoints $p_n$ for $x$ and $y$;
that is,  
\[\dist{x}{p_n}{}\to\tfrac12\cdot\dist{x}{y}{}
\quad\text{and}\quad
\dist{y}{p_n}{}
\to\tfrac12\cdot\dist{x}{y}{}
\eqlbl{eq:to|p1p2|/2}\]
as $n\to\infty$.

Consider model triangles $[\tilde p_n\tilde x\tilde y]=\modtrig(p_nxy)$.
Let $\tilde z$ be the midpoint of $\tilde x$ and $\tilde y$.
By \ref{eq:to|p1p2|/2}, we have that 
\[\dist{\tilde p_n}{\tilde z}{}\to 0\] as $n\to\infty$.

By $\CAT(0)$ comparison, 
\[\dist{p_n}{p_m}{\spc{X}}\le\dist{\tilde p_n}{\tilde z}{}+\dist{\tilde p_m}{\tilde z}{}.\]
Therefore $\dist{p_n}{p_m}{}\to 0$ as $m,n\to\infty$;
that is, $(p_n)$ is Cauchy.
Clearly the limit of the sequence $p_n$ is a midpoint of $x$ and $y$.
Applying \ref{lem:mid>geod:geod}, we get that $\spc{X}$ is geodesic.

It remains to prove uniqueness.
Suppose there are two geodescics between $x$ and $y$.
Then we can choose two points $p\ne q$ on these geodesics such that $\dist{x}{p}{}=\dist{x}{q}{}$ and therefore $\dist{y}{p}{}=\dist{y}{q}{}$.

Observe that the model triangles $[\tilde p\tilde x\tilde y]=\modtrig(pxy)$ and $[\tilde q\tilde x\tilde y]\z=\modtrig(qxy)$ are degenerate and moreover $\tilde p=\tilde q$.
Applying $\CAT(0)$ comparison with $\tilde z=\tilde p=\tilde q$,
we get that $\dist{p}{q}{}=0$, a contradiction.
\qeds

The following exercise is an analogous statement for $\CBB$ spaces.
In general complete length $\CBB(0)$ space might fail to be geodesic
and uniqueness of geodesic usually does not hold.

\begin{thm}{Exercise}\label{ex:CBB-geodesic}
Let $\spc{X}$ be a complete length $\CBB(0)$ space.
Show that if two geodesics from $x$ to $y$ share yet another point $z$,
then they coincide.
\end{thm}


\section{Alexandrov's lemma}

\begin{thm}{Lemma}
\index{Alexandrov's lemma}
\label{lem:alex}  
Let $p,x,y,z$ be distinct points in a metric space such that $z\in \left]x y\right[$.
Then 
the following expressions for the Euclidean model angles have the same sign:

\begin{subthm}{lem-alex-difference}
$\angk x p y
-\angk x p z$,
\end{subthm} 

\begin{subthm}{lem-alex-angle}
$\angk z p x
+\angk z p y -\pi$.
\end{subthm}

\begin{wrapfigure}{r}{25mm}
\vskip-0mm
\centering
\includegraphics{mppics/pic-730}
\end{wrapfigure}

Moreover,
\[\angk p x y \ge \angk p x z +  \angk p z y,\]
with equality if and only if the expressions in \ref{SHORT.lem-alex-difference} and \ref{SHORT.lem-alex-angle} vanish.

The same holds for the hyperbolic and spherical model angles, 
but in the latter case one has to assume in addition that
\[\dist{p}{z}{}+\dist{p}{y}{}+\dist{x}{y}{}< 2\cdot\pi.\]

\end{thm}


\parit{Proof.} 
Consider the model triangle $\trig{\tilde x}{\tilde p}{\tilde z}=\modtrig(x p z)$.
Take 
a point $\tilde y$ on the extension of 
$[\tilde x \tilde z]$ beyond $\tilde z$ so that $\dist{\tilde x}{\tilde y}{}=\dist{x}{y}{}$ (and therefore $\dist{\tilde x}{\tilde z}{}=\dist{x}{z}{}$). 

\begin{wrapfigure}{r}{33mm}
\vskip-0mm
\centering
\includegraphics{mppics/pic-740}
\end{wrapfigure}

Since increasing the opposite side in a plane triangle increases the corresponding angle, 
the following expressions have the same sign:
\begin{enumerate}[(i)]
\item $\mangle\hinge{\tilde x}{\tilde p}{\tilde y}-\angk{x}{p}{y}$,
\item $\dist{\tilde p}{\tilde y}{}-\dist{p}{y}{}$,
\item $\mangle\hinge{\tilde z}{\tilde p}{\tilde y}-\angk{z}{p}{y}$.
\end{enumerate}
Since 
\[\mangle\hinge{\tilde x}{\tilde p}{\tilde y}=\mangle\hinge{\tilde x}{\tilde p}{\tilde z}=\angk{x}{p}{z}\]
and
\[ \mangle\hinge{\tilde z}{\tilde p}{\tilde y}
=\pi-\mangle\hinge{\tilde z}{\tilde x}{\tilde p}
=\pi-\angk{z}{x}{p},\]
the first statement follows.

For the second statement, construct a model triangle $\trig{\tilde p}{\tilde z}{\tilde y'}\z=\modtrig(pzy)_{\EE^2}$ on the opposite side of $[\tilde p\tilde z]$ from $\trig{\tilde x}{\tilde p}{\tilde z}$.  
Note that 
\begin{align*}
\dist{\tilde x}{\tilde y'}{}
&\le \dist{\tilde x}{\tilde z}{} + \dist{\tilde z}{\tilde y'}{}=
\\
&=\dist{x}{z}{}+\dist{z}{y}{}=
\\
&=\dist{x}{y}{}.
\intertext{Therefore}
\angk{p}{x}{z} + \angk{p}{z}{y} 
&
= 
\mangle\hinge{\tilde p}{\tilde x}{\tilde z}+ \mangle\hinge{\tilde p}{\tilde z}{\tilde y'} 
=
\\
&
= 
\mangle\hinge{\tilde p}{\tilde x}{\tilde y'}
\le
\\
&\le  \angk p x y.
\end{align*}
Equality holds if and only  if $\dist{\tilde x}{\tilde y'}{}=\dist{x}{y}{}$, 
as required.
\qeds

\begin{wrapfigure}{r}{25mm}
\vskip-0mm
\centering
\includegraphics{mppics/pic-750}
\end{wrapfigure}

\begin{thm}{Exercise}\label{ex:noncreasing} Given $\hinge p x y$ in a metric space $\spc{X}$, consider the function 
\[f\:(\dist{p}{\bar x}{},\dist{p}{\bar y}{})\mapsto \angk p{\bar x}{\bar y},\]
where $\bar x\in\left]p x\right]$ and $\bar y\in\left]p y\right]$.

\begin{subthm}{ex:noncreasing-CAT}
Suppose $\spc{X}$ is $\CAT(0)$.
Show that $f$ is nondecreasing in each argument.
\end{subthm}

\begin{subthm}{ex:noncreasing-CBB}
Suppose $\spc{X}$ is $\CBB(0)$.
Show that $f$ is nonincreasing in each argument.
\end{subthm}

Conclude that for any hinge in a $\CAT(0)$ or $\CBB(0)$ space has defined angle.
\end{thm}

\begin{thm}{Exercise}\label{ex:contractible}
Fix a point $p$ in a  be a complete length $\CAT(0)$ space~$\spc{X}$.
Given a point $x\in \spc{X}$, denote by $\gamma_x$ a (necessary unique) geodesic path from $p$ to $x$.

Show that the family of maps $h_t\: \spc{X}\to \spc{X}$ defined by 
\[h_t(x)= \gamma_x(t)\]
is a homotopy; it is called \index{geodesic homotopy}\emph{geodesic homotopy}. 
Conclude that $\spc{X}$ is contractible.
\end{thm}

The geodesic homotopy introduced in the previous exercise should help to solve the next one.

\begin{thm}{Exercise}\label{ex:CAT-mnfld=>ext.geod}
Let $\spc{X}$ be a complete length $\CAT(0)$ space.
Assume $\spc{X}$ is a topological manifold.
Show that any geodesic in $\spc{X}$ can be extended 
as a two-side infinite geodesic.
\end{thm}
 

\section{Thin and fat triangles}

Recall that a \index{triangle}\emph{triangle} $\trig xyz$ in a space $\spc{X}$ 
is a triple of minimizing geodesics $[xy]$, $[yz]$ and $[zx]$.
Consider the  model triangle $\trig{\tilde x}{\tilde y}{\tilde z}=\modtrig{}({x}{y}{z})_{\EE^2}$ in the Euclidean plane.
The \index{natural map}\emph{natural map} $\trig{\tilde x}{\tilde y}{\tilde z}\to \trig{x}{y}{z}$ 
sends a point $\tilde p\in[\tilde x\tilde y]\cup[\tilde y\tilde z]\cup[\tilde z\tilde x]$ to the corresponding point $p\in[ x y]\cup[y z]\cup[ z x]$;
that is, if $\tilde p$ lies on $[\tilde y\tilde z]$,
then $p\in [y z]$ and $\dist{\tilde y}{\tilde p}{}=\dist{y}{p}{}$ (and therefore $\dist{\tilde z}{\tilde p}{}=\dist{z}{p}{}$).

In the same way, the natural map can be defined for the spherical model triangle $\modtrig{}({x}{y}{z})_{\SSS^2}$.
 
\begin{thm}{Definition}\label{def:k-thin}
A triangle $\trig{x}{y}{z}$ in the metric space $\spc{X}$ 
is called \index{thin triangle}\emph{thin} (or \index{fat triangle}\emph{fat}) if the natural map $\modtrig{}({x}{y}{z})_{\EE^2}\to \trig{x}{y}{z}$ is distance nonincreasing (or respectively distance nondecreasing).

{\sloppy 

Analogously, a triangle $\trig{x}{y}{z}$ 
is called \index{spherically thin}\emph{spherically thin} or \index{spherically fat}\emph{spherically fat} if
the natural map from the spherical model triangle $\modtrig{}({x}{y}{z})_{\SSS^2}$ to $\trig{x}{y}{z}$ is distance nonincreasing or nondecreasing.

}
\end{thm}

\begin{thm}{Proposition}\label{prop:thin=cat}
A geodesic space is $\CAT(0)$ 
($\CAT(1)$) 
if and only if 
all its triangles are thin (respectively, all its triangles of perimeter $<2\cdot\pi$ are spherically thin).
\end{thm}

\parit{Proof; if part.} 
Apply  the triangle inequality and thinness of triangles $\trig pxy$ and $\trig qxy$, where $p$, $q$, $x$ and $y$ are as in the definition of $\CAT(\kappa)$ comparison (\ref{sec:def-CAT}).

\parit{Only-if part.} 
Applying $\CAT(0)$ comparison to a quadruple $p,q,x,y$ with $q\in [xy]$ shows that any triangle satisfies \index{point-side comparison}\emph{point-side comparison}, that is, the distance from a vertex to a  point on the opposite side is no greater than the corresponding distance in the Euclidean model triangle.  

Now consider a triangle $\trig{x}{y}{z}$ and let $p\in [xy]$ and $q\in [xz]$.
Let $\tilde p$, $\tilde q$ be the corresponding points on the sides of the model triangle $\modtrig({x}{y}{z})_{\EE^2}$.
Applying \ref{ex:noncreasing-CAT}, we get that
\[\angk {x} {y} {z}_{\EE^2} \ge \angk {x} p q _{\EE^2}.\]
Therefore $ \dist{\tilde p}{\tilde q}{\EE^2}\ge \dist{p}{q}{}$.

The $\CAT(1)$ argument is the same.
\qeds

\begin{thm}{Exercise}\label{ex:fat-triangle}
Show that any triangle is a $\CBB(0)$ space is fat. 
\end{thm}

\begin{thm}{Exercise}\label{ex:convex-dist}
Suppose $\gamma_1,\gamma_2\:[0,1]\to \spc{U}$ be two geodesic paths in a complete length $\CAT(0)$ space $\spc{U}$.
Show that
\[t\mapsto\dist{\gamma_1(t)}{\gamma_2(t)}{\spc{U}}\]
is a convex function.
\end{thm}

\begin{thm}{Exercise}\label{ex:convex-nbhd}
Let $A$ be a convex closed set in a proper length $\CAT(0)$ space $\spc{U}$;
that is, if $x,y\in A$, then $[xy]\subset A$.
Show that for any $r>0$ the closed $r$-neighborhood of $A$ is convex;
that is, the set
\[A_r=\set{x\in \spc{U}}{\distfun_Ax\le r}\]
is convex.
\end{thm}

\begin{thm}{Exercise}\label{ex:closest-point}
Let  $\spc{U}$ be a proper length $\CAT(0)$ space 
and $K\subset \spc{U}$ be a closed convex set.
Show that: 

\begin{subthm}{ex:closest-point:a}
For each point $p\in \spc{U}$ there is unique point $p^*\in K$ that minimizes the distance $\dist{p}{p^*}{}$.
\end{subthm}

\begin{subthm}{ex:closest-point:b}
The closest-point projection $p\mapsto p^*$ defined by \ref{SHORT.ex:closest-point:a} is short. 
\end{subthm}

\end{thm}

Recall that a set $A$ in a metric space $\spc{U}$ is called \index{locally convex set}\emph{locally convex} if for any point $p\in A$ there is an open neighborhood $\spc{U}\ni p$ such that any geodesic in $\spc{U}$ with  ends in $A$ lies in~$A$. 

\begin{thm}{Exercise}\label{ex:locally-convex}
Let $\spc{U}$ be a proper length $\CAT(0)$ space.
Show that any closed, connected, locally convex set in $\spc{U}$ is convex.
\end{thm}



\section{Other descriptions}

In this section we will list few ways to describe $\CAT(0)$ and $\CBB(0)$ spaces.
We do not give proofs of these statements, altho they are not hard \cite[see][ and the references therein]{alexander-kapovitch-petrunin-2025}.

These conditions will not be used in the sequel, but they might help to build right intuition.  

\parbf{Convexity of function.}
The following condition might help to adapt intuition from real analysis.

Let $\spc{X}$ be a metric space and $\lambda\in\RR$.
A function $f\:\spc{X}\to \RR$ is called \index{$\lambda$-convex}\emph{$\lambda$-convex} (\index{$\lambda$-concave}\emph{$\lambda$-concave}) if 
the real-to-real function 
\[t\mapsto f\circ\gamma(\gamma)-\tfrac{\lambda}{2}\cdot t^2\] 
is convex (respectively concave)
for any geodesic $\gamma\:\II\to \RR$.

The $\lambda$-convex  and $\lambda$-concave functions can be thought as functions satisfying inequalities $f''\ge\lambda$ and respectively $f''\le\lambda$ in a generalized sense.
Note that a smooth real-to-real function $f$ is $\lambda$-convex ($\lambda$-concave) if it satisfies inequality $f''\ge\lambda$ (respectively $f''\le\lambda$).

\begin{thm}{Proposition}
Let $\spc{X}$ be a geodesic space.
Then $\spc{X}$ is $\CAT(0)$ (respectively $\CBB(0)$) if and only if for any point $p\in \spc{X}$ the function
\[f(x)=\tfrac12\cdot\dist[2]{p}{x}{\spc{X}}\]
is 1-convex (respectively 1-concave).
\end{thm}

\parbf{Angle comparison.}
The following condition might help to adapt intuition from Euclidean geometry.

Recall that in $\CAT(0)$ and $\CBB(0)$ spaces any hinge has defined angle; see \ref{ex:noncreasing}.

\begin{thm}{Proposition}\label{prop:convexity}
Let $\spc{X}$ be a geodesic space such that any hinge in $\spc{X}$ has defined angle.
Then 

\begin{subthm}{prop:convexity:CAT}
$\spc{X}$ is $\CAT(0)$ if and only if 
\[\mangle\hinge pxy\le \angk pxy.\]
\end{subthm}

\begin{subthm}{prop:convexity:CBB}
$\spc{X}$ is $\CBB(0)$ if and only if 
\[\mangle\hinge pxy\ge \angk pxy\]
and 
\[\mangle\hinge pxy+\mangle\hinge pxz=\pi\]
for any {}\emph{adjacent hinges} $\hinge pxy$ and $\hinge pxz$;
that is, the union of the sides $[px]$ and $[pz]$ of the hinges form a geodesic $[xy]$.
\end{subthm}

\end{thm}

It is unknown if the condition on adjacent hinges in \ref{SHORT.prop:convexity:CBB} can be removed (even in the two-dimesional case).

\parbf{Kirszbraun property.}
We include the following condition only because it is beautiful.

The following theorem was proved by Mojżesz Kirszbraun \cite{kirszbraun} and rediscovered later by Frederick Valentine \cite{valentine}.

\begin{thm}{Theorem}
Let $A\subset \EE^m$.
Then any short map $f\:A\to \EE^n$
admits a short extension $f\:\EE^m\to \EE^n$.
\end{thm}

The conclusion of the theorem holds for some other metric spaces instead of $\EE^m$ and $\EE^n$.
For example instead of $\EE^n$ one might take any injective space (\ref{def:injective}) and instead of $\EE^m$ one may take any compact ultrametric space (\ref{ex:ultrametric}).
On the other hand, the existence of extension to/from a Euclidean space is a much weaker condition than in \ref{def:injective} and \ref{ex:ultrametric}.
As the following theorems state, these conditions are closely related to the $\CBB(0)$ and $\CAT(0)$ conditions.

\begin{thm}{Theorem}
Let $\spc{X}$ be a complete length space and $n\ge 2$.
Then $\spc{X}$ is $\CBB(0)$ if and only if for any set $A\subset \spc{X}$, 
any short map $f\:A\to \EE^n$ 
admits a short extension $F\:\spc{X}\to \EE^n$.
\end{thm}

\begin{thm}{Theorem}
Let $\spc{Y}$ be a metric space with and $m\ge 2$.
Assume any two points in $\spc{Y}$ are joint by unique geodesic.
Then $\spc{Y}$ is $\CAT(0)$ if and only if for any set $A\subset \EE^m$, 
any short map $f\:\EE^m\to \spc{Y}$ 
admits a short extension $F\:\EE^m\to \spc{Y}$.
\end{thm} 


\section{History}

The idea that the essence of curvature lies in a condition on quadruples of points apparently originated with Abraham Wald.
It is found in his publication on ``coordinate-free differential geometry'' \cite{wald} written under the supervision of Karl Menger;
the story of this discovery can be found in \cite{menger}.
In 1941, similar definitions were rediscovered independently by 
Alexandr Danilovich Alexandrov %Alexandr is the right spelling
\cite{alexandrov:def}.
In Alexandrov's work the first fruitful applications of this approach were given.
Mainly:
\begin{itemize}
\item Alexandrov's embedding theorem --- 
\textit{metrics of non-negative curvature on the sphere, and only they, are isometric to closed convex surfaces in Euclidean 3-space}. 
\item Gluing theorem, which tells  when the sphere obtained by gluing of two discs along their boundaries has non-negative curvature in the sense of Alexandrov.
\end{itemize}
These two results together gave  a very intuitive geometric tool for studying  embeddings and bending of surfaces in  Euclidean space, and changed this subject dramatically.
They formed the foundation of the branch of geometry now called {}\emph{Alexandrov geometry}.

The study of  spaces with curvature bounded above started later.
The first paper on the subject was written by Alexandrov \cite{alexandrov:strong-angle}.
It was based on work of Herbert Busemann \cite{busemann-CBA}, who studied spaces satisfying a weaker condition.
