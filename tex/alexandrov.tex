\chapter{Definitions}

\section{Manifesto of Alexandrov geometry}

Alexandrov geometry can use ``back to Euclid'' as a slogan.
Alexandrov spaces are defined via axioms similar to those given by Euclid,
but certain  equalities are changed to inequalities. 
Depending on the sign of the inequalities, we get Alexandrov spaces with \emph{curvature bounded above} or \emph{curvature bounded below}.
The definitions of the two classes of spaces are similar, but their properties and known applications are quite different.


Consider the space $\mathcal{M}_4$ of all isometry classes of 4-point metric spaces.
Each element in $\mathcal{M}_4$ can be described by 6 numbers 
 --- the distances between all 6 pairs of its points, say $\ell_{i,j}$ for $1\le i< j\le 4$ modulo permutations of the index set $(1,2,3,4)$.
These 6 numbers are subject to 12 triangle inequalities; that is,
\[\ell_{i,j}+\ell_{j,k}\ge \ell_{i,k}\]
holds for all $i$, $j$ and $k$, where we assume that $\ell_{j,i}=\ell_{i,j}$ and $\ell_{i,i}=0$.

The space $\mathcal{M}_4$ can be though of the cone in $\RR^6$ defined by triangle inequalities that is factorized by permutations of the 4-points of the space.
The same topology is induced on $\mathcal{M}_4$ by the Gromov--Hausdorff metric.

Consider the subset $\mathcal{E}_4\subset \mathcal{M}_4$ of all isometry classes of 4-point metric spaces that admit isometric embeddings into Euclidean space.

\begin{thm}{Claim}\label{clm:two-components-of-M4}
The complement $\mathcal{M}_4\backslash \mathcal{E}_4$ has two connected components.
\end{thm}

\begin{wrapfigure}{o}{33mm}
\vskip-0mm
\centering
\includegraphics{mppics/pic-70}
\end{wrapfigure}

One of the components will be denoted by $\mathcal{P}_4$ and the other by~$\mathcal{N}_4$.
Here $\mathcal{P}$ and $\mathcal{N}$ stand for {}\emph{positive} 
and {}\emph{negative curvature} because spheres have no quadruples of type $\mathcal{N}_4$ and 
hyperbolic space
has no quadruples of type~$\mathcal{P}_4$.

A metric space, with length metric, 
that has no quadruples of points of type $\mathcal{P}_4$ or $\mathcal{N}_4$
respectively 
is called an Alexandrov space with non-positive or non-negative curvature.

The following argument is based on idea from \cite{petrunin-quest}.

\parit{Sketch of proof.}
Let $\spc{X}$ be a 4-point metric space.

Fix a tetrahedron $\triangle$ in~$\RR^3$.
The vertices of $\triangle$, 
say $x_0$, $x_1$, $x_2$, $x_3$, can be identified with the points of~$\spc{X}$.

Note that there is a unique quadratic form $W$ on $\RR^3$
such that 
\[W(x_i-x_j)=\dist[2]{x_i}{x_j}{\spc{X}}\]
for all $i$ and~$j$.

By the triangle inequality, $W(v)\ge 0$ 
for any vector $v$ parallel to one of the faces of~$\triangle$.

Note that $\spc{X}$ is isometric to a 4-point subset in the Euclidean space
%???the Euclidean space or just Euclidean space? I mean the 3D  Euclidean space. A.
if and only if $W(v)\ge 0$ for any vector $v$ in~$\RR^3$.

\begin{wrapfigure}{o}{33mm}
\vskip-0mm
\centering
\includegraphics{mppics/pic-71}
\end{wrapfigure}

Therefore, if $\spc{X}$ is not of type $\mathcal{E}_4$, then $W(v)<0$ for some vector~$v$.
From above, the vector $v$ must be transversal to each of the 4 faces of~$\triangle$.
Therefore if we project $\triangle$ along $v$ to a plane transversal to $v$ we see one of the two pictures on the right.

Note that the set of vectors $v$ such that $W(v)<0$ has two connected components;
the opposite vectors $v$ and $-v$ lie in the different components.
If one moves $v$ continuously, keeping $W(v)<0$,
then the corresponding projection moves continuously and the projections of the 4 faces 
cannot degenerate. 
It follows that the combinatorics of the picture do not depend on the choice of~$v$. 
Hence $\mathcal{M}_4\backslash\mathcal{E}_4$ is not connected. 

It remains to show that if the combinatorics of the pictures for two spaces is the same, then one can continuously deform one space into the other.
This can be easily done by deforming $W$ and applying a permutation of $x_0$, $x_1$, $x_2$, $x_3$ if necessary.
\qeds

Here is an exercise, solving which would force the reader to rebuild a considerable part of Alexandrov geometry.

\begin{thm}{Advanced exercise}\label{ex:convex-set}
Assume $\spc{X}$ is a complete metric space with length metric, 
containing only quadruples of type~$\mathcal{E}_4$.
Show that $\spc{X}$ is isometric to a convex set in a Hilbert space.
\end{thm}

In fact, it might be helpful to spend some time thinking about this exercise before proceeding.

In the definition above, 
instead of  Euclidean space 
one can take  
hyperbolic space of curvature~$-1$.
In this case,
one obtains the definition of spaces with curvature bounded above or below by~$-1$.

To define spaces with curvature bounded above or below by $1$,
one has to take the unit 3-sphere 
and specify that only the quadruples of points such that each of the four triangles has perimeter 
less than $2\cdot\pi$ are checked.
The latter condition could be considered as a part of the {}\emph{spherical triangle inequality}.
